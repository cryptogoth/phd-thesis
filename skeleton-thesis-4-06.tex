\documentclass{article}

\usepackage{amssymb}

%    Q-circuit version 1.06
%    Copyright (C) 2004  Steve Flammia & Bryan Eastin

%    This program is free software; you can redistribute it and/or modify
%    it under the terms of the GNU General Public License as published by
%    the Free Software Foundation; either version 2 of the License, or
%    (at your option) any later version.
%
%    This program is distributed in the hope that it will be useful,
%    but WITHOUT ANY WARRANTY; without even the implied warranty of
%    MERCHANTABILITY or FITNESS FOR A PARTICULAR PURPOSE.  See the
%    GNU General Public License for more details.
%
%    You should have received a copy of the GNU General Public License
%    along with this program; if not, write to the Free Software
%    Foundation, Inc., 59 Temple Place, Suite 330, Boston, MA  02111-1307  USA

\usepackage[matrix,frame,arrow]{xy}
\usepackage{amsmath}
\newcommand{\bra}[1]{\left\langle{#1}\right\vert}
\newcommand{\ket}[1]{\left\vert{#1}\right\rangle}
    % Defines Dirac notation.
\newcommand{\qw}[1][-1]{\ar @{-} [0,#1]}
    % Defines a wire that connects horizontally.  By default it connects to the object on the left of the current object.
    % WARNING: Wire commands must appear after the gate in any given entry.
\newcommand{\qwx}[1][-1]{\ar @{-} [#1,0]}
    % Defines a wire that connects vertically.  By default it connects to the object above the current object.
    % WARNING: Wire commands must appear after the gate in any given entry.
\newcommand{\cw}[1][-1]{\ar @{=} [0,#1]}
    % Defines a classical wire that connects horizontally.  By default it connects to the object on the left of the current object.
    % WARNING: Wire commands must appear after the gate in any given entry.
\newcommand{\cwx}[1][-1]{\ar @{=} [#1,0]}
    % Defines a classical wire that connects vertically.  By default it connects to the object above the current object.
    % WARNING: Wire commands must appear after the gate in any given entry.
\newcommand{\gate}[1]{*{\xy *+<.6em>{#1};p\save+LU;+RU **\dir{-}\restore\save+RU;+RD **\dir{-}\restore\save+RD;+LD **\dir{-}\restore\POS+LD;+LU **\dir{-}\endxy} \qw}
    % Boxes the argument, making a gate.
\newcommand{\meter}{\gate{\xy *!<0em,1.1em>h\cir<1.1em>{ur_dr},!U-<0em,.4em>;p+<.5em,.9em> **h\dir{-} \POS <-.6em,.4em> *{},<.6em,-.4em> *{} \endxy}}
    % Inserts a measurement meter.
\newcommand{\measure}[1]{*+[F-:<.9em>]{#1} \qw}
    % Inserts a measurement bubble with user defined text.
\newcommand{\measuretab}[1]{*{\xy *+<.6em>{#1};p\save+LU;+RU **\dir{-}\restore\save+RU;+RD **\dir{-}\restore\save+RD;+LD **\dir{-}\restore\save+LD;+LC-<.5em,0em> **\dir{-} \restore\POS+LU;+LC-<.5em,0em> **\dir{-} \endxy} \qw}
    % Inserts a measurement tab with user defined text.
\newcommand{\measureD}[1]{*{\xy*+=+<.5em>{\vphantom{#1}}*\cir{r_l};p\save*!R{#1} \restore\save+UC;+UC-<.5em,0em>*!R{\hphantom{#1}}+L **\dir{-} \restore\save+DC;+DC-<.5em,0em>*!R{\hphantom{#1}}+L **\dir{-} \restore\POS+UC-<.5em,0em>*!R{\hphantom{#1}}+L;+DC-<.5em,0em>*!R{\hphantom{#1}}+L **\dir{-} \endxy} \qw}
    % Inserts a D-shaped measurement gate with user defined text.
\newcommand{\multimeasure}[2]{*+<1em,.9em>{\hphantom{#2}} \qw \POS[0,0].[#1,0];p !C *{#2},p \drop\frm<.9em>{-}}
    % Draws a multiple qubit measurement bubble starting at the current position and spanning #1 additional gates below.
    % #2 gives the label for the gate.
    % You must use an argument of the same width as #2 in \ghost for the wires to connect properly on the lower lines.
\newcommand{\multimeasureD}[2]{*+<1em,.9em>{\hphantom{#2}}\save[0,0].[#1,0];p\save !C *{#2},p+LU+<0em,0em>;+RU+<-.8em,0em> **\dir{-}\restore\save +LD;+LU **\dir{-}\restore\save +LD;+RD-<.8em,0em> **\dir{-} \restore\save +RD+<0em,.8em>;+RU-<0em,.8em> **\dir{-} \restore \POS !UR*!UR{\cir<.9em>{r_d}};!DR*!DR{\cir<.9em>{d_l}}\restore \qw}
    % Draws a multiple qubit D-shaped measurement gate starting at the current position and spanning #1 additional gates below.
    % #2 gives the label for the gate.
    % You must use an argument of the same width as #2 in \ghost for the wires to connect properly on the lower lines.
\newcommand{\control}{*-=-{\bullet}}
    % Inserts an unconnected control.
\newcommand{\controlo}{*!<0em,.04em>-<.07em,.11em>{\xy *=<.45em>[o][F]{}\endxy}}
    % Inserts a unconnected control-on-0.
\newcommand{\ctrl}[1]{\control \qwx[#1] \qw}
    % Inserts a control and connects it to the object #1 wires below.
\newcommand{\ctrlo}[1]{\controlo \qwx[#1] \qw}
    % Inserts a control-on-0 and connects it to the object #1 wires below.
\newcommand{\targ}{*{\xy{<0em,0em>*{} \ar @{ - } +<.4em,0em> \ar @{ - } -<.4em,0em> \ar @{ - } +<0em,.4em> \ar @{ - } -<0em,.4em>},*+<.8em>\frm{o}\endxy} \qw}
    % Inserts a CNOT target.
\newcommand{\qswap}{*=<0em>{\times} \qw}
    % Inserts half a swap gate. 
    % Must be connected to the other swap with \qwx.
\newcommand{\multigate}[2]{*+<1em,.9em>{\hphantom{#2}} \qw \POS[0,0].[#1,0];p !C *{#2},p \save+LU;+RU **\dir{-}\restore\save+RU;+RD **\dir{-}\restore\save+RD;+LD **\dir{-}\restore\save+LD;+LU **\dir{-}\restore}
    % Draws a multiple qubit gate starting at the current position and spanning #1 additional gates below.
    % #2 gives the label for the gate.
    % You must use an argument of the same width as #2 in \ghost for the wires to connect properly on the lower lines.
\newcommand{\ghost}[1]{*+<1em,.9em>{\hphantom{#1}} \qw}
    % Leaves space for \multigate on wires other than the one on which \multigate appears.  Without this command wires will cross your gate.
    % #1 should match the second argument in the corresponding \multigate. 
\newcommand{\push}[1]{*{#1}}
    % Inserts #1, overriding the default that causes entries to have zero size.  This command takes the place of a gate.
    % Like a gate, it must precede any wire commands.
    % \push is useful for forcing columns apart.
    % NOTE: It might be useful to know that a gate is about 1.3 times the height of its contents.  I.e. \gate{M} is 1.3em tall.
    % WARNING: \push must appear before any wire commands and may not appear in an entry with a gate or label.
\newcommand{\gategroup}[6]{\POS"#1,#2"."#3,#2"."#1,#4"."#3,#4"!C*+<#5>\frm{#6}}
    % Constructs a box or bracket enclosing the square block spanning rows #1-#3 and columns=#2-#4.
    % The block is given a margin #5/2, so #5 should be a valid length.
    % #6 can take the following arguments -- or . or _\} or ^\} or \{ or \} or _) or ^) or ( or ) where the first two options yield dashed and
    % dotted boxes respectively, and the last eight options yield bottom, top, left, and right braces of the curly or normal variety.
    % \gategroup can appear at the end of any gate entry, but it's good form to pick one of the corner gates.
    % BUG: \gategroup uses the four corner gates to determine the size of the bounding box.  Other gates may stick out of that box.  See \prop. 
\newcommand{\rstick}[1]{*!L!<-.5em,0em>=<0em>{#1}}
    % Centers the left side of #1 in the cell.  Intended for lining up wire labels.  Note that non-gates have default size zero.
\newcommand{\lstick}[1]{*!R!<.5em,0em>=<0em>{#1}}
    % Centers the right side of #1 in the cell.  Intended for lining up wire labels.  Note that non-gates have default size zero.
\newcommand{\ustick}[1]{*!D!<0em,-.5em>=<0em>{#1}}
    % Centers the bottom of #1 in the cell.  Intended for lining up wire labels.  Note that non-gates have default size zero.
\newcommand{\dstick}[1]{*!U!<0em,.5em>=<0em>{#1}}
    % Centers the top of #1 in the cell.  Intended for lining up wire labels.  Note that non-gates have default size zero.
\newcommand{\Qcircuit}{\xymatrix @*=<0em>}
    % Defines \Qcircuit as an \xymatrix with entries of default size 0em.


\begin{document}

\section{The Activities of Thesis Writing}

The common activity to all these strategies is that you \emph{write} 
something.

\begin{enumerate}

\item Write about things you want to do the next day. This is especially
useful if you did not do a lot of thinking or reading in the current day.

\item Write about things that you read today. Make a list of questions
that you have.

\item Write a general introduction of things from memory. (This seems the
hardest to do)

\item Write about what you are feeling, and how to deal with it, which
may delve into Buddhism and mindfulness.

\end{enumerate}

\section{True Quantum Division}

The first step is to reduce finding the ratio of two numbers $x$ and $y$
with finding the reciprocal $y^{-1}$ with some precision (binary fraction 
digits). Then taking the product $x y^{-1}$, since we already know how to
do multiplication.

So what are the steps in finding the reciprocal?

First, we know that $y$ is less than one. I think it is not the case that
$y$ is an arbitrary number, but rather that it is a number between 1 and 2.

We do this by scaling $x$ as $2^s y$ where $s$ is the largest integer such
that $2^s \le y \le 2^{s+1}$. Scaling $x$ by $2^s$, which we call $x'$ is
just bitshifting. Then $y' = y \cdot 2^{-s}$, which is just truncating bits.
We assume that we have a leading one by convention.

So now we know that $1 \le y' < 2$, and we want to find $y'^{-1}$, which
will be $\frac{1}{2} < y'^{-1} \le 1$. Let's say this is as easy as
taking the bits of $y'$:

\begin{equation}
y' = 1.y_1 y_2 y_3 y_4 \ldots y_n
\end{equation} 

Every bit of $y'^{-1}$, let's call it $z$, is simply the inverse of the
corresponding bit in $y'$.

\begin{equation}
z' = 0.1\cdot (1-y_1) \cdot (1-y_2) \cdot (1-y_3) \ldots
\end{equation}

FACT CHECK THIS. It seems too simplistic.

Now assume that we know the reciprocal, $z' = y'^{-1}$, we can shift it
up by $2^s$, which is technically part of $x'$, and then multiply.

\section{What is the Plan to Fail Softly}

Do not try to understand the division algorithm, or powering, or 
multiplication, in detail. Instead, concentrate on the resources needed.
If you have time, go back and try to understand the algorithm, or try to
derive the explicit polynomial.

Do not be seduced with doing new work. Some new work might be needed in
Chapter 4, when you try to simulate local Hamiltonians. That is the
most fearsome part.

Finish Chapter 1, and let others give you feedback and what you need to
do. Read the Harmonic Analysis paper, or at least understand better its
results which are used in the depth-efficient paper, to see if you can
get asymptotic bounds without knowing the explicit polynomial.

Isolate yourself from pretty girls, and social scenes where you will compete
and come to focus on what you do not have, rather than on what you have.

\section{The Desire for Vagueness}

Often there is a sense of wanting to avoid specifics, of meeting the
interface of your ignorance, by not formulating things in symbols. After
all, once you have put it in symbols, you could be wrong, and you could
fact check it.

Be vigilant against this. Use as many symbols as possible. Over-bias
yourself towards using symbols. And formulating things, and calculating
quantities.

\section{Quantum Circuit Coherence}

What can we state about the quantum circuit coherence and the circuits
we have calculated so far? Certainly, we know that fanning-out increases
coherence by a certain amount.
Fanning out a qubit by $n$ increases the circuit coherence at least by
$n\cdot D$, but as $D = O(1)$, we increase the circuit coherence by a
linear amount.

\section{Deterrences}

It is not just the fear of losing money which would make you want to
keep writing thesis pages for a particular day. It is also painful to
not have things to write, and so the memory of it would work on the
next day for you to read, and so have something to write about.

What are the things you want to read about tomorrow?
Multiplication, and how it scales up from the product of two numbers
to multiple product, and its relationship to POWERING.

Print out the paper on harmonic analysis. Print out the paper on the
power of small weights. Be aware of perfectionism or procrastination
as deterring you from you goal of finishing, since it is often easier,
even though still painful, to wallow in your not understanding of a thing
than to finish it, and submit an imperfect thing. As long as you prolong
the finishing, you preserve the fantasy of a future perfect thing.

When you were writing your generals exam, you made great progress in
reading papers and then summarizing them, in the style of a
buccaneer-scholar. Other techniques you used for isolation. You pretty
much cut yourself off from other society in order to get your work done.
Did you institute daily page quotas? Unknown. You definitely made it a
point to sit and write for several hours at a time. Maybe you have a
notebook account of the kind of work you did during this time.
A lot of meditation, which you are missing this time. Because you put off
the writing until the end of the day.

Not just insight into what you are doing. But also recognizing thoughts as
thoughts, and behaviors of unworthiness. Three minutes to go.

Printing out and reading other papers for a timeboxed amount of time.
You may have reached the limit of social pressure, where you are doing
more work to uphold your social network than you are getting from it,
in terms of suffering from guilt and so forth. It is not a nice way
to view your life, as business transactions. How much discomfort
is necessary?

\section{The Threshold Circuit}

Things that could affect the circuit resources of a threshold circuit.
One is the number of inputs. One is the magnitude of the weights.
One is the magnitude or precision of the thresholds. If, as I suspect,
the weights and the thresholds are scaled angles of $0 \le \phi \le 2\pi$,
then the precision matters, and is inversely proportional to the
magnitude. Polynomially-bounded weights in the classical threshold case
corresponds to inverse polynomial precision in the quantum case.

The circuit size of each threshold gate, when reduced to a 2D CCNTC
model (or maybe can even make it a 1D CCNTC model) depends on the
size of the gates, or the number of inputs, since this is the number
that must be used in transporting the qubits using constant-depth
teleportation.

\section{Connecting Themes of the Current Work}

Circuit coherence as a new resource for measuring time-space product and
tradeoffs. We need to show a separation between space on the lower bound
and depth-width on the upper bound.

Mapping other addition approaches to CCNTC, like the configurable-depth
Takahashi-Tani addition scheme.

The nearest-neighbor implementation of KSV compiling, or rather, using
the nearest-neighbor implementation of constant-depth addition.

\section{The Feeling of Retreating from Reality is Hard}

\end{document}