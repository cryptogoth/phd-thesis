\documentclass{article}

%    Q-circuit version 1.06
%    Copyright (C) 2004  Steve Flammia & Bryan Eastin

%    This program is free software; you can redistribute it and/or modify
%    it under the terms of the GNU General Public License as published by
%    the Free Software Foundation; either version 2 of the License, or
%    (at your option) any later version.
%
%    This program is distributed in the hope that it will be useful,
%    but WITHOUT ANY WARRANTY; without even the implied warranty of
%    MERCHANTABILITY or FITNESS FOR A PARTICULAR PURPOSE.  See the
%    GNU General Public License for more details.
%
%    You should have received a copy of the GNU General Public License
%    along with this program; if not, write to the Free Software
%    Foundation, Inc., 59 Temple Place, Suite 330, Boston, MA  02111-1307  USA

\usepackage[matrix,frame,arrow]{xy}
\usepackage{amsmath}
\newcommand{\bra}[1]{\left\langle{#1}\right\vert}
\newcommand{\ket}[1]{\left\vert{#1}\right\rangle}
    % Defines Dirac notation.
\newcommand{\qw}[1][-1]{\ar @{-} [0,#1]}
    % Defines a wire that connects horizontally.  By default it connects to the object on the left of the current object.
    % WARNING: Wire commands must appear after the gate in any given entry.
\newcommand{\qwx}[1][-1]{\ar @{-} [#1,0]}
    % Defines a wire that connects vertically.  By default it connects to the object above the current object.
    % WARNING: Wire commands must appear after the gate in any given entry.
\newcommand{\cw}[1][-1]{\ar @{=} [0,#1]}
    % Defines a classical wire that connects horizontally.  By default it connects to the object on the left of the current object.
    % WARNING: Wire commands must appear after the gate in any given entry.
\newcommand{\cwx}[1][-1]{\ar @{=} [#1,0]}
    % Defines a classical wire that connects vertically.  By default it connects to the object above the current object.
    % WARNING: Wire commands must appear after the gate in any given entry.
\newcommand{\gate}[1]{*{\xy *+<.6em>{#1};p\save+LU;+RU **\dir{-}\restore\save+RU;+RD **\dir{-}\restore\save+RD;+LD **\dir{-}\restore\POS+LD;+LU **\dir{-}\endxy} \qw}
    % Boxes the argument, making a gate.
\newcommand{\meter}{\gate{\xy *!<0em,1.1em>h\cir<1.1em>{ur_dr},!U-<0em,.4em>;p+<.5em,.9em> **h\dir{-} \POS <-.6em,.4em> *{},<.6em,-.4em> *{} \endxy}}
    % Inserts a measurement meter.
\newcommand{\measure}[1]{*+[F-:<.9em>]{#1} \qw}
    % Inserts a measurement bubble with user defined text.
\newcommand{\measuretab}[1]{*{\xy *+<.6em>{#1};p\save+LU;+RU **\dir{-}\restore\save+RU;+RD **\dir{-}\restore\save+RD;+LD **\dir{-}\restore\save+LD;+LC-<.5em,0em> **\dir{-} \restore\POS+LU;+LC-<.5em,0em> **\dir{-} \endxy} \qw}
    % Inserts a measurement tab with user defined text.
\newcommand{\measureD}[1]{*{\xy*+=+<.5em>{\vphantom{#1}}*\cir{r_l};p\save*!R{#1} \restore\save+UC;+UC-<.5em,0em>*!R{\hphantom{#1}}+L **\dir{-} \restore\save+DC;+DC-<.5em,0em>*!R{\hphantom{#1}}+L **\dir{-} \restore\POS+UC-<.5em,0em>*!R{\hphantom{#1}}+L;+DC-<.5em,0em>*!R{\hphantom{#1}}+L **\dir{-} \endxy} \qw}
    % Inserts a D-shaped measurement gate with user defined text.
\newcommand{\multimeasure}[2]{*+<1em,.9em>{\hphantom{#2}} \qw \POS[0,0].[#1,0];p !C *{#2},p \drop\frm<.9em>{-}}
    % Draws a multiple qubit measurement bubble starting at the current position and spanning #1 additional gates below.
    % #2 gives the label for the gate.
    % You must use an argument of the same width as #2 in \ghost for the wires to connect properly on the lower lines.
\newcommand{\multimeasureD}[2]{*+<1em,.9em>{\hphantom{#2}}\save[0,0].[#1,0];p\save !C *{#2},p+LU+<0em,0em>;+RU+<-.8em,0em> **\dir{-}\restore\save +LD;+LU **\dir{-}\restore\save +LD;+RD-<.8em,0em> **\dir{-} \restore\save +RD+<0em,.8em>;+RU-<0em,.8em> **\dir{-} \restore \POS !UR*!UR{\cir<.9em>{r_d}};!DR*!DR{\cir<.9em>{d_l}}\restore \qw}
    % Draws a multiple qubit D-shaped measurement gate starting at the current position and spanning #1 additional gates below.
    % #2 gives the label for the gate.
    % You must use an argument of the same width as #2 in \ghost for the wires to connect properly on the lower lines.
\newcommand{\control}{*-=-{\bullet}}
    % Inserts an unconnected control.
\newcommand{\controlo}{*!<0em,.04em>-<.07em,.11em>{\xy *=<.45em>[o][F]{}\endxy}}
    % Inserts a unconnected control-on-0.
\newcommand{\ctrl}[1]{\control \qwx[#1] \qw}
    % Inserts a control and connects it to the object #1 wires below.
\newcommand{\ctrlo}[1]{\controlo \qwx[#1] \qw}
    % Inserts a control-on-0 and connects it to the object #1 wires below.
\newcommand{\targ}{*{\xy{<0em,0em>*{} \ar @{ - } +<.4em,0em> \ar @{ - } -<.4em,0em> \ar @{ - } +<0em,.4em> \ar @{ - } -<0em,.4em>},*+<.8em>\frm{o}\endxy} \qw}
    % Inserts a CNOT target.
\newcommand{\qswap}{*=<0em>{\times} \qw}
    % Inserts half a swap gate. 
    % Must be connected to the other swap with \qwx.
\newcommand{\multigate}[2]{*+<1em,.9em>{\hphantom{#2}} \qw \POS[0,0].[#1,0];p !C *{#2},p \save+LU;+RU **\dir{-}\restore\save+RU;+RD **\dir{-}\restore\save+RD;+LD **\dir{-}\restore\save+LD;+LU **\dir{-}\restore}
    % Draws a multiple qubit gate starting at the current position and spanning #1 additional gates below.
    % #2 gives the label for the gate.
    % You must use an argument of the same width as #2 in \ghost for the wires to connect properly on the lower lines.
\newcommand{\ghost}[1]{*+<1em,.9em>{\hphantom{#1}} \qw}
    % Leaves space for \multigate on wires other than the one on which \multigate appears.  Without this command wires will cross your gate.
    % #1 should match the second argument in the corresponding \multigate. 
\newcommand{\push}[1]{*{#1}}
    % Inserts #1, overriding the default that causes entries to have zero size.  This command takes the place of a gate.
    % Like a gate, it must precede any wire commands.
    % \push is useful for forcing columns apart.
    % NOTE: It might be useful to know that a gate is about 1.3 times the height of its contents.  I.e. \gate{M} is 1.3em tall.
    % WARNING: \push must appear before any wire commands and may not appear in an entry with a gate or label.
\newcommand{\gategroup}[6]{\POS"#1,#2"."#3,#2"."#1,#4"."#3,#4"!C*+<#5>\frm{#6}}
    % Constructs a box or bracket enclosing the square block spanning rows #1-#3 and columns=#2-#4.
    % The block is given a margin #5/2, so #5 should be a valid length.
    % #6 can take the following arguments -- or . or _\} or ^\} or \{ or \} or _) or ^) or ( or ) where the first two options yield dashed and
    % dotted boxes respectively, and the last eight options yield bottom, top, left, and right braces of the curly or normal variety.
    % \gategroup can appear at the end of any gate entry, but it's good form to pick one of the corner gates.
    % BUG: \gategroup uses the four corner gates to determine the size of the bounding box.  Other gates may stick out of that box.  See \prop. 
\newcommand{\rstick}[1]{*!L!<-.5em,0em>=<0em>{#1}}
    % Centers the left side of #1 in the cell.  Intended for lining up wire labels.  Note that non-gates have default size zero.
\newcommand{\lstick}[1]{*!R!<.5em,0em>=<0em>{#1}}
    % Centers the right side of #1 in the cell.  Intended for lining up wire labels.  Note that non-gates have default size zero.
\newcommand{\ustick}[1]{*!D!<0em,-.5em>=<0em>{#1}}
    % Centers the bottom of #1 in the cell.  Intended for lining up wire labels.  Note that non-gates have default size zero.
\newcommand{\dstick}[1]{*!U!<0em,.5em>=<0em>{#1}}
    % Centers the top of #1 in the cell.  Intended for lining up wire labels.  Note that non-gates have default size zero.
\newcommand{\Qcircuit}{\xymatrix @*=<0em>}
    % Defines \Qcircuit as an \xymatrix with entries of default size 0em.


\begin{document}

\section{Constant-Depth Fanout and Unfanout}

The technique of the constant-depth fanout may at first appear mysterious,
but we shall explain it as follows. First, we will begin with the
constant-depth creation of an $n$-qubit cat state, an equal superposition of
$\ket{0}^{\otimes n}$ and $\ket{1}^{\otimes n}$. We begin with
the creation of fixed-size
cat states is natural, since if we want to go from a completely product
state (of all $\ket{0}$'s) to a maximally entangled state (the $n$-qubit
cat state), we can expect that any intermediate state will be a product of
$k$-qubit cat states, where $k < n$. Three-qubit cat states are the
minimum we can use, since as we will see later, our technique depends on
linking these smaller cat states together using the first and last qubit
in each triplet to project the two states into the Bell basis and then
measure them.

In the naive case, we go from $3$-qubit cat states to $6$-qubit cat states,
simply for each triplet, we apply a CNOT from the last qubit of the first
triplet to the first qubit of the second triplet.

Namely, we start out in the product state of two triplets:

\begin{equation}
\frac{1}{\sqrt{2}}(\ket{000} + \ket{111})\otimes
\frac{1}{\sqrt{2}}(\ket{000} + \ket{111})
\end{equation}

This gives us four basis states, with two contiguous blocks of $\ket{000}$
or $\ket{111}$.

\begin{equation}
\frac{1}{2}(\ket{000000} + \ket{000111} + \ket{111000} + \ket{111111})
\end{equation}

By entangling the middle two qubits with CNOTs, we now no longer have a
product state, but an entangled state, since we have removed the symmetry
separating the first register of qubits from the second.

\begin{equation}
\frac{1}{2}(\ket{000000} + \ket{000111} + \ket{111100} + \ket{111011})
\end{equation}

Now depending on which qubit we choose to measure, we can project the remaining
qubits into a subspace. We would like this subspace to be one where the
remaining qubits are in a cat state. Therefore, the qubit we measure should be
the least entangled with the remaining qubits. That rules out the qubits
1 through 3, since they are all $\ket{0}$ or all $\ket{1}$ in every
basis state. Likewise, this rules out qubits 5 and 6. This leaves us with
qubit 4, and indeed, if we measure it and get a $0$, we have the following
cat state on the remaining qubits:

\begin{equation}
\frac{1}{2}(\ket{00000} + \ket{11111})
\end{equation}

If we measure it and get a $1$, then we have the following cat state:

\begin{equation}
\frac{1}{2}(\ket{00011} + \ket{11100})
\end{equation}

Which we can correct into the first cat state by applying Pauli $X$ gates
to either the first qubits or the last two qubits. Cat states are symmetric
so it doesn't really matter.

How then, can be we extend this principle to multiple triplets? It turns out,
we can simultaneously apply a CNOT to an arbitrary number of triplets, or
$3$-qubit cat states, by linking the first and third qubit in every triplet
to the adjacent triplet. Imagine all the triplets sitting on a line.

We are essentially starting with a product state of $n/3$ $3$-qubit cat
states, which is a superposition of $2^{n/3}$ cat states.
By entangling them together, the target qubits of each CNOT will be measured,
and thus projected to a classical bit. It will not be part of our
final cat state, but it will tell us whether to apply a Pauli $X$
correction to our remaining cat state. The Pauli $X$ corrections for each
triplet cascade, so that (without loss of generality), every triplet to the
right will incorporate the Pauli $X$ corrections of a triplet to the left.
Because cat states are symmetric, we could also have Pauli $X$ corrections
cascade from the right to the left. The two approaches are equivalent.

Therefore, there are $(n/3)-1$ CNOT gates, and of the $n$ original qubits,
roughly two-thirds of them are part of the final cat state.

\begin{equation}
\frac{2}{3}n + 2
\end{equation}

Using this method, and discarding the intermediate scaffold qubits that we
have measured, assume we have an $n$-qubit cat state. Since the creation
of the fixed-size ($3$-qubit) cat state takes constant depth (depth 3 by
definition), and the interlinking CNOTs also happen concurrently, and the
Pauli $X$ corrections can also take place concurrently, and resetting the
intermediate qubits to $\ket{0}$ can also happen concurrently, the whole
circuit for cat state creation happens in depth $6$.

Now we wish to use this cat state to fan out an arbitrary qubit
$\ket{\psi} = \alpha\ket{0} + \beta\ket{1}$. Let's try the same trick of
linking them via a CNOT, measuring the target of the CNOT to project into
a subspace, and then using the measurement outcome to perform Pauli $X$
corrections.

\begin{equation}
(\alpha\ket{0} + \beta\ket{1})\otimes \frac{1}{\sqrt{2}}(\ket{0}^{\otimes n} + \ket{1}^{\otimes n})
\end{equation}

Apply the CNOT with control on the source qubit and targeting the first qubit
(say) of the cat state. We will denote this qubit as $\ket{x}_m$.

\begin{equation}
\frac{\alpha}{\sqrt{2}}\ket{0}\ket{0}_m\ket{0}^{\otimes n-1} + 
\frac{\alpha}{\sqrt{2}}\ket{0}\ket{1}_m\ket{1}^{\otimes n-1} + 
\frac{\beta}{\sqrt{2}}\ket{1}\ket{1}_m\ket{0}^{\otimes n-1} + 
\frac{\beta}{\sqrt{2}}\ket{1}\ket{0}_m\ket{1}^{\otimes n-1} + 
\end{equation}

Now if you project onto $\ket{0}_m$, then the resulting state is

\begin{equation}
\alpha\ket{0}^{\otimes n} + \beta\ket{1}^{\otimes n}
\end{equation}

Which is exactly the source qubit $\ket{\psi}$ fanned out $n-1$ times.

Now if you project onto $\ket{1}_m$, then the resulting state is

\begin{equation}
\alpha\ket{0}\ket{1}^{\otimes n-1} + \beta\ket{1}\ket{0}^{\otimes n-1}
\end{equation}

Unfortunately, this is not what we want. We could perform a layer of
$n-1$ parallel Pauli $X$ corrections to the qubits formerly of the cat state.
That would still keep us constant depth. This is not the most elegant
solution, but we will leave it alone for now. If only we could undo the
fanout in this case, and try again. Since both measurement outcomes
occur with equal probability, we can drive down our error probability
exponentially by repeating this fanout a linear number of times.

Now we will turn to the problem of \emph{unfanout}, which ideally would
return to us our source qubit $\ket{\psi}$ as well as an $n$-qubit cat
state for us to reuse.

How does one unmeasure, given that we know the classical outcome?

\end{document}