\chapter {Shor's Factoring Algorithm on a Nearest-Neighbor Architecture}

\section{Abstract}
%%%%%%%%%%%%%%%%%%%%
% put abstract here
%%%%%%%%%%%%%%%%%%%%
We present a 2D nearest-neighbor
quantum architecture for Shor's algorithm to factor an $n$-bit number in $O(\log^3n)$ depth.
Our implementation uses
%(1)
parallel phase estimation,
%(due to Kitaev, Shen, and Vyalyi),
%(2)
constant-depth fanout and teleportation,
%(due to Harrow, Fowler, and Taylor),
and
%(3)
constant-depth carry-save modular addition.
%(due to Gossett).
%We introduce a novel 2D architectural variation on Gossett's modular arithmetic
%and interleave constant-depth fanout and teleportation circuits for
%nearest-neighbor and long-distance communication channels, and ultimately use
%our circuit within parallel phase estimation to achieve quantum factoring.
We derive upper bounds on the circuit resources of our architecture under a
new 2D model which allows a classical controller and parallel, communicating
modules.
We provide a comparison to all previous nearest-neighbor factoring
implementations.  
Our circuit results in an exponential improvement in nearest-neighbor circuit depth at the cost of a polynomial increase in circuit size and width.

\input{ppham-thesis-factor-story}

\input{ppham-thesis-factor-intro}

\section{Background}
\label{sec:fpl-bg}

Quantum architecture is the design of physical qubit layouts
and their allowed interactions to execute
quantum algorithms efficiently in time, space, and other
resources.
In this paper, we focus on designing a realistic nearest-neighbor circuit for running
Shor's factoring algorithm on two-dimensional
architectural models of a physical quantum device with nearest-neighbor
interactions.

%%%%%%%%%%%%%%%%%%%%%%%%%%%%%%%%%%%%%%%%%%%%%%%%%%%%%%%%%%%%%%%%%%%%%%%%%%%%%%%
\subsection{Architectural Models and Circuit Resources}
\label{subsec:models}

Following Van Meter and Itoh \cite{VanMeter2005},
we distinguish between a model and an architectural implementation as follows.
A \emph{model} is a set of constraints and rules for the placement and
interaction of qubits.
An \emph{architecture} (or interchangeably, an \emph{implementation} 
or a \emph{circuit}) is a particular
spatial layout of qubits (as a graph of vertices) and allowed interactions (edges between the vertices),
following the constraints of a given model. In this section, we describe
several models which try to incorporate resources of physical interest from
experimental work. We also introduce a new model,
\textsc{2D CCNTCM}, which we will use to analyze our current circuit.

The most general model is called Abstract Concurrent (\textsc{AC})
and allows arbitrary, long-range interactions between any qubits and concurrent
operation of quantum gates.
This corresponds to a complete graph with an edge between every pair of nodes.
It is the model assumed in most quantum algorithms.

A more specialized model restricts interactions to nearest-neighbor, two-qubit,
concurrent gates (\textsc{NTC}) in a regular one-dimensional chain (1D NTC),
which is sometimes called linear nearest-neighbor (\textsc{LNN}).
This corresponds to a line graph. This is a more realistic model than
\textsc{AC}, but correspondingly, circuits in this model may incur greater
resource overheads.

To relieve movement congestion,
we can consider a two-dimensional regular grid
(2D NTC), where each
qubit has four planar neighbors, and 
there is an extra degree of freedom over the 1D model
in which to move data.
In this paper, we extend the \textsc{2D NTC} model in three ways.
The first two extensions are described in Section \ref{subsec:2dccntc},
and the third extension is described in Section \ref{subsec:2dccntcm}.

\subsection{\textsc{2D CCNTC}: Two-Dimensional Nearest-Neighbor Two-Qubit Concurrent Gates with Classical Controller}
\label{subsec:2dccntc}

The first extension allows arbitrary planar graphs
with bounded degree, rather than a regular square lattice.
Namely, we assume qubits lie in a plane and edges are not allowed to intersect.
All qubits are accessible from above
or below by control and measurement apparatus.
Whereas 2D NTC conventionally assumes each qubit
has four neighbors, we consider up to six neighbors in a roughly hexagonal
layout. The edge length in this model is no more than twice the edge length
in a regular 2D NTC lattice. The second extension is the realistic assumption
that classical control (CC) can
access every qubit in parallel, and we do not count these classical
resources in our implementation since they are polynomially bounded. The
classical controllers
correspond to fast digital computers which are
available in actual experiments and are necessary for constant-depth
communication in the next section.

We call an AC or NTC model augmented by these two extensions
\textsc{CCAC} and \textsc{CCNTC}, respectively. Before we describe the
third extension, let us formalize our model for \textsc{2D CCNTC}, with definitions that are (asymptotically) equivalent to those in 
\cite{Rosenbaum2012}.

\begin{definition}
A 2D CCNTC architecture consists of

\begin{itemize}
\item a quantum computer $QC$ which is represented by a planar graph $(V,E)$. A
node $v \in V$ represents a qubit which is acted upon in a circuit, and an
undirected edge $(u,v) \in E$ represents 
an allowed two-qubit interaction between qubits $u,v \in V$. Each node has
degree at most $6$.
\item a universal gate set $\mathcal{G} = \{X, Z, H, T, T^{\dagger}, CNOT, MeasureZ\}$.

\item a deterministic machine (classical controller) $CC$ that applies a sequence
of concurrent gates in each of $D$ timesteps.
\item In timestep $i$, $CC$ applies a set of
gates $G_i = \{g_{i,j} \in \mathcal{G} \}$.
Each $g_{i,j}$ operates in one of the following two ways:
\begin{enumerate}
\item It is a single-qubit gate from $\mathcal{G}$ acting on a single qubit $v_{i,j} \in V$
\item
It is the gate CNOT from $\mathcal{G}$ acting on two qubits $v^{(1)}_{i,j}, v^{(2)}_{i,j} \in V$ where
$(v^{(1)}_{i,j}, v^{(2)}_{i,j}) \in E$
\end{enumerate}
All the $g_{i,j}$ can only operate on
disjoint qubits for a given timestep $i$. We define the support of $G_i$
as $V_i$, the set of all qubits acted upon during timestep $i$.

\begin{equation}
V_i = \bigcup_{j: g_{i,j} \in G_i} v_{i,j} \cup v^{(1)}_{i,j} \cup v^{(2)}_{i,j}
\end{equation}

\end{itemize}
\end{definition}

We can then define the three conventional circuit resources in this model.

\begin{description}
\item[circuit depth ($D$):] the number of concurrent timesteps.
\item[circuit size ($S$):] the total number of non-identity gates applied
from $\mathcal{G}$, equal to $\sum_{i=1}^D |G_i|$.
\item[circuit width ($W$):] the total number of qubits operated upon by
any gate, including inputs, outputs, and ancillae. It is equal to $| \bigcup_{i=1}^D V_i|$.
\end{description}

We observe that the following relationship holds between the circuit resources.
The circuit size is bounded above by
the product of circuit depth and circuit width, since in the worst case,
every qubit is acted upon by a gate for every timestep of a circuit.
The circuit depth is also bounded above by the size, since in the worst case,
every gate is executed serially without any concurrency.

\begin{equation}
D \le S \le D\cdot W
\label{eqn:depth-width}
\end{equation}

The set $\mathcal{G}$ includes measurement in the $Z$ basis, which is
actually not a unitary operation but which may be slower than unitary
operations in actual practice \cite{DiVincenzo2007}.
Therefore we count it in our resource
estimates.
All other gates
in $\mathcal{G}$ form a universal set of unitary
gates \cite{Kitaev2002}.
 In this paper we
will treat the operations in $\mathcal{G}$ as \emph{elementary gates}.
We can also define a Bell basis measurement using operations
from $\mathcal{G}$. A circuit performing this measurement is shown
in Figure \ref{fig:bell-measure} and has depth $4$,
size $4$, and width $2$.

\begin{figure*}[tb!]
\begin{center}
\begin{displaymath}
\begin{array}{ccc}
\Qcircuit @C=1em @R=1em {
& \qw & \multimeasureD{1}{\mbox{Bell}} & \cw & \rstick{j} \\
& \qw & \ghost{\mbox{Bell}}            & \cw & \rstick{k}
}
& \qquad \equiv \qquad &
\Qcircuit @C=1em @R=1em {
& \qw & \ctrl{1} & \qw & \gate{H} & \qw & \meter & \cw & \rstick{j} \\
& \qw & \targfix & \qw & \qw      & \qw & \meter & \cw & \rstick{k}
}
\end{array}
\end{displaymath}
\centerline{}
\fcaption{A circuit for measurement in the Bell state basis.}
\label{fig:bell-measure}
\end{center}\end{figure*}

The third extension to our model, and the most significant, is to consider
multiple disconnected planar graphs, each of which is a 2D CCNTC
architecture. This is described in more detail in the next section.

\subsection{\textsc{2D CCNTCM}: Two-Dimensional Nearest-Neighbor Two-Qubit Concurrent Gates with Classical Controller and Modules}
\label{subsec:2dccntcm}

A single, contiguous
2D lattice which contains an entire quantum architecture which may be prohibitively large to manufacture. In practice,
scalable experiments will probably use many
smaller quantum computers which communicate by means of shared
entanglement \cite{Monroe2012}.
We call these individual machines \emph{modules}, each of
which is a self-contained \textsc{2D CCNTC} lattice. This should not be
confused with the word ``modular'' as in ``modular arithmetic'' or as
referring to the modulus $m$ which we are trying to factor.

We treat these modules
and teleportations between them as nodes and edges, respectively,
in a higher-level planar graph. The teleportations each transmit one qubit
from one module to another, from any location within the source module
to any location within the destination module, making use of the
omnipresent classical controller. The modules can be arbitrarily far
apart physically, but they have bounded-degree connectivity with other
modules, and their edges are planar (they cannot intersect).

A single module can be part of multiple teleportation operations in a single timestep, as long as they involve disjoint qubits within the module.
We justify this assumption in that it is
possible to establish entanglement between multiple
quantum computers
in parallel. We call this new model \textsc{2D CCNTCM},
and we argue that is captures the essential aspects of 2D architectures
without being overly sensitive to the exact geometry of the lattices involved.
An graphic depiction of three modules in \textsc{2D CCNTCM} is shown in
Figure \ref{fig:modules}. Each module contains within it a
\textsc{2D CCNTC} lattice. We can equivalently consider the omnipresent,
single
classical controller as a collection of multiple classical controllers, one
for each module or teleportation operation, which can inter-communicate
classically and share a clock.

\begin{figure}[btp!]
\begin{center}
\includegraphics[width=4in]{./modules.pdf}
\end{center}
\fcaption{Three modules in the \textsc{2D CCNTCM} model}
\label{fig:modules}
\end{figure}

\begin{definition}
A \textsc{2D CCNTCM} architecture consists of

\begin{itemize}
\item a quantum computer $\overline{QC}$ which is represented by a planar graph $(\overline{V},\overline{E})$. A
node $\overline{v} \in \overline{V}$ represents a module, or a graph $(V,E)$
from a \textsc{2D CCNTC} architecture defined previously. It can have
unbounded degree.
An
undirected edge $(\overline{u},\overline{v}) \in \overline{E}$ represents an
allowed teleportation from any qubit in module $\overline{u}$ to
another qubit in module $\overline{v}$.
\item All modules are restricted to be linear in the number of their qubits:
$|V| = \Theta(n)$ for all $(V,E) \in \overline{V}$.
\item a universal gate set $\mathcal{G} = \{X, Z, H, T, T^{\dagger}, CNOT,
MeasureZ\}$
for the qubits \emph{within the same} modules which is the same as for \textsc{2D CCNTC},
and an additional operation $Teleport$ which only operates on qubits
\emph{in
different} modules.
\item a deterministic machine (classical controller) $\overline{CC}$ that applies a sequence
of concurrent gates in each of $D+\overline{D}$ timesteps.
This can be a separate classical controller
for every pair of modules.
\item In timestep $i$, $\overline{CC}$ applies
gates $G_i = \{g_{i,j} : g_{i,j} \in \mathcal{G} \lor g_{i,j} = Teleport \}$.
That is, there are two kinds of timesteps with respect to the kinds of gates
which operate within them.
\begin{enumerate}
\item In the first kind, gates are exclusively from $\mathcal{G}$, and
they operate within modules as described
for \textsc{2D CCNTC} above. We say there are $D$ such timesteps.
\item In the second kind, gates are exclusively $Teleport$ gates between two qubits $v^{(1)}_{i,j} \in \overline{v}_1$ and
$v^{(2)}_{i,j} \in \overline{v}_2$ for
(possibly non-distinct) modules $\overline{v}_1, \overline{v}_2 \in \overline{V}$.
Again, all such qubits much be distinct within a timestep.
We say there are $\overline{D}$ such timesteps.
\end{enumerate}

Again, we define the support of $G_i$
as $V_i$, the set of all qubits acted upon by any $g_{i.j}$, which
includes all the modules.
\begin{equation}
V_i = \bigcup_{j: g_{i,j} \in G_i} v_{i,j} \cup v^{(1)}_{i,j} \cup v^{(2)}_{i,j} 
\end{equation}

\end{itemize}
\end{definition}

We measure the efficiency of a circuit in this new module using not just
the three conventional circuit resources, but with three novel resources
based on modules.

\begin{description}

%, depicted in Figure \ref{fig:resources}:
\item[module depth ($\overline{D}$):] the depth of consecutive teleportations between modules.
\item[module size ($\overline{S}$):] the number of total qubits teleported between any two modules over all timesteps.
\item[module width ($\overline{W}$):] the number of modules whose qubits are
acted upon during any timestep.

\end{description}

%We can make an observation analogous to Equation \ref{eqn:depth-width} but
%for modules in Equation \ref{eqn:module-depth-width}.

%\begin{equation}
%\overline{D} \le \overline{S} \le \overline{D}\cdot \overline{W}
%\label{eqn:module-depth-width}
%\end{equation}

We note the following relationship between circuit width and
module width.

\begin{equation}
W = O(n\overline{W})
\label{eqn:module-width}
\end{equation}

This restriction imposes some locality on our model by constraining it to
nearest-neighbor gates within a linear-sized group of qubits, but allowing
it long-range teleportation to circumvent onerous geometric constraints.
Using the constant-depth communication in Section \ref{sec:cdc}, and for
the specific case of factoring, we
can simulate arbitrary connectivity between modules with only a polynomial
increase in the module size and a constant increase in module depth.

\subsection{Circuit Resource Comparisons}

Counting gates from $\mathcal{G}$ as having unit size and unit depth
is
an overestimate compared to the model in \cite{Kutin2006}, in which a
two-qubit gate has unit size and unit depth and
absorbs the depth and size of any adjacent single-qubit gates. We intend
for this more pessimistic estimate to reflect the practical difficulties
in compiling these gates using a non-Clifford gate in a fault-tolerant way,
such as the $T$ gate or the Toffoli gate
\cite{Fowler2011}.
%However, these difficulties may be mitigated by using
%Toffoli gates directly, which can be fault-tolerantly implemented using
%magic-state distillation according to recent works \cite{Eastin2012,Jones2013a}.

In both our resource counting method and that of \cite{Fowler2004,Kutin2006}, multiple gates acting on disjoint qubits
can occur in parallel during the same timestep. For each building block,
from modular addition to modular multiplication and finally to modular
exponentiation, we provide closed form equations upper-bounding the required circuit
resources as a function of $n$, the size of the modulus $m$ to be factored.
We will use the
term \emph{numerical upper bound} to distinguish these formulae from asymptotic
upper bounds.

It is possible to reduce the numerical constants with more detailed analysis,
which would be important for any physical implementation.
However, we have chosen instead to simplify the number of terms in the formulae
for the current work. We do not intend for these upper bounds to represent
the optimal or final work in this area.

The modular adder in Section \ref{sec:csa-mod-add} and its carry-save
subcomponents only occur within a single module, so we only give their
circuit resources in terms of circuit depth, circuit size, and circuit width. 
For the modular multiplier in
Section \ref{sec:csa-mod-mult} and the modular exponentiator in
Section \ref{sec:modexp}, we also give circuit resources in
terms of module depth, module size, and module width.

\section{Constant-depth Teleportation, Fanout, and Unfanout}
\label{subsec:fanout}

Communication, namely the \emph{moving} and \emph{copying} of quantum information, in nearest-neighbor quantum architectures is challenging.
The first challenge of moving quantum information from one site to another over
arbitrarily long distances can be addressed by using
%A related problem is how to teleport a qubit an arbitrary distance.
% in an
%architecture through ancillae prepared in some initial state.
the constant-depth teleportation circuit
shown in Figure \ref{fig:cdt}, illustrated using standard quantum circuit
notation \cite{Nielsen2000}. This requires the circuit resources shown in
Table \ref{tab:cd-resources}. The depth includes a layer of $H$ gates; a layer of CNOTs; an interleaved layer of Bell basis measurements; and two layers of
Pauli corrections ($X$ and $Z$ for each qubit), occurring concurrently with
resetting the $\ket{j}$ and $\ket{k}$ qubits back to $\ket{0}$.
These correction layers are not shown in the circuit.

\begin{figure*}[tb!]
\begin{center}
\begin{displaymath}
%\begin{array}{ccc}
\Qcircuit @C=1em @R=1em {
\lstick{\ket{\psi}}	& \qw      & \qw      & \qw & \qw & \qw & \qw & \qw                                          & \qw & \qw & \multimeasureD{1}{\mbox{Bell}} & \cw & \rstick{j_1} \\
\lstick{\ket{0}}    & \gate{H} & \ctrl{1} & \qw & \qw & \qw & \qw & \qw                                          & \qw & \qw & \ghost{\mbox{Bell}}            & \cw & \rstick{k_1} \\
\lstick{\ket{0}}    & \qw      & \targfix & \qw & \qw & \qw & \qw & \qw_{Z^{j_1}X^{k_1}\ket{\psi}}               & \qw & \qw & \multimeasureD{1}{\mbox{Bell}} & \cw & \rstick{j_2} \\
\lstick{\ket{0}}    & \gate{H} & \ctrl{1} & \qw & \qw & \qw & \qw & \qw                                          & \qw & \qw & \ghost{\mbox{Bell}}            & \cw & \rstick{k_2} \\
\lstick{\ket{0}}    & \qw      & \targfix & \qw & \qw & \qw & \qw & \qw_{Z^{j_2}Z^{j_1}X^{k_2}X^{k_1}\ket{\psi}} & \qw & \qw & \multimeasureD{1}{\mbox{Bell}} & \cw & \rstick{j_3} \\
\lstick{\ket{0}}    & \gate{H} & \ctrl{1} & \qw & \qw & \qw & \qw & \qw                                          & \qw & \qw & \ghost{\mbox{Bell}}            & \cw & \rstick{k_3} \\
\lstick{\ket{0}}    & \qw      & \targfix & \qw & \qw & \qw & \qw & \qw & \qw_{Z^{j_1}Z^{j_2}Z^{j_3}X^{k_3}X^{k_2}X^{k_1}\ket{\psi}} & \qw & \qw              & \qw & \qw \\
}
\end{displaymath}
\centerline{}
\caption{Constant-depth circuit based on \protect{\cite{Broadbent2007,Browne2009}} for teleportation over $n=5$ qubits \protect{\cite{Rosenbaum2012}}.}
\label{fig:cdt}
\end{center}\end{figure*}

\begin{figure*}[tb!]
\begin{center}
\begin{displaymath}
%& \qquad \qquad \qquad &
\Qcircuit @C=1em @R=1em {
\lstick{\ket{\psi}}	& \qw      & \qw      & \qw & \qw & \qw & \multimeasureD{1}{\mbox{Bell}'} & \cw & \rstick{j_1} \\
\lstick{\ket{0}}    & \gate{H} & \ctrl{1} & \qw & \qw      & \qw & \ghost{\mbox{Bell}'}            & \cw & \rstick{k_1} \\
\lstick{\ket{0}_1}    & \qw      & \targfix & \qw & \ctrl{1} & \qw & \qw      & \qw & \rstick{Z^{j_1}X^{k_1}\ket{\ell}_1}\\
\lstick{\ket{0}}	& \qw      & \qw      & \qw & \targfix & \qw & \multimeasureD{1}{\mbox{Bell}} & \cw & \rstick{j_2} \\
\lstick{\ket{0}}    & \gate{H} & \ctrl{1} & \qw & \qw      & \qw & \ghost{\mbox{Bell}}           & \cw & \rstick{k_2} \\
\lstick{\ket{0}_2}    & \qw      & \targfix & \qw & \ctrl{1} & \qw & \qw      & \qw & \rstick{Z^{j_2}X^{k_2}X^{k_1}\ket{\ell}_2}\\
\lstick{\ket{0}}	& \qw      & \qw      & \qw & \targfix & \qw & \multimeasureD{1}{\mbox{Bell}} & \cw & \rstick{j_3} \\
\lstick{\ket{0}}    & \gate{H} & \ctrl{1} & \qw & \qw      & \qw & \ghost{\mbox{Bell}}           & \cw & \rstick{k_3} \\
\lstick{\ket{0}_3}    & \qw      & \targfix & \qw & \ctrl{1} & \qw & \qw      & \qw & \rstick{Z^{j_3}X^{k_3}X^{k_2} X^{k_1}\ket{\ell}_3}\\
\lstick{\ket{0}_4}	& \qw      & \qw      & \qw & \targfix & \qw & \qw      & \qw & \rstick{X^{k_3}X^{k_2} X^{k_1}\ket{\ell}_4}\\
}
%& & \\
%(a) & & (b)
%\end{array}
\end{displaymath}
\centerline{}
\caption{Constant-depth circuits based on \protect{\cite{Broadbent2007,Browne2009}} for fanout \protect{\cite{Harrow2012}} of one qubit to $n=4$ entangled copies.}
\label{fig:cdf}
\end{center}\end{figure*}

Although general cloning is
impossible \cite{Nielsen2000}, the second challenge of copying information can be addressed by performing an unbounded quantum
fanout operation:
$\ket{x,y_1,\ldots,y_n} \rightarrow \ket{x,y_1\oplus x, \ldots, y_n\oplus x}$.
This is used in our arithmetic circuits when
a single qubit needs to control (be entangled with) a large quantum register
(called a \emph{fanout rail}).
We employ a constant-depth circuit due to insight from
measurement-based quantum computing \cite{Raussendorf2003}
that relies on the creation of an
$n$-qubit cat state \cite{Browne2009}.

This circuit requires $O(1)$-depth, $O(n)$-size, and $O(n)$-width. Approximately
two-thirds of the ancillae are reusable and can be reset to $\ket{0}$ after
being measured. Numerical upper bounds are given in Table \ref{tab:cd-resources}.
The constant-depth fanout circuit is shown in Figure \ref{fig:cdf} for the case of fanning out a given single-qubit state
$\ket{\psi} = \alpha\ket{0} + \beta\ket{1}$ to four qubits.
The technique works by creating multiple small
cat states of a fixed size (in this case, three qubits), linking them
together into a larger cat state of unbounded size with Bell basis measurements,
and finally entangling them with the source qubit to be fanned out.
The qubits marked $\ket{\ell}$ are
entangled into the larger fanned out state given in Equation \ref{eqn:cat4}.
The Pauli corrections from the cat state creation are denoted by
$X^{k_2}$, $X^{k_3}$, $Z^{j_2}$ and $Z^{j_3}$ on qubits ending in
states $\ket{\ell}_1$, $\ket{\ell}_2$,
$\ket{\ell}_3$, and $\ket{\ell}_4$. The Pauli corrections
$X^{k_1}$ and $Z^{j_1}$ are from the Bell basis measurement
entangling the cat state with the source qubit (denoted $\text{Bell}'$).
\begin{equation}
Z_1^{j_1}X_1^{k_1}Z_2^{j_2}X_2^{k_2}X_2^{k_1}Z_{3}^{j_3}X_{3}^{k_3}X_{3}^{k_2}X_{3}^{k_1}X_{4}^{k_3}X_{4}^{k_2}X_{4}^{k_1}
\left(\alpha \ket{0}_1\ket{0}_2\ket{0}_3\ket{0}_4 + \beta \ket{1}_1\ket{1}_2\ket{1}_3\ket{1}_4 \right)
\label{eqn:cat4}
\end{equation}
%
The operators $X^k_i$ and $Z^j_{h}$ denote Pauli $X$ and $Z$ operators
on qubits $i$ and $h$, controlled by classical bits $k$ and $j$,
respectively. These corrections are enacted by the classical controller based on
the Bell measurement outcomes (not depicted).
Note the cascading nature of these corrections.
There can be up to
$n-1$ of these $X$ and $Z$
corrections on the same qubit, which can be simplified by the classical
controller to a single $X$ and $Z$ operation and then applied with a circuit of
depth 2 and size 2. Also, given the symmetric nature of the cat state, there
is an alternate set of Pauli corrections which would give the same state and
is of equal size to corrections given above.

Reversing the fanout (un-fanout) in constant depth is an interesting
problem. Doing so would allow us to improve the overall depth of our
factoring implementation to $O(\log^2 n)$ instead of $O(\log^3 n )$.
In this work it is sufficient to perform un-fanout using alternating rounds of
teleportation and CNOT among the $n$ fanned-out qubits in a logarithmic-depth
binary tree. The resources for this are given in
Table \ref{tab:cd-resources}.

% From Notebook #16, p. 212
% From Notebook #16, p. 66
\begin{table}
\begin{displaymath}
\begin{tabular}{|c|c|c|c|}
\hline
\text{Circuit Name} & \text{Depth} & \text{Size} & \text{Width}\\
\hline
\text{Teleportation from Figure \ref{fig:cdt}} & 7 & 3n + 4 & n+1\\
\hline
\text{Fanout from Figure \ref{fig:cdf}} & 9 & 10n - 9 & 3n-1 \\
\hline
\text{Un-fanout} & $8\log_2(2n)$ & $33n\log_2(2n) + 10\log^2_2(2n)$ & $3n-1$ \\
\hline
\end{tabular}
\end{displaymath}
\centerline{}
\caption{Circuit resources for teleportation, fanout, and un-fanout
(consisting of
alternating rounds of constant-depth teleportation and CNOT).}
\label{tab:cd-resources}
\end{table}

From an experimental perspective, it is physically efficient to create
a cat state in trapped ions using the M{\o}lmer-S{\o}rensen gate
\cite{Sorensen2000}\cite{Benhelm2008}. However, the fanout circuit for
the 2D CCNTCM model would still be useful for other technologies, such
as superconducting qubits on a two-dimensional lattice.

%Unfortunately, this
%``consumes'' the cat state in that there is no known way to unentangle the
%source qubit from the cat state after they have been jointly measured \cite{Rosenbaum2012}.

%\input{factor2d-related}
%%%%%%%%%%%%%%%%%%%%%%%%%%%%%%%%%%%%%%%%%%%%%%%%%%%%%%%%%%%%%%%%%%%%%%%%%%%%%%


\input{ppham-thesis-factor-related}

\input{ppham-thesis-factor-carrysave}

\section{Quantum Modular Addition}
\label{sec:csa-mod-add}

To perform addition of two numbers $a$ and $b$ modulo $m$,
we consider the variant problem of modular addition of three numbers to
two numbers:
%
%\begin{quote}
Given three $n$-bit input numbers $a$, $b$, and $c$, and an $n$-bit modulus $m$,
compute
%\begin{equation}
$(u+v) = (a+b+c)[m]$,
%\end{equation}
where $(u+v)$ is a CSE number.

In this section, we provide an alternate, pedagogical explanation of
Gossett's modular reduction \cite{Gossett1998}. Later, we contribute a mapping of this adder
to a 2D architecture,
using unbounded fanout to maintain constant depth for adding back
modular residues. This last step is absent in Gossett's original approach.

To start, we will demonstrate the basic method of modular addition and reduction
on an $n$-bit conventional number. In general, adding two $n$-bit conventional
numbers will produce an overflow bit of significance $2^n$, which we can truncate as long as
we add back its modular residue $2^n \bmod m$. How can we guarantee that we won't
generate another overflow bit by adding back the modular residue? It turns out
we can accomplish this by allowing
a slightly larger input and output number ($n+1$ bits in this case), truncating
multiple overflow bits, and adding back their $n$-bit modular residues.

For two $(n+1)$-bit conventional numbers $x$ and $y$,
we truncate the three high-order bits of their sum $z_{n-1,n+3}$
and
add back their modular residue $x_{(n-1,n)}[m]$:
%
\begin{eqnarray}
x + y \bmod m &=& z_{(0,n+1)}[m] \nonumber \\
&=& z_{(0,n-2)} + z_{(n-1,n+1)}[m].
\end{eqnarray}
%
Since both the truncated number $z_{(0,n-2)}$ and the modular residue
are $n$-bit numbers, their sum is an $(n+1)$-bit number as desired, equivalent
to $x[m]$.

Now we must do the same modular reduction on a CSE number $(u+v)$,
which in this case represents an $(n+2)$-bit conventional number and has
$2n+3$ bits.
%This is the special case mentioned in the
%previous
%section \label{star:csa-special}, where $x$ is the result of a single
%CSA layer, not repeated CSA layers alternating with truncation.
%
%Assume for now that this modular reduction works;
%in the next section we walk through an illustrated concrete example.
%We present a more formal argument in Section \ref{subsec:mod-reduce-1}.
%
First, we truncate the three high-order bits ($v_{n}, u_{n-1}, v_{n-1}$)
of $(u+v)$, yielding an $n$-bit
conventional number with a CSE representation of $2n$ bits:
$\{u_0, u_1, \ldots, u_{n-1}\} \cup \{v_1, v_2, \ldots, v_{n-1}\}$.
Then we add back the three modular residues
$(v_{(n+1)}[m], u_{(n)}[m], v_{(n)}[m])$, and we are guaranteed not to
generate additional overflow bits (of significance $2^{n}$ or higher). This equivalence
is shown in Equation \ref{eqn:mod-reduce}.
\begin{eqnarray}
(u+v)[m] &=& \left(u_{(0,n+1)} + v_{(1,n+2)}\right)[m] \nonumber \\
 &=& u_{(0,n)} +
     v_{(1,n)} + \nonumber \\
 & & u_{(n+1)}[m] +
     v_{(n+1)}[m] + v_{(n+2)}[m]
\label{eqn:mod-reduce}
\end{eqnarray}

\begin{lemma}[Modular Reduction in Constant Depth]
The modular addition of three $n$-bit numbers to two $n$-bit numbers can be
accomplished
in constant depth with $O(n)$ width in \textsc{2D CCNTC}.
\end{lemma}

%\begin{proof}
\vspace*{12pt}
\noindent
{\bf Proof:}
Our goal is to show how to perform modular addition while keeping our numbers
of a fixed size by treating overflow bits correctly.
We map the proof of \cite{Gossett1998} to \textsc{2D CCNTC} and show that
we meet our required depth and width.
First, we enlarge our registers to allow the addition of $(n+2)$-bit numbers,
while keeping our modulus of size $n$ bits.
(In Gossett's original approach, he takes the equivalent step of restricting
the modulus to be of size $(n-2)$ bits.) We accomplish the modular addition
by first performing a layer of non-modular addition, truncating the three high-order
overflow bits, and then adding back modular residues controlled on these
bits in three successive layers, where we are guaranteed that no additional
overflow bits are generated in each layer.
This is illustrated for a $3$-bit modulus and $5$-bit registers
in Figure \ref{fig:csa-proof}.

\begin{center}
\begin{figure*}[h!tb]
\begin{displaymath}
\renewcommand\arraystretch{1.5}
\begin{array}{ccccccll}
        & a_4 & a_3 & a_2 & a_1 & a_0 & 5\text{-bit input number } a &\\
        & b_4 & b_3 & b_2 & b_1 & b_0 & 5\text{-bit input number } b & \\
        & c_4 & c_3 & c_2 & c_1 & c_0 & 5\text{-bit input number } c & \text{[Layer 1]}\\
\hline
        & u_4 & u_3 & u_2 & u_1 & u_0 & \text{truncate } u_{4} & \\
    v_5 & v_4 & v_3 & v_2 & v_1 &     & \text{truncate } v_{4},v_{5} & \\
        &     &     & c^{v_4}_2 & c^{v_4}_1 & c^{v_4}_0 & \text{add back } 2^4 \bmod m \text{ controlled on } v_4 & \text{[Layer 2]}\\
\hline
        &      & u'_3 & u'_2 & u'_1 & u'_0 & & \\
        & v'_4 & v'_3 & v'_2 & v'_1 &      & & \\
        &      &    & c^{u_4}_2 & c^{u_4}_1 & c^{u_4}_0  & \text{add back } 2^4 \bmod m \text{ controlled on } u_4 & \text{[Layer 3]}\\
\hline
        & u''_4 & u''_3 & u''_2 & u''_1 & u''_0 & \text{the bit } u''_4 \text{ is the same as } v'_4 & \\
        & v''_4 & v''_3 & v''_2 & v''_1 &       &  & \\
        &       &    & c^{v_5}_2 & c^{v_5}_1 & c^{v_5}_0 & \text{add back } 2^5 \bmod m \text{ controlled on } v_5 & \text{[Layer 4]}\\
\hline
        & u'''_4 & u'''_3 & u'''_2 & u'''_1 & u'''_0 & \text{ Final CSE output with } 5 \text{ bits} &\\
        & v'''_4 & v'''_3 & v'''_2 & v'''_1 &        & \text{ Final CSE output with } 5 \text{ bits} & \\
\end{array}
%\begin{array}{cccccccr}
%        & a_{n+1} & a_{n} & a_{n-1} & \ldots & a_1 & a_0 & \text{input number } a\\
%        & b_{n+1} & b_{n} & b_{n-1} & \ldots & b_1 & b_0 & \text{input number } b\\
%        & c_{n+1} & c_{n} & c_{n-1} & \ldots & c_1 & c_0 & \text{input number } c\\
%\hline
%        & u_{n+1} & u_{n} & u_{n-1} & \ldots & u_1 & u_0 & \text{truncate } u_{n+1} \\
%v_{n+2} & v_{n+1} & v_{n} & v_{n-1} & \ldots & v_1 & 0   & \text{truncate } v_{n+1},v_{n+2} \\
%        &         &       & x_{n-1} & \ldots & x_1 & x_0 \\
%\hline
%        &         & u'_{n} & u'_{n-1} & \ldots & u'_1 & u'_0 & \\
%        & v'_{n+1} & v'_{n} & v'_{n-1} & \ldots & v'_1 & 0 &  \\
%        &         &       & x_{n-1} & \ldots & x_1 & x_0 \\
%\hline
%        & u''_{n+1} & u''_{n} & u''_{n-1} & \ldots & u''_1 & u''_0 & \\
%        & v''_{n+1} & v''_{n} & v''_{n-1} & \ldots & v''_1 & 0 &  \\
%        &         &       & y_{n-1} & \ldots & y_1 & y_0 \\
%\hline
%        & u'''_{n+1} & u'''_{n} & u'''_{n-1} & \ldots & u'''_1 & u'''_0 & \\
%        & v'''_{n+1} & v'''_{n} & v'''_{n-1} & \ldots & v'''_1 & 0 &  \\
%\hline
%\end{array}
\end{displaymath}
\fcaption{A schematic proof of Gossett's constant-depth modular reduction for $n=3$.}
\label{fig:csa-proof}
\end{figure*}
\end{center}

We use the following notation.
The non-modular sum of the first layer is $u$ and $v$.
The CSE output of the first modular reduction layer
is $u'$ and $v'$, and the modular residue is
written as $c^{v_{n+1}}$ to mean the precomputed value $2^{n+1} \bmod m$
controlled on $v_{n+1}$.
The CSE output of the second modular reduction layer
is $u''$ and $v''$, and the modular residue is written as
$c^{u_{n+1}}$ to mean the precomputed value $2^{n+1} \bmod m$
controlled on $u_{n+1}$.
The CSE output of the third and final modular reduction layer
is $u'''$ and $v'''$, and the modular residue is written as
$c^{v_{n+2}}$ to mean the precomputed value $2^{n+2} \bmod m$
controlled on $v_{n+2}$.

We show that no layer generates an overflow $(n+2)$-bit, namely in the
$v$ component of any CSE output. (The $u$ component will never exceed the
size of the input numbers.) First, we know that no $v'_{n+2}$ bit
is generated after the first modular reduction layer, because we have
truncated away all $(n+1)$-bits. Second, we know that no $v''_{n+2}$ bit is
generated because we only have one $(n+1)$-bit to add, $v'_{n+1}$.
Finally, we need to show that $v'''_{n+2} = 0$ in the third modular reduction
layer. 

Since $u'_{(n)} + v'_{(n+1)} =
u_{(n)} + v_{(n)} \le 2^{n+1}$, the bits $u'_n$ and $v'_{n+1}$ cannot both be $1$.
But $u''_{n+1} = v'_{n+1}$ and $v''_{n+1} = u'_n\land v'_n$, so $u''_{n+1}$ and
$v''_{n+1}$ cannot both be $1$, and hence $v'''_{n+2} = 0$.
%This bit is the majority of
%$u''_{n+1}$, $v''_{n+1}$, and $c^{v_{n+2}}_{n+1} = 0$. This means we only have
%to guarantee that at most one of $u''_{n+1}$ and $v''_{n+1}$ has value 1.
%This is equivalent to requiring that
%$u''_{(n,n+1)} + v''_{(n+1)} \le 3\cdot 2^{n}$, that is, the sum of these
%three bits has value at most $3$. Bit $u''_{n+1}$ is copied directly from
%$v'_{n+1}$ by the rules of CSA, which requires the following condition for
%the second modular reduction layer:
%$u'_{(n)} + v'_{(n,n+1)} \le 3\cdot 2^n$. This is true because
%$u'_{(n)} + v'_{(n+1)} = u_{(n)} + v_{(n)} \le 2$ and $v'_{(n)} \le 1$.
Everywhere
we use the fact that the modular residues are restricted to $n$ bits.
Therefore, the modular sum is computed as the sum of two $(n+2)$-bit numbers
with no overflows in constant-depth.
%\end{proof}
\square\,

As a side note, we can perform modular reduction in one layer instead of
three by decoding the three overflow bits into one of seven different
modular residues. This can also be done in constant depth, and in this case
we only need to enlarge all our registers to $(n+1)$ bits instead of $(n+2)$
as in the proof above. We omit the proof for brevity.

In the following two subsections, we give a concrete example to illustrate
the modular addition circuit as well as a numerical upper bound for the
general circuit resources.

%%%%%%%%%%%%%%%%%%%%%%%%%%%%%%%%%%%%%%%%%%%%%%%%%%%%%%%%%%%%%%%%%%%%%%%%%%%%%%%
\subsection{A Concrete Example of Modular Addition}
\label{subsec:concrete}

\begin{center}
\begin{figure*}[h!bt]
\centerline{
\includegraphics[width=6.5in]{./mod-add-fixed.pdf}
}
\fcaption{Addition and three rounds of modular reduction for a 3-bit
modulus.}
\label{fig:csa-add-4}
\end{figure*}
\end{center}

A \textsc{2D CCNTC} circuit for modular addition of $5$-bit numbers using
four layers of parallel CSA's is shown graphically in Figure \ref{fig:csa-add-4}
which corresponds directly to the schematic proof in Figure \ref{fig:csa-proof}.
Note that in Figure \ref{fig:csa-add-4}, the least significant qubits are
on the left, and in Figure \ref{fig:csa-proof}, the least significant qubits are
on the right.
Figure \ref{fig:csa-add-4} also represents the approximate
physical layout of the qubits as they would look if this
circuit were to be fabricated.
Here, we convert the sum of three
$5$-bit integers into the modular sum of two $5$-bit integers, with a
$3$-bit modulus $m$.
In the first layer,
we perform 4 CSA's in parallel on the input numbers ($a,b,c$) and produce the
output numbers ($u, v$).

As described above, we truncate
the three high-order bits during the initial CSA round
(bits $u_4, v_4, v_5$) to retain a $4$-bit number.
Each of these bits serves as a control for adding its modular residue to
a running total. We can classically precompute $2^4[m]$ for the two
additions controlled on $u_4$ and $v_4$ and
$2^5[m]$ for the addition controlled on $v_5$.

In Layer 2,
we use a constant-depth fanout rail (see Figure \ref{fig:cdf}) to
distribute the control bit $v_4$ to its modular residue, which we denote as
%%\begin{equation}
$\ket{c^{v_4}} \equiv \ket{2^4[m]\cdot v_4}$.
%%\end{equation}
%This fanout requires constant depth;
The register $c^{v_4}$ has $n$ bits, which we add to the CSE results of layer 1.
The results $u_i$ and $v_{i+1}$ are teleported into layer 3. The exception is
$v'_4$ which is teleported into layer 4, since there are no other $4$-bits
to which it can be added. Wherever there are only
two bits of the same significance, we use the 2-2 adder from
Section \ref{sec:csa}.

Layer 3
%%, shown in Figure \ref{fig:csa-add-3},
operates similarly to layer 2, except that the modular residue is controlled on
$u_4$:
%%\begin{equation}
$\ket{c^{u_4}} \equiv \ket{2^4[m] \cdot u_4}$.
%%\end{equation}
%This fanout again requires constant depth;
The register $c^{u_4}$ has $3$ bits, which we
add to the CSE results of layer 2, where $u'_i$ and $v'_{i+1}$ are teleported
forward into layer 4.

Layer 4
%%, shown in Figure \ref{fig:csa-add-4},
is similar to layers 2 and 3, with the modular residue controlled on $v_5$:
%%\begin{equation}
$\ket{c^{v_5}} \equiv \ket{2^5[m] \cdot v_5}$.
%%\end{equation}
%This fanout is constant depth;
The register $c^{v_5}$ has $3$ bits, which we
add to the CSE results of layer 3.
There is no overflow bit $v'''_5$, and no carry bit from $v''_4$ and $v'_4$
as argued in the proof of Lemma 1.
The final modular sum $(a+b+c)[m]$ is $u'''+v'''$.

The general circuit for adding three $n$-qubit quantum integers to
two $n$-qubit quantum integers is called a \emph{CSA tile}. Each CSA tile in our architecture 
corresponds to its own module, and it will be represented by the symbol in 
Figure \ref{fig:csa-tile-symbol} for the rest of this paper. We call this
an $n$-bit modular adder, even though it accepts $(n+2)$-bit inputs, because
the size of the modulus is still $n$ bits.

\begin{center}
\begin{figure*}[h!bt]
\centerline{
\includegraphics[width=1.5in]{./csa-tile-symbol.pdf}
}
\fcaption{Symbol for an $n$-bit 3-to-2 modular adder, also called a CSA tile.}
\label{fig:csa-tile-symbol}
\end{figure*}
\end{center}


\subsection{Quantum Circuit Resources for Modular Addition}

We now calculate numerical upper bounds for the circuit resources of
the $n$-bit $3$-to-$2$ modular adder described in the previous section.
There are four layers of non-modular $n'$-bit $3$-to-$2$ adders, each of which
consists of $n'$ parallel single-bit adders whose
resources are detailed in Table \ref{tab:csa-tile-resources}. For factoring
an $n$-bit modulus, we have $n'=n+2$ in the first and fourth layers
and $n'=n+1$ in the second and third layers.

After each of the first three layers, we must move the output qubits
across the fanout rail to be the inputs of the next layer. We use
two swap gates, which have a depth and size of $6$ CNOTs each, since
the depth of teleportation is only more efficient for moving more than
two qubits. The control bit for each modular residue needs to be
teleported $0$, $4$, and $7$ qubits respectively according to the
diagram in Figure \ref{fig:csa-add-4}, before being fanned out $n$
times along the fanout rails, where the fanned out copies will end up
in the correct position to be added as inputs.

%The detailed resources for a Toffoli gate and the single-bit adder that uses
%them are given in Table \ref{tab:csa-tile-resources}.

The resources for the $n$-bit $3$-to-$2$ modular adder depicted in Figure
\ref{fig:csa-add-4} are given below.
The formulae reflect the resources needed for both computing the output
in the forward direction (including creating an entangled fanned-out state
controlled on overflow qubits)
and also uncomputing ancillae in the backward
direction (including disentangling previous fanned-out copies).

The circuit depth is $O(1)$:

\begin{equation}
374\text{.}
\end{equation}

The circuit size is $O(n)$:

\begin{equation}
551n + 757\text{.}
\end{equation}

The circuit width is $O(n)$:

\begin{equation}
33n + 47\text{.}
\end{equation}

\input{ppham-thesis-factor-modmult}

\input{ppham-thesis-factor-modexp}

\input{ppham-thesis-factor-threshold}

\section{Controlled Rotations for Factoring}
\label{sec:factor-crz}

In Section \ref{sec:related}, we stated that we wished to avoid
factoring implementations that used a QFT due to the fine
single-qubit rotations involved. Due to the requirements of
fault-tolerance on a particular physical implementation,
we can usually implement only a set of gates that is fixed
(it does not change with the problem input size), discrete (of finite size),
and universal. This last property is necessary for us to approximate any
other gate \emph{not} in our set, especially single-qubit phase rotations
of the form $\Lambda(e^{i \phi})$. Such an approximation would involve
a quantum compiling procedure, such as Solovay-Kitaev, which is the
subject of Chapter \ref{chap:qcompile}. However, we mention it here
because the choice of our universal set determines the true depth
of any circuit.

In our polylogarithmic factoring implementation, we were able to reduce
all our arithmetic circuits to such a fixed, discrete universal set.
These arithmetic circuits are discrete and classical and nature, so it is
not surprising that we can implement them in a discrete way.
However, to reduce the depth further, we need to introduce the idea of
a quantum threshold gate, which 
the Toffoli gate, CNOT, and fixed set of
single-qubit gates.



\begin{figure}[tb!]
\begin{center}
\begin{displaymath}
\begin{array}{ccc}
\Qcircuit @C=1.5em @R=1.5em {
   & \qw      & \ctrl{1}                   & \qw \\
   & \qw      & \gate{\frac{\pi}{2^{d}}} & \qw \\
 }
&
\begin{array}{c}
\\
\\
\\
= \\
\end{array}
&
\Qcircuit @C=1.5em @R=1.5em {
& \qw & \qw & \qw & \ctrl{1} & \qw & \gate{\frac{\pi}{2^{d+1}}} & \qw & \ctrl{1} & \qw\\
 & \qw & \gate{\frac{\pi}{2^{d+1}}} & \qw & \targfix & \qw & \gate{\frac{\pi}{2^{d+1}}} & \qw & \targfix & \qw
}
\end{array}
\end{displaymath}
\caption{Decomposition of a controlled-$R_z$ rotation}
\label{fig:crz}
\end{center}\end{figure}


\section{Quantum Majority Circuits for Modular Exponentiation}

We now consider circuits made from majority gates, which we denote
as $\text{MAJ}_n \equiv \text{TH}^{\lceil n/2 \rceil}_n$. These have the
advantage of being simpler to implement than general threshold gates
$\text{TH}^t_n$ while being equivalent in power. Namely, depth-$k$
majority circuits are equivalent in power to depth-$k$ threshold circuits
with polynomially-bounded weights: $\textsf{MAJ}_k = \hat{\textsf{LT}}_k$
\cite{Alon1994,Goldmann1994}.

In many works on classical majority circuits, the fanin of a majority
gate is not treated as an important consideration and can in 

\begin{theorem}{Multiple product in constant depth and polynomial size.\cite{Yeh1996}}
The $n^2$-bit product of $n\times n$-bit numbers can be computed by a
circuit of depth $O(\frac{1}{\epsilon})$,
size and width $O(\frac{1}{\epsilon}n^{3+2\epsilon})$, and
fanin $O(n)$.
\end{theorem}

\begin{theorem}{Modular reduction in constant depth and polynomial size \cite{Yeh1996}}
The $n$-bit binary representation of the modular residue $x \bmod m$, where
$x$ is an $n^2$-bit number and $m$ is an $n$-bit modulus, can be computed
by a circuit of depth $O(\frac{1}{\epsilon})$,
size and width $O(\frac{1}{\epsilon}n^{1 + 2\epsilon})$, and
fanin $O(n^2)$.
\end{theorem}



\section{A Majority Gate in 2D CCNTCM in Sub-logarithmic Depth}

To map a majority gate to 2D CCNTCM, we use the $\textsf{AC}_f^0$ 
constructions from Takahashi-Tani \cite{Takahashi2011},
based on the results of
Hoyer-Spalek \cite{Hoyer2002}.
We combine this with our architectural techniques
from the previous chapter, including the 2D CCNTCM circuits
for constant-depth teleportation, fanout, and unfanout.

The majority gate is a threshold gate, with unit weights, a
fan-in (input size) $n' = O(n)$ in the size of the input number to Shor's 
factoring algorithm, and a threshold of $\lceil n/2 \rceil$.

We describe the construction for $\text{MAJ}_{n'}$ below, given the
$n'$-qubit input $x$.

\begin{enumerate}

\item
We compute in parallel the gates $\text{EX}^i_{n}$ for
$0 \le i \le \lceil n/2 \rceil$ to determine if the quantum
threshold for majority is reached. There are at most $(n/2) + 1$
such gates.

\begin{enumerate}
\item 
Compute the constant-depth reduction from $\text{EX}^t_{n'}$ to
$\text{EX}^t_{m}$ where $m = \lceil \log_2(n'=1) \rceil$, using
the reduction from $OR_n$ to $OR_{\log_n}$ \cite{Hoyer2002}.
For $1 \le k \le m$, do the following:

\begin{enumerate}
\item
Compute the qubit $\mu^{|x|-t}_{\phi_k}$, which is the rotation by Hamming 
weight of $x$, with a threshold $t$ subtracted, by the angle $\phi_k = 2\pi / 2^k$. Note that the precision
of this angle is $O(\log \log n)$. This can be done by a 2D CCNTCM circuit
with $O(1)$-depth, $O(n^2)$-size, and $O(n^2)$-width, as described in
Section \ref{subsec:or-reduce}.
\end{enumerate}

At the end of this step, we have $O(\log_2 n)$ bits $\ket{y_k}$ which 
represent the degree to which the Hamming weight of $x$ is close to the
threshold $t = \lceil n/2 \rceil$.

\item
Apply the circuit for exact $\text{OR}_{\log n}$ from Lemma 2 of
\cite{Takahashi2011} to the output of the previous step. This can
be done with a 2D CCNTCM circuit with $O(1)$-depth, $O(n \log n)$-size,
and $O()$.

\end{enumerate}

\item
Apply the gate $\text{PA}_{\lceil n/2 \rceil}$ to the result of
the previous step. This can be done by a 2D CCNTCM circuit of
$O(1)$-depth, $O(n)$-size, and $O(n)$-width using constant-depth
fanout, as described in Section \ref{sec:cdc}, and conjugated by
Hadamards on every qubit as described in \cite{Moore1998}.

\item
Apply a NOT to the output of the previous step. This final
output is the output of the quantum majority gate $MAJ_{n}$.

\end{enumerate}


\subsection{OR Reduction of Hoyer-Spalek}
\label{subsec:or-reduce}

This step involves $m \le (n/2)+1$ gates $\mu^{|x|-t}_{\phi_k}$ for
$1 \le k \le m$. Each gate is the same as Hoyer and Spalek's reduction from
$O(n)$ bits to $O(\log n)$ bits as described in \cite{Hoyer2002}.

The $\mu$ circuits 

\subsection{Exact OR Circuit of Takahashi-Tani}
\label{subsec:or-exact}



\subsection{Parity Circuit}
\label{subsec:parity}



\section{Asymptotic Results}
\label{sec:fpl-results}

The asymptotic resources required for our approach,
as well as the resources for other nearest-neighbor approaches,
are listed in Table \ref{tab:results},
where we assume a fixed constant error
probability for each round of QPF. Not all resources are
provided directly by the referenced source.

Resources in square brackets
are inferred using Equation \ref{eqn:depth-width}.
These upper bounds are correct,
but may not be tight with the upper bounds
calculated by their respective authors.
In particular, a more detailed analysis
could give a better upper bound for circuit size than the
depth-width product. Also note that the
work by Beckman et al. \cite{Beckman1996} is unique in that it uses
efficient multi-qubit gates inherent to linear ion trap technology which at first
seem to
be more powerful than \textsc{1D NTC}. However, use of these gates does not result in an
asymptotic improvement over \textsc{1D NTC}.

%, say $\epsilon=1/4$.
% and $\delta' = 1/2$ for KSV-QPF.
%Note that the
%number of measurements are included for completeness.
%, since these are
%not counted as gates in our model but may be comparable in terms of
%execution time.
%Some table cells are blank if the entries are not relevant to the current comparison, or if the entires were not %calculated in the prior work.
We achieve an exponential
improvement in nearest-neighbor circuit depth (from quadratic to polylogarithmic)
with our approach at the cost of a polynomial increase in
circuit size and width. Similar depth improvements at the cost of width increases can be achieved using the modular multipliers
of other factoring implementations
by arranging them in a parallel modular exponentiator.
Our approach is the first implementation for factoring on \textsc{2D NTC},
augmented with a classical controller and parallel, communicating
modules (\textsc{2D CCNTCM}).
%
\begin{table}[htb!]
\begin{center}
\begin{tabular}{|c|c|c|c|c|}
\hline
Implementation             & Architecture      & Depth   & Size   & Width     \\
\hline
Vedral, et al. \cite{Vedral1996}   & \textsc{AC}      & $[O(n^3)]$ & $O(n^3)$    & $O(n)$ \\
Gossett \cite{Gossett1998}                   & \textsc{AC}       & $O(n \log n)$  & $[O(n^3\log n)]$  & $O(n^2)$  \\
Beauregard \cite{Beauregard2002}                & \textsc{AC}       & $O(n^3)$      & $O(n^3 \log n)$ & $O(n)$ \\
Zalka \cite{Zalka1998}                     & \textsc{AC}       & $O(n^2)$      & $[O(n^3)]$ & $O(n)$     \\
Takahashi \& Kunihiro \cite{Takahashi2006}     & \textsc{AC}       & $O(n^3)$      & $O(n^3\log n)$ & $O(n)$ \\
Cleve \& Watrous \cite{Cleve2000}           & \textsc{AC}       & $O(\log^3 n)$ & $O(n^3)$ & $[O(n^3 / \log^3n)]$ \\
\hline
Beckman et al. \cite{Beckman1996} & \textsc{Ion trap}   & $O(n^3)$ & $O(n^3)$ & $O(n)$\\
\hline
Fowler, et al. \cite{Fowler2004} & \textsc{1D NTC}   & $O(n^3)$ & $O(n^4)$ & $O(n)$\\
Van Meter \& Itoh \cite{VanMeter2006} & \textsc{1D NTC}   & $O(n^2 \log n)$ & $[O(n^4\log n)]$ & $O(n^2)$\\
Kutin \cite{Kutin2006}                     & \textsc{1D NTC}   & $O(n^2)$ & $O(n^3)$ & $O(n)$\\
\hline
Current Work               & \textsc{2D CCNTCM}   & $O(\log^2{n})$ & $O(n^4)$ & $O(n^4)$   \\
\hline
\end{tabular}
\end{center}
\tcaption{Asymptotic circuit resource usage for quantum factoring of an $n$-bit number.}
\label{tab:results}
\end{table}

\section{Improving the Depth Beyond Sublogarithmic}

It is now natural to ask, given such dramatic improvement in
nearest-neighbor circuit
depth from quadratic \cite{Kutin2006} to polylogarithmic, can we
decrease depth further? Surprisingly, the answer is yes. In this
chapter we now decrease the depth below polylogarithmic, in fact,
to be $O((\log \log n)^2)$.

\section{The Block-Save Technique}

This doesn't really go anywhere, but I will mention here out of
completeness.
As an aside, one may think that one can iterate the carry-save
adder. Instead of re-encoding the sum of 3 bits as the sum of
2 bits, we could also re-encode the sum of 7 bits to be the
sum of 3 bits. In analogy to the 3-2 adder, we call this a 7-3 adder.
How would this re-encoding work?

In the 3-2 adder, there were two output bits of significance $1$
and $2$. The $1$-bit was the parity of the three bits and
the $2$-bit was the majority of the three bits. The truth tables
for these functions are shown in Table \ref{tab:3-2}

\begin{table}
\begin{tabular}{cc|c}
\hline
$x_0$ & $x_1$ & $c$ \\
\hline
0 & 0 & 0 \\
0 & 1 & 1 \\
1 & 0 & 1 \\
1 & 1 & 0 \\
\hline
\end{tabular}
\caption{Truth tables for}
\label{tab:3-2}
\end{table}

 In a 7-3 adder,
likewise, the $1$-bit is the parity of \emph{single} bits.
The $2$-bit is the parity of \emph{pairs} of bits, 

\section{Circuit Complexity Classes}

A circuit is a directed acyclic graph in which the nodes are
logical gates drawn from a certain (universal) set and the edges 
represent
the connection of the output of one gate to the input of another
gate. For classical circuits, they implement Boolean functions, which take in $n$ input
bits to one output bit.

\begin{equation}
f:{0,1}^n \rightarrow {0,1}
\end{equation}

For quantum circuits, they implement reversible functions on
$(n+1)$-qubits.


 We can also define special nodes which are not gates, but rather
are placeholder ``sources'' which provide the inputs to the circuit and 
``sinks'' which provide the outputs to the circuit. The in-degree of a 
node is also known as its \emph{fanin} and the out-degree of a node is
also known as its \emph{fanout}.

We denote a gate by its fanin as a subscript and an optional
second parameter as a superscript.

\begin{equation}
\text{GATE}_n^k
\end{equation}

\subsection{Classical Circuit Complexity Classes}

It is useful to define complexity classes of circuits based on the
set of allowed gates. In classical circuits, we take unbounded fanout
for granted (any node can have arbitrary out-degree). These are common
in the literature of classical circuits. We will list them in order
of the size of their universal set, where each subsequent class adds
more gates.
Obviously, since the set is universal to begin with, adding more gates
does not increase the power of the class to represent more Boolean
functions. Rather, it can decrease the depth or size of the circuit.
Once we define the classes, we can design subclasses.

\begin{definition}
\item[\textsf{NC}]
circuits consisting of $\text{NOT}_1$ and $\text{AND}_2$ and
$\text{OR}_2$ gates.
\item[\textsf{AC}]
NC circuits augmented with $\text{AND}_n$ and $\text{OR}_n$ gates,
for $n \ge 2$.
\item[\textsf{TC}]
AC circuits augmented with $\text{TH}_n^t$ gates, for $n \ge 2$ and
$0 \le t \le n$.
\end{definition}

A $\text{TH}^t_n$ gate is a threshold gate that is $1$ if the number
of input bits in greater than or equal to the threshold $t$ and $0$
otherwise.

\begin{equation}
% TODO insert block piecewise bracket
\end{equation}

We are often interested in the computing power of the above
circuit classes restricted in some way, usually shallow depth.
We denote by a superscript $k$ a complexity class of
functions implementable by circuits of depth $k$.

For these classical circuit classes, it is known that containment
is proper.

\begin{equation}
\textsf{NC}^0 \subsetneq \textsf{AC}^0 \subsetneq \textsf{TC}^0
\end{equation}

\subsection{Quantum Gates}

Unbounded fanout is taken for granted in classical circuit classes
because it is physically realistic to implement using electrical
devices. However, for quantum circuits, we must make use of the
unbounded quantum fanout, $\text{FANOUT}_n$, to copy one output
qubit of each gate.
We can define quantum analogues of the above circuit complexity 
classes by defining quantum analogues for each of the gates which
is reversible.
$NOT_1$ is already reversible, and we can use it as is.
To replace $AND_2$ we can use the reversible $3$-qubit Toffoli gate,
the so-called controlled-controlled-NOT.
To replace $OR_2$ we can use the circuit given in
Figure \ref{fig:or2}

\begin{figure}
% TODO 
\caption{A perfectly normal $\text{OR}_2$ gate minding its own business.}
\label{fig:or2}
\end{figure}

In fact this is special case of a much more powerful construction
that will let us define a quantum $OR_n$ gate on unbounded inputs.

\section{Quantum Compiling Rotations from a Fixed Set}

WAIT! Insight. We only need to compile gates of the form
$2\pi / n\log n$. I don't think that actually gives us anything.
But if we use a co-prime multiple of $2^n$.

\section{Circuit Resources}
