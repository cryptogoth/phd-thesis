\chapter{Nearest-Neighbor Factoring in Sublogarithmic Depth}
\label{chap:fsl}

\section{Previous Threshold Circuits and Related Work}

In order to understand techniques of decreasing depth for quantum 
circuits, we can leverage a well-studied area of classical circuit
complexity, namely that of threshold circuits.

There is a more general class of threshold gates than $\text{TH}_n^t$
defined earlier. These allow weights other than $+/-1$. In general, we
allow them to be any real number, although this doesn't allow them to
be 

Classical circuit complexity studies the resources needed
to implement a boolean function
$f:\{0,1\}^n \rightarrow {0,1}$. The resources are usually the
depth and size of the circuit in gates taken from some universal
set, which may be of fixed size or vary with the size of the input.
By adding gates, restricting their properties, or setting
constraints on circuit depth and size, we can define what boolean
functions are computable using the resulting circuit complexity
class.

A \emph{linear threshold element} (LTE) is a logic gate whose output is the
sign of the weighted sum of $n$ input bits $x = \{x_1, x_2, \ldots, x_n\}$,
where initially we take
the weights $w_i$ to be any real number. There is a weight that is not
associated with any input bit which is called the \emph{bias} of the
LTE. It is the most elementary logic element in threshold circuits.
Analogous to a simplified model of neurons in the human brain, an LTE can
accept many inputs through its dendrites, which are the outputs of other
neurons or sensory organs, and compute a function based on ``learned'' 
weights. If the weighted sum passes some threshold (the bias), the neuron
``fires,'' or sends signal down its axon, which then becomes the input to
another neuron or ends in an actuator, like a muscle.

\begin{equation}
f(x) = \sgn\left( w_0 \sum_{i=1}^n w_i x_i \right)
\end{equation}

The class of boolean functions implementable with a single layer
of LTE's is called
$\textsf{LT}_1$. In fact, a circuit of LTEs of depth $1$ consists only
of $1$ LTE gate. We are primarily interested in constant depth threshold
circuits, which we denote $\textsf{LT}_k$ for depths $k>0$.

An alternative way of expressing boolean functions is using a polynomial
threshold element (PTE), which is a weighted sum over \emph{multilinear terms}
in the input variables. In this representation, it is useful to
map the bit values $\{0,1\}$ to $\{+1,-1\}$.

\begin{equation}
f(x) = \sgn\left( w_0 \sum_{a \in \{0,1\}^n} w_a X_a \right) \qquad X_a \equiv \prod_{i=1}^n x_i^{a_i}
\end{equation}

We call the class of functions implementable by $k$ layers of PTE's
$\textsf{PT}_k$, where $\textsf{PT}_1$ indicates the class of functions
implementable by a single PTE.
It is easy to see that any boolean function can be implemented using
a sum-of-products, with one multilinear term for each of the $2^n$
rows in the function's truth table. The surprising relationship between
these two primitive gates 


\begin{enumerate}

\item
consider from exponential number of terms to polynomial number
of terms. This definitely decreases power, but has a practical
motivation in that we will quickly run out of space.

\item
change weights from real numbers to rational numbers. in fact,
integers with $n \log n$ bits can be used. (doesn't change power) need citation. motivation is because we often have limited precision is
implementing gates physically.

\item
change exponential weights to polynomially bounded weights (can simulate in constant-depth, polynomial increase in size), need
citation.

\item
change to unit weights (majority gates). can simulate in
constant depth polynomial increase in size) need citation.
We can allow weights of $+1$ and $-1$ simply by putting a
NOT gate in front of an input we want.

\end{enumerate}

There are (non-constructive) existence proofs of many interesting 
arithmetic functions that can be implemented by threshold circuits
of constant depth (called \emph{bounded depth} in the literature),
often of very small sizes (5 or less). These exist in the class
$\textsf{TC}^0$, and it is especially surprising because they are known
not to be in the class $\textsf{AC}^0$, which proves the proper
containment $\textsf{TC}^0 \subsetneq \textsf{AC}^0$. That is, in
the classical setting, threshold gates appear to contain some
special power that cannot be captured by AND and OR, gates, even those
with unbounded fanin, in contant depth.

In \cite{Reif1992}, it is shown that threshold circuits
are equivalent (up to constant depth) to another class of
circuits known as finite field ($Z_{P(n)}$) circuits, where each node
computes either multiple sums or products of integers modulo a
prime $P(n)$. However, in both of these cases, the fanin

An important technique in any operation involving multiplication is
the conversion between a binary representation, where numbers are
treated as a linear combination from the basis of elements of
size $O(2^n)$, to a so-called ``Chinese Remainder'' representation,
where numbers are treated as a linear combination of residues
modulo primes of size $O(n)$. This latter representation can be thought
of as a ``mixed-radix'' representation, and is responsible for
the depth-reduction of multiplication related operations in
threshold circuits.

\section{Controlled Rotations for Factoring}
\label{sec:factor-crz}

In Section \ref{sec:related}, we stated that we wished to avoid
factoring implementations that used a QFT due to the fine
single-qubit rotations involved. Due to the requirements of
fault-tolerance on a particular physical implementation,
we can usually implement only a set of gates that is fixed
(it does not change with the problem input size), discrete (of finite size),
and universal. This last property is necessary for us to approximate any
other gate \emph{not} in our set, especially single-qubit phase rotations
of the form $\Lambda(e^{i \phi})$. Such an approximation would involve
a quantum compiling procedure, such as Solovay-Kitaev, which is the
subject of Chapter \ref{chap:qcompile}. However, we mention it here
because the choice of our universal set determines the true depth
of any circuit.

In our polylogarithmic factoring implementation, we were able to reduce
all our arithmetic circuits to such a fixed, discrete universal set.
These arithmetic circuits are discrete and classical and nature, so it is
not surprising that we can implement them in a discrete way.
However, to reduce the depth further, we need to introduce the idea of
a quantum threshold gate, which 
the Toffoli gate, CNOT, and fixed set of
single-qubit gates.



\begin{figure}[tb!]
\begin{center}
\begin{displaymath}
\begin{array}{ccc}
\Qcircuit @C=1.5em @R=1.5em {
   & \qw      & \ctrl{1}                   & \qw \\
   & \qw      & \gate{\frac{\pi}{2^{d}}} & \qw \\
 }
&
\begin{array}{c}
\\
\\
\\
= \\
\end{array}
&
\Qcircuit @C=1.5em @R=1.5em {
& \qw & \qw & \qw & \ctrl{1} & \qw & \gate{\frac{\pi}{2^{d+1}}} & \qw & \ctrl{1} & \qw\\
 & \qw & \gate{\frac{\pi}{2^{d+1}}} & \qw & \targfix & \qw & \gate{\frac{\pi}{2^{d+1}}} & \qw & \targfix & \qw
}
\end{array}
\end{displaymath}
\caption{Decomposition of a controlled-$R_z$ rotation}
\label{fig:crz}
\end{center}\end{figure}


\section{Quantum Majority Circuits for Modular Exponentiation}

We now consider circuits made from majority gates, which we denote
as $\text{MAJ}_n \equiv \text{TH}^{\lceil n/2 \rceil}_n$. These have the
advantage of being simpler to implement than general threshold gates
$\text{TH}^t_n$ while being equivalent in power. Namely, depth-$k$
majority circuits are equivalent in power to depth-$k$ threshold circuits
with polynomially-bounded weights: $\textsf{MAJ}_k = \hat{\textsf{LT}}_k$
\cite{Alon1994,Goldmann1994}.

In many works on classical majority circuits, the fanin of a majority
gate is not treated as an important consideration and can in 

\begin{theorem}{Multiple product in constant depth and polynomial size.\cite{Yeh1996}}
The $n^2$-bit product of $n\times n$-bit numbers can be computed by a
circuit of depth $O(\frac{1}{\epsilon})$,
size and width $O(\frac{1}{\epsilon}n^{3+2\epsilon})$, and
fanin $O(n)$.
\end{theorem}

\begin{theorem}{Modular reduction in constant depth and polynomial size \cite{Yeh1996}}
The $n$-bit binary representation of the modular residue $x \bmod m$, where
$x$ is an $n^2$-bit number and $m$ is an $n$-bit modulus, can be computed
by a circuit of depth $O(\frac{1}{\epsilon})$,
size and width $O(\frac{1}{\epsilon}n^{1 + 2\epsilon})$, and
fanin $O(n^2)$.
\end{theorem}



\section{A Majority Gate in 2D CCNTCM in Sub-logarithmic Depth}

To map a majority gate to 2D CCNTCM, we use the $\textsf{AC}_f^0$ 
constructions from Takahashi-Tani \cite{Takahashi2011},
based on the results of
Hoyer-Spalek \cite{Hoyer2002}.
We combine this with our architectural techniques
from the previous chapter, including the 2D CCNTCM circuits
for constant-depth teleportation, fanout, and unfanout.

The majority gate is a threshold gate, with unit weights, a
fan-in (input size) $n' = O(n)$ in the size of the input number to Shor's 
factoring algorithm, and a threshold of $\lceil n/2 \rceil$.

We describe the construction for $\text{MAJ}_{n'}$ below, given the
$n'$-qubit input $x$.

\begin{enumerate}

\item
We compute in parallel the gates $\text{EX}^i_{n}$ for
$0 \le i \le \lceil n/2 \rceil$ to determine if the quantum
threshold for majority is reached. There are at most $(n/2) + 1$
such gates.

\begin{enumerate}
\item 
Compute the constant-depth reduction from $\text{EX}^t_{n'}$ to
$\text{EX}^t_{m}$ where $m = \lceil \log_2(n'=1) \rceil$, using
the reduction from $OR_n$ to $OR_{\log_n}$ \cite{Hoyer2002}.
For $1 \le k \le m$, do the following:

\begin{enumerate}
\item
Compute the qubit $\mu^{|x|-t}_{\phi_k}$, which is the rotation by Hamming 
weight of $x$, with a threshold $t$ subtracted, by the angle $\phi_k = 2\pi / 2^k$. Note that the precision
of this angle is $O(\log \log n)$. This can be done by a 2D CCNTCM circuit
with $O(1)$-depth, $O(n^2)$-size, and $O(n^2)$-width, as described in
Section \ref{subsec:or-reduce}.
\end{enumerate}

At the end of this step, we have $O(\log_2 n)$ bits $\ket{y_k}$ which 
represent the degree to which the Hamming weight of $x$ is close to the
threshold $t = \lceil n/2 \rceil$.

\item
Apply the circuit for exact $\text{OR}_{\log n}$ from Lemma 2 of
\cite{Takahashi2011} to the output of the previous step. This can
be done with a 2D CCNTCM circuit with $O(1)$-depth, $O(n \log n)$-size,
and $O()$.

\end{enumerate}

\item
Apply the gate $\text{PA}_{\lceil n/2 \rceil}$ to the result of
the previous step. This can be done by a 2D CCNTCM circuit of
$O(1)$-depth, $O(n)$-size, and $O(n)$-width using constant-depth
fanout, as described in Section \ref{sec:cdc}, and conjugated by
Hadamards on every qubit as described in \cite{Moore1998}.

\item
Apply a NOT to the output of the previous step. This final
output is the output of the quantum majority gate $MAJ_{n}$.

\end{enumerate}


\subsection{OR Reduction of Hoyer-Spalek}
\label{subsec:or-reduce}

This step involves $m \le (n/2)+1$ gates $\mu^{|x|-t}_{\phi_k}$ for
$1 \le k \le m$. Each gate is the same as Hoyer and Spalek's reduction from
$O(n)$ bits to $O(\log n)$ bits as described in \cite{Hoyer2002}.

The $\mu$ circuits 

\subsection{Exact OR Circuit of Takahashi-Tani}
\label{subsec:or-exact}



\subsection{Parity Circuit}
\label{subsec:parity}



\section{Improving the Depth Beyond Sublogarithmic}

It is now natural to ask, given such dramatic improvement in
nearest-neighbor circuit
depth from quadratic \cite{Kutin2006} to polylogarithmic, can we
decrease depth further? Surprisingly, the answer is yes. In this
chapter we now decrease the depth below polylogarithmic, in fact,
to be $O((\log \log n)^2)$.

\section{The Block-Save Technique}

This doesn't really go anywhere, but I will mention here out of
completeness.
As an aside, one may think that one can iterate the carry-save
adder. Instead of re-encoding the sum of 3 bits as the sum of
2 bits, we could also re-encode the sum of 7 bits to be the
sum of 3 bits. In analogy to the 3-2 adder, we call this a 7-3 adder.
How would this re-encoding work?

In the 3-2 adder, there were two output bits of significance $1$
and $2$. The $1$-bit was the parity of the three bits and
the $2$-bit was the majority of the three bits. The truth tables
for these functions are shown in Table \ref{tab:3-2}

\begin{table}
\begin{tabular}{cc|c}
\hline
$x_0$ & $x_1$ & $c$ \\
\hline
0 & 0 & 0 \\
0 & 1 & 1 \\
1 & 0 & 1 \\
1 & 1 & 0 \\
\hline
\end{tabular}
\caption{Truth tables for}
\label{tab:3-2}
\end{table}

 In a 7-3 adder,
likewise, the $1$-bit is the parity of \emph{single} bits.
The $2$-bit is the parity of \emph{pairs} of bits, 

\section{Circuit Complexity Classes}

A circuit is a directed acyclic graph in which the nodes are
logical gates drawn from a certain (universal) set and the edges 
represent
the connection of the output of one gate to the input of another
gate. For classical circuits, they implement Boolean functions, which take in $n$ input
bits to one output bit.

\begin{equation}
f:{0,1}^n \rightarrow {0,1}
\end{equation}

For quantum circuits, they implement reversible functions on
$(n+1)$-qubits.


 We can also define special nodes which are not gates, but rather
are placeholder ``sources'' which provide the inputs to the circuit and 
``sinks'' which provide the outputs to the circuit. The in-degree of a 
node is also known as its \emph{fanin} and the out-degree of a node is
also known as its \emph{fanout}.

We denote a gate by its fanin as a subscript and an optional
second parameter as a superscript.

\begin{equation}
\text{GATE}_n^k
\end{equation}

\subsection{Classical Circuit Complexity Classes}

It is useful to define complexity classes of circuits based on the
set of allowed gates. In classical circuits, we take unbounded fanout
for granted (any node can have arbitrary out-degree). These are common
in the literature of classical circuits. We will list them in order
of the size of their universal set, where each subsequent class adds
more gates.
Obviously, since the set is universal to begin with, adding more gates
does not increase the power of the class to represent more Boolean
functions. Rather, it can decrease the depth or size of the circuit.
Once we define the classes, we can design subclasses.

\begin{definition}
\item[\textsf{NC}]
circuits consisting of $\text{NOT}_1$ and $\text{AND}_2$ and
$\text{OR}_2$ gates.
\item[\textsf{AC}]
NC circuits augmented with $\text{AND}_n$ and $\text{OR}_n$ gates,
for $n \ge 2$.
\item[\textsf{TC}]
AC circuits augmented with $\text{TH}_n^t$ gates, for $n \ge 2$ and
$0 \le t \le n$.
\end{definition}

A $\text{TH}^t_n$ gate is a threshold gate that is $1$ if the number
of input bits in greater than or equal to the threshold $t$ and $0$
otherwise.

\begin{equation}
% TODO insert block piecewise bracket
\end{equation}

We are often interested in the computing power of the above
circuit classes restricted in some way, usually shallow depth.
We denote by a superscript $k$ a complexity class of
functions implementable by circuits of depth $k$.

For these classical circuit classes, it is known that containment
is proper.

\begin{equation}
\textsf{NC}^0 \subsetneq \textsf{AC}^0 \subsetneq \textsf{TC}^0
\end{equation}

\subsection{Quantum Gates}

Unbounded fanout is taken for granted in classical circuit classes
because it is physically realistic to implement using electrical
devices. However, for quantum circuits, we must make use of the
unbounded quantum fanout, $\text{FANOUT}_n$, to copy one output
qubit of each gate.
We can define quantum analogues of the above circuit complexity 
classes by defining quantum analogues for each of the gates which
is reversible.
$NOT_1$ is already reversible, and we can use it as is.
To replace $AND_2$ we can use the reversible $3$-qubit Toffoli gate,
the so-called controlled-controlled-NOT.
To replace $OR_2$ we can use the circuit given in
Figure \ref{fig:or2}

\begin{figure}
% TODO 
\caption{A perfectly normal $\text{OR}_2$ gate minding its own business.}
\label{fig:or2}
\end{figure}

In fact this is special case of a much more powerful construction
that will let us define a quantum $OR_n$ gate on unbounded inputs.

\section{Quantum Compiling Rotations from a Fixed Set}

WAIT! Insight. We only need to compile gates of the form
$2\pi / n\log n$. I don't think that actually gives us anything.
But if we use a co-prime multiple of $2^n$.

\section{Circuit Resources}
