This is the first of two chapters which contribute a low-depth
implementation of Shor's factoring algorithm on our hybrid nearest-neighbor
architecture, \textsf{2D CCNTCM}. In support of our thesis,
we are studying the depth of factoring architectures by minimizing
their depth, and therefore decreasing the running time of this
particular algorithm. Our main result is the
factoring of an
$n$-bit number in circuit depth $O(\log^2(n))$
with a circuit size and width of $O(n^4)$.
This is an exponential improvement in circuit depth
over all previous implementations
\cite{Beauregard2002,Kutin2006,VanMeter2006,VanMeter2005,VanMeterIL2005}
at the cost of a polynomial increase in circuit size and width.

We use the
following three key techniques:
parallel phase estimation due to Kitaev-Shen-Vyalyi \cite{Kitaev2002};
constant-depth communication introduced in Section \ref{sec:intro-cdc}
including our circuit for quantum unfanout,
and 
constant-depth carry-save modular addition due to Gossett \cite{Gossett1998}.

Section \ref{sec:fpl-related} places our work in the context of existing
results.
In Section \ref{sec:csa}, we provide a self-contained pedagogical review
of the carry-save technique and encoding.
We use this foundation to construct building blocks of increasing complexity,
deriving numerical upper bounds on circuit and module resources
on \textsf{2D CCNTCM} as we go along.
In Section \ref{sec:csa-mod-add} we modify and extend the carry-save technique to a 2D
modular adder,
which we then use as a basis for a modular multiplier
(Section \ref{sec:csa-mod-mult}) and a modular exponentiator
(Section \ref{sec:modexp}), which contains our
main result of polylogarithmic-depth factoring.


In Section \ref{sec:fpl-results}, we provide a comparison of our \textsf{2D CCNTCM}
hybrid factoring architecture to all previous nearest-neighbor factoring
implementations. We also compare our hybrid architecture with a
variant on the non-hybrid model \textsf{2D CCNTC} to show the
benefit of using modules. Finally, we conclude our exploration
of polylogarithmic-depth hybrid factoring in Section
\ref{sec:fpl-conclude}.