\section{Conclusion}
\label{sec:fpl-conclude}

In this chapter, we've presented the first main result of
our dissertation in Section \ref{sec:modexp}:
factoring on a hybrid nearest-neighbor architecture in polylogarithmic depth.
We've place it in the context of previous factoring implementations on a
variety of different models. We provided a firm grounding
in the carry-save technique before using it to construct
a modular adder, a modular multiplier, and finally a
modular exponentiator. Combining the carry-save technique
with constant-depth
communication and parallel phase estimation yielded
this exponential improvement over the previous
state-of-the-art, a quadratic-depth nearest-neighbor
factoring architecture.

We have also examined the
effect of introducing modules by comparing our
implementation on both a hybrid model \textsf{2D CCNTCM}
and a non-hybrid model \textsf{2D CCNTC}.
Our hybrid model has reduced circuit resources
($D$, $S$, $W$) which better
capture the essential computational locality of Shor's factoring
algorithm while measuring part of the communication
costs in the module resources ($\overline{D}$, $\overline{S}$, $\overline{W}$).

The key point in this chapter is that
low-depth (sub-linear) factoring is possible
on a hybrid nearest-neighbor architecture
at only a polynomial increase in circuit size
and 
circuit width. It is possible to decrease
circuit size and width further by increasing
module size and module width.

Using our numerical
upper bounds, we can compute the amount of
physical resources to compromise a $4096$-bit
RSA key, currently regarded as a secure key size
for at least several decades. At a rate of
1 millisecond per timestep or long-range
teleportation, our implementation would
complete in 10 days.
 Our architecture
would require 5,736 communicating quantum
computers. If each quantum computer generated
enough entangled pairs for long-range
teleportation at a rate of 1 Hz, it would
take 4.5 hours to generate all required pairs.
At the current cost of residential electricity 
in Seattle of 10.71 cents per kilowatt-hour and
at least 100 millijoules of laser
power per gate,
the architecture would cost $2.8$ billion
to power.
 Given a typical
separation of ions of 5 microns\footnote{Provided by Tom Noel for barium ions in the lab of Boris Blinov.},
the entire apparatus would occupy at least 1.26
square miles, about twice the land area of the city
of Seattle. Such figures, while large, are
within the realm of possibility in 50 years, especially
for governments.\footnote{For example,
1.26 square miles in low earth orbit would be cheap.
However, the commute would be quite expensive.}

Even if our figures were off by an order of magnitude,
our results would still confirm our thesis. We
would have provided a quantum architecture to solve
the human problem of factoring that is
buildable and runnable within the author's lifetime.
However, progress rarely stops with what is
feasible and usually moves on to what is possible.
Can the depth of factoring be improved even futher?
This question will be examined
in the next chapter.

The work in this chapter was partially supported by
Microsoft Research. The latest version is
available on the arXiv \cite{Pham2012}.