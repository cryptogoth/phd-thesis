\section{Asymptotic Results}
\label{sec:fpl-results}

The asymptotic resources required for our approach,
as well as the resources for other nearest-neighbor approaches,
are listed in Table \ref{tab:fpl-results},
where we assume a fixed constant error
probability for each round of QPF. Not all resources are
provided directly by the referenced source.

Resources in square brackets
are inferred using Equation \ref{eqn:depth-width}.
These upper bounds are correct,
but may not be tight with the upper bounds
calculated by their respective authors.
In particular, a more detailed analysis
could give a better upper bound for circuit size than the
depth-width product.
These related works all use architectural models defined in
Sections \ref{sec:intro-arch} and \ref{sec:intro-modules}.
 Also note that the
work by Beckman et al. \cite{Beckman1996} is unique in that it uses
efficient multi-qubit gates inherent to linear ion trap technology which
at first seem to
be more powerful than \textsf{1D NTC}. However, use of these gates does not
result in an asymptotic improvement over \textsf{1D NTC}.

%, say $\epsilon=1/4$.
% and $\delta' = 1/2$ for KSV-QPF.
%Note that the
%number of measurements are included for completeness.
%, since these are
%not counted as gates in our model but may be comparable in terms of
%execution time.
%Some table cells are blank if the entries are not relevant to the current comparison, or if the entires were not %calculated in the prior work.
We achieve an exponential
improvement in hybrid nearest-neighbor circuit depth (from quadratic to polylogarithmic)
with our approach at the cost of a polynomial increase in
circuit size and width.
Similar depth improvements at the cost of width increases can be achieved
in other factoring implementations simply by rearranging their
existing modular multipliers in a parallel, logarithmic-depth binary tree.
Our approach is the first implementation for factoring on any
\textsc{2D} model. The resulting circuit size $S$ (which measures
intra-module computation and communication) and module size $\overline{S}$
(which measures inter-module communication)
both have the same asymptotic order of growth: $O(n^4)$. This satisfies
our heuristic from Section \ref{subsec:module-compare} to choose
a module size such that roughly equal resources are devoted to
computation and communication.

One objection to our results is that it is not a purely nearest-neighbor approach,
and therefore not a fair comparison to other nearest-neighbor implementations.
To address this issue, we have presented a variation of our hybrid architecture
that runs on the non-hybrid model \textsf{2D CCNTC}. That is, all long-range
interactions are directly simulated using constant-depth communication on a
contiguous \textsc{2D} lattice, without accounting for computation and
communication operations separately. The resulting depth is asymptotically
the same as before ($O(\log^2 n)$) but the size and width have now
increased to $O(n^6)$ versus $O(n^4)$. While is is only a polynomial increase
asymptotically, the actual numerical bounds may make it vastly intractable
to implement. We present the formula for \textsf{2D CCNTC} circuit size below
for comparison to Equation \ref{eqn:sme}.
%
\begin{equation}
\tilde{S}_{ME} = 245955 n^6 + 3.3n^5 + 1.6n^4 + O(n^3)
\end{equation}
%
\begin{table}[htb!]
\begin{center}
\begin{tabular}{|c|c|c|c|c|}
\hline
Implementation             & Architecture      & $D$   & $S$  & $W$     \\
\hline
Vedral, et al. \cite{Vedral1996}   & \textsf{AC}      & $[O(n^3)]$ & $O(n^3)$    & $O(n)$ \\
Gossett \cite{Gossett1998}                   & \textsf{AC}       & $O(n \log n)$  & $[O(n^3\log n)]$  & $O(n^2)$  \\
Beauregard \cite{Beauregard2002}                & \textsf{AC}       & $O(n^3)$      & $O(n^3 \log n)$ & $O(n)$ \\
Zalka \cite{Zalka1998}                     & \textsf{AC}       & $O(n^2)$      & $[O(n^3)]$ & $O(n)$     \\
Takahashi \& Kunihiro \cite{Takahashi2006}     & \textsf{AC}       & $O(n^3)$      & $O(n^3\log n)$ & $O(n)$ \\
Cleve \& Watrous \cite{Cleve2000}           & \textsf{AC}       & $O(\log^3 n)$ & $O(n^3)$ & $[O(n^3 / \log^3n)]$ \\
\hline
Beckman et al. \cite{Beckman1996} & \textsf{ION TRAP}   & $O(n^3)$ & $O(n^3)$ & $O(n)$\\
\hline
Fowler, et al. \cite{Fowler2004} & \textsf{1D NTC}   & $O(n^3)$ & $O(n^4)$ & $O(n)$\\
Van Meter \& Itoh \cite{VanMeter2006} & \textsf{1D NTC}   & $O(n^2 \log n)$ & $[O(n^4\log n)]$ & $O(n^2)$\\
Kutin \cite{Kutin2006}                     & \textsf{1D NTC}   & $O(n^2)$ & $O(n^3)$ & $O(n)$\\
\hline
                           & \textsf{2D CCNTC}    & $O(\log^2{n})$ & $O(n^6)$ & $O(n^6)$   \\
Current Work               & \textsf{2D CCNTCM}   & $O(\log^2{n})$ & $O(n^4)$ & $O(n^4)$   \\
                           &                      & $\overline{D}$ & $\overline{S}$ & $\overline{W}$ \\
                           &                      & $O(\log^2{n})$ & $O(n^4)$ & $O(n^3)$ \\
\hline
\end{tabular}
\end{center}
\caption{Asymptotic circuit resource usage for quantum factoring of an $n$-bit number.}
\label{tab:fpl-results}
\end{table}