\section{Asymptotic Results}
\label{sec:fpl-results}

The asymptotic resources required for our approach,
as well as the resources for other nearest-neighbor approaches,
are listed in Table \ref{tab:results},
where we assume a fixed constant error
probability for each round of QPF. Not all resources are
provided directly by the referenced source.

Resources in square brackets
are inferred using Equation \ref{eqn:depth-width}.
These upper bounds are correct,
but may not be tight with the upper bounds
calculated by their respective authors.
In particular, a more detailed analysis
could give a better upper bound for circuit size than the
depth-width product. Also note that the
work by Beckman et al. \cite{Beckman1996} is unique in that it uses
efficient multi-qubit gates inherent to linear ion trap technology which at first
seem to
be more powerful than \textsc{1D NTC}. However, use of these gates does not result in an
asymptotic improvement over \textsc{1D NTC}.

%, say $\epsilon=1/4$.
% and $\delta' = 1/2$ for KSV-QPF.
%Note that the
%number of measurements are included for completeness.
%, since these are
%not counted as gates in our model but may be comparable in terms of
%execution time.
%Some table cells are blank if the entries are not relevant to the current comparison, or if the entires were not %calculated in the prior work.
We achieve an exponential
improvement in nearest-neighbor circuit depth (from quadratic to polylogarithmic)
with our approach at the cost of a polynomial increase in
circuit size and width. Similar depth improvements at the cost of width increases can be achieved using the modular multipliers
of other factoring implementations
by arranging them in a parallel modular exponentiator.
Our approach is the first implementation for factoring on \textsc{2D NTC},
augmented with a classical controller and parallel, communicating
modules (\textsc{2D CCNTCM}).
%
\begin{table}[htb!]
\begin{center}
\begin{tabular}{|c|c|c|c|c|}
\hline
Implementation             & Architecture      & Depth   & Size   & Width     \\
\hline
Vedral, et al. \cite{Vedral1996}   & \textsc{AC}      & $[O(n^3)]$ & $O(n^3)$    & $O(n)$ \\
Gossett \cite{Gossett1998}                   & \textsc{AC}       & $O(n \log n)$  & $[O(n^3\log n)]$  & $O(n^2)$  \\
Beauregard \cite{Beauregard2002}                & \textsc{AC}       & $O(n^3)$      & $O(n^3 \log n)$ & $O(n)$ \\
Zalka \cite{Zalka1998}                     & \textsc{AC}       & $O(n^2)$      & $[O(n^3)]$ & $O(n)$     \\
Takahashi \& Kunihiro \cite{Takahashi2006}     & \textsc{AC}       & $O(n^3)$      & $O(n^3\log n)$ & $O(n)$ \\
Cleve \& Watrous \cite{Cleve2000}           & \textsc{AC}       & $O(\log^3 n)$ & $O(n^3)$ & $[O(n^3 / \log^3n)]$ \\
\hline
Beckman et al. \cite{Beckman1996} & \textsc{Ion trap}   & $O(n^3)$ & $O(n^3)$ & $O(n)$\\
\hline
Fowler, et al. \cite{Fowler2004} & \textsc{1D NTC}   & $O(n^3)$ & $O(n^4)$ & $O(n)$\\
Van Meter \& Itoh \cite{VanMeter2006} & \textsc{1D NTC}   & $O(n^2 \log n)$ & $[O(n^4\log n)]$ & $O(n^2)$\\
Kutin \cite{Kutin2006}                     & \textsc{1D NTC}   & $O(n^2)$ & $O(n^3)$ & $O(n)$\\
\hline
Current Work               & \textsc{2D CCNTCM}   & $O(\log^2{n})$ & $O(n^4)$ & $O(n^4)$   \\
\hline
\end{tabular}
\end{center}
\tcaption{Asymptotic circuit resource usage for quantum factoring of an $n$-bit number.}
\label{tab:results}
\end{table}