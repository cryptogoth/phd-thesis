%%%%%%%%%%%%%%%%%%%%%%%%%%%%%%%%%%%%%%%%%%%%%%%%%%%%%%%%%%%%%%%%%%%%%%%%%%%%%%
\section{Related Work}
\label{sec:fpl-related}

Our work builds upon ideas in classical digital and reversible logic and their extension to quantum logic.
Any circuit implementation for Shor's algorithm requires a quantum adder.
Gossett proposed a quantum algorithm for addition using classical carry-save techniques to add
in constant-depth and multiply in logarithmic-depth, with a quadratic
cost in qubits (circuit width) \cite{Gossett1998}. The techniques relies on encoded addition, sometimes
called a $3 \rightarrow 2$ adder, and derives from classical Wallace trees \cite{Wallace1964}.

Takahashi and Kunihiro discovered a linear-depth
and linear-size quantum adder using zero ancillae \cite{Takahashi2005}.
They also developed an adder with tradeoffs between $O(n/d(n))$ ancillae and
$O(d(n))$-depth for $d(n) = \Omega(\log n)$ \cite{Takahashi2009}. 
Their approach assumes unbounded fanout, which had not previously been mapped to a
nearest-neighbor circuit until our present work.

Studies of architectural constraints, namely restriction to a 2D planar layout, 
were experimentally motivated. For example, these layouts correspond
to early ion trap proposals \cite{Kielpinski2002}
and were later analyzed at the level of physical qubits and error correction in the context of Shor's algorithm \cite{Kubi09}.
Choi and Van Meter designed one of the first adders targeted to a 2D architecture 
and showed it runs in $\Theta(\sqrt{n})$-depth on \textsf{2D NTC} \cite{Choi2010}
using $O(n)$-qubits with dedicated, special-purpose areas of a physical
circuit layout.

%Once an adder implementation is chosen, it can be extended 
%To perform modular reduction, modular multiplication, 
%Modular exponentiation, and ultimately
%quantum period finding (QPF), the only quantum part of the factoring algorithm.
Modular exponentiation is a key component of quantum period-finding (QPF),
and its efficiency relies on that of its underlying adder implementation.
Since Shor's algorithm is a probabilistic algorithm, multiple rounds of
QPF are required to amplify success probability arbitrarily close to 1.
It suffices to determine the resources
required for a single round of QPF with a fixed, modest success probability
(in the current work, $3/4$).

The most common approach to QPF performs controlled
modular exponentiation followed by an inverse quantum Fourier transform
(QFT) \cite{Nielsen2000}. We will call this \emph{serial QPF}, which is
used by the following implementations.
%Quantum circuits proposed for factoring on a nearest-neighbor architecture have assumed a serial QPF circuit.

Beauregard \cite{Beauregard2002}
constructs a cubic-depth quantum period-finder using only $2n+3$ qubits on
\textsf{AC}.
It combines the ideas of Draper's transform adder \cite{Draper2000},
Vedral et al.'s modular arithmetic blocks \cite{Vedral1996}, and a
semi-classical QFT.
This approach was subsequently adapted to \textsf{1D NTC} by Fowler, Devitt,
and Hollenberg
\cite{Fowler2004} to achieve resource counts for an $O(n^3)$-depth
quantum period-finder. Kutin \cite{Kutin2006} later improved this using
an idea from Zalka for approximate multipliers to produce a QPF circuit on
\textsf{1D NTC}
in $O(n^2)$-depth. Thus, there is only a constant overhead from
Zalka's own factoring implementation on \textsf{AC}, which also has
quadratic depth \cite{Zalka1998}.
Takahashi and Kunihiro extended their earlier $O(n)$-depth adder to a factoring
circuit in $O(n^3)$-depth with linear width \cite{Takahashi2006}.
Van Meter and Itoh explore many different approaches for serial QPF,
with their lowest achievable depth being $O(n^2\log n)$ with
$O(n^2)$ on \textsf{NTC} \cite{VanMeter2005}. Cleve and Watrous
calculate a factoring circuit depth of $O(\log^3 n)$ and corresponding
circuit size of $O(n^3)$ on \textsf{AC},
assuming an adder which has depth $O(\log n)$ and
$O(n)$ size and width. We beat this depth and provide a concrete
architectural implementation using an adder with $O(1)$-depth and $O(n)$
size and width.

In the current work, we assume that errors do not affect the storage of qubits
during the circuit's operation. An alternate approach is taken by
Miquel \cite{Miquel1996} and Garcia-Mata \cite{GarciaMata2007}, who both
numerically simulate Shor's algorithm for factoring specific
numbers to determine its sensitivity to errors. Beckman et al. provide a
concrete factoring implementation in ion traps with $O(n^3)$ depth and size and
$O(n)$ width \cite{Beckman1996}.

In all the previous works,
it is assumed that qubits are expensive (width) and that
execution time (depth) is not the limiting constraint.
We make the alternative assumption that ancillae are cheap and that fast classical control
is available which allows access to all qubits in parallel.
Therefore, we optimize circuit depth at the expense of width.
We compare our work primarily to Kutin's method \cite{Kutin2006}.

These works also rely on serial QPF which in turn relies on an inverse QFT.
On an AC architecture, even when approximating the (inverse) QFT by truncating two-qubit
$\pi/2^k$ rotations beyond $k = O(\log n)$, 
the depth is $O(n \log n)$ to factor an $n$-bit number.
To be implemented fault-tolerantly on a quantum device, rotations in the QFT must then be compiled into a discrete gate basis.
This requires at least a $O(\log(1/\epsilon))$ overhead in depth to approximate a rotation with precision $\epsilon$ \cite{Harrow2002, Kitaev2002}.
We would like to avoid the use of a QFT due to its compilation overhead.

There is an alternative, parallel version of phase estimation 
\cite{Cleve2000,Kitaev2002}, which we call \emph{parallel QPF} (we refer the reader to Section 13 of \cite{Kitaev2002} for details), which decreases depth in exchange
for increased width and additional classical post-processing.
This eliminates the need to do an inverse QFT.
%We refer the reader to \cite{Kitaev2002} 
%and \cite{Pham2011b} 
%for details.
We develop a nearest-neighbor factoring circuit based on parallel QPF and our proposed 2D quantum arithmetic circuits.
We show that it is asymptotically more efficient than the serial QPF method. 
We compare the circuit resources required by our work with existing serial QPF implementations in Table
\ref{tab:fpl-results} of Section \ref{sec:fpl-results}.
However, a recent result by \cite{Jones2013} allows one to enact a
QFT using only Clifford gates and a Toffoli gate in $O(\log^2 n)$ expected depth.
This would allow us to
greatly improve the constants in our circuit resource upper bounds in Section \ref{sec:modexp} by combining a QFT with parallel multiplication similar to
the approach described in \cite{VanMeter2005,Cleve2000}.


We also note that recent results by Browne, Kashefi, and Perdrix (BKP) connect the power of
measurement-based quantum computing to the quantum circuit model augmented with
unbounded fanout \cite{Browne2009}. Their model, which we adapt and call
\textsf{CCNTC}, uses the classical controller mentioned in Section \ref{sec:cdc}.
Using results by H{\o}yer and {\v S}palek \cite{Hoyer2002} that
unbounded quantum fanout would allow for a constant-depth factoring algorithm,
they conclude that a probabilistic polytime classical machine with access
to a constant-depth one-way quantum computer would also be able to factor
efficiently.