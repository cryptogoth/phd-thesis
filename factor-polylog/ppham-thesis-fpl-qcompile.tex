\section{Controlled Rotations for Factoring}
\label{sec:factor-crz}

In Section \ref{sec:related}, we stated that we wished to avoid
factoring implementations that used a QFT due to the fine
single-qubit rotations involved. Due to the requirements of
fault-tolerance on a particular physical implementation,
we can usually implement only a set of gates that is fixed
(it does not change with the problem input size), discrete (of finite size),
and universal. This last property is necessary for us to approximate any
other gate \emph{not} in our set, especially single-qubit phase rotations
of the form $\Lambda(e^{i \phi})$. Such an approximation would involve
a quantum compiling procedure, such as Solovay-Kitaev, which is the
subject of Chapter \ref{chap:qcompile}. However, we mention it here
because the choice of our universal set determines the true depth
of any circuit.

In our polylogarithmic factoring implementation, we were able to reduce
all our arithmetic circuits to such a fixed, discrete universal set.
These arithmetic circuits are discrete and classical and nature, so it is
not surprising that we can implement them in a discrete way.
However, to reduce the depth further, we need to introduce the idea of
a quantum threshold gate, which requires controlled-rotations similar to
the QFT. This controlled-rotation gate is shown in Figure \ref{fig:crz}
along with its decomposition into CNOTs and single-qubit $R_Z(\phi)$
rotations.
These rotations depend on the input size.

If we were given pre-generated ancillae of the form
$\ket{0} + e^{i\phi}\ket{1}$, we could use the method of programmable ancillae rotation (PAR) from
\cite{Jones2012}
to apply the gate $R_Z(\phi)$ probabilistically
using the circuit from Figure \ref{fig:par}. Note that we have a $50\%$
chance
of applying the opposite rotation $-\phi$, and so we must chain
several rounds of the given circuit. In expectation, we achieve the
desired rotation in a constant depth of $2$, and now we must consider
all circuit resources in terms of expected (average) values. The 
probability of $k$ PAR rounds all failing is $2^{-k}$, and the cost
of the circuit depth 

\begin{description}
\item[expected circuit depth ($\langle D \rangle $):] the expected number of concurrent timesteps.
\item[expected circuit size ($\langle S \rangle$):] the total number of non-identity gates applies
from $\mathcal{G}$, equal to $\sum_{i=1}^D |G_i|$.
\item[circuit width ($W$):] is the total number of qubits operated upon by
any gate, including inputs, outputs, and ancillae. It is equal to $| \bigcup_{i=1}^D V_i|$.

\end{description}

, we can prepare these bits in advance, using one of
the quantum compiling methods described in Chapter \ref{chap:qcompile}, and 
We do not count the quantum compiling resources as part of the
circuit resources for factoring in this chapter. 

\begin{figure}[tbp!]
\begin{displaymath}
\Qcircuit @C=1.5em @R=1.5em {
\lstick{\ket{\psi}}                  & \qw & \targfix  & \qw & \gate{X^j} & \qw & \rstick{R_Z((-1)^j \phi)\ket{\psi}} \\
 %\normtwo \left( \ket{0} + e^{i\phi}\ket{1} \right) & \qw & \ctrl{-1} & \qw & \measureD  & \cw & \rstick{j}
\lstick{ \normtwo \left( \ket{0} + e^{i\phi}\ket{1} \right) } & \qw & \ctrl{-1} & \qw & \measureD{Z}  & \cw & \rstick{j}
 }
 \end{displaymath}
\caption{One round of programmable ancillae rotation (PAR) to probabilistically achieve arbitrary single-qubit rotations \cite{Jones2012}}
\label{fig:crz}
\end{figure}

\begin{figure}[tb!]
\begin{center}
\begin{displaymath}
\begin{array}{ccc}
\Qcircuit @C=1.5em @R=1.5em {
   & \qw      & \ctrl{1}                   & \qw \\
   & \qw      & \gate{\frac{\pi}{2^{d}}} & \qw \\
 }
&
\begin{array}{c}
\\
\\
\\
= \\
\end{array}
&
\Qcircuit @C=1.5em @R=1.5em {
& \qw & \qw & \qw & \ctrl{1} & \qw & \gate{\frac{\pi}{2^{d+1}}} & \qw & \ctrl{1} & \qw\\
 & \qw & \gate{\frac{\pi}{2^{d+1}}} & \qw & \targfix & \qw & \gate{\frac{\pi}{2^{d+1}}} & \qw & \targfix & \qw
}
\end{array}
\end{displaymath}
\caption{Decomposition of a controlled-$R_z$ rotation}
\label{fig:crz}
\end{center}\end{figure}


We now modify our 2D CCNTCM model from Section \ref{subsec:2dccntcm} to
allow teleportations from a separate ``quantum compiler'' module which
can produce ancillary qubits of the form 
$\ket{0} + e^{i\phi}\ket{1}$. The angles $\phi$ are determined by
the particular algorithm. In the case of majority gates for factoring,
as described in Section \ref{sec:factor-majority}, the angles are of the
form $\phi_k = \frac{2\pi}{2^k}$ where $k \le (\log_2 n' + 2)$ for a majority
gate with fanin $n'$. Since this represents the most fine-grained resolution of
rotation, our allowable error is $2^{-(k+1)} = 2^{-(\log_2 n' + 3)}$.

This quantum compiler resource has a parameter $\epsilon$ which is the
allowable error in the angles of the return PAR qubits.
We assume they have some precision $\epsilon = 2^{-k}$ from the
true desired angle $\tilde{\phi}$.

\begin{equation}
| \phi - \tilde{\phi} | \le \epsilon = 2^{-(k+1)} = 2^{-(\log_2 n' + 3)}
\end{equation}.

Often, quantum compiler resources are given in terms of the quantity
$(1/\epsilon)$, which we can re-express in terms of the
majority gate fanin as $O(1/n')$. 
While we will delay a discussion and comparison of
quantum compilers until Chapter \ref{chap:qcompile}, for now
we assume an upper bound of the Kitaev-Shen-Vyalyi (KSV) algorithm
using parallel phase estimation from Section 13 of \cite{Kitaev2002}.
This has the resources given in Table \ref{tab:fsl-ksv}.

\begin{table}[tbp!]
\begin{tabular}{|c|c|c|}
\hline
Circuit Size & $O((\log n')^2\log\log n')$ \\
\hline
Circuit Depth & $O((\log\log n')^2)\\
\hline
\caption{KSV quantum compiling resources for single-qubit rotations in majority gates of fanin $n'$.}
\label:tab:fsl-ksv}
\end{table}

We call this augmented model 2D CCNTCP and 

