\section{Constant-depth Teleportation, Fanout, and Unfanout}
\label{subsec:fanout}

Communication, namely the \emph{moving} and \emph{copying} of quantum information, in nearest-neighbor quantum architectures is challenging.
The first challenge of moving quantum information from one site to another over
arbitrarily long distances can be addressed by using
%A related problem is how to teleport a qubit an arbitrary distance.
% in an
%architecture through ancillae prepared in some initial state.
the constant-depth teleportation circuit
shown in Figure \ref{fig:cdt}, illustrated using standard quantum circuit
notation \cite{Nielsen2000}. This requires the circuit resources shown in
Table \ref{tab:cd-resources}. The depth includes a layer of $H$ gates; a layer of CNOTs; an interleaved layer of Bell basis measurements; and two layers of
Pauli corrections ($X$ and $Z$ for each qubit), occurring concurrently with
resetting the $\ket{j}$ and $\ket{k}$ qubits back to $\ket{0}$.
These correction layers are not shown in the circuit.

\begin{figure*}[tb!]
\begin{center}
\begin{displaymath}
%\begin{array}{ccc}
\Qcircuit @C=1em @R=1em {
\lstick{\ket{\psi}}	& \qw      & \qw      & \qw & \qw & \qw & \qw & \qw                                          & \qw & \qw & \multimeasureD{1}{\mbox{Bell}} & \cw & \rstick{j_1} \\
\lstick{\ket{0}}    & \gate{H} & \ctrl{1} & \qw & \qw & \qw & \qw & \qw                                          & \qw & \qw & \ghost{\mbox{Bell}}            & \cw & \rstick{k_1} \\
\lstick{\ket{0}}    & \qw      & \targfix & \qw & \qw & \qw & \qw & \qw_{Z^{j_1}X^{k_1}\ket{\psi}}               & \qw & \qw & \multimeasureD{1}{\mbox{Bell}} & \cw & \rstick{j_2} \\
\lstick{\ket{0}}    & \gate{H} & \ctrl{1} & \qw & \qw & \qw & \qw & \qw                                          & \qw & \qw & \ghost{\mbox{Bell}}            & \cw & \rstick{k_2} \\
\lstick{\ket{0}}    & \qw      & \targfix & \qw & \qw & \qw & \qw & \qw_{Z^{j_2}Z^{j_1}X^{k_2}X^{k_1}\ket{\psi}} & \qw & \qw & \multimeasureD{1}{\mbox{Bell}} & \cw & \rstick{j_3} \\
\lstick{\ket{0}}    & \gate{H} & \ctrl{1} & \qw & \qw & \qw & \qw & \qw                                          & \qw & \qw & \ghost{\mbox{Bell}}            & \cw & \rstick{k_3} \\
\lstick{\ket{0}}    & \qw      & \targfix & \qw & \qw & \qw & \qw & \qw & \qw_{Z^{j_1}Z^{j_2}Z^{j_3}X^{k_3}X^{k_2}X^{k_1}\ket{\psi}} & \qw & \qw              & \qw & \qw \\
}
\end{displaymath}
\centerline{}
\caption{Constant-depth circuit based on \protect{\cite{Broadbent2007,Browne2009}} for teleportation over $n=5$ qubits \protect{\cite{Rosenbaum2012}}.}
\label{fig:cdt}
\end{center}\end{figure*}

\begin{figure*}[tb!]
\begin{center}
\begin{displaymath}
%& \qquad \qquad \qquad &
\Qcircuit @C=1em @R=1em {
\lstick{\ket{\psi}}	& \qw      & \qw      & \qw & \qw & \qw & \multimeasureD{1}{\mbox{Bell}'} & \cw & \rstick{j_1} \\
\lstick{\ket{0}}    & \gate{H} & \ctrl{1} & \qw & \qw      & \qw & \ghost{\mbox{Bell}'}            & \cw & \rstick{k_1} \\
\lstick{\ket{0}_1}    & \qw      & \targfix & \qw & \ctrl{1} & \qw & \qw      & \qw & \rstick{Z^{j_1}X^{k_1}\ket{\ell}_1}\\
\lstick{\ket{0}}	& \qw      & \qw      & \qw & \targfix & \qw & \multimeasureD{1}{\mbox{Bell}} & \cw & \rstick{j_2} \\
\lstick{\ket{0}}    & \gate{H} & \ctrl{1} & \qw & \qw      & \qw & \ghost{\mbox{Bell}}           & \cw & \rstick{k_2} \\
\lstick{\ket{0}_2}    & \qw      & \targfix & \qw & \ctrl{1} & \qw & \qw      & \qw & \rstick{Z^{j_2}X^{k_2}X^{k_1}\ket{\ell}_2}\\
\lstick{\ket{0}}	& \qw      & \qw      & \qw & \targfix & \qw & \multimeasureD{1}{\mbox{Bell}} & \cw & \rstick{j_3} \\
\lstick{\ket{0}}    & \gate{H} & \ctrl{1} & \qw & \qw      & \qw & \ghost{\mbox{Bell}}           & \cw & \rstick{k_3} \\
\lstick{\ket{0}_3}    & \qw      & \targfix & \qw & \ctrl{1} & \qw & \qw      & \qw & \rstick{Z^{j_3}X^{k_3}X^{k_2} X^{k_1}\ket{\ell}_3}\\
\lstick{\ket{0}_4}	& \qw      & \qw      & \qw & \targfix & \qw & \qw      & \qw & \rstick{X^{k_3}X^{k_2} X^{k_1}\ket{\ell}_4}\\
}
%& & \\
%(a) & & (b)
%\end{array}
\end{displaymath}
\centerline{}
\caption{Constant-depth circuits based on \protect{\cite{Broadbent2007,Browne2009}} for fanout \protect{\cite{Harrow2012}} of one qubit to $n=4$ entangled copies.}
\label{fig:cdf}
\end{center}\end{figure*}

Although general cloning is
impossible \cite{Nielsen2000}, the second challenge of copying information can be addressed by performing an unbounded quantum
fanout operation:
$\ket{x,y_1,\ldots,y_n} \rightarrow \ket{x,y_1\oplus x, \ldots, y_n\oplus x}$.
This is used in our arithmetic circuits when
a single qubit needs to control (be entangled with) a large quantum register
(called a \emph{fanout rail}).
We employ a constant-depth circuit due to insight from
measurement-based quantum computing \cite{Raussendorf2003}
that relies on the creation of an
$n$-qubit cat state \cite{Browne2009}.

This circuit requires $O(1)$-depth, $O(n)$-size, and $O(n)$-width. Approximately
two-thirds of the ancillae are reusable and can be reset to $\ket{0}$ after
being measured. Numerical upper bounds are given in Table \ref{tab:cd-resources}.
The constant-depth fanout circuit is shown in Figure \ref{fig:cdf} for the case of fanning out a given single-qubit state
$\ket{\psi} = \alpha\ket{0} + \beta\ket{1}$ to four qubits.
The technique works by creating multiple small
cat states of a fixed size (in this case, three qubits), linking them
together into a larger cat state of unbounded size with Bell basis measurements,
and finally entangling them with the source qubit to be fanned out.
The qubits marked $\ket{\ell}$ are
entangled into the larger fanned out state given in Equation \ref{eqn:cat4}.
The Pauli corrections from the cat state creation are denoted by
$X^{k_2}$, $X^{k_3}$, $Z^{j_2}$ and $Z^{j_3}$ on qubits ending in
states $\ket{\ell}_1$, $\ket{\ell}_2$,
$\ket{\ell}_3$, and $\ket{\ell}_4$. The Pauli corrections
$X^{k_1}$ and $Z^{j_1}$ are from the Bell basis measurement
entangling the cat state with the source qubit (denoted $\text{Bell}'$).
\begin{equation}
Z_1^{j_1}X_1^{k_1}Z_2^{j_2}X_2^{k_2}X_2^{k_1}Z_{3}^{j_3}X_{3}^{k_3}X_{3}^{k_2}X_{3}^{k_1}X_{4}^{k_3}X_{4}^{k_2}X_{4}^{k_1}
\left(\alpha \ket{0}_1\ket{0}_2\ket{0}_3\ket{0}_4 + \beta \ket{1}_1\ket{1}_2\ket{1}_3\ket{1}_4 \right)
\label{eqn:cat4}
\end{equation}
%
The operators $X^k_i$ and $Z^j_{h}$ denote Pauli $X$ and $Z$ operators
on qubits $i$ and $h$, controlled by classical bits $k$ and $j$,
respectively. These corrections are enacted by the classical controller based on
the Bell measurement outcomes (not depicted).
Note the cascading nature of these corrections.
There can be up to
$n-1$ of these $X$ and $Z$
corrections on the same qubit, which can be simplified by the classical
controller to a single $X$ and $Z$ operation and then applied with a circuit of
depth 2 and size 2. Also, given the symmetric nature of the cat state, there
is an alternate set of Pauli corrections which would give the same state and
is of equal size to corrections given above.

Reversing the fanout (un-fanout) in constant depth is an interesting
problem. Doing so would allow us to improve the overall depth of our
factoring implementation to $O(\log^2 n)$ instead of $O(\log^3 n )$.
In this work it is sufficient to perform un-fanout using alternating rounds of
teleportation and CNOT among the $n$ fanned-out qubits in a logarithmic-depth
binary tree. The resources for this are given in
Table \ref{tab:cd-resources}.

% From Notebook #16, p. 212
% From Notebook #16, p. 66
\begin{table}
\begin{displaymath}
\begin{tabular}{|c|c|c|c|}
\hline
\text{Circuit Name} & \text{Depth} & \text{Size} & \text{Width}\\
\hline
\text{Teleportation from Figure \ref{fig:cdt}} & 7 & 3n + 4 & n+1\\
\hline
\text{Fanout from Figure \ref{fig:cdf}} & 9 & 10n - 9 & 3n-1 \\
\hline
\text{Un-fanout} & $8\log_2(2n)$ & $33n\log_2(2n) + 10\log^2_2(2n)$ & $3n-1$ \\
\hline
\end{tabular}
\end{displaymath}
\centerline{}
\caption{Circuit resources for teleportation, fanout, and un-fanout
(consisting of
alternating rounds of constant-depth teleportation and CNOT).}
\label{tab:cd-resources}
\end{table}

From an experimental perspective, it is physically efficient to create
a cat state in trapped ions using the M{\o}lmer-S{\o}rensen gate
\cite{Sorensen2000}\cite{Benhelm2008}. However, the fanout circuit for
the 2D CCNTCM model would still be useful for other technologies, such
as superconducting qubits on a two-dimensional lattice.

%Unfortunately, this
%``consumes'' the cat state in that there is no known way to unentangle the
%source qubit from the cat state after they have been jointly measured \cite{Rosenbaum2012}.

%\input{factor2d-related}
%%%%%%%%%%%%%%%%%%%%%%%%%%%%%%%%%%%%%%%%%%%%%%%%%%%%%%%%%%%%%%%%%%%%%%%%%%%%%%
