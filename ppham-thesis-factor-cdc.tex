%%%%%%%%%%%%%%%%%%%%%%%%%%%%%%%%%%%%%%%%%%%%%%%%%%%%%%%%%%%%%%%%%%%%%%%%%%%%%%%
\section{Constant-depth Teleportation, Fanout, and Unfanout}
\label{sec:cdc}

Communication, namely the \emph{moving} and \emph{copying} of quantum information, in nearest-neighbor quantum architectures is challenging.
In this section we quote known results for teleportation and
fanout in constant depth while also contributing a novel construction
for unfanout.

The first challenge of moving quantum information from one site to another over
arbitrarily long distances can be addressed by using
%A related problem is how to teleport a qubit an arbitrary distance.
% in an
%architecture through ancillae prepared in some initial state.
the constant-depth teleportation circuit
shown in Figure \ref{fig:cdt} due to Rosenbaum \cite{Rosenbaum2012}, illustrated using standard quantum circuit
notation \cite{Nielsen2000}. This requires the circuit resources shown in
Table \ref{tab:cd-resources}. The depth includes a layer of $H$ gates; a layer of CNOTs; an interleaved layer of Bell basis measurements; and two layers of
Pauli corrections ($X$ and $Z$ for each qubit), occurring concurrently with
resetting the $\ket{j}$ and $\ket{k}$ qubits back to $\ket{0}$.
These correction layers are not shown in the circuit.

\begin{figure*}[tb!]
\begin{center}
\begin{displaymath}
%\begin{array}{ccc}
\Qcircuit @C=1em @R=1em {
\lstick{\ket{\psi}}	& \qw      & \qw      & \qw & \qw & \qw & \qw & \qw                                          & \qw & \qw & \multimeasureD{1}{\mbox{Bell}} & \cw & \rstick{j_1} \\
\lstick{\ket{0}}    & \gate{H} & \ctrl{1} & \qw & \qw & \qw & \qw & \qw                                          & \qw & \qw & \ghost{\mbox{Bell}}            & \cw & \rstick{k_1} \\
\lstick{\ket{0}}    & \qw      & \targfix & \qw & \qw & \qw & \qw & \qw_{Z^{j_1}X^{k_1}\ket{\psi}}               & \qw & \qw & \multimeasureD{1}{\mbox{Bell}} & \cw & \rstick{j_2} \\
\lstick{\ket{0}}    & \gate{H} & \ctrl{1} & \qw & \qw & \qw & \qw & \qw                                          & \qw & \qw & \ghost{\mbox{Bell}}            & \cw & \rstick{k_2} \\
\lstick{\ket{0}}    & \qw      & \targfix & \qw & \qw & \qw & \qw & \qw_{Z^{j_2}Z^{j_1}X^{k_2}X^{k_1}\ket{\psi}} & \qw & \qw & \multimeasureD{1}{\mbox{Bell}} & \cw & \rstick{j_3} \\
\lstick{\ket{0}}    & \gate{H} & \ctrl{1} & \qw & \qw & \qw & \qw & \qw                                          & \qw & \qw & \ghost{\mbox{Bell}}            & \cw & \rstick{k_3} \\
\lstick{\ket{0}}    & \qw      & \targfix & \qw & \qw & \qw & \qw & \qw & \qw_{Z^{j_1}Z^{j_2}Z^{j_3}X^{k_3}X^{k_2}X^{k_1}\ket{\psi}} & \qw & \qw              & \qw & \qw \\
}
\end{displaymath}
\centerline{}
\fcaption{Constant-depth circuit based on \protect{\cite{Broadbent2007,Browne2009}} for teleportation over $n=5$ qubits \protect{\cite{Rosenbaum2012}}.}
\label{fig:cdt}
\end{center}\end{figure*}

\begin{figure*}[tb!]
\begin{center}
\begin{displaymath}
%& \qquad \qquad \qquad &
\Qcircuit @C=1em @R=1em {
\lstick{\ket{\psi}}	& \qw      & \qw      & \qw & \qw & \qw & \multimeasureD{1}{\mbox{Bell}'} & \cw & \rstick{j_1} \\
\lstick{\ket{0}}    & \gate{H} & \ctrl{1} & \qw & \qw      & \qw & \ghost{\mbox{Bell}'}            & \cw & \rstick{k_1} \\
\lstick{\ket{0}_1}    & \qw      & \targfix & \qw & \ctrl{1} & \qw & \qw      & \qw & \rstick{Z^{j_1}X^{k_1}\ket{\ell}_1}\\
\lstick{\ket{0}}	& \qw      & \qw      & \qw & \targfix & \qw & \multimeasureD{1}{\mbox{Bell}} & \cw & \rstick{j_2} \\
\lstick{\ket{0}}    & \gate{H} & \ctrl{1} & \qw & \qw      & \qw & \ghost{\mbox{Bell}}           & \cw & \rstick{k_2} \\
\lstick{\ket{0}_2}    & \qw      & \targfix & \qw & \ctrl{1} & \qw & \qw      & \qw & \rstick{Z^{j_2}X^{k_2}X^{k_1}\ket{\ell}_2}\\
\lstick{\ket{0}}	& \qw      & \qw      & \qw & \targfix & \qw & \multimeasureD{1}{\mbox{Bell}} & \cw & \rstick{j_3} \\
\lstick{\ket{0}}    & \gate{H} & \ctrl{1} & \qw & \qw      & \qw & \ghost{\mbox{Bell}}           & \cw & \rstick{k_3} \\
\lstick{\ket{0}_3}    & \qw      & \targfix & \qw & \ctrl{1} & \qw & \qw      & \qw & \rstick{Z^{j_3}X^{k_3}X^{k_2} X^{k_1}\ket{\ell}_3}\\
\lstick{\ket{0}_4}	& \qw      & \qw      & \qw & \targfix & \qw & \qw      & \qw & \rstick{X^{k_3}X^{k_2} X^{k_1}\ket{\ell}_4}\\
}
%& & \\
%(a) & & (b)
%\end{array}
\end{displaymath}
\centerline{}
\fcaption{Constant-depth circuits based on \protect{\cite{Broadbent2007,Browne2009}} for fanout \protect{\cite{Harrow2012}} of one qubit to $n=4$ entangled copies.}
\label{fig:cdf}
\end{center}\end{figure*}

Although general cloning is
impossible \cite{Nielsen2000}, the second challenge of copying information can be addressed by performing an unbounded quantum
fanout operation:
$\ket{x,y_1,\ldots,y_n} \rightarrow \ket{x,y_1\oplus x, \ldots, y_n\oplus x}$.
This is used in our arithmetic circuits when
a single qubit needs to control (be entangled with) a large quantum register
(called a \emph{fanout rail}).
We employ a constant-depth circuit due to insight from
measurement-based quantum computing \cite{Raussendorf2003}
that relies on the creation of an
$n$-qubit cat state \cite{Browne2009} which was communicated to
us by Harrow and Fowler \cite{Harrow2012}.

This circuit requires $O(1)$-depth, $O(n)$-size, and $O(n)$-width. Approximately
two-thirds of the ancillae are reusable and can be reset to $\ket{0}$ after
being measured. Numerical upper bounds are given in Table \ref{tab:cd-resources}.
The constant-depth fanout circuit is shown in Figure \ref{fig:cdf} for the case of fanning out a given single-qubit state
$\ket{\psi} = \alpha\ket{0} + \beta\ket{1}$ to four qubits.
The technique works by creating multiple small
cat states of a fixed size (in this case, three qubits), linking them
together into a larger cat state of unbounded size with Bell basis measurements,
and finally entangling them with the source qubit to be fanned out.
The qubits marked $\ket{\ell}$ are
entangled into the larger fanned out state given in Equation \ref{eqn:cat4}.
The Pauli corrections from the cat state creation are denoted by
$X^{k_2}$, $X^{k_3}$, $Z^{j_2}$ and $Z^{j_3}$ on qubits ending in
states $\ket{\ell}_1$, $\ket{\ell}_2$,
$\ket{\ell}_3$, and $\ket{\ell}_4$. The Pauli corrections
$X^{k_1}$ and $Z^{j_1}$ are from the Bell basis measurement
entangling the cat state with the source qubit (denoted $\text{Bell}'$).
\begin{equation}
Z_1^{j_1}X_1^{k_1}Z_2^{j_2}X_2^{k_2}X_2^{k_1}Z_{3}^{j_3}X_{3}^{k_3}X_{3}^{k_2}X_{3}^{k_1}X_{4}^{k_3}X_{4}^{k_2}X_{4}^{k_1}
\left(\alpha \ket{0}_1\ket{0}_2\ket{0}_3\ket{0}_4 + \beta \ket{1}_1\ket{1}_2\ket{1}_3\ket{1}_4 \right)
\label{eqn:cat4}
\end{equation}
%
The operators $X^k_i$ and $Z^j_{h}$ denote Pauli $X$ and $Z$ operators
on qubits $i$ and $h$, controlled by classical bits $k$ and $j$,
respectively. These corrections are enacted by the classical controller based on
the Bell measurement outcomes (not depicted).
Note the cascading nature of these corrections.
There can be up to
$n-1$ of these $X$ and $Z$
corrections on the same qubit, which can be simplified by the classical
controller to a single $X$ and $Z$ operation and then applied with a circuit of
depth 2 and size 2. Also, given the symmetric nature of the cat state, there
is an alternate set of Pauli corrections which would give the same state and
is of equal size to the corrections given above.

\begin{figure*}[tb!]
\begin{center}
\begin{displaymath}
\Qcircuit @C=1em @R=1em {
& \lstick{\ket{\ell}}	& \qw & \gate{H} & \qw & \ctrl{1} & \qw & \qw      & \qw &  \measureD{Z} & \cw & \rstick{j_1} & \\
& \lstick{\ket{\ell}}	& \qw & \gate{H} & \qw & \targfix & \qw & \ctrl{1} & \qw & \measureD{Z} & \cw & \rstick{j_2} & \\
& \lstick{\ket{\ell}}	& \qw & \gate{H} & \qw & \ctrl{1} & \qw & \targfix & \qw & \measureD{Z} & \cw & \rstick{j_3} & \\
& \lstick{\ket{\ell}}	& \qw & \gate{H} & \qw & \targfix & \qw & \ctrl{1} & \qw & \measureD{Z} & \cw & \rstick{j_4} & \\
& \lstick{\ket{\ell}}	& \qw & \gate{H} & \qw & \ctrl{1} & \qw & \targfix & \qw & \measureD{Z} & \cw & \rstick{j_5} & \\
& \lstick{\ket{\ell}}	& \qw & \gate{H} & \qw & \targfix & \qw & \ctrl{1} & \qw & \measureD{Z} & \cw & \rstick{j_6} & \\
& \lstick{\ket{\ell}}	& \qw & \gate{H} & \qw & \qw      & \qw & \targfix & \qw & \gate{H} & \qw & \rstick{Z^{j_2 \oplus j_4}(\alpha\ket{0} + \beta\ket{1})}
}
\end{displaymath}
\centerline{}
\fcaption{A novel, constant-depth circuit for unbounded quantum unfanout on
CCNTC, from the $7$-qubit entangled state $\alpha\ket{0}^{\otimes 7} + \beta\ket{1}^{\otimes 7}$ to the
target product state $(\alpha\ket{0} + \beta\ket{1})\otimes\ket{0}^{\otimes 6}$.}
\label{fig:cdu}
\end{center}\end{figure*}

Reversing the fanout is an operation called \emph{unfanout}. Unfanout
takes as input 
the following entangled $n$-qubit state which is the result of a fanout.

\begin{equation}
\normtwo (\ket{0}^{\otimes n} + \ket{1}^{\otimes n})
\label{eqn:fanned-out}
\end{equation}

The output of unfanout, after Pauli corrections, is the product state
consisting of all $\ket{0}$'s except for a single target qubit $\alpha\ket{0} + \beta\ket{1}$, which is in the
same state as the original source qubit of the fanout.

In the model of \cite{Hoyer2002}, the fanout and unfanout were identical, elementary
operations. In CCNTC, the operations are not identical due to
the one-way nature of the measurement and the
constraints of NTC. In Figure \ref{fig:cdu}, we contribute a novel quantum circuit for unbounded quantum
unfanout for $n=7$ in constant depth on 2D CCNTC.
Note that the state in Equation \ref{eqn:fanned-out}
is completely symmetric in that all qubits are
equivalent entangled copies of each other. Therefore, the asymmetry 
of the final target qubit is entirely determined by the unfanout circuit,
which in this case selects the bottom qubit in the figure.

The initial fanned out state lives in a $2$-dimensional subspace. The
round of Hadamard gates increases its dimension to $2^n$, and the two
interleaved layers of CNOTs in a sense ``disentangle'' the qubits from
one another, up to a Pauli $Z$ correction. This correction, on the
final target qubit, is controlled by the parity of the classical measurements
on every ``even'' qubit ($j_2$ and $j_4$ in the figure), excluding the 
next-to-last qubit ($j_6$ in the figure). Each measurement projects the state of the
target qubit
into a subspace with half the dimension, so $n-1$ measurements project
the target qubit into a final $2$-dimensional subspace, which is the
qubit $\alpha\ket{0} + \beta\ket{1}$.

Although the circuit show works for odd $n$, we can easily take into
account even $n$ with an initial CNOT to ``uncopy'' one qubit from its
neighbors. The unfanout circuit in Figure \ref{fig:cdu} is the
functional inverse of the
fanout circuit in 
Figure \ref{fig:cdf}, but it relies on the fanned-out qubits
being teleported back into adjacent positions,
which is only possible in a 2D layout.
The target qubit of unfanout is usually chosen to be in the same location
as the source qubit of the corresponding fanout. 
The resources for unfanout are given in
Table \ref{tab:cd-resources}.

% From Notebook #16, p. 212
% From Notebook #16, p. 66
\begin{table}
\begin{displaymath}
\begin{tabular}{|c|c|c|c|}
\hline
\text{Circuit Name} & \text{Depth} & \text{Size} & \text{Width}\\
\hline
\text{Teleportation from Figure \ref{fig:cdt}} & 7 & 3n + 4 & n+1\\
\hline
\text{Fanout from Figure \ref{fig:cdf}} & 9 & 10n - 9 & 3n-1 \\
\hline
\text{Unfanout} & $ 6 $ & $ 3n+2 $ & $ n$ \\
\hline
\end{tabular}
\end{displaymath}
\centerline{}
\tcaption{Circuit resources for teleportation, fanout, and unfanout
(consisting of
alternating rounds of constant-depth teleportation and CNOT).}
\label{tab:cd-resources}
\end{table}

From an experimental perspective, it is physically efficient to create
a cat state in trapped ions using the M{\o}lmer-S{\o}rensen gate
\cite{Sorensen2000}\cite{Benhelm2008}. However, the fanout circuit for
the 2D CCNTCM model would still be useful for other technologies, such
as superconducting qubits on a 2D lattice.
