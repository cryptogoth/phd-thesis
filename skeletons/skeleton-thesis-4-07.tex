\documentclass{article}

\usepackage{amssymb}

%    Q-circuit version 1.06
%    Copyright (C) 2004  Steve Flammia & Bryan Eastin

%    This program is free software; you can redistribute it and/or modify
%    it under the terms of the GNU General Public License as published by
%    the Free Software Foundation; either version 2 of the License, or
%    (at your option) any later version.
%
%    This program is distributed in the hope that it will be useful,
%    but WITHOUT ANY WARRANTY; without even the implied warranty of
%    MERCHANTABILITY or FITNESS FOR A PARTICULAR PURPOSE.  See the
%    GNU General Public License for more details.
%
%    You should have received a copy of the GNU General Public License
%    along with this program; if not, write to the Free Software
%    Foundation, Inc., 59 Temple Place, Suite 330, Boston, MA  02111-1307  USA

\usepackage[matrix,frame,arrow]{xy}
\usepackage{amsmath}
\newcommand{\bra}[1]{\left\langle{#1}\right\vert}
\newcommand{\ket}[1]{\left\vert{#1}\right\rangle}
    % Defines Dirac notation.
\newcommand{\qw}[1][-1]{\ar @{-} [0,#1]}
    % Defines a wire that connects horizontally.  By default it connects to the object on the left of the current object.
    % WARNING: Wire commands must appear after the gate in any given entry.
\newcommand{\qwx}[1][-1]{\ar @{-} [#1,0]}
    % Defines a wire that connects vertically.  By default it connects to the object above the current object.
    % WARNING: Wire commands must appear after the gate in any given entry.
\newcommand{\cw}[1][-1]{\ar @{=} [0,#1]}
    % Defines a classical wire that connects horizontally.  By default it connects to the object on the left of the current object.
    % WARNING: Wire commands must appear after the gate in any given entry.
\newcommand{\cwx}[1][-1]{\ar @{=} [#1,0]}
    % Defines a classical wire that connects vertically.  By default it connects to the object above the current object.
    % WARNING: Wire commands must appear after the gate in any given entry.
\newcommand{\gate}[1]{*{\xy *+<.6em>{#1};p\save+LU;+RU **\dir{-}\restore\save+RU;+RD **\dir{-}\restore\save+RD;+LD **\dir{-}\restore\POS+LD;+LU **\dir{-}\endxy} \qw}
    % Boxes the argument, making a gate.
\newcommand{\meter}{\gate{\xy *!<0em,1.1em>h\cir<1.1em>{ur_dr},!U-<0em,.4em>;p+<.5em,.9em> **h\dir{-} \POS <-.6em,.4em> *{},<.6em,-.4em> *{} \endxy}}
    % Inserts a measurement meter.
\newcommand{\measure}[1]{*+[F-:<.9em>]{#1} \qw}
    % Inserts a measurement bubble with user defined text.
\newcommand{\measuretab}[1]{*{\xy *+<.6em>{#1};p\save+LU;+RU **\dir{-}\restore\save+RU;+RD **\dir{-}\restore\save+RD;+LD **\dir{-}\restore\save+LD;+LC-<.5em,0em> **\dir{-} \restore\POS+LU;+LC-<.5em,0em> **\dir{-} \endxy} \qw}
    % Inserts a measurement tab with user defined text.
\newcommand{\measureD}[1]{*{\xy*+=+<.5em>{\vphantom{#1}}*\cir{r_l};p\save*!R{#1} \restore\save+UC;+UC-<.5em,0em>*!R{\hphantom{#1}}+L **\dir{-} \restore\save+DC;+DC-<.5em,0em>*!R{\hphantom{#1}}+L **\dir{-} \restore\POS+UC-<.5em,0em>*!R{\hphantom{#1}}+L;+DC-<.5em,0em>*!R{\hphantom{#1}}+L **\dir{-} \endxy} \qw}
    % Inserts a D-shaped measurement gate with user defined text.
\newcommand{\multimeasure}[2]{*+<1em,.9em>{\hphantom{#2}} \qw \POS[0,0].[#1,0];p !C *{#2},p \drop\frm<.9em>{-}}
    % Draws a multiple qubit measurement bubble starting at the current position and spanning #1 additional gates below.
    % #2 gives the label for the gate.
    % You must use an argument of the same width as #2 in \ghost for the wires to connect properly on the lower lines.
\newcommand{\multimeasureD}[2]{*+<1em,.9em>{\hphantom{#2}}\save[0,0].[#1,0];p\save !C *{#2},p+LU+<0em,0em>;+RU+<-.8em,0em> **\dir{-}\restore\save +LD;+LU **\dir{-}\restore\save +LD;+RD-<.8em,0em> **\dir{-} \restore\save +RD+<0em,.8em>;+RU-<0em,.8em> **\dir{-} \restore \POS !UR*!UR{\cir<.9em>{r_d}};!DR*!DR{\cir<.9em>{d_l}}\restore \qw}
    % Draws a multiple qubit D-shaped measurement gate starting at the current position and spanning #1 additional gates below.
    % #2 gives the label for the gate.
    % You must use an argument of the same width as #2 in \ghost for the wires to connect properly on the lower lines.
\newcommand{\control}{*-=-{\bullet}}
    % Inserts an unconnected control.
\newcommand{\controlo}{*!<0em,.04em>-<.07em,.11em>{\xy *=<.45em>[o][F]{}\endxy}}
    % Inserts a unconnected control-on-0.
\newcommand{\ctrl}[1]{\control \qwx[#1] \qw}
    % Inserts a control and connects it to the object #1 wires below.
\newcommand{\ctrlo}[1]{\controlo \qwx[#1] \qw}
    % Inserts a control-on-0 and connects it to the object #1 wires below.
\newcommand{\targ}{*{\xy{<0em,0em>*{} \ar @{ - } +<.4em,0em> \ar @{ - } -<.4em,0em> \ar @{ - } +<0em,.4em> \ar @{ - } -<0em,.4em>},*+<.8em>\frm{o}\endxy} \qw}
    % Inserts a CNOT target.
\newcommand{\qswap}{*=<0em>{\times} \qw}
    % Inserts half a swap gate. 
    % Must be connected to the other swap with \qwx.
\newcommand{\multigate}[2]{*+<1em,.9em>{\hphantom{#2}} \qw \POS[0,0].[#1,0];p !C *{#2},p \save+LU;+RU **\dir{-}\restore\save+RU;+RD **\dir{-}\restore\save+RD;+LD **\dir{-}\restore\save+LD;+LU **\dir{-}\restore}
    % Draws a multiple qubit gate starting at the current position and spanning #1 additional gates below.
    % #2 gives the label for the gate.
    % You must use an argument of the same width as #2 in \ghost for the wires to connect properly on the lower lines.
\newcommand{\ghost}[1]{*+<1em,.9em>{\hphantom{#1}} \qw}
    % Leaves space for \multigate on wires other than the one on which \multigate appears.  Without this command wires will cross your gate.
    % #1 should match the second argument in the corresponding \multigate. 
\newcommand{\push}[1]{*{#1}}
    % Inserts #1, overriding the default that causes entries to have zero size.  This command takes the place of a gate.
    % Like a gate, it must precede any wire commands.
    % \push is useful for forcing columns apart.
    % NOTE: It might be useful to know that a gate is about 1.3 times the height of its contents.  I.e. \gate{M} is 1.3em tall.
    % WARNING: \push must appear before any wire commands and may not appear in an entry with a gate or label.
\newcommand{\gategroup}[6]{\POS"#1,#2"."#3,#2"."#1,#4"."#3,#4"!C*+<#5>\frm{#6}}
    % Constructs a box or bracket enclosing the square block spanning rows #1-#3 and columns=#2-#4.
    % The block is given a margin #5/2, so #5 should be a valid length.
    % #6 can take the following arguments -- or . or _\} or ^\} or \{ or \} or _) or ^) or ( or ) where the first two options yield dashed and
    % dotted boxes respectively, and the last eight options yield bottom, top, left, and right braces of the curly or normal variety.
    % \gategroup can appear at the end of any gate entry, but it's good form to pick one of the corner gates.
    % BUG: \gategroup uses the four corner gates to determine the size of the bounding box.  Other gates may stick out of that box.  See \prop. 
\newcommand{\rstick}[1]{*!L!<-.5em,0em>=<0em>{#1}}
    % Centers the left side of #1 in the cell.  Intended for lining up wire labels.  Note that non-gates have default size zero.
\newcommand{\lstick}[1]{*!R!<.5em,0em>=<0em>{#1}}
    % Centers the right side of #1 in the cell.  Intended for lining up wire labels.  Note that non-gates have default size zero.
\newcommand{\ustick}[1]{*!D!<0em,-.5em>=<0em>{#1}}
    % Centers the bottom of #1 in the cell.  Intended for lining up wire labels.  Note that non-gates have default size zero.
\newcommand{\dstick}[1]{*!U!<0em,.5em>=<0em>{#1}}
    % Centers the top of #1 in the cell.  Intended for lining up wire labels.  Note that non-gates have default size zero.
\newcommand{\Qcircuit}{\xymatrix @*=<0em>}
    % Defines \Qcircuit as an \xymatrix with entries of default size 0em.


\begin{document}

\section{The Overall View}

In a collection of seminal works by Jehoshua Bruck and others,
the circuit complexity of computing various arithmetic functions
is studied using a linear threshold element (LTE) as a basic gate.
Interestingly, it turns out what many arithmetic functions needed
for Shor's factoring algorithm can be done in constant-depth in
a threshold circuit.

These include

\begin{enumerate}
\item Powering: the raising of an $n$-bit constant $b$ to an
$n$-bit power $\ket{x}$, yielding an $n^2$-bit final product $z$
as the first stage in modular exponentiation. This takes depth
4 in LTE's.

\item Division: finding the quotient of an $n^2$-bit integer $z$ with
an $n$-bit integer $m$, as the first step in modular reduction, which itself
is the second step in modular exponentiation.

\item Multiplication: finding the product of the quotient $q$ in the 
previous step and the divisor $b$ also from the previous step, to get
$y$, the largest multiple of $m$ less than $z$, as the second step
of modular reduction.

\item Subtraction: subtracting $y$ from $z$, to get the modular residue
$w = b^{\ket{x}} \bmod m$.
\end{enumerate}

Due to Hoyer and Spalek, it is known how to perform a quantum
threshold gate in constant-depth using constant-depth teleportation
and fanout. In particular, this was calculated by Andrea McCool to take
the following resources:

Fill this in here from Andrea's report, I think it was for AC.

Therefore, the main task of mapping these results from classical circuit
complexity to find a nearest-neighbor quantum architecture for factoring
is as follows:

\begin{enumerate}
\item Mapping the quantum threshold gate from AC to CCNTC, using the 
Clifford group and Toffoli and finding 
the overhead incurred in resources. Find a way to use Cody Jone's Fourier
state distillation.
\item Finding the circuit size, in LTE's, for the classical circuits above
(powering, division, subtraction, multiplication),
using constant-depth fanout and teleportation.
\item Finding the circuit width, in LTE's, for the classical circuits above,
(powering, division, subtraction, multiplication)
using constant-depth fanout and teleportation.
\item Calculate the lower bounds for the circuit size and width, using
the lemma mentioned in the next section.
\end{enumerate}

\section{Linear Threshold Elements}

The linear threshold element is a primitive which we assume can implement
any linear threshold function. This is a reasonable assumption given our
construction of a quantum threshold gate which can perform something
something.

Also fact check the resolution needed for phase rotations, in rotation
by Hamming value and by Hamming weight, needed for the quantum threshold 
gate.

\section{Summary of Bounded-Depth Quantum Circuit Complexity}

Here we can summarize the results from Takahashi and Tani, namely
something about $AC^0$ being properly contained in $TC^0$, when using
only quantum AND, quantum OR gates, but with the containment known to
collapse when we are allowed to use unbounded quantum-fanout.

\section{Motvations}

We have several motivations for studying the complexity of threshold
circuits. The first is that some basic functions such as PARITY and MAJORITY
are known to be impossible in constant-depth circuits consisting only of
AND, OR, XOR, and NOT. However, when considering threshold gates, or
a special case MAJORITY, as a basic gate, then complicated arithmetic
functions such as PARITY, ADDITION, EXACT, COUNT, and so forth, become
possible in constant-depth.

The second motivation is the use of beautiful new tools that were
seemingly unrelated previous to circuit complexity.

\begin{enumerate}

\item
The harmonic analysis of boolean 
functions, via the spectral norms of its coefficients,
which provides a beautiful connection between the idea of
a function's spectral norm to bound the number of terms in its
polynomial representation. We can characterize the terms in this
representation, and therefore lower bound the size of the circuit
needed to implement the function in LTE's, without knowing the
explicit function itself.
This uses the probabilistic method and the creation of random polynomials.
\item The polynomial representation of a Boolean function and its
relationship to the Sylvester-form of the Hadamard matrix.
\item Something else? I feel like there was another connection.

\end{enumerate}


\section{Spectral Norms}

Every Boolean function in $n$ variables
can be represented as a polynomial threshold 
function with an exponential number of terms and with
exponential weights. However, these can be difficult to implement.
We are interested in functions which can be implemented with
polynomially-bounded terms (so-called \emph{sparse polynomials})
and polynomially-bounded weights.

That is

\begin{equation}
f:\{+1, -1\} \rightarrow \{+1, -1\}
\end{equation}

where we assume without loss of generality that $F(X) \ne 0$:

\begin{equation}
f(X) = sgn(F(X)) = +1 \text{ if } F(X) > 0, -1 \text{ if } F(X) < 0
\end{equation}

and

\begin{equation}
F(X) = \sum_{\alpha \in \{0,1\}^n} w_{\alpha} X^{\alpha} \qquad
X^{\alpha} = \prod_{i=1}^n x_i^{\alpha_i}
\end{equation}

We call the weights $w_{\alpha}$ the spectral coefficients of $f$.
Although they can be real-valued, it is known that restricting them
to the rational numbers does not decrease the power of the resulting
class of boolean functions (citation needed).

In fact, there is a unique representation of every boolean function $f$
in terms of these coefficients, which we call the
\emph{polynomial representation} of $f$ in the monomials which consist
of all possible multilinear combinations of the input variables
$\{x_1, \ldots, x_n\}$.

It is now possible to define a norm on these spectral coefficients for
$f$, the so-called \emph{spectral norms} from which our lower-bounding
method applies. FACT-CHECK: Do these polynomials provide an upper bound
or a lower-bound? I think it is a lower-bound. The explicit constructions
are the ones that provide an upper bound.

The $L_1$ norm is defined as follows:

\begin{equation}
L_1(f) = \sum_{\alpha\in \{0,1\}^n} |w_{\alpha}|
\end{equation}

The $L_2$ norm is defined as follows:

\begin{equation}
L_2(f) = \sum_{\alpha\in \{0,1\}^n |w_{\alpha}|^2}
\end{equation}

The $L_{\infty}$ norm is defined as follows:

\begin{equation}
L_{\infty}(f) = \max_{\alpha\in \{0,1\}^n} |w_\alpha|
\end{equation}

We can state the following facts

\section{Uniqueness of Polynomial Representation}

We can in fact determine the spectral coefficients in the polynomial 
representation directly (and uniquely) from the boolean function
on $n$ variables using the $2^n \times 2^n$ Hadamard matrix:

\begin{equation}
A_{2^n} = H_{2^n}p_{2^n}
\end{equation}

\section{Lower Bounding Terms in the Polynomial Representation}

\end{document}