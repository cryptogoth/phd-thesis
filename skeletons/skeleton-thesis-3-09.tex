\documentclass{article}

%    Q-circuit version 1.06
%    Copyright (C) 2004  Steve Flammia & Bryan Eastin

%    This program is free software; you can redistribute it and/or modify
%    it under the terms of the GNU General Public License as published by
%    the Free Software Foundation; either version 2 of the License, or
%    (at your option) any later version.
%
%    This program is distributed in the hope that it will be useful,
%    but WITHOUT ANY WARRANTY; without even the implied warranty of
%    MERCHANTABILITY or FITNESS FOR A PARTICULAR PURPOSE.  See the
%    GNU General Public License for more details.
%
%    You should have received a copy of the GNU General Public License
%    along with this program; if not, write to the Free Software
%    Foundation, Inc., 59 Temple Place, Suite 330, Boston, MA  02111-1307  USA

\usepackage[matrix,frame,arrow]{xy}
\usepackage{amsmath}
\newcommand{\bra}[1]{\left\langle{#1}\right\vert}
\newcommand{\ket}[1]{\left\vert{#1}\right\rangle}
    % Defines Dirac notation.
\newcommand{\qw}[1][-1]{\ar @{-} [0,#1]}
    % Defines a wire that connects horizontally.  By default it connects to the object on the left of the current object.
    % WARNING: Wire commands must appear after the gate in any given entry.
\newcommand{\qwx}[1][-1]{\ar @{-} [#1,0]}
    % Defines a wire that connects vertically.  By default it connects to the object above the current object.
    % WARNING: Wire commands must appear after the gate in any given entry.
\newcommand{\cw}[1][-1]{\ar @{=} [0,#1]}
    % Defines a classical wire that connects horizontally.  By default it connects to the object on the left of the current object.
    % WARNING: Wire commands must appear after the gate in any given entry.
\newcommand{\cwx}[1][-1]{\ar @{=} [#1,0]}
    % Defines a classical wire that connects vertically.  By default it connects to the object above the current object.
    % WARNING: Wire commands must appear after the gate in any given entry.
\newcommand{\gate}[1]{*{\xy *+<.6em>{#1};p\save+LU;+RU **\dir{-}\restore\save+RU;+RD **\dir{-}\restore\save+RD;+LD **\dir{-}\restore\POS+LD;+LU **\dir{-}\endxy} \qw}
    % Boxes the argument, making a gate.
\newcommand{\meter}{\gate{\xy *!<0em,1.1em>h\cir<1.1em>{ur_dr},!U-<0em,.4em>;p+<.5em,.9em> **h\dir{-} \POS <-.6em,.4em> *{},<.6em,-.4em> *{} \endxy}}
    % Inserts a measurement meter.
\newcommand{\measure}[1]{*+[F-:<.9em>]{#1} \qw}
    % Inserts a measurement bubble with user defined text.
\newcommand{\measuretab}[1]{*{\xy *+<.6em>{#1};p\save+LU;+RU **\dir{-}\restore\save+RU;+RD **\dir{-}\restore\save+RD;+LD **\dir{-}\restore\save+LD;+LC-<.5em,0em> **\dir{-} \restore\POS+LU;+LC-<.5em,0em> **\dir{-} \endxy} \qw}
    % Inserts a measurement tab with user defined text.
\newcommand{\measureD}[1]{*{\xy*+=+<.5em>{\vphantom{#1}}*\cir{r_l};p\save*!R{#1} \restore\save+UC;+UC-<.5em,0em>*!R{\hphantom{#1}}+L **\dir{-} \restore\save+DC;+DC-<.5em,0em>*!R{\hphantom{#1}}+L **\dir{-} \restore\POS+UC-<.5em,0em>*!R{\hphantom{#1}}+L;+DC-<.5em,0em>*!R{\hphantom{#1}}+L **\dir{-} \endxy} \qw}
    % Inserts a D-shaped measurement gate with user defined text.
\newcommand{\multimeasure}[2]{*+<1em,.9em>{\hphantom{#2}} \qw \POS[0,0].[#1,0];p !C *{#2},p \drop\frm<.9em>{-}}
    % Draws a multiple qubit measurement bubble starting at the current position and spanning #1 additional gates below.
    % #2 gives the label for the gate.
    % You must use an argument of the same width as #2 in \ghost for the wires to connect properly on the lower lines.
\newcommand{\multimeasureD}[2]{*+<1em,.9em>{\hphantom{#2}}\save[0,0].[#1,0];p\save !C *{#2},p+LU+<0em,0em>;+RU+<-.8em,0em> **\dir{-}\restore\save +LD;+LU **\dir{-}\restore\save +LD;+RD-<.8em,0em> **\dir{-} \restore\save +RD+<0em,.8em>;+RU-<0em,.8em> **\dir{-} \restore \POS !UR*!UR{\cir<.9em>{r_d}};!DR*!DR{\cir<.9em>{d_l}}\restore \qw}
    % Draws a multiple qubit D-shaped measurement gate starting at the current position and spanning #1 additional gates below.
    % #2 gives the label for the gate.
    % You must use an argument of the same width as #2 in \ghost for the wires to connect properly on the lower lines.
\newcommand{\control}{*-=-{\bullet}}
    % Inserts an unconnected control.
\newcommand{\controlo}{*!<0em,.04em>-<.07em,.11em>{\xy *=<.45em>[o][F]{}\endxy}}
    % Inserts a unconnected control-on-0.
\newcommand{\ctrl}[1]{\control \qwx[#1] \qw}
    % Inserts a control and connects it to the object #1 wires below.
\newcommand{\ctrlo}[1]{\controlo \qwx[#1] \qw}
    % Inserts a control-on-0 and connects it to the object #1 wires below.
\newcommand{\targ}{*{\xy{<0em,0em>*{} \ar @{ - } +<.4em,0em> \ar @{ - } -<.4em,0em> \ar @{ - } +<0em,.4em> \ar @{ - } -<0em,.4em>},*+<.8em>\frm{o}\endxy} \qw}
    % Inserts a CNOT target.
\newcommand{\qswap}{*=<0em>{\times} \qw}
    % Inserts half a swap gate. 
    % Must be connected to the other swap with \qwx.
\newcommand{\multigate}[2]{*+<1em,.9em>{\hphantom{#2}} \qw \POS[0,0].[#1,0];p !C *{#2},p \save+LU;+RU **\dir{-}\restore\save+RU;+RD **\dir{-}\restore\save+RD;+LD **\dir{-}\restore\save+LD;+LU **\dir{-}\restore}
    % Draws a multiple qubit gate starting at the current position and spanning #1 additional gates below.
    % #2 gives the label for the gate.
    % You must use an argument of the same width as #2 in \ghost for the wires to connect properly on the lower lines.
\newcommand{\ghost}[1]{*+<1em,.9em>{\hphantom{#1}} \qw}
    % Leaves space for \multigate on wires other than the one on which \multigate appears.  Without this command wires will cross your gate.
    % #1 should match the second argument in the corresponding \multigate. 
\newcommand{\push}[1]{*{#1}}
    % Inserts #1, overriding the default that causes entries to have zero size.  This command takes the place of a gate.
    % Like a gate, it must precede any wire commands.
    % \push is useful for forcing columns apart.
    % NOTE: It might be useful to know that a gate is about 1.3 times the height of its contents.  I.e. \gate{M} is 1.3em tall.
    % WARNING: \push must appear before any wire commands and may not appear in an entry with a gate or label.
\newcommand{\gategroup}[6]{\POS"#1,#2"."#3,#2"."#1,#4"."#3,#4"!C*+<#5>\frm{#6}}
    % Constructs a box or bracket enclosing the square block spanning rows #1-#3 and columns=#2-#4.
    % The block is given a margin #5/2, so #5 should be a valid length.
    % #6 can take the following arguments -- or . or _\} or ^\} or \{ or \} or _) or ^) or ( or ) where the first two options yield dashed and
    % dotted boxes respectively, and the last eight options yield bottom, top, left, and right braces of the curly or normal variety.
    % \gategroup can appear at the end of any gate entry, but it's good form to pick one of the corner gates.
    % BUG: \gategroup uses the four corner gates to determine the size of the bounding box.  Other gates may stick out of that box.  See \prop. 
\newcommand{\rstick}[1]{*!L!<-.5em,0em>=<0em>{#1}}
    % Centers the left side of #1 in the cell.  Intended for lining up wire labels.  Note that non-gates have default size zero.
\newcommand{\lstick}[1]{*!R!<.5em,0em>=<0em>{#1}}
    % Centers the right side of #1 in the cell.  Intended for lining up wire labels.  Note that non-gates have default size zero.
\newcommand{\ustick}[1]{*!D!<0em,-.5em>=<0em>{#1}}
    % Centers the bottom of #1 in the cell.  Intended for lining up wire labels.  Note that non-gates have default size zero.
\newcommand{\dstick}[1]{*!U!<0em,.5em>=<0em>{#1}}
    % Centers the top of #1 in the cell.  Intended for lining up wire labels.  Note that non-gates have default size zero.
\newcommand{\Qcircuit}{\xymatrix @*=<0em>}
    % Defines \Qcircuit as an \xymatrix with entries of default size 0em.


\begin{document}

In the previous section, we discussed a particular technique of parallelization,
that is, identifying when information regarding the inputs is known in advance,
and inserting them into the circuit as soon as possible. In many cases, such
as in factoring, the inputs (the control qubits $\ket{p_j}$
and the successive powers-of-two of the base $a^{2^j}$ used for
modular multiplication) are all known in advance, which allows us to create
all $O(n)$ numbers in parallel at the beginning. However, this often comes at
a cost in width, or number of qubits, which introduces another geometric
constraint, when you consider a 2D circuit which you must eventually
manufacture.

Instead of inserting inputs into a problem as you go along in a serial fashion,
perhaps reusing qubits by uncomputing ancillae back to zero and making
efficient use of space, now you have a state which is spread out across
many qubits. This introduces a purely geometric communication problem of
moving the information around this spread out state. As in the 2D factoring
circuit in the previous example, the majority of the circuit size and circuit
width ($O(n^6)$ over all the modular multipliers in use at any one time) is
spent on moving the control qubits $\ket{x_i \cdot y_j}$ from their site of
generation to the site of the $z$-numbers, which are spread out over a
space $O(n^3)$ qubits long and requiring up to $O(n^4)$ as the maximum length
of teleportation rail for any single qubit to be teleported.
Since there are $O(n^2)$ control qubits, the total amount of operations
is $O(n^6)$, even though the teleportations themselves happen concurrently
in constant depth.

This often happens in establishing models, especially architectural models.
Some constraint, such as nearest-neighbor operations occurring on a lattice,
is established to approximate physical realism and more realistically
model experimental settings. Most experimental implementations of quantum
computing can only feasibly perform Hamiltonians which have locally-constrained
interactions on a constant number of qubits. However, many experiments
(cite the MUSIQC architecture here) no longer try to contain an entire
circuit on a single physical substrate. Rather, many architectures believe that
a future quantum computer will consist of many physical machines operating in
parallel, with entanglement created by photon pairs exchanged by optical fiber
and beamsplitters (citation needed here).

It is infeasible to imagine fabricating circuit board tracks on a silicon wafer
for, for example, superconducting qubits, which is possibly many meters squared,
and of which the majority consists of intermediate, reusable ancillae only
useful for a few steps of constant-depth teleportation. It is much more likely
that such a circuit would be split along boundaries into modules requiring the least
amount of communication (similar to the classical problem of MAX-CUT, where
the nodes represent qubits and the edges represent allowed interactions).
These modules, which have minimum entanglement, can then be divided among
communicating physical quantum computers.

Indeed, this is the amazing insight of recent work into constant-depth
quantum circuits, so-called $\textsc{QNC}^0$, (cite BKP paper and Aram's
distributed communication paper). Even though
they are restricted to constant depth and polynomial width, they can still
perform problems that are considered classically hard to do in
constant-depth, such as factoring and even something as simple as parity
and the majority function. Intuitively, if you have arbitrarily many
small, fixed-size quantum computers that can communicate via
nearest-neighbors (I actually don't know if this nearest-neighbor constraints
on the larger module/QC level is valid, fact check this), you can compute
the above functions in constant depth.

This represents the essence of the time-space tradeoff. In our previous
architecture, we have sacrificed width, by allowing it to increase polynomially,
in exchange for a lower depth (polylogarithmic). This would allow us to
complete a circuit for factoring a reasonable-sized RSA key (4096 bits)
in a matter of 8 days, versus the 427 years possibly needed by a 1D
implementation, even one that used much fewer qubits. This would appear to
be an advantage of our implementation, but qualified on the fact that it may
take us many decades to scale up to widths of billions of qubits, or to
sustain entanglement among millions of communication quantum computers long
enough to run our 8 day computation. Indeed, given current rates of
entangled photon pair generation (1 every few seconds, citation needed),
such massive communication costs may be the major bottleneck.

Toward this even, we propose modifying the CCNTC to allow multiple
communicating machines, and as another resource, we propose circuit
modules, or the number of communicating sub-blocks needed. The circuit
modules exist as nodes on a graph, again where allowed interactions are
represented by edges and only nearest-neighbors can interact. Therefore,
we can use this additional resource to model what is likely the most
feasible model of scalable quantum computation.

\end{document}