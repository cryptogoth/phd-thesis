\documentclass{article}

\usepackage{amssymb}

\input{Qcircuit}

\begin{document}

\section{Settling Previous Questions}

\begin{enumerate}
\item
The relationship between majority gates and threshold gates are as follows.
Threshold gates of polynomially-bounded weight can be simulated by
(I think up to three layers of) threshold gates with unit weights, which
can then be trivially simulated by majority gates. I think the only 
difference between the last two are that majority gates never have a bias,
that is, the weight $w_0$. The reverse direction is trivial.

\item
As Aram pointed out in our meeting, the $O(\log n)$ block size in the
block-save adder is probably to polynomially bound the weights in the
threshold circuit, something that I had not previously considered, but
seems obvious in retrospect.

\end{enumerate}

Okay that's it folks, move along.

\section{The Overall 2D Layout}

There are several intermediate results, such as the $n^2$-bit result of the
multiple product, the quotient of this with the modulus $m$, which involves
finding the reciprocal of $m$, the product of
the quotient and the modulus, and the subtraction.

Using our model with modules, we can teleport / fanout copies to different
modules and not worry about the exact geometry of the relationship between
them.

\section{To Do After These Pages Are Written}

Deeper understanding of the Reif and Tate results are needed, but perhaps
that can be delayed or deferred until after we have a final version of
Chapter 1.

Soft failure can look like just using the results of Reif and Tate.
But some deeper understanding is needed. For example, how to compute the
reciprocal. This can be done using the expansion given in Kitaev, but
then this reduces to creating the products in constant depth.

We don't necessarily need to use threshold gates to do this, since the
Toffoli is much easier, and can still be made constant.

As pointed out in group meeting, David's reordering circuit could be
used to rearrange the qubits, after being fanned out, into the right
order. But we could do this in blocks. What are the size of the blocks?
What \emph{are} the blocks?

\section{Creating Partial Product Bits}

In this section, I will describe how the partial product creation I gave in
my 2D factoring paper.

\section{Multiple Product via Reif and Tate}

In the 1992 paper by Reif and Tate, which appears to have appeared
contemporaneously with a bunch of papers by Bruck, and so was not cited
until a 1996 paper by Yeh and Varvarigos when the dust had settled, they
address the question of multiple product. Actually it is multiple product
modulo a number $p$, which as far as I can tell does not strictly have to
be prime, or if it is prime, is a smaller prime and one chosen so it is
bigger than the largest possible output number, so that no bits are
truncated. It does not correspond to the number $m$, the modulus to be
factored in Shor's factoring algorithm.

It remains to be seen whether I can use this or not. Yeh and Varvarigos
seem to think so, as long as I choose $p$ large enough. But then I need
to do modular reduction.

\section{The equivalence of threshold circuits to $Z_p$ circuits}

They can simulate each other in constant depth and polynomial increase in 
size. Certainly, this seems useful for the so-called
``Chinese Remaindering'' procedure, since we are dividing up a number into
factors of this ``mixed radix'' where the weights never become exponential,
but rather, remain of size $O(n)$, or at least polynomial.

\section{Size and depth tradeoffs}

Also from reading the Yeh-Varvarigos paper, I became aware of new parameters
in threshold circuits, namely the tradeoff between size and depth. The
size is never very critical for me, as long as it's polynomial, and 
surprisingly even with the parameter $\epsilon$ set to its maximum value,
the size is never more than $O(n^2)$ and usually $O(n)$.

\section{Sum of $n$ bits}

There is a construction given in Yeh-Varvarigos for the sum of $n$ bits,
which can then be used, I guess, for iterated sum, or multiple addition.
At some point, hours need to be set aside for finding the exact
construction for the building blocks that I am going to use.

For iterated addition, we can pretty much use the Siu-Bruck original
construction. And time's up.

\section{Explicit construction of simulation}

The Goldmann and Karpinski paper showed an explicit construction for
simulating threshold gates with exponential weights with those of
polynomial size weights. This is in contrast to the probabilistic,
existential, non-constructive proof given in the Siu-Bruck
paper ``On the Dynamic Range of Linear Threshold Elements.'' and
one other paper where the division and multiplication functions are
given.

I think this might be useful for lower bounds, or back when it was thought
that multiplication and division required exponentially many terms.
(Citation needed!)

\end{document}