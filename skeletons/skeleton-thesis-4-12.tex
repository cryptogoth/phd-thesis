\documentclass{article}

\usepackage{amssymb}

\input{Qcircuit}

\begin{document}

\section{The Model \textsf{2D CCNTCM}}

It occurred to me while meditating that the use of teleportation in
between modules was more powerful than I realized, in that it allowed
us to reorder qubits and eliminate the need for David's 2D reordering
construction.

It would be interesting to correlate the two of them, and show the 
equivalence.

\section{Other Activities}

During the day, when I am preparing things to write, here are useful
things to do.

\begin{enumerate}
\item
Write down interesting facts, theorems, lemmas, and whatnot on notecards
to remind yourself later. I already do this. Maybe I should do this in
my notebook, and number them in sections. This would certainly be more
portable, but less viscerally satisfying than the index cards.

\item
The above, but especially mathematical formulae. It would be good for you
to have practice formulating things in terms of symbols.

\item
Going through old previous skeleton pages and answering their questions,
and then typing them up. Maybe you could do that right now!

\item
Plan sections and actually type them into your skeleton pages, the skeleton
for the skeleton. Then go research each one for bursts of ten minutes,
and come back and write about them.

\end{enumerate}

\section{Chinese Remaindering}

Or the conversion from binary representation to a residue number system
(Chinese Remainder basis). And why is it necessary, instead of
just generating all the product bits for $n\times n$-bit numbers and
using iterated product. This then is the procedure of Chinese Remaindering,
or converting into a residue number system.

Let $\{p_1, \ldots p_k\}$ be relatively prime numbers (say, even the
$k$ smallest primes greater than $n$), and
$P_k = \prod_{i=1^k} p_i$ be the product of all these numbers. Let $Z$ be
an $n^2$-bit number, which would result from, for example, powering or
multiple product. Then we know that $Z < 2^{n^2} < P_k$.

Does this work out? We need to reconcile all the different versions of
Chinese Remaindering. Because by Cebysev's Theorem in Yeh-Varvarigos we show
that $k = \lceil n / \log_2 n \rceil$, so if $k$ such numbers of $\log^n$
bits each, we can represent any $n$-bit number. To do something similar
for 

So in order to do something similar for an $n^2$-bit number, we can just 
choose $k = \lceil n^2 / \log_2 n^2$.

The advantage of doing this is that if the weights are related to the
size of the numbers involved, then the original $O(n)$-bit number would
have exponentially large ($2^O(n)$) weights, and each of the new
modular arithmetic operations performed on each of the moduli in paralllel
would be $O(\log n)$-bit numbers, and the weights would be polynomially-large
($2^{\log n} = O(n)$).

However, it is still not clear to be how finding $r_i = Z \mod p_i$ can
be done as a sum of polynomially many $\hat{LT}_1$ functions.

\section{Reciprocal Finding as Expansion into Multiplication, Powering, Addition}

The expansion of reciprocal finding can be found in Kitaev on p. 190,
where we divide out the largest power-of-two from $x$, so that we are
left with the binary expansion (of finite precision):

\begin{equation}
\tilde{x} = 1.x_1 x_2 \ldots x_n
\end{equation}

Then $1 \le \tilde{x} < 2$, which saturates the lower bound if all the
$x_i$'s are
zero. Also note that we have imperfect knowledge of $x$ in this case.
If we divide $\tilde{x}$ by two, we get $1/2 \le \tilde{x}/2 < 1$.
Then let us define a new variable $y$ which is difference of $x/2$ of $1$,

\begin{equation}
y = 1 - x/2
\end{equation}

which is bounded by $0 < y \le 1/2$.

\section{The Best Known Concrete Construction for Multiple Sum}



\end{document}