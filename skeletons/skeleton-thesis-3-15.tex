\documentclass{article}

%    Q-circuit version 1.06
%    Copyright (C) 2004  Steve Flammia & Bryan Eastin

%    This program is free software; you can redistribute it and/or modify
%    it under the terms of the GNU General Public License as published by
%    the Free Software Foundation; either version 2 of the License, or
%    (at your option) any later version.
%
%    This program is distributed in the hope that it will be useful,
%    but WITHOUT ANY WARRANTY; without even the implied warranty of
%    MERCHANTABILITY or FITNESS FOR A PARTICULAR PURPOSE.  See the
%    GNU General Public License for more details.
%
%    You should have received a copy of the GNU General Public License
%    along with this program; if not, write to the Free Software
%    Foundation, Inc., 59 Temple Place, Suite 330, Boston, MA  02111-1307  USA

\usepackage[matrix,frame,arrow]{xy}
\usepackage{amsmath}
\newcommand{\bra}[1]{\left\langle{#1}\right\vert}
\newcommand{\ket}[1]{\left\vert{#1}\right\rangle}
    % Defines Dirac notation.
\newcommand{\qw}[1][-1]{\ar @{-} [0,#1]}
    % Defines a wire that connects horizontally.  By default it connects to the object on the left of the current object.
    % WARNING: Wire commands must appear after the gate in any given entry.
\newcommand{\qwx}[1][-1]{\ar @{-} [#1,0]}
    % Defines a wire that connects vertically.  By default it connects to the object above the current object.
    % WARNING: Wire commands must appear after the gate in any given entry.
\newcommand{\cw}[1][-1]{\ar @{=} [0,#1]}
    % Defines a classical wire that connects horizontally.  By default it connects to the object on the left of the current object.
    % WARNING: Wire commands must appear after the gate in any given entry.
\newcommand{\cwx}[1][-1]{\ar @{=} [#1,0]}
    % Defines a classical wire that connects vertically.  By default it connects to the object above the current object.
    % WARNING: Wire commands must appear after the gate in any given entry.
\newcommand{\gate}[1]{*{\xy *+<.6em>{#1};p\save+LU;+RU **\dir{-}\restore\save+RU;+RD **\dir{-}\restore\save+RD;+LD **\dir{-}\restore\POS+LD;+LU **\dir{-}\endxy} \qw}
    % Boxes the argument, making a gate.
\newcommand{\meter}{\gate{\xy *!<0em,1.1em>h\cir<1.1em>{ur_dr},!U-<0em,.4em>;p+<.5em,.9em> **h\dir{-} \POS <-.6em,.4em> *{},<.6em,-.4em> *{} \endxy}}
    % Inserts a measurement meter.
\newcommand{\measure}[1]{*+[F-:<.9em>]{#1} \qw}
    % Inserts a measurement bubble with user defined text.
\newcommand{\measuretab}[1]{*{\xy *+<.6em>{#1};p\save+LU;+RU **\dir{-}\restore\save+RU;+RD **\dir{-}\restore\save+RD;+LD **\dir{-}\restore\save+LD;+LC-<.5em,0em> **\dir{-} \restore\POS+LU;+LC-<.5em,0em> **\dir{-} \endxy} \qw}
    % Inserts a measurement tab with user defined text.
\newcommand{\measureD}[1]{*{\xy*+=+<.5em>{\vphantom{#1}}*\cir{r_l};p\save*!R{#1} \restore\save+UC;+UC-<.5em,0em>*!R{\hphantom{#1}}+L **\dir{-} \restore\save+DC;+DC-<.5em,0em>*!R{\hphantom{#1}}+L **\dir{-} \restore\POS+UC-<.5em,0em>*!R{\hphantom{#1}}+L;+DC-<.5em,0em>*!R{\hphantom{#1}}+L **\dir{-} \endxy} \qw}
    % Inserts a D-shaped measurement gate with user defined text.
\newcommand{\multimeasure}[2]{*+<1em,.9em>{\hphantom{#2}} \qw \POS[0,0].[#1,0];p !C *{#2},p \drop\frm<.9em>{-}}
    % Draws a multiple qubit measurement bubble starting at the current position and spanning #1 additional gates below.
    % #2 gives the label for the gate.
    % You must use an argument of the same width as #2 in \ghost for the wires to connect properly on the lower lines.
\newcommand{\multimeasureD}[2]{*+<1em,.9em>{\hphantom{#2}}\save[0,0].[#1,0];p\save !C *{#2},p+LU+<0em,0em>;+RU+<-.8em,0em> **\dir{-}\restore\save +LD;+LU **\dir{-}\restore\save +LD;+RD-<.8em,0em> **\dir{-} \restore\save +RD+<0em,.8em>;+RU-<0em,.8em> **\dir{-} \restore \POS !UR*!UR{\cir<.9em>{r_d}};!DR*!DR{\cir<.9em>{d_l}}\restore \qw}
    % Draws a multiple qubit D-shaped measurement gate starting at the current position and spanning #1 additional gates below.
    % #2 gives the label for the gate.
    % You must use an argument of the same width as #2 in \ghost for the wires to connect properly on the lower lines.
\newcommand{\control}{*-=-{\bullet}}
    % Inserts an unconnected control.
\newcommand{\controlo}{*!<0em,.04em>-<.07em,.11em>{\xy *=<.45em>[o][F]{}\endxy}}
    % Inserts a unconnected control-on-0.
\newcommand{\ctrl}[1]{\control \qwx[#1] \qw}
    % Inserts a control and connects it to the object #1 wires below.
\newcommand{\ctrlo}[1]{\controlo \qwx[#1] \qw}
    % Inserts a control-on-0 and connects it to the object #1 wires below.
\newcommand{\targ}{*{\xy{<0em,0em>*{} \ar @{ - } +<.4em,0em> \ar @{ - } -<.4em,0em> \ar @{ - } +<0em,.4em> \ar @{ - } -<0em,.4em>},*+<.8em>\frm{o}\endxy} \qw}
    % Inserts a CNOT target.
\newcommand{\qswap}{*=<0em>{\times} \qw}
    % Inserts half a swap gate. 
    % Must be connected to the other swap with \qwx.
\newcommand{\multigate}[2]{*+<1em,.9em>{\hphantom{#2}} \qw \POS[0,0].[#1,0];p !C *{#2},p \save+LU;+RU **\dir{-}\restore\save+RU;+RD **\dir{-}\restore\save+RD;+LD **\dir{-}\restore\save+LD;+LU **\dir{-}\restore}
    % Draws a multiple qubit gate starting at the current position and spanning #1 additional gates below.
    % #2 gives the label for the gate.
    % You must use an argument of the same width as #2 in \ghost for the wires to connect properly on the lower lines.
\newcommand{\ghost}[1]{*+<1em,.9em>{\hphantom{#1}} \qw}
    % Leaves space for \multigate on wires other than the one on which \multigate appears.  Without this command wires will cross your gate.
    % #1 should match the second argument in the corresponding \multigate. 
\newcommand{\push}[1]{*{#1}}
    % Inserts #1, overriding the default that causes entries to have zero size.  This command takes the place of a gate.
    % Like a gate, it must precede any wire commands.
    % \push is useful for forcing columns apart.
    % NOTE: It might be useful to know that a gate is about 1.3 times the height of its contents.  I.e. \gate{M} is 1.3em tall.
    % WARNING: \push must appear before any wire commands and may not appear in an entry with a gate or label.
\newcommand{\gategroup}[6]{\POS"#1,#2"."#3,#2"."#1,#4"."#3,#4"!C*+<#5>\frm{#6}}
    % Constructs a box or bracket enclosing the square block spanning rows #1-#3 and columns=#2-#4.
    % The block is given a margin #5/2, so #5 should be a valid length.
    % #6 can take the following arguments -- or . or _\} or ^\} or \{ or \} or _) or ^) or ( or ) where the first two options yield dashed and
    % dotted boxes respectively, and the last eight options yield bottom, top, left, and right braces of the curly or normal variety.
    % \gategroup can appear at the end of any gate entry, but it's good form to pick one of the corner gates.
    % BUG: \gategroup uses the four corner gates to determine the size of the bounding box.  Other gates may stick out of that box.  See \prop. 
\newcommand{\rstick}[1]{*!L!<-.5em,0em>=<0em>{#1}}
    % Centers the left side of #1 in the cell.  Intended for lining up wire labels.  Note that non-gates have default size zero.
\newcommand{\lstick}[1]{*!R!<.5em,0em>=<0em>{#1}}
    % Centers the right side of #1 in the cell.  Intended for lining up wire labels.  Note that non-gates have default size zero.
\newcommand{\ustick}[1]{*!D!<0em,-.5em>=<0em>{#1}}
    % Centers the bottom of #1 in the cell.  Intended for lining up wire labels.  Note that non-gates have default size zero.
\newcommand{\dstick}[1]{*!U!<0em,.5em>=<0em>{#1}}
    % Centers the top of #1 in the cell.  Intended for lining up wire labels.  Note that non-gates have default size zero.
\newcommand{\Qcircuit}{\xymatrix @*=<0em>}
    % Defines \Qcircuit as an \xymatrix with entries of default size 0em.


\begin{document}

To continue our discussion from the previous section, we now go into more
details of how the idea of \emph{modules} effectively captures a realistic
resource in quantum computing. Using a strictly CCNTC model, either in 1D
or 2D, makes architectures very sensitive to geometry. That is, the exact
placement and distribution of data throughout a lattice can vastly change
the resources needed to communicate and move data throughout such an
arrangement. While such details are necessary for actual implementations,
in preliminary studies we found that such communication formed the bulk of
such communication for factoring on a 2D CCNTC architecture.

Our goal is to study more general architectural features that are not so
sensitive to placement, or moreover, that obviate the need by treating
each module as a separate quantum computer. The key insight of
parallelization results such as Browne-Kashefi-Perdrix and
Kashefi-Broadbent is that the power of communication between fixed-size,
parallel quantum computers can greatly increase their overall computing
power.

However, given the difficult of establishing remote entanglement, we
are limiting the communication between modules to be of constant size,
although the modules themselves do not have to be constant-sized. For example,
the rate of generating entangled photon pairs in current laboratory
experiments is one every few seconds (citation needed, maybe from NIST),
which is much larger than the coherence times of many physical qubits,
including trapped ions. Therefore, generating scalable entanglement resources
for communication between separate quantum computers seems a difficult task
for the foreseeable future. This is the rationale behind the choice of
restrictions behind the $O(1)$ communication between modules and the
$O(n)$ size of the modules themselves, which otherwise might seem
arbitrary.

Another subtlety of the model we wish to discuss is the concept of
\emph{reusable ancillae}, that is, a qubit which starts out initialized to
$\ket{0}$ at the beginning of a circuit and is returned to $\ket{0}$ by the
end of the circuit, where we use the circuit here loosely to mean the
smallest sequence of operations in which an ancillae is used
(entangled with the computation state) and then becomes disentangled again.
Clearly, such an ancillae can be used multiple times, becoming entangled
and then disentangled from the computation state through the lifetime of a
complicated algorithm, each time performing a different purpose.

Unlike
much of the previous literature, we do not treat such ancillae as having a
fixed location and needing to be maintained in a coherent state throughout
the life of the computation. Rather, it only has to be maintained for the
above \emph{minimal computation time}. Such an ancillae may be measured
in the middle of a computation in order to make it a product state, or
to disentangle it from the computation state, to free up qubits which would
otherwise have to be maintained by a quantum error-correcting code. These
ancillae, while they are in a $\ket{0}$ state, may even be dumped out of the
quantum computer, for example, like ions that can be lost out of the trap,
or moved to a different area of the trap where they are needed. The end
result is that we treat such ancillae as instantly (re)-initializable and
available in arbitrary locations within a module ``just-in-time''. This is
consistent with how we currently understanding experimental quantum information
systems to be built. That is, in segmented ion traps, not all fabricated locations
hold ions, only those that currently need to be involved in the computation.
In superconducting systems, such as the D-Wave architecture, not all
couplings need to be active. For those ancillae which lie dormant, all
couplings to them from adjacent qubits can be turned off.

The idea of the computation state is one that persists over time, and has
the following definition. In the last timestep of a circuit, the computation
state is defined by
the final qubits which are projectively measured to obtain a classical
answer. At every timestep before that, say step $i$,
the computation state is one which is connected to the computation state in
step $i+1$ by an entangling gate (most two-qubit gates, cite Nielsen at al
paper). In analogy to classical computing, any qubit that is not part of the
computation state at any timestep can safely be removed from the circuit
without affecting the computation. It is similar to dead code analysis in a
classical compiler, where code that cannot possibly affect the outcome of
a function is pruned away.

The computation state is used to measure the resource of \emph{circuit coherence},
the total amount of error-correction, or state maintenance, needed to complete
a circuit. If at any timestep of a quantum circuit, one of the qubits involved
in the computation state becomes corrupted, the success probability of the
entire computation is decreased. Trivially, the circuit coherence is
bounded above by the product of circuit depth and circuit width, since it
may be the case that every qubit is in the computation step at every timestep.
This is not always the case, per our discussion of reusable ancillae above,
and in fact, it is our goal to show that there is an asymptotic separation
between the circuit coherence and the circuit depth width product for
factoring.

\end{document}