\documentclass{article}

\usepackage{amssymb}

%    Q-circuit version 1.06
%    Copyright (C) 2004  Steve Flammia & Bryan Eastin

%    This program is free software; you can redistribute it and/or modify
%    it under the terms of the GNU General Public License as published by
%    the Free Software Foundation; either version 2 of the License, or
%    (at your option) any later version.
%
%    This program is distributed in the hope that it will be useful,
%    but WITHOUT ANY WARRANTY; without even the implied warranty of
%    MERCHANTABILITY or FITNESS FOR A PARTICULAR PURPOSE.  See the
%    GNU General Public License for more details.
%
%    You should have received a copy of the GNU General Public License
%    along with this program; if not, write to the Free Software
%    Foundation, Inc., 59 Temple Place, Suite 330, Boston, MA  02111-1307  USA

\usepackage[matrix,frame,arrow]{xy}
\usepackage{amsmath}
\newcommand{\bra}[1]{\left\langle{#1}\right\vert}
\newcommand{\ket}[1]{\left\vert{#1}\right\rangle}
    % Defines Dirac notation.
\newcommand{\qw}[1][-1]{\ar @{-} [0,#1]}
    % Defines a wire that connects horizontally.  By default it connects to the object on the left of the current object.
    % WARNING: Wire commands must appear after the gate in any given entry.
\newcommand{\qwx}[1][-1]{\ar @{-} [#1,0]}
    % Defines a wire that connects vertically.  By default it connects to the object above the current object.
    % WARNING: Wire commands must appear after the gate in any given entry.
\newcommand{\cw}[1][-1]{\ar @{=} [0,#1]}
    % Defines a classical wire that connects horizontally.  By default it connects to the object on the left of the current object.
    % WARNING: Wire commands must appear after the gate in any given entry.
\newcommand{\cwx}[1][-1]{\ar @{=} [#1,0]}
    % Defines a classical wire that connects vertically.  By default it connects to the object above the current object.
    % WARNING: Wire commands must appear after the gate in any given entry.
\newcommand{\gate}[1]{*{\xy *+<.6em>{#1};p\save+LU;+RU **\dir{-}\restore\save+RU;+RD **\dir{-}\restore\save+RD;+LD **\dir{-}\restore\POS+LD;+LU **\dir{-}\endxy} \qw}
    % Boxes the argument, making a gate.
\newcommand{\meter}{\gate{\xy *!<0em,1.1em>h\cir<1.1em>{ur_dr},!U-<0em,.4em>;p+<.5em,.9em> **h\dir{-} \POS <-.6em,.4em> *{},<.6em,-.4em> *{} \endxy}}
    % Inserts a measurement meter.
\newcommand{\measure}[1]{*+[F-:<.9em>]{#1} \qw}
    % Inserts a measurement bubble with user defined text.
\newcommand{\measuretab}[1]{*{\xy *+<.6em>{#1};p\save+LU;+RU **\dir{-}\restore\save+RU;+RD **\dir{-}\restore\save+RD;+LD **\dir{-}\restore\save+LD;+LC-<.5em,0em> **\dir{-} \restore\POS+LU;+LC-<.5em,0em> **\dir{-} \endxy} \qw}
    % Inserts a measurement tab with user defined text.
\newcommand{\measureD}[1]{*{\xy*+=+<.5em>{\vphantom{#1}}*\cir{r_l};p\save*!R{#1} \restore\save+UC;+UC-<.5em,0em>*!R{\hphantom{#1}}+L **\dir{-} \restore\save+DC;+DC-<.5em,0em>*!R{\hphantom{#1}}+L **\dir{-} \restore\POS+UC-<.5em,0em>*!R{\hphantom{#1}}+L;+DC-<.5em,0em>*!R{\hphantom{#1}}+L **\dir{-} \endxy} \qw}
    % Inserts a D-shaped measurement gate with user defined text.
\newcommand{\multimeasure}[2]{*+<1em,.9em>{\hphantom{#2}} \qw \POS[0,0].[#1,0];p !C *{#2},p \drop\frm<.9em>{-}}
    % Draws a multiple qubit measurement bubble starting at the current position and spanning #1 additional gates below.
    % #2 gives the label for the gate.
    % You must use an argument of the same width as #2 in \ghost for the wires to connect properly on the lower lines.
\newcommand{\multimeasureD}[2]{*+<1em,.9em>{\hphantom{#2}}\save[0,0].[#1,0];p\save !C *{#2},p+LU+<0em,0em>;+RU+<-.8em,0em> **\dir{-}\restore\save +LD;+LU **\dir{-}\restore\save +LD;+RD-<.8em,0em> **\dir{-} \restore\save +RD+<0em,.8em>;+RU-<0em,.8em> **\dir{-} \restore \POS !UR*!UR{\cir<.9em>{r_d}};!DR*!DR{\cir<.9em>{d_l}}\restore \qw}
    % Draws a multiple qubit D-shaped measurement gate starting at the current position and spanning #1 additional gates below.
    % #2 gives the label for the gate.
    % You must use an argument of the same width as #2 in \ghost for the wires to connect properly on the lower lines.
\newcommand{\control}{*-=-{\bullet}}
    % Inserts an unconnected control.
\newcommand{\controlo}{*!<0em,.04em>-<.07em,.11em>{\xy *=<.45em>[o][F]{}\endxy}}
    % Inserts a unconnected control-on-0.
\newcommand{\ctrl}[1]{\control \qwx[#1] \qw}
    % Inserts a control and connects it to the object #1 wires below.
\newcommand{\ctrlo}[1]{\controlo \qwx[#1] \qw}
    % Inserts a control-on-0 and connects it to the object #1 wires below.
\newcommand{\targ}{*{\xy{<0em,0em>*{} \ar @{ - } +<.4em,0em> \ar @{ - } -<.4em,0em> \ar @{ - } +<0em,.4em> \ar @{ - } -<0em,.4em>},*+<.8em>\frm{o}\endxy} \qw}
    % Inserts a CNOT target.
\newcommand{\qswap}{*=<0em>{\times} \qw}
    % Inserts half a swap gate. 
    % Must be connected to the other swap with \qwx.
\newcommand{\multigate}[2]{*+<1em,.9em>{\hphantom{#2}} \qw \POS[0,0].[#1,0];p !C *{#2},p \save+LU;+RU **\dir{-}\restore\save+RU;+RD **\dir{-}\restore\save+RD;+LD **\dir{-}\restore\save+LD;+LU **\dir{-}\restore}
    % Draws a multiple qubit gate starting at the current position and spanning #1 additional gates below.
    % #2 gives the label for the gate.
    % You must use an argument of the same width as #2 in \ghost for the wires to connect properly on the lower lines.
\newcommand{\ghost}[1]{*+<1em,.9em>{\hphantom{#1}} \qw}
    % Leaves space for \multigate on wires other than the one on which \multigate appears.  Without this command wires will cross your gate.
    % #1 should match the second argument in the corresponding \multigate. 
\newcommand{\push}[1]{*{#1}}
    % Inserts #1, overriding the default that causes entries to have zero size.  This command takes the place of a gate.
    % Like a gate, it must precede any wire commands.
    % \push is useful for forcing columns apart.
    % NOTE: It might be useful to know that a gate is about 1.3 times the height of its contents.  I.e. \gate{M} is 1.3em tall.
    % WARNING: \push must appear before any wire commands and may not appear in an entry with a gate or label.
\newcommand{\gategroup}[6]{\POS"#1,#2"."#3,#2"."#1,#4"."#3,#4"!C*+<#5>\frm{#6}}
    % Constructs a box or bracket enclosing the square block spanning rows #1-#3 and columns=#2-#4.
    % The block is given a margin #5/2, so #5 should be a valid length.
    % #6 can take the following arguments -- or . or _\} or ^\} or \{ or \} or _) or ^) or ( or ) where the first two options yield dashed and
    % dotted boxes respectively, and the last eight options yield bottom, top, left, and right braces of the curly or normal variety.
    % \gategroup can appear at the end of any gate entry, but it's good form to pick one of the corner gates.
    % BUG: \gategroup uses the four corner gates to determine the size of the bounding box.  Other gates may stick out of that box.  See \prop. 
\newcommand{\rstick}[1]{*!L!<-.5em,0em>=<0em>{#1}}
    % Centers the left side of #1 in the cell.  Intended for lining up wire labels.  Note that non-gates have default size zero.
\newcommand{\lstick}[1]{*!R!<.5em,0em>=<0em>{#1}}
    % Centers the right side of #1 in the cell.  Intended for lining up wire labels.  Note that non-gates have default size zero.
\newcommand{\ustick}[1]{*!D!<0em,-.5em>=<0em>{#1}}
    % Centers the bottom of #1 in the cell.  Intended for lining up wire labels.  Note that non-gates have default size zero.
\newcommand{\dstick}[1]{*!U!<0em,.5em>=<0em>{#1}}
    % Centers the top of #1 in the cell.  Intended for lining up wire labels.  Note that non-gates have default size zero.
\newcommand{\Qcircuit}{\xymatrix @*=<0em>}
    % Defines \Qcircuit as an \xymatrix with entries of default size 0em.


\begin{document}

\section{The Overall Plan}

Here are the main steps remaining for the constant-depth factoring
implementation:

\begin{enumerate}

\item
Finish testing out and writing up constant-depth fanout, which I've
given steps for below.

\item
Calculate the resource overhead for circuit depth, size, width,
including module equivalents, for overall multiple product and
modular reduction, in 2D CCNTCM model.

\item
Calculate the resources for a single majority gate in 2D CCNTCM,
using constant-depth unfanout resources.

\item
Calculate the same resources for implementing a majority gate
directly with rotation by Hamming value and a threshold of
$n/2$. See whether this is less than Step 3.

\item
Combine steps 2 and 3 (via multiplication) to get total resources
for iterated log-depth factoring.

\end{enumerate}

Here are the substeps to do for constant-depth unfanout.

\begin{enumerate}
\item Figure out a compact, or at least systematic, way of correcting the
resulting source qubit. Otherwise, we can hardly be said to have a
constant-depth fanout. Do this for the odd case.

\item Figure this out for the even case (do the above for even $n$).

\item Unify the two of them into the same general formulae.

\end{enumerate}

\section{Constant-Depth Unfanout}

We now present a circuit for constant-depth unfanout in CCNTC
assuming that all the fanned out qubits are adjacent in some
contiguous line. The actual unfanout only requires 1D,
however, normally the fanned out qubits have been teleported else
and must be teleported back into a contiguous line. This operation
normally requires 2D in order to be constant-depth. However, we
neglect that here, only assuming that the ``user'' or ``client''
module of constant-depth unfanout is able to get their qubits
in the right form.

% TODO 

\begin{figure}
\caption{CCNTC Circuit for unfanout for odd number of qubits}
\end{figure}

\begin{figure}
\caption{CCNTC Circuit for unfanout for even number of qubits}
\end{figure}

The general technique for both fanning out and unfanning out is to
increase the subspace inhabited by the state. For fanning out,
we first consider the submodule of cat state creation. An
unbounded product state of $k$ constant-sized cat states inhabits
a subspace of $2^k$ computational basis states. CNOTs provided
entanglement between the constant-sized cat states, thereby 
making irreducible back to a product of $k$ cat states, but
without reducing the dimension of the subspace from $2^k$.
Then the measurements (they do not need to be Bell basis measurements)
project the entangled state onto a smaller subspace by half.
Therefore, to get back from a $2^k$-dimensional subspace to a
$2$-dimensional subspace, we need $k-1$ measurements, which indeed
is what happens.

For the unfanout, we cannot simply perform the same fanout procedure
twice.  (Why not?) However, we can take the same approach.
We start with a $2$-dimensional subspace, the fanned out state
which is analogous to the $n$-qubit cat state with coefficients
$\alpha$ and $\beta$ instead of $1/\sqrt{2}$. To do this, we
must enlarge the subspace of our entangled state back to $2^n$,
which we can do by applying Hadamards to every single qubit. At this
point, the state is still entangled, and we need to disentangle them
the same way that we entangled them in the first place.
To do so, it turns out 
we enact two layers of interleaved CNOTs and must project the entangled
state back down to a $2$-dimensional subspace, namely by projective
measurements on all qubits except the ``target'' qubit.

Because the fanned out state is symmetric, any of these qubits can be
left in the original ``source'' state of
$\ket{\psi} = \alpha\ket{0} + \beta\ket{1}$.
At the end of which the
state is no longer entangled, i.e., it can be expressed as the
tensor product of $n$ single-qubit states. Therefore, we must perform
some \emph{asymmetric} operation to the fanned-out state to choose
which qubit is the target, which does not necessarily have to be the
same as the source qubit. However, it will be in a similar location
to the original source qubit, if further uncomputing is necessary,
of which this unfanout is merely a subroutine. This asymmetric action
is to projectively measure in the $Z$-basis on all other qubits
\emph{except} the target qubit. On the target qubit, we apply the
Hadamard, since it will be left in the source qubit state but in
an alternate basis.

Finally, the resulting target qubit state will need some Pauli
correction (it turns out, only a Pauli $Z$ correction is necessary)
based on the classical measurement outcomes of the other qubits.
This is a classical boolean function with $1$-bit outputs and
$n-1$-bit inputs, (define what $n$ is above), so this can be
done on a classical computer in polynomial time simply by using
a lookup table. (If this requires $2^n$ exponential space, however,
can we still claim this? I think it is necessary to find a compact
representation, via Karnaugh maps or the like).

Using this new constant-depth unfanout, it is now possible to reduce
the factoring depth of the previous implementation to $O(\log ^2 n)$
rather than $O(\log^3 n)$.

The question, remains, is this optimal? That is also a question for
another time.

\section{The Threshold Model}

We must emphasize, as Aram says, that our implementation of factoring
down to the CCNTC model is $O(\log \log n)$ depth overhead due to
quantum compiling, which any other implementation will have to face,
even if they use the results from classical threshold circuit complexity.
However, the threshold gate itself can be performed in CCNTC with only
constant-depth overhead, and a polynomial (say $O(n^2)$) increase in
size and width (for ancillae)

\section{New Idea: Recursive Approximation of KSV}

Instead of approximating the rotation $R_Z(1/2^n)$ in one go, what if
we approximated two rotations whose difference would equal the
desired rotation, but would be asymptotically faster to approximate?
Or at least, in expectation, be faster. A bold new idea, for another time.

\end{document}