\documentclass{article}

%    Q-circuit version 1.06
%    Copyright (C) 2004  Steve Flammia & Bryan Eastin

%    This program is free software; you can redistribute it and/or modify
%    it under the terms of the GNU General Public License as published by
%    the Free Software Foundation; either version 2 of the License, or
%    (at your option) any later version.
%
%    This program is distributed in the hope that it will be useful,
%    but WITHOUT ANY WARRANTY; without even the implied warranty of
%    MERCHANTABILITY or FITNESS FOR A PARTICULAR PURPOSE.  See the
%    GNU General Public License for more details.
%
%    You should have received a copy of the GNU General Public License
%    along with this program; if not, write to the Free Software
%    Foundation, Inc., 59 Temple Place, Suite 330, Boston, MA  02111-1307  USA

\usepackage[matrix,frame,arrow]{xy}
\usepackage{amsmath}
\newcommand{\bra}[1]{\left\langle{#1}\right\vert}
\newcommand{\ket}[1]{\left\vert{#1}\right\rangle}
    % Defines Dirac notation.
\newcommand{\qw}[1][-1]{\ar @{-} [0,#1]}
    % Defines a wire that connects horizontally.  By default it connects to the object on the left of the current object.
    % WARNING: Wire commands must appear after the gate in any given entry.
\newcommand{\qwx}[1][-1]{\ar @{-} [#1,0]}
    % Defines a wire that connects vertically.  By default it connects to the object above the current object.
    % WARNING: Wire commands must appear after the gate in any given entry.
\newcommand{\cw}[1][-1]{\ar @{=} [0,#1]}
    % Defines a classical wire that connects horizontally.  By default it connects to the object on the left of the current object.
    % WARNING: Wire commands must appear after the gate in any given entry.
\newcommand{\cwx}[1][-1]{\ar @{=} [#1,0]}
    % Defines a classical wire that connects vertically.  By default it connects to the object above the current object.
    % WARNING: Wire commands must appear after the gate in any given entry.
\newcommand{\gate}[1]{*{\xy *+<.6em>{#1};p\save+LU;+RU **\dir{-}\restore\save+RU;+RD **\dir{-}\restore\save+RD;+LD **\dir{-}\restore\POS+LD;+LU **\dir{-}\endxy} \qw}
    % Boxes the argument, making a gate.
\newcommand{\meter}{\gate{\xy *!<0em,1.1em>h\cir<1.1em>{ur_dr},!U-<0em,.4em>;p+<.5em,.9em> **h\dir{-} \POS <-.6em,.4em> *{},<.6em,-.4em> *{} \endxy}}
    % Inserts a measurement meter.
\newcommand{\measure}[1]{*+[F-:<.9em>]{#1} \qw}
    % Inserts a measurement bubble with user defined text.
\newcommand{\measuretab}[1]{*{\xy *+<.6em>{#1};p\save+LU;+RU **\dir{-}\restore\save+RU;+RD **\dir{-}\restore\save+RD;+LD **\dir{-}\restore\save+LD;+LC-<.5em,0em> **\dir{-} \restore\POS+LU;+LC-<.5em,0em> **\dir{-} \endxy} \qw}
    % Inserts a measurement tab with user defined text.
\newcommand{\measureD}[1]{*{\xy*+=+<.5em>{\vphantom{#1}}*\cir{r_l};p\save*!R{#1} \restore\save+UC;+UC-<.5em,0em>*!R{\hphantom{#1}}+L **\dir{-} \restore\save+DC;+DC-<.5em,0em>*!R{\hphantom{#1}}+L **\dir{-} \restore\POS+UC-<.5em,0em>*!R{\hphantom{#1}}+L;+DC-<.5em,0em>*!R{\hphantom{#1}}+L **\dir{-} \endxy} \qw}
    % Inserts a D-shaped measurement gate with user defined text.
\newcommand{\multimeasure}[2]{*+<1em,.9em>{\hphantom{#2}} \qw \POS[0,0].[#1,0];p !C *{#2},p \drop\frm<.9em>{-}}
    % Draws a multiple qubit measurement bubble starting at the current position and spanning #1 additional gates below.
    % #2 gives the label for the gate.
    % You must use an argument of the same width as #2 in \ghost for the wires to connect properly on the lower lines.
\newcommand{\multimeasureD}[2]{*+<1em,.9em>{\hphantom{#2}}\save[0,0].[#1,0];p\save !C *{#2},p+LU+<0em,0em>;+RU+<-.8em,0em> **\dir{-}\restore\save +LD;+LU **\dir{-}\restore\save +LD;+RD-<.8em,0em> **\dir{-} \restore\save +RD+<0em,.8em>;+RU-<0em,.8em> **\dir{-} \restore \POS !UR*!UR{\cir<.9em>{r_d}};!DR*!DR{\cir<.9em>{d_l}}\restore \qw}
    % Draws a multiple qubit D-shaped measurement gate starting at the current position and spanning #1 additional gates below.
    % #2 gives the label for the gate.
    % You must use an argument of the same width as #2 in \ghost for the wires to connect properly on the lower lines.
\newcommand{\control}{*-=-{\bullet}}
    % Inserts an unconnected control.
\newcommand{\controlo}{*!<0em,.04em>-<.07em,.11em>{\xy *=<.45em>[o][F]{}\endxy}}
    % Inserts a unconnected control-on-0.
\newcommand{\ctrl}[1]{\control \qwx[#1] \qw}
    % Inserts a control and connects it to the object #1 wires below.
\newcommand{\ctrlo}[1]{\controlo \qwx[#1] \qw}
    % Inserts a control-on-0 and connects it to the object #1 wires below.
\newcommand{\targ}{*{\xy{<0em,0em>*{} \ar @{ - } +<.4em,0em> \ar @{ - } -<.4em,0em> \ar @{ - } +<0em,.4em> \ar @{ - } -<0em,.4em>},*+<.8em>\frm{o}\endxy} \qw}
    % Inserts a CNOT target.
\newcommand{\qswap}{*=<0em>{\times} \qw}
    % Inserts half a swap gate. 
    % Must be connected to the other swap with \qwx.
\newcommand{\multigate}[2]{*+<1em,.9em>{\hphantom{#2}} \qw \POS[0,0].[#1,0];p !C *{#2},p \save+LU;+RU **\dir{-}\restore\save+RU;+RD **\dir{-}\restore\save+RD;+LD **\dir{-}\restore\save+LD;+LU **\dir{-}\restore}
    % Draws a multiple qubit gate starting at the current position and spanning #1 additional gates below.
    % #2 gives the label for the gate.
    % You must use an argument of the same width as #2 in \ghost for the wires to connect properly on the lower lines.
\newcommand{\ghost}[1]{*+<1em,.9em>{\hphantom{#1}} \qw}
    % Leaves space for \multigate on wires other than the one on which \multigate appears.  Without this command wires will cross your gate.
    % #1 should match the second argument in the corresponding \multigate. 
\newcommand{\push}[1]{*{#1}}
    % Inserts #1, overriding the default that causes entries to have zero size.  This command takes the place of a gate.
    % Like a gate, it must precede any wire commands.
    % \push is useful for forcing columns apart.
    % NOTE: It might be useful to know that a gate is about 1.3 times the height of its contents.  I.e. \gate{M} is 1.3em tall.
    % WARNING: \push must appear before any wire commands and may not appear in an entry with a gate or label.
\newcommand{\gategroup}[6]{\POS"#1,#2"."#3,#2"."#1,#4"."#3,#4"!C*+<#5>\frm{#6}}
    % Constructs a box or bracket enclosing the square block spanning rows #1-#3 and columns=#2-#4.
    % The block is given a margin #5/2, so #5 should be a valid length.
    % #6 can take the following arguments -- or . or _\} or ^\} or \{ or \} or _) or ^) or ( or ) where the first two options yield dashed and
    % dotted boxes respectively, and the last eight options yield bottom, top, left, and right braces of the curly or normal variety.
    % \gategroup can appear at the end of any gate entry, but it's good form to pick one of the corner gates.
    % BUG: \gategroup uses the four corner gates to determine the size of the bounding box.  Other gates may stick out of that box.  See \prop. 
\newcommand{\rstick}[1]{*!L!<-.5em,0em>=<0em>{#1}}
    % Centers the left side of #1 in the cell.  Intended for lining up wire labels.  Note that non-gates have default size zero.
\newcommand{\lstick}[1]{*!R!<.5em,0em>=<0em>{#1}}
    % Centers the right side of #1 in the cell.  Intended for lining up wire labels.  Note that non-gates have default size zero.
\newcommand{\ustick}[1]{*!D!<0em,-.5em>=<0em>{#1}}
    % Centers the bottom of #1 in the cell.  Intended for lining up wire labels.  Note that non-gates have default size zero.
\newcommand{\dstick}[1]{*!U!<0em,.5em>=<0em>{#1}}
    % Centers the top of #1 in the cell.  Intended for lining up wire labels.  Note that non-gates have default size zero.
\newcommand{\Qcircuit}{\xymatrix @*=<0em>}
    % Defines \Qcircuit as an \xymatrix with entries of default size 0em.


\begin{document}

\section{The New Plan}

After meeting with Aram today, I decided to timebox my work on
constant-depth factoring. For example, after taking some finite steps
in the next day or two, concentrating exclusively on writing up what I do
know and moving on. This will take a lot o willpower, but as a consolation,
I can keep a list of concrete open problems to work on for later.

We did discuss the difficulty of making classical progress on purely
classical results, especially ones that are old.

Here are the concrete steps:

\begin{enumerate}
\item
See what other papers cited the ones by Bruck, especially after 1992,
and see if any are more recent than the Hoyer-Spalek paper.
\item
Ask Paul Beame if he knows of any results which are more recent.
\end{enumerate}

These steps I could do even tonight.
This one paper by Goldmann and Karpinski seems to contain an explicit
construction for the simulation of an exponentially-weighted threshold
gate by a polynomially-weighted one, proving an optimal / tight bound
for the probabilistic construction. Is that enough? There are several
mentions made that multiplication and division require exponentially
large weights, and therefore a simulation step would be necessary to
turn them into something implementable on a quantum computer.

We must fail softly. At least a day should be spent thinking what
soft failure would look like, and outlining the sections in completing
chapter one.

What are the things discussed in the meeting with Aram?

\begin{enumerate}
\item
The fact that QFT can be performed in constant depth, but the modular
multiplication part still is not known. However, there could be ways of
doing it in the QFT domain, using unbounded quantum fanout and repeated
multiplication (multiple product) and modular reduction, that don't
rely on this classical circuit construction.
\item
It is possible to construct a non-optimal constant-depth threshold circuit
for multiple product, by clever creation of partial products. A good first
step would be the multiplication of two $n$-bit numbers in constant-depth,
by showing the creation of the partial products in constant-depth, and then
using multiple sum.
\item
It is possible to create a circuit for division by concentrating on
reciprocal finding, and reducing reciprocal finding to multiplication
of $O(\log_2 n)$ terms. Using the $(1 + y^{2^r})$ expansion from
Kitaev's book, page 134.
\end{enumerate}

It seems so doable. But I may need to do it alone.

\section{Soft Failure}

I outline the steps needed for\ldots waiting the clock out.
Let's concentrate on the division task from above.
Even if I were to multiply the partial products from above in
constant depth, how would I create them? Independently, for each term
(I guess I should call them factors since I am multiplying them),
I can raise $y$ to a power, and then raise it to a power-of-two,
which is equivalent to just bitshifting.

This task appears to be further increasing my sense of self.
And I am unable to relax.
The reason why I am unable to be with others, is because I feel
threatened and cannot maintain my self-concept. As long as I am by myself,
I can pretend that I am as smart or noble or generous as I want.

\section{The Phases of Writing}

\begin{enumerate}
\item
Planning. This can be pleasant, in the sense that watching clouds or
daydreaming is pleasant.
\item
Things are a mess. You have started writing a bunch of disjointed
segments, and you don't know how they fit together. You feel the pang
of not even being able to write a stub section, or insert a TODO to
create a graphic here for later.

% TODO
% insert a figure here for later of the steps of writing
\begin{figure}
\end{figure}

\item
Things have settled into a mostly established form, but you still feel
the urge to rip out huge chunks and reorganize everything.

\item
Obsessive re-editing of increasingly minute details and mistakes.
This can feel extremely painful, in that you will judge yourself for not
being able to make an end.

\item
The end. This has stopped being painful. You know you are going to finish,
and then you simply do it.

\end{enumerate}

\section{Skills That You Want To Learn}

The practice of writing a thesis has some useful skills to teach that are
not directly related to the technical content matter of the thesis itself.
The first skill is how to overcome fear, and move forward while staying
with any feelings of unworthiness, that somehow you have to work
overwhelmingly hard in order to be good enough. This is difficult to say
the least, since the majority of my motivation to do work comes from
feeling bad. It is why I have to be careful about enjoyment or exercise
making me feel too good, because then I won't feel bad enough about myself
to continue doing whatever it is that I am doing.

A useful guide to the future would be that whatever I don't feel like
doing after a great period of exercise, I should not be doing. Anything
that requires a great deal of extraordinary effort\ldots well here it
becomes tricky. Certainly things worth doing are often hard, and require
sustained effort. Strong desire and purpose and connecting to higher values
are needed in order to overcome moments of doubt. You could roughly
call this perseverance. But when does it cross
the line into forcing yourself? And isn't forcing yourself sometimes
necessary? Isn't that what I'm doing now?

The second skill is mindfulness, of being able to recognize my process.
My mindfulness right now says that I am procrasting, so I will put a
hold on this section.

But briefly, the third skill is free thinking. Creative acts are a mess.
They are a jumble, and out of them you create something new, and polish it
until it shines. But you have to be okay with the mess, and not retreat from
it, otherwise you won't stick with it until the very end, when things seem
easy or awesome or you begin to think you could maybe possibly good at
this thing.

The fourth skill. How to relax. How to be productive in a relaxed way.

\end{document}