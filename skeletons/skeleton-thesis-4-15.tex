\documentclass{article}

\usepackage{amssymb}

\input{Qcircuit}

\begin{document}

\section{The Model \textsf{2D CCNTCM}}

In the ``best'' case, we have $\overline{W} = O(1)$, that is, a
constant number of modules. This is good because it decreases the
amount of teleportations in between modules that has to happen.
However, this is bad in the sense that it concentrates all
scalability to one contiguous lattice, which may make the circuit
size and width overly sensitive to geometry.

In general, we would like the modules to at least be linear in
size.

\begin{equation}
\frac{W}{\overline{W}} = \Omega(n)
\end{equation}

Therefore, we can write something like the following.

\begin{equation}
cn \le \frac{W}{\overline{W}} \le W
\end{equation}

\section{Majority Gate Fan-In}

How can we upper-bound the fan-out needed for any one majority gate?
Actually, we just need to upper-bound the \emph{total} fan-out of any
of the majority gates, which is actually the circuit size (in
majority gates). This does not increase the depth at all.
Doesn't this add to circuit size by a constant amount? See next section,
we need to first find out the majority circuit, and double-check the
figures in Andrea's report.

QUESTION: Can we asymptotically affect circuit width by uncomputing,
and without asymptotically affecting circuit depth?

For each majority gate, we can uncompute everything in constant-depth
except the fanout. And that we only know how to do in log-depth,
which would ruin the log-log-squared thing that we have going on. And
the unfanout, would that necessarily restore all the ancillae to
$\ket{0}$?

Also, this highlights the need for circuit coherence over just
circuit width. Even if we uncompute these ancillae and make them
reusable, they are in different modules, sitting idle.

Given that the resources for fan-out are linear in size and in width,
we would add at most a polynomial in size and width (and only constant
depth) to each CCNTCM implementation of a majority gate.

\section{Majority Gate Resources}

For the $\mu$ gate, we count depth as one fanout gate (depth 7) plus a
row of controlled $R_z(\phi)$ rotations.

QUESTION: How are these controlled rotations performed? Do we use a CNOT
decomposition? Such as might be done in a QFT. I feel like I used to know
this like the back of my hand.

Plus we do an unfanout. The unfanout is not just necessary to recover
ancillae, it is actually an inherent part of the algorithm. Therefore,
the ability to do this in constant-depth is a necessary part of your
thesis. I don't think this counts as soft failure.
You can't put this off after all. Exercising your independent scientific
judgment.

By Andrea's counting this gives us depth $3$, width of $2n+2$, and
size of $n+2$.

QUESTION: What is $m$ in this case? For the $\mu$ gate.

\section{Total Resources}

The resources for multiple product are: depth $O(\frac{1}{\epsilon})$,
size $O(\frac{1}{\epsilon}n^{3+2\epsilon})$, fan-in $O(n)$,
width $O(\frac{1}{\epsilon}n^{4+2\epsilon})$. There doesn't seem to be a
need to choose an $\epsilon$ at this point.

\end{document}