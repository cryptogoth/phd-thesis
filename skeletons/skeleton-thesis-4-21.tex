\documentclass{article}

\usepackage{amssymb}

%    Q-circuit version 1.06
%    Copyright (C) 2004  Steve Flammia & Bryan Eastin

%    This program is free software; you can redistribute it and/or modify
%    it under the terms of the GNU General Public License as published by
%    the Free Software Foundation; either version 2 of the License, or
%    (at your option) any later version.
%
%    This program is distributed in the hope that it will be useful,
%    but WITHOUT ANY WARRANTY; without even the implied warranty of
%    MERCHANTABILITY or FITNESS FOR A PARTICULAR PURPOSE.  See the
%    GNU General Public License for more details.
%
%    You should have received a copy of the GNU General Public License
%    along with this program; if not, write to the Free Software
%    Foundation, Inc., 59 Temple Place, Suite 330, Boston, MA  02111-1307  USA

\usepackage[matrix,frame,arrow]{xy}
\usepackage{amsmath}
\newcommand{\bra}[1]{\left\langle{#1}\right\vert}
\newcommand{\ket}[1]{\left\vert{#1}\right\rangle}
    % Defines Dirac notation.
\newcommand{\qw}[1][-1]{\ar @{-} [0,#1]}
    % Defines a wire that connects horizontally.  By default it connects to the object on the left of the current object.
    % WARNING: Wire commands must appear after the gate in any given entry.
\newcommand{\qwx}[1][-1]{\ar @{-} [#1,0]}
    % Defines a wire that connects vertically.  By default it connects to the object above the current object.
    % WARNING: Wire commands must appear after the gate in any given entry.
\newcommand{\cw}[1][-1]{\ar @{=} [0,#1]}
    % Defines a classical wire that connects horizontally.  By default it connects to the object on the left of the current object.
    % WARNING: Wire commands must appear after the gate in any given entry.
\newcommand{\cwx}[1][-1]{\ar @{=} [#1,0]}
    % Defines a classical wire that connects vertically.  By default it connects to the object above the current object.
    % WARNING: Wire commands must appear after the gate in any given entry.
\newcommand{\gate}[1]{*{\xy *+<.6em>{#1};p\save+LU;+RU **\dir{-}\restore\save+RU;+RD **\dir{-}\restore\save+RD;+LD **\dir{-}\restore\POS+LD;+LU **\dir{-}\endxy} \qw}
    % Boxes the argument, making a gate.
\newcommand{\meter}{\gate{\xy *!<0em,1.1em>h\cir<1.1em>{ur_dr},!U-<0em,.4em>;p+<.5em,.9em> **h\dir{-} \POS <-.6em,.4em> *{},<.6em,-.4em> *{} \endxy}}
    % Inserts a measurement meter.
\newcommand{\measure}[1]{*+[F-:<.9em>]{#1} \qw}
    % Inserts a measurement bubble with user defined text.
\newcommand{\measuretab}[1]{*{\xy *+<.6em>{#1};p\save+LU;+RU **\dir{-}\restore\save+RU;+RD **\dir{-}\restore\save+RD;+LD **\dir{-}\restore\save+LD;+LC-<.5em,0em> **\dir{-} \restore\POS+LU;+LC-<.5em,0em> **\dir{-} \endxy} \qw}
    % Inserts a measurement tab with user defined text.
\newcommand{\measureD}[1]{*{\xy*+=+<.5em>{\vphantom{#1}}*\cir{r_l};p\save*!R{#1} \restore\save+UC;+UC-<.5em,0em>*!R{\hphantom{#1}}+L **\dir{-} \restore\save+DC;+DC-<.5em,0em>*!R{\hphantom{#1}}+L **\dir{-} \restore\POS+UC-<.5em,0em>*!R{\hphantom{#1}}+L;+DC-<.5em,0em>*!R{\hphantom{#1}}+L **\dir{-} \endxy} \qw}
    % Inserts a D-shaped measurement gate with user defined text.
\newcommand{\multimeasure}[2]{*+<1em,.9em>{\hphantom{#2}} \qw \POS[0,0].[#1,0];p !C *{#2},p \drop\frm<.9em>{-}}
    % Draws a multiple qubit measurement bubble starting at the current position and spanning #1 additional gates below.
    % #2 gives the label for the gate.
    % You must use an argument of the same width as #2 in \ghost for the wires to connect properly on the lower lines.
\newcommand{\multimeasureD}[2]{*+<1em,.9em>{\hphantom{#2}}\save[0,0].[#1,0];p\save !C *{#2},p+LU+<0em,0em>;+RU+<-.8em,0em> **\dir{-}\restore\save +LD;+LU **\dir{-}\restore\save +LD;+RD-<.8em,0em> **\dir{-} \restore\save +RD+<0em,.8em>;+RU-<0em,.8em> **\dir{-} \restore \POS !UR*!UR{\cir<.9em>{r_d}};!DR*!DR{\cir<.9em>{d_l}}\restore \qw}
    % Draws a multiple qubit D-shaped measurement gate starting at the current position and spanning #1 additional gates below.
    % #2 gives the label for the gate.
    % You must use an argument of the same width as #2 in \ghost for the wires to connect properly on the lower lines.
\newcommand{\control}{*-=-{\bullet}}
    % Inserts an unconnected control.
\newcommand{\controlo}{*!<0em,.04em>-<.07em,.11em>{\xy *=<.45em>[o][F]{}\endxy}}
    % Inserts a unconnected control-on-0.
\newcommand{\ctrl}[1]{\control \qwx[#1] \qw}
    % Inserts a control and connects it to the object #1 wires below.
\newcommand{\ctrlo}[1]{\controlo \qwx[#1] \qw}
    % Inserts a control-on-0 and connects it to the object #1 wires below.
\newcommand{\targ}{*{\xy{<0em,0em>*{} \ar @{ - } +<.4em,0em> \ar @{ - } -<.4em,0em> \ar @{ - } +<0em,.4em> \ar @{ - } -<0em,.4em>},*+<.8em>\frm{o}\endxy} \qw}
    % Inserts a CNOT target.
\newcommand{\qswap}{*=<0em>{\times} \qw}
    % Inserts half a swap gate. 
    % Must be connected to the other swap with \qwx.
\newcommand{\multigate}[2]{*+<1em,.9em>{\hphantom{#2}} \qw \POS[0,0].[#1,0];p !C *{#2},p \save+LU;+RU **\dir{-}\restore\save+RU;+RD **\dir{-}\restore\save+RD;+LD **\dir{-}\restore\save+LD;+LU **\dir{-}\restore}
    % Draws a multiple qubit gate starting at the current position and spanning #1 additional gates below.
    % #2 gives the label for the gate.
    % You must use an argument of the same width as #2 in \ghost for the wires to connect properly on the lower lines.
\newcommand{\ghost}[1]{*+<1em,.9em>{\hphantom{#1}} \qw}
    % Leaves space for \multigate on wires other than the one on which \multigate appears.  Without this command wires will cross your gate.
    % #1 should match the second argument in the corresponding \multigate. 
\newcommand{\push}[1]{*{#1}}
    % Inserts #1, overriding the default that causes entries to have zero size.  This command takes the place of a gate.
    % Like a gate, it must precede any wire commands.
    % \push is useful for forcing columns apart.
    % NOTE: It might be useful to know that a gate is about 1.3 times the height of its contents.  I.e. \gate{M} is 1.3em tall.
    % WARNING: \push must appear before any wire commands and may not appear in an entry with a gate or label.
\newcommand{\gategroup}[6]{\POS"#1,#2"."#3,#2"."#1,#4"."#3,#4"!C*+<#5>\frm{#6}}
    % Constructs a box or bracket enclosing the square block spanning rows #1-#3 and columns=#2-#4.
    % The block is given a margin #5/2, so #5 should be a valid length.
    % #6 can take the following arguments -- or . or _\} or ^\} or \{ or \} or _) or ^) or ( or ) where the first two options yield dashed and
    % dotted boxes respectively, and the last eight options yield bottom, top, left, and right braces of the curly or normal variety.
    % \gategroup can appear at the end of any gate entry, but it's good form to pick one of the corner gates.
    % BUG: \gategroup uses the four corner gates to determine the size of the bounding box.  Other gates may stick out of that box.  See \prop. 
\newcommand{\rstick}[1]{*!L!<-.5em,0em>=<0em>{#1}}
    % Centers the left side of #1 in the cell.  Intended for lining up wire labels.  Note that non-gates have default size zero.
\newcommand{\lstick}[1]{*!R!<.5em,0em>=<0em>{#1}}
    % Centers the right side of #1 in the cell.  Intended for lining up wire labels.  Note that non-gates have default size zero.
\newcommand{\ustick}[1]{*!D!<0em,-.5em>=<0em>{#1}}
    % Centers the bottom of #1 in the cell.  Intended for lining up wire labels.  Note that non-gates have default size zero.
\newcommand{\dstick}[1]{*!U!<0em,.5em>=<0em>{#1}}
    % Centers the top of #1 in the cell.  Intended for lining up wire labels.  Note that non-gates have default size zero.
\newcommand{\Qcircuit}{\xymatrix @*=<0em>}
    % Defines \Qcircuit as an \xymatrix with entries of default size 0em.


\begin{document}

\section{Questions}

To begin with

\begin{enumerate}
\item
What is quantum Hamming weight? Why does it make sense to take the
Hamming weight of a quantum state, and how is the output qubit entangled
with each input state?

\item
What is the rotation by Hamming value used for?

\item
In Hoyer-Spalek, the constant-depth of Or and other quantum logic gates
seems to depend on taking the gates from a arbitrary single-qubit
rotations, rather than a fixed set. However, in the beginning sections,
we do not make any such assumptions, and yet the assumption is made
of constant-depth using a fixed basis. How do we get around this
quantum compiling problem?

\item
Useful papers to add to our reading list include the result that
a rotation about the $Z$-axis of an irrational multiple of $\pi$,
together with Hadamard and CNOT, are universal, when in fact we
know that Hadamard, the T gate ($R_z(\pi/8)$), and CNOT are normally
considered universal everywhere else. How do we resolve this consistency?

\item
Is it worth computing exact parameters for KSV compiling? Soft failure
says no, just keep it at asymptotics at this stage.

\item
What is the implementation of controlled-$R_z$, using some
standard decomposition?

\end{enumerate}

\section{Exact Gate in 2D CCNTC}

The $exact[t]$ gate is defined as the $(n+1)$-qubit operation which takes
$\ket{x}\ket{y} \rightarrow \ket{x}\ket{y \oplus g(x)}$ where
we define $g(x) = 1$ iff $|x| = t$. It is achieved by a rotation by Hamming
weight using a basic angle $\phi = 2\pi / m$, where $m$ is determined by
the $or$ gate, and an added rotation of $-\phi\cdot t$.

\section{Or Gate in 2D CCNTC}

From Hoyer-Spalek, the OR gate 

\section{Rotation by Hamming Weight in 2D CCNTC}

This is a basic gate in Hoyer-Spalek which makes use of unbounded-fanout.
It takes as input $n$ input qubits and an output qubit in the
$\ket{+}$ state which is an equal superposition of $\ket{0}$ and $\ket{1}$.
It modifies the phase on the $\ket{1}$ component of the output qubit 
based on the Hamming weight of the input. Since the input register $\ket{x}$
is in general a superposition of $2^n$ computational basis states, the
output qubit is entangled to encode a rotation based on the Hamming weight
of a particular input $x$.

Therefore, the notion of a \emph{quantum Hamming weight} appears to be ill-
defined.
writes the output in a single
qubit, giving the following $(n+1)$-qubit operation.

\begin{equation}
\sum_{x \in \mathcal{B}^n} \ket{x}\left(\frac{\ket{0} + \ket{1}}{\sqrt{2}}\right)
\rightarrow
\sum_{x \in mathcal{B}^n} \ket{x}\left(\frac{\ket{0} + e^{i\phi|x|}\ket{1}}{\sqrt{2}}\right)
\end{equation}

It is denoted $\mu^{|x|}_\phi$, meaning that the phase of output qubit
is shifted by $|x|\cdot \phi$. Note that we do not mention what to do
with the output qubit. Obviously this phase is not measurable, but under
certain circumstances mentioned in the succeeding sections, a Hadamard
can be applied to this gate, in which case the phase does become
measurable in the $Z$ basis. Based on the values of $\phi$, and additional
rotations, this can be used to implement various logic gates.

As it stands however, assuming that controlled-$R_z$ gates can be done
in constant depth, the depth of this gate is one for the Hadamard,
the depth of unbounded quantum fanout (which we know to be $9$), the
controlled-$R_z$ rotation, which we can assume to approximate with
resolution $2^-\delta$, so that the depth according to KSV-style
approximation is $O(\log^2 \delta)$, plus a depth of $6$ for the
unfanout. For $\delta = O(\log n')$, which is the case if we can
bound the fan-in of this gate by some $n'$ related to the larger
overall problem, size, the depth then becomes $16 + O((\log\log \delta)^2)$.

The size of this gate is one for the Hadamard, plus $10n-9$ for the
fanout, plus $n \times O()$ (fact check this, we need the size for KSV),
plus $3n+2$ for the unfanout.

The width of this gate is just $2n +1$.

\section{Half-or-Nothing Gate in 2D CCNTC}

The $\mu^{|x|}_\phi$ gate from the previous section is useful for
implementing approximately 

The basic gate from Hoyer-Spalek lets us approximate a ``half-or-nothing'' 
gate on $n$ input qubits $x = x_1 x_2  \ldots x_n$. That is, it returns a
$1$ if $|x| = n/2$ and $0$ if $|x| = 0$, exactly, whatever that means
for quantum Hamming weight. Note that this is not a
total function, since we are not always promised that $x$ has Hamming weight
of exactly half ones or all zeros. In this case, we use the $\mu$ gate
previously with $\phi = m$.

What is the relationship between $m$ and $n$ in the case of this gate?
We seem to need the idea of an \emph{expected Hamming weight}.

Is this gate exact? Yes, for the partial function that it is.

\section{Expected Hamming Weight}

WRITE ABOUT THIS. This appears to be the key ingredient for doing everything
else. It is contained in the proof of Theorem 4, on page 7 of Hoyer-Spalek.

\section{OR Gate}



\end{document}