%%%%%%%%%%%%%%%%%%%%%%%%%%%%%%%%%%%%%%%%%%%%%%%%%%%%%%%%%%%%%%%%%%%%%%%%%%%%%%
% Ph.D. Thesis proposal
% Paul Pham
% University of Washington
%%%%%%%%%%%%%%%%%%%%%%%%%%%%%%%%%%%%%%%%%%%%%%%%%%%%%%%%%%%%%%%%%%%%%%%%%%%%%%

%\documentclass[runningheads]{llncs}
% Suppress page numbers
%\documentclass[a4paper]{llncs}
% For arXiv, and eprint support
\documentclass[]{article}

\usepackage{amssymb}
\usepackage{amsthm}
\usepackage{graphicx}
\usepackage{hyperref}
\usepackage{eprint}
\usepackage[osf]{mathpazo} % Use Palatino / Euler fonts

\usepackage{url}
\urldef{\mailsa}\path|{alfred.hofmann, ursula.barth, ingrid.beyer, christine.guenther,|
\urldef{\mailsb}\path|frank.holzwarth, anna.kramer, erika.siebert-cole, lncs}@springer.com|
\newcommand{\keywords}[1]{\par\addvspace\baselineskip
\noindent\keywordname\enspace\ignorespaces#1}

% Right brace for multirows in tables / arrays
\newcommand\coolrightbrace[2]{%
\left.\vphantom{\begin{matrix} #1 \end{matrix}}\right\}#2}

% To fix Qcircuit target with new Xypic
\newcommand{\targfix}{\qw {\xy {<0em,0em> \ar @{ - } +<.39em,0em>
\ar @{ - } -<.39em,0em> \ar @{ - } +
<0em,.39em> \ar @{ - }
-<0em,.39em>},<0em,0em>*{\rule{.01em}{.01em}}*+<.8em>\frm{o}
\endxy}}

% To get Roman uppercase Greek characters
\newcommand{\X}[1]{$#1$}

\input{Qcircuit}

\newcommand{\braket}[2]{\langle #1|#2 \rangle}
\newcommand{\normtwo}{\frac{1}{\sqrt{2}}}
\newcommand{\norm}[1]{\parallel #1 \parallel}
\newcommand{\email}[1]{\href{mailto:#1}{#1}}
\theoremstyle{plain} \newtheorem{lemma}{Lemma}

\begin{document}

%\mainmatter  % start of an individual contribution

% first the title is needed
\title{Thesis Proposal: Low-depth quantum architectures}

% a short form should be given in case it is too long for the running head
%\titlerunning{A 2D Nearest-Neighbor Quantum Architecture for Factoring}

% the name(s) of the author(s) follow(s) next
%
% NB: Chinese authors should write their first names(s) in front of
% their surnames. This ensures that the names appear correctly in
% the running heads and the author index.
%
\author{Paul Pham\\
University of Washington\\
\email{ppham@cs.washington.edu}
}
% if the list of authors exceeds the space for a headline,
% an abbreviated author list is needed
%\authorrunning{P. Pham \and K.M. Svore}
% (feature abused for this document to repeat the title also on left hand pages)

\maketitle

\section{Committee}

\begin{itemize}
\item Aram Harrow (chair)
\item Paul Beame
\item Mark Oskin
\item Boris Blinov (GSR)
\end{itemize}

\section{Abstract}

Building upon my generals exam, I combine the common threads of
realistic architectural constraints (e.g. nearest-neighbor
interactions),
quantum circuit construction techniques (e.g. parallelization),
quantum compiling, and
a new circuit resource called circuit coherence. I develop these threads
in the context of
the well-known Shor factoring algorithm and then generalize them to
a new algorithm, Hamiltonian simulation. I calculate closed formulae
for the resources needed
to run these algorithms on input sizes of interest. Thus, I characterize the
possibility of trading space (circuit width, up to billions of
qubits, for time (circuit depth, up to centuries of time).

\textbf{Thesis Statement: Advances in nearest-neighbor quantum architectures,
quantum compilation, and quantum circuit analysis can help us design
quantum computers to solve human problems within human lifetimes.}

\pagebreak
%%%%%%%%%%%%%%%%%%%%%%%%%%%%%%%%%%%%%%%%%%%%%%%%%%%%%%%%%%%%%%%%%%%%%%%%%%%%%%
%%%%%%%%%%%%%%%%%%%%%%%%%%%%%%%%%%%%%%%%%%%%%%%%%%%%%%%%%%%%%%%%%%%%%%%%%%%%%%
\section{Updated and Proposed Timeline}

\begin{description}
\item[February 21]
I sent this proposal to committee, accept revisions.

\item[March 30]
Finish first part of Chapter 1, on 2D factoring in polylogarithmic
depth.

\item[April 30]
Finish remaining part of Chapter 1, on 2D factoring in sublogarithmic
depth.

\item[May 7]
Finish Chapter 2 on quantum compiling. Submit to reading committee.

\item[May 14]
Finish Chapter 3 on quantum circuit coherence.
Submit to reading committee.

\item[May  24th]
Finish Chapter 4 on hamiltonian simulation.
Submit to reading committee.
Beginning of two week reading period.

\item[June 7]
Tentative final exam with full committee present, in CSE 503, 10:30am - 12:30pm.

\end{description}

%%%%%%%%%%%%%%%%%%%%%%%%%%%%%%%%%%%%%%%%%%%%%%%%%%%%%%%%%%%%%%%%%%%%%%%%%%%%%%
%%%%%%%%%%%%%%%%%%%%%%%%%%%%%%%%%%%%%%%%%%%%%%%%%%%%%%%%%%%%%%%%%%%%%%%%%%%%%%
\section{Proposed Chapters}

\begin{enumerate}

%%%%%%%%%%%%%%%%%%%%%%%%%%%%%%%%%%%%%%%%%%%%%%%%%%%%%%%%%%%%%%%%%%%%%%%%%%%%%%
% Chapter 1

\item
\textbf{Factoring on a 2D nearest-neighbor architecture.}

In this chapter, we discuss the work
Pham-Svore 2012 \cite{Pham2012}, where we mapped Shor's factoring algorithm
to a 2D nearest-neighbor quantum architecture (with classical controller) in
polylogarithmic depth, an exponential improvement over the previous
best-known architecture (in 1D).
Furthermore, we will improve these
results to be constant-depth, which is optimal even if we allow an
architecture with arbitrary interactions. We will use the techniques,
namely the unbounded quantum fanout of H{\o}yer-{\v S}palek 2002
\cite{Hoyer2002}, the quantum threshold circuit given in
Takahashi-Tani 2011 \cite{Takahashi2011}, and the original classical circuit
given in Siu et al. 1993 \cite{Siu1993}.
We will calculate the circuit resources required by our factoring architectures
and compare them to previous implementations, especially the best-known
1D nearest-neighbor algorithm by Kutin 2006 \cite{Kutin2006}.

This chapter will also include a background of quantum architecture, and
physical justification for the realism and utility of our chosen models.
It will also review other work on mapping quantum gates of
arbitrary connectivity to a nearest-neighbor architecture such as
Rosenbaum 2012 \cite{Rosenbaum2012} and Biels et al. 2012 \cite{Beals2012}.

\pagebreak

%%%%%%%%%%%%%%%%%%%%%%%%%%%%%%%%%%%%%%%%%%%%%%%%%%%%%%%%%%%%%%%%%%%%%%%%%%%%%%
% Chapter 2

\item
\textbf{Quantum compiling on a nearest-neighbor architecture.}

This section is primarily a pedagogical literature review which builds upon my
quals project.

Quantum compiling is the approximation of any quantum gate to arbitrary
precision using gates from a universal, finite, discrete set. Along with
error correction, it provides one of the biggest overheads in the realistic
implementation of a quantum algorithm. In this case,
we limit ourselves to approximating single-qubit gates, which is sufficient
to approximate multi-qubit gates using known decompositions given in
Kitaev-Shen-Vyalyi 2002 \cite{Kitaev2002}, Svore-Aho 2003 \cite{Aho2003},
and Saeedi-Markov 2011 \cite{Saeedi2011}.

There has been much recent
work in approximating single-qubit gates and improving empirical performance
using heuristics, such as 
Amy et al. 2012 \cite{Amy2012}, Booth 2012 \cite{Booth2012},
Kliuchnikov et al. 2012 \cite{Kliuchnikov2012a},
Eastin 2012 \cite{Eastin2012}
Selinger 2012 \cite{Selinger2012}, Bocharov-Svore 2012 \cite{Bocharov2012}.
An interesting model which uses the technique of Kitaev-Shen-Vyalyi 2002
\cite{Kitaev2002} to prepare programmable ancillae offline and then apply them
in constant depth at runtime is given in Jones et al. \cite{Jones2012}.
There has also been an asymptotic improvement by Duclos-Cianci-Svore 2012 \cite{Duclos-Cianci2012}.
These all improve upon the original results by Solovay-Kitaev 1995-1997 as
formulated in Dawson-Nielsen 2005 \cite{Dawson2005} and a later result by
Kitaev-Shen-Vyalyi 2002 \cite{Kitaev2002}.

We will give a pedagogical
review all of these works and compare their relative performances on
architectures with arbitrary interactions. As well, we will
compute the circuit resource overhead in mapping one or more of these
methods to our 2D nearest-neighbor architecture with classical controller,
as time allows.

%%%%%%%%%%%%%%%%%%%%%%%%%%%%%%%%%%%%%%%%%%%%%%%%%%%%%%%%%%%%%%%%%%%%%%%%%%%%%%
% Chapter 3

\item
\textbf{Quantum coherence versus measurement patterns.}

In this chapter, I re-introduce 
a new circuit resource first mentioned in my generals report
which I now call \emph{circuit coherence}. Roughly defined,
it is the time-space product of the coherent computation
state that ends in measurement to produce a circuit's final classical output.
It has units of \emph{qubits$\cdot$ timestep}. Intuitively, it measures the
amount of experimental labor and error-correction needed to maintain a
coherent, entangled state for computation.

Circuit size is counted as the
number of two-qubit (nearest-neighbor) gates in a circuit, and a timestep is
the unit depth of a two-qubit gate. Then circuit coherence is upper-bounded by the
product of circuit depth times width, which in the worst case means that
all qubits in a quantum computer must remain entangled for the entire
runtime of the circuit. It is lower-bounded by the circuit size, which in
the worst case means that every two-qubit gate is alternately entangling
or disentagling a qubit from the computation state.

The questions examined in this section are as follows:

\begin{enumerate}
\item
Is circuit coherence a well-defined circuit resource? Are there pathological
cases to be handled, and if so, what are the workarounds to the definition
given above?

\item
Are there transformations to a circuit that decrease circuit coherence while
still computing the same function? What are some specific examples? Can we
generalize some principles, properties, or pseudocode that would allow us to
automatically determine that a circuit is ``canonical''?
That is, a canonical circuit has the minimum circuit coherence among
all its equivalent representations.

\item
Is circuit coherence asymptotically separated from the upper and lower bounds
given above? If not, what are specific examples where there is a separation
between coherence and depth $\times$ width? 

\item
Is circuit coherence different from the measurement patterns defined in
Broadbent-Kasheffi 2007 \cite{Broadbent2007} ?

\end{enumerate}

%%%%%%%%%%%%%%%%%%%%%%%%%%%%%%%%%%%%%%%%%%%%%%%%%%%%%%%%%%%%%%%%%%%%%%%%%%%%%%%
% Chapter 4

\item
\textbf{Mapping the Hamiltonian simulation algorithm to a 2D nearest-neighbor
architecture.}

A very different flavor of quantum algorithm is Hamiltonian simulation, the
original problem Richard Feynman proposed for which a quantum mechanical
machine might be exponentially faster than a classical machine. Hamiltonian
simulation, or the simulation of an inherently quantum physical system
with many-body interactions, has been attracting a lot of attention recently.
We can
conceivably demonstrate quantum speedups in experimental implementations
of Hamiltonian simulation in the
near future that match classical supercomputer performance on modest input
sizes. We cannot say the same for factoring for at least many more years.

This chapter is partly a review of related work on Hamiltonian simulation
but is primarily original research on mapping a $k$-sparse Hamiltonian
to our 2D
nearest-neighbor architecture with classical controller. We also calculate
the circuit resources for such a mapping. This gives a concrete
implementation for the basic building block of the Hamiltonian simulation
steps in Berry et al. 2005 \cite{Berry2006} and Aharanov and Ta-Shma
2003 \cite{Aharonov2003}.

This chapter is not meant to be a comprehensive survey, but rather a
novel and concrete application of the previous three chapters
(architectural lessons learned from the factoring algorithm,
quantum compiling as a subroutine to quantum algorithms, and circuit coherence)
to a new algorithm.

\item
\textbf{Conclusion.} We conclude with interesting open problems and ways to
extend the work of these four chapters in the future.

\end{enumerate}

\pagebreak

\bibliography{proposal}
\bibliographystyle{tocplain}

\end{document}
