%  ========================================================================
%  Copyright (c) 1995-2012 The University of Washington
%
%  Licensed under the Apache License, Version 2.0 (the "License");
%  you may not use this file except in compliance with the License.
%  You may obtain a copy of the License at
%
%      http://www.apache.org/licenses/LICENSE-2.0
%
%  Unless required by applicable law or agreed to in writing, software
%  distributed under the License is distributed on an "AS IS" BASIS,
%  WITHOUT WARRANTIES OR CONDITIONS OF ANY KIND, either express or implied.
%  See the License for the specific language governing permissions and
%  limitations under the License.
%  ========================================================================
%

% Documentation for UW thesis document style for LaTeX
% by Jim Fox
% fox@washington.edu
%
%    Revised for version 2012/06/19 of uwthesis.cls
%
%    This document is contained in a single file ONLY because
%    I wanted to be able to distribute it easily.  A real thesis ought
%    to be contained on many files (e.g., one for each chapter, at least).
%
%    To help you identify the files and sections in this large file
%    I use the string '==========' to identify new files.
%
%    To help you ignore the unusual things I do with this sample document
%    I try to use the notation
%       
%    % --- sample stuff only -----
%    special stuff for my document, but you don't need it in your thesis
%    % --- end-of-sample-stuff ---


%    Printed in twoside style now that that's allowed
%
 
\documentclass [11pt, twoside] {uwthesis}[2012/06/19]
 
%
% The following line would print the thesis in a postscript font 

% \usepackage{natbib}
% \def\bibpreamble{\protect\addcontentsline{toc}{chapter}{Bibliography}}

\setcounter{tocdepth}{1}  % Print the chapter and sections to the toc
 

% ==========   Local defs and mods
%

% --- sample stuff only -----
% These format the sample code in this document

\usepackage{alltt}  % 
\newenvironment{demo}
  {\begin{alltt}\leftskip3em
     \def\\{\ttfamily\char`\\}%
     \def\{{\ttfamily\char`\{}%
     \def\}{\ttfamily\char`\}}}
  {\end{alltt}}
 
% metafont font.  If logo not available, use the second form
%
% \font\mffont=logosl10 scaled\magstep1
\let\mffont=\sf
% --- end-of-sample-stuff ---
 
\usepackage{epsfig}
\usepackage{graphicx}% Include figure files
\usepackage{dcolumn}% Align table columns on decimal point
\usepackage{bm}% bold math
\usepackage{hyperref}% add hypertext capabilities
\usepackage{amsthm}
\usepackage{amsmath}
\usepackage{amssymb}
\usepackage[osf]{mathpazo} % Use Palatino / Euler fonts
\usepackage{algorithmic}
\usepackage{lscape}

\newcommand{\sgn}{\text{sgn}}
\newcommand{\avg}[1]{\langle #1 \rangle}
\newcommand{\braket}[2]{\langle #1|#2 \rangle}
\newcommand{\normtwo}{\frac{1}{\sqrt{2}}}
\newcommand{\norm}[1]{\parallel #1 \parallel}
\newcommand{\card}[1]{\left| #1 \right|}
\newcommand{\ci}{\perp\!\!\!\perp}
\newtheorem{definition}{Definition}
\newtheorem{lemma}{Lemma}
\newtheorem{theorem}{Theorem}
\newtheorem{conjecture}{Conjecture}

% To fix Qcircuit target with new Xypic
\newcommand{\targfix}{\qw {\xy {<0em,0em> \ar @{ - } +<.39em,0em>
\ar @{ - } -<.39em,0em> \ar @{ - } +
<0em,.39em> \ar @{ - }
-<0em,.39em>},<0em,0em>*{\rule{.01em}{.01em}}*+<.8em>\frm{o}
\endxy}}


\input{Qcircuit}

\begin{document}
 
% ==========   Preliminary pages
%
% ( revised 2012 for electronic submission )
%

\prelimpages
 
%
% ----- copyright and title pages
%
\Title{Low-depth quantum architectures}
\Author{Paul Pham}
\Year{2013}
\Program{University of Washington Computer Science \& Engineering}

\Chair{Aram Harrow}{Visiting Professor}{Computer Science \& Engineering}
\Signature{Paul Beame}
\Signature{Mark Oskin}
\Signature{Boris Blinov}

\copyrightpage

% \titlepage  

% --- sample stuff only -----
% unusual footnote not found in a real thesis
% You just use the \titlepage as commented out above

{\Degreetext{A dissertation \\
%  \footnote[2]{an egocentric imitation, actually}\\
  submitted in partial fulfillment of the\\ requirements for the degree of}
 \def\thefootnote{\fnsymbol{footnote}}
 \let\footnoterule\relax
 \titlepage
 }
\setcounter{footnote}{0}

% --- end-of-sample-stuff ---
 
%
% ----- signature and quoteslip are gone
%

%
% ----- abstract
%


\setcounter{page}{-1}
\abstract{%

} 
%
% ----- contents & etc.
%
\tableofcontents
\listoffigures
%\listoftables  % I have no tables
 
%
% ----- glossary 
%
\chapter*{Glossary}      % starred form omits the `chapter x'
\addcontentsline{toc}{chapter}{Glossary}
\thispagestyle{plain}
%
\begin{glossary}
\item[quantum architecture] the study of how to lay out quantum bits and their
allowed interactions in a physical system to execute quantum algorithms
efficiently in time, space, and other resources.
 
\end{glossary}
 
%
% ----- acknowledgments
%
\acknowledgments{% \vskip2pc
  % {\narrower\noindent
  This section remains to be written.
    % \par}
}

%
% ----- dedication
%
\dedication{\begin{center}to be decided\end{center}}

%
% end of the preliminary pages
 
 
 
%
% ==========      Text pages
%

\textpages
 
% ========== Chapter 1
\chapter {Shor's Factoring Algorithm on a Nearest-Neighbor Architecture}
\label{chap:factor-polylog}

\section{Abstract}
%%%%%%%%%%%%%%%%%%%%
% put abstract here
%%%%%%%%%%%%%%%%%%%%
We present a 2D nearest-neighbor
quantum architecture for Shor's algorithm to factor an $n$-bit number in $O(\log^3n)$ depth.
Our implementation uses
%(1)
parallel phase estimation,
%(due to Kitaev, Shen, and Vyalyi),
%(2)
constant-depth fanout and teleportation,
%(due to Harrow, Fowler, and Taylor),
and
%(3)
constant-depth carry-save modular addition.
%(due to Gossett).
%We introduce a novel 2D architectural variation on Gossett's modular arithmetic
%and interleave constant-depth fanout and teleportation circuits for
%nearest-neighbor and long-distance communication channels, and ultimately use
%our circuit within parallel phase estimation to achieve quantum factoring.
We derive upper bounds on the circuit resources of our architecture under a
new 2D model which allows a classical controller and parallel, communicating
modules.
We provide a comparison to all previous nearest-neighbor factoring
implementations.  
Our circuit results in an exponential improvement in nearest-neighbor circuit depth at the cost of a polynomial increase in circuit size and width.

This is the first of two chapters which contribute a low-depth
implementation of Shor's factoring algorithm on our hybrid nearest-neighbor
architecture, \textsf{2D CCNTCM}. In support of our thesis,
we are studying the depth of factoring architectures by minimizing
their depth, and therefore decreasing the running time of this
particular algorithm. Our main result is the
factoring of an
$n$-bit number in circuit depth $O(\log^2(n))$
with a circuit size and width of $O(n^4)$.
This is an exponential improvement in circuit depth
over all previous implementations
\cite{Beauregard2002,Kutin2006,VanMeter2006,VanMeter2005,VanMeterIL2005}
at the cost of a polynomial increase in circuit size and width.

We use the
following three key techniques:
parallel phase estimation due to Kitaev-Shen-Vyalyi \cite{Kitaev2002};
constant-depth communication introduced in Section \ref{sec:intro-cdc}
including our circuit for quantum unfanout,
and 
constant-depth carry-save modular addition due to Gossett \cite{Gossett1998}.

Section \ref{sec:fpl-related} places our work in the context of existing
results.
In Section \ref{sec:csa}, we provide a self-contained pedagogical review
of the carry-save technique and encoding.
We use this foundation to construct building blocks of increasing complexity,
deriving numerical upper bounds on circuit and module resources
on \textsf{2D CCNTCM} as we go along.
In Section \ref{sec:csa-mod-add} we modify and extend the carry-save technique to a 2D
modular adder,
which we then use as a basis for a modular multiplier
(Section \ref{sec:csa-mod-mult}) and a modular exponentiator
(Section \ref{sec:modexp}), which contains our
main result of polylogarithmic-depth factoring.


In Section \ref{sec:fpl-results}, we provide a comparison of our \textsf{2D CCNTCM}
hybrid factoring architecture to all previous nearest-neighbor factoring
implementations. We also compare our hybrid architecture with a
variant on the non-hybrid model \textsf{2D CCNTC} to show the
benefit of using modules. Finally, we conclude our exploration
of polylogarithmic-depth hybrid factoring in Section
\ref{sec:fpl-conclude}.

%\input{factor-polylog/fpl-arch}

%\input{factor-polylog/fpl-cdc}

%%%%%%%%%%%%%%%%%%%%%%%%%%%%%%%%%%%%%%%%%%%%%%%%%%%%%%%%%%%%%%%%%%%%%%%%%%%%%%
\section{Related Work}
\label{sec:fpl-related}

Our work builds upon ideas in classical digital and reversible logic and their extension to quantum logic.
Any circuit implementation for Shor's algorithm requires a quantum adder.
Gossett proposed a quantum algorithm for addition using classical carry-save techniques to add
in constant-depth and multiply in logarithmic-depth, with a quadratic
cost in qubits (circuit width) \cite{Gossett1998}. The techniques relies on encoded addition, sometimes
called a $3 \rightarrow 2$ adder, and derives from classical Wallace trees \cite{Wallace1964}.

Takahashi and Kunihiro discovered a linear-depth
and linear-size quantum adder using zero ancillae \cite{Takahashi2005}.
They also developed an adder with tradeoffs between $O(n/d(n))$ ancillae and
$O(d(n))$-depth for $d(n) = \Omega(\log n)$ \cite{Takahashi2009}. 
Their approach assumes unbounded fanout, which had not previously been mapped to a
nearest-neighbor circuit until our present work.

Studies of architectural constraints, namely restriction to a 2D planar layout, 
were experimentally motivated. For example, these layouts correspond
to early ion trap proposals \cite{Kielpinski2002}
and were later analyzed at the level of physical qubits and error correction in the context of Shor's algorithm \cite{Kubi09}.
Choi and Van Meter designed one of the first adders targeted to a 2D architecture 
and showed it runs in $\Theta(\sqrt{n})$-depth on \textsf{2D NTC} \cite{Choi2010}
using $O(n)$-qubits with dedicated, special-purpose areas of a physical
circuit layout.

%Once an adder implementation is chosen, it can be extended 
%To perform modular reduction, modular multiplication, 
%Modular exponentiation, and ultimately
%quantum period finding (QPF), the only quantum part of the factoring algorithm.
Modular exponentiation is a key component of quantum period-finding (QPF),
and its efficiency relies on that of its underlying adder implementation.
Since Shor's algorithm is a probabilistic algorithm, multiple rounds of
QPF are required to amplify success probability arbitrarily close to 1.
It suffices to determine the resources
required for a single round of QPF with a fixed, modest success probability
(in the current work, $3/4$).

The most common approach to QPF performs controlled
modular exponentiation followed by an inverse quantum Fourier transform
(QFT) \cite{Nielsen2000}. We will call this \emph{serial QPF}, which is
used by the following implementations.
%Quantum circuits proposed for factoring on a nearest-neighbor architecture have assumed a serial QPF circuit.

Beauregard \cite{Beauregard2002}
constructs a cubic-depth quantum period-finder using only $2n+3$ qubits on
\textsf{AC}.
It combines the ideas of Draper's transform adder \cite{Draper2000},
Vedral et al.'s modular arithmetic blocks \cite{Vedral1996}, and a
semi-classical QFT.
This approach was subsequently adapted to \textsf{1D NTC} by Fowler, Devitt,
and Hollenberg
\cite{Fowler2004} to achieve resource counts for an $O(n^3)$-depth
quantum period-finder. Kutin \cite{Kutin2006} later improved this using
an idea from Zalka for approximate multipliers to produce a QPF circuit on
\textsf{1D NTC}
in $O(n^2)$-depth. Thus, there is only a constant overhead from
Zalka's own factoring implementation on \textsf{AC}, which also has
quadratic depth \cite{Zalka1998}.
Takahashi and Kunihiro extended their earlier $O(n)$-depth adder to a factoring
circuit in $O(n^3)$-depth with linear width \cite{Takahashi2006}.
Van Meter and Itoh explore many different approaches for serial QPF,
with their lowest achievable depth being $O(n^2\log n)$ with
$O(n^2)$ on \textsf{NTC} \cite{VanMeter2005}. Cleve and Watrous
calculate a factoring circuit depth of $O(\log^3 n)$ and corresponding
circuit size of $O(n^3)$ on \textsf{AC},
assuming an adder which has depth $O(\log n)$ and
$O(n)$ size and width. We beat this depth and provide a concrete
architectural implementation using an adder with $O(1)$-depth and $O(n)$
size and width.

In the current work, we assume that errors do not affect the storage of qubits
during the circuit's operation. An alternate approach is taken by
Miquel \cite{Miquel1996} and Garcia-Mata \cite{GarciaMata2007}, who both
numerically simulate Shor's algorithm for factoring specific
numbers to determine its sensitivity to errors. Beckman et al. provide a
concrete factoring implementation in ion traps with $O(n^3)$ depth and size and
$O(n)$ width \cite{Beckman1996}.

In all the previous works,
it is assumed that qubits are expensive (width) and that
execution time (depth) is not the limiting constraint.
We make the alternative assumption that ancillae are cheap and that fast classical control
is available which allows access to all qubits in parallel.
Therefore, we optimize circuit depth at the expense of width.
We compare our work primarily to Kutin's method \cite{Kutin2006}.

These works also rely on serial QPF which in turn relies on an inverse QFT.
On an AC architecture, even when approximating the (inverse) QFT by truncating two-qubit
$\pi/2^k$ rotations beyond $k = O(\log n)$, 
the depth is $O(n \log n)$ to factor an $n$-bit number.
To be implemented fault-tolerantly on a quantum device, rotations in the QFT must then be compiled into a discrete gate basis.
This requires at least a $O(\log(1/\epsilon))$ overhead in depth to approximate a rotation with precision $\epsilon$ \cite{Harrow2002, Kitaev2002}.
We would like to avoid the use of a QFT due to its compilation overhead.

There is an alternative, parallel version of phase estimation 
\cite{Cleve2000,Kitaev2002}, which we call \emph{parallel QPF} (we refer the reader to Section 13 of \cite{Kitaev2002} for details), which decreases depth in exchange
for increased width and additional classical post-processing.
This eliminates the need to do an inverse QFT.
%We refer the reader to \cite{Kitaev2002} 
%and \cite{Pham2011b} 
%for details.
We develop a nearest-neighbor factoring circuit based on parallel QPF and our proposed 2D quantum arithmetic circuits.
We show that it is asymptotically more efficient than the serial QPF method. 
We compare the circuit resources required by our work with existing serial QPF implementations in Table
\ref{tab:fpl-results} of Section \ref{sec:fpl-results}.
However, a recent result by \cite{Jones2013} allows one to enact a
QFT using only Clifford gates and a Toffoli gate in $O(\log^2 n)$ expected depth.
This would allow us to
greatly improve the constants in our circuit resource upper bounds in Section \ref{sec:modexp} by combining a QFT with parallel multiplication similar to
the approach described in \cite{VanMeter2005,Cleve2000}.


We also note that recent results by Browne, Kashefi, and Perdrix (BKP) connect the power of
measurement-based quantum computing to the quantum circuit model augmented with
unbounded fanout \cite{Browne2009}. Their model, which we adapt and call
\textsf{CCNTC}, uses the classical controller mentioned in Section \ref{sec:intro-cdc}.
Using results by H{\o}yer and {\v S}palek \cite{Hoyer2002} that
unbounded quantum fanout would allow for a constant-depth factoring algorithm,
they conclude that a probabilistic polytime classical machine with access
to a constant-depth one-way quantum computer would also be able to factor
efficiently.

\section{The Constant-Depth Carry-Save Technique}
\label{sec:csa}

Our 2D factoring approach rests on the central technique of the constant-depth
carry-save adder (CSA) \cite{Gossett1998}, which converts the sum of three
numbers $a$, $b$, and $c$, to the sum of two numbers $u$ and $v$:
%\begin{equation}
$a+b+c = u+v$. The explanation of this technique and how it achieves constant depth requires the following definitions.
%\end{equation}

A \emph{conventional number} $x$ can be represented in $n$ bits as
%\begin{equation}
$x = \sum_{i=0}^{n-1} 2^i x_i$,
%\end{equation}
where $x_i \in \{0,1\}$ denotes the $i$-th bit of $x$, which we call
an $i$-bit and has significance $2^i$, and the $0$-th bit is the low-order bit.\footnote{It will be clear from the context whether we mean an
$i$-bit, which has significance $2^i$, or an $i$-bit number.}
Equivalently, $x$ can be represented as a (non-unique)
sum of two smaller, $(n-1)$-bit, conventional numbers, $u$ and $v$.
We say $(u+v)$ is a \emph{carry-save encoded}, or CSE, number.
The CSE representation of an $n$-bit conventional number
consists of $2n-2$ individual
bits where $v_0$ is always $0$ by convention.

Consider a CSA operating on three bits instead of three numbers; 
then a CSA converts the sum of three
$i$-bits into the sum of an $i$-bit (the \emph{sum} bit) and an $(i+1)$-bit
(the \emph{carry} bit):
%\begin{equation}
%\label{eqn:csa-3-2}
$a_i+b_i+c_i = u_i+v_{i+1}$.
%\end{equation}
By convention, the bit $u_i$ is the parity of the input bits
($u_i = a_i \oplus b_i \oplus c_i$) and
the bit $v_{i+1}$ is the majority of $\{a_i, b_i, c_i\}$.
Figure \ref{fig:csa-encoding} gives a concrete example, where
$(u+v)$ has $2n-2 = 8$ bits, not counting $v_0$.

%
It will also be useful to refer to a subset of the bits in a conventional
number using subscripts to indicate a range of indices:
\begin{equation}
x_{(j,k)} \equiv \sum_{i=j}^k 2^ix_i \qquad
x_{(i)} \equiv x_{(i,i)} = 2^ix_i.
\end{equation}
%
Using this notation, the following identity holds:
\begin{equation}
x_{(j,k)} = x_{(j,\ell)} + x_{(\ell+1,k)}, \qquad \text{ for all } j \le \ell < k.
\end{equation}
%
We can express the relationship between the bits of $x$ and $(u+v)$ as follows:
%
\begin{equation}
x = x_{(0,n-1)} \equiv u+v = u_{(0,n-2)} + v_{(1,n-1)}.
\end{equation}
%
Finally, we denote arithmetic modulo $m$ with square brackets.

\begin{equation}
x_{(j,k)} \bmod m = x_{(j,k)}[m]
\end{equation}

\begin{center}
\begin{figure*}[tb!]
\begin{displaymath}
x = 30 = u+v = 8 + 22 = \left\{
\begin{array}{ccccc}
    & u_3 & u_2 & u_1 & u_0 \\
v_4 & v_3 & v_2 & v_1 &    \\
\hline
x_4 & x_3 & x_2 & x_1 & x_0
\end{array}
\right\}
=
\left\{
\begin{array}{ccccc}
    & 1 & 0 & 0 & 0 \\
  1 & 0 & 1 & 1 &   \\
\hline
1 & 1 & 1 & 1 & 0
\end{array}
\right\}
\end{displaymath}
\caption{An example of carry-save encoding for the 5-bit conventional number 30.}
\label{fig:csa-encoding}
\end{figure*}
\end{center}
%

\begin{figure}[tb!]
\begin{center}
\begin{displaymath}
\centerline{
\Qcircuit @C=2em @R=2em {
\lstick{\ket{0}}   & \qw      & \qw & \qw                        & \qw & \qw                        & \targfix  & \qw & \qw_{\ket{a_i \wedge (b_i \oplus c_i)}} & \targfix  & \qw       & \qw       & \qw_{\ket{(b_i \wedge c_i) \oplus a_i \wedge (b_i \oplus c_i)}} & \qw & \qswap      & \qswap      & \qw & \rstick{\ket{u_i}} \\
\lstick{\ket{a_i}} & \qw      & \qw & \qw                        & \qw & \qw                        & \ctrl{-1} & \qw & \qw                                     & \qw       & \targfix  & \qw       & \qw_{\ket{a_i \oplus b_i \oplus c_i}}                           & \qw & \qw \qwx    & \qswap \qwx & \qw & \rstick{\ket{0}} \\
\lstick{\ket{b_i}} & \ctrl{1} & \qw & \targfix                   & \qw & \qw_{\ket{b_i \oplus c_i}} & \ctrl{-1} & \qw & \qw                                     & \qw       & \ctrl{-1} & \targfix  & \ctrl{1}                                                        & \qw & \qw \qwx    & \qw         & \qw & \rstick{\ket{b_i}} \\
\lstick{\ket{c_i}} & \ctrl{1} & \qw & \ctrl{-1}                  & \qw & \qw                        & \qw       & \qw & \qw                                     & \qw       & \qw       & \ctrl{-1} & \ctrl{1}                                                        & \qw & \qw \qwx    & \qw         & \qw & \rstick{\ket{c_i}} \\
\lstick{\ket{0}}   & \targfix & \qw & \qw_{\ket{b_i \wedge c_i}} & \qw & \qw                        & \qw       & \qw & \qw                                     & \ctrl{-4} & \qw       & \qw       & \targfix                                                        & \qw & \qswap \qwx & \qw         & \qw & \rstick{\ket{v_{i+1}}}}
}
\end{displaymath}
\caption{Carry-save adder circuit for a single bit position $i$: $a_i+b_i+c_i = u_i + v_{i+1}$.}
\label{fig:csa-circuit}
\end{center}\end{figure}

\begin{figure}
\begin{center}
\begin{displaymath}
\begin{tabular}{p{0.5in} m{0.1in} p{2in}}

\Qcircuit @C=1em @R=2.2em { 
	& \qw & \ctrl{1} & \qw & \qw \\
	& \qw & \ctrl{1} & \qw & \qw \\
	& \qw & \targfix & \qw & \qw
}

&
\qquad
=
\qquad
&

\Qcircuit @C=1em @R=.7em { 
	& \gate{T^{\dagger}} & \qw & \targfix  & \qw & \gate{T} & \qw & \targfix  & \qw & \gate{T^{\dagger}} & \qw & \targfix  & \qw & \gate{T}           & \qw & \targfix  & \qw & \qw \\ 
	& \gate{T^{\dagger}} & \qw & \qw       & \qw & \ctrl{1} & \qw & \ctrl{-1} & \qw & \ctrl{1}           & \qw & \qw       & \qw & \qw                & \qw & \ctrl{-1} & \qw & \qw \\
	& \gate{H}           & \qw & \ctrl{-2} & \qw & \targfix & \qw & \gate{T}  & \qw & \targfix           & \qw & \ctrl{-2} & \qw & \gate{T^{\dagger}} & \qw & \gate{H}  & \qw & \qw
}
\end{tabular}
\end{displaymath}
\caption{The depth-efficient Toffoli gate decomposition from \cite{Amy2012}.}
\label{fig:toffoli}
\end{center}
\end{figure}

% From Notebook #16, pp. 68-69
\begin{table}
\begin{displaymath}
\begin{tabular}{|c|c|c|c|}
\hline
\text{Circuit Name} & \text{Depth} & \text{Size} & \text{Width} \\
\hline
\text{Toffoli gate from \cite{Amy2012}} and Figure \ref{fig:toffoli} & 8 & 15 & 3 \\
\hline
\text{Single-bit } 3\text{-to-}2 \text{ adder from Figure \ref{fig:csa-circuit}} & 33 & 55 & 5 \\
%\hline
%$n$ \text{-bit modular } 3\text{-to-}2 \text{ adder from Figure \ref{fig:csa-add-4}} & 356 & 572n + 724 & 33n+47 \\
\hline
\end{tabular}
\end{displaymath}
\centerline{}
\caption{Circuit resources for Toffoli and single-bit addition.}
\label{tab:csa-tile-resources}
\end{table}

Figure \ref{fig:csa-circuit} gives a circuit description of carry-save addition (CSA) for a single bit position $i$.
The resources for this circuit are given in Table \ref{tab:csa-tile-resources}, using
the resources for the Toffoli gate (in the same table) based on
\cite{Amy2012}. We note here
that a more efficient decomposition for the Toffoli is possible using a
distillation approach described in \cite{Jones2013a}.

We must lay out the circuit to satisfy a 2D NTC model.
The Toffoli gate decomposition in \cite{Amy2012}, duplicated in
Figure \ref{fig:toffoli}, requires two control
qubits and a single target qubit to be
mutually connected to each other. Given this constraint, and the
interaction of the CNOTs in Figure \ref{fig:csa-circuit}, we can
rearrange these qubits on a 2D planar grid and obtain the layout shown
in Figure \ref{fig:csa-3-2}, which satisfies our 2D NTC model.
Qubits $\ket a_i$, $\ket b_i$, and $\ket c_i$ reside at the top of Figure~\ref{fig:csa-3-2}, while qubits $\ket{u_i}$ and $\ket{v_{i+1}}$ are initialized to $\ket 0$.
Upon completion of the circuit, qubit $\ket{a_i}$ is in state $\ket 0$, as seen from the output in Figure~\ref{fig:csa-circuit}. 
Note that this construction uses more gates and one more ancilla than the equivalent
quantum full adder circuit in Figure 5 of \cite{Gossett1998}. However this
is necessary in order to meet our architectural constraints and does not change the
asymptotic results.
Also in Figure \ref{fig:csa-3-2}
is a variation called a $2\rightarrow 2$ adder, which simply re-encodes two $i$-bits
into an $i$-bit and an $(i+1)$-bit. The $2 \rightarrow 2$ adder uses at most the resources
of a $3\rightarrow 2$ adder, so we can count it as such in our calculations.
It will be useful in the next section.

\begin{figure}[b!]
\begin{center}
\includegraphics[width=3in]{factor-polylog/figures/csa-32-22.pdf}
\end{center}
\caption{The carry-save adder (CSA), or $3\rightarrow 2$ adder, and carry-save $2\rightarrow 2$ adder.}
\label{fig:csa-3-2}
\end{figure}

At the level of numbers, the sum of three $n$-bit numbers can be converted into
the sum of two $n$-bit numbers by applying a \emph{CSA layer} of
$n$ parallel, single-bit
CSA circuits (Fig.~\ref{fig:csa-circuit}). Since each CSA operates in constant depth, the entire layer also
operates in constant depth, and we have achieved (non-modular) addition.
%
%An important consideration is the circuit width. The circuit above
%requires two additional qubits to contain the output
%out-of-place and produces two garbage qubits: the original inputs
%$b_i$ and $c_i$. 
Each single addition of three $n$-bit numbers requires $O(n)$ circuit width.

\input{factor-polylog/fpl-modadd}

\section{Quantum Modular Multiplication}
\label{sec:csa-mod-mult}

We can build upon our carry-save adder to implement quantum modular
multiplication in logarithmic depth. We start with a completely classical
problem to illustrate the principle of multiplication by repeated addition.
Then we consider modular multiplication of two quantum integers in a serial
and a parallel fashion in Section
\ref{subsec:csa-mod-mult-qq}. Both of these problems use as subroutines
\emph{partial product creation}, which we define and solve
 in Section \ref{subsec:ppc} and
 \emph{modular multiple addition}, which we define and solve
in Section \ref{subsec:mma}.

%%%%%%%%%%%%%%%%%%%%%%%%%%%%%%%%%%%%%%%%%%%%%%%%%%%%%%%%%%%%%%%%%%%%%%%%%%%%%%%
First we consider a completely classical problem:
given three $n$-bit classical numbers $a$, $b$, and $m$,
compute $c = ab \bmod m$, where $c$ is allowed to be in CSE.

We only have to add shifted
multiples of $a$ to itself, ``controlled'' on the bits of $b$. There are
$n$ shifted multiples of $a$, let's call them $z^{(i)}$, one for every bit of $b$:
%%\begin{equation}
$z^{(i)} = 2^i a b_i \bmod m$.
%%\end{equation}
We can parallelize the addition of $n$ numbers in a logarithmic depth
binary tree to get a total depth of $O(\log n)$.

%%%%%%%%%%%%%%%%%%%%%%%%%%%%%%%%%%%%%%%%%%%%%%%%%%%%%%%%%%%%%%%%%%%%%%%%%%%%%%
\subsection{Modular Multiplication of Two Quantum Integers}
\label{subsec:csa-mod-mult-qq}

We now consider the problem of multiplying a classical number controlled
on a quantum bit with a
\emph{quantum integer},
which is a
quantum superposition of classical numbers.\footnote{In this paper, an $n$-qubit 
quantum integer is a
general superposition of up to $2^n$ classical integers. As a special case,
a classical number controlled on a single qubit is a superposition of
$2$ classical integers. This should not be confused with quantum numbers
in physics.}

\begin{quote}
Given an $n$-qubit quantum integer $\ket{x}$, a control qubit $\ket{p}$,
and two $n$-bit classical numbers $a$
and $m$,
compute $\ket{c} = \ket{xa[m]}$, where $c$ is allowed to be in CSE.
\end{quote}

This problem occurs naturally in modular exponentiation (described in
the next section) and can be considered \emph{serial multiplication},
in that $t$ quantum integers are multiplied in series to a single
quantum register. This is used in serial QPF as mentioned in
Section \ref{sec:fpl-related}.


We first create $n$ quantum integers $\ket{z^{(i)}}$,
which are shifted multiples of the classical number $a$ controlled on the bits
of $x$:
\begin{equation}
\ket{z^{(i)}} \equiv \ket{2^i a[m] \cdot x_i }.
\end{equation}
These are typically called \emph{partial products} in a classical multiplier.
How do we create these numbers, and what is the depth of the procedure?
First, note that $\ket{2^i a[m]}$ is a classical number, so we can
precompute them classically and prepare them in parallel using single-qubit
operations
on $n$ registers, each consisting of $n$ ancillae qubits. Each $n$-qubit
register will hold a future $\ket{z^{(i)}}$ value.
We then fan out each of the
$n$ bits of $x$, $n$ times each, using an unbounded fanout operation so that
$n$ copies of each bit $\ket{x_i}$ are next to register $\ket{z^{(i)}}$.
This takes a total of $O(n^2)$ parallel CNOT operations.
We then entangle each $\ket{z^{(i)}}$ with the corresponding $x_i$.
%The schematic for this is shown in Figure \ref{fig:mod-mult-create}.
After this, we interleave these numbers into groups of three using
constant-depth teleportation. This reduces to the task of modular
multiple addition in order to add these numbers down to a single
(CSE) number modulo $m$, which is described in Section \ref{subsec:mma}.

%\begin{figure*}[htp!]
%\centerline{
%\includegraphics[width=4.5in]{./znumbers.pdf}
%}
%\caption{Creating $n=4$ shifted values $\{z^{(0)},z^{(1)},z^{(2)},z^{(3)}\}$
%for an input number $x$.}
%\label{fig:mod-mult-create}
%\end{figure*}

Finally, we tackle the most interesting problem:
\begin{quote}
Given two $n$-qubit quantum integers $\ket{x}$ and
$\ket{y}$ and an $n$-bit classical number
$m$,
compute $\ket{c} = \ket{xy \bmod m}$,
where $\ket{c}$ is allowed to be in CSE.
\end{quote}

This can be considered \emph{parallel multiplication} and is responsible
for our logarithmic speedup in modular exponentiation and parallel QPF.

Instead of creating $n$ quantum integers $\ket{z^{(i)}}$, we must create
up to $n^2$ numbers
$\ket{z^{i,j}}$ for all possible pairs of quantum bits $x_i$ and $y_j$,
$i,j \in \{0,\ldots,n-1\}$:
\begin{equation}
\ket{z^{i,j}} \equiv \ket{2^i2^j[m]\cdot x_i \cdot y_j}.
\label{eqn:zij-pp}
\end{equation}
We create these numbers using a similar procedure to the previous problem.
Adding $n^2$ quantum integers of $n$ qubits each takes depth
$O(\log(n^2))$, which is still $O(\log n)$.
Creating $n^2\times n$-bit quantum integers takes width $O(n^3)$.
Numerical constants are given for these resource estimates in
Section \ref{subsec:mod-mult-resources} for the entire modular multiplier.

Here is an outline of our modular multiplier construction, combining the
two halves of partial product creation (Section \ref{subsec:ppc}) and
modular multiple addition (Section \ref{subsec:mma}).
This is the first building block which requires multiple modules,
and hence the full generality of the \textsf{2D CCNTCM} model. 

\begin{enumerate}
\item Initially, the inputs consist of the CSE quantum integers $x$ and $y$,
each with $2n+3$ bits, sitting on adjacent edges of a square lattice that has
sides of length $3(2n+3)$ qubits.
\item For each of $\lceil \log_2 (2n+3) \rceil$ rounds:
\begin{enumerate}
\item Of the existing $\{x_i\}$ and $\{y_j\}$ bits, apply a CNOT to create an
entangled copy in an adjacent qubit.
\item Teleport this new copy halfway between its current location and the
new copy.
\item At every site where an $\ket{x_i}$ and an $\ket{y_j}$ meet,
apply a Toffoli gate to create $\ket{x_i \cdot y_j}$.
\item Teleport $\ket{x_i \cdot y_j}$ to the correct $z$-site module.
\end{enumerate}
\item Within each $z$-site module, fanout $\ket{x_i \cdot y_j}$ up to $n$
times, corresponding to each $1$ in the modular residue $2^i 2^j \bmod m$,
to create the $n$-qubit quantum integer $\ket{z^{(i,j)}}$.
\item For each triplet of $z$-site modules, teleport the quantum integers
$\ket{z^{(i,j)}}$ to a CSA tile module, interleaving the three numbers so that
bits of the same significance are adjacent. This concludes partial product
creation (Section \ref{subsec:ppc}).
\item Perform modular multiple addition (described in Section \ref{subsec:mma})
on $t'$ $n$-qubit quantum integers down to 2 $n$-qubit quantum integers (one CSE number).
\item Uncompute all the previous steps to restore ancillae to $\ket{0}$.
\end{enumerate}
%%%%%%%%%%%%%%%%%%%%%%%%%%%%%%%%%%%%%%%%%%%%%%%%%%%%%%%%%%%%%%%%%%%%%%%%%%%%%%%
\subsection{Partial Product Creation}
\label{subsec:ppc}

This subroutine describes the procedure of creating $t'=O(n^2)$ partial products of
the CSE quantum integers $x$ and $y$, each with $2n+3$ bits each. We will now
discuss only the case of parallel multiplication. Although we
will not provide an explicit circuit for this subroutine, we will outline
our particular construction and give a numerical upper bound on the
resources required.

First, we need to generate the product bits
$\ket{x_i\cdot y_j}$ for all possible $(2n+3)^2$ pairs of $\ket{x_i}$ and
$\ket{y_j}$.
A particular product bit $\ket{x_i \cdot y_j}$
controls a particular classical number, the
$n$-bit modular residue $2^i 2^j [m]$, to form the partial product
$\ket{z^{(i,j)}}$ defined in Equation \ref{eqn:zij-pp}.
However, some of these partial products
consist of only a single qubit, if $2^i 2^j < 2^n$, which is the minimum
value for an $n$-bit modulus $m$. There are at least $2n^2 - 2n + 1$
such single-bit partial products, which can be grouped into at most
$(2n+3)\times n$-bit numbers. Of the $(2n+3)^2$ possible partial products,
this leaves the number of remaining $n$-bit partial products as at most
$2n^2 + 14n +8$. Therefore we have a maximum number of $n$-bit
partial products, which we will simply refer to as $t'$ from now on.
% from N16, p. 65
\begin{equation}
t'=2n^2+16n+11
\label{eqn:tprime}
\end{equation}
%
The creation of the product bits $\ket{x_i \cdot y_j}$ occurs on a
square lattice of $(3(2n+3))^2$ qubits, with the numbers $\ket{x_i}$ and
$\ket{y_j}$ located on adjacent edges. The factor of $3$ in the size of the lattice
allows the $\ket{x_i}$ and $\ket{y_j}$ bits to teleport past each other.
The $\ket{x_i}$ bits are teleported along an axis that is perpendicular to
the teleportation axis for the $\ket{y_j}$ bits, and vice versa.
Product bit creation, and this square lattice, comprise a single module.
In $\lceil \log_2 (2n+3) \rceil$
rounds, these bits are copied via a CNOT and teleported to the middle of
a recursively halved interval of the grid. The copied bits $\ket{x_i}$ and
$\ket{y_j}$
first form $1$ line, then $3$ lines, then $7$ lines, and so forth,
intersecting at $1$ site, then $9$ sites, then $49$ sites, and so forth.
There are $\lceil \log_2 (2n+3) \rceil$ such rounds.

At each intersection, a Toffoli gate is used to create $\ket{x_i \cdot y_j}$
from the given $\ket{x_i}$ and $\ket{y_j}$. These product bits are then
teleported away from this qubit, out of this product bit module, to different
modules where the $\ket{z^{(i,j)}}$ numbers are later generated,
called $z$-sites. There are $t'$ $z$-site modules which each contain 
an $n$-qubit quantum integer. Any
round of partial product generation will produce at most as many product
bits $x_i \cdot y_j$ as in the last round, which is half the total number
of $(2n+3)^2$.
%These product bits are teleported out the two sides of the
%square lattice that are opposite the input numbers $x$ and $y$, which means the
%square lattice has dimension at most $((2(2n+3)-1)(n+2))^2 = O(n^4)$.

%The $z$-sites have total width of $3nt'$ qubits, so the maximum total
%teleportation length of all qubits is $(2n+3)^2$ multiplied by the maximum
%length of any single teleportation length,
%Our construction consists of the following steps:

We now present the resources for partial product creation, the first half of
a modular multiplier, including the reverse computation.

The circuit depth is $O(\log n)$:
%
\begin{equation}
D_{PPC} = 32\log_2 n + 150\text{.}
\end{equation}
%
The module depth is $O(1)$:
%
\begin{equation}
\overline{D}_{PPC} = 8\text{.}
\end{equation}
%
The circuit size is $O(n^3)$:
%
\begin{eqnarray}
S_{PPC} & = & (6n + 9)\log_2 n +\\
        &   & (26n^3 + 232n^2 + 224n + 159)\text{.}
\end{eqnarray}
%
The module size is $O(n^2)$:
%
\begin{eqnarray}
\overline{S}_{PPC} = 6n^2 + 26n + 19\text{.}
\end{eqnarray}
%
The circuit width is $O(n^3)$:
%
\begin{eqnarray}
W_{PPC} = 6n^3 + 48n^2 - 8n + 1\text{.}
\end{eqnarray}
%
The module width is $O(n^2)$:
% N18 p. 3
\begin{eqnarray}
\overline{W}_{PPC} = 2n^2 + 16n + 12\text{.}
\end{eqnarray}

%%%%%%%%%%%%%%%%%%%%%%%%%%%%%%%%%%%%%%%%%%%%%%%%%%%%%%%%%%%%%%%%%%%%%%%%%%%%%%%
\subsection{Modular Multiple Addition}
\label{subsec:mma}

As a subroutine to modular multiplication, we define the operation of
repeatedly adding multiple numbers down to a single CSE number, called
\emph{modular multiple addition}.

The modular multiple addition circuit generically adds down $t'\times n$-bit
conventional numbers to an $n$-bit CSE number:
%
\begin{equation}
z^{(1)} + z^{(2)} + \ldots z^{(t')} \equiv (u+v)[m].
\end{equation}
%
It does not matter how the
$t'$ numbers are generated, as long as they are divided into groups of three
and have their bits interleaved to be the inputs of a CSA tile.
From the previous section, serial multiplication results in
$t' \in O(n)$ and parallel multiplication results in $t' \in O(n^2)$. Each CSA tile
is contained in its own module. These modules are arranged in layers within
a logarithmic depth binary tree, where 
the first layer contains $\lceil t'/3 \rceil$ modules. A modular addition
occurs in all the modules of the first layer in parallel. The outputs from this
first layer are then teleported to be the inputs of the next layer of modules,
which have at most two-thirds as many modules. This continues until the
tree terminates in a single module, whose output is a CSE number $u+v$ which
represents the modular product of all the original $t'$ numbers. The resulting
height of the tree is $(\lceil \log_{3/2}(t'/3) \rceil + 1)$ modules.

As the parallel modular additions proceed by layers, all previous layers
must be maintained in a coherent state, since the modular addition leaves
garbage bits behind. Only at the end of modular multiple addition, after
the final answer $u+v$ is obtained, can all the previous layers be
uncomputed in reverse to free up their ancillae.\footnote{
Note this is a naive, worst-case strategy. In Chapter \ref{chap:coherence},
we will see that
that there may be some benefit to uncomputing intermediate results.
}

These steps are best illustrated with a concrete
example in Figure \ref{fig:mod-mult}. The module for each CSA tile is
represented by the symbol from Figure \ref{fig:csa-tile-symbol}.
The arrows indicate the
teleportation of output numbers from the source tile to be input numbers
into a destination tile.

\begin{figure*}[htb!]
\centerline{
\includegraphics[width=5.5in]{factor-polylog/figures/mod-mult-add.pdf}
}
\caption[Modular multiple addition of quantum integers on a CSA tile
architecture for $t'=18$]
{Modular multiple addition of quantum integers on a CSA tile
architecture for $t'=18$ in a logarithmic-depth tree with height
$(\lceil \log_{\frac{3}{2}}(t'/3) \rceil + 1) = 6$. Arrows represent teleportation
in between modules.}
\label{fig:mod-mult}
\end{figure*}
%

Now we can analyze the circuit resources for multiplying $n$-bit
quantum integers, which requires $(t'-2)$ modular additions, for $t'$ from
Equation \ref{eqn:tprime}.
The circuit width is the sum of the $O(n^3)$ ancillae
needed for partial product creation and the ancillae required for $O(n^2)$
modular additions. Each modular addition has width $O(n)$ and depth $O(1)$
from the previous
section. There are
$\lceil \log_{3/2}(n^2 / 3) \rceil +1 $ timesteps of modular addition. Therefore
the entire modular multiplier circuit has depth $O(\log n)$ and width $O(n^3)$.

\subsection{Modular Multiplier Resources}
\label{subsec:mod-mult-resources}.

The circuit depth of the entire modular multiplier is $O(\log n)$:
%
\begin{equation}
D_{MM} = 1383 \log_2 n + 3930\text{.}
\end{equation}
%
The module depth is $O(\log n)$:
\begin{equation}
\overline{D}_{MM} = 2\log_2 n + 11\text{.}
\end{equation}
%
The circuit size is $O(n^3)$:
%
\begin{eqnarray}
S_{MM} = & (6n + 9)\log_2 n +\\
        & (1152n^3 + 10780n^2 + 17628n + 7082)\text{.}
\end{eqnarray}
%
The module size is $O(n^3)$:
%
\begin{equation}
\overline{S}_{MM} = 15n^3 + 127n^2 + 178n + 50{.}
\end{equation}
%
The circuit width is $O(n^3)$:
%
\begin{equation}
W_{MM} = 66n^3 + 558n^2 + 870n + 290\text{.}
\end{equation}
%
The module width is $O(n^2)$:
%
\begin{equation}
\overline{W}_{MM} = 4n^2 + 28n + 15\text{.}
\end{equation}

\section{Quantum Modular Exponentiation}
\label{sec:modexp}

We now extend our arithmetic to modular exponentiation, which is repeated
modular multiplication controlled on qubits supplied by a phase estimation
procedure.
If we wish to multiply an $n$-qubit quantum input number $\ket{x}$ by
$t$ classical numbers $a^{(j)}$, we can multiply them in series.
% as shown in
%Figure \ref{fig:modexp-qc-series}.
This requires depth $O(t\log n)$ in modular multiplication operations.

%\begin{figure}[htp!]
%\begin{center}
%\includegraphics[width=5.5in]{figures/modexp-qc-series.pdf}
%\end{center}
%\caption{Multiplying a quantum number $\ket{x}$ by $t$ classical numbers
%$\{a^{0}, a^{1}, \ldots, a^{n-1}\}$ in series.}
%\label{fig:modexp-qc-series}
%\end{figure}

Now consider the same procedure, but this time each classical number $a^{(j)}$
is controlled on a quantum bit $p_j$. This is a special case of
multiplying by $t$ quantum integers in series, since a classical number
entangled with a quantum integer is also quantum.
%This is shown in
%Figure \ref{fig:modexp-qq-series}.
It takes the same depth $O(t\log n)$ as the previous case.
%
%\begin{figure}[htp!]
%\begin{center}
%\includegraphics[width=5.5in]{figures/modexp-qq-series.pdf}
%\end{center}
%\caption{Multiplying a quantum number $\ket{x}$ by $t$ quantum numbers
%$\{\ket{a^{0}p_0}, \ket{a^{1}p_1}, \ldots, \ket{a^{n-1}p_{n-1}}\}$ in series.}
%\label{fig:modexp-qq-series}
%\end{figure}

Finally, we consider multiplying $t$ quantum integers
$\{x^{(1)}, x^{(2)}, \ldots, x^{(t-1)}, x^{(t)}\}$ in a parallel,
logarithmic-depth binary tree.
This is shown in Figure \ref{fig:modexp-qq-parallel}, where arrows indicate multiplication.
The tree has depth $\log_2(t)$ in modular multiplier operations. Furthermore,
each
modular multiplier has depth $O(\log(n))$ and width $O(n^3)$ for $n$-qubit
numbers. Therefore, the overall depth of this parallel modular exponentiation
structure is $O(\log(t)\log(n))$ with width $O(tn^3)$.
In phase estimation for QPF, it is
sufficient to take $t = O(n)$ \cite{Nielsen2000,Kitaev2002}. Therefore our total depth is
$O(\log^2(n))$ and our total size and total width are $O(n^4)$, as desired. At this point, combined with the parallel phase
estimation procedure of \cite{Kitaev2002}, we have a complete factoring
implementation in our 2D nearest-neighbor architecture in polylogarithmic
depth.
%
\begin{figure*}[tb!]
\centerline{
\includegraphics[width=5.5in]{factor-polylog/figures/mod-exp-par.pdf}
}
\caption{Parallel modular exponentiation: multiplying $t$ quantum integers
%$\{\ket{x^{(0)}}, \ket{x^{(1)}}, \ldots, \ket{x^{(t-1)}}\}$ in parallel,
in a $O(\log{(t)}\log{(n)})$-depth binary tree. Arrows indicate modular
multiplication.}
\label{fig:modexp-qq-parallel}
\end{figure*}

We will now calculate numerical
constants to upper bound circuit resources.

According to the Kitaev-Shen-Vyalyi parallelized phase estimation procedure
\cite{Kitaev2002},
for a constant success probability of $3/4$,
it is sufficient to multiply together $t' = 2867n$ quantum integers,
controlled on the qubits $\ket{p_j}$, in parallel.

In Section \ref{subsec:qcla}, we describe the last step of modular
exponentiation in CSE. In Section \ref{subsec:modexp-resources}, we
state the final circuit resources for the entire modular exponentiation
circuit,
and therefore, our quantum period-finding procedure.

%%%%%%%%%%%%%%%%%%%%%%%%%%%%%%%%%%%%%%%%%%%%%%%%%%%%%%%%%%%%%%%%%%%%%%%%%%%%%%
\subsection{Converting Back to a Unique Conventional Number}
\label{subsec:qcla}

The final product of all $t$ quantum integers is in CSE which is not
unique. As stated in Gossett's original paper \cite{Gossett1998}, this
must be converted back to a conventional number using, for example, the
quantum carry-lookahead adder (QCLA) from \cite{Draper2004}. We can convert
this to a nearest-neighbor architecture by using the qubit reordering
construction of \cite{Rosenbaum2012}. We now compute the resources
needed for this last step.

To add two $(n+2)$-bit numbers in a QCLA, we have a circuit width of
$k = (4(n+2) - 2\log_2 n - 1)$. The depth is at most $4\log_2 n +2$ gates,
and some of them act on qubits that are not nearest-neighbors. Therefore,
we add in between each gate a reordering circuit that takes $k^2$
(reusable) ancillae
qubits and uses two rounds of constant-depth teleportation to rearrange
the qubits into a new order where all the gates are nearest-neighbor.
Adding in the teleportation circuit resources from Table \ref{tab:cd-resources},
we can calculate the following resources.

The circuit depth is $O(\log n)$:

\begin{equation}
56\log_2 n + 28\text{.}
\end{equation}

The circuit size is $O(n^2 \log n)$:

\begin{eqnarray}
96 \log_2^3 n & - & (384n + 624)\log_2^2 n \nonumber \\
              & + & (384n^2 + 1152n + 840) \log_2 n \nonumber \\
              & + & (192n^2 + 672n + 588)\text{.}
\end{eqnarray}

The circuit width is $O(n^2)$:

\begin{equation}
4 \log_2^2 n - (16n + 30)\log_2 n + 16n^2 + 60n + 56\text{.}
\end{equation}


%%%%%%%%%%%%%%%%%%%%%%%%%%%%%%%%%%%%%%%%%%%%%%%%%%%%%%%%%%%%%%%%%%%%%%%%%%%%%%
\subsection{Circuit Resources for Modular Exponentiator}
\label{subsec:modexp-resources}

This leads to the following circuit resource upper bounds for a modular exponentiator. Therefore, these are the total resources for running a
single round of parallel QPF as part of Shor's factoring algorithm.

The circuit depth is $O(\log^2 n)$:

% From Notebook #16 p. 225
\begin{equation}
D_{ME} = 1383\log_2^2(n) + 21253\log_2(n) + 49095\text{.}
\end{equation}

The module depth is $O(\log n)$:

\begin{equation}
\overline{D}_{ME} = 3\log_2 n + 24
\end{equation}

The circuit size is $O(n^4)$:

\begin{eqnarray}
S_{ME} & = & 96 \log_2^3 n + \nonumber \\
       & - & (384n + 624)\log_2^2 n \nonumber \\
       & + & (384n^2 + 1152n + 840) \log_2 n \nonumber \\
       &   & 3302324 n^4 + 30900797 n^3 + 50521837 n^2  + 20284306 n + 
  -6494\text{.}
\end{eqnarray}

The module size is $O(n^2)$:

\begin{equation}
\overline{S}_{ME} = 5749n^2 + 8725n +175\text{.}
\end{equation}

The circuit width is $O(n^4)$:

\begin{equation}
W_{ME} = 94598n^4 + 799749 n^3 + 1246692 n^2 + 415222 n - 145\text{.}
\end{equation}

% From N17 p. 192
The module width is $O(n)$:

\begin{equation}
\overline{W}_{ME} = 11468 n^3 + 80276 n^2 + 43005 n\text{.}
\end{equation}

\section{Asymptotic Results}
\label{sec:fpl-results}

The asymptotic resources required for our approach,
as well as the resources for other nearest-neighbor approaches,
are listed in Table \ref{tab:fpl-results},
where we assume a fixed constant error
probability for each round of QPF. Not all resources are
provided directly by the referenced source.

Resources in square brackets
are inferred using Equation \ref{eqn:depth-width}.
These upper bounds are correct,
but may not be tight with the upper bounds
calculated by their respective authors.
In particular, a more detailed analysis
could give a better upper bound for circuit size than the
depth-width product.
These related works all use architectural models defined in
Sections \ref{sec:intro-arch} and \ref{sec:intro-modules}.
 Also note that the
work by Beckman et al. \cite{Beckman1996} is unique in that it uses
efficient multi-qubit gates inherent to linear ion trap technology which
at first seem to
be more powerful than \textsf{1D NTC}. However, use of these gates does not
result in an asymptotic improvement over \textsf{1D NTC}.

%, say $\epsilon=1/4$.
% and $\delta' = 1/2$ for KSV-QPF.
%Note that the
%number of measurements are included for completeness.
%, since these are
%not counted as gates in our model but may be comparable in terms of
%execution time.
%Some table cells are blank if the entries are not relevant to the current comparison, or if the entires were not %calculated in the prior work.
We achieve an exponential
improvement in hybrid nearest-neighbor circuit depth (from quadratic to polylogarithmic)
with our approach at the cost of a polynomial increase in
circuit size and width.
Similar depth improvements at the cost of width increases can be achieved
in other factoring implementations simply by rearranging their
existing modular multipliers in a parallel, logarithmic-depth binary tree.
Our approach is the first implementation for factoring on any
\textsc{2D} model. The resulting circuit size $S$ (which measures
intra-module computation and communication) and module size $\overline{S}$
(which measures inter-module communication)
both have the same asymptotic order of growth: $O(n^4)$. This satisfies
our heuristic from Section \ref{subsec:module-compare} to choose
a module size such that roughly equal resources are devoted to
computation and communication.

One objection to our results is that it is not a purely nearest-neighbor approach,
and therefore not a fair comparison to other nearest-neighbor implementations.
To address this issue, we have presented a variation of our hybrid architecture
that runs on the non-hybrid model \textsf{2D CCNTC}. That is, all long-range
interactions are directly simulated using constant-depth communication on a
contiguous \textsc{2D} lattice, without accounting for computation and
communication operations separately. The resulting depth is asymptotically
the same as before ($O(\log^2 n)$) but the size and width have now
increased to $O(n^6)$ versus $O(n^4)$. While is is only a polynomial increase
asymptotically, the actual numerical bounds may make it vastly intractable
to implement. We present the formula for \textsf{2D CCNTC} circuit size below
for comparison to Equation \ref{eqn:sme}.
%
\begin{equation}
\tilde{S}_{ME} = 245955 n^6 + 3.3n^5 + 1.6n^4 + O(n^3)
\end{equation}
%
\begin{table}[htb!]
\begin{center}
\begin{tabular}{|c|c|c|c|c|}
\hline
Implementation             & Architecture      & $D$   & $S$  & $W$     \\
\hline
Vedral, et al. \cite{Vedral1996}   & \textsf{AC}      & $[O(n^3)]$ & $O(n^3)$    & $O(n)$ \\
Gossett \cite{Gossett1998}                   & \textsf{AC}       & $O(n \log n)$  & $[O(n^3\log n)]$  & $O(n^2)$  \\
Beauregard \cite{Beauregard2002}                & \textsf{AC}       & $O(n^3)$      & $O(n^3 \log n)$ & $O(n)$ \\
Zalka \cite{Zalka1998}                     & \textsf{AC}       & $O(n^2)$      & $[O(n^3)]$ & $O(n)$     \\
Takahashi \& Kunihiro \cite{Takahashi2006}     & \textsf{AC}       & $O(n^3)$      & $O(n^3\log n)$ & $O(n)$ \\
Cleve \& Watrous \cite{Cleve2000}           & \textsf{AC}       & $O(\log^3 n)$ & $O(n^3)$ & $[O(n^3 / \log^3n)]$ \\
\hline
Beckman et al. \cite{Beckman1996} & \textsf{ION TRAP}   & $O(n^3)$ & $O(n^3)$ & $O(n)$\\
\hline
Fowler, et al. \cite{Fowler2004} & \textsf{1D NTC}   & $O(n^3)$ & $O(n^4)$ & $O(n)$\\
Van Meter \& Itoh \cite{VanMeter2006} & \textsf{1D NTC}   & $O(n^2 \log n)$ & $[O(n^4\log n)]$ & $O(n^2)$\\
Kutin \cite{Kutin2006}                     & \textsf{1D NTC}   & $O(n^2)$ & $O(n^3)$ & $O(n)$\\
\hline
                           & \textsf{2D CCNTC}    & $O(\log^2{n})$ & $O(n^6)$ & $O(n^6)$   \\
Current Work               & \textsf{2D CCNTCM}   & $O(\log^2{n})$ & $O(n^4)$ & $O(n^4)$   \\
                           &                      & $\overline{D}$ & $\overline{S}$ & $\overline{W}$ \\
                           &                      & $O(\log^2{n})$ & $O(n^4)$ & $O(n^3)$ \\
\hline
\end{tabular}
\end{center}
\caption{Asymptotic circuit resource usage for quantum factoring of an $n$-bit number.}
\label{tab:fpl-results}
\end{table}

\section{Conclusion}
\label{sec:fpl-conclude}

Insert acknowledgements for 2D factoring work, mention paper in which these results appear. 


% ========== Chapter 2
\chapter{Hybrid Nearest-Neighbor Factoring in Sub-logarithmic Depth}
\label{chap:factor-sublog}

It is now natural to ask: given such dramatic improvement in circuit depth
for hybrid nearest-neighbor factoring
from quadratic \cite{Kutin2006} to poly-logarithmic in the last chapter, can we
decrease depth further? Surprisingly, the answer is yes. In this
chapter, we now decrease the depth below poly-logarithmic, in fact,
to be $O((\log \log n)^2)$. To do this, we take inspiration from
two main lines of related work. First, it is known how to compute
many useful arithmetic functions, including those used in modular
exponentiation, in constant depth by introducing a threshold gate.
Second, a similar construction (on \textsf{AC}) gives us a
quantum OR gate. Using these results,
we construct a \emph{quantum majority gate} on \textsf{2D CCNTCM} to
achieve quantum modular exponentiation, and therefore factoring, in the above
depth.

In Section \ref{sec:fsl-circuits} we provide background for classical circuit complexity,
including common universal gate sets, how they allow us to define circuit
complexity classes, and the relationships between those classes. We also
discuss the powerful threshold gate and its variations. Finally, we provide
quantum analogues for all these notions.
The quantum threshold
gate can be decomposed into simpler operations, namely CNOT and
arbitrary single-qubit gates, while maintaining constant-depth.

However,
even this simple gate set must be compiled down to \textsf{2D CCNTCM}.
This accounts for the discrepancy between the
non-constant quantum depth upper bound and the constant classical
depth lower bound mentioned above. Our solution is to augment our factoring circuit
with quantum compiler modules using the Kitaev-Shen-Vyalyi method \cite{Kitaev2002}
and the programmable ancillae rotation method of Jones et al. \cite{Jones2012}.
We discuss this special case of quantum
compiling overhead and its effect on our factoring implementation in
Section \ref{sec:fsl-qcompile}.

We then discuss our main result in Section \ref{sec:fsl-majority},
a \textsf{2D CCNTCM} implementation
of a quantum majority gate with fanin $n$ having circuit
depth $O((\log \log n)^2)$ and
circuit size and width $O(n^2\log^2 n)$. Using this quantum
majority gate and the constant-depth majority circuits of
Reif-Tate \cite{Reif1992} and Yeh-Varvarigos \cite{Yeh1996},
we achieve a complete circuit for quantum modular exponentiation.
We conclude in Section \ref{sec:fsl-conclude} with open problems
related to factoring architectures.

\section{Circuit Complexity Classes}
\label{sec:fsl-circuits}

A circuit can be thought of as a directed acyclic graph in which the nodes are
logical gates drawn from a certain (universal) set and the edges 
represent
the connection of the output of one gate to the input of another
gate. This graph is not equivalent to, but is related to, the graph of an architecture
as described in Chapter \ref{chap:factor-polylog}. The edges of a circuit graph can
be mapped to the nodes (qubits) of an architectural graph; the nodes of the
circuit graph, for 2D CCNTC, can be mapped to single nodes or connected pairs of nodes
in the architectural graph.

\begin{figure}
\caption{An example of a classical circuit implementing a Boolean function.}
\end{figure}

We can also define special nodes which are not gates, but rather
are placeholder ``sources'' which provide the inputs to the circuit and 
``sinks'' which provide the outputs to the circuit. The in-degree of a 
node is also known as its \emph{fanin} and the out-degree of a node is
also known as its \emph{fanout}.
Classical circuits implement Boolean functions, which take in $n$ input
bits to one output bit.

\begin{equation}
f:{0,1}^n \rightarrow {0,1}
\end{equation}

We denote a gate by its fanin as a subscript and an optional
second parameter as a superscript.

\begin{equation}
\text{GATE}_n^k
\end{equation}

The fanin $n$ will be neglected where it is obvious,
such as for the following well-known gate set which is universal
for classical circuits:

\begin{itemize}
\item $\text{NOT} = \text{NOT}_1$
\item $\text{AND} = \text{AND}_2$
\item $\text{OR} = \text{OR}_2$
\end{itemize}

\subsection{Gates Based on the Hamming Weight}

While the gates above have a simple truth table, it is
useful to describe a wider class of gates with general
fanin $n$ in a compact way as a function on the input
Hamming weight. These include the gates named below, which
are $1$ when the condition to their right is met, and $0$
otherwise.

\begin{itemize}
\item The logical OR gate $\text{OR}_n: |x| > 0$
\item The logical AND gate $\text{AND}_n: |x| = n$
\item The modular gate $\text{MOD}[q]_n: |x| \bmod q = 0$
\item The parity gate $\text{PA}_n: |x| \bmod 2 = 0$
\item The exact gate $\text{EX}^t_n: |x| = t$
\item The threshold gate $\text{TH}^t_n: |x| \ge t$
\item The majority $\text{MAJ}_n: |x| \ge n/2$
\end{itemize}

Many of these gates are also related in interesting ways,
as noted in \cite{Takahashi2011}.
$\text{OR}_n$ and $\text{EX}^0_n$ are
negations of each other. $\text{PA}_n$ is equivalent to
$\text{MOD}[2]_n$. $\text{TH}^t_n$ can be implemented with
$n-t$ parallel copies of $\text{EX}^k_n$ for $t \ge k \ge n$
followed by $\text{PA}_{n-t}$ on the outputs.

\subsection{Classical Circuit Complexity Classes}

We have introduced a menagerie of interesting gates,
all of which are universal with the $NOT$ gate. However depending
on what gates are in our universal set, our circuits may have
different depth and size. Therefore, we can define complexity classes
of circuits based on the allowed gate set and study more general
relationships among these classes beyond the special cases mentioned
as the end of the last section. This will let us formalize the notion of
which gates are more powerful than other gates and which are equivalent.
 
In classical circuits, we take unbounded fanout
for granted (any node can have arbitrary out-degree). These are common
in the literature of classical circuits. We will list them in order
of the size of their universal set, where each subsequent class adds
more gates.

\begin{definition}
\item[\textsf{NC}]
circuits consisting of $\text{NOT}_1$ and $\text{AND}_2$ and
$\text{OR}_2$ gates.
\item[\textsf{AC}]
NC circuits augmented with $\text{AND}_n$ and $\text{OR}_n$ gates,
for $n \ge 2$.
\item[\textsf{TC}]
AC circuits augmented with $\text{TH}_n^t$ gates, for $n \ge 2$ and
$0 \le t \le n$.
\end{definition}

A $\text{TH}^t_n$ gate is a threshold gate that is $1$ if the number
of input bits in greater than or equal to the threshold $t$ and $0$
otherwise. This is a special case of the linear threshold element (LTE)
in Section \ref{sec:threshold}, but we can simulate any
LTE with a circuit of $\text{TH}^t_n$ gates in constant depth
and polynomial size (citation needed).

\begin{equation}
% TODO insert block piecewise bracket
\end{equation}

We are often interested in the computing power of the above
circuit classes restricted in some way, usually shallow depth.
We denote by a superscript $k$ a complexity class of
functions implementable by circuits of depth $k$.

For these classical circuit classes, it is known that containment
is proper.

\begin{equation}
\textsf{NC}^0 \subsetneq \textsf{AC}^0 \subsetneq \textsf{TC}^0
\end{equation}

\subsection{Quantum Gates}

Unbounded fanout is taken for granted in classical circuit classes
because it is physically realistic to implement using electrical
devices. However, for quantum circuits, we must make use of the
unbounded quantum fanout, $\text{FANOUT}_n$, to copy one output
qubit of each gate.
We can define quantum analogues of the above circuit complexity 
classes by defining quantum analogues for each of the gates which
is reversible.
$NOT_1$ is already reversible, and we can use it as is (the Pauli $X$ gate).
To replace $AND_2$ we can use the reversible $3$-qubit Toffoli gate,
the so-called controlled-controlled-NOT.
To replace $OR_2$ we can use the circuit given in
Figure \ref{fig:or2}

\begin{figure}
% TODO 
\caption{A quantum $\text{OR}_2$ gate (citation needed).}
\label{fig:or2}
\end{figure}

In fact this is special case of a much more powerful construction
that will let us define a quantum $OR_n$ gate on unbounded inputs.
We will use this construction in Section \ref{sec:fsl-majority}
to build a quantum majority gate.

\begin{definition}
\item[$\textsf{QNC}^0_f$]
constant-depth quantum circuits consisting of CNOT and single-qubit gates.
\item[$\textsf{QAC}^0_f$]
constant-depth $\textsf{QNC}^0_f$ circuits augmented with quantum $\text{AND}_n$ and $\text{OR}_n$ gates,
for $n \ge 2$.
\item[$\textsf{QTC}^0_f$]
constant-depth $\textsf{QAC}^0_f$ circuits augmented with $\text{TH}_n^t$ gates, for $n \ge 2$ and
$0 \le t \le n$.
\end{definition}

\subsection{}

\input{factor-sublog/fsl-qcompile}

\section{Quantum Majority Circuits for Modular Exponentiation}
\label{sec:fsl-majority}

A quantum majority circuit is made from quantum $\text{MAJ}_n$ gates.
As mentioned before, depth-$k$
majority circuits are equivalent in power to depth-$k$ LTE circuits
with polynomially-bounded weights: $\textsf{MAJ}_k = \hat{\textsf{LT}}_k$
\cite{Alon1994,Goldmann1994}.

We are also interest in majority gates of polynomial fanin.
In a majority circuit, the fanin of any one majority gate is
bounded above by the circuit size, since in the worst case, one
gate receives an output from every other gate as input. As long
as we restrict ourselves to polynomial-size circuits, we are
assured that our circuit fanin is also polynomial. This is the
primary consideration in achieving sublogarithmic depth for
quantum compiling single-qubit rotations, and therefore for
factoring.

We now contribute a quantum majority circuit for 
modular exponentiation
of an $n$-bit modulus on \textsf{2D CCNTCM}. In this section, we show that each majority circuit
can be implemented in a single module with $O(n)$ qubits. Any
reordering needed between the output qubits of one timestep in a majority circuit
and the input qubits of the next timestep is handled by teleportation
in between modules, as allowed on \textsf{2D CCNTCM}. Therefore, it suffices for
us to do the following:

\begin{enumerate}
\item Show that a (classical) majority circuit exists for (classical) modular exponentiation.
We can translate this to a \textsf{2D CCNTCM} architecture where each $\text{MAJ}_n$ gate
is translated to $O(1)$ modules of $\Omega(n)$ width.
\item Show a \textsf{2D CCNTCM} quantum architecture for a single $\text{MAJ}_n$ gate in
$O(1)$ modules of $\Omega(n)$ width,
assuming the incoming teleportation of qubits of this form $\normtwo(\ket{0} + e^{i\phi}\ket{1})$.
These qubits are used to perform the corresponding single-qubit rotations using the
procedure of Jones et al. \cite{Jones2012}.
\item Include the circuit resources for (KSV) quantum compiler modules that can produce
the qubits in the previous step.
\end{enumerate}

We do the first item by relying on the following two theorems from Yeh-Varvarigos,
which we re-state below without proof. Both of these theorems
allow an extra parameter $\epsilon \in (0,1]$ which determines the
tradeoff between circuit depth and circuit size. They apply to \emph{classical majority circuits}.
Therefore, we must scale all their circuit resources by those for a quantum majority
gate, calculated in Section \ref{subsec:maj-gate}.

\begin{theorem}{\textbf{(Yeh-Varvarigos) Multiple product in constant depth and polynomial size: \cite{Yeh1996}.}}
The $n^2$-bit product of $n\times n$-bit numbers can be computed by a
majority circuit of depth $O(\frac{1}{\epsilon})$,
size and width $O(\frac{1}{\epsilon}n^{3+2\epsilon})$, and
fanin $O(n)$.
\label{thm:mult-prod}
\end{theorem}

\begin{theorem}{\textbf{(Yeh-Varvarigos) Modular reduction in constant depth and polynomial size \cite{Yeh1996}.}}
The $n$-bit binary representation of the modular residue $x \bmod m$, where
$x$ is an $n^2$-bit number and $m$ is an $n$-bit modulus, can be computed
by a majority circuit of depth $O(\frac{1}{\epsilon})$,
size and width $O(\frac{1}{\epsilon}n^{1 + 2\epsilon})$, and
fanin $O(n^2)$.
\label{thm:mod-reduce}
\end{theorem}

Both of these theorems rely on a Chinese Remainder representation for
an $n$-bit number. A conventional binary representation of a number
treats bits as coefficients for weights of $O(2^n)$, which requires exponential
weight to represent in an LTE. In a Chinese Remainder representation,
a number is given a ``mixed-radix'' representation, where each coefficient
is associated with a modular residue for a prime with $O(n)$ bits. In this
way, the weights needed to represent each coefficient are bounded
polynomially and not exponentially.
A more detailed reference of this technique can be found in \cite{Reif1992}.

We delay discussion of quantum compiler considerations until Section
\ref{sec:fsl-qcompile}.

We now accomplish the second item (a concrete architecture for a quantum majority gate)
in the remainder of this section by a sequence of building blocks, each on
\textsf{2D CCNTC} with constant depths and polynomially-bounded sizes and widths.

\begin{itemize}

\item a $\text{BIAS}^{t,\phi}_n$ gate which distinguishes between $|x| = t$ and $|x| = (\lceil n/2 \rceil - t) \bmod n$ with a measurement bias of $e^{i\phi}$.
This is described in Section \ref{subsec:mu-gate}.
\item an $\text{EX}^t_{n\rightarrow \log_n}$ gate which reduces from $\text{EX}^t_n$ (on $n$ qubits)
to $\text{EX}^t_{\log n}$ (on $\lceil \log_2(n+1) \rceil$ qubits).
This is described in Section \ref{subsec:ex-reduce}.
\item an $\text{EX}^t_{\log_n}$ gate which acts on $O(\log n)$ qubits.
This is described in Section \ref{subsec:or-log}.
\end{itemize}

%%%%%%%%%%%
\subsection{BIAS Gate}
\label{subsec:mu-gate}

We define the BIAS gate using the results in \cite{Hoyer2002}, where it is called a $\mu^{|x|-t}_{\phi}$ gate.
We can also think of it as rotating the output qubit $\ket{+}$ by Hamming weight with a threshold $t$ subtracted.
It operates as follows:

\begin{equation}
\text{BIAS}^{t,\phi}_n\ket{x}\ket{+} \rightarrow \ket{x}\ket{\mu^{|x|-t}_{\phi}}
\end{equation}

The output qubit begins in the state $\ket{+}$, which has equal probability of
being measured in the $\ket{0}$ state or the $\ket{1}$ state. It ends in
the following state, which introduces a bias between measuring $\ket{0}$
or $\ket{1}$ proportional to the difference $(|x|-t)$.

\begin{equation}
\ket{\mu^{|x|-t}_{\phi}} = \frac{1 + e^{i\phi(|x|-t)}}{2}\ket{0} + \frac{1 - e^{i\phi(|x|-t)}}{2}\ket{1}
\end{equation}

When $\phi = 2\pi / n$, then the BIAS gate allows us to distinguish
between the case of $|x| = t$ or $|x| = (\lceil n/2 \rceil - t) \bmod n$. As we will
see in the next section, rotations by multiples of $2\pi / n$ will allow us to reduce the
size of an $\text{EX}^t_n$ gate.

\begin{theorem}{\textbf{Constant-depth BIAS gate.}}
The $BIAS^{t,\phi}_n$ gate can be implemented on \textsf{2D CCNTCM} with
a depth of $O(1)$, a size and width of $O(n)$, and
expected $O(n)$ teleported PAR qubits of the form $(\ket{0} + e^{i\phi}\ket{1})$
and $(\ket{0} + e^{-i\phi\cdot t}\ket{1})$.
\label{thm:bias}
\end{theorem}

\begin{proof}
We can lay out the circuit from Figure 1 in \cite{Takahashi2011} on a \textsf{2D CCNTC} lattice as
shown in Figure \ref{fig:mu-circuit}. The size includes a Hadamard to transform
the output qubit into $\ket{+}$, a fanout of this qubit over $n+1$ qubits,
$O(n)$ gates to apply the rotations $R_Z(\phi)$ and $O(1)$ gates to apply
the rotation $R_Z(-\phi\cdot t)$, and a corresponding unfanout, which is
$O(n)$ total. This occurs on $O(n)$ qubits, and can be arranged to take
$O(1)$ depth. This circuit is contained within a single module.
\end{proof}

% TODO: Fill this in
\begin{figure}
\caption{The layout for a BIAS gate on 2D CCNTC.}
\label{fig:mu-circuit}
\end{figure}

%%%%%%%%%%%
\subsection{EX Logarithmic Reduction}
\label{subsec:ex-reduce}

Now we wish to show how to reduce $\text{EX}^t_n$ gate (on $n$ qubits) to an $\text{EX}^t_{\log n}$
gate (on $O(\log n)$ qubits). That is, we wish to implement the following gate
$\text{EX}^t_{n\rightarrow \log_2 n}$ on an $n$-qubit input register $\ket{x}$
to produce an $m$-qubit output register $\ket{y}$, where
$m = \lceil \log_2 n + 1 \rceil$. Running $\text{EX}^t_n$
on $\ket{x}$ should produce the same output qubit $\ket{z}$ as running
$\text{EX}^t_{m}$ on $\ket{y}$. This is formally defined below.

\begin{eqnarray}
\text{EX}^t_{n\rightarrow \log_2 n} \ket{x}\ket{0^m} & \rightarrow &\ket{x}\ket{y} \\
\text{EX}^t_n \ket{x}\ket{0} & \rightarrow & \ket{x}\ket{z} \\
\text{EX}^t_m \ket{y}\ket{0} & \rightarrow & \ket{y}\ket{z}
\end{eqnarray}

\begin{theorem}{\textbf{Constant-depth EX reduction gate.}}
The $EX^t_{n\rightarrow \log_2 n}$ gate can be implemented on \textsf{2D CCNTCM} with
a depth of $O(1)$, a size and width of $O(n\log n)$, and expected
$O(n\log n)$ teleported PAR qubits of the form $(\ket{0} + e^{i\phi_k}\ket{1})$
and $(\ket{0} + e^{-i\phi_k\cdot t}\ket{1})$
\label{thm:ex-reduce}
\end{theorem}

\begin{proof}
We map the construction from Theorem 19 in \cite{Hoyer2002} onto
\textsf{2D CCNTCM}.
This step involves $m = \lceil \log_2 (n+1) \rceil$ parallel $\text{BIAS}^{t,\phi_k}_n$ gates from the last section for
$1 \le k \le m$, where $\phi_k = \frac{2\pi}{m}k$, each in their own modules.
This maintains constant circuit depth while only increasing circuit size by
a $O(\log n)$ factor.
\end{proof}

%%%%%%%%%%%
\subsection{OR on Logarithmic Qubits}
\label{subsec:or-log}

We now map an exact OR gate from Takahashi-Tani \cite{Takahashi2011} to \textsf{2D CCNTCM}.
Unlike the approximate OR gate (and its variation, an approximate EXACT gate) from Hoyer-Spalek \cite{Hoyer2002}, it is both
complete and sound with probability $1$ and it completes in
constant depth. However, it has size exponential in its input size $n'$: $O(n'2^{n'})$.
Therefore, the previous logarithmic reduction in Section \ref{subsec:ex-reduce}
is necessary to reduce $O(n)$ qubits, the fanin of our majority circuit,
to $n' = O(\log n)$, to give us a polynomial circuit size of $O(n\log n)$.
We will denote this gate as $OR^t_{\log n} = OR^t_{n'}$ to emphasize that it
can only be efficiently used when the number of qubits have been reduced to
be logarithmic in the input size of the overall problem (in this case, factoring).
The circuit depth will be constant no matter what, but circuit size is
only polynomially-bounded if the previous condition is met.

\begin{theorem}{\textbf{Constant-depth exact OR gate on logarithmic qubits.}}
The $OR^t_{\log n}$ gate can be implemented on \textsf{2D CCNTCM} with a
circuit depth of $O(1)$, a circuit size and width of $O(n\log n)$, and expected
$O(n \log n)$ teleported PAR qubits of the form $(\ket{0} + e^{i \frac{2\pi}{O(n)}}\ket{1})$.
This takes module depth of $O(1)$, module size of $O(n\log n)$, and
module width $O(n\log n)$.
\label{thm:or-log}
\end{theorem}

\begin{proof}
Following the notation of Lemma 2 from \cite{Takahashi2011}, we have two
modules:

\begin{enumerate}
\item One module contains the input qubits $x_i$ and the $R_j$ and $S$
registers. We will call this the $RS$ module.
\item One module contains the $T$ register, which we call the $T$ module.
\end{enumerate}

First, we consider the $RS$ module.
This has a total of $n'2^{n'} = O(n\log n)$ qubits each.
We use a sorting network from Rosenbaum \cite{Rosenbaum2012} to
reorder qubits and allow for nearest-neighbor interactions.
This requires size and width $O(n^2\log^2 n)$ and depth $O(1)$
using constant-depth teleportation. We also need to perform $O(n\log n)$
Hadamards.
In addition, we need to
perform the following fanouts (and corresponding unfanout).
This is illustrated in Figure 3 from
\cite{Takahashi2011}, which we repeat in Figure \ref{fig:exact-or}.

\begin{itemize}
\item
$n' = O(\log n)$ constant-depth fanouts of $2^{n'} = O(n)$
target qubits from each $x_i$ to each $R_j$ register.
\item
$2^{n'} = O(n)$ constant-depth fanouts of $2^{n'} = O(n)$
target qubits from each $S$ qubit to the $R_j$ registers.
\end{itemize}

The two kinds of fanouts above take, in total, $O(1)$ depth, $O(n^2)$ size,
and $O(n^2)$ width. These are subsumed by the resources needed
for reordering qubits above. The operations in the $RS$ module
do not require any PAR ancillae qubits to be teleported in.

Second, we consider the $T$ module, into which the $O(\log n)$ input qubits
$x_i$ and the $O(n)$ qubits from $S$ are teleported from the $RS$ module.
There is a single
Hadamard gate on the output qubit and a fanout
(and corresponding unfanout) of $O(2^{n'}) = O(n)$ qubits,
which takes a depth of $O(1)$ and a size and width of $O(n)$.
Finally, we are left with $O(n)$ controlled-$R_z$ rotations of
angle $\pi / 2^{n-1}$ which can be done in expected depth of $O(1)$,
 size and width of $O(n)$, and $O(n)$ teleported PAR qubits
 of the form $(\ket{0} + e^{i\phi_k}\ket{1})$.

 This gives us the required resources stated in the theorem.
\end{proof}

\begin{figure}[hbt!]
\caption{A circuit for exact $\text{OR}_3$ from Takahashi-Tani \cite{Takahashi2011}.}
\label{fig:exact-or}
\end{figure}

%%%%%%%%%%%
\subsection{A Majority Gate in \textsf{2D CCNTCM} in Sublogarithmic Depth}
\label{subsec:maj-gate}

Now we have all the building blocks needed to construct a
$\text{MAJ}_n$ gate in sublogarithmic depth. Using the constant-depth
majority circuits of $O(n)$ fanin from \cite{Yeh1996} mentioned at
the beginning of this section, we will then have a sublogarithmic
quantum circuit for modular exponetiation, and therefore for
Shor's factoring algorithm.

\begin{theorem}{\textbf{Constant-depth quantum $\text{MAJ}_n$ gate on \textsf{2D CCNTCM}.}}
A quantum $\text{MAJ}_n$ gate can be implemented on \textsf{2D CCNTCM} with
circuit depth $O(1)$, circuit size and width $O(n^2\log^2 n)$,
module depth $O(1)$, and module size and width $O(n \log n)$.
\label{thm:maj-gate}
\end{theorem}

\begin{proof}
We combine the gates 
Below we give the construction for $\text{MAJ}_{n}(x)$ on the
$n$-qubit input $x$.



\begin{enumerate}

\item
We compute in parallel the gates $\text{EX}^i_{n}(x)$ for
$0 \le i \le \lceil n/2 \rceil$ to determine if the quantum
threshold for majority is reached. There are at most $(n/2) + 1$
such gates. To do this, we need to use Theorem \ref{thm:or-log}.
Each gate $\text{EX}^i_{n}$ requires the following steps:

\begin{enumerate}
\item 
Compute the constant-depth reduction from $\text{EX}^t_{n}$ to
$\text{EX}^t_{m}$ where $m = \lceil \log_2(n+1) \rceil$, using
the reduction from $\text{OR}_n$ to $\text{OR}_{\log_n}$ \cite{Hoyer2002}.
For $1 \le k \le m$, do the following:

\begin{enumerate}
\item
Compute the qubit $\ket{\mu^{|x|-t}_{\phi_k}}$, which is the rotation by Hamming 
weight of $x$, with a threshold $t$ subtracted, by the angle $\phi_k = 2\pi / 2^k$. Note that the
required precision
of this angle is $O(\log \log n)$. This uses the result of Theorem \ref{thm:bias}.
This can be done by a \textsf{2D CCNTCM} circuit
with $O(1)$-depth, $O(n^2)$-size, and $O(n^2)$-width.
\end{enumerate}

At the end of this step, we have $m = O(\log_2 n)$ bits $\ket{y_k}$. If
$|x| \ge t$ then the qubits $\ket{y_k}$ will be orthogonal
to the state $\ket{0^m}$.

\item
Apply the \textsf{2D CCNTCM} circuit for exact $\text{OR}_{\log n}$ from
Theorem \ref{thm:or-log} of
\cite{Takahashi2011} to the output of the previous step. This can
be done with a \textsf{2D CCNTCM} circuit with $O(1)$-depth, $O(n \log n)$-size,
and $O()$ width.

\end{enumerate}

At the end of this step, we have used expected circuit depth of
$O(1)$, expected circuit size and width of $O(n^3 \log^2 n)$,
expected module depth of $O(1)$, and expected module size and
width of $O(n \log n)$.

\item
Apply the gate $\text{PA}_{\lceil n/2 \rceil}$ to the result of
the previous step. This can be done by a \textsf{2D CCNTCM} circuit of
$O(1)$-depth, $O(n)$-size, and $O(n)$-width using constant-depth
fanout, as described in Section \ref{sec:cdc}, and conjugated by
Hadamards on every qubit as described in \cite{Moore1998}. See
the equivalence of these two steps in Figure \ref{fig:pa-fanout}.

\begin{figure}[htb!]
% TODO insert Figure 1 from Hoyer-Spalek
\caption{The equivalence of $\text{PA}_n$ and $\text{FANOUT}_n$ conjugated by Hadamards.}
\label{fig:pa-fanout}
\end{figure}

\item
Apply a NOT to the output of the previous step. This final
output is the output of the quantum majority gate $MAJ_{n}$.

\end{enumerate}

The final resources for a quantum $\text{MAJ}_n$ gate is shown in
Table \ref{tab:maj-resources}, which subsumes those of the PK-KSV quantum compiler
resources from Table \ref{tab:ksv-resources}.

\begin{table}[htb!]
\begin{tabular}{c|c|}
\hline
$\langle D \rangle$ & $O(1)$ \\
\hline
$\langle S \rangle$ & $O(n^3\log^2 n)$ \\
\hline
$\langle W \rangle$ & $O(n^3 \log^2 n)$ \\
\hline
$\langle \overline{D} \rangle$ & $O(1)$ \\
\hline
$\langle \overline{S} \rangle$ & $O(n\log n)$ \\
\hline
$\langle \overline{W} \rangle$ & $O(n\log n)$ \\
\hline
\end{tabular}
\caption{Expected circuit resources for a quantum $\text{MAJ}_n$ gate.}
\label{tab:maj-resources}
\end{table}

\end{proof}

%\section{The Block-Save Technique}

This doesn't really go anywhere, but I will mention here out of
completeness.
As an aside, one may think that one can iterate the carry-save
adder. Instead of re-encoding the sum of 3 bits as the sum of
2 bits, we could also re-encode the sum of 7 bits to be the
sum of 3 bits. In analogy to the 3-2 adder, we call this a 7-3 adder.
How would this re-encoding work?

In the 3-2 adder, there were two output bits of significance $1$
and $2$. The $1$-bit was the parity of the three bits and
the $2$-bit was the majority of the three bits. The truth tables
for these functions are shown in Table \ref{tab:3-2}

\begin{table}
\begin{tabular}{cc|c}
\hline
$x_0$ & $x_1$ & $c$ \\
\hline
0 & 0 & 0 \\
0 & 1 & 1 \\
1 & 0 & 1 \\
1 & 1 & 0 \\
\hline
\end{tabular}
\caption{Truth tables for}
\label{tab:3-2}
\end{table}

 In a 7-3 adder,
likewise, the $1$-bit is the parity of \emph{single} bits.
The $2$-bit is the parity of \emph{pairs} of bits, 


\section{Conclusion}
\label{sec:fsl-conclude}

In this section, we have contributed a nearest-neighbor factoring architecture with
sub-logarithmic depth based on majority circuits. To do so, we've combined results from classical threshold
circuit complexity and our low-depth quantum architectural techniques from
Chapter $\ref{chap:factor-polylog}$.
We dsicuss the effect of quantum compiling single-qubit rotations to a fixed, finite basis.
To that end, we've given a concrete circuit for
a quantum majority gate on \textsf{2D CCNTCM} to fit into a classical, constant-depth majority circuit
for factoring. Table \ref{tab:sublog-resources} summarizes the final $\textsf{2D CCNTCM}$ resources for our sublogarithmic
factoring architecture, combining Theorems \ref{thm:mult-prod}, \ref{thm:mod-reduce}, and \ref{thm:maj-gate}.

\begin{table}[htb!]
\begin{tabular}{c|c|}
\hline
$\langle D \rangle$ & $O(\frac{1}{\epsilon}(\log\log n)^2)$ \\
\hline
$\langle S \rangle$ & $O(\frac{1}{\epsilon}n^{6 + 2\epsilon}\log^4 n\log\log n)$ \\
\hline
$\langle W \rangle$ & $O(\frac{1}{\epsilon}n^{6 + 2\epsilon}\log^2 n)$ \\
\hline
$\langle \overline{D} \rangle$ & $O(\frac{1}{\epsilon})$ \\
\hline
$\langle \overline{S} \rangle$ & $O(\frac{1}{\epsilon}n^{4+2\epsilon}\log n)$ \\
\hline
$\langle \overline{W} \rangle$ & $O(\frac{1}{\epsilon}n^{4+2\epsilon}\log n)$ \\
\hline
\end{tabular}
\caption{Expected circuit resources for a sublogarithmic factoring architecture with time-space tradeoff $\epsilon = (0,1]$.}
\label{tab:sublog-resources}
\end{table}

Now we conclude with some interesting open questions.
Although we are able to compile arbitrary rotations down to \textsf{CCNTC} in $O((\log \log n)^2)$-depth
for the resolution needed for factoring, can we reduce this to $O(1)$ depth if we relax our
requirements? For example, if we have a finite basis which is fault-tolerant, but which is not fixed;
it may vary based on the input size. quantum compiling makes truly constant-depth factoring architecture a challenging
open problem. Perhaps we will solve this in Chapter \ref{chap:qcompile}, when we discuss quantum compiling on
nearest-neighbor architectures.
Another open question is: is our factoring architectures now optimal?
What is the lower-bound for factoring on \textsf{CCNTC}, versus the $\Theta(1)$ bound for factoring on \textsf{CCAC}?
Can we determine the optimality of our factoring architectures using some other time-space product?
Perhaps we will this in Chapter \ref{chap:coherence}, when we discuss quantum circuit coherence. 

%We've also discussed an alternative
%approach to sub-logarithmic factor called iterated carry-save, which could
%potentially beat our majority circuit construction.


% In addition,
%Quantum compiling itself is a procedure which can be mapped to a quantum
%architecture, which is the topic of the next chapter. 

% ========== Chapter 3

\chapter{Quantum Compiling}
\label{chap:qcompile}

Quantum compiling is the approximation of an $n$-qubit
unitary operation using a fixed, finite, universal set
of simpler gates. Like quantum architecture, quantum
compiling plays an intermediate role between quantum
algorithms (which determine the high-level gates in a circuit),
and quantum error correction (which determines the
fault-tolerant gate set).
Quantum compiling helps mitigate one source of error,
the difference $\epsilon$ between a desired gate and its
finite approximation,
and consumes its own circuit resources, usually measured in
$(1 / \epsilon)$.
Therefore, it is important to study efficient quantum 
compiling so that we don't lose any quantum algorithmic speedups.
Most interestingly to this dissertation, quantum compiling itself
is an algorithm and can be mapped to a low-depth, nearest-neighbor
architecture. That is the subject of this chapter.

In Section \ref{sec:qcompile-bg}, we build upon the background
of Section \ref{sec:intro-basis} to rigorously define the problem of
quantum compiling and define useful notation and circuit resources,
building upon the primer on circuit bases in Section \ref{sec:intro-basis}.
We also discuss
variations and subtasks of quantum compiling that are common themes
in the literature, as well as their inter-relationships.
These themes include exact synthesis versus approximation,
single-qubit gates versus multi-qubit gates, and deterministic
versus randomized compiling procedures.

In Section \ref{sec:qcompile-sk}, we review the foundational
result in this field, the Solovay-Kitaev algorithm, and how it
continues to shape quantum compiler research. We also discuss
lower bounds on any SK-style approach to quantum compiling.
Building upon this,
in Section \ref{sec:qcompile-review}, we review the large
amount of recent literature on single-qubit quantum compiling with
constant width. We present
the cross-cutting themes of this research and provide an
at-a-glance comparison of all known single-qubit quantum compiling methods
to date in Section \ref{sec:qcompile-compare}.

As an alternative method of single-qubit quantum compiling,
in Section \ref{sec:qcompile-qfs}, we
discuss a popular method of trading low depth for increased
width: combining phase kickback with a quantum Fourier state.

In Section \ref{sec:qcompile-ksv}, we contribute an improved
algorithm for generating quantum Fourier states based on
Kitaev-Shen-Vyalyi \cite{Kitaev2002}. We also provide the calculation of
parameters necessary for any practical implementation as well as
the particular circuit resources consumed on \textsf{2D CCNTCM}.
We then compare the KSV approach to a recent alternative
by Jones which distills quantum Fourier states recursively \cite{Jones2013}.
Finally, in Section \ref{sec:qcompile-maj}, we contribute a method for
quantum compiling the single-qubit rotations needed for a
sub-logarithmic depth quantum majority gate from Chapter \ref{chap:factor-sublog}.

The last two sections represent the main contributions of this chapter.

%Finally, in Section \ref{sec:qcompile-conclude}, we conclude by summarizing
%the overall themes of quantum compiling and presenting interesting
%directions for future research.

%\section{Quantum Gates and Circuit Bases}
\label{sec:qcompile-basis}

To compile, or implement, arbitrary quantum algorithms, we must construct circuits
out of gates from a universal set, which we call a \emph{circuit basis},
or just \emph{basis}.
This should not be confused with a basis for a vector space.
Therefore, we will now review quantum gates, how to combine them into
circuits, and what it means to be universal.

A quantum gate on $n$-qubits is a $2^n \times 2^n$ unitary matrix
(an element of $U(2^n)$). We can consider this the overall circuit width.
Often, we find it useful to neglect a
global phase, since these cannot be measured in quantum mechanics.
However, a global
phase on a particular system $S$ may result in a measurable relative phase
in a large system $S'$ of which $S$ is a subsystem. Therefore, for our
purposes we will only distinguish between $U(2^n)$ and
$SU(2^n) = U(2^n) / U(1)$ in the few cases where it matters for
quantum compiling. As in the discussion of classical circuits from
Section \ref{sec:fsl-circuits}, the distinction between a quantum circuit
and a quantum gate is relative; often we consider a quantum gate as a
fundamental primitive of our physical technology, and a circuit as a
composite of these gates corresponding to a quantum algorithm.

In Section \ref{subsec:pauli} we
will review the Pauli single-qubit gates and their corresponding group.
In Section \ref{subsec:clifford} we will introduce the Clifford group.
In Section \ref{subsec:controlled} we will introduce controlled operations
and the Toffoli gate.
In Section \ref{subsec:qcompile-single} we will discuss \emph{single-qubit compiling}
and how a
general single-qubit gate can be compiled into rotations about Bloch sphere axes.
In Section \ref{subsec:distance} we will present distance metrics to
measure the quality of our single-qubit (and later multi-qubit) approximations.
In Section \ref{subsec:qcompile-bases} we will finally
define what it means to be a universal gate set, or a circuit basis.

%%%%%%%%%%%%%%%%%%%%%%%%%%%%%%%%%%%%%%%%%%%%%%%%%%%%%%%%%%%%%%%%%%%%%%%%%%%%%%
\subsection{Pauli Group}
\label{subsec:pauli}

We review here the Pauli group on one qubit, $\mathcal{P}_1 = \{I, X, Y, Z\}$.
These last three represent
rotations of $\pi$ on Bloch sphere about the $x$-axis, $y$-axis, and $z$-axis,
using the homomorphism between $SU(2)$ and $SO(3)$, as well as the
identity matrix $I$. The group $\mathcal{P}_1$ also serves as a complex vector
basis for generating elements of $U(2)$.

\begin{equation}
X = \sigma_x
 \left[
  \begin{array}{cc}
    0 & 1 \\
    1 & 0 \\
  \end{array} \right]
\qquad
Y = \sigma_z =
 \left[
  \begin{array}{cc}
    0 & i \\
   -i & 0 \\
  \end{array} \right]
\qquad
Z = \sigma_z =
 \left[
  \begin{array}{cc}
    1 & 0 \\
    0 & -1 \\
  \end{array} \right]
\qquad
I = \sigma_0 =
 \left[
  \begin{array}{cc}
    1 & 0 \\
    0 & 1 \\
  \end{array} \right]
\end{equation}

We define the Pauli group $\mathcal{P}_n$ on $n$ qubits as the set of
all $n$-qubit operators which are tensor products of elements from
$\mathcal{P}_1$.

%%%%%%%%%%%%%%%%%%%%%%%%%%%%%%%%%%%%%%%%%%%%%%%%%%%%%%%%%%%%%%%%%%%%%%%%%%%%%%
\subsection{The Clifford Group}
\label{subsec:clifford}

We define the normalizer of $\mathcal{P}_n$ as the
Clifford group $\mathcal{C}_n$ on $n$ qubits.

\begin{equation}
\mathcal{C}_n = \{ C \in U(n) | CPC^{\dagger} \in \mathcal{P}_n \quad \forall P \in \mathcal{P}_n \}
\end{equation}

Of particular interest to us is the two-qubit Clifford group $\mathcal{C}_2$,
which is generated by the following matrices (and their adjoints):

\begin{equation}
\mathcal{C}_2 = \langle H, S, CNOT \rangle
\end{equation}

The first two Clifford generator matrices are single-qubit gates ($2 \times 2$ unitary matrices) and
their inclusion means they can be applied on either the first or the second
qubit.
The matrix $H$ is known as the Hadamard gate, and it is a special case of the
general Walsh-Hadamard transform. It is its own adjoint: $H^{\dagger} = H$.
The matrix $S$ is known as the phase gate, and it can be considered the
``square root'' of the Pauli $Z$ gate (up to a phase): $S^2 = Z$.
Equivalently, it can be viewed as a $\pi/2$ rotation about the Bloch sphere
$z$-axis, and its adjoint $S^{\dagger}$ is the reverse rotation of $-\pi /2$.
These matrices are defined below.

\begin{displaymath}
H = \normtwo
 \left[
  \begin{array}{cc}
    1 & 1 \\
    1 & -1 \\
  \end{array} \right]
\qquad
S = 
 \left[
  \begin{array}{cc}
    1 & 0 \\
    0 & i \\
  \end{array} \right]
\qquad
S^{\dagger} = 
 \left[
  \begin{array}{cc}
    1 & 0 \\
    0 & -i \\
  \end{array} \right]
\end{displaymath}

The Hadamard matrix also has the special property that it changes between the
$X$ basis and the $Z$ basis, that is, the vector basis for single-qubit
states consisting of eigenstates of the Pauli $X$ and Pauli $Z$ gates,
respectively. In fact, using the identities $X = HZH$ and $S^2 = Z$, it
is easy to see why $X$ and $Z$ are often listed as generators of the
Clifford group as well.

The last Clifford generator matrix is a two-qubit gate (a $4 \times 4$ unitary matrix) which
also represents a \emph{controlled} operation. That is, based on the
$\ket{1}$ component of the \emph{control} qubit, it applies a single-qubit
gate (in this case, Pauli $X$) to the \emph{target} qubit.
In fact,
both $CNOT$ and $X$ are also fundamental gates in classical reversible
logic as well, where $X$ is also the Boolean $NOT$ gate on classical bits.
That is why the gate is called $CNOT$, for ``controlled-NOT.'' Its inclusion
in the generating set for $\mathcal{C}_2$ means that it can be applied
in either direction: with control on qubit 1 and target on qubit 2 or
vice versa. $CNOT$ is defined below.

\begin{displaymath}
CNOT = 
 \left[
  \begin{array}{cccc}
    1 & 0 & 0 & 0 \\
    0 & 1 & 0 & 0 \\
    0 & 0 & 0 & 1 \\
    0 & 0 & 1 & 0
  \end{array} \right]
\end{displaymath}

Likewise, general $\mathcal{C}_n$ can be generated from the same set
as $\mathcal{C}_2$, with gates understood to apply to any of the $n$ qubits.
The gate $CNOT$ has historical importance in quantum computing partly
due to its use in many
early quantum gate decompositions and its ability to
be performed fault-tolerantly in many physical technologies. It will be our
primary two-qubit gate.
Along with arbitrary single-qubit gates, it is universal for quantum computation \cite{Barenco1995a}.
Therefore, we will give it a special name:

\begin{equation}
\mathcal{Q} = \{ U(2) \cup CNOT \}
\end{equation}

%%%%%%%%%%%%%%%%%%%%%%%%%%%%%%%%%%%%%%%%%%%%%%%%%%%%%%%%%%%%%%%%%%%%%%%%%%%%%%
\subsection{Controlled Gates}
\label{subsec:controlled}

The principle of a controlled gate can be generalized to multiple
controls using the ``meta-operator'' notation from \cite{Kitaev2002}.
By $\Lambda^n(U)$, we mean an $(n+1)$-qubit gate ($2^{n+1} \times 2^{n+1}$
unitary matrix) with $n$ control qubits and a single-qubit target gate
$U \in U(2)$. An important multiply-controlled gate, which is universal
for classical reversible circuits, is the Toffoli gate, or controlled-controlled-$NOT$.

\begin{equation}
Toffoli = \Lambda^2(X) = 
 \left[
  \begin{array}{cccccccc}
    1 & 0 & 0 & 0 & 0 & 0 & 0 & 0 \\
    0 & 1 & 0 & 0 & 0 & 0 & 0 & 0 \\
    0 & 0 & 1 & 0 & 0 & 0 & 0 & 0 \\
    0 & 0 & 0 & 1 & 0 & 0 & 0 & 0 \\
    0 & 0 & 0 & 0 & 1 & 0 & 0 & 0 \\
    0 & 0 & 0 & 0 & 0 & 1 & 0 & 0 \\
    0 & 0 & 0 & 0 & 0 & 0 & 0 & 1 \\
    0 & 0 & 0 & 0 & 0 & 0 & 1 & 0
  \end{array} \right]
\end{equation}

As seen above, multiply-controlled single-qubit gates $\Lambda^n(U)$ have a
special, sparser structure than general $n$-qubit gates in $U(2^n)$. At the
same time, it is not known how to implement them generally on physical systems
which have nearest-neighbor constraints. Combining these two facts, we can
use a common heuristic for the general task of quantum compiling:
(a) first reducing them to $\Lambda^{n-1}(U)$ gates, and then (b) compiling
the $\Lambda^{n-1}(U)$ gates to $\mathcal{Q}$.
Task (a) is discussed in Section \ref{subsec:qcompile-multi}.
We will not discuss task (b) any further except for the special case of
singly-controlled gates of the form $\Lambda(U)$ below. We refer the
interested reader to Rosenbaum \cite{Rosenbaum2012} who has shown
optimal circuits for $\Lambda^n(U)$ gates over the basis $\mathcal{Q}$
on \textsf{CCNTC}.

There is also a special case of a ``targetless'' controlled single-qubit
gate which simply rotates the $\ket{1}$ component of a single-qubit state.

\begin{equation}
\Lambda(e^{i\phi}) = 
 \left[
  \begin{array}{cc}
    1 & 0 \\
    0 & e^{i\phi} \\
  \end{array} \right]
\end{equation}

This gate is a key tool and simplification for single-qubit compiling.

%%%%%%%%%%%%%%%%%%%%%%%%%%%%%%%%%%%%%%%%%%%%%%%%%%%%%%%%%%%%%%%%%%%%%%%%%%%%%%
\subsection{Single-Qubit Compiling}
\label{subsec:qcompile-single}

A seemingly simpler task than general compiling is single-qubit compiling.
This will illustrate the basic principles of quantum compiling and the
structure that we will exploit later to choose an effective basis. Moreover,
it will reveal a general relationship between many of the single-qubit
gates that we have already introduced.

First, we review how to decompose a general $U \in U(2)$ into three single-qubit
rotations about the Bloch sphere $x$-axis and $z$-axis, the so-called
Euler angle decompositions \cite{Nielsen2000}. This gives rise to a factor of $3$
which commonly appears in resource calculations in the literature.

\begin{equation*}
U = e^{i\delta}R_Z(\gamma)R_X(\beta)R_Z(\alpha)
\end{equation*}

The gate $R_Z(\phi)$ represents a rotation about the Bloch sphere $z$-axis,
of which the Pauli $Z$ gate is a special case of a $\pi$ rotation. In fact,
it is the same as the controlled-phase gate we introduced in the previous section.

\begin{equation}
R_Z(\phi) = \Lambda(e^{i\phi}) =
\left[
  \begin{array}{cc}
    1 & 0 \\
    0 & e^{i\phi} \\
  \end{array} \right]
=
\left[
  \begin{array}{cc}
    e^{-i\phi/2} & 0 \\
    0 & e^{i\phi/2} \\
  \end{array} \right]
\end{equation}

We can now state the relationship between $S$ and $Z$, as well as introduce
an important new gate $T$ which is the square root of $S$ up to a phase. All three
are rotations about the Bloch $z$-axis by power-of-two fractions of $\pi$.

\begin{equation}
Z = R_Z(\pi) =
\left[
  \begin{array}{cc}
    1 & 0 \\
    0 & -1 \\
  \end{array} \right]
\qquad
S = R_Z(\pi/2) =
\left[
  \begin{array}{cc}
    1 & 0 \\
    0 & i \\
  \end{array} \right]
\qquad
T = R_Z(\pi/4) =
\left[
  \begin{array}{cc}
    1 & 0 \\
    0 & e^{i\pi / 4} \\
  \end{array} \right]
\end{equation}

Likewise, the gate $R_X(\phi)$ represents a rotation about the Bloch sphere $x$-axis,
of which the Pauli $X$ gate is a special case of a $\pi$ rotation.

\begin{equation}
R_X(\phi) =
\left[
  \begin{array}{cc}
    \cos \phi & -i \sin \phi \\
    -i \sin \phi & \cos \phi \\
  \end{array} \right]
\end{equation}

Similar decompositions can be given in terms of $R_X$ and $R_Y$, or in
terms of $R_Y$ and $R_Z$. Solving for the angles $\{ \alpha, \beta, \gamma, \delta \}$
involves writing four equations in four variables, which can be found in
the standard textbook \cite{Nielsen2000}. We will not say
more about their solution here, except that we can implement the
global phase shift $e^{i\delta}$ using the identities below, which are
adapted from \cite{Kitaev2002}.

\begin{eqnarray}
e^{i\delta} & = & R_Z(\phi)\cdot X \cdot R_Z(\phi) \cdot X \\
X & = & R_X(\pi) \\
Z & = & R_Z(\pi) \\
R_X(\phi) & = & H \cdot R_Z(\phi) \cdot H
\end{eqnarray}

It now seems that a reasonable basis for single-qubit compiling are
arbitrary $R_Z(\phi)$ and $R_X(\phi)$ rotations, along with $H$.
However, in practice our classical control can only implement
rotations with finite precision. How can we measure this, or any
other, precision arising from approximation?

%%%%%%%%%%%%%%%%%%%%%%%%%%%%%%%%%%%%%%%%%%%%%%%%%%%%%%%%%%%%%%%%%%%%%%%%%%%%%%
\subsection{Distance Metrics}
\label{subsec:distance}

Each gate from our basis is a unitary matrix of bounded dimension, and the action of an entire
$n$-qubit compiled circuit $\tilde{C}$
is also a $2^n \times 2^n$ unitary matrix. This matrix can be formed
by the product of $2^n \times 2^n$ matrices $G_i$ which are a tensor
product of gate matrices from our basis. In fact, the $G_i$'s are the timesteps from
our \textsf{2D CCNTCM} model defined in Section \ref{subsec:models}, and
are formed from the tensor product of all single-qubit and two-qubit gates
which execute concurrently on disjoint qubits. Our desired target matrix $C$
is itself
a matrix from $U(2^n)$, and therefore we will need a distance metric
that operates on matrices (specifically, the difference of matrices).

\begin{equation}
\mathcal{C} = G_{D}G_{D-1}\cdots G_{2} G_{1}
\end{equation}

One distance metric used in theoretical literature
is the operator norm of a matrix $M$,
is defined as the maximum amount it scales the vector norm
of all unit-length vectors. This is sometimes also called the
infinity-norm, or supremum-norm (sup-norm).

\begin{equation}
\| M \|_{\infty} = \max_{\| \ket{v} \| = 1} \| M \ket{v} \|
\end{equation}

However, this is not an operational definition.
Moreover, we often wish to neglect a global phase in a unitary matrix,
which is not measurable in quantum physics. This is equivalent to
defining the set of valid $n$-qubit quantum gates as the
group $SU(2^n) = U(2^n) / U(1)$. To measure phase-independent
distance between two unitary matrices, we can use the following
distance measure due to Fowler \cite{Fowler2011}.

\begin{equation}
dist(U, V) = \| U - V\| \equiv \sqrt{\frac{2^n - |tr(U^{\dag}V)|}{2^n}}
\end{equation}

Now can quantify the quality of our approximations through an
error $\epsilon$.

\begin{equation}
\| C - \tilde{C}\| = \| C -  G_D\cdots G_1 \| < \epsilon
\end{equation}

Often $\epsilon$ will be small, and we will upper bound it by some
power of $\frac{1}{2}$. Therefore, we define a
new parameter which is the number of bits needed to encode the exponent
of this increasingly small fraction.

\begin{equation}
\epsilon = \frac{1}{2^n} \qquad
n = \log(1/\epsilon)
\end{equation}

It is natural to suppose that compiling better approximations requires
more resources, and these resources are expressed as functions of these
parameters $\epsilon$ and $n$.\footnote{This is not the same as our circuit
width $n$, and we will distinguish between the two when there is ambiguity.}
In fact, often the efficiency and the
capabilities of a quantum
compiler depend on its basis. Therefore, we conclude this section by
discussing circuit bases.

%%%%%%%%%%%%%%%%%%%%%%%%%%%%%%%%%%%%%%%%%%%%%%%%%%%%%%%%%%%%%%%%%%%%%%%%%%%%%%
\subsection{Circuit Bases}
\label{subsec:qcompile-bases}

\begin{definition}{\textbf{Circuit basis.}}
A basis for a quantum circuit is a universal set of
bounded-qubit (usually $3$-qubit) gates.
We call a basis \emph{finite} if it contains a finite
number of gates; that is, it contains discrete gates and not an infinite
continuum of gates. We call a basis \emph{fixed} if its members are independent
of any input size.
\end{definition}

For fault-tolerant quantum computing, we are interested in compiling
circuits to a fixed, finite basis. What does it mean for a fixed, finite
basis to be universal for an infinite group like $SU(2^n)$?

\begin{definition}{\textbf{Universal approximation.}}
We call a fixed, finite set of gates $\mathcal{G}$ \emph{universal} for
a group $G$ iff for every desired target $C \in G$ and
desired error $\epsilon$, we can return a
sequence of gates $(g_1,g_2,\ldots,g_S)$ from $\mathcal{G}$ where

\begin{equation}
\| C - g_1 g_2 \cdots g_S \| \le \epsilon
\end{equation}

\end{definition}

This defines whether a gateset is a basis, or whether universal approximation
is even possible (non-constructively). We will see in later sections that
quantum compilers are concerned with constructive approaches to
\emph{efficiently} return such a compiled sequence $\tilde{C} = \prod g_i$

What gatesets are known to be fixed, finite, and universal, and therefore
suitable bases for quantum compilation?

We recall one such basis from our definition of \textsf{2D CCNTCM} in
Section \ref{subsec:models}, with the exception of the non-unitary operation
$MeasureZ$. However, $MeasureZ$ is implicitly assumed as part of any basis,
and is the means by which we can offload postprocessing to a classical
controller.

\begin{equation}
\mathcal{G}'' = \{X, Z, H, Toffoli, CNOT\}
\end{equation}

It is important to note that the only non-Clifford gate in the above basis
is $Toffoli$.
The Clifford group $\mathcal{C}_n$ by itself is \emph{not}
universal.
This is unfortunate given that many quantum error-correcting codes have
efficient implementations \emph{only} for Clifford gates. In fact, it is provable
that \emph{any} universal gateset must possess at least one
non-Clifford gate \cite{Zeng2011}.

Two popular choices for the non-Clifford gate in a basis are the $T = R_Z(\pi/4)$
gate and the $Toffoli = \Lambda^2(X)$ gate. Since these are not ``natively''
supported (non-transveral) in many codes, they must often be implemented
probabilistically using only Clifford operations and $MeasureZ$, usually
by way of a so-called ``magic'' state.
Therefore, many quantum compilers
use the Clifford+$T$ basis $\mathcal{C}_2 \cup \{ T \}$ or the Clifford+$Toffoli$ basis $\mathcal{C}_2 \cup \{ Toffoli \}$, and measure
the non-Clifford gate as the most expensive resource. It is an area of
active research
whether $T$ or $Toffoli$ is more efficient to implement
\cite{Jones2013a,Eastin2012}.

For single-qubit compilation, the $\{H,T, T^{\dagger}\}$ gateset is universal and
plays an important role in the literature. Other compilers may add
the Clifford gates $S$ and $S^{\dagger}$ to the basis $\mathcal{G}'$ above,
but this does not change its universality nor its asymptotic efficiency for
compiling.

We are now prepared to discuss resources for measuring the efficiency of
a quantum compiler.

\section{Quantum Compiling Themes}
\label{sec:qcompile-bg}

Quantum compiling is a classical procedure for transforming quantum circuits.
The image to keep in mind throughout this entire section is shown in
Figure \ref{fig:qcompile}.

\begin{figure}
\begin{center}
\begin{displaymath}
\begin{array}{ccc}

%%%%%%%%%%%%%%%%
% Source circuit
%\underbrace{
\begin{array}{c}
S = 2 \\
\Qcircuit @C=0.5em @R=.5em { 
	& \multigate{4}{U_1} & \qw & \multigate{4}{U_2} & \qw \\ 
	& \ghost{U_1}        & \qw & \ghost{U_2}        & \qw \\
	& \ghost{U_1}        & \qw & \ghost{U_2}        & \qw \\
	& \ghost{U_1}        & \qw & \ghost{U_2}        & \qw \\
	& \ghost{U_1}        & \qw & \ghost{U_2}        & \qw 
	\gategroup{1}{2}{5}{4}{.7em}{--}
}\\
\xymatrix {
  & D=2 \ar[l] \ar[r] & \\
 }
\end{array}
%}_{C}

%& 
%\begin{array}{c}
%\textsc{Quantum Compiler} \\
\rightarrow
%\end{array}
%&

%%%%%%%%%%%%%%%%
% Target circuit
%\underbrace{
\begin{array}{c}
S' = 15 \\
\Qcircuit @C=0.5em @R=.5em { 
	& \gate{H} & \qw & \ctrl{1} & \gate{H} & \qw & \qw      & \ctrl{1} & \qw \\ 
	& \gate{H} & \qw & \targfix & \ctrl{2} & \qw & \gate{K} & \targfix & \qw \\
	& \gate{H} & \qw & \gate{K} & \qw      & \qw & \gate{H} & \qw      & \qw \\
	& \gate{H} & \qw & \ctrl{1} & \targfix & \qw & \gate{H} & \qw      & \qw \\
	& \gate{H} & \qw & \targfix & \gate{H} & \qw & \qw      & \qw      & \qw
	\gategroup{1}{2}{5}{9}{.7em}{--}
}\\
\xymatrix {
  & & D'=5 \ar[ll] \ar[rr] & & \\
 }
\end{array}
%}_{C}

\end{array}
\end{displaymath}

\caption{An arbitrary quantum circuit being compiled into single-qubit gates and $CNOT$.}
\label{fig:qcompile}
\end{center}
\end{figure}

General quantum compiling can be subdivided into more special-purpose tasks along several axes,
which are cross-cutting themes in any literature review of quantum compilers.
These themes also provide a context for understanding the resource consumption
for a wide variety of quantum compilers.

These axes are:

\begin{enumerate}
\item single-qubit compiling versus multi-qubit compiling
\item exact synthesis versus approximative quantum compiling
\item deterministic versus probabilistic quantum compiling
\item compilers with provable upper bounds versus conjectured upper bounds
\end{enumerate}

The first axis is 
single-qubit compiling
(mentioned previously in Section \ref{subsec:qcompile-single}) versus
multi-qubit compiling. Some algorithms which work on single-qubit compiling
can be generalized directly to the multi-qubit case. In fact, all known
examples of these generalized algorithms can accept an arbitrary circuit
basis $\mathcal{B}$ \cite{Amy2012,Dawson2005,Fowler2011,Booth2012}.
That is, they do not exploit any special structure of
a particular basis. The circuit basis is another input to the algorithm,
possibly to an additional classical preprocessing step. Whether the algorithm
is a single-qubit or a multi-qubit algorithm depends on whether the basis
is single-qubit or multi-qubit.

There is an intermediate point on this axis, between single-qubit and multi-qubit,
which is the reduction of a multi-qubit circuit into a basis of
single-qubit and two-qubit gates. This task is often called \emph{quantum circuit synthesis},
and we will discuss it in Section \ref{subsec:qcompile-multi}.

The second axis is compiling a circuit exactly or approximately.
Exact synthesis refers to the case of determining whether a
target circuit $C$ is implementable from a basis $\mathcal{B}$
with no error ($\epsilon = 0$). If this is possible, a quantum compiler
should return the sequence of gates which constitute the exact
synthesis. Furthermore, exact synthesizers often have a goal of
returning the \emph{optimal} sequence of compiled gates, that is,
one with minimal length $\ell$. In the compilers that we review
in Section \ref{sec:qcompile-review}, $\ell$ stands for the optimal
depth of non-Clifford resources in a basis which also contains Clifford
gates. Non-Clifford resources are always more expensive than
Clifford gates in most error-correcting codes. There is evidence
that the Clifford resources needed to synthesis the non-Clifford gates $T$ and $Toffoli$
are within a small constant factor of each other \cite{Eastin2012,Jones2012}

Exact synthesizers often enumerate over all circuits of
a certain length from a certain basis $\mathcal{B}$. Therefore, their
resources are upper bounded by a brute force search, which takes
time upper-bounded by $|\mathcal{B}|^{\ell}$.
Approximative quantum compiling conforms to our usual notion where
$\epsilon > 0$, and achieving smaller error costs more resources. Many
exact synthesis algorithms can be used to build basic approximations
for the Solovay-Kitaev algorithm more efficiently, and therefore help
achieve better approximative upper bounds as verified by numerical
simulation over random unitaries.

What is the relationship between $\ell$ and $\epsilon$? By a volume
argument, the minimum number of points in an $\epsilon_0$-net for
$SU(d=2^n)$ is $1/(\epsilon^{d^2 - 1})$. If we were to do an approximative
search within error $\epsilon_0$
for a circuit in $SU(d)$ which is known to have optimal length
$\ell$, we would have to enumerate all sequences from a basis $\mathcal{B}$
of up to length $\ell$ in the worst case, of which there are $|\mathcal{B}|^{\ell}$.
Therefore, we have the following relationship.

\begin{equation}
\ell \ge (d^2 - 1) \log_{|B|}(1/\epsilon) + O(1)
\end{equation}

The third axis is whether a quantum compiling algorithm uses randomness
or is completely deterministic. For known randomized algorithms, it is
an open problem whether the algorithm can be derandomized or not
\cite{Kliuchnikov2012a}, and numerical verification is necessary to
show the desired distribution of running times.

The fourth axis is whether a quantum compiler has upper bounds
(usually on running time or compiled sequence length) that are provable or
based on a conjecture. Both deterministic and randomized
algorithms can have provable upper bounds, although
in the latter case, one calculates the average-case and upper bounds the
variance. Likewise, both deterministic and randomized algorithms can
be based on a conjecture. One example is a deterministic algorithm
whose resources are too difficult to compute in any other way than
numerical simulation and fitting a curve to the data.

These four axes can be used to classify quantum compilers, although some
algorithms can be placed in multiple categories. For example, many
single-qubit quantum compilers which perform exact synthesis can be
incorporated into a hybrid algorithm which then performs
approximation. And of course, some single-qubit quantum compilers can be generalized
into multi-qubit algorithms.

A fifth axis could be formed, which is whether the compiled circuit requires
arbitrarily long interactions for $CNOT$ or is nearest-neighbor. Such a
quantum compiler could also divide up a compiled circuit into an optimal
number of modules on \textsf{2D CCNTCM} to also minimize module depth and
module size (inter-module teleportations). This is an interesting direction
for future research.

%%%%%%%%%%%%%%%%%%%%%%%%%%%%%%%%%%%%%%%%%%%%%%%%%%%%%%%%%%%%%%%%%%%%%%%%%%%%%%
\subsection{Quantum Compiler Resources}

Just as a quantum algorithm with arbitrary long-range interactions incurs
some overhead in being mapped to a nearest-neighbor architecture,
a quantum compiler itself is an algorithm. It always has a classical
component, which runs on a digital computer, and transforms a classical
description of an input quantum circuit into an output circuit from
a basis $\mathcal{B}$. The compiled output circuit then runs on a
quantum computer. In general, the compiled output circuit $\tilde{C}$ consumes
resources which are greater than those of the input circuit $C$.

Not all quantum compilers are ``total functions.'' Some of them, notably
single-qubit compilers, are ``promise functions'' in that they can
only compile gates of a certain form (usually $R_Z(\phi)$) and require
prior decomposition of a multi-qubit gate down to the set of
$Q \cup \{R_Z(\phi)\}$.

\begin{description}
\item[classical runtime $R$:] the classical time it takes to return a 
compiled quantum circuit.
\item[input depth $D$:] the depth of the input quantum circuit in arbitrary
$n$-qubit gates.
\item[input size $S$:] the size of the input quantum circuit in arbitrary
$n$-qubit gates.
\item[input width $W$:] the width of the input quantum circuit in qubits.
\item[compiled depth $D'$:] the compiled quantum circuit depth, equal to
the compiled sequence length for single-qubit circuits.
\item[compiled size $S$:] the compiled quantum circuit size, which is
identical to compiled depth if no ancillae are used (compiled width is zero).
\item[compiled width $W$:] the compiled quantum circuit width, which includes
the width of the input circuit as well as any ancillae introduced by
the compiler.
\end{description}

All but the first resource are quantum in nature, and follow the definitions
for circuit resources from Chapter \ref{chap:factor-polylog}. Because
compilation incurs some overhead, we have $D' \ge D$, $S' \ge S$, and
$W' \ge W$.

It's also known that
in order to approximate a circuit with $S$ gates to a total precision of
$\epsilon$
requires each gate to be approximated to a precision of
$n = O(\log(S/\epsilon)$ \cite{Lloyd1995}. We denote this per-gate precision
$n$, since it serves as an independent parameter for compiling. For
single-qubit gates, $S = 1$, and this corresponds exactly with our previous
definition for $n$ in Section \ref{sec:qcompile-basis}.

We do not measure classical space requirements, although these may be
exponential. This would be a useful metric for comparison for future work.

%%%%%%%%%%%%%%%%%%%%%%%%%%%%%%%%%%%%%%%%%%%%%%%%%%%%%%%%%%%%%%%%%%%%%%%%%%%%%%
\subsection{Decomposition to Bounded-Qubit Gates}
\label{subsec:qcompile-multi}

Restricting ourselves to the simplest case of
single-qubit circuits allows us to exploit a lot of structure
in the group $U(2)$ (or its related subgroups $SU(2)$ and $PSU(2)$).
From a volume argument, we can derive a general
lower bound for the efficiency of the multi-qubit case \cite{Harrow2002},
as well as determine how our compiling efficiency scales with dimensionality.
Any
SK-style algorithm produces worst-case sequence lengths $\ell_d$ that
are longer than worst-case single-qubit sequence lengths $\ell_1$ by a certain multiplicative
prefactor. This prefactor has a dependence that is at least
polynomial in $d = 2^n$. 

\begin{equation}
% TODO fact check this!
\ell_d / \ell_1 = \Omega \left( \frac{d^2 - 1}{ \log |\mathcal{B}| } \right )
\end{equation}

This is an example of task modularity which allows us
to divide the effort of quantum compiling between the
single-qubit case and then decomposition to single-qubit gates and
$CNOT$. It is a heuristic which often results
in simple decompositions to implement in (classical) software.
It may not be asymptotically optimal compared to generic 
multi-qubit protocols. However, for small input sizes, it is often
tractable to run on modern digital computers.

Now that we have handled the single-qubit case, how can we leverage this
to compile general $n$-qubit gates? We need a reduction to the basis
$\mathcal{Q} = U(2) \cup \{ CNOT \}$, as originally depicted in
Figure \ref{fig:qcompile}.
It turns out that almost any two-qubit gate plus arbitrary single-qubit
rotations are universal \cite{Bremner2002}. However, we will stick with CNOT
due to its other useful properties. A table of known
upper and lower bounds for this task are given in Table \ref{tab:multi}.
A standard two-level decomposition, such as provided on page 70 of \cite{Kitaev2002},
decomposes a general $U(d=2^n)$ matrix down to $O(d^2)$ ``two-level'' matrices which can
be implemented with multiply-controlled single qubit gates $\Lambda^{n-1}(U)$
for $U \in U(2)$. These gates are implementable with $O(n)$ $CNOT$ gates each.

The recent Giles-Selinger proves a conjecture by Kliuchnikov-Mosca-Maslov \cite{Kliuchnikov2012e}
that $n$-qubit circuits implementable by the basis $\mathcal{C}_2 \cup \{ T \}$ is
equivalent to all $U(2^n)$ matrices with elements from the ring $\mathbb{Z}\left[i,\frac{1}{\sqrt{2}}\right]$.
Their construction to find this exact synthesis of an $n$-qubit gate is meant to
prove this equivalent, and is not optimal.

The optimal bound needed for this in terms of $CNOT$ gates in
the compiled output (the dominant cost) is still exponential
$O(4^n)$ \cite{Shende2004}.

\begin{table}[hbt!]
\centerline{
\begin{tabular}{|c|c|}
\hline
Decomposition Method & $CNOT$ Cost 
\hline
Giles-Selinger \cite{Giles2012} & $O(9^n nk)$\\
Two-level \cite{Kitaev2002} & $O(4^n n)$ \\
Vartiainen-M\"{o}tt\"{o}nen-Salomaa \cite{Vartainen2003} & $o(11\cdot 4^n)$ \\
Aho-Svore \cite{Aho2003} & $o(1.17\cdot 4^n - 3.51\cdot 4^n + 3.34)$ \\
Shende-Bullock-Markov \cite{Shende2004a} & $o(0.48\cdot 4^n - 1.50\cdot 2^n + 1.34)$ \\
Shende-Bullock-Markov \cite{Shende2004} & $\omega(0.25\cdot 4^n - 3n - 1)$ \\
BBC+ \cite{Barenco1995a} & $\omega(0.10\cdot 4^n - 0.34n - 0.12)$ \\
\hline
\end{tabular}
}
\label{tab:multi}
\caption{Comparison of multi-qubit circuit synthesizers in $CNOT$ cost, both upper and lower bounds.
The $o(\cdot)$ and $\omega(\cdot)$ notation are used to indicate when multiplicative constants are known.}
\end{table}

\section{The Solovay-Kitaev Algorithm}
\label{sec:qcompile-sk}

As one of the central results of quantum computation, it is worth
reviewing here the venerable Solovay-Kitaev (SK) algorithm for the
approximation of $SU(d)$ gates by a fixed, finite basis $\mathcal{B}$.
Although many recent results have surpassed SK in terms of efficiency,
they often do so by improving the base-level approximation of the
original SK algorithm. Moreover, many techniques for analyzing
and understanding quantum compilers were developed for SK, which
continues to be the best way to learn them. Finally, the overall
structure of the original SK algorithm is so simple, it is surprising
that it works so well. Therefore, it is worth
our time to review it here.

The essential structure of SK is to recursively generate nets of unitary
operators with
successively finer precision. At any given level of recursion, the input 
gate is divided up into two halves which can be
approximated with less precision. Our pseudo-code and explanation
follows the exposition in \cite{Dawson2005} and \cite{Harrow2001}

\begin{algorithmic}[1]
\STATE \textsc{function} $\tilde{U}_i \leftarrow$ SK$(U,n)$
\IF{$i = 0$}
\STATE $\tilde{U}_i \leftarrow $ BASIC-APPROX$(U)$
\ELSE
\STATE $\tilde{U}_{i-1} \leftarrow$ SK$(U, i-1)$
\STATE $A,B \leftarrow $ FACTOR$(U\tilde{U}^\dagger_{i-1})$
\STATE $\tilde{A}_{i-1} \leftarrow $ SK$(A, i-1)$
\STATE $\tilde{B}_{i-1} \leftarrow $ SK$(B, i-1)$
\STATE $\tilde{U}_i \leftarrow \tilde{A}_{i-1}\tilde{B}_{i-1}\tilde{A}^\dagger_{i-1}\tilde{B}^\dagger_{i-1}\tilde{U}_{i-1}$
\ENDIF
\STATE return $\tilde{U}_i$
\end{algorithmic}

Since the recursion must eventually bottom out, we must precompute some sequences
of gates from $\mathcal{B}$ up to length $l_0$. This is the classical
preprocessing step which requires upfront storage space for this
coarsest-grained net, where
each sequence is no more than $\epsilon_0$ from its nearest neighbor. According
to \cite{Dawson2005} the values of $l_0=16$ and $\epsilon_0 = 0.14$ using
operator norm distance is sufficient for most applications.
This step can be
done offline and reused across multiple runs of the compiler, assuming
$\mathcal{B}$ for your quantum computer doesn't change.

The BASIC-APPROX function above does a lookup (e.g. using some kd-tree search
maneuvers through higher-dimensional vector spaces) using this $\epsilon_0$-net,
and all higher recursive calls to SK are effectively constructing
finer $\epsilon$-nets on the fly as needed.

The FACTOR function performs a balanced group commutator decomposition,
$U = ABA^\dagger B^\dagger$, and then recursively approximates the $A$ and $B$
operators, again using SK. We denote by $\tilde{U}_i$ as the approximation
of $U$ using $i$ levels of SK recursion. When they are multiplied
together again, along with their inverses, their errors (which go like
$\epsilon$) are symmetric and cancel out in such a way that the resulting
product $U$ has errors which go like $\epsilon^2$, using the properties of
the balanced group commutator. In this manner, we can
eventually sharpen our desired error down to any value. A
geometric decomposition is used in the Dawson-Nielsen implementation \cite{Dawson2005},
while one based on a Baker-Campbell-Hausdorff approximation is used in the
Harrow implementation \cite{Harrow2001}. It is not
known which method converges more quickly to a desired gate in general.

%%%%%%%%%%%%%%%%%%%%%%%%%%%%%%%%%%%%%%%%%%%%%%%%%%%%%%%%%%%%%%%%%%%%%%%%%%%%%%%
%%%%%%%%%%%%%%%%%%%%%%%%%%%%%%%%%%%%%%%%%%%%%%%%%%%%%%%%%%%%%%%%%%%%%%%%%%%%%%%
% we still need to reconcile this 5^n with log^{3.97}(1/\epsilon)
% this is an evernote item, maybe it is already reconciled

Each level $i$ of recursion solves the problem of compiling 
$U_i$ by combining five gates compiled at the lower $i-1$ level.
Therefore, the compiled circuit size at the top-level is upper-bounded by $5^n$,
and
this is in general the same as the circuit depth (since not all gates in
$\mathcal{B}$ commute).

%%%%%%%%%%%%%%%%%%%%%%%%%%%%%%%%%%%%%%%%%%%%%%%%%%%%%%%%%%%%%%%%%%%%%%%%%%%%%%%
%%%%%%%%%%%%%%%%%%%%%%%%%%%%%%%%%%%%%%%%%%%%%%%%%%%%%%%%%%%%%%%%%%%%%%%%%%%%%%%
% we still need a discussion somewhere of the relationship between commuting
% operators and parallelizing them into the same timestep

\section{Quantum Compiler Review}
\label{sec:qcompile-review}

\subsection{Alternative Bases and Resources}

Now we turn to three recent works which use magic state distillation to compile
arbitrary single-qubit gates. These works either compile to an alternate
basis from the Clifford group $\mathcal{C}_n$ and $T$

A recent approach by Bocharov-Gurevich-Svore \cite{Bocharov2013}
compiles to subsets of the Clifford group augmented with the non-Clifford $V$-basis, which was proven to permit the lower-bound of compiled sequence length $O(\log^1(1/\epsilon)$
\cite{Harrow2003}.

\begin{equation}
V_1 = TODO \qquad V_2 = TODO \qquad V_3 = TODO
\end{equation}

This work uses the properties of Lipschitz quaternions with norms $5^l$, ($l \in \mathbb{Z}, l \ge 0$). It
contains a randomized algorithm whose running time is based on a conjecture from geometric number theory.
There is currently no complete, fault-tolerant method of compiling all three gates from the $V$ basis into
our usual universal set of $\mathcal{C}_1 \cup \{T\}$. However, the appendix of \cite{Bocharov2013}
gives a method for implementing the exact $V_2$ gate using the (probabilistic) magic state distillation of
Duclos-Cianci and Svore \cite{DuclosCianci2012}. 
We cannot compare it directly to previous algorithms which consider the number of $T$ gates ($T_c$)
the primary resource, or the compiled sequence length $D'$ as an upper bound to $T_c$.

\section{Quantum Compiler Comparison}
\label{sec:qcompile-compare}

\begin{landscape}

\begin{table}[hbt!]
\begin{center}
\begin{tabular}{|c|c|c|c|c|c|c|c|c|}
\hline
Algorithm                  & A/E & D/R & P/C & Multi? & $R$                             & $D'$                    & $S'$                  & $W'$ \\
\hline
Fowler\cite{Fowler2011}     & A   & D   & P  & Y    & $O(\ell |\mathcal{B}|^{\ell})$    &                         &                       &     \\
Booth \cite{Booth2012}      & A   & D   & P  & Y    & $O(\ell |\mathcal{B}|^{\ell/2})$  &                         &                       & 1    \\
\hline
AMMR \cite{Amy2012}         & E   & D   & P  & Y    & $O(\ell |\mathcal{B}|^{\ell/2})$  &                         &                       & 1    \\
BS \cite{Bocharov2012}      & E   & D   & P  & N    & $O(\ell |\mathcal{B}|^{\ell/4})$  &                         &                       & 1    \\
%BSG \cite{Bocharov2013}     & A   & D   & N    & $O(\ell |\mathcal{B}|^{\ell/4})$ & $\ell$           & $D'$ & 1    \\
KMM-e\cite{Kliuchnikov2012e} & E   & D   & P  & ?    & $\ell^2$                          &                         &                       & 1    \\
\hline
SK-DN\cite{Dawson2005}      & A   & D   & P  & Y    & $O(\eta^{2.71} + \mathcal{B}^{l_0})$ & $O(D \eta^{3.97})$         &                       & 1    \\
SK-KSV\cite{Kitaev2002}     & A   & D   & P  & N    & $O(\eta^{3+o(1)})$                   & $O(D \eta^{3+o(1)})$        &                       & 1    \\
SK-BS \cite{Bocharov2012}   & A   & D   & C  & N    & $O(\eta^{2.71} + \mathcal{B}^{l_0})$ & $O(D\eta^{3.4})$           &                       & 1    \\
KMM-a\cite{Kliuchnikov2012a}& A   & P   & C  & N    & $O(\eta^2\log n)$                    & $O(D\eta)$                 &                       & 1    \\
Selinger\cite{Selinger2012} & A   & P   & C  & N    & $O(\eta^4)$                          & $D(48\eta + 44)$           &                       & 1    \\
\hline
PK-KSV \cite{Kitaev2002,Cleve2000} & E & D & P & N  & $O(1)$                            & $O(D\log \eta + \log^2 \eta)$ & $O(S \eta + \eta^2 \log \eta)$ & $O(\eta^2)$ \\
PK-Jones \cite{Jones2013}   & E  & D    & P  & N     & $O(1)$                           & $O(D\log \eta + \eta \log \eta)$ & $O(S \eta + \eta \log \eta)$   & $2\eta + O(1)$ \\
\hline
\end{tabular}
\caption[A comparison of single-qubit quantum compilers]{A comparison of single-qubit quantum compilers, where $\eta = \log_2(1/\epsilon)$, for a desired error $\epsilon$.
Blank depths $D'$ are $\ell$. The maximum basic approximation length is $\ell_0$ \cite{Dawson2005}. Blank sizes $S'$ are equal to $D'$. Blank widths $W'$ are equal to $1$.
A/E: Approximate versus exact-synthesis. D/R: Deterministic versus randomized. P/C: provable bounds versus conjectured bounds + numerical verification.
Multi?: Y if it can be generalized to a multi-qubit compiler.}
\label{tab:qcompile-compare}
\end{center}
\end{table}

\end{landscape}


\section{Phase Kickback and Quantum Fourier States}
\label{sec:qfs}

In this section, we discuss the \emph{phase kickback} procedure for
producing a qubit with arbitrary phase: $\ket{0} + e^{i\phi}\ket{1}$.
From the PAR procedure, a qubit could be prepared in such a state
``offline' and then teleported later into a \textsf{2D CCNTCM} module to
probabilistically enact rotations $R_Z(\phi)$ \cite{Jones2011}.

Phase kickback requires controlled addition on a target state in 
the quantum Fourier basis. This will provide us the foundation
to understand the KSV procedure for generating such quantum Fourier
states in the next section.

In Section \ref{subsec:qfs-basis}, we discuss the properties of the
basis of quantum Fourier states, including their relationship with
addition modulo $2^n$. In Section \ref{subsec:qfs-resources},
we calculate circuit resources for performing these adders
on \textsf{2D CCNTCM}. Although the adders are generic, they will be
used in the resource calculations of KSV phase estimation in
Section \ref{sec:ksv-pe}, which operates on quantum Fourier states.

%%%%%%%%%%%%%%%%%%%%%%%%%%%%%%%%%%%%%%%%%%%%%%%%%%%%%%%%%%%%%%%%%%%%%%%%%%%%%%
\subsection{Properties of the Quantum Fourier Basis}
\label{subsec:qfs-basis}

The following states are well-known as $n$-qubit quantum Fourier states,
the result of applying the quantum Fourier transform (QFT)
to the $n$-qubit computational basis state. 

\begin{equation}
\ket{\psi^{(k)}_{n}} = \frac{1}{\sqrt{2^n}} \sum_{j=0}^{2^n-1}
e^{-2\pi i j k / 2^n} \ket{j}
\end{equation}

These states form an alternate, orthonormal basis indexed by
$0 \le k < 2^n$. We will often just call these Fourier states,
and neglect the subscript $n$, which is implied.
Note that the state $\ket{\psi^{(0)}}$, sometimes
called the fundamental Fourier state, is simply the equal superposition
of all computational basis states, or the tensor product of
$n$ qubits in the state $\ket{+}$, or the result of applying
$n$ Hadamard's qubit-wise to a state beginning in $\ket{0}^n$.

Note that these states $\ket{\psi}^{(k)}_n$ are the
QFT states of the $n$-qubit computational basis. The usual method of
creating these states involves performing phase estimation of the
modular addition operator. These are implicitly hard in that
all known procedures take size $O(n\log n)$, even if the depths
can be decreased to $O(\log n)$ \cite{Jones2012} or in some cases $O(1)$
given unbounded quantum fanout
\cite{Browne2009}.

However, we can create a superposition in constant depth
over all odd $k=(2s-1)$
by starting in the state $\ket{0}^{\otimes n}$,
then applying a Hadamard and $\sigma^z$ to the most significant qubit.

\begin{equation}
\ket{\eta} = \normtwo \ket{0} - \normtwo \ket{2^{n-1}} =
\frac{1}{\sqrt{2^{n-1}}} \sum_{s=1}^{2^{n-1}} \ket{\psi_{n,2s-1}}
\end{equation}

More obviously relevant to our overall goal of approximating
$\Lambda(e^{i\phi})$, we can enact a phase
shift simply by performing the following modular addition operator, for
which $\ket{\psi_{n,k}}$ are eigenstates.

\begin{equation}
A\ket{j} \rightarrow \ket{j+1 \bmod 2^n}
\end{equation}

Applying this operator to its eigenstates results in a phase shift which
depends on the particular eigenstate.
 
\begin{equation}
A\ket{\psi_{n,k}} = e^{2\pi i \phi_k} \ket{\psi_{n,k}}
\end{equation}

Finding the eigenvalue $e^{2\pi i \phi_k}$ corresponds to finding
the phase $\phi_k = k / 2^n$.
Repeated application of $A$ (say $p$ times) would result in a phase
added to the eigenstate equal to a multiple of $e^{2\pi i p / 2^n}$

\begin{equation}
A^p\ket{\psi_{n,k}} = e^{2\pi i \phi_k / 2^n} \ket{\psi_{n,k}}
\end{equation}

This explains why we don't find even $k$ interesting,
since then we would not get a
cyclic distribution of $2^n$ different phases,
since only odd $k$
are coprime with $2^n$. The exception is $k=0$, since this is the
equal superposition of computational basis states, which we can also
efficiently create. This will be a useful starting point later on to
create addition eigenstates
for odd $k$.

\begin{displaymath}
\ket{\psi_{n,0}} = H^{\otimes n}\ket{0^n}
\end{displaymath}

Suppose we have a certain state $\ket{\psi_{n,k}}$ but we want to get enact
a phase shift $e^{2\pi i l / 2^n}$. We can do this by solving $p=p(s,l)$
in this equation:

\begin{equation}
\label{eqn:psl}
(2s-1)p \equiv l (\bmod 2^n)
\end{equation}

Stipulating $k$ to be odd guarantees that there is a unique solution $p$.

We then apply $A^p$ as follows, where $\Upsilon_n(A)$ means to
apply $A$ to the second register $p$ times, where $p$ is an $n$-qubit
number in the first register.

\begin{equation}
\label{eqn:upsilon}
\Upsilon_n(A) \ket{p}\ket{\psi_{n,k}} \rightarrow
e^{2\pi i l/2^n} \ket{p}\ket{\psi_{n,k}}
\end{equation}

If we control the operation of $\Upsilon_n(A)$ on a source qubit $\ket{+}$,
it will acquire the phase $e^{2\pi i l}$.

\begin{equation}
\Lambda(\Upsilon_n(A))\ket{+}\ket{p}\ket{\psi}^{(k)} \rightarrow
\left( \ket{0} + e^{2\pi i l}\ket{1} \right) \ket{p}\ket{\psi}^{(k)}
\end{equation}

This is not quite the phase kickback procedure, since
we must still solve for $p$ using the equation below by finding the
modular inverse $(2s - 1)^{-1} \bmod 2^n$,
making use of the following expansion from Section 13
of \cite{Kitaev2002}

\begin{equation}
p \equiv -l\sum_{j=0}^{m-1} (2s)^j \equiv -l \prod_{r=1}^{t-1}\left(1 + (2s)^{2^r}\right) \mod 2^n
\end{equation}

where $m = O(n)$ is $n$ rounded to the nearest power of $2$.  In general,
this requires a circuit of size $O(n^2 \log n)$ and depth $O((\log n)^2)$ and
represents the most expensive part of the KSV procedure as originally
presented in \cite{Kitaev2002}.

Ideally, we would obviate the need for the expensive circuit above
by ensuring that $k=1$, in which case
$p = l$. We will see how to do this in Section \ref{sec:ksv-pe}.

Finally, to copy the state $\ket{\psi_{n,k}}$ it suffices to apply the following
operator which only uses subtraction (addition with one addend and the
outcome negated in two's complement representation).

\begin{equation*}
\ket{\psi_{n,k}}^{\otimes m} = W^{-1}\left( \ket{\psi_{n,0}}^\otimes(m-1) \otimes \ket{\psi_{n,k}} \right)
\end{equation*}

where $W$ is the operator on $m$ registers, each of consisting of $n$ qubits,
which adds all the registers into its final register (modulo $2^n$).

\begin{multline}
W : \ket{x_1,\ldots,x_{m-1},x_m} \rightarrow \\
 \ket{x_1,\ldots,x_{m-1},x_1+\ldots+x_m \bmod 2^n}
\end{multline}

Having complained about the cost of modular division, how can we now
implement the operators $A$ and $W$ more efficiently on our
nearest-neighbor architecture?

%%%%%%%%%%%%%%%%%%%%%%%%%%%%%%%%%%%%%%%%%%%%%%%%%%%%%%%%%%%%%%%%%%%%%%%%%%%%%%
\subsection{Circuit Resources for Adders on Quantum Fourier States}

First we prove a general result for mapping any \textsf{AC} circuit
to a \textsf{2D CCNTCM} circuit.
Then we use it to 
map the operator $A$ (increment modulo $2^n$) and the operator
$W$ (multiple addition modulo $2^n$) to \textsf{2D CCNTCM}.

\begin{lemma}
Suppose an \textsf{AC} circuit $\mathcal{C}$ has
depth $D(n)$, size $S(n)$, and width $W(n)$. Then it can be mapped
onto a \textsf{2D CCNTCM} circuit $\mathcal{C'}$ with circuit depth
$D(n)$, circuit size $S(n)$, circuit width $D(n)\cdot W(n)$,
module depth $D(n)$, module size $D(n)\cdot W(n)$, and module width
$D(n)$.
\label{lem:ac-ccntcm}
\end{lemma}

\begin{proof}
Each of $D(n)$ layers in $\mathcal{C}$ could operate on non-nearest-neighbors.
We teleport all
$W(n)$ qubits from the first module to the last module in sequence,
reordering them at each module so that we can execute the gates of that
layer only on nearest-neighbor qubits.
This gives the desired circuit and module resources on
\textsf{2D CCNTCM}.
\end{lemma}

\begin{lemma}
The operator $A\ket{j} \rightarrow \ket{j+1 \bmod 2^n}$ on
$n$-qubit register can be
implemented in circuit/module depth $O(\log n)$, circuit size $O(n)$, 
circuit width $O(n\log n)$, module size of $O(n \log n)$, and
module width of $O(\log n)$ on
\textsf{2D CCNTCM}.
\label{lem:a}
\end{lemma}

\begin{proof}
We modify the QCLA adder from
Section \ref{subsec:qcla} mapped to \textsf{2D CCNTCM} using
Lemma \ref{lem:ac-ccntcm}.
To add $2 \times n$-bit numbers, the original \text{AC} adder has
depth $O(\log n)$ and size and width of $O(n)$.
Therefore, we have the desired resources for a \textsf{2D CCNTCM} circuit.
\end{proof}

\begin{lemma}
The operator $W\ket{x_1}\ket{x_2}\cdots\ket{x_m} \rightarrow \ket{x_1}\ket{x_2}\cdots\ket{x_1 + x_2 + \ldots + x_m \bmod 2^n}$,
which operates on $m \times n$-qubit registers, can be
implemented in circuit/module depth $O(\log n + \log nm)$, circuit size $O(mn)$, 
circuit width $O(mn + n\log n)$, module size of $O(mn + n \log n)$, and
module width of $O(m + \log n)$ on
\textsf{2D CCNTCM}.
\label{lem:w}
\end{lemma}

\begin{proof}
This reduces to the problem of modular multiple addition from
Section \ref{subsec:mma}, with the following variation. Instead of
addition modulo $m < 2^n$, we perform addition modulo $2^n$, which is
much simpler. It does not involve any truncation and adding back of
modular residues. Namely, we just perform one round of constant-depth
$3 \rightarrow 2$ addition, then we redo the computation of the
highest-order bit $v_n$ to uncompute it.

Therefore, we have a $O(\log m)$-depth binary tree of $O(m)$ modules,
which requires a total of $O(mn)$ teleportation of qubits between them.
Within each module, the addition takes depth $O(1)$ and size and width
of $O(n)$. Combine this in series with the circuit for $A$ in
Lemma \ref{lem:a} and we have the desired resource bounds.
\end{proof}

Armed with these properties, we're now ready to describe our version of the
KSV phase estimation procedure to product quantum Fourier states.

\section{Circuit Resources for the KSV Quantum Compiler}
\label{sec:qcompile-ksv}

This section gives a pedagogical review of the quantum compiling method
that combines phase kickback and QFS generation due to
Kitaev-Shen-Vyalyi \cite{Kitaev2002}. In this form, it compiles single-qubit
of the form $R_Z(\phi)$. Using a multi-qubit
decomposition such as those presented in Section \ref{subsec:qcompile-multi},
it can be extended to compile multi-qubit gates.
Furthermore, we present an
original optimization
called \emph{early measurement} which does not asymptotically increase
our compilation resources (see Appendix \ref{app:ksv-error}). We refer to
our optimized method as PK-KSVe, which still shares much in common with
PK-KSV up to the phase estimation step.
Finally, we contribute the architectural resources on $\textsf{2D CCNTCM}$ for running 
three different QFS generation.

First, in Section \ref{subsec:ksv-diff} we discuss our optimization of early measurement, which distinguished
PK-KSVe from the original PK-KSV algorithm.
Next,
in Section \ref{subsec:ksv-steps} we present the high-level overview
of the KSV algorithm. In Section \ref{subsec:ppe} the most important
and most resource-intensive step in PK-KSV and PK-KSVe.
Section \ref{subsec:ksv-classical} describes the classical post-processing
which completes phase estimation. While this is done on a classical computer,
we will discuss how this step affects our parameters and resources
for generating a quantum circuit. We also discuss a new, faster
phase estimation algorithm \cite{Svore2013}. Finally, we map the QFS
stage of PK-KSV,
PK-KSVe, and PK-Jones to \textsf{2D CCNTCM} and compare their
asymptotic resource usage in Section \ref{subsec:ksv-compare}.

%%%%%%%%%%%%%%%%%%%%%%%%%%%%%%%%%%%%%%%%%%%%%%%%%%%%%%%%%%%%%%%%%%%%%%%%%%%%%%
\subsection{The Optimization of Early Measurement}
\label{subsec:ksv-diff}

In this section, we describe the differences between PK-KSV, as originally
presented in Section 13 of \cite{Kitaev2002} and PK-KSVe, which includes our
novel contribution of early measurement.

Given a circuit $C$ to compile, we
precompile it into gates from $\mathcal{G} \cup \{R_Z(2\pi x / 2^n)\}$
using the results from Sections \ref{subsec:qcompile-single} and
\ref{subsec:qcompile-multi} in $O(1)$ time, depth, and size.
Now we are done with the single-qubit gates and CNOT, and we have computed
the values $\{x_1, \ldots , x_m\}$ that allow us to approximate our
desired $m$
$R_Z(\phi)$ gates as $\phi \approx \frac{\pi l}{2^n}$ to within precision
$2^{-n}$. We now need to generate the QFS $\ket{\psi^{(k)}}$ to use with
phase kickback to either approximate $R_Z(\phi)$ directly, or to generate
PAR qubits as intermediaries for later enacting $R_Z(\phi)$ probabilistically.

The original PK-KSV procedure first created the state $\ket{\psi^{(0)}}$
and applied phase estimation to it to enact the operator $\Upsilon(e^{-2\pi i / 2^n})$
and get $\ket{\psi^{(1)}}$ as a result. This state can be copied using
$W$: multiple addition modulo $2^n$. However, it involves using coherent measurement
in order to enact a phase $\ket{x} \rightarrow e^{2\pi i x / 2^n}\ket{x}$ on
all the components of $\ket{\psi^{(0)}} = \sum_{x = 0}^{2^n} \ket{x}$.

The PK-KSVe does not enact the operator $\Upsilon(e^{-2\pi i / 2^n})$ but instead
yields a QFS $\ket{\psi{(k)}}$ for random odd $k$, as well as the index $k$ itself.
This can be used to solve the modular inverse equation classically:

\begin{equation}
kp \equiv x \bmod 2^n
\end{equation}

to get the solution $p = p(k,x)$.
 This integer 
$p$ is added to $\ket{\psi^{(k)}}$, controlled on some qubit $\ket{phi}$
to get the desired phase shift: $R_Z(2\pi x / 2^n)\ket{\phi}$.
To copy $\ket{\psi^{(k)}}$, consider the controlled addition operator $\Upsilon(A)_n$
on two $n$-qubit registers.

\begin{equation}
\Upsilon(A)_n \ket{x}\ket{y} \rightarrow \ket{x}\ket{y + x \bmod 2^n}
\end{equation}

Applying this operator to two QFS's has the following interesting property
as noted in \cite{Jones2013}.

\begin{equation}
\Upsilon(A)_n \ket{\psi^{(k')}}\ket{\psi^{(k)}} \rightarrow \ket{\psi^{(k'-k)}}\ket{\psi^{(k)}}
\end{equation}

Therefore, we can copy a state $\ket{\psi^{(k)}}$ in the second register by
putting $\ket{\psi^{(0)}}$ in the first register to get $\ket{\psi^{(-k)}} = \ket{\psi^{(2^n - k)}}$.
This is not exactly the same state as $\ket{\psi^{(k)}}$, but it can similarly be used
with solving modular inverse to get an $x'$ and enact a desired rotation $R_Z(2\pi x' / 2^n)$.

%%%%%%%%%%%%%%%%%%%%%%%%%%%%%%%%%%%%%%%%%%%%%%%%%%%%%%%%%%%%%%%%%%%%%%%%%%%%%%
\subsection{PK-KSVe Steps}
\label{subsec:ksv-steps}

Here now are the main steps of PK-KSVe.

The first stage involves QFS generation.

\begin{enumerate}
\item Create the equal superposition of all $\ket{\psi^{(k)}}$ for odd $k$:
\begin{equation}
\ket{nu} = \normtwo \left( \ket{2^{n-1}} + \ket{0} \right)
\end{equation}
in register 1.
\item Use phase estimation with error $\epsilon = 2^{-3n}$ and early
measurement to get classical Bernoulli series outcomes $y$ in register 2
and a corresponding QFS $\ket{\psi^{(k)}}$. This takes a circuit of
depth $O(\log n)$ and size $O(n^2)$ (see Section \ref{subsec:ppe}).
\item Perform classical post-processing from Section \ref{subsec:ksv-classical} to
recover the phase $k$ by which can identify the state $\ket{\psi^{(k)}}$.
\end{enumerate}

Now we have $\ket{\psi^{(k)}}$, or a state that has high overlap with it
(calculated in Section \ref{subsec:ksv-compare}).
We can copy it $m$ times using multiple addition $W$ in depth
$O(m \log n)$ and size $O(mn)$. This is part of any complete
quantum compiler solution for PK-KSVe, but we omit it here and
focus on QFS generation.

%%%%%%%%%%%%%%%%%%%%%%%%%%%%%%%%%%%%%%%%%%%%%%%%%%%%%%%%%%%%%%%%%%%%%%%%%%%%%%
\subsection{Parallelized Phase Estimation}
\label{subsec:ppe}

Parallelized phase estimation is the central component of both
PK-KSV and PK-KSVe which distinguishes it from all other quantum
compiling approaches. It is used to randomly ``pick'' an
eigenstate $\ket{\psi_k}$ of a unitary operator $U$ and
measure its corresponding eigenvalue (phase) $\phi_k$
with some degree of precision $\delta = 2^{-n}$ and
error probability $\epsilon = 2^{-l}$. As $n$ increases, the phases
generally become closer together, which is why we need exponential precision
to distinguish between them.

\begin{displaymath}
U\ket{\psi_k} = e^{2\pi i \phi_k} \ket{\psi_k}
\end{displaymath}

Phase estimation holds some superposition of eigenstates
$\sum_{i} \alpha_i \ket{\psi_i}$
in an $n$-qubit target register, to which it applies repeated measuring
operators $\Lambda(U^{2^k})$
controlled on some $t$-qubit register, which holds an
approximation $\tilde{\phi}$ to the real phase $\phi$.
The unitary $U$ is applied in successive powers of two to get
power-of-two multiples of the phase for increased precision.
The error probability of approximating the phase to within a given
precision is given by the following:

\begin{displaymath}
\Pr\left[ | \phi - \tilde{\phi} | \ge \delta \right] \le \epsilon
\end{displaymath}

The parameter $t = t(\delta, \epsilon)$ encodes the dependence of the number
of $\Lambda(U)$ measuring operators as a function of our desired
$\delta$ and $\epsilon$.
It varies according to the exact phase estimation procedure
used.

The popular version of phase estimation presented in \cite{Nielsen2000},
requires $t$ repeated controlled applications of some unitary
$U$ (and its successive powers as $U^{2^k}$, $0<k<2^t$)
to a target state which holds some superposition of its eigenvectors,
controlled by $t$ bits which will hold the approximation to a corresponding
eigenvalue (phase).
This version requires applying an inverse quantum Fourier
transform (QFT).

To achieve our desired low-depth, we can ``parallelize'' the application of
$\Lambda(U)$ by interpreting the
$t = (n+2)s$ control bits as an $n$-bit number $q$ and
apply $\Upsilon(A)\ket{q}$ only once.
Recall that $A$ is the addition operator on an $n$ qubit
target register containing $\ket{\psi^{(k)}}$, so we can
only effectively add the lowest $n$ bits of $q$.
Furthermore, the eigenvalues of $A$ are rational with a fixed
denominator, $\phi_k = k / 2^n$.
To avoid the inverse QFT, which often has small rotations of the form 
$R_Z(\phi)$, 
we can do a classical post-processing step as described in Section \ref{subsec:ksv-classical}.

The main steps in parallelized phase estimation as applied to PK-KSV
are as follows. As our most important input parameter, we calculate $t$, the number
of measurements that we need (and the number of control qubits that we have).
For QFS generation, it suffices to determine the phase with resolution $\frac{1}{2^{n+2}}$,
where each bit of the phase is determined with a series of Bernoulli (biased coin flips)
of $s$ trials each. As in the original PK-KSV, we have two sets of measurements,
one to measure $\phi$ as a bias projected on the real axis of the unit circle (called
the cosine measurements), and one to perform the same measurements rotated by $\pi/2$
(called the sine measurements). These are illustrated below:

\begin{equation}
\Pr(1|k) = \frac{1 + \sin(2\pi\phi_k)}{2} \qquad \Pr(0|k) = \frac{1 + \cos(2\pi\phi_k)}{2}
\end{equation}

Therefore, we have:

\begin{equation}
t = 2(n+2)s
\end{equation}

\begin{enumerate}

\item Begin with a $t$-qubit ``phase'' register initialized to $\ket{0}^{\otimes t}$.
Recall that we have an $n$-qubit ``QFS'' register initialized to $\ket{\nu}$, the
equal superposition of $\ket{\psi^{(k)}}$ for all odd $0 < k  < 2^n$.
%\textsc{Resources} $= [0,0,0,0,0,t]$

\item Place the $t$-qubit register into an equal superposition by
applying $n$ Hadamard gates.
%\textsc{Resources} $= [0,0,0,n,1,0]$

\item Treat $t$ as $2s$ groups of bits, each encoding an $(n+2)$-bit number
which we call $\ket{q_i}$.
This does not involve any addition or other operation, it just determines
how we interpret its measurement outcome after we projectively measure it later.
%Sum them up out-of-place, retaining only the lowest $n$-bits,
%to get the superposition
%of all $n$-bit numbers, $1/(\sqrt{2^{n}}) \sum_{i=0}^{2^n-1} \ket{q_i}$.
%Call this register $\ket{q_i}$.
%\textsc{Resources} $= ADD-OUT(2s \times n)$

\item Apply the gate $\Upsilon(A)_{n+2}$ to a target ancilla register $\ket{h_i}$ controlled
on $\ket{q_i}$, which we interpret as an $(n+2)$-bit number. The register
$\ket{h_1}$ is initialized to all $\ket{0}$'s. This can be done in
parallel for all $0 \le i < 2s$ using constant-depth $3\rightarrow 2$ addition.
Now we have $2s$ quantum integers, plus
the original register $\ket{\nu}$, that we add down using modular multiple
addition in logarithmic depth, including the final addition of a CSE number
down to a conventional number using QCLA \cite{Draper2004}. We use the adders described
in Section \ref{subsec:qfs-adder} on \textsf{2D CCNTCM}, uncomputing all
other intermediate adder qubits back to $\ket{0}$
so that we are only left with $\ket{\nu}$.
%\textsc{Resources} $= ADD-IN(2 \times n)$

\item Reverse the second step by applying another $n$ Hadamards.
%\textsc{Resources} $= [0,0,0,n,1,0]$

\item Projectively measure all $t$ control
qubits in the phase register. These qubits
now contain classical $0$ or $1$ as outcomes of $(n+2)s$ Bernoulli trials.
The target $\ket{nu}$ now contains a QFS $\ket{\psi^{(k)}}$ for some
as yet unknown $k$ with high fidelity.

\item Read out these outcomes into our classical controller
and perform the post-processing in Section \ref{subsec:ksv-classical}.
We get an approximation of $\phi$ with precision $\delta$ and
success probability $1-\epsilon$.

\end{enumerate}

%%%%%%%%%%%%%%%%%%%%%%%%%%%%%%%%%%%%%%%%%%%%%%%%%%%%%%%%%%%%%%%%%%%%%%%%%%%%%%
\subsection{Classical Postprocessing}
\label{subsec:ksv-classical}

It is now the point to mention that Kitaev's phase estimation procedure
contains a post processing step which is completely classical in
character, in that they involve a measurement. If this measurement is
projective and the outcomes are completely classical, the remaining steps
can be done on our classical computer,
and the results fed back into our quantum algorithm to perform controlled
addition (phase kickback). Therefore, as long as we can perform these classical
algorithms in polynomial time (which we can), we don't really care
about the equivalent circuit size and depth.

The steps of classical postprocessing, which will determine some of the
parameters in the earlier, quantum part of phase estimation are as follows.

\begin{enumerate}

\item
Estimate the phase and its power-of-two multiples
$2^j \phi_k$ to
some constant, modest precision $\delta''$, where
$0 \le j < (n+2)$. For each $j$, we
apply a series of $s$ measuring operators targeting the state $\ket{\nu}$
controlled on $s$ qubits in the state $(\ket{0}+\ket{1})/\sqrt{2}$,
essentially encoding the $2^j \phi_k$ as a bias in a coin, and flipping the
coin $s$ times in a Bernoulli trial, counting the number of $1$ outcomes,
and using that fraction to approximate the real $2^j \phi_k$.
\item
Sharpen our estimate to exponential precision $\frac{1}{2^{n+2}}$ using the
$(n+2)$ estimates, each for different bits in the binary expansion of
$\phi_k$. Multiplication by successive powers-of-two shift these bits
up to a fixed position behind the zero in a binary fraction representation,
where we can use a finite-automata and a constant number of
bits to refine our $O(n)$-length running approximation.
\end{enumerate}

Three things are worth mentioning about the interrelation of the parameters
between these two steps. Since our phases all have a denominator of $2^n$,
there is no need to run the continued fractions algorithm on multiple
convergents, as is the case with period-finding in Shor's factoring algorithm.
Furthermore, the phases are $1/2^n$ apart, therefore it suffices to approximate
the phases to within $1/2^{n+2}$ in order to break ties, which is where
our range for $j$ comes from above.

The number of trials $s$ comes from the Chernoff bound:

\begin{displaymath}
\Pr \left[ | s^{-1}\sum_{r=1}^s v_r - p_* | \ge \delta'' \right]
\le 2e^{-2\delta'^2 s}
\end{displaymath}

Setting this equal to the desired error probability $\epsilon$ we get

\begin{displaymath}
s = \frac{1}{2\delta''^2}\ln \frac{1}{\epsilon}
\end{displaymath}

We are actually estimating the values $\cos(2\pi \cdot 2^j \phi_k)$ and
$\sin(2\pi \cdot 2^j \phi_k)$, so if we wish to know $2^j \phi_k$ with
precision $\delta''$, we actually need to determine the $\cos(\cdot)$ and
$\sin(\cdot)$ values with a different precision $\delta'$, lower-bounding
it with the steepest part of the cosine and sine curves.

\begin{displaymath}
\delta' = 1 + cos(\pi - \delta'')
\end{displaymath}

It is possible to improve this bound by adding a bias angle of $\pi/4$
before measurement, and then subtracting it from the recovered phase
angle after classical post-processing. This bias angle is
inspired by the SHF phase estimation \cite{Svore2013}. Then we have
$\delta'' \approx \delta'$.

The factor $\frac{1}{2\delta''^2}$ depends on the constant precision with
which we determine our $2^j \phi_k$ values. Since classical time is
cheap and quantum gates are expensive, it makes sense to minimize the number
of trials $s$. Table \ref{tab:ksv-parameters} shows the corresponding values of $1/(2\delta''^2)$
and $\delta'$ as a function of various choices for $\delta'$.

\begin{table}
\centerline{
\begin{tabular}{|c|c|c|c|}
\hline
$\delta''$ & $1/(2\delta''^2)$ & $\delta'$ & $1/(2\delta'^2)$ \\
\hline
$1/16$     & $128$             & $0.0019525$ & $131,160$\\
$1/8$      & $32$              & $0.0078023$ & $  8,213$\\
$1/4$      & $8$               & $0.0310880$ & $    517$\\
\hline
\end{tabular}
}
\caption{Parameters for the number of measurements in PK-KSV.}
\label{tab:ksv-parameters}
\end{table}

By making our $\delta'$
exponentially worse (doubling it) we are only increasing the range of
$j$ a linear amount (by one). In general, for $\delta'=\frac{1}{2^l}$, we get
a final estimate for $\phi = 2^{m-3}$

Projective measurements are irreversible, and it is not so important that
we are left with (classical, unentangled) garbage in our $t$-qubit ancillae register.
After all, we only run phase estimation once
to get our initial $\ket{\psi^{(k)}}$ state.
The reason why
the authors of \cite{Kitaev2002} go to some care to show that all the classical
postprocessing steps can be done in polynomial-size and logarithmic-depth
is that these must be done to a quantum state $\ket{\psi^{(0)}}$ in order
to turn it into $\ket{\psi^{(1)}}$ in original PK-KSV. However, our new procedure
sidesteps this requirement, so we are free to offload this processing to a
classical controller, which is available in \textsf{2D CCNTCM}.

Since we have seen KSV-style phase estimation in
Chapter \ref{chap:factor-polylog}, it is important to mention here several
differences between applying phase estimation to factoring versus
applying it to QFS generation. The first difference is that a constant
success probability of $\frac{3}{4}$ is no longer good enough. We would
like to drive down our error exponentially low as $\epsilon = 2^{-l}$.
Second, the operator whose phase we are estimating is now
addition modulo $2^n$, which is more efficient in circuit resources
than modular multiplication (see Section \ref{subsec:qfs-adder}).

In the procedure above, we measured power-of-two multiples of the phase
$2^{j}\phi_k$ to recover single bits of $\phi_k$ at a time.
However, a remarkable recent result by Svore-Hasting-Freedman (SHF) shows
how to use information theoretic and signal processing techniques to
improve the number of measurements needed from $O(n^2)$ as in conventional
KSV to $O(n\log^{*}n)$.
Their key technique is to select random measurements of this form:

\begin{equation}
(2^{j_1}+2^{j_2}+\ldots + 2^{j_S})\phi_k
\end{equation}

This would allow us to recover multiple bits at a time.
SHF empirically determines that the number of trials $s$ in
each Bernoulli series scales as $O(\log n)$ for $n = {1000,10000}$,
on the same order of magnitude for running factoring on
keys of 2048 to 4096 bits.
Furthermore, they extend this approach to multiple rounds. Given
that the number of measurements in the final round is
$s = O(\log n)$ and that each round requires logarithmically
fewer measurements than the previous round, they
achieve the iterated-logarithmic depth above.
These techniques could be used to improve PK-KSV, but we
remain conservative for now until numerical upper bounds can be
calculated.

%%%%%%%%%%%%%%%%%%%%%%%%%%%%%%%%%%%%%%%%%%%%%%%%%%%%%%%%%%%%%%%%%%%%%%%%%%%%%%%%%%%%%
\subsection{Resource Comparisons}
\label{subsec:ksv-compare}

We now map both the KSV and Jones algorithms for QFS generation
to \textsf{2D CCNTCM} and compare their resources. We do not consider
the phase kickback part of the quantum compiler, since this controlled addition
is the same for both QFS algorithms. Like PK-Jones \cite{Jones2012}, we
set our error as $\epsilon = \left(\frac{\pi}{2^n}\right)^2$. It is not clear why
this error was chosen, since for factoring or other polynomial-sized circuits,
an inverse-polynomial error precision is sufficient. However, for the
purpose of comparison, we set the same error.

The condition for error for PK-Jones is one minus the overlap between a
pure Fourier state $\ket{\psi}^{(1)}_n$ and a distilled
Fourier state $\ket{\tilde{\psi}}^{(1,m)}_n$ after $m$ rounds.

\begin{equation}
1 - | \braket{\tilde{\psi}^{(1,m)}_n}{\psi^{(1)}} |^2 = \epsilon \le \left( \frac{\pi}{2^n} \right)^2 
\end{equation}

The condition for error in PK-KSV is that an incorrect phase indexed by $k$ is returned,
one that does not correspond to the eigenstate $\ket{\psi}^{(1)}$. The target
register for phase estimation begins in a superposition of eigenstates
$\ket{1} = \sum_{s = 1}^{2^{n-1}} \ket{\psi^{(2s-1)}_n}$, and does not deteriorate
since none of the measuring operators mix between the orthogonal decomposition of the
QFS basis. We make use of Theorem \ref{thm:projective} in Appendix \ref{app:ksv-error},
where the measured index $y$ corresponds to a subspace $\mathcal{M}_y$ of all Bernoulli series outcomes $\Delta$ that would
correspond to the phase $k = (2s-1)$, and the state left in $\mathcal{N}$ is a (possibly impure)
QFS $\mathcal{\tilde{\psi}^{(k)}}$ which has large overlap with $\mathcal{\psi^{(k)}}$.
We take the set of $\Omega$ of subspaces to be the odd QFS indices $\{1, 3, 5, \ldots 2^n - 1\}$
of which there are $2^{n-1}$ elements. Taking our phase estimation error to be $\epsilon \left( \frac{1}{2^{3n-1}} \right)$,
we then have the following inequality.

\begin{equation}
1 - | \braket{\tilde{\psi}^{(y)}}{\psi^{(k)}} | = \sum_{s = 1}^{2^{n-1}} \sum_{y \in \Delta: f(j) \ne (2s-1)} \Pr(2s-1|y) \le |\Omega|\epsilon = \frac{1}{2^{2n}} \le \left( \frac{\pi}{2^n} \right)^2 
\end{equation}

Several differences exist in resource
calculation methods and those of PK-Jones \cite{Jones2012}. First, that
author calculates expected resources, since the distillation method is
probabilistic and involves post-selection. Our resources are worst-case,
with still a small probability of failure calculated to match the
case of PK-Jones with all successful post-selection. We also compare our
improvements with those of the original PK-KSV algorithm, for the QFS generation stage.
These are given in Table \ref{tab:ksv-resources}.

\begin{table}[hbt!]
\begin{tabular}{|c|c|c|c|}
\hline
Resource       & PK-KSVe     & PK-KSV         & PK-Jones\\
\hline
$D$            & $O(\log n)$ & $O(\log^2 n)$  & $O(\log^2 n)$ \\
$S$            & $O(n^2)$    & $O(n^2\log n)$ & $O(n \log n)$ \\
$W$            & $O(n^2)$    & $O(n^2)$       & $2n + O(1)$ \\
$\overline{D}$ & $O(\log n)$ & $O(\log^2 n)$  & $O(\log n)$ \\
$\overline{S}$ & $O(n^2)$    & $O(n^2)$       & $O(n^2)$ \\
$\overline{W}$ & $O(n)$      & $O(n)$         & $O(n)$ \\
\hline
\end{tabular}
\caption{A comparison of circuit resources for QFS generation for PK-KSVe, PK-KSV, and PK-Jones on \textsf{2D CCNTCM} for error $\epsilon \le \left( \frac{\pi}{2^n} \right)^2$.}
\label{tab:ksv-resources}
\end{table}

A useful extension of this work would be to calculate numerical upper bounds and compare the
two QFS methods for an application, such as Shor's factoring algorithm for realistic key sizes of
2048-4096 bits.

\section{Single-Qubit Rotations for Quantum Majority Gate}
\label{sec:qcompile-maj}

We now present a result delayed from Chapter \ref{chap:factor-sublog},
the quantum compiling procedure which completes the quantum majority gate.
Recall from that chapter that within each quantum majority gate, we
needed to implement single-qubit rotations of the form
$2\pi / k$, where $k = poly(n)$, where $n$ is the input size of the
number for factoring. We can augment any quantum
majority circuit with quantum compiler modules that produce
rotated ancillae of the form $\normtwo(\ket{0} + e^{i\phi}\ket{1})$.

To maintain our sub-logarithmic depth, we choose the KSV method for
generating quantum Fourier states and combine it with phase kickback
(PK-KSV).
The precision for quantum compiling is $k = 2^{-\eta}$, and the
resulting resources for applying PK-KSV on \textsf{2D CCNTCM}
in this case are
given in Table \ref{tab:pk-ksv-resources}, based on calculations
from Section \ref{sec:qcompile-ksv}. Note that these resources
are deterministic, since they represent the worst case for any
single rotation $R_Z(e^{i\phi})$. To convert from $\eta$ to $n$,
we use the relationship $\eta = O(\log k) = O(\log n)$.
These resources are for the generation of a single QFS, which is
reusable for creating PAR qubits one-at-a-time. To create
$m = poly(n)$ $R_Z(e^{i\phi})$ rotations for an entire
quantum majority circuit, we would need to copy the first
QFS $m$ times using the $W$ operator from Lemma \ref{lem:w}.
This would take depth $O(\log m) = O(\log n)$,
which is not sub-logarithmic. Therefore, it is more
depth-efficient to create a new QFS from scratch for each
PAR qubit needed, in parallel.

We note that it is possible to modify PK-Jones to have
similar sub-logarithmic depth for QFS generation if a
constant-depth adder were used.

\begin{table}[hbt!]
\begin{center}
\begin{tabular}{|c|c|}
\hline
$D'$ & $O((\log \log n)^2)$ \\
$S'$ & $O(\log^2 n)$ \\
$W'$ & $O(\log^2 n)$ \\
\hline
\end{tabular}
\caption{Quantum compiling resources for PK-KSV for quantum majority gates
in factoring an $n$-bit number.}
\label{tab:pk-ksv-resources}
\end{center}
\end{table}

Given such a factory for producing such ancillae, we can compile the
rotations $R_Z(2\pi / m)$, for $m = poly(n)$ as in Theorem \ref{thm:maj-gate},
by directly using the KSV compiler with parameters $\eta = \log n$ as in the
previous table.

We now present a more general result for
producing rotations which are integer multiples of $2\pi / m$ 
using a \emph{finite}
basis in constant depth and polynomial size.
This is not necessary for our sub-logarithmic factoring implementation,
since we can always produce our desired angles of $2\pi / m$ directly
with our quantum compiler.
In fact, no polynomial-size quantum circuit will require more than
polynomial precision for compiling single-qubit rotations. However, we
present our result in the hopes that it will be useful for other
quantum algorithms, perhaps one where we must produce the
PAR qubit $\normtwo ( \ket{0} + e^{2\pi / m}\ket{1} )$ ``offline'' and then produce
the rotation $R_Z(2\pi k / m)$ ``online.''

\begin{theorem}{\textbf{Compiling a single-qubit rotation over a non-fixed, finite basis.}}
The single-qubit rotation $R_Z(2\pi k /m)$, where $m = poly(n)$,
$k \in \mathbb{Z}_m$,
can be implemented in expected depth $O(1)$ and expected size and width $O(k)$ on
\textsf{2D CCNTCM} over the finite
(but not fixed) basis $\mathcal{G} \cup \{R_Z(2\pi / m)\}$.
\label{thm:qcompile}
\end{theorem}

\begin{proof}
We use the quantum parallelism method of Hoyer-Spalek \cite{Hoyer2002},
which relies on quantum fanout and unfanout on \textsf{2D CCNTCM}.
Our use of the basis $\mathcal{G} \cup \{R_Z(2\pi / m)\}$ implies that
we have access to quantum compiler modules for producing the
PAR qubits $\normtwo ( \ket{0} + e^{2\pi / m}\ket{1} )$. Teleport $O(k)$ such qubits
into our current circuit.
Our desired rotation of $R_Z(2\pi k / m)$ on a target qubit $\ket{\psi}$
can be produced as $k$
parallel applications of $R_Z(2\pi / m)$, which are already diagonal in
the same (computational) basis. Fan out the qubit $\ket{\psi}$ $k$ times,
apply the rotations $R_Z(2\pi /m)$ to each fanned out qubit in parallel
using the PAR procedure, then unfanout the
qubits.
This requires expected $O(k)$ PAR qubits.
\end{proof}

We note here two possible conjectures for improving the above result.
The first would allow us to achieve $O(\log m)$
expected size, expected width, and
expected number of teleported PAR qubits. The second would allow us to
compile arbitrary rotations to a basis that is both fixed and finite
in constant depth.

\begin{conjecture}{\textbf{Logarithmic Reduction of Compiling Circuit Size and Width.}}
Theorem \ref{thm:qcompile} can be accomplished in expected size
and width $O(\log n)$ rather than $O(n)$.
\end{conjecture}

The size and width of the above circuit depend on whatever additional,
finite, set of 
gates
$\{ R_Z(\phi_{k_i}) \}$ used to augment the usual \textsf{2D CCNTCM} basis
$\mathcal{G}$. Let $\phi_{k_i} = 2\pi k_i / m$, then the size and width of
a circuit applying \emph{only} $R_Z(\phi_{k_i})$
are proportional to the order of $k_i$ in
$\mathbb{Z}_m$, or equivalently, the number of times we must apply
the rotation $\phi_{k_i}$ in parallel to equal the desired rotation
$\phi_k$. Suppose we are able to find a Chinese Remainder number system
for $m$, that is, a set of pairwise coprime numbers $\{m_1, \ldots m_{t}\}$
such that $m = \prod_{i=1}^t m_i $, where $t$ and the number of bits
needed to encode each $m_i$ are $O(m)$ \cite{Yeh1996}.
The Chinese Remainder representation of $k$
is the set of $(\log_2 m)$-bit numbers
$x_i = k \bmod m_i$. 
Then we conjecture that
the finite basis $\mathcal{G} \cup \{R_Z(2\pi x_i / m\}$ satisfies the
properties above.

\begin{conjecture}{\textbf{Constant Reduction of Compiling Depth.}}
Theorem \ref{thm:qcompile} can be accomplished using a fixed,
finite basis in constant depth.
\end{conjecture}

The above bases are finite but still depend on the problem input
size $n$. It may be possible to find a basis that
is both fixed and finite that would allow for compiling
arbitrary single-qubit rotations in constant depth and polynomial
size and width, still to precision $1 / poly(n)$. This fixed
basis would be ``polynomially universal'' in that it would be
the same for all inputs of any size.
We conjecture that products of single-qubit gates in
$\mathcal{G}$ which, when diagonalized, represent $R_Z$ rotations
of irrational multiples of $\pi$, would form such a basis.

% Old stuff, from old intro, maybe we can fit it into here to add color
In digital computing, the boundary between architecture and compilers is quite porous and is determined by a processor's instruction set. Architecture studies processor resources to solve an algorithm given a particular instruction set which is fixed in hardware. This instruction set is produced by a compiler, a piece of (low-level) software which transforms over pieces of (high-level) software. This instruction set can change based on which algorithms it allows to solve efficiently as well as which processors it allows to manufacture efficiently as well as which operations it allows humans to understand easily. All of these factors combine to make architecture an art and an engineering discipline rather than merely a science.

Quantum computers make this problem even more difficult due to the nature of a quantum bit. Because transformations between quantum states vary continuously over the space of unitary matrices with complex coefficients, we can only approximate desired quantum logic gates using a fixed set, given to us by fault-tolerance.


%\section{Conclusion}
\label{sec:qcompile-conclude}

In this chapter, we examined quantum compiling as a necessary
puzzle piece to complete our previous factoring architectures
as well as
an active field of research in its own right. This chapter
combined a pedagogical review at the beginning of quantum
compiling in general and phase kickback quantum compiling
using quantum Fourier states (QFS) in particular. It then ended
with new, more depth-efficient results in QFS generation.

Expanding beyond single-qubit gates and circuit bases, we
studied the main themes of quantum compiler research,
including multi-qubit compiling, exact versus approximative,
deterministic versus randomized, and provably-efficient versus
conjectured-efficient. We provided a wide-ranging resource
comparison
of the current state-of-the-art in single-qubit
compiling. We then explicated in detail QFS generation
using the Kitaev-Shen-Vyalyi (KSV) method of parallelized
phase estimation with classical postprocessing. We called
this method PK-KSV.

In our main contribution, we presented
an optimized version of PK-KSV using early measurement
which we called PK-KSVe. We contributed depth-efficient
adders on \textsf{2D CCNTCM} for any phase kickback
compiler in Lemma \ref{lem:w}. Then we compared the resources required
for three different phase kick compilers,
PK-KSV, PK-KSVe, and a method by Jones \cite{Jones2013}
which we call PK-Jones. Our optimized approach
PK-KSVe has the lowest depth of these approaches.
Finally, we contribute a method for completing
our factoring architecture from the previous chapter
while maintaining sub-logarithmic depth in
Theorem \ref{thm:qcompile}.

This chapter has also raised interesting open question for
future research.
Among the quantum compiling themes of Section \ref{sec:qcompile-bg},
we could form a
fifth axis which is whether the compiled circuit obeys (hybrid) nearest-neighbor
constraints or not. Such compilers would be judged based on how well
they partitioned an input circuit to have an optimal
number of modules on \textsf{2D CCNTCM} to also minimize module depth and
module size (inter-module teleportations).
When comparing existing related works in Section \ref{sec:qcompile-review},
we did not measure classical space requirements, although these may be
exponential. This would be a useful metric for future comparison.
To extend
the comparison of PK-KSV(e) and PK-Jones in Section \ref{sec:qcompile-ksv},
one could calculate numerical upper bounds
for performing QFT rotations sufficient for
Shor's factoring algorithm on realistic key sizes of
2048-4096 bits. Finally, in Section \ref{sec:qcompile-maj},
we presented two conjectures which have deep implications for
depth-efficient fault-tolerance. 
 
% ========== Chapter 4

\chapter{Quantum Circuit Coherence}
\label{chap:coherence}

In the first two chapters, we presented low-depth nearest-neighbor
architectures for factoring an $n$-bit number. We improved the depth first to
be sublinear and then sublogarithmic, but at a polynomial increase in size
and width. This represents a time-space tradeoff which can be upper-bounded
by the product of the circuit depth and circuit width. Although the title of
this dissertation indicates that it is beneficial to decrease depth, in
experimental implementations, other practical considerations may constrain
our depth reduction.

Toward this end, we introduce a new circuit resource call \emph{coherence}
which quantifies the amount of error-correction that must be performed
to maintain a coherent quantum computing state. This can be analogous to
the amount of classical controller time or electrical power that a
quantum computing experiment consumes while running an algorithm. We
define our new resource in Section \ref{sec:cohere-def}, discuss its
relationship to other circuit resources on \textsf{2D CCNTCM}, and
calculate coherence for some simple examples to illustrate how it
captures the notion of parallelizability for a circuit.

Since circuit coherence measures a time-space tradeoff in order to
quantify circuit parallelism, it is natural to
compare it to other resources that are measured in the literature.
This is done in Section \ref{sec:coherence-tradeoff}, where we
especially review the model measurement-based quantum computing and
how it relates to circuit coherence.

In Section \ref{sec:cohere-factor}, we calculate the circuit coherence
for two arithmetic building blocks useful in factoring: addition
and multiplication. We use this to draw some conclusions about the
usefulness of decreasing the depth, and making the circuit depth configurable
so that an experimental architect can make the appropriate choice.

Finally, having exhausted our fascination with factoring,
we apply our low-depth techniques (compiling and circuit coherence)
to a new quantum algorithm: hamiltonian
simulation. In Section \ref{cohere-hs-bg}, we discuss the background of
this problem, and in Section \ref{cohere-hs-calc} we parallelize one
aspect of it, that of decreasing the depth of simulating
a $1$-sparse Hamiltonian matrix.

\section{Definition of Circuit Coherence}
\label{sec:cohere-def}

Usually quantum circuits neglect to draw identity gates. When a bare
quantum wire appears, what is meant is that the qubit maintains its
coherent state until the next non-identity gate comes along to transform it.
However, most quantum circuits are drawn at a logical qubit level,
assuming no errors occur and a coherent state is maintained.
So far we have maintained that abstraction in this dissertation by studying
quantum compiling and quantum architecture independently from
quantum error correction. In this chapter, we move closer to
this abstraction barrier by studying
the effort to maintain a coherent quantum state at the logical qubit level
only for those qubits and only during those timesteps which are useful
for computation.
We quantify this effort with a new circuit resource called \emph{circuit coherence}.

First, we will define what we mean by an \emph{influencing} gate and
a \emph{qubit instance}
in Section \ref{subsec:cohere-entangle}. Then we will build upon this
to define a computational subset of qubits in every
timestep, in Section \ref{subsec:cohere-subset}. Finally,
in Section \ref{subsec:cohere-algo}, we will use
the previous two definitions to provide an algorithm for
computing reachability and therefore the computational subset for a circuit.
This in turn lets us define the resource circuit coherence and
describe its relationship to the other circuit resources: depth, size, and
width. 

%%%%%%%%%%%%%%%%%%%%%%%%%%%%%%%%%%%%%%%%%%%%%%%%%%%%%%%%%%%%%%%%%%%%%%%%%%%%%%
\subsection{Influence and Qubit Instances}
\label{subsec:cohere-entangle}

In the rest of this chapter, we only consider pure quantum states, which are
sufficient for describing circuit coherence. While in reality
mixed quantum states may occur on implementations of our circuits, it
will not affect our definitions here. We will not precede every
instance of a quantum state with the word ``pure,'' but it should be
assumed.

The basic unit of circuit coherence is a \emph{qubit instance} which is
a qubit at a particular timestep. There are $D\cdot W$ qubit instances
in a circuit, but only some of them are computationally useful, depending
on how they interact with each other via gates. We will refer to
qubits and qubit instances with the same label ($u$ and $v$) if a particular
timestep is implied. Qubit instances which
need to be maintained by error-correction are said to \emph{influence}
the result of the output qubits in the last timestep, what we call
the \emph{output qubit instances}. We will make this more formal
as we build up to an algorithm for calculating coherence.

An \emph{entangled} pure quantum state is one which cannot be expressed as the
tensor product of two smaller states. An unentangled pure quantum state
is also known as a \emph{product} state. 
We call the gate $E_{uv}$ \emph{entangling} between two qubits $u$ and $v$
(and between $u$ and $L = V - \{u\}$) given the input states $\rho^{u}$ and $\rho^{L}$ if
the new global state $\rho^{V}$ after applying $E_{uv}$ cannot be expressed as a product state
$\sigma^{u} \otimes \sigma^{L}$.

However, it is useful to characterize a gate as ``potentially entangling''
as a simplification, so that we can reason about the circuit without
worrying about the input states to each gate.
%
\begin{definition}
A gate $E_{uv}$ is \emph{influencing}, or potentially entangling, if
there exists any pure, product quantum input state $\rho^{V} = \rho^{u} \otimes \rho^{L}$
(where $v \in L$) such that the output state $\sigma^{V}$ is entangled.
We can also say that the qubit instance $u$ influences
the qubit instance $v$, or $u$ influences
$L$, in the timestep in which $E_{uv}$ occurs.
We can also say $E_{uv}$ \emph{connects} influence from $u$ to $v$.
\end{definition}

Influencing gates are the means by which any qubit instance can influence
the output qubit instances. This will tell us when we need to \emph{begin}
maintaining a qubit's state, which is after a gate that is
\emph{influencing}.
But how do we know when to \emph{stop} maintaining a qubit's state?
For this purpose of conserving our error-correcting effort, we define
a related concept: an operator which is known to \emph{disconnect}
a qubit instance from influencing other qubit instances.

\begin{definition}
An operator $M$ on a pure $n$-qubit input state $\rho$ is 
\emph{disconnecting} with respect to a given input state $\rho^{V}$,
if the resulting output state is product:
$\rho^{u} \otimes \rho^{L}$ where $L = V - \{u\}$.
We say $u$ is the
qubit instance which has been disconnected.
\end{definition}

We only consider single-qubit and two-qubit disconnecting operators.
The only single-qubit disconnecting operator that we consider
is a projective measurement operator. This projector is always disconnecting,
regardless of the input state.

The disconnecting nature of a gate
may not be apparent just by examining a circuit locally.
To make things more difficult, a disconnecting gate is also often
influencing (for example, CNOT).
However, a quantum algorithm designer able to specify a circuit in terms of
single-qubit and
two-qubit gates often knows when gates are connecting influence and when
they are disconnecting it.
Moreover, the algorithm designer knows which qubits are garbage and that reversing
part of a quantum circuit will uncompute these garbage ancillae back to $\ket{0}$.
In those cases, it is known by design which gates are disconnecting.

This is the case for
quantum circuits in a certain layered form which we describe in
Section \ref{sec:cohere-tradeoff},
of which the 
well-known QFT and factoring circuits are special cases.
As an overestimate of circuit coherence, we can also consider all
two-qubit gates influencing in the worst case
with
only single-qubit projective measurement considered as disconnecting.
However, this will usually
not give an upper-bound separation between coherence and the depth-width product.
We only consider two-qubit gates that are either influencing or disconnecting
for some input states; all other two-qubit gates are a tensor product of
single-qubit gates and will be treated as such.

%%%%%%%%%%%%%%%%%%%%%%%%%%%%%%%%%%%%%%%%%%%%%%%%%%%%%%%%%%%%%%%%%%%%%%%%%%%%
\subsection{Reachability and Computational Subsets}
\label{subsec:cohere-subset}

We refer back to our definition of a quantum circuit on
\textsf{CCNTC}, which is represented by a graph $G = (V,E)$,
with input qubits $I \subseteq V$ and output qubits $O \subseteq V$, and a
classical controller. In particular, the set of all qubits is $V$,
and its size is $|V|=W$, the circuit width.
Our notion of circuit coherence will not depend
on the modules defined in \textsf{CCNTCM}.

%%%%%%%%%%%%%%%%%% DEFINITION
\begin{definition}{\textbf{Influencing Paths.}}
We denote by $E^{(i)}_{uv}$ a two-qubit gate which acts in
timestep $i$ which is influencing and \emph{not} disconnecting for its current states
on $\rho^{u}$ and $\rho^{L}$ for $L = V - \{u\}$, $v \in L$.
An \emph{influencing path} of gates from qubit $u$ in timestep $i_1$ to
qubit $v$ in timestep $i_n$ is
any sequence of influencing (and not disconnecting)
gates $(E^{(i_1)}, E^{(i_2)}, \ldots, E^{(i_n)})$
where the following conditions are met:

\begin{enumerate}
\item
$E^{(i_1)}$ operates on qubit $u$ and $E^{(i_n)}$ operates on qubit $v$.

%\item
%influencing paths always move forward in time: $i_j > i_{j-1}$.
\item
any two consecutive gates in the sequence $(E^{(i_j)},E^{(i_{j+1})})$
act on a common qubit $w$.
\item
any two consecutive gates in the sequence either occur in
consecutive timesteps ($i_j = i_{j-1} + 1$) or are only separated by
gates which are not single-qubit measurements on $w$ in intervening timesteps $i_j < i < i_{j+1}$.
%\item
%every gate $E^{(i_j)}$ encountered in the sequence satisfies the
%following two conditions:

%\begin{enumerate}
%\item it is influencing if
%the path exits it in the forward direction ($i_{j+1} = i_j + 1$)
%\item it is disentangling if the path exits it in the backward direction
%($i_{j+1} = i_j - 1$).
%\end{enumerate}
\end{enumerate}

\end{definition}

%%%%%%%%%%%%%%%%
We can now make our definition of influence more formal.

\begin{definition}{\textbf{Influence and Reachability.}}
A qubit instance $u$ at timestep $i$ is \emph{reachable} from another qubit
instance $v$ in
another (possibly the same) timestep $i'$ if there is some path of influencing
gates that connects them. Conversely, we also say $v$ influences $u$.
Influence is directed and asymmetric; qubit
instances can only be influenced from qubit instances in earlier timesteps.
\end{definition}

Influence also applies to classical measurement outcomes which are
connected by the classical controller.
If one qubit instance occurs right after a single-qubit measurement
(that is, its state contains a classical measurement outcome), any other
qubit instance (including another classical measurement outcome) which
depends on that outcome is influenced by it.

We now define a standard form for circuit in which circuit coherence will be
well-defined.

\begin{definition}{\textbf{Standard form circuits for calculating circuit coherence.}}
Standard form circuits must have the following properties:

\begin{description}
\item[output qubits $O \subseteq V$:] These qubits are semantically defined as
containing the useful outputs of a quantum circuit. They do not have to be
projectively measured. They may, for example, be the control for a
later coherent measurement when cascaded with another quantum circuit.
\item[input qubits $I \subseteq V$:] These qubits are prepared in a 
classical product state (the computational basis)
and are all reachable in timestep $1$ from the
output qubits in timestep $D$.
\item[ancillae qubits:] these are prepared in the product state of all $\ket{0}$'s.
\end{description}
\end{definition}.

We will assume all circuits from now on are in standard form.

\begin{definition}{\textbf{Computational subset, computational set, computational state.}}
The computational subset in timestep $i$ (abbreviated $L_i$) is the subset of
the qubit instances in that timestep
which are reachable from the output qubits $O$.
The \emph{computational set} is the set of computational subsets across
all timesteps.

\begin{equation}
M = \{L_1, L_2, \ldots, L_D \}
\end{equation}
\end{definition}

The computational set are those qubit instances on which an entangled, coherent
quantum state (which we will call the computational state) evolves over time from
the initial preparation of the input qubits $I \subset V$ in timestep $1$
until the output qubits $O \subset V$ are
measured in timestep $D$.
It is measured in qubit instances,
and potentially grows (if influencing, non-disconnecting gates are applied), shrinks
(if disconnecting gates including measurements are applied) or stays the same in size
in every timestep. We note the following
relationships for well-formed circuits.

\begin{equation}
L_1 = I \qquad L_D = O \qquad L_i \subseteq V
\end{equation}

The computational subset can be computed in two passes through
the quantum circuit, one forward to determine reachability and one backward
to determine the computational subset. We can perform this algorithm efficiently on a
layered quantum circuit as defined in Section \ref{subsec:cohere-lqc},
since it is known which gates are disconnecting in that case.

In each timestep $i$, we partition all $W$ qubits in $V$ into disjoint
subsets. We denote these other
(possibly non-computational) qubit subsets as $\{\tilde{L}^{(j)}_i\}$,
of which one is the same as the current
computational subset $L_i$.

This partitioning, like $L_i$, is updated in
every timestep.

Following the definition of \textsf{2D CCNTCM}, each qubit subset is a
contiguous subgraph of the main graph $G$. No two qubit subgraphs share any vertices,
but all vertices are covered and the subgraphs may share edges. All influencing/disconnecting gates $E^{(i)}_{uv}$
that occur during a timestep $i$ are contained in the set $G_i$.

Qubit subsets may potentially share common qubits and become connected by an influencing gate
in a past timestep $i' < i$ from the current timestep $i$.
We keep track of all qubit subsets at a given timestep $i$
in a collection $M_i = \{\tilde{L}^{(j)}_i \}$.

%%%%%%%%%%%%%%%%%%%%%%%%%%%%%%%%%%%%%%%%%%%%%%%%%%%%%%%%%%%%%%%%%%%%%%%%%%%%%%
\subsection{The Coherence Calculation Algorithm}
\label{subsec:cohere-algo}

The following algorithm takes as input a quantum circuit in standard form
with graph $G = (V,E)$, gates in groups $G_i$
over the basis of single-qubit and two-qubit gates
$U(2) \cup U(4)$ including single-qubit measurements in the $Z$-basis.
Gates in group $G_i$ execute in timestep $i$.
The algorithm returns as output the computational subset at every timestep $\{ L_1, L_2, \ldots, L_D \}$.

The data type of qubit subset $\tilde{L}^{(j)}_i$ is the tuple
$(\tilde{V}, P)$ where:

\begin{itemize}
\item $\tilde{V} \subseteq V$ are the qubits (qubit instances) in the subset
\item $P$ is a pointer to the parent qubit subset in timestep ${i-1}$, initially \textsc{NULL}.
\end{itemize}

When we assign one qubit subset to another, we assume we are only assigning
the qubits part of the tuple, $\tilde{V}$.

%%%%%%%%%%%%%%%%% ENUMERATE 1
\begin{enumerate}
\item
Initialize the following:
\begin{itemize}
\item
$L_1 \leftarrow \{ I \}$.
\item
$\tilde{L}^{(j)}_1 = v_j \in V \setminus I$ (all non-input qubits $v_j$ get their own subset)
\item
$M_1 = \left( \bigcup_j \{ \tilde{L}^{(j)}_1 \} \right) \cup \{ L_1 \} $.
\end{itemize}

\item
In timestep $i \in \{2, \ldots, D \}$:

%%%%%%%%%%%%%%%%%%%%%%%%%%%%% ENUMERATE 2
\begin{enumerate}

\item
Compute the classical description of the state on all the qubits $\rho^{V}_i$
from the state $\rho^{V}_{i-1}$
given $M_{i-1}$.
This is assumed to be efficient (e.g. for layered quantum circuits). 
\item
Initialize $M_i \leftarrow \{\}$.
\item
Create two sets:
\begin{itemize}
\item $T_e$ which contains every two-qubit
gate $E_{uv} \in G_i$ that is influencing and not disconnecting
\item $T_d$ which contains every two-qubit
gate $E_{uv} \in G_i$ that is disconnecting,
along with all single-qubit measurements $M_u \in G_i$.
\end{itemize}

\item
For every qubit subset $\tilde{L}^{(j)}_{i-1} \in M_{i-1}$:

%%%%%%%%%%%%%%%%%%%%%%%%%%%%%%%%%%%%%%%% ENUMERATE 3
\begin{enumerate}
\item \textbf{Connecting Case.} Check whether $\tilde{L}^{(j)}_{i-1}$ contains a qubit acted upon by a
$E_{uv} \in T_e$.

%%%%%%%%%%%%%%%%%%%%%%%%%%%%%%%%%%%%%%%%%%%%%%%%%%%% ENUMERATE 4
\begin{enumerate}
\item If it does, call that qubit
$u \in \tilde{L}^{(j)}_{i-1}$.
Check whether $v$ is in any other qubit subset
(call it $\tilde{L}^{(j')}_{i-1}$).

% Too deeply nested
%%%%%%%%%%%%%%%%%%%%%%%%%%%%%%%%%%%%%%%%%%%%%%%%%%%%%%%%%%%%%%%% ENUMERATE 5
%\begin{enumerate}
\item
If it is, create a new qubit subset
$\tilde{L}^{(j)}_{i}$ equal to the union of the two qubit subsets from
step $i-1$:

\begin{equation*}
\tilde{L}^{(j)}_{i} \leftarrow \tilde{L}^{(j)}_{i-1} \cup \tilde{L}^{(j')}_{i-1}
\end{equation*}

Update the parent pointers of $\tilde{L}^{(j)}_{i}$ accordingly.

%\item
%If $v$ is \emph{not} in any other qubit subset for timestep $i-1$,
%then simply add it to a new qubit subset for timestep $i$. Note that it will
%not have the tag \textsc{Computation Subset}.

%\begin{equation*}
%\tilde{L}^{(j)}_{i} = \{v\}
%\end{equation*}
%\end{enumerate}
%%%%%%%%%%%%%%%%%%%%%%%%%%%%%%%%%%%%%%%%%%%%%%%%%%%%%%%%%%%%%%% ENUMERATE 5

\item
Add the current qubit subset to the current timestep's set of qubit subsets $M_i$.

\begin{equation*}
M_i \leftarrow M_i \cup \{ \tilde{L}^{(j)}_{i} \}
\end{equation*}

\end{enumerate}
%%%%%%%%%%%%%%%%%%%%%%%%%%%%%%%%%%%%%%%%%%%%%%%%%%%% ENUMERATE 4

\item \textbf{Disconnecting Case.} Check whether $\tilde{L}^{(j)}_{i-1}$ matches the following two cases
and take the corresponding actions.

%%%%%%%%%%%%%%%%%%%%%%%%%%%%%%%%%%%%%%%%%%%%%%%%%%%% ENUMERATE 4
\begin{enumerate}
\item If $\tilde{L}^{(j)}_{i-1}$ contains two qubits $u$ and $v$
acted upon by some
$E_{uv} \in T_d$, then check whether $E_{uv}$ is
disconnecting between any partitioning of $\tilde{L}^{(j)}_{i-1}$ into two
subsets $V_1 \ni u$ and $V_2 \ni v$.

% Too deeply nested
%%%%%%%%%%%%%%%%%%%%%%%%%%%%%%%%%%%%%%%%%%%%%%%%%%%%%%%%%%%%%%%% ENUMERATE 5
%\begin{enumerate}
\item
If it does, add these two subsets to our collection $M_i$.
Set their parent pointers to $\tilde{L}^{(j)}_{i-1}$.

\begin{equation*}
M_i \leftarrow M_i \cup \{ V_1, V_2 \}
\end{equation*}

\item
Otherwise, just set the current subset $\tilde{L}^{(j)}_{i} \leftarrow \tilde{L}^{(j)}_{i-1}$
with the appropriate parent pointer, and add it.

\begin{equation*}
M_i \leftarrow M_i \cup \{ \tilde{L}^{(j)}_{i} \}
\end{equation*}

%\end{enumerate}
%%%%%%%%%%%%%%%%%%%%%%%%%%%%%%%%%%%%%%%%%%%%%%%%%%%%%%%%%%%%%%%% ENUMERATE 5

\item
If $\tilde{L}^{(j)}_{i+1}$ contains a qubit $u$ acted upon by some $M_u \in T_d$,
then create two new qubit subsets. One just removes the qubit $u$
from the current qubit subset.
%, inheriting the tag \textsc{Computational Subset}
%if present.
The other is a single-qubit subset consisting
only of $u$.

\begin{eqnarray*}
\tilde{L}^{(j)}_{i} & \leftarrow & \tilde{L}^{(j)}_{i-1} - \{u\} \\
\tilde{L}^{(j')}_{i} & \leftarrow & \{ u \}
\end{eqnarray*}

Set the parent pointer of $\tilde{L}^{(j)}_{i}$ accordingly.
Add these to our collection.

\begin{equation*}
M_i \leftarrow M_i \cup \{ \tilde{L}^{(j)}_{i}, \tilde{L}^{(j')}_{i} \}
\end{equation*}

\end{enumerate}
%%%%%%%%%%%%%%%%%%%%%%%%%%%%%%%%%%%%%%%%%%%%%%%%%%%%% ENUMERATE 4

\end{enumerate}
%%%%%%%%%%%%%%%%%%%%%%%%%%%%%%%%%%%%%%%% ENUMERATE 3

\item For every qubit subset $\tilde{L}^{(j)}_{i+1} \in M_{i-1}$ not
operated upon by any of the previous steps, copy it unmodified 
as $\tilde{L}^{(j)}_i$ into
$M_i$ with the appropriate parent pointer.

\end{enumerate}
%%%%%%%%%%%%%%%%%%%%%%%%%%%%% ENUMERATE 2

\item
Do a backward pass from the outputs to the inputs to discover the computational subset $L_i$ in each timestep.

\begin{enumerate}
\item
Verify that the output qubits exactly correspond to one of the qubit subsets
in $M_D$. Call this $L_D$.
\item
Initialize the computational set $M \leftarrow \{ L_D \}$.
\item
Working backwards for timestep $i$ in $(D, D-1, D-2, \ldots, 3, 2)$:
\begin{enumerate}
\item
Find the parent(s) of $L_i$. Create a new set that is the union of them called $L_{i-1}$
and add them to $M$.
\begin{equation}
M \leftarrow M \cup \{L_{i-1}\}
\end{equation}
\end{enumerate}

\item Verify that $L_1 = I$ are exactly the input qubits.

\item Output $M = \{L_1, \ldots, L_D\}$. This is the computational (sub)set of qubits.
\end{enumerate}

\end{enumerate}
%%%%%%%%%%%%%%% ENUMERATE 1

\begin{definition}{\textbf{Instantaneous coherence}.}
Instantaneous coherence $Q_i$ is the size of the computational subset
(in qubits) in timestep $i$. From the algorithm above,
\begin{equation}
Q_i = |L_i|\text{.}
\end{equation}
\end{definition}

\begin{definition}{\textbf{Circuit coherence}.}
Circuit coherence $Q$ is the sum of the computation subset size (in qubits)
over all $D$ timesteps of a quantum circuit's execution. It is measured
in qubit-timesteps, which is the amount of error-correcting effort to
maintain the coherent state of one logical qubit for one timestep of a circuit.

\begin{equation}
Q = \sum_{i=1}^D Q_i = \sum_{i=1}^D |L_i|
\end{equation}
\end{definition}

We will say that a circuit has greater coherence than another circuit if
its $Q$ has a higher value. This should not be confused with the
meaning of coherence as resisting decoherence.

The following relationships hold with other circuit resources.

\begin{equation}
D \le S \le Q \le D\cdot W
\end{equation}

The first inequality holds in the least parallel case, each of $S$ gates is executed in sequence
and $S=D$. The second inequality holds in the least coherent case, when of $S$ gates
either connects or disconnects another qubit from the computational subset in every timestep, and
there are no identity wires within the circuit. The third inequality holds in the
most coherent case, all of $W$ qubits are part of the computational subset for each of $D$ timesteps.

As an example, we can bound the circuit coherence of modular multiplication of $2\times n$-bit
CSE numbers, as described in Section \ref{subsec:mma}. The overall width is $W = O(n^3)$ and
depth is $D=O(\log n)$, so the coherence is upper-bounded by $O(n^3\log n)$. We will suggest possible
ways to reduce this resource in Section \ref{sec:cohere-tradeoff}.


\section{Circuit Coherence as a Time-Space Tradeoff}
\label{sec:cohere-tradeoff}

Although circuit coherence's motivation was to capture another resource
of interest that may not be 

%%%%%%%%%%%%%%%%%%%%%%%%%%%%%%%%%%%%%%%%%%%%%%%%%%%%%%%%%%%%%%%%%%%%%%%%%%%%%%
\subsection{Measurement-based Quantum Computing Time-Space Tradeoffs}
\label{subsec:cohere-mbqc}

Our model \textsf{2D CCNTCM} bears some resemblance to the
cluster state used to prove universality of the one-way quantum computer
model by Briegel=Raussendorf \cite{Briegel}. Therefore, a natural question
is to study the relationship between these two models and characterize
how the resources for an algorithm on these two models compare to each other.

First, we must make some simplifications to normalize these two models.
\textsf{2D CCNTCM} as defined in Chapter $\ref{chap:factor-polylog}$ allowed
an irregular planar graph, with constant but bounded degree (no more than
6 for factoring). We will constrain ourself to the regular
2D lattice, as in the cluster state graph. We state without proof that
every \textsf{2D CCNTCM} lattice with $n$ qubits can be embedded in
a regular 2D lattice with at most a constant factor increase in qubits
($O(n)$).

Both models contain a classical controller

\subsection{Other Time-space Tradeoffs}
\label{subsec:cohere-ts-other}

Another study of quantum time-space tradeoffs related to our notion of
circuit width is the bounded space regime by Klauck \cite{Klauck2003}.
In that model, the input is read-only and accessed through an oracle.
Time is counted as the number of 

Klauck discovered time-space tradeoff upper bounds for the specific
problem of sorting $n$ numbers.

In comparison to the classical time-space tradeoff for sorting
discovered by Borodin-Cook \cite{Borodin1982} of $\Omega(TS)$.	

%%%%%%%%%%%%%%%%%%%%%%%%%%%%%%%%%%%%%%%%%%%%%%%%%%%%%%%%%%%%%%%%%%%%%%%%%%%%%%
\subsection{The Pebble Game and Reversible Time-Space Tradeoffs}
\label{subsec:cohere-pebble}

An important time-space tradeoff for classical reversible Turing machines
originates from the pebble game as studied by Bennett \cite{Bennett1973}.
This is relevant to quantum time-space tradeoffs when simulating
completely classical circuits on quantum inputs, such as many arithmetic
functions and a large part of Shor's factoring algorithm. Moreover, this
pebble game models how a reversible machine can compute an irreversible
function. It has a direct connection to circuit coherence as we shall see
below, since quantum computations, especially low-depth ones, can leave
garbage behind which must be uncomputed.

The pebble game is a stylized setting for studying time-space tradeoffs.
Although it may take place on general graphs, we study a line graph
in analogy to the mechanism of an MBQC pattern and our factoring architectures
from Chapters \ref{chap:factor-polylog} and \ref{chap:factor-sublog}.
In short, imagine a row of $n$ tiles in sequence, each of which may
contain at most one pebble. One pebble is placed
on tile $1$ in the first timestep, and the goal is to place a pebble
on tile $n$. The only allowable move is that at every timestep,
you may add or remove a pebble from tile $i+1$ if there is a pebble on
tile $i$. Therefore, you cannot remove the pebble from tile $1$.

The number of timesteps it takes to place a pebble on tile $n$ is known
as the time $T$. The number of pebbles present on all tiles is known
as the space $S$. The obvious strategy for winning the pebble game is
to place a pebble on tile $i$ in timestep $i$, without removing any of them.
This completes in time $T=n$ and space $S = n$. In the case of unbounded
space (unlimited pebbles), this is the optimal depth. However, by
bounding space, we can introduce a time-space product $TS$ and attempt to
upper-bound and lower-bound it.

Knill gave a lower bound for the minimum pebble-game time-space tradeoff
\cite{Knill1995} which is bounded above by $n^3$.

\begin{equation}
TS(n) = 2^{2\sqrt{\log(n)}(1 + o(1))}n = o(n^3), \omega(n^2)
\end{equation}

As a consequence, he obtains a minimum time-space tradeoff for
Shor's factoring algorithm on \textsf{AC}.

\begin{equation}
TS(n) = 2^{2\sqrt{n}(1 + o(1))}n^3 = o(n^4), \omega(n^3)
\end{equation}

The minimum time-space tradeoff for factoring is indeed consistent with the depth-width product of all known
factoring implementations from Table \ref{tab:fpl-results}, including
the current work. The one exception is the approximate 1D NTC factoring
implementation by Kutin \cite{Kutin2006}, which beats the above lower bound.
This suggests that the earlier 1D NTC work by Fowler-Devitt-Hollenberg
\cite{Fowler2004} may achieve the optimal depth-width product.

TODO We need to define a standard form for coherent circuits

Note that in applying
pebble-game time-space tradeoffs to a quantum algorithm, each layer
in a parallel quantum circuit now takes the place of a tile in our
scenario above. Therefore, the space (circuit width) taken by placing a ``pebble'' on
each layer is no longer a constant. We will define the \emph{layer width}
$W_l$
of a layer $l$ in a quantum circuit as the number of new qubits which are operated on in
a given timestep. The maximum layer width $W_l^{max}$
is then well-defined over all the
layers, and the overall circuit width is upper-bounded by the sum of the
layer widths.

\begin{equation}
W \le \sum_{l} W_l
\end{equation}

In fact, $LW$ 

Finally, we conclude by describing the connection between the pebble game
and circuit coherence as defined in Section \ref{sec:cohere-def}.
Let us call the number of pebbles present at any timestep of the pebble game
the instantaneous space, and we will scale each pebble by the 
width of that layer in the quantum circuit (not the width of the entire circuit).
This scaled instantaneous space is equal to the computational subset, and
sum of these scaled instantaneous spaces (the scaled pebble-game space) is
an upper-bound, within a factor of the maximum layer width in a quantum
circuit

\section{Circuit Coherence of Factoring Architectures}
\label{sec:cohere-factor}

In this section, we apply circuit coherence and pebble-game
uncomputing techniques from the previous section to our
factoring architectures. The techniques are generic for
any layered circuit, and are not specific to factoring or
even nearest-neighbor circuits, except that the layered circuits
are nearest-neighbor at the layer level.

In Section \ref{subsec:cohere-conject}, we provide a conjecture
for decreasing circuit coherence for modular multiplication while
only marginally increasing our depth.

In Section \ref{subsec:cohere-factor}, we provide a generalized,
configurable-depth factoring architecture based on
Chapter \ref{chap:factor-polylog}.

%%%%%%%%%%%%%%%%%%%%%%%%%%%%%%%%%%%%%%%%%%%%%%%%%%%%%%%%%%%%%%%%%%%%%%%%%%%%%%
\subsection{Reducing Circuit Coherence with Intermediate Uncomputing}
\label{subsec:cohere-conject}

The naive strategy for computing modular multiplication has
depth $D = O(\log n)$, size $O(n^2)$, and width $O(n^3)$. Therefore, the
circuit coherence is upper-bounded by $O(n^3 \log n)$. However,
Bennett \cite{Bennett1989} with corrections from
Sherman-Levine \cite{Levine1990} have shown that an irreversible
pebble game with time $T$ and space $S$ can be simulated reversibly
with overhead $T' = O(T^{1+\epsilon})$ and $S = O(\epsilon 2^{1/\epsilon} S \log T)$
Therefore, we propose the following conjecture.

\begin{conjecture}
Define a layered circuit $C$ for modular multiplication
with layer depth $\tilde{D} = O(\log n)$ and maximum layer width
$w_{max} = O(n^3)$.
Define $P$ as the pebble game corresponding to the optimal reversible simulation of
an irreversible pebble game on $\tilde{D}$ tiles on a reversible
pebble game with the same number of tiles, as proved by
Li \cite{Li1998} based on the authors above.
Using the results of Theorem \ref{thm:pg-cc}, we can create a new
layered circuit $C'$ with uncomputation that performs the
same modular multiplication with the following conjectured resources:
depth $D' = O(D^{1+\epsilon})$,
the same width $W' = W$
and reduced circuit coherence $Q = O(\epsilon 2^{1/\epsilon} n^3 \log\log n)$.
\end{conjecture}

% To improve: add this back in if time permits

%From the previously best-known architectural depth of
%$O(n^2)$, we improved the depth to $O(\log^2 n)$ in Chapter \ref{chap:factor-polylog},
%and then even beneath that to $O((\log\log n)^2)$ in Chapter \ref{chap:factor-sublog}.
%This last depth was limited only by the depth of quantum compiling a
%single-qubit rotation for the quantum majority gate. However, the improvement
%in depth came at the cost of increasing width and size.
%For the polylogarithmic depth, we calculated a size and width of $O(n^4)$.

%How does this compare with our technique of hand-optimized, nearest-neighbor
%mapping in Chapters \ref{chap:factor-polylog} and \ref{chap:factor-sublog}?
%We repeat the relevant rows from 
%Table \ref{tab:fpl-results} and Table \ref{tab:sublog-resources}, namely
%those that correspond to \textsf{AC} and \textsf{2D CCNTCM} implementations
%over the circuit basis of the PK-KSV quantum compiler, which is
%Clifford+$Toffoli$.

%\begin{table}[hbt!]
%\begin{tabular}{|c|c|c|c|c|}
%\hline
%Implementation                           & Architecture  & $D$           & $S$                                             & $W$ \\
%\hline
%Shor \cite{Shor1995}                     & \textsf{AC}   & [$O(n)$]      & $O(n^2\log n \log\log n)$                       & $O(n\log n \log\log n)$ \\
%Browne-Kashefi-Perdrix \cite{Browne2009} & \textsf{CCAC} & $O(1)$        & $O(\frac{1}{\epsilon}n^{6+2\epsilon}\log^{4}n)$ & $$ \\
%Cleve-Watrous \cite{Cleve2000}           & \textsf{AC}   & $O(\log^3 n)$ & $O(n^3)$                                        & $O(n^3)$ \\
%\hline
%Polylog Depth, Chapter \ref{chap:factor-polylog} & \textsf{2D CCNTCM} & $O((\log\log n)^2)$ & $$ & $$ \\
%\hline

%\end{tabular}
%\caption{A comparison for \textsf{AC} factoring architectures.}
%\label{tab:cohere-ac-factor}
%\end{table}



%\begin{table}[hbt!]
%\begin{tabular}{|c|c|c|c|c|}
%\hline
%Implementation                           & Architecture  & $D$ & $S$ & $W$ \\
%\hline
%Shor \cite{Shor1995}                     & \textsf{AC}   & [$O(n)$] & $O(n^2\log n \log\log n)$ & $O(n\log n \log\log n)$ \\
%Browne-Kashefi-Perdrix \cite{Browne2009} & \textsf{CCAC} & $O((\log\log n)^2)$ & $$ & $$ \\
%Cleve-Watrous \cite{Cleve2000}           & \textsf{AC}   & $O((\log\log n)^2)$ & $$ & $$ \\
%\hline
%Polylog Depth, Chapter \ref{chap:factor-polylog} & \textsf{2D CCNTCM} & $O((\log\log n)^2)$ & $$ & $$ \\
%\hline

%\end{tabular}
%\caption{A comparison of hand-optimized and automated nearest-neighbor mappings
%for \textsf{AC} factoring architectures.}
%\label{tab:mbqc-mapping}
%\end{table}

%%%%%%%%%%%%%%%%%%%%%%%%%%%%%%%%%%%%%%%%%%%%%%%%%%%%%%%%%%%%%%%%%%%%%%%%%%%%%%
\subsection{Configurable-Depth Factoring}
\label{subsec:cohere-factor}

When we decrease our nearest-neighbor factoring depth from $O(\log^2 n)$
in Chapter \ref{chap:factor-polylog} to $O((\log\log n)^2)$ in
Chapter \ref{chap:factor-sublog}, we calculated a disproportionate
increase in circuit size and width from
$O(n^4)$ to $\Omega(n^6\log^2 n)$. This seems to be quite an
unfavorable depth-width tradeoff, and it is natural to ask whether
we could have some configurable depth in between
poly-logarithmic and sub-logarithmic that would let a quantum
systems architect choose the right tradeoff for a particular
implementation.

In this section, we provide such a configurable depth
factoring architecture by generalizing our implementation in
Chapter \ref{chap:factor-polylog}. In that chapter, we
wanted to multiply $n\times n$-bit quantum integers in parallel.
To do so, we divided them up into $\lceil n/2 \rceil$ groups of two.
In each group of two quantum integers, call them $\ket{x}$ and
$\ket{y}$, in order to get all the product bits
$\ket{x_i\cdot y_j}$ we need to generate $n^2 \times n$-bit modular residues
$2^i 2^j \bmod m$ controlled on two qubits. We then add these
down with modular multiple addition back to an $n$-bit (CSE) number,
and we are then left with $\lceil n/2 \rceil$ numbers to multiply in
the second level. This takes place in depth $O(\log n)$ and takes
width and size $O(n^3)$.
Modular exponentiation has $\lceil \log_2 n \rceil + 1$ such
levels and perform $(n-1)$ multiplications total, giving us the
final depth of $O(\log n)$ and size and width $O(n^4)$.

The configurable parameter is how we group quantum integers for
expansion. If instead of groups of two, we used groups of
$4$, then each multiplication would require expansion
into $n^4 \times n$-bit numbers. These would still add down to
a single $O(n)$-bit CSE number in $O(\log n)$ depth, but
would now take size and width $O(n^5)$. Modular exponentiation
would still take total depth $O(\log^2 n)$ but would now take
size and width $O(n^6)$, with hidden constants dependent on
the parameter.

We now name the configurable parameter $d$, where in each level
of modular exponentiation we group quantum integers into groups of
$2^d$, and we also expand into $n^{2^d}\times n$ product bits.
We compare this new depth-width tradeoff in Table \ref{tab:config-tradeoff}.

\begin{table}[hbt!]
\centerline{
\begin{tabular}{|c|c|c|c|}
\hline
Implementation                           & $D$ & $W$ \\
\hline
Polylog Depth ($d=1$), Chapter \ref{chap:factor-polylog} & $O(\log^2 n)$ & $O(n^4)$ \\
Config Log Depth ($d$) & $O(\frac{2^d}{d}\log^2 n)$ & $O(\frac{1}{2^d}n^{{2^d}+2})$ \\
($d = \lceil \log_2 n \rceil$)  & $O(n\log n)$ & $O(n^{n+2})$ \\
($d = 1/n$)  & $O(n2^{1/n}\log^2 n)$ & $O(\frac{1}{2^{1/n}}n^{2^{1/n}+2})$ \\
Sublog Depth, Chapter \ref{chap:factor-polylog} & $O((\log\log n)^2)$ & $O(\frac{1}{\epsilon}n^{6 + 2\epsilon}\log^2 n)$ \\
\hline
\end{tabular}
}
\caption{A comparison of configurable-depth factoring architectures and their depth-width tradeoffs.}
\label{tab:config-tradeoff}
\end{table}

The depth-width product for configurable parameter $d$ is
$DW = O(n^{2^d + 3})$. By setting it to be less than one $d = 1/k$
for $k \ge 2$, we are
actually splitting each $n$-bit number into $2^{k-1}$ pieces and grouping
them together.

%Of all these implementations, the one that is the closest to the
%minimum $TS$ from Ref. \cite{Knill1995} is the $O(\log^2 n)$-depth
%factoring implementation from Chapter \ref{chap:factor-polylog).

%Other approaches to configuring 

Finally, we suggest a construction for low-coherence factoring circuits.
In the original quantum period-finding (QPF) procedure of
Nielsen-Chuang \cite{Nielsen2000}, $n$ quantum integers are multiplied in
series, giving a depth in multiplications of $O(n)$.
In the parallelized construction of Kitaev-Shen-Vyalyi \cite{Kitaev2002},
all $n$ quantum integers are multiplied in parallel with a multiplication
depth of $O(\log n)$. Instead, one can consider another kind of configurable-depth
QPF where $n/2^{d-1}$ integers are multiplied in parallel, % in depth $O(d)$,
and there are $2^{d-1}$ such groups multiplied in serial,
%, giving a total depth of $O(2^d \log n)$,
where $d = \lceil \log_2 n \rceil$ for the serial QPF above,
and $d=1$ for the parallel QPF above.

Determining other bounds for circuit coherence and extending this
to other quantum algorithms remains a promising area for future research.

%The best choice of $d$ for either these configurable-coh

\input{coherence/coherence-hs-bg.tex}

\section{Depth-Reduction for Hamiltonian Simulation}
\label{sec:cohere-hs-calc}

% ========== Chapter 5

%\chapter{Hamiltonian Simulation Architectures}
\label{chap:hamsim}

We now apply techniques of parallelization to an alternative
problem, the one of Hamiltonian simulation.

\input{hamsim/hamsim-history.tex}

\input{hamsim/hamsim-compile.tex}
%\chapter{Hamiltonian Simulation on a Nearest-Neighbor Architecture}
%\label{chap:hamsim}

\printendnotes

%
% ==========   Appendices
%
\appendix
\raggedbottom\sloppy

% Uncomment if you ever figure out the error calculation
% and if it is really needed
\chapter{KSV Error from Early Measurement}
\label{app:ksv-error}

In this appendix, we provide a detailed calculation for the error in
phase estimation due to Kitaev, Shen, and Vyalyi (KSV) where projective
measurement is used instead of a coherent measurement. This is called
\emph{early measurement} in Section \ref{sec:qcompile-ksv} of
Chapter \ref{chap:qcompile} where it is used.
We will mainly expand
upon
the exposition in Chapter 12 and 13 in \cite{Kitaev2002}.The use
of projective measurement is to offload the solution of a
modular inverse equation (Equation \ref{eqn:mod-inverse})
and many trigonometric operations
onto a classical controller instead of doing them coherently on a quantum
computer. The cost of projective measurement is leaving garbage qubits in
the ancillae of the target register, but this is a small ($O(n)$)
amount relative to
the size of circuits ($poly(n)$) that we may wish to compile using the KSV procedure.
Our goal is to show that this increase in error by a square root
factor is negligible and may be preferable in realistic implementations.

To begin with, we review (coherent) measurement operators as a generalization of
controlled quantum operators in Section \ref{sec:meas-ops} and introduce
notation in Section \ref{sec:meas-notation}.
We define what it means for a measuring operator to measure a function
in Section \ref{sec:meas-func}. Then we extend our definition in
Section \ref{sec:meas-ancillae} in three ways which require ancillae
qubits: garbage qubits left in the ancillae, non-copy operations
out of the ancillae, and approximate measurement.
Next, we calculate the error of a coherent measurement operation which
incorporates all of these extensions in Theorem \ref{thm:coherent}.
Finally, in our main result, we upper bound the error for
early projective measurement in Theorem \ref{thm:projective}.
We show that the error in the projective case is greater than the
error in the coherent case by a factor equal to the size of
the orthogonal basis decomposition of the measurement object subsystem.

\input{appendices/ksv-error-background}

\input{appendices/ksv-error-measure-ancillae}

%%%%%%%%%%%%%%%%%%%%%%%%%%%%%%%%%%%%%%%%%%%%%%%%%%%%%%%%%%%%%%%%%%%%%%%%%%%%%%
%%%%%%%%%%%%%%%%%%%%%%%%%%%%%%%%%%%%%%%%%%%%%%%%%%%%%%%%%%%%%%%%%%%%%%%%%%%%%%
\section{Error Calculations}

In this section, we will calculate two error probabilities $\delta$ for
an approximate measuring operator $\tilde{Y}$, given that we are able to
approximate a function $f$ with error probability $\epsilon$. The first
error probability, which follows exactly the development in Problem 12.2
of Ref \cite{Kitaev2002}, assumes that we measure \emph{coherently}, meaning
that at no point do we projectively measure, and we are able to perfectly
uncompute all garbage. This is the ideal case, which we calculate in
\ref{subsec:error-noproj}. The second error probability
involves projectively measuring as part of the operator $W$,
which involves some purely classical reversible operations that can be
offloaded to a classical controller. Afterwards, we execute $V$ as before,
but it is now impossible to run $W^{-1}$ because of the projective measurement.
This leaves some amount of garbage in the ancillary
subsystem $\mathcal{B}^N$, and we calculate it in \ref{subsec:error-proj}.

%%%%%%%%%%%%%%%%%%%%%%%%%%%%%%%%%%%%%%%%%%%%%%%%%%%%%%%%%%%%%%%%%%%%%%%%%%%%%%
%%%%%%%%%%%%%%%%%%%%%%%%%%%%%%%%%%%%%%%%%%%%%%%%%%%%%%%%%%%%%%%%%%%%%%%%%%%%%%
\subsection{Error Probability With Coherent Measurement}
\label{subsec:error-noproj}

First, we use this preliminary lemma. using properties of
the operator norm.

%%%%%%%%%%%%%%%%%%%%%%%%%%%%%%%%%%%%%%%%%%%%%%%%%%%%%%%%%%%%%%%%%%%%%%%%%%%%%%
\begin{lemma}[Solution 12.1, p. 230 \cite{Kitaev2002}]
\label{lemma:sum-norm}
Let $X_j : \mathcal{N}_j \rightarrow \mathcal{M}_j $
be a collection of operators which operate on pairwise orthogonal
subspaces, where each $X_j$ takes $\mathcal{N}_j$ to $\mathcal{M}_j$.
Then the norm of the operator $X$ formed as a direct product of these
$X_j$'s has an operator norm equal to the maximum of any of the
$X_j$'s.

\begin{equation}
X = \bigoplus_j X_j : \bigoplus_j \mathcal{N}_j \rightarrow \mathcal{M}_j
\end{equation}

\begin{equation}
|| X || = \max_{j} ||X_j||
\end{equation}
\end{lemma}

\begin{proof}
This follows from the fact that the operator norm measures how much
an operator scales any non-zero vector. If the vector comes from the
space which is the direct sum of the $\mathcal{N}_j$'s, it is in a
particular fixed subspace $\mathcal{N}_j$. Therefore, it cannot be scaled
more than the maximum operator norm of any of the $X_j$'s.
\end{proof}

We now apply this to the case of unitary operators
to show how to approximate a measuring operator with
ancillae which is the direct sum of projectors onto pairwise orthogonal subspaces.

%%%%%%%%%%%%%%%%%%%%%%%%%%%%%%%%%%%%%%%%%%%%%%%%%%%%%%%%%%%%%%%%%%%%%%%%%%%%%%
\begin{lemma}[Problem 12.1 \cite{Kitaev2002}]
\label{lemma:error-sum}
Let $W$ be a unitary operator which acts on a space with two subsystems
$\mathcal{N}$ and $\mathcal{K}$ and is the direct sum of projectors onto
pairwise orthogonal subspaces of $\mathcal{N} = \bigoplus_j \mathcal{L}_j$
tensored with unitary operators on $\mathcal{K}$.
Let $\tilde{W}$ be an analogous operator except that the unitaries $\tilde{U}_j$
now operate on $\mathcal{K}$ tensored with an ancillary subsystem
$\mathcal{B}^N$.
%
\begin{eqnarray}
U_j : \mathcal{K}\\
\tilde{U}_j : \mathcal{K} \otimes \mathcal{B}^{\otimes N}\\
W : \mathcal{N} \otimes \mathcal{K} = \bigoplus_j \Pi_{\mathcal{L}_j} \otimes U_j \\
\tilde{W} : \mathcal{N} \otimes \mathcal{K} \otimes \mathcal{B}^N =
\bigoplus_j \Pi_{\mathcal{L}_j} \otimes \tilde{U}_j
\end{eqnarray}
%
Suppose that for each $j$,
$\tilde{U}_j$ approximates (with ancillae) $U_j$ with error $\nu$
according to the
definition below.
%
\begin{equation}
\forall_j || \tilde{U}_j - (U_j \otimes I_{\mathcal{B}^{\otimes N}}) || \le \nu
\label{eqn:uj_approx}
\end{equation}
%
Then the measuring operator
$\tilde{W} = \oplus_j \Pi_{\mathcal{L}_j} \otimes \tilde{U}_j$ approximates
with ancillae the measuring operator
$W = \sum_j \Pi_{\mathcal{L}_j} \otimes U_j$ with the same error $\nu$.
%
\begin{equation}
|| \tilde{W} - (W \otimes I_{\mathcal{B}^{\otimes N}}) || \le \nu
\end{equation}
%
\end{lemma}

\begin{proof}
We decompose $W$ using its definition.
%
\begin{eqnarray}
|| \tilde{W} - (W \otimes I_{\mathcal{B}^{\otimes N}}) || & = &
|| (\bigoplus_j \Pi_{\mathcal{L}_j} \otimes \tilde{U}_j) -
   (\bigoplus_j \Pi_{\mathcal{L}_j} \otimes U_j \otimes I_{\mathcal{B}^{\otimes N}}\\
& = & || \bigoplus_j \Pi_{\mathcal{L}_j} \otimes (\tilde{U}_j - (U_j \otimes I_{\mathcal{B}^{\otimes N}})) || \\
& \le & \max_j ||    \Pi_{\mathcal{L}_j} \otimes (\tilde{U}_j - (U_j \otimes I_{\mathcal{B}^{\otimes N}})) ||
\end{eqnarray}
%
In the last step above, we use Lemma \ref{lemma:error-sum} to reduce the
error of approximating $W$ with ancillae to the largest error of approximating
any $U_j$ with ancillae. Now we use Equation \ref{eqn:uj_approx} and continue.
%
\begin{eqnarray}
|| \tilde{W} - (W \otimes I_{\mathcal{B}^{\otimes N}}) || & \le &
\max_j || (\tilde{U}_j - (U_j \otimes I_{\mathcal{B}^{\otimes N}}) || \\
& = & \nu
\end{eqnarray}
%
In the step above, we used the fact that when taking the norm of an operator
which is a projector onto a subspace
tensored with a unitary, this reduces to taking the norm
of just the unitary. This is because the vectors outside of the subspace,
which are projected away to the zero vector, cannot affect the operator norm.

Therefore, we have that $\tilde{W}$ \emph{with} ancillae approximates $\tilde{W}$
\emph{without} ancillae with the same error $\nu$ of approximating the unitaries
with ancillae within any subspace.
\end{proof}

We now return to the setting of Section \ref{subsec:approx} and repeat
its definitions here. Given a measuring operator $W$ which approximately
measures a function, what is the error of approximation
of a two stage measurement using $W$ and intermediate ancillae?
We will see later that such an approximate two-stage measurement
corresponds to parallelized phase estimation.

%%%%%%%%%%%%%%%%%%%%%%%%%%%%%%%%%%%%%%%%%%%%%%%%%%%%%%%%%%%%%%%%%%%%%%%%%%%%%%%
\begin{theorem}[Problem 12.2, \cite{Kitaev2002}]
We consider the space
$\mathcal{N} \otimes \mathcal{B}^{\otimes N} \otimes \mathcal{K}$ with the
following orthogonal subsystem decompositions:
$\mathcal{N} = \bigoplus_{j \in \Omega} \mathcal{L}_j$ and
$\mathcal{B}^{\otimes N} = \bigoplus_{y \in \Delta} \mathcal{M}_y$. We
define two operators, $W$ which measures $\mathcal{N}$ as object into
$\mathcal{B}^{\otimes N}$ as instrument and $V$ which measures
$\mathcal{B}^{\otimes N}$ as object into $\mathcal{K}$ as instrument.
%
\begin{eqnarray}
W : \mathcal{N} \otimes \mathcal{K}\\
\tilde{W} : \mathcal{N} \otimes \mathcal{K} \otimes \mathcal{B}^N
\end{eqnarray}
%
If $W$ approximately measures a function $f : \Omega \rightarrow \Delta$
with error $\epsilon$, then the operator $\tilde{Y} = W^{-1}VW$ approximates
the following operator $Y$ with error $2\sqrt{\epsilon}$.
\label{thm:coherent}
\end{theorem}

%%%%%%%%%%%%%%%%%%%%%%%%%%%%%%%%%%%%%%%%%%%%%%%%%%%%%%%%%%%%%%%%%%%%%%%%%%%%%%%
\begin{proof}
Even though we have already defined $Y$ and $\tilde{Y}$ in terms of
operators on the whole space
$\mathcal{N} \otimes \mathcal{K} \otimes \mathcal{J}$, it is now useful to
define $\tilde{Y}$ in an alternative way: operators $P_j$ which operate on the
space $\mathcal{K} \otimes \mathcal{J}$ which can further be expressed
in terms of operators $Q_y$ operating on $\mathcal{K}$ and operators
$R_j$ operating on $\mathcal{J}$.
%
\begin{equation}
\tilde{Y} = \sum_{j \in \Omega} \Pi_{\mathcal{L}_j} \otimes P_j \qquad
P_j = \sum_{y \in \Delta} Q_y \otimes (R_j^{\dagger}\Pi_{\mathcal{M}_y}R_j)
\end{equation}
%
Using the results of Lemma \ref{lemma:error-sum}, we only need to show that
$P_j$ approximates (using ancillae) $Q_{f(j)}$ (without using ancillae) for
all $j$.

To begin with, let's examine the action of $P_j$ and $Q_{f(j)}$ on
an arbitrary state $\ket{\xi} \in \mathcal{K}$, possibly augmented with
ancillae $\ket{0^N}$.
We further define the following states and the difference between them.
%
\begin{eqnarray}
\ket{\eta}         & = & Q_{f(j)} \ket{\xi} \\
\ket{\tilde{\eta}} & = & P_j(\ket{\xi} \otimes \ket{0^N}) \\
\ket{\psi}         & = & \ket{\tilde{\eta}} - \ket{\eta}
\end{eqnarray}
%
We want to minimize the norm of $\ket{\psi}$, which represents the error
in a state operated on by the desired unitary $Q_{f(j)}$ and the state
operated on by the approximation with ancillae $P_j$.

That's how we begin these next calculations.
%
\begin{eqnarray}
\braket{\psi}{\psi} & = & \left[ \bra{\tilde{\eta}} - (\bra{\eta} \otimes \bra{0^N}) \right]
                    \left[ \ket{\tilde{\eta}} - (\ket{\eta} \otimes \ket{0^N}) \right] \\
              & = & \left[ \braket{\tilde{\eta}}{\tilde{\eta}} \right ] - \\
              &   & \left[ (\bra{\eta} \otimes \bra{0^N})(\ket{\tilde{\eta}}) \right] - \\
              &   & \left[ (\bra{\tilde{\eta}} \otimes (\ket{\eta} \otimes \ket{0^N}) \right] + \\
              &   & \left[ \braket{\eta}{\eta} \right] \\
              & = & 2 - \left(\bra{\eta} \otimes \bra{0^N} \right) \ket{\tilde{\eta}} - \\
              &   & \bra{\tilde{\eta}} \left( \ket{\eta} \otimes \ket{0^N} \right)
\end{eqnarray}
%
In the last line, we use the fact that $\ket{\eta} \otimes \ket{0^N}$ and
$\ket{\tilde{\eta}}$ are both normalized states of unit norm. The two braket
terms remaining are complex conjugates of each other (call them $a+bi$ and
$a-bi$) so their sum is just $2a$, or the real part of either complex number.

This is where we continue our calculations, using the definition of $\ket{\eta}$
and $\ket{\tilde{\eta}}$ to express their overlaps in terms of $Q_y$,
$R_j$, and projectors onto the orthogonal decomposition of $\mathcal{J}$.
%
\begin{eqnarray}
\braket{\psi}{\psi} & = & 2 - 2\Re\left[ \left(\bra{\eta} \otimes \bra{0^N}\right) \ket{\tilde{\eta}} \right] \\
                    & = & 2 - 2\Re\left[ \sum_{y \in \Delta} \bra{\xi}Q_{f(j)}^{\dagger}Q_y \ket{\xi}
                                                             \bra{0^N}R_j^{\dagger}\Pi_{\mathcal{M}_y}R_j\ket{0^N} \right] \label{eqn:sym-real}
\end{eqnarray}
%
In the last line above, we use the fact that we can distribute a braket
through a tensor product.

We can now factor out the braket
$\bra{0^N}R_j^{\dagger}\Pi_{\mathcal{M}_y}R_j\ket{0^N}$ from the product
inside the real operation above since the braket is completely real. This is
because $\Pi_{\mathcal{M}_y}R_j \ket{0^N}$ is either the zero vector or
a normalized state. We will henceforce call this projected state
$\ket{\phi_{(y,j)}}$, and we will write its overlap with itself
as the braket $\braket{\phi}{\phi}$.

Now we also need to use the fact that the real part of a sum over complex
numbers is at most the sum of real parts of each complex number.
%
\begin{equation}
\Re\left[ \sum_{i} c_i \right] \quad \le \quad \sum_{i} \Re( c_i )
\end{equation}
%
Furthermore, the following equality holds.
%
\begin{equation}
1 - \sum_{i} \Re(c_i) \quad = \quad \sum_{i} \left(1 - \Re(c_i) \right)
\end{equation}
%
Substituting this
back in our original calculation, we get:
%
\begin{eqnarray}
\braket{\psi}{\psi} & \le & 2 - 2\sum_{y \in \Delta} \Re\left(
  \bra{\xi}Q_{f(j)}^{\dagger}Q_y \ket{\xi}
  \braket{\phi_{(y,j)}}{\phi_{(y,j)}} \right) \\
   & \le & 2 - 2\sum_{y \in \Delta} \Re\left(
           \bra{\xi}Q_{f(j)}^{\dagger}Q_y \ket{\xi}\Pr(y | j) \right) \\
   & \le & 2 - 2\sum_{y \in \Delta} \Re\left(
           \bra{\xi}Q_{f(j)}^{\dagger}Q_y \ket{\xi}\right) \Pr(y | j) \\
   & \le & 2 \sum_{y \in \Delta}\left[
           1 - \Re\left( \bra{\xi}Q_{f(j)}^{\dagger}Q_y \ket{\xi} \right)
           \Pr(y | j)
           \right]
\end{eqnarray}
%
In the second line, we used the definition of
$\bra{0^N}R_j^{\dagger}\Pi_{\mathcal{M}_y}R_j\ket{0^N} = \braket{\phi_{(y,j)}}{\phi_{(y,h)}}$
as the probability of getting outcome $y$ in the register
$\mathcal{J}$ given that we have projected onto outcome $j$ in register
$\mathcal{N}$ according to Section \ref{subsec:approx}.

Next, we use the fact that the real part of the overlap
$\bra{\xi}Q_{f(j)}^{\dagger}Q_y \ket{\xi}$ is between $-1$ and $1$, inclusive,
because any $Q_y\ket{\xi}$ is a normalized state, and therefore one minus
this quantity is at most $2$. We also exclude the case where
$y = f(j)$, since in that case $Q_{f_(j)}^{\dagger}Q_y = I$ and the overlap
is 1, contributing zero to the sum.
%
\begin{eqnarray}
\braket{\psi}{\psi} & \le & 2 \sum_{y \ne f(j)} 2\Pr(y|j) \label{eqn:prob-sum}\\
                    & \le & 4\epsilon
\end{eqnarray}
%
In the final line above, we use the definition of $\Pr(y|j)$ from
Section \ref{subsec:approx}. The actual answer we want is the norm
of the difference vector $\ket{\psi}$, which is the square root of the
magnitude of the overlap.
%
\begin{equation}
\lvert\lvert \ket{\psi} \rvert\rvert \le 2\sqrt{\epsilon}
\end{equation}
%
This completes the proof.

As an additional note, if $V$ is the copy operator, we have
$Q_{f(j)}^{\dagger}Q_y = \delta_{y,f(j)}I_{\mathcal{K}}$.
For each $y = f(j)$, the overlap $\bra{\xi}Q_{f(j)}^{\dagger}Q_y \ket{\xi}$
is $1$ and contributes zero to the sum of probabilities.
For each $y \ne f(j)$, the overlap is exactly $0$, and contributes
$\Pr(y|j)$ to the sum.
Therefore, instead of line \ref{eqn:prob-sum} above, we get:
%
\begin{eqnarray}
\braket{\psi}{\psi} & \le & 2\sum_{y \ne f(j)} \Pr(y|j)\\
                    & \le & 2\epsilon
\end{eqnarray}
%
And the final error is
%
\begin{equation}
\lvert\lvert \ket{\psi} \rvert\rvert \le \sqrt{2\epsilon}
\end{equation}
%
This makes intuitive sense, since the first bound is overly general.
It works for any $V$. If we know $V$, we can actually make additional
assumptions and get a better (smaller) upper bound for the error, in this
case by a factor $\sqrt{2}$.
\end{proof}

%%%%%%%%%%%%%%%%%%%%%%%%%%%%%%%%%%%%%%%%%%%%%%%%%%%%%%%%%%%%%%%%%%%%%%%%%%%%%%
%%%%%%%%%%%%%%%%%%%%%%%%%%%%%%%%%%%%%%%%%%%%%%%%%%%%%%%%%%%%%%%%%%%%%%%%%%%%%%
\subsection{Error Probability With Projective Measurement}
\label{subsec:error-proj}

Now what happens if we perform the operator $\hat{Y} = VW$? That is, we
perform $W$ to measure some function $f$ and then we extract the useful
information using $V$, but we don't wish to uncompute the results by
performing $W^{-1}$? There are four related reasons we may wish to do this:

\begin{enumerate}
\item
because $W$ may be practically difficult to perform, and we don't wish
to essentially do it twice
\item
because we have projectively measured
in the register $\mathcal{K}$ so that we can offload some post-processing to
a classical computer, and there is no way to uncompute $W$ at that point.
\item
the states in $\mathcal{K}$, after having been projected by $\hat{Y}$,
are close to classical, and we wish to avoid maintaining them as coherent
quantum states
\item
the states in $\mathcal{N}$, after having been entangled with $\mathcal{K}$ and
projected by $\hat{Y}$, are still ``mostly'' within their subspaces
$\mathcal{L}_j$, and we wish to use states in these subspaces as a resource.
\end{enumerate}

All of these reasons are true for the problem of quantum Fourier state
generation, as described in Section \ref{sec:qcompile-qfs}. However,
the last reason is the most important, since in the case of phase
estimation for phase kickback quantum compiling, a quantum Fourier
state (or something close to one) is left in the subsystem $\mathcal{N}$.

These are both practical reasons in that they don't affect the
asymptotic resources required by our algorithm. However, they do affect
the asymptotic error of our algorithm. In this section, we will calculate
this error based on Theorem \ref{thm:coherent}. This is the main original
contribution of this work beyond the exposition in \cite{Kitaev2002}.

%%%%%%%%%%%%%%%%%%%%%%%%%%%%%%%%%%%%%%%%%%%%%%%%%%%%%%%%%%%%%%%%%%%%%%%%%%%%%%%
\begin{theorem}
Given the setting in Theorem \ref{thm:coherent}
where $W$ approximately measures a function $f : \Omega \rightarrow \Delta$
with error $\epsilon$: If the operator $\hat{Y} = VW$ followed by projective
measurement on $\mathcal{K}$ yields classical outcome $y \in \Delta$,
then the
overlap of the state $\ket{\tilde{\psi}}$ in $\mathcal{N}$ with $\mathcal{L}_j$ such
that $f(j) = y$ is upper-bounded by the number of subspaces $|\Omega|$ times the
$\epsilon$.
%
\begin{equation}
\sum_{j \in \Omega: f(j) \ne y} \bra{\tilde{\psi}}\Pi_{\mathcal{L}_j}\ket{\tilde{\psi}} \le |\Omega|\epsilon
\end{equation}
\label{thm:projective}
\end{theorem}
%
\begin{proof}
Consider the probability distribution of states over $\mathcal{N} \otimes \mathcal{J} \otimes \mathcal{K}$.
Decomposing the entire system over the subspaces of $\mathcal{N} \bigoplus_{j \in \Omega} \mathcal{L}_j$,
within each $\mathcal{L}_j$ there is a probability of $\epsilon$ associated with error states $y \in \Delta$
such that $y \ne f(j)$.
%
\begin{equation}
\sum_{y \ne f(j)} \Pr(y | j) \le \epsilon
\end{equation}
%
We can define a total error probability by summing over all values of $j \in Omega$, or
$\mathcal{L_j} \in \mathcal{N}$.

\begin{equation}
\sum_{j \in \Omega} \sum_{y \ne f(j)} \Pr(y | j) \le |\Omega|\epsilon
\end{equation}

However, if we projectively measure $y$ in register $\mathcal{K}$, what is the probability
mass associated with ``impurities'' in register $\mathcal{N}$? That is, we want to
find $\sum_{j: f(j) \ne y} \Pr(j | y)$. The maximum contribution to this error for
any $y$ is the total error probability above. Therefore, we have:

\begin{equation}
\sum_{j: f(j) \ne y} \Pr(j | y) \le |\Omega|\epsilon
\end{equation}

This argument is independent of any ancillae.
\end{proof}

Theorem \ref{thm:projective} upper-bounds the error for
early measurement in KSV parallel phase estimation used to
generate a quantum Fourier state. It is necessary to compute the resources
for a quantum compiler in Section \ref{sec:qcompile-ksv}.


%
% ==========   Bibliography
%
\nocite{*}   % include everything in the uwthesis.bib file
\bibliographystyle{plain}
\bibliography{ppham-thesis}
 

\end{document}
