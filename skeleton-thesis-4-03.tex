\documentclass{article}

\begin{document}

\section{The Overall Sketch of a Constant-Depth Implementation to Factoring}

The simplest but most disappointing discovery of today was that we
could not do constant-depth factoring in a ``nice'' way using the
modular exponentiation in depth-4 or even use modular reduction in depth 4
from the Siu-Bruck paper, since those only work for a modulus that is
polynomially-bounded, that is $m < n^k$ for some $k > 0$. In fact, for
Shor's algorithm, we allow exponential moduli, that is, $m < 2^n$.
So that is out.

There is still a way though, but it uses the icky division circuit in
Siu-Bruck, which is some ways seems superficially similar to the division
and true division circuits that I spent a lot of time understanding
two summers ago during my MSR internship. So we first use the depth-four
circuit for multiple multiplication, the multiple product of $n\times n$-bit
numbers into a single $n^2$-bit number in depth-4. This probably works
because the product of any two $n$-bit numbers is in general a $2n$-bit number. We can multiply $n$ such numbers together in a $\log_2 n$-depth
binary tree, where at the end we have a $n\cdot 2^{\log_2 n} = n^2$-bit
number as desired.

Then we can do division to get the remainder. It's not clear to me what
$DIV_k(x/y)$ means in the Siu-Bruck paper, whether that gives us the
remainder or what. But modular reduction seems pointless because we only
needed it to keep a reasonably-sized register right? Do we even need
modular multiplication to be unitary? Oh, I think it is because phase
estimation only works for selecting a random eigenstate if the operators
you apply have as eigenvectors the superposition of the initial state.
There was probably a better way to say that.
You know what I meant.

Okay more writing, less thinking. Something to ponder later.

\section{Implementation of a Threshold Gate in Nearest-Neighbor}

Andrea appears to have done some of this work already in her report.
The threshold gate performs a rotation by Hamming value on each of the
inputs equal to the weight, and then something something final overall
threshold value. However, if I remember correctly, it involves two layers
of gates, a threshold of thresholds if you will. So constant-depth
teleportation could be used to move the qubits from the first layer to
be adjacent for the second layer. Also, I have a feeling that the first
layer requires constant-depth fanout to create enough copies of the
original inputs to operate in parallel. To create the correct weights for
each input? That doesn't make sense. Fact check this!

\section{Relationship Between a Threshold Gate and a Linear Threshold Function}

This appears to be a one-to-one mapping.
So it's probably a stupid section to have.

\section{Relationship Between a Linear Threshold Function and a
Polynomial Threshold Function}

A linear threshold function as defined by Siu-Bruck is a sign function on
$n$ inputs ${x_1, \ldots, x_n}$ with polynomial weights, including an
initial weight $w_0$ that doesn't depend on any input that will let us
program the ``offset'' of this function. This is what they call the class
$LT_1$, meaning the class of functions that can be computed this way. Or
rather, when they say a function $F\in LT_1$, they mean a family of 
functions which depend on input size $n$ can be implemented with a single
threshold gate. They are being rather pedantic, I think.

However! The linear threshold function could be the result of exponentially
big weights. Therefore, we consider weights (real-valued, could be negative)
which are bounded by $n$, that is $|w_i| < n^c$. for some $c>0$. We call 
this class $\hat{LT}_1$, and it is the one we are primarily interested in.
For the sake of tractability, if a linear threshold function as 
exponentially big weights, it could be difficult to implement.
(How? Why?) And what is the relationship between linear threshold function
and threshold gates?

One thing we can say for sure is that $\hat{LT}_1 \subseteq LT_1$.

I am guessing that a polynomial threshold function is a multinomial
function in the input variables $\{x_1, \ldots, x_n\}$ where each
monomial contains up to $n$ of these input variables, but the number of
multinomials $m$ is polynomially bounded by $n$, that is, $m < n^c$ for
some $c > 0$. This is what Siu and Bruck call a \emph{sparse polynomial},
a term which I like. I think also each multinomial can have a weight
$w_i$ which may also be polynomially bound, $|w_i| < c$.
How are these relation to linear threshold functions? Fact check this!

\section{Relationship Between Majority Gate and Threshold Gate}

A Majority gate is a threshold gate where the weights only take on values of
$-1$ and $+1$, where the inputs $x_i \in {-1,+1}$. This seems to be a 
special case of threshold gates, and so we should be safe in saying that
$MAJ_1 \subseteq \hat{LT}_1 \subseteq LT_1$.

What is a monotone function? Why do they talk about non-monotone analogues 
of the conventional majority gate?

\section{Delta Polynomials}

In analogy to the Dirac delta function, which has a large weight
(theoretically infinite) at a particular point and zero weight anywhere 
else, a delta polynomial over input variables $\{x_1, \ldots, x_n\}$ has
a value greater than some $a$ for the all $+1$'s input:
$\epsilon_1 = (1,\ldots,1)$ and
some value less than some $a/c$ for any other input.

This allows us to reward a condition with a large positive value by 
encoding it in $\epsilon_1$ while simultaneously making sure that
$\epsilon_1$ is not accidentally created by some other combination of 
inputs. This is combines out notion of completeness. We do this by
creating some polynomial $P(\cdot)$ on a truncated input, that is, only
on a subset of the input variables $x_i$'s which slides depending on
which portion we want to consider. This sliding window notion is natural
in many arithmetic functions when we can define a kind of recursion or
dependency.

For comparison\ldots

However, there is still soundness. We want undesirable inputs to generate
low weight, and so there is often an additional coefficient controlled
by the inputs

\section{Explicit Polynomials}

The existence of such polynomial threshold functions for given arithmetic
functions such as COMPARE and ADDITION is proven in general terms by
arguing from the symmetry of the functions (how they are unchanged by
permuting the inputs) and constructing a composite circuit where
the inputs to a linear threshold function are themselves outputs of
a previous layer of linear threshold functions. FACT CHECK.

However, explicit polynomials for these can be constructed by using
so-called \emph{delta polynomials} described in a previous section.

\section{Explicit Function for Comparison}

\end{document}