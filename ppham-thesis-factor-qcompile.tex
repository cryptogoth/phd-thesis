\section{Controlled Rotations for Factoring}
\label{sec:factor-crz}

In Section \ref{sec:related}, we stated that we wished to avoid
factoring implementations that used a QFT due to the fine
single-qubit rotations involved. Due to the requirements of
fault-tolerance on a particular physical implementation,
we can usually implement only a set of gates that is fixed
(it does not change with the problem input size), discrete (of finite size),
and universal. This last property is necessary for us to approximate any
other gate \emph{not} in our set, especially single-qubit phase rotations
of the form $\Lambda(e^{i \phi})$. Such an approximation would involve
a quantum compiling procedure, such as Solovay-Kitaev, which is the
subject of Chapter \ref{chap:qcompile}. However, we mention it here
because the choice of our universal set determines the true depth
of any circuit.

In our polylogarithmic factoring implementation, we were able to reduce
all our arithmetic circuits to such a fixed, discrete universal set.
These arithmetic circuits are discrete and classical and nature, so it is
not surprising that we can implement them in a discrete way.
However, to reduce the depth further, we need to introduce the idea of
a quantum threshold gate, which 
the Toffoli gate, CNOT, and fixed set of
single-qubit gates.



\begin{figure}[tb!]
\begin{center}
\begin{displaymath}
\begin{array}{ccc}
\Qcircuit @C=1.5em @R=1.5em {
   & \qw      & \ctrl{1}                   & \qw \\
   & \qw      & \gate{\frac{\pi}{2^{d}}} & \qw \\
 }
&
\begin{array}{c}
\\
\\
\\
= \\
\end{array}
&
\Qcircuit @C=1.5em @R=1.5em {
& \qw & \qw & \qw & \ctrl{1} & \qw & \gate{\frac{\pi}{2^{d+1}}} & \qw & \ctrl{1} & \qw\\
 & \qw & \gate{\frac{\pi}{2^{d+1}}} & \qw & \targfix & \qw & \gate{\frac{\pi}{2^{d+1}}} & \qw & \targfix & \qw
}
\end{array}
\end{displaymath}
\caption{Decomposition of a controlled-$R_z$ rotation}
\label{fig:crz}
\end{center}\end{figure}
