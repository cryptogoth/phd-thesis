\section{Quantum Architectures and Models}
\label{sec:intro-arch}

In this dissertation, we will refer to a quantum architecture (or simply,
an architecture) interchangeably with the terms \emph{implementation} 
or a \emph{circuit}, although in the latter case it is understood that the
circuit has architectural constraints. We will make our definition from
the introduction more formal here.

\begin{definition}
A quantum architecture consists of a quantum circuit $C$ with
two main constraints.

\begin{enumerate}
\item a fault-tolerant basis (fixed, finite) basis of
gates $\mathcal{G}$
\item a layout graph $(V,E)$ which defines the connectivity (allowed two-qubit
interactions, or edges in $E$) between qubits (which are nodes in $V$).
\end{enumerate}
\end{definition}

The constraints are chosen to match some aspect of experimental reality.
The circuit basis is usually chosen to meet fault-tolerance requirements.
The layout graph is chosen to match the nearest-neighbor nature of
many physical interactions.

Following Van Meter and Itoh \cite{VanMeter2005},
we distinguish between a model and an architectural implementation as follows.
A \emph{model} includes a set of constraints for circuit basis and layout
graph, which may be common to many quantum architectures. Models allow us
to abstract our study of quantum architectures further and reason about
general properties of the constraints themselves.
We say that an architecture is implemented \emph{on} such-and-such a model,
or that a model is meant to closely correspond to a physical technology such as
superconducting qubits.

In the rest of this section, we give definitions of
several important models that are common in the literature and how they are
related to each other.
In particular, we define the
model \textsf{2D CCNTC} as a foundation for a new \emph{hybrid}
model that we
introduce in Section \ref{sec:intro-modules}. Finally, we define how
circuit resources are counted using \textsf{2D CCNTC} in the rest of this
dissertation.

%%%%%%%%%%%%%%%%%%%%%%%%%%%%%%%%%%%%%%%%%%%%%%%%%%%%%%%%%%%%%%%%%%%%%%%%%%%%%%%
\subsection{Architectural Models}
\label{subsec:models}

The most general model is called Abstract Concurrent (\textsf{AC})
and allows arbitrary, long-range interactions between any qubits and concurrent
operation of quantum gates on disjoint qubits.
The layout graph is a complete graph with an edge between every pair of nodes.
It is the model assumed in most abstract quantum algorithms. 
The circuit basis
for this model is arbitrary (infinite) single-qubit operations and some
two-qubit operation that completes it as a universal gateset (usually $CNOT$).
All operations are usually counted to have unit depth and unit size.

A more specialized model restricts interactions to nearest-neighbor, two-qubit,
concurrent gates (\textsf{NTC}). This is a submodel of \textsf{AC} in that it
has the same circuit basis, but loosens a restriction on the layout graph,
which no longer has to be complete. All \textsf{AC} architectures
are also \textsf{NTC} architectures, but not vice versa.

In a more realistic restriction of \textsf{NTC}, we confine the qubits to
lie in a regular one-dimensional chain. We call this new model
\textsf{1D NTC} or more commonly
linear nearest-neighbor (\textsf{LNN}).
The layout graph is now a line graph.
Since Shor's original algorithm was
presented in \textsf{AC} \cite{Shor1994}, we expect an increase in circuit resources to
simulate these same long-range interaction using the nearest-neighbor
interactions of \textsf{NTC}. This model more realistically captures multiple
ion traps that are arranged in a linear chain, perhaps one where each
trap encodes a single logical qubit.

An even more realistic model is described in Beckman et al.
\cite{Beckman1996} called \textsf{ION TRAP}, which unsurprisingly,
corresponds to a single linear trap where all ions are coupled by
common vibrational modes \cite{Cirac1995}.
The layout graph consists of a star topology,
where all ion qubits are connected to the motional qubit, and the basis
again consists of arbitrary single-qubit gates and fixed two qubit
gates (usually $CNOT$ or the controlled phase gate $\Lambda(Z)$).
All gates require
a constant number of laser pulses to implement, although some are more
expensive than others. An interesting consequence of this model is that
multiply-controlled single-qubit gates of the form $\Lambda^n(U)$ for
$U \in U(2)$ can be performed with $O(n)$ pulses and no ancillary qubits
beyond the motional qubit. Unfortunately, it is difficult to scale
this configuration due to the slowing of operations for large strings of
ions \cite{Haffner2008}.

To relieve movement congestion up and down a linear chain,
we can consider a two-dimensional regular grid
(\textsf{2D NTC}), where each
qubit has four planar neighbors, and 
there is an extra degree of freedom over the \textsc{1D} model
in which to move data. While this extra degree of freedom allows us to
decrease depth over \textsf{1D NTC}, it possible to improve it further.
To allow this, we extend this model in three powerful yet still physically
realistic ways, described in the next section.

%%%%%%%%%%%%%%%%%%%%%%%%%%%%%%%%%%%%%%%%%%%%%%%%%%%%%%%%%%%%%%%%%%%%%%%%%%%%
\subsection{The Model \textsf{2D CCNTC}}
\label{subsec:2dccntc}

Two-dimensional models are the focus of this dissertation,
especially our factoring architectures in Chapters \ref{chap:factor-polylog}
and \ref{chap:factor-sublog}.

The first extension to \textsf{2D NTC} allows arbitrary planar graphs
with bounded degree, rather than a regular square lattice.
Namely, we assume qubits lie in a plane and edges are not allowed to intersect.
All qubits are accessible from above
or below by control and measurement apparatus.
Whereas \textsf{2D NTC} conventionally assumes each qubit
has four neighbors, we consider up to six neighbors in a roughly hexagonal
layout. The edge length in this model is no more than twice the edge length
in a regular \textsf{2D NTC} lattice.

The second extension is the realistic assumption
that classical control (CC) can
access every qubit in parallel, and we do not count these classical
resources in our implementation since they are polynomially bounded. The
classical controllers
correspond to fast digital computers which are
available in actual experiments and are necessary for constant-depth
communication in the next section.

We call an \textsf{AC} or \textsf{NTC} model augmented by these two extensions
\textsf{CCAC} and \textsf{CCNTC}, respectively.
We will frequently refer to \textsf{NTC} and \textsf{CCNTC} without
reference to dimensionality when we are making general statements about
nearest-neighbor architectures, with or without classical controllers,
regardless of the connectivity of the layout graph.
Let us now formalize our model for \textsf{2D CCNTC},
with definitions that are (asymptotically) equivalent to those in 
\cite{Rosenbaum2012}.

In our third extension, we restrict our gate set to be the
fault-tolerant, finite, fixed basis $\mathcal{G}$ defined
in the previous section and repeated below.
Traditionally, quantum architecture papers that
implement Shor's factoring algorithm have focused exclusively on
nearest-neighbor constraints for the layout graph (see Section \ref{sec:fpl-related} for a full
bibliography). A crucial difference in our architectures is
the use of this more realistic circuit basis.
We retain Rosenbaum's name \textsf{2D CCNTC}, although
his work does not necessarily use the same basis.

Now we define our formal model combining these three
extensions.

\begin{definition}
A \textsf{2D CCNTC} architecture consists of

\begin{itemize}
\item a quantum computer $QC$ which is represented by a planar graph $(V,E)$. A
node $v \in V$ represents a qubit which is acted upon in a circuit, and an
undirected edge $(u,v) \in E$ represents 
an allowed two-qubit interaction between qubits $u,v \in V$. Each node has
degree at most $6$.
\item a circuit basis $\mathcal{G} = \{X, Z, H, Toffoli, CNOT, MeasureZ\}$.

\item a deterministic machine (classical controller) $CC$ that applies a sequence
of concurrent gates in each of $D$ timesteps.
\item In timestep $i$, $CC$ applies a set of
gates $G_i = \{g_{i,j} \in \mathcal{G} \}$.
Each $g_{i,j}$ operates in one of the following two ways:
\begin{enumerate}
\item It is a single-qubit gate from $\mathcal{G}$ acting on a single qubit $v_{i,j} \in V$
\item
It is the gate CNOT from $\mathcal{G}$ acting on two qubits $v^{(1)}_{i,j}, v^{(2)}_{i,j} \in V$ where
$\left(v^{(1)}_{i,j}, v^{(2)}_{i,j}\right) \in E$
\end{enumerate}
All the $g_{i,j}$ can only operate on
disjoint qubits for a given timestep $i$. We define the support of $G_i$
as $V_i$, the set of all qubits acted upon during timestep $i$.

\begin{equation}
V_i = \bigcup_{j: g_{i,j} \in G_i} v_{i,j} \cup v^{(1)}_{i,j} \cup v^{(2)}_{i,j}
\end{equation}

\end{itemize}
\end{definition}

We can then define the three conventional circuit resources in this model.

\begin{description}
\item[circuit depth ($D$):] the number of concurrent timesteps.
\item[circuit size ($S$):] the total number of (non-identity) gates applied
from $\mathcal{G}$, equal to $\sum_{i=1}^D |G_i|$.
\item[circuit width ($W$):] the total number of qubits operated upon by
any gate, including inputs, outputs, and ancillae. It is equal to $| \bigcup_{i=1}^D V_i|$.
\end{description}

We observe that the following relationship holds between the circuit resources.
The circuit size is bounded above by
the product of circuit depth and circuit width, henceforce known as the
\emph{depth-width product}. This inequality holds since in the worst case,
every qubit is acted upon by a gate for every timestep of a circuit.
The circuit depth is also bounded above by the size, since in the worst case,
every gate is executed serially without any concurrency.

\begin{equation}
D \le S \le D\cdot W
\label{eqn:depth-width}
\end{equation}

The set $\mathcal{G}$ includes measurement in the $Z$ basis, which is
actually not a unitary operation but which may be slower than unitary
operations in actual practice \cite{DiVincenzo2007}.
Therefore we count it in our resource
estimates.
All other gates
in $\mathcal{G}$ form a universal set of unitary
gates \cite{Kitaev2002}.
 In this paper we
will treat the operations in $\mathcal{G}$ as \emph{elementary gates}, with
unit depth and unit size.
We also define a useful measurement in the Bell basis using operations
from $\mathcal{G}$. A circuit performing this measurement is shown
in Figure \ref{fig:bell-measure} and has depth $4$,
size $4$, and width $2$.

\begin{figure*}[tb!]
\begin{center}
\begin{displaymath}
\begin{array}{ccc}
\Qcircuit @C=1em @R=1em {
& \qw & \multimeasureD{1}{\mbox{Bell}} & \cw & \rstick{j} \\
& \qw & \ghost{\mbox{Bell}}            & \cw & \rstick{k}
}
& \qquad \equiv \qquad &
\Qcircuit @C=1em @R=1em {
& \qw & \ctrl{1} & \qw & \gate{H} & \qw & \meter & \cw & \rstick{j} \\
& \qw & \targfix & \qw & \qw      & \qw & \meter & \cw & \rstick{k}
}
\end{array}
\end{displaymath}
\centerline{}
\caption{A circuit for measurement in the Bell state basis.}
\label{fig:bell-measure}
\end{center}\end{figure*}

Counting gates from $\mathcal{G}$ as having unit size and unit depth
is
an overestimate compared to the models \textsf{NTC} and \textsf{AC} which are
used in previous works.
We intend
for this more pessimistic estimate to reflect the practical difficulties
in approximating these gates using a non-Clifford gate in a fault-tolerant way,
such as the $T$ gate or the Toffoli gate
\cite{Fowler2011}.
However, we will still be able to use this model to achieve asymptotically
superior architectures.

Next, we will show how the
seemingly simple basis $\mathcal{G}$ can nevertheless allow it to simulate
long-range interactions on a contiguous lattice using some
communication building blocks.
For each building block, including those for arithmetic in
Chapter \ref{chap:factor-polylog},
we provide closed-form equations upper-bounding the required circuit
resources as a function of the circuit width (input size) $n$.
We will use the
term \emph{numerical upper bound} to distinguish these formulae from asymptotic
upper bounds.