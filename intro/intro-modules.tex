\section{Modules and Long-Range Interactions}
\label{sec:intro-modules}

%%%%%%%%%%%%%%%%%%%%%%%%%%%%%%%%%%%%%%%%%%%%%%%%%%%%%%%%%%%%%%%%%%%%%%%%%%%%%
%% Subsection 0: Motivation, why modules?

Unfortunately, the model \textsf{2D CCNTC} has several shortcomings
which are highlighted by the constant-depth communication in the
previous section. First and foremost, it is difficult to realize physically. 
A single, contiguous
2D lattice that contains an entire quantum architecture
may be prohibitively large to manufacture or too difficult to control
coherently. In practice,
scalable experiments already prefer to use many
smaller quantum computers that communicate by means of shared
entanglement \cite{Monroe2012}. Second, for highly parallelized
algorithms, the amount of quantum information undergoing active
computation is so large that communicating data across the
quantum processor consumes more resources than the computation
itself. Finally, as quantum architects we are interested in distilling
the essential locality of a computation, and distinguishing it from
communication (which itself may be an interesting problem). All of these
notions will be made more rigorous below, when we develop a new model to
overcome these limitations.

Running an
algorithm on a single \textsf{2D CCNTC} architecture makes it
sensitive to the geometry of the different computation sites and how they
are arranged on the same lattice. Calculating the circuit resources of different
geometric arrangements will have vastly different numerical constants but will asymptotically
be the same. We would like a model that abstracts away this sensitivity.
The key to characterizing this essential locality is the long-range interaction. Rather than simulating them using short-range interactions,
we account for them as separate circuit resources. We define our
new model as a
\emph{hybrid architectural model}, one which accounts for long-range
and short-range interactions separately and also constrains them
according to independent graphs.

In fact, an example of this hybrid approach has already been proposed
in the \emph{quantum multicomputer} of Van Meter et al.,
which we abbreviate as \textsf{QMC} \cite{VanMeter2008}.
This hybrid model is also called a distributed quantum computer since it contains several communicating nodes operating in parallel.
Our approach is
very similar in that we divide a computation, which normally occurs on a single connected qubit graph $G$, into smaller subdivisions connected in
a graph $\overline{G}$. If two qubits $\ket{u}$ and $\ket{v}$ are connected
in $G$, then their two containing subdivisions $\overline{u} \ni \ket{u}$
and $\overline{v} \ni \ket{v}$ are connected in $\overline{G}$.
Like \textsf{QMC}, we also include ``flying qubits,'' or shared
entanglement, to allow for long-range, inter-node interactions.
Finally, we also regard short-range interactions between
qubits in the same node as being faster and cheaper, although this is not
a formal part of the model.

However, our model is different from \textsf{QMC} in several crucial
respects. We call our circuit subdivisions \emph{modules}
to emphasize the following differences from
\textsf{QMC} nodes.\footnote{This should not be
confused with the word ``modular'' as in ``modular arithmetic'' or as
referring to the modulus $m$ that we are trying to factor.}
Each module is a self-contained \textsf{2D CCNTC} lattice, where the qubits
are constrained to have nearest-neighbor interactions. Qubits in a
\textsf{QMC} node,
in contrast, are \textsf{AC} circuits with arbitrary connectivity. The
Van Meter paper is more empirical, providing numerical calculations
based on estimates of current physical technology as well as detailed
schematic implementations of long-range teleportations. Our work is more
abstract, providing asymptotic formulae for circuit resources. Importantly,
our model allows concurrent teleportations, even among non-disjoint
pairs of modules, whereas \textsf{QMC} nodes can only perform long-range
teleportations one-at-a-time. Finally,
while both \textsf{QMC} nodes and modules have configurable size, we have come to the
conclusion that modules should contain $O(n)$ qubits each, whereas
the Van Meter paper concludes that nodes have an optimal $O(1)$ size for 
factoring.
We discuss module size in more detail in Section \ref{subsec:module-future}.

A related work by Beals et al. \cite{Beals2012} describes how to achieve
low-depth, generic simulations of any \textsf{AC} circuit, either by
rearranging qubits using a sorting network on \textsf{NTC} or
by mapping it to \textsf{QMC}
(even with fixed-size nodes connected in a \textsf{LNN} topology). The
depth overhead in either case for an $n$-qubit \textsf{AC} circuit is $O(\log^2 n)$. Their first approach of \textsf{NTC} sorting networks can be used
as an alternative to hybrid architectures. As we will see later in
Chapter \ref{chap:factor-polylog}, the asymptotic depth of the generic sorting network approach will be the same as our ``hand-crafted'' architecture for factoring,
but possibly with worse constants and asymptotically larger width and size. 

We have now described the motivation for a new hybrid architectural model.
We have described a general feature set which
allows us to capture the essential locality of a quantum algorithm while
still accounting for long-range interactions.

%%%%%%%%%%%%%%%%%%%%%%%%%%%%%%%%%%%%%%%%%%%%%%%%%%%%%%%%%%%%%%%%%%%%%%%%%%%%
\subsection{The Model \textsf{2D CCNTCM}}
\label{subsec:2dccntcm}

The main feature of our new submodel is that modules
and long-range teleportations between them are nodes and edges, respectively,
in a higher-level planar graph. The teleportations each transmit one qubit
from one module to another, from any location within the source module
to any location within the destination module, making use of the
omnipresent classical controller.

We do not discuss the physical technology underlying the long-range
teleportation other than acknowledging that it is based on shared
entanglement. Namely, the two modules which are the source and destination
of any long-range teleportation must each possess one half of an
Einstein-Podolsky-Rosen (EPR) pair, as shown in Equation \ref{eqn:epr}.
%
\begin{equation}
\normtwo \left( \ket{00} + \ket{11} \right)
\label{eqn:epr}
\end{equation}

%Therefore, each long-range teleportation includes a preprocessing step in
%which the EPR pairs are generated and transported, along the bounded-degree
%module graph, to the modules where future teleportations will take place.

 The modules can be arbitrarily far
apart physically and have arbitrary connectivity. This is a reasonable
assumption because entangled pairs can be generated and distributed between
any two modules. This generation and distribution is counted as part of
the (preprocessing) costs of the long-range teleportation itself as a
one-to-one correspondence in the worst case. One could imagine that 
generating and distributing many entangled pairs in a batch might be more
efficient physically than doing so for individual pairs. Moreover, the
distribution of entangled pairs may occur along a constrained graph of
module connectivity, but such constraints can be overcome with only
polylogarithmic depth overhead as previously noted \cite{Beals2012}.

A single module can be part of multiple teleportation operations in a single
timestep, as long as they involve disjoint qubits within the module.
We justify this assumption in that it is
possible for a quantum computer to share multiple EPR pairs with a
second quantum computer, and even multiple other quantum computers,
simultaneously.

We call this new model \textsf{2D CCNTCM}, since it generalizes the
model \textsf{2D CCNTC} to have more than one module. Equivalently,
each module corresponds exactly to a \textsf{2D CCNTC} architecture.
All modules can share a single omnipresent
classical controller or a collection of multiple, inter-communicating classical controllers.
A graphic depiction of three modules in \textsf{2D CCNTCM} is shown in
Figure \ref{fig:modules}.

\begin{figure}[btp!]
\begin{center}
\includegraphics[width=4in]{factor-polylog/figures/modules.pdf}
\end{center}
\caption{Three modules in the \textsf{2D CCNTCM} model.}
\label{fig:modules}
\end{figure}

\begin{definition}
A \textsf{2D CCNTCM} architecture consists of

\begin{itemize}
\item a quantum computer $\overline{QC}$ which is represented by a graph
$(\overline{V},\overline{E})$ of arbitrary degree.
A node $\overline{v} \in \overline{V}$ represents a module,
or a graph $(V,E)$
from a \textsf{2D CCNTC} architecture defined previously on page \pageref{subsec:2dccntc}.
An
undirected edge $(\overline{u},\overline{v}) \in \overline{E}$ represents an
allowed teleportation from any qubit in module $\overline{u}$ to
another qubit in module $\overline{v}$.
\item a circuit basis $\mathcal{G} = \{X, Z, H, T, T^{\dagger}, CNOT,
MeasureZ\}$
for the qubits \emph{within the same} modules which is the same as for \textsf{2D CCNTC}.
\item
an additional operation $Teleport$ which only operates on qubits
\emph{in
different} modules.
\item a deterministic machine (classical controller) $\overline{CC}$ that applies a sequence
of concurrent gates in each of $D+\overline{D}$ timesteps.
This can be a separate classical controller
for every pair of modules.
\item In timestep $i$, $\overline{CC}$ applies
gates $G_i = \{g_{i,j} : g_{i,j} \in \mathcal{G} \cup \{Teleport\} \}$.
That is, there are two kinds of timesteps with respect to the kinds of gates
which operate within them.
\begin{enumerate}
\item In the first kind, gates are exclusively from $\mathcal{G}$, and
they operate within modules as described
for \textsf{2D CCNTC} above. We say there are $D$ such timesteps.
\item In the second kind, gates are exclusively $Teleport$ gates between two qubits $v^{(1)}_{i,j} \in \overline{v}_1$ and
$v^{(2)}_{i,j} \in \overline{v}_2$ for
(possibly non-distinct) modules $\overline{v}_1, \overline{v}_2 \in \overline{V}$.
Again, all such qubits much be distinct within a timestep but the
modules need not be distinct.
We say there are $\overline{D}$ such timesteps.
\end{enumerate}

Again, we define the support of $G_i$
as $V_i$, the set of all qubits acted upon by any $g_{i.j}$, which
includes all the modules.
\begin{equation}
V_i = \bigcup_{j: g_{i,j} \in G_i} v_{i,j} \cup v^{(1)}_{i,j} \cup v^{(2)}_{i,j} 
\end{equation}

\end{itemize}
\end{definition}

We measure the efficiency of a circuit in this new module using not just
the three conventional circuit resources defined for \textsf{2D CCNTC}, but with three novel resources
based on modules.

\begin{description}

\item[circuit depth ($D$):] the depth of all operations over all modules.
\item[circuit size ($S$):] the total operations on all qubits within all
modules.
\item[circuit width ($W$):] the total number of qubits, equal to the sum of all the module sizes.
%, depicted in Figure \ref{fig:resources}:
\item[module depth ($\overline{D}$):] the depth of parallel teleportations
of disjoint qubits between modules.
\item[module size ($\overline{S}$):] the number of total qubits teleported between any two modules over all timesteps.
\item[module width ($\overline{W}$):] the number of modules whose qubits are
acted upon during any timestep.

\end{description}

We can define the size of a module $\overline{v} \in \overline{V}$
as the number of qubits within it,
where all the module sizes add up to the total circuit width.

\begin{equation}
\sum_{\overline{v} \in \overline{V}} |\overline{v}| = W
\end{equation}

%%%%%%%%%%%%%%%%%%%%%%%%%%%%%%%%%%%%%%%%%%%%%%%%%%%%%%%%%%%%%%%%%%%%%%%%%%%%
\subsection{Module Size and Model Comparisons}
\label{subsec:module-compare}

Modules in a \textsf{2D CCNTCM} architecture do not all need to have the same
size. However, it is useful to upper bound the size of any module and call
this the \emph{module size} for the entire architecture, denoted $\hat{W}$.
In this 
dissertation, we only consider \emph{uniformly-sized} \textsf{CCNTCM} architectures, ones where
the module size $\hat{W}$
is upper-bounded by the average of all the module sizes, up to a constant
factor. In the definition below, we iterate over all modules $\overline{v}$
in the set of all modules $\overline{V}$, where the number of modules is
defined as $\overline{W} = |\overline{V}|$.

\begin{equation}
\hat{W} = \max_{\overline{v} \in \overline{V}} |v| \equiv O(W / \overline{W})
\end{equation}

We argue that module size is a key, configurable parameter of any
hybrid architecture that greatly affects its resource usage. To see this,
we will now compare the overhead of simulating a generic architecture
(an \textsf{AC} circuit) on three different models with nearest-neighbor
interactions: \textsf{2D NTC}, \textsf{2D CCNTC}, and \textsf{2D CCNTCM}.
This will demonstrate how \textsf{2D CCNTCM} simplifies our view of computation
versus communication resources.

%Related to module
%size, we define two restrictions we can make to the \textsf{CCNTCM} model
%to simplify the analyses in this dissertation. Neither of these is strictly
%part of the \textsf{CCNTCM} model as we've defined it above.

%\begin{definition}
%A \emph{one-way} \textsf{CCNTCM} architecture is one where each module
%only interacts with other modules either before or after computation
%within the module itself. That is, all long-range, inter-module 
%teleportations that are counted as part of $\overline{S}$ take place in
%a cohort of operations either before or after the short-range interactions
%and gates that are counted as part of $S$. Furthermore, all
%inter-module teleportations occur in one-direction. Once 

%The second is that we only consider \emph{efficient} partitionings such
%that a module only participates in long long-range teleportations 
%\end{definition}

% IV. Compare & Contrast

In each of these three models, we must simulate the long-range
interactions of a completely general \textsf{AC} circuit (with
depth $D$, size $S$, and width $W$), using only the
constraints of each particular model. We will define a depth overhead
of $\tilde{D}$, a size overhead $\tilde{S}$, a width overhead $\tilde{W}$
associated purely with communication (not computation) associated with
this long-range simulation. The first two models do not use modules
($\overline{W} = 1$),
whereas the last model has a module size parameter $\hat{W}$ as defined above, with $1 \le \hat{W} \le W$ depending on the number of modules
$\overline{W}$. Modules are now seen as an elegant framework for
interpolating our hybrid model between a completely unconstrained and a completely nearest-neighbor architectural model. When $\overline{W} = 1$,
there is a contiguous lattice which corresponds exactly to \textsf{2D CCNTC}.
When $\overline{W} = W$, there is an unconstrained network of single qubits
which corresponds exactly to \textsf{AC}.
%% TODO I really love something about the sentences below,
%% find some way to put them back in
%This new model then, \textsf{2D CCNTCM}, is a hybrid
%-between \textsf{AC} and \textsf{2D CCNTC} with module size as a parameter.
%-\textsf{AC} and \textsf{2D CCNTC} represent the two extremes of long-range
%-interactions, where \textsf{AC} permits all long-range interactions and
%-\textsf{2D CCNTC} permits none.

Table \ref{tab:mod-comp} summarizes the resources used by these models
to to simulate an \textsf{AC} circuit \emph{with lowest depth}.
In this comparison, \textsf{AC} is
the most efficient architecture in all resources
but also the most unrealistic. All other
architectures pay an overhead for simulating interactions with
varying degrees of nearest-neighbor realism.

\begin{table}
\begin{tabular}{|c|c|c|c|c|}
\hline
Circuit Resources & \textsf{AC} & \textsf{2D NTC} & \textsf{2D CCNTC} & \textsf{2D CCNTCM} \\
\hline
$D + \tilde{D}$ & $D$ & $D\cdot (1 + O(\sqrt{W})$    & $D(1 + O(1))$   & $D(1 + O(1))$ \\
$S + \tilde{S}$ & $S$ & $S + D\cdot O(W)$ & $S + D\cdot O(W^2)$ & $S + D\cdot \overline{W} \cdot O(\hat{W}^2)$ \\
$W + \tilde{W}$ & $W$ & $W$               & $W(1 + W)$    & $W(1 + \hat{W})$\\
\hline
$\overline{D}$  & $D$ &                   &               & $\overline{D}$\\
$\overline{S}$  & $S$ &                   &               & $\overline{S}$\\
$\overline{W}$  & $W$ &                   &               & $\overline{W}$\\
\hline
\end{tabular}
\caption{A comparison of simulating \textsf{AC} interactions on three
other models. Unmarked entries signify $\overline{D}=\overline{S} = 0$ and
$\overline{W} = 1$.}
\label{tab:mod-comp}
\end{table}

A \textsf{2D NTC} circuit can only
use $SWAP$ gates to rearrange each of the $W$ qubits after each of $D$
timesteps, so that all gates can occur between nearest neighbors. We choose
to use a sorting network approach of Beals et al. \cite{Beals2012} which
has the following consequences. The sorting that occurs after every timestep
has depth $O(\sqrt{W})$ and total communication depth $\tilde{D} = D\cdot O(\sqrt{W})$.
This sorting after every timestep has size $O(W)$ (the maximum number
of $SWAP$'s required) and total communication size
$\tilde{S} = O(D\cdot W)$. Because sorting happens in-place,
no additional ancillary qubits are needed ($\tilde{W} = 0$). 

A \textsf{2D CCNTC} circuit achieves lowest-depth by using the constant-depth
communication of Section \ref{sec:intro-cdc} and Rosenbaum's
construction for reordering qubits \cite{Rosenbaum2012}. Other, more
efficient re-orderings may be possible for a specific algorithm, but
the Rosenbaum reordering is completely general. This doesn't
asymptotically increase the depth ($\tilde{D} = D\cdot O(1)$) but it does
increase size and width by the ancillary qubits needed for the reordering
grid ($\tilde{W} = W^2$, $\tilde{S} = O(W^2)$).

In the first two models, there was only one contiguous lattice
($\overline{W} = 1$) and therefore no long-range, inter-module interactions
($\overline{D} = \overline{S} = 0$). However, in the third model, we now
consider multiple modules which represent a partitioning of the qubits
in the original \textsf{AC} circuit. In the worst case, we still need to
perform a Rosenbaum reordering after each of $D$ timesteps $(\tilde{D} = D\cdot O(1)$), but this time
only within smaller modules of size at most $\hat{W}$. This gives the
number of communication operations $\tilde{S} = D\cdot \overline{W} \cdot O(\hat{W}^2)$.

The comparison between \textsf{2D NTC} and \textsf{2D CCNTC} underscores the
tradeoff between a large depth on one hand and a large ancillary qubit
overhead on the other hand.
The comparison between \textsf{2D CCNTC} and \textsf{2D CCNTCM} has two
noteworthy features.
There is a reduction in number of intra-module operations over \textsf{CCNTC} 
if the number of modules is non-constant:
$\overline{W} = \omega(1)$. However, this reduction in $S$ must be
balanced with the relative cost of $\overline{S}$, which is non-zero in
\textsf{CCNTCM} and zero in all other models.
Only physical experiments can determine the best tradeoff, but our
model will facilitate such a calculation. Achieving exact bounds on
$\overline{S}$ will be calculated later for factoring in Chapter
\ref{chap:factor-polylog}.

%We can make an observation analogous to Equation \ref{eqn:depth-width} but
%for modules in Equation \ref{eqn:module-depth-width}.

%\begin{equation}
%\overline{D} \le \overline{S} \le \overline{D}\cdot \overline{W}
%\label{eqn:module-depth-width}
%\end{equation}

%%%%%%%%%%%%%%%%%%%%%%%%%%%%%%%%%%%%%%%%%%%%%%%%%%%%%%%%%%%%%%%%%%%%%%%%%%%
\subsection{Objections to Modules and Future Directions}
\label{subsec:module-future}

Although we have demonstrate the usefulness of modules in capturing
the locality of a quantum algorithm, our new \textsf{CCNTCM} model also
contains some weaknesses. Chief among them are the possibilities of
wasteful modules
and the misrepresentation of the costs of long-range teleportation. We will
address objections to these weaknesses here before presenting useful
open questions needed to fully explore the potential of modules in
hybrid architectures.

It is allowable under the \textsf{CCNTCM} model for an architecture
to have only a small,
or constant, number of qubits in each module be useful for computation.
Alternatively, the qubits could be partitioned among modules such that most
computationally useful interactions must be long-range and involve
qubits in other modules.
Such modules are wasteful in the sense that they are not capturing any
locality.
We argue that such an architecture, and partitioning of qubits, can be
further optimized to use fewer resources, including both long-range
$\overline{S}$-type interactions and short-range $S$-type interactions.
Even an \textsf{AC} circuit can have extraneous
qubits and extraneous gates that are not useful for computation.
Bad implementations are possible with any model, and so the wasteful module
objection is not specific to \textsf{CCNTCM}.

Another objection to \textsf{CCNTCM} is that it may conceal the true
costs of long-range teleportation. Whereas the models \textsf{NTC} and
\textsf{CCNTC} present the communication costs of a circuit in a unified
way alongside computation, it can be argued that \textsf{CCNTCM} is unfairly
optimizing $S$ and $W$ compared to these two models. Indeed, for any
sufficiently localized algorithm, a \textsf{CCNTCM} architecture can be designed with asymptotically lower $S$ and $W$ than a corresponding \textsf{NTC} or \textsf{CCNTC}. However, the costs of communication are not being
misrepresented, they are just counted separately
in the module resources $\overline{D}$, $\overline{S}$, and
$\overline{W}$. Later, when physical experiments have determined the
true cost of each resource, architects can make informed decisions about
the best tradeoffs between module size and long-range interactions.
The separate accounting of \textsf{CCNTCM} reflects the belief that
practically, $\overline{S}$-type operations (or their equivalent simulation
on a contiguous lattice) will always be much slower
and expensive than $S$-type operations. A numerical comparison between
\textsf{2D CCNTC} and \textsf{2D CCNTCM} for one of our
factoring architectures is presented in Section \ref{sec:fpl-results}.

Finally, the distribution of entangled pairs necessary to allow
arbitrary connectivity of long-range teleportations has some
potential pitfalls. The polylogarithmic depth overhead
of EPR pair distribution is subsumed by the polylogarithmic
depth of our first factoring architecture in
Chapter \ref{chap:factor-polylog}, but not by our
second, sublogarithmic-depth architecture in
Chapter \ref{chap:factor-sublog}. These issues are
dealt with in more detail in the conclusions of those chapters.
More importantly, entangled pairs are likely to be generated
and distributed ``just-in-time'' to avoid unnecessary
error-correction in storing them.
Therefore, this part of shared entanglement is not included as part
of \textsf{CCNTCM} because
the technology and optimal strategy for entangled pair management
is still not well-developed.
Physical implementations could use a completely-connected
optical switch, such as the one described in \cite{Monroe2012}.
Furthermore, the qubits in a long-range teleportation always begin and end
on the peripheries of their source and destination modules,
on the edges of their respective \textsf{2D CCNTC} lattices.
Therefore, no additional teleportation is necessary to move
those qubits into position for computation.

Beyond these objections, there is still more work to be done to fully
understand the usefulness of modules and the \textsf{CCNTCM} model.
First and foremost, the connectivity
of the modules themselves. We allow teleportation between any two modules
because an entangled pair can be generated and distributed between any
two modules. However, in reality, this distribution might occur along
constrained paths between modules. Although 

Furthermore, \textsf{CCNTCM} can be defined for other dimensionalities,
which constrain the \textsf{CCNTC} lattices in each modules.
However, we concentrate on the \textsf{2D} case in this dissertation,
although the module graph $\overline{G}$ could be connected with
a \textsc{1D} topology independently of the modules, which could contain
\textsc{2D} lattices. It is a subject of future research to determine
the optimal setting of connectivities of the graphs $G$
and $\overline{G}$ independently.

Another useful investigation would be
the particular tradeoff between the number of
long-range interactions $\overline{S}$ and the module size $\hat{W}$.
In our comparison in the previous section, we make no assumptions about the 
partitioning of qubits into
modules or how the $\overline{S}$ long-range interactions are distributed
in the circuit. This prevents us from relating $\overline{S}$,
$\overline{W}$, and $W$ in the general case.
What particular features of an algorithm make this tradeoff better or worse?
Unfortunately, such investigations are beyond the scope of the current
work. In Chapter \ref{chap:factor-polylog}, we use the heuristic of linear 
size modules ($\hat{W} = O(n)$) to match the amount of computation and
communication within each module. This appears to give reasonably efficient
numerical results for our hybrid factoring architectures. We hope that
the importance of
both factoring and realistic hybrid architectures drive further work in this
area.

