\chapter{Introduction}
\label{chap:intro}

% I. Quantum computing and perspectives
The combination of existing scientific disciplines provides fertile ground
for the development of entirely new scientific discoveries. Computer science
and quantum physics are now coming together in this way to give birth to
quantum computing. Computer science and quantum physics originated
as contemporaries in the early part of the 20th century, culminating in
technology that decisively ended World War II. They have remained
largely independent until recently, but the common thread connecting them
has always existed.

From the perspective of computer science, engineering is problem-driven.
We humans wish to solve a problem, like calculating ballistic weapon
trajectories, and we work backwards from there. To calculate,
we need to perform fast
arithmetic on stored numbers, for which we need to design devices in a
certain configuration, for which we need certain materials from nature.
At heart, it is the study of structured problem-solving which happens to have a strong
mechanical bias. This perspective asks the questions: what do we want to do,
and how can we
do it? It is like a child going into a LEGO store with a pre-existing plan
and only buying the bricks needed. It is an efficient plan for brick procurement,
but not
necessarily creative.

From the perspective of quantum physics, engineering is phenomena-based.
We humans wish to describe reality and observe nature. We discover objects
and effects in nature, and afterwards we may
then discover that they
can be useful in performing some useful task. An example is
the discovery of quantum tunnelling and its subsequent use to build a
field-effect transistor (FET) for digital electronics. This perspective asks the
questions: what is available to us, and what could we do it? It is like a child
going into a LEGO store with no pre-conceived plan, investigating all the
parts and dreaming up ways to use them. This will generate completely new
designs and models, but it is much slower than brick-shopping.

These perspectives are intertwined and serve complementary purposes,
and we often alternate
between the two of them. In our transistor example, we probably would not
have thought to build a FET if we had not already needed fast electrical
switching, for which vacuum tubes were proving inadequate. Moreover, we can
go beyond using quantum physics to simulate a classical device. There is
a remarkable example
of a completely new algorithm designed using the building blocks of
quantum physics. In 1994, Peter Shor contributed a quantum factoring algorithm
to solve a human problem efficiently,
one that has defied all classical algorithms before or since.
where no known classical algorithm, previously or since.
This kickstarted the entire field of quantum computing; united many
disciplines including computer science, physics, engineering, and
mathematics; and provided the foundation for this dissertation.

% II. Factoring
The problem of factorization, or factoring,
is simple to state and difficult to solve.
Given an integer $m$, return the integers $\{p_1, \ldots, p_k\}$ whose
product of their powers is $m$: $m = p_1^{\alpha_1}\cdots p_k^{\alpha_k}$.
It is a beautiful theoretical problem posed by Carl Friedrich Gauss in 1801,
one which he felt that every self-respecting scientist should devote their
efforts to solving. Factoring is an important problem in the theory of
numbers, which Gauss considered the crown jewel of mathematics, itself
considered the queen of the sciences.\footnote{\emph{Gauss zum Ged\"achtniss} (1856) by Wolfgang Sartorius von Waltershausen}
It is a quirk of cryptography that the hardness of problems can be used
to secure communication. Therefore, factoring also has immense practical
usefulness as the basis of the RSA cryptosystem, the most widely-used
cryptosystem around the world and on the internet.

The somewhat sinister,
but financially important, importance to
governments is the ability to gather intelligence on each other and on their
citizens. The importance to citizens is the ability to choose security
parameters so that their correspondence is protected for a limited time.
The organizers of the popular democratic uprisings in the Middle East known
as the Arab Spring in 2011 definitely benefitted from using a cryptosystem that was
secure against compromise given the resources of the regimes that they toppled.
A government able to hack the organizers' Twitter accounts or Facebook pages
could misdirect protest actions or identify activists for imprisonment.
Similarly, being able to accurately estimate the resources needed to factor
an RSA key of a certain size may help protect future citizens against
governments with a quantum computers, as governments will be the first
organizations interested in and able to afford such a device.

How can we reason about the computational resources needed to solve the
problem of factoring an $n$-bit number?
Before 1994, the best known algorithm was the
number field sieve (citation needed) which had a running time of
$exp(\sqrt[3]{log n})$. This can be considered a pre-quantum (classical)
algorithm in a pre-quantum era. Its sub-exponential (but still super-polynomial)
running time is considered intractable and empirical proof of its hardness.
This conjectured hardness is why users of the RSA cryptosystem above have faith
that their information is resistant to compromise, while at the same time
giving hope to the possibility of discovering some as yet unknown improvement.

In 1994, Peter Shor fulfilled this hope in a landmark
discovery that using the more general computational power of quantum physics,
factoring could be performed in running time $O(n^3\log n \log\log n)$ (citation
needed). 
The computational model in which this remarkable feat was achieved
made some assumptions about quantum computing devices, which are now
standard in the field. 
However, many remain skeptical of this achievement due to the perceived
plausibility of these assumptions, which is why RSA still remains the
most widely-used cryptosystem in the world.\footnote{Despite recommendations
from the NSA to switch to ECC, itself susceptible to quantum attacks (citation needed).}
These assumptions include that robust error-correction
can be performed to preserve fragile quantum information and that quantum gates
can be performed with high fidelity. Whether these assumptions can be
realized is an active area of experimental physics research, and is not the
topic of this current work. However, given that these assumptions will
eventually be realized, how can we optimize Shor's factoring algorithm?
Quantum computation is now a rapidly maturing field, and we cannot provide
an adequate background in this dissertation. The interested reader is referred
to the standard textbook in the field by Nielsen and Chuang \cite{Nielsen2000}.
Building upon this background, we can provide an introduction into the subfield
of quantum architecture.

% III. Quantum circuits and quantum architecture
First, we will introduce a quantum circuit.
Analogous to the classical circuit model, a quantum circuit consists of
a network, or graph, of gates operating on quantum bits (qubits). One can
arrange this graph such that directed edges represent qubits and
nodes represent reversible quantum gates with in-degree equal to out-degree.
See Figure \ref{fig:intro-qcirc} for a representation of a quantum circuit
as a special class of directed
acyclic graphs. All flows travel from the input state of the qubits at the
first timestep (on the far
left) to the final output state on the same qubits in the last timestep
(on the far right). This induces a natural definition of circuit depth,
size, and width, which we will make more formal later.

\begin{figure}
\caption{A quantum circuit as a directed acyclic graph.}
\label{fig:intro-qcirc}
\end{figure}

A quantum architecture is a quantum circuit with constraints. For this
dissertation, we will concern ourselves with two main constraints:
(1) the layout graph of qubits which constrain two-qubit gate interactions and
(2) the decomposition of all single-qubit gates to a discrete, finite,
set of gates which are universal combined with a fixed two-qubit gate.
Quantum architecture provides an intermediate layer in between working
with actual hardware (the physical technology in a laboratory) and the
abstract theory (algorithms divorced of all physical constraints).
As quantum architects, we can optimize algorithms in a general way which
is relevant to many kinds of laboratory experiments. At the same time,
we can take into account the more complicated nature of physical devices
to provide more pessimistic, accurate, and numerical resource estimates than more
high-level theoretical results.
The role of quantum architecture is two-fold. First,
to quantify and improve the resources---such as circuit depth,
size, and width mentioned above---required to run a quantum
algorithm on a system of varying degrees of realism. Second, to provide
tradeoffs between circuit resources, or conceptual knobs used to tune a
particular implementation. This allows experimentalists to configure a
quantum architecture to meet their needs in a laboratory while allowing
theorists to improve tradeoffs or design new algorithms which take
physical constraints into account.


% Point 
We are now ready to state the thesis that we investigate in this dissertation.

\begin{quote}
Studying the depth of quantum architectures can help us design quantum computers that can solve
human problems with a human lifetime.
\end{quote}

Here, or somewhere else, we must distinguish between quantum architecture as a field and quantum architecture as a
circuit with hardware constraints. Actually, the two definitions are pretty close.

We also need to provide preliminary definitions of a quantum bit.

This deserves some unpacking. A quantum architecture is a circuit designed to implement a particular algorithm given certain
hardware constraints. The depth of a quantum architecture is the amount of time it takes for the circuit to complete.
Given a particular input $x$, how long (in human clock time) does it take to return the output of some function $f(x)$?
Quantum computers are interesting because for some algorithms, namely factoring, the depth of a quantum circuit is
less than the depth of any known classical circuit. However, this depth may correspond to a running time which is
still much longer than a single human lifetime. In particular this author, who is 32 years old at the time of this
writing, is interested in techniques to optimize the successful completion of a problem within his own lifetime.

Therefore, one meaning of ``studying the depth'' means to \emph{decrease} depth. The main technique for decreasing
depth is to parallelize a quantum circuit, that is, to divide and duplicate information to allow them to be processed
in a parallel way, using multiple classical controllers. However, parallelizing can increase other circuit resources,
usually due to this information duplication. This introduces a time-space tradeoff. In practical reality, we cannot
decrease depth at the expense of other circuit resources, such as the amount of information in a form suitable for
operation over time.

Decreasing depth can increase circuit size (the total number of operations to be performed) and circuit width
(the total amount of quantum information that can exist at any one time). This increase may again far
outstrip human resources, and delay completion of a quantum computer's construction, let alone its
successful execution, until well after that author's death. Therefore, the useful study of depth to help
quantum computers solve human problems must also limit this time-space tradeoff overhead. It should also
involve defining the exact tradeoff that is useful to minimize.

We have outlined the problems of designing such a low-depth quantum architecture so far in a very general,
abstract form. In this thesis, we will make these concepts more concrete by describing a formalism for
quantum circuits and quantum computation that is well-known to the research community.
Building on this formalism, we will propose a new architectural model and present our solutions to factoring in them.


Section \ref{sec:intro-arch} will discuss architectural models. We will present the definition of an
architectural models and give examples of them from the
current literature, with names like \textsf{AC} and \textsf{NTC}. We will discuss the idea of architectural
sub-models, with names like \textsf{2D CCNTC}. Finally,
we will describe our contribution of a new architectural model, called \textsf{2D CCNTCM}, which is a weakly 2D,
hybrid architecture that introduces the idea of allowing both short-range (local) interactions and long-range
interactions, and providing a configurable tradeoff in between them.

In order to build long-range interactions out of short-range interactions, Section \ref{sec:intro-cdc} 
gives a thorough background on recent constant-depth communication techniques. These include three
basic constructs for teleporting, copying, and uncopying quantum information across an unbounded
lattice.

All operations on a quantum architecture can be reduced to single-qubit operations and two-qubit operations.
The latter operations are what we have been calling ``interactions,'' and we have studied how to overcome
constraints on nearest-neighbor interactions using the constant-depth communication techniques previously.

Ultimately, however, the limits to depth of a quantum architecture are
dependent on the single-qubit operations. 

The remainder of this chapter is devoted to the necessary background needed
to understand quantum architecture in general and factoring architectures
in particular. Quantum architecture has potentially many concerns, but we
focus on two of them: interaction distance and fault-tolerant quantum compiling.
Without loss of generality, the gates of a quantum circuit can be reduced to
single-qubit and two-qubit gates. In fact, we can use a fixed two-qubit gate
and combine it with arbitrary single-qubit gates.
Section \ref{sec:intro-basis} provides the necessary background for
understanding single-qubit gates, which do not impose any constraints on
which qubits can interact with each other.

, its relationship to quantum compiling,
recent constant-depth techniques in communicating quantum information across
a lattice, and most importantly,
the usefulness of architectural modules,
and the conc

\section{Quantum Gates and Circuit Bases}
\label{sec:intro-basis}

To compile, or implement, arbitrary quantum algorithms, we must construct circuits
out of gates from a universal set, which we call a \emph{circuit basis},
or just \emph{basis}.
This should not be confused with a basis for a vector space.
Therefore, we will now review quantum gates and how to combine them into
circuits. This procedure is known as quantum compiling, but we will not
delve into its details until we need them in Chapter \ref{chap:qcompile}.

A quantum gate on $n$-qubits is a $2^n \times 2^n$ unitary matrix
(an element of $U(2^n)$). We can consider this the overall circuit width.
Often, we find it useful to neglect a
global phase, since these cannot be measured in quantum mechanics.
However, a global
phase on a particular system $S$ may result in a measurable relative phase
in a large system $S'$ of which $S$ is a subsystem. Therefore, for our
purposes we will only distinguish between $U(2^n)$ and
$SU(2^n)$ in the few cases where it matters for
quantum compiling. The distinction between a quantum circuit
and a quantum gate is relative; often we consider a quantum gate as a
fundamental primitive of our physical technology, and a circuit as a
composite of these gates corresponding to a quantum algorithm.

In Section \ref{subsec:pauli} we
will review the Pauli single-qubit gates and their corresponding group.
In Section \ref{subsec:clifford} we will introduce the Clifford group.
In Section \ref{subsec:controlled} we will introduce controlled operations
and the Toffoli gate.
In Section \ref{subsec:qcompile-single} we will discuss \emph{single-qubit compiling}
and how a
general single-qubit gate can be compiled into rotations about Bloch sphere axes.
In Section \ref{subsec:distance} we will present distance metrics to
measure the quality of our single-qubit (and later multi-qubit) approximations.
In Section \ref{subsec:qcompile-bases} we will finally
define what it means for a gate set to be universal.

%%%%%%%%%%%%%%%%%%%%%%%%%%%%%%%%%%%%%%%%%%%%%%%%%%%%%%%%%%%%%%%%%%%%%%%%%%%%%%
\subsection{Pauli Group}
\label{subsec:pauli}

We review here the Pauli group on one qubit, $\mathcal{P}_1 = \{I, X, Y, Z\}$.
These last three represent
rotations of $\pi$ on the Bloch sphere about the $x$-axis, $y$-axis, and $z$-axis,
using the homomorphism between $SU(2)$ and $SO(3)$. The $2\times 2$ identity matrix
is denoted $I$. The group $\mathcal{P}_1$ also serves as a complex vector
basis for generating elements of $U(2)$.

\begin{IEEEeqnarray}{rcrclCrcrcl}
X & = & \sigma_x & = &
 \left[
  \begin{array}{cc}
    0 & 1 \\
    1 & 0 \\
  \end{array} \right]
& \qquad &
Y & = & \sigma_y & = &
 \left[
  \begin{array}{cc}
    0 & i \\
   -i & 0 \\
  \end{array} \right]
\\
Z & = & \sigma_z & = &
 \left[
  \begin{array}{cc}
    1 & 0 \\
    0 & -1 \\
  \end{array} \right]
& \qquad &
I & = & \sigma_0 & = &
 \left[
  \begin{array}{cc}
    1 & 0 \\
    0 & 1 \\
  \end{array} \right]
\end{IEEEeqnarray}

We define the Pauli group $\mathcal{P}_n$ on $n$ qubits as the set of
all $n$-qubit operators which are tensor products of elements from
$\mathcal{P}_1$.

%%%%%%%%%%%%%%%%%%%%%%%%%%%%%%%%%%%%%%%%%%%%%%%%%%%%%%%%%%%%%%%%%%%%%%%%%%%%%%
\subsection{The Clifford Group}
\label{subsec:clifford}

We define the normalizer of $\mathcal{P}_n$ as the
Clifford group $\mathcal{C}_n$ on $n$ qubits.

\begin{equation}
\mathcal{C}_n = \{ C \in U(2^n) | CPC^{\dagger} \in \mathcal{P}_n \quad \forall P \in \mathcal{P}_n \}
\end{equation}

Of particular interest to us is the two-qubit Clifford group $\mathcal{C}_2$,
which is generated by the following matrices:

\begin{equation}
\mathcal{C}_2 = \langle H, S, CNOT \rangle
\end{equation}

The first two Clifford generator matrices are single-qubit gates ($2 \times 2$ unitary matrices) and
their inclusion means they can be applied on either the first or the second
qubit.\footnote{Standard convention writes $H_i$ to mean $H$ on qubit $i$ and likewise for $S_i$.}
The matrix $H$ is known as the Hadamard gate, and it is a special case of the
general Walsh-Hadamard transform. It is its own adjoint: $H^{\dagger} = H$.
The matrix $S$ is known as the phase gate, and it can be considered the
``square root'' of the Pauli $Z$ gate (up to a phase): $S^2 = Z$.
Equivalently, it can be viewed as a $\pi/2$ rotation about the Bloch sphere
$z$-axis, and its adjoint $S^{\dagger}$ is the reverse rotation of $-\pi /2$.
These matrices are defined below.

\begin{equation}
H = \normtwo
 \left[
  \begin{array}{cc}
    1 & 1 \\
    1 & -1 \\
  \end{array} \right]
\qquad
S = 
 \left[
  \begin{array}{cc}
    1 & 0 \\
    0 & i \\
  \end{array} \right]
\qquad
S^{\dagger} = 
 \left[
  \begin{array}{cc}
    1 & 0 \\
    0 & -i \\
  \end{array} \right]
\end{equation}

The Hadamard matrix also has the special property that it changes between the
$X$ basis and the $Z$ basis, that is, the vector basis for single-qubit
states consisting of eigenstates of the Pauli $X$ and Pauli $Z$ gates,
respectively. In fact, using the identities $X = HZH$ and $S^2 = Z$, it
is easy to see why $X$ and $Z$ are often listed as generators of the
Clifford group as well.

The last Clifford generator matrix is a two-qubit gate (a $4 \times 4$ unitary matrix) which
also represents a \emph{controlled} operation. That is, based on the
$\ket{1}$ component of the \emph{control} qubit, it applies a single-qubit
gate (in this case, Pauli $X$) to the \emph{target} qubit.
In fact,
both CNOT and $X$ are also fundamental gates in classical reversible
logic as well, where $X$ is also the Boolean $NOT$ gate on classical bits.
That is why the gate is called CNOT, for ``controlled-NOT.'' Its inclusion
in the generating set for $\mathcal{C}_2$ means that it can be applied
in either direction: with control on qubit 1 and target on qubit 2 or
vice versa. CNOT is defined below.
%
\begin{equation}
CNOT = 
 \left[
  \begin{array}{cccc}
    1 & 0 & 0 & 0 \\
    0 & 1 & 0 & 0 \\
    0 & 0 & 0 & 1 \\
    0 & 0 & 1 & 0
  \end{array} \right]
\end{equation}
%
Likewise, the general Clifford group on $n$ qubits $\mathcal{C}_n$
can be generated from the same set
as $\mathcal{C}_2$, with gates understood to apply to any of the $n$ qubits.
With three CNOTs, we can implement the $SWAP$ gate which exchanges the
states of two qubits.
%
\begin{equation}
SWAP = 
 \left[
  \begin{array}{cccc}
    1 & 0 & 0 & 0 \\
    0 & 0 & 1 & 0 \\
    0 & 1 & 0 & 0 \\
    0 & 0 & 0 & 1
  \end{array} \right]
\end{equation}
%
The gate CNOT has historical importance in quantum computing partly
due to its use in many
early quantum gate decompositions and its ability to
be performed fault-tolerantly in many physical technologies. It will be our
primary two-qubit gate.
Along with arbitrary single-qubit gates, it is universal for quantum computation \cite{Barenco1995a}.
Therefore, we will give this basis a special name:
%
\begin{equation}
\mathcal{Q} = \{ U(2) \cup CNOT \}
\end{equation}

%%%%%%%%%%%%%%%%%%%%%%%%%%%%%%%%%%%%%%%%%%%%%%%%%%%%%%%%%%%%%%%%%%%%%%%%%%%%%%
\subsection{Controlled Gates}
\label{subsec:controlled}

The principle of a controlled gate can be generalized to multiple
controls using the ``meta-operator'' notation from \cite{Kitaev2002}.
By $\Lambda^n(U)$, we mean an $(n+1)$-qubit gate ($2^{n+1} \times 2^{n+1}$
unitary matrix) with $n$ control qubits and a single-qubit target gate
$U \in U(2)$. An important multiply-controlled gate, which is universal
for classical reversible circuits, is the Toffoli gate, or controlled-controlled-$NOT$.

\begin{equation}
\text{Toffoli} = \Lambda^2(X) = 
 \left[
  \begin{array}{cccccccc}
    1 & 0 & 0 & 0 & 0 & 0 & 0 & 0 \\
    0 & 1 & 0 & 0 & 0 & 0 & 0 & 0 \\
    0 & 0 & 1 & 0 & 0 & 0 & 0 & 0 \\
    0 & 0 & 0 & 1 & 0 & 0 & 0 & 0 \\
    0 & 0 & 0 & 0 & 1 & 0 & 0 & 0 \\
    0 & 0 & 0 & 0 & 0 & 1 & 0 & 0 \\
    0 & 0 & 0 & 0 & 0 & 0 & 0 & 1 \\
    0 & 0 & 0 & 0 & 0 & 0 & 1 & 0
  \end{array} \right]
\end{equation}

As seen above, multiply-controlled single-qubit gates $\Lambda^n(U)$ have a
special, sparser structure than general $n$-qubit gates in $U(2^n)$.
These play a special role in many multi-qubit decompositions, about which
we will say more in Chapter \ref{chap:qcompile}.

There is also a special case of a ``targetless'' controlled single-qubit
gate which simply rotates the $\ket{1}$ component of a single-qubit state.

\begin{equation}
\Lambda(e^{i\phi}) = 
 \left[
  \begin{array}{cc}
    1 & 0 \\
    0 & e^{i\phi} \\
  \end{array} \right]
\end{equation}

This gate is a key tool and simplification for single-qubit compiling.

%%%%%%%%%%%%%%%%%%%%%%%%%%%%%%%%%%%%%%%%%%%%%%%%%%%%%%%%%%%%%%%%%%%%%%%%%%%%%%
\subsection{Single-Qubit Compiling}
\label{subsec:qcompile-single}

A seemingly simpler task than general compiling is single-qubit compiling.
This will illustrate the basic principles of quantum compiling and the
structure that we will exploit later to choose an effective basis. Moreover,
it will reveal a general relationship between many of the single-qubit
gates that we have already introduced.

First, we review how to decompose a general $U \in U(2)$ into three single-qubit
rotations about the Bloch sphere $x$-axis and $z$-axis, the so-called
Euler angle decompositions \cite{Nielsen2000}. This gives rise to a factor of $3$
which commonly appears in resource calculations in the literature.

\begin{equation}
U = e^{i\delta}R_Z(\gamma)R_X(\beta)R_Z(\alpha)
\end{equation}

The gate $R_Z(\phi)$ represents a rotation about the Bloch sphere $z$-axis,
of which the Pauli $Z$ gate is a special case of a $\pi$ rotation. In fact,
it is the same as the controlled-phase gate we introduced in the previous section,
up to a global phase.

\begin{equation}
R_Z(\phi) =
\left[
  \begin{array}{cc}
    e^{-i\phi/2} & 0 \\
    0 & e^{i\phi/2} \\
  \end{array} \right]
=
e^{-i\phi / 2} \left[
  \begin{array}{cc}
    1 & 0 \\
    0 & e^{i\phi} \\
  \end{array} \right]
= e^{-i\phi / 2} \Lambda(e^{i\phi})
\end{equation}

We can now state the relationship between $S$ and $Z$, as well as introduce
an important new gate $T$ which is the square root of $S$ up to a phase. All three
are rotations about the Bloch $z$-axis by power-of-two fractions of $\pi$.

\begin{equation}
Z = R_Z(\pi) =
\left[
  \begin{array}{cc}
    1 & 0 \\
    0 & -1 \\
  \end{array} \right]
\qquad
S = R_Z(\tfrac{\pi}{2}) =
\left[
  \begin{array}{cc}
    1 & 0 \\
    0 & i \\
  \end{array} \right]
\qquad
T = R_Z(\tfrac{\pi}{4}) =
\left[
  \begin{array}{cc}
    1 & 0 \\
    0 & e^{i\pi / 4} \\
  \end{array} \right]
\end{equation}
%
Likewise, the gate $R_X(\phi)$ represents a rotation about the Bloch sphere $x$-axis,
of which the Pauli $X$ gate is a special case of a $\pi$ rotation.
%
\begin{equation}
R_X(\phi) =
\left[
  \begin{array}{cc}
    \cos \tfrac{\phi}{2} & -i \sin \tfrac{\phi}{2} \\
    -i \sin \tfrac{\phi}{2} & \cos \tfrac{\phi}{2} \\
  \end{array} \right]
\end{equation}

Similar decompositions can be given in terms of $R_X$ and $R_Y$, or in
terms of $R_Y$ and $R_Z$. Solving for the angles $\{ \alpha, \beta, \gamma, \delta \}$
involves writing four equations in four variables, which can be found in
the standard textbook \cite{Nielsen2000}. We will not say
more about their solution here, except that we can implement the
global phase shift $e^{i\delta}$ using the identities below, which are
adapted from \cite{Kitaev2002}.

\begin{eqnarray}
e^{i\delta} & = & R_Z(\phi)\cdot X \cdot R_Z(\phi) \cdot X \\
X & = & R_X(\pi) \\
Z & = & R_Z(\pi) \\
R_X(\phi) & = & H \cdot R_Z(\phi) \cdot H
\end{eqnarray}

It now seems that a reasonable basis for single-qubit compiling are
arbitrary $R_Z(\phi)$ and $R_X(\phi)$ rotations, along with $H$.
However, in practice our classical control can only implement
rotations with finite precision. How can we measure this, or any
other, precision arising from approximation?

%%%%%%%%%%%%%%%%%%%%%%%%%%%%%%%%%%%%%%%%%%%%%%%%%%%%%%%%%%%%%%%%%%%%%%%%%%%%%%
\subsection{Distance Metrics}
\label{subsec:distance}

Each gate from our basis is a unitary matrix of bounded dimension, and the action of an entire
$n$-qubit compiled circuit $\tilde{C}$
is also a $2^n \times 2^n$ unitary matrix. This matrix can be formed
by the product of $2^n \times 2^n$ matrices $G_i$ which are a tensor
product of gate matrices from our basis operating on individual qubits.
Our desired target matrix $C$ is itself
a matrix from $U(2^n)$, and therefore we will need a distance metric
that operates on matrices (specifically, the difference of matrices).

\begin{equation}
\tilde{C} = G_{D}G_{D-1}\cdots G_{2} G_{1}
\end{equation}

One distance metric used in theoretical literature
is the operator norm of a matrix $M$,
is defined as the maximum amount it scales the vector norm
of all unit-length vectors. This is sometimes also called the
infinity-norm, or supremum-norm (sup-norm).

\begin{equation}
\| M \|_{\infty} = \max_{\| \ket{v} \| = 1} \| M \ket{v} \|
\end{equation}

However, this is not an operational definition.
Moreover, we often wish to neglect a global phase in a unitary matrix,
which is not measurable in quantum physics. To measure phase-independent
distance between two unitary matrices, we can use the following
distance measure due to Fowler \cite{Fowler2011}.

\begin{equation}
dist(U, V) = \| U - V\| \equiv \sqrt{\frac{2^n - |tr(U^{\dag}V)|}{2^n}}
\end{equation}

Now can quantify the quality of our approximations through an
error $\epsilon$.

\begin{equation}
\| C - \tilde{C}\| = \| C -  G_D\cdots G_1 \| < \epsilon
\end{equation}

Often $\epsilon$ will be small, and we will upper bound it by some
power of $\frac{1}{2}$. Therefore, we define a
new parameter $\nu$ which is the number of bits needed to encode the exponent
of this increasingly small fraction.

\begin{equation}
\epsilon = \frac{1}{2^\nu} \qquad
\nu = \log(1/\epsilon)
\end{equation}

It is natural to suppose that compiling better approximations requires
more resources, and these resources are expressed as functions of
these
parameters, increasing with decreasing $\epsilon$ and increasing $\nu$.
In fact, often the efficiency and the
capabilities of a quantum
compiler depend on its basis. Therefore, we conclude this section by
discussing circuit bases.

%%%%%%%%%%%%%%%%%%%%%%%%%%%%%%%%%%%%%%%%%%%%%%%%%%%%%%%%%%%%%%%%%%%%%%%%%%%%%%
\subsection{Circuit Bases}
\label{subsec:qcompile-bases}

\begin{definition}{\textbf{Circuit basis.}}
A basis for a quantum circuit (family) is a universal set of
bounded-qubit gates (usually operating on no more than 3 qubits each).
We call a basis \emph{finite} if it contains a finite
number of gates; that is, it contains discrete gates and not an infinite
continuum of gates. We call a basis \emph{fixed} if its members are independent
of the number of qubits in the input circuit (its width).
\end{definition}

For fault-tolerant quantum computing, we are interested in compiling
circuits to a fixed, finite basis. What does it mean for a fixed, finite
basis to be universal for an infinite group like $SU(2^n)$?

\begin{definition}{\textbf{Universal approximation.}}
We call a fixed, finite set of gates $\mathcal{G}$ \emph{universal} for
a group $U(2^n)$ iff for every desired target $C \in U(2^n)$ and
desired error $\epsilon$, we can return a
sequence of gates $(g_1,g_2,\ldots,g_S)$ from $\mathcal{G}$ where

\begin{equation}
\| C - g_1 g_2 \cdots g_S \| \le \epsilon
\end{equation}

\end{definition}

This defines whether a gateset is a basis, or whether universal approximation
is even possible (non-constructively). We will see in
Chapter \ref{chap:qcompile} that
quantum compilers are concerned with constructive approaches to
\emph{efficiently} return such a compiled sequence $\tilde{C} = \prod g_i$.
Examples of efficiency metrics are the number of returned gates $S$ or
the running time of the quantum compiling algorithm (which itself is
primarily classical).

What gatesets are known to be fixed, finite, and universal, and therefore
suitable bases for quantum compilation? We state without proof that
the following basis satisfies these properties, and we will use it later as
our circuit basis in Chapters \ref{chap:factor-polylog} and
\ref{chap:factor-sublog}.

\begin{equation}
\mathcal{G} = \{X, Z, H, \text{Toffoli}, \text{CNOT}\}
\end{equation}

It is important to note that the only non-Clifford gate in the above basis
is Toffoli.
The Clifford group $\mathcal{C}_n$ by itself is \emph{not}
universal.
In fact, it is provable
that \emph{any} universal gateset must possess at least one
non-Clifford gate \cite{Zeng2011}.
Two popular choices for the non-Clifford gate in a basis are the $T = R_Z(\pi/4)$
gate and the $\text{Toffoli} = \Lambda^2(X)$ gate. Since these are not ``natively''
supported (non-transversal) in many codes, they must often be implemented
probabilistically using only Clifford operations and $MeasureZ$, usually
by way of a so-called ``magic'' state.
Therefore, many quantum compilers
use the Clifford+$T$ basis ($\mathcal{C}_2 \cup \{ T \}$)
or the Clifford+Toffoli basis ($\mathcal{C}_2 \cup \{ \text{Toffoli} \}$),
and measure
the non-Clifford gate as the most expensive resource. It is an area of
active research
whether $T$ or Toffoli is more efficient to implement
\cite{Jones2013a,Eastin2012}.

For single-qubit compilation, the $\{H,T, T^{\dagger}\}$ gateset is universal and
plays an important role in the literature. Other compilers may add
the Clifford gates $S$ and $S^{\dagger}$ to the bases above,
but this does not change their universality nor its asymptotic efficiency for
compiling. From our discussion in this section
on quantum gates and circuit bases, we are now prepared to add some additional
circuit constraints to form quantum architectures.

\section{Quantum Architecture}
\label{sec:intro-arch}

As defined in the introduction, a quantum architecture is a quantum circuit
with two main constaints: a fault-tolerant basis (fixed, finite) basis of
gates and a layout graph which defines the connectivity (allowed two-qubit
interactions) between qubits.
In this section, we define an architectural model and give definitions of
several important models that are common in the literature. We define
how submodels are related to parent models. In particular, we define the
important model \textsf{2D CCNTC} as a foundation for a new model that we
introduce in Section \ref{sec:intro-modules}. Finally, we define how
circuit resources are counted using this model in the rest of this
dissertation.

%%%%%%%%%%%%%%%%%%%%%%%%%%%%%%%%%%%%%%%%%%%%%%%%%%%%%%%%%%%%%%%%%%%%%%%%%%%%%%%
\subsection{Architectural Models}
\label{subsec:models}

Following Van Meter and Itoh \cite{VanMeter2005},
we distinguish between a model and an architectural implementation as follows.
A \emph{model} is a set of constraints and rules for the placement and
interaction of qubits.
An \emph{architecture} (or interchangeably, an \emph{implementation} 
or a \emph{circuit}) is a particular
spatial layout of qubits (as a graph of vertices) and allowed interactions (edges between the vertices),
following the constraints of a given model. In this section, we describe
several models that try to incorporate resources of physical interest from
experimental work.

The most general model is called Abstract Concurrent (\textsf{AC})
and allows arbitrary, long-range interactions between any qubits and concurrent
operation of quantum gates.
This corresponds to a complete graph with an edge between every pair of nodes.
It is the model assumed in most quantum algorithms.

A more specialized model restricts interactions to nearest-neighbor, two-qubit,
concurrent gates (\textsf{NTC}) in a regular one-dimensional chain (\textsf{1D NTC}),
which is sometimes called linear nearest-neighbor (\textsf{LNN}).
This corresponds to a line graph. This is a more realistic model than
\textsf{AC}, but correspondingly, circuits in this model may incur greater
resource overheads.
Traditionally, quantum architecture papers that
implement Shor's factoring algorithm, have focused exclusively on
nearest-neighbor constraints (see Section \ref{fpl-related} for a full
bibliography). Since Shor's original algorithm was
presented in \textsf{AC}, we expect an increase in circuit resources to
simulate these same long-range interaction using the nearest-neighbor
interactions of \textsf{NTC}.
In this dissertation, we focus on designing a realistic nearest-neighbor circuit for running
Shor's factoring algorithm on two-dimensional
architectural models of a physical quantum device with nearest-neighbor
interactions.


To relieve movement congestion,
we can consider a two-dimensional regular grid
(\textsf{2D NTC}), where each
qubit has four planar neighbors, and 
there is an extra degree of freedom over the 1D model
in which to move data.
We can extend this model in two powerful ways, which are described in the next section.

\subsection{\textsf{2D CCNTC}: Two-Dimensional Nearest-Neighbor Two-Qubit Concurrent Gates with Classical Controller}
\label{subsec:2dccntc}

The first extension allows arbitrary planar graphs
with bounded degree, rather than a regular square lattice.
Namely, we assume qubits lie in a plane and edges are not allowed to intersect.
All qubits are accessible from above
or below by control and measurement apparatus.
Whereas 2D NTC conventionally assumes each qubit
has four neighbors, we consider up to six neighbors in a roughly hexagonal
layout. The edge length in this model is no more than twice the edge length
in a regular 2D NTC lattice. The second extension is the realistic assumption
that classical control (CC) can
access every qubit in parallel, and we do not count these classical
resources in our implementation since they are polynomially bounded. The
classical controllers
correspond to fast digital computers which are
available in actual experiments and are necessary for constant-depth
communication in the next section.

We call an AC or NTC model augmented by these two extensions
\textsf{CCAC} and \textsf{CCNTC}, respectively. Before we describe the
third extension, let us formalize our model for \textsf{2D CCNTC}, with definitions that are (asymptotically) equivalent to those in 
\cite{Rosenbaum2012}.

\begin{definition}
A \textsf{2D CCNTC} architecture consists of

\begin{itemize}
\item a quantum computer $QC$ which is represented by a planar graph $(V,E)$. A
node $v \in V$ represents a qubit which is acted upon in a circuit, and an
undirected edge $(u,v) \in E$ represents 
an allowed two-qubit interaction between qubits $u,v \in V$. Each node has
degree at most $6$.
\item a universal gate set $\mathcal{G} = \{X, Z, H, T, T^{\dagger}, CNOT, MeasureZ\}$.

\item a deterministic machine (classical controller) $CC$ that applies a sequence
of concurrent gates in each of $D$ timesteps.
\item In timestep $i$, $CC$ applies a set of
gates $G_i = \{g_{i,j} \in \mathcal{G} \}$.
Each $g_{i,j}$ operates in one of the following two ways:
\begin{enumerate}
\item It is a single-qubit gate from $\mathcal{G}$ acting on a single qubit $v_{i,j} \in V$
\item
It is the gate CNOT from $\mathcal{G}$ acting on two qubits $v^{(1)}_{i,j}, v^{(2)}_{i,j} \in V$ where
$(v^{(1)}_{i,j}, v^{(2)}_{i,j}) \in E$
\end{enumerate}
All the $g_{i,j}$ can only operate on
disjoint qubits for a given timestep $i$. We define the support of $G_i$
as $V_i$, the set of all qubits acted upon during timestep $i$.

\begin{equation}
V_i = \bigcup_{j: g_{i,j} \in G_i} v_{i,j} \cup v^{(1)}_{i,j} \cup v^{(2)}_{i,j}
\end{equation}

\end{itemize}
\end{definition}

We can then define the three conventional circuit resources in this model.

\begin{description}
\item[circuit depth ($D$):] the number of concurrent timesteps.
\item[circuit size ($S$):] the total number of non-identity gates applied
from $\mathcal{G}$, equal to $\sum_{i=1}^D |G_i|$.
\item[circuit width ($W$):] the total number of qubits operated upon by
any gate, including inputs, outputs, and ancillae. It is equal to $| \bigcup_{i=1}^D V_i|$.
\end{description}

We observe that the following relationship holds between the circuit resources.
The circuit size is bounded above by
the product of circuit depth and circuit width, since in the worst case,
every qubit is acted upon by a gate for every timestep of a circuit.
The circuit depth is also bounded above by the size, since in the worst case,
every gate is executed serially without any concurrency.

\begin{equation}
D \le S \le D\cdot W
\label{eqn:depth-width}
\end{equation}

The set $\mathcal{G}$ includes measurement in the $Z$ basis, which is
actually not a unitary operation but which may be slower than unitary
operations in actual practice \cite{DiVincenzo2007}.
Therefore we count it in our resource
estimates.
All other gates
in $\mathcal{G}$ form a universal set of unitary
gates \cite{Kitaev2002}.
 In this paper we
will treat the operations in $\mathcal{G}$ as \emph{elementary gates}.
We can also define a Bell basis measurement using operations
from $\mathcal{G}$. A circuit performing this measurement is shown
in Figure \ref{fig:bell-measure} and has depth $4$,
size $4$, and width $2$.

\begin{figure*}[tb!]
\begin{center}
\begin{displaymath}
\begin{array}{ccc}
\Qcircuit @C=1em @R=1em {
& \qw & \multimeasureD{1}{\mbox{Bell}} & \cw & \rstick{j} \\
& \qw & \ghost{\mbox{Bell}}            & \cw & \rstick{k}
}
& \qquad \equiv \qquad &
\Qcircuit @C=1em @R=1em {
& \qw & \ctrl{1} & \qw & \gate{H} & \qw & \meter & \cw & \rstick{j} \\
& \qw & \targfix & \qw & \qw      & \qw & \meter & \cw & \rstick{k}
}
\end{array}
\end{displaymath}
\centerline{}
\caption{A circuit for measurement in the Bell state basis.}
\label{fig:bell-measure}
\end{center}\end{figure*}

The third extension to our model, and the most significant, is to consider
multiple disconnected planar graphs, each of which is a 2D CCNTC
architecture. This is described in more detail in the next section.


\subsection{Circuit Resource Comparisons}

Counting gates from $\mathcal{G}$ as having unit size and unit depth
is
an overestimate compared to the model in \cite{Kutin2006}, in which a
two-qubit gate has unit size and unit depth and
absorbs the depth and size of any adjacent single-qubit gates. We intend
for this more pessimistic estimate to reflect the practical difficulties
in compiling these gates using a non-Clifford gate in a fault-tolerant way,
such as the $T$ gate or the Toffoli gate
\cite{Fowler2011}.
%However, these difficulties may be mitigated by using
%Toffoli gates directly, which can be fault-tolerantly implemented using
%magic-state distillation according to recent works \cite{Eastin2012,Jones2013a}.

In both our resource counting method and that of \cite{Fowler2004,Kutin2006}, multiple gates acting on disjoint qubits
can occur in parallel during the same timestep. For each building block,
from modular addition to modular multiplication and finally to modular
exponentiation, we provide closed form equations upper-bounding the required circuit
resources as a function of $n$, the size of the modulus $m$ to be factored.
We will use the
term \emph{numerical upper bound} to distinguish these formulae from asymptotic
upper bounds.

It is possible to reduce the numerical constants with more detailed analysis,
which would be important for any physical implementation.
However, we have chosen instead to simplify the number of terms in the formulae
for the current work. We do not intend for these upper bounds to represent
the optimal or final work in this area.

The modular adder in Section \ref{sec:csa-mod-add} and its carry-save
subcomponents only occur within a single module, so we only give their
circuit resources in terms of circuit depth, circuit size, and circuit width. 
For the modular multiplier in
Section \ref{sec:csa-mod-mult} and the modular exponentiator in
Section \ref{sec:modexp}, we also give circuit resources in
terms of module depth, module size, and module width.

%%%%%%%%%%%%%%%%%%%%%%%%%%%%%%%%%%%%%%%%%%%%%%%%%%%%%%%%%%%%%%%%%%%%%%%%%%%%%%%
\section{Constant-depth Communication}
\label{sec:cdc}

Communication, namely the \emph{moving} and \emph{copying} of quantum information, in nearest-neighbor quantum architectures is challenging.
In this section we quote known results for teleportation and
fanout in constant depth while also contributing a novel construction
for unfanout.

The first challenge of moving quantum information from one site to another over
arbitrarily long distances can be addressed by using
%A related problem is how to teleport a qubit an arbitrary distance.
% in an
%architecture through ancillae prepared in some initial state.
the constant-depth teleportation circuit
shown in Figure \ref{fig:cdt} due to Rosenbaum \cite{Rosenbaum2012}, illustrated using standard quantum circuit
notation \cite{Nielsen2000}. This requires the circuit resources shown in
Table \ref{tab:cd-resources}. The depth includes a layer of $H$ gates; a layer of CNOTs; an interleaved layer of Bell basis measurements; and two layers of
Pauli corrections ($X$ and $Z$ for each qubit), occurring concurrently with
resetting the $\ket{j}$ and $\ket{k}$ qubits back to $\ket{0}$.
These correction layers are not shown in the circuit.

\begin{figure*}[tb!]
\begin{center}
\begin{displaymath}
%\begin{array}{ccc}
\Qcircuit @C=1em @R=1em {
\lstick{\ket{\psi}}	& \qw      & \qw      & \qw & \qw & \qw & \qw & \qw                                          & \qw & \qw & \multimeasureD{1}{\mbox{Bell}} & \cw & \rstick{j_1} \\
\lstick{\ket{0}}    & \gate{H} & \ctrl{1} & \qw & \qw & \qw & \qw & \qw                                          & \qw & \qw & \ghost{\mbox{Bell}}            & \cw & \rstick{k_1} \\
\lstick{\ket{0}}    & \qw      & \targfix & \qw & \qw & \qw & \qw & \qw_{Z^{j_1}X^{k_1}\ket{\psi}}               & \qw & \qw & \multimeasureD{1}{\mbox{Bell}} & \cw & \rstick{j_2} \\
\lstick{\ket{0}}    & \gate{H} & \ctrl{1} & \qw & \qw & \qw & \qw & \qw                                          & \qw & \qw & \ghost{\mbox{Bell}}            & \cw & \rstick{k_2} \\
\lstick{\ket{0}}    & \qw      & \targfix & \qw & \qw & \qw & \qw & \qw_{Z^{j_2}Z^{j_1}X^{k_2}X^{k_1}\ket{\psi}} & \qw & \qw & \multimeasureD{1}{\mbox{Bell}} & \cw & \rstick{j_3} \\
\lstick{\ket{0}}    & \gate{H} & \ctrl{1} & \qw & \qw & \qw & \qw & \qw                                          & \qw & \qw & \ghost{\mbox{Bell}}            & \cw & \rstick{k_3} \\
\lstick{\ket{0}}    & \qw      & \targfix & \qw & \qw & \qw & \qw & \qw & \qw_{Z^{j_1}Z^{j_2}Z^{j_3}X^{k_3}X^{k_2}X^{k_1}\ket{\psi}} & \qw & \qw              & \qw & \qw \\
}
\end{displaymath}
\centerline{}
\caption{Constant-depth circuit based on \protect{\cite{Broadbent2007,Browne2009}} for teleportation over $n=5$ qubits \protect{\cite{Rosenbaum2012}}.}
\label{fig:cdt}
\end{center}\end{figure*}

\begin{figure*}[tb!]
\begin{center}
\begin{displaymath}
%& \qquad \qquad \qquad &
\Qcircuit @C=1em @R=1em {
\lstick{\ket{\psi}}	& \qw      & \qw      & \qw & \qw & \qw & \multimeasureD{1}{\mbox{Bell}'} & \cw & \rstick{j_1} \\
\lstick{\ket{0}}    & \gate{H} & \ctrl{1} & \qw & \qw      & \qw & \ghost{\mbox{Bell}'}            & \cw & \rstick{k_1} \\
\lstick{\ket{0}_1}    & \qw      & \targfix & \qw & \ctrl{1} & \qw & \qw      & \qw & \rstick{Z^{j_1}X^{k_1}\ket{\ell}_1}\\
\lstick{\ket{0}}	& \qw      & \qw      & \qw & \targfix & \qw & \multimeasureD{1}{\mbox{Bell}} & \cw & \rstick{j_2} \\
\lstick{\ket{0}}    & \gate{H} & \ctrl{1} & \qw & \qw      & \qw & \ghost{\mbox{Bell}}           & \cw & \rstick{k_2} \\
\lstick{\ket{0}_2}    & \qw      & \targfix & \qw & \ctrl{1} & \qw & \qw      & \qw & \rstick{Z^{j_2}X^{k_2}X^{k_1}\ket{\ell}_2}\\
\lstick{\ket{0}}	& \qw      & \qw      & \qw & \targfix & \qw & \multimeasureD{1}{\mbox{Bell}} & \cw & \rstick{j_3} \\
\lstick{\ket{0}}    & \gate{H} & \ctrl{1} & \qw & \qw      & \qw & \ghost{\mbox{Bell}}           & \cw & \rstick{k_3} \\
\lstick{\ket{0}_3}    & \qw      & \targfix & \qw & \ctrl{1} & \qw & \qw      & \qw & \rstick{Z^{j_3}X^{k_3}X^{k_2} X^{k_1}\ket{\ell}_3}\\
\lstick{\ket{0}_4}	& \qw      & \qw      & \qw & \targfix & \qw & \qw      & \qw & \rstick{X^{k_3}X^{k_2} X^{k_1}\ket{\ell}_4}\\
}
%& & \\
%(a) & & (b)
%\end{array}
\end{displaymath}
\centerline{}
\caption{Constant-depth circuits based on \protect{\cite{Broadbent2007,Browne2009}} for fanout \protect{\cite{Harrow2012}} of one qubit to $n=4$ entangled copies.}
\label{fig:cdf}
\end{center}\end{figure*}

Although general cloning is
impossible \cite{Nielsen2000}, the second challenge of copying information can be addressed by performing an unbounded quantum
fanout operation:
$\ket{x,y_1,\ldots,y_n} \rightarrow \ket{x,y_1\oplus x, \ldots, y_n\oplus x}$.
This is used in our arithmetic circuits when
a single qubit needs to control (be entangled with) a large quantum register
(called a \emph{fanout rail}).
We employ a constant-depth circuit due to insight from
measurement-based quantum computing \cite{Raussendorf2003}
that relies on the creation of an
$n$-qubit cat state \cite{Browne2009} which was communicated to
us by Harrow and Fowler \cite{Harrow2012}.

This circuit requires $O(1)$-depth, $O(n)$-size, and $O(n)$-width. Approximately
two-thirds of the ancillae are reusable and can be reset to $\ket{0}$ after
being measured. Numerical upper bounds are given in Table \ref{tab:cd-resources}.
The constant-depth fanout circuit is shown in Figure \ref{fig:cdf} for the case of fanning out a given single-qubit state
$\ket{\psi} = \alpha\ket{0} + \beta\ket{1}$ to four qubits.
The technique works by creating multiple small
cat states of a fixed size (in this case, three qubits), linking them
together into a larger cat state of unbounded size with Bell basis measurements,
and finally entangling them with the source qubit to be fanned out.
The qubits marked $\ket{\ell}$ are
entangled into the larger fanned out state given in Equation \ref{eqn:cat4}.
The Pauli corrections from the cat state creation are denoted by
$X^{k_2}$, $X^{k_3}$, $Z^{j_2}$ and $Z^{j_3}$ on qubits ending in
states $\ket{\ell}_1$, $\ket{\ell}_2$,
$\ket{\ell}_3$, and $\ket{\ell}_4$. The Pauli corrections
$X^{k_1}$ and $Z^{j_1}$ are from the Bell basis measurement
entangling the cat state with the source qubit (denoted $\text{Bell}'$).
\begin{equation}
Z_1^{j_1}X_1^{k_1}Z_2^{j_2}X_2^{k_2}X_2^{k_1}Z_{3}^{j_3}X_{3}^{k_3}X_{3}^{k_2}X_{3}^{k_1}X_{4}^{k_3}X_{4}^{k_2}X_{4}^{k_1}
\left(\alpha \ket{0}_1\ket{0}_2\ket{0}_3\ket{0}_4 + \beta \ket{1}_1\ket{1}_2\ket{1}_3\ket{1}_4 \right)
\label{eqn:cat4}
\end{equation}
%
The operators $X^k_i$ and $Z^j_{h}$ denote Pauli $X$ and $Z$ operators
on qubits $i$ and $h$, controlled by classical bits $k$ and $j$,
respectively. These corrections are enacted by the classical controller based on
the Bell measurement outcomes (not depicted).
Note the cascading nature of these corrections.
There can be up to
$n-1$ of these $X$ and $Z$
corrections on the same qubit, which can be simplified by the classical
controller to a single $X$ and $Z$ operation and then applied with a circuit of
depth 2 and size 2. Also, given the symmetric nature of the cat state, there
is an alternate set of Pauli corrections which would give the same state and
is of equal size to the corrections given above.

\begin{figure*}[tb!]
\begin{center}
\begin{displaymath}
\Qcircuit @C=1em @R=1em {
& \lstick{\ket{\ell}}	& \qw & \gate{H} & \qw & \ctrl{1} & \qw & \qw      & \qw &  \measureD{Z} & \cw & \rstick{j_1} & \\
& \lstick{\ket{\ell}}	& \qw & \gate{H} & \qw & \targfix & \qw & \ctrl{1} & \qw & \measureD{Z} & \cw & \rstick{j_2} & \\
& \lstick{\ket{\ell}}	& \qw & \gate{H} & \qw & \ctrl{1} & \qw & \targfix & \qw & \measureD{Z} & \cw & \rstick{j_3} & \\
& \lstick{\ket{\ell}}	& \qw & \gate{H} & \qw & \targfix & \qw & \ctrl{1} & \qw & \measureD{Z} & \cw & \rstick{j_4} & \\
& \lstick{\ket{\ell}}	& \qw & \gate{H} & \qw & \ctrl{1} & \qw & \targfix & \qw & \measureD{Z} & \cw & \rstick{j_5} & \\
& \lstick{\ket{\ell}}	& \qw & \gate{H} & \qw & \targfix & \qw & \ctrl{1} & \qw & \measureD{Z} & \cw & \rstick{j_6} & \\
& \lstick{\ket{\ell}}	& \qw & \gate{H} & \qw & \qw      & \qw & \targfix & \qw & \gate{H} & \qw & \rstick{Z^{j_2 \oplus j_4}(\alpha\ket{0} + \beta\ket{1})}
}
\end{displaymath}
\centerline{}
\caption{A novel, constant-depth circuit for unbounded quantum unfanout on
CCNTC, from the $7$-qubit entangled state $\alpha\ket{0}^{\otimes 7} + \beta\ket{1}^{\otimes 7}$ to the
target product state $(\alpha\ket{0} + \beta\ket{1})\otimes\ket{0}^{\otimes 6}$.}
\label{fig:cdu}
\end{center}\end{figure*}

Reversing the fanout is an operation called \emph{unfanout}. Unfanout
takes as input 
the following entangled $n$-qubit state which is the result of a fanout.

\begin{equation}
\normtwo (\ket{0}^{\otimes n} + \ket{1}^{\otimes n})
\label{eqn:fanned-out}
\end{equation}

The output of unfanout, after Pauli corrections, is the product state
consisting of all $\ket{0}$'s except for a single target qubit $\alpha\ket{0} + \beta\ket{1}$, which is in the
same state as the original source qubit of the fanout.

In the model of \cite{Hoyer2002}, the fanout and unfanout were identical, elementary
operations. In Figure \ref{fig:cdf}, the given CCNTC fanout circuit is not its own self-inverse
due to
the one-way nature of the measurement and the assumption that ancillary qubits begin in the
$\ket{0}$ state. Therefore, one major contribution of this current work is
a \textsc{CCNTC} circuit which performs unfanout, or the inverse of the fanout circuit in
Figure \ref{fig:cdf}. The relationship of our fanout and unfanout constructions and their
effect on intermediate quantum states is shown below.
\begin{equation}
(\alpha\ket{0} + \beta\ket{1})\otimes\ket{0}^{\otimes n} \rightarrow^{fanout}
\alpha\ket{0}^{\otimes n} + \beta\ket{1}^{\otimes n} \rightarrow^{unfanout}
(\alpha\ket{0} + \beta\ket{1})\otimes\ket{0}^{\otimes n}
\end{equation}
We give a concrete example of our unfanout circuit in Figure \ref{fig:cdu}
for $n=7$ which executes in constant depth on \textsf{2D CCNTC}.
Note that the state in Equation \ref{eqn:fanned-out}
is completely symmetric in that all qubits are
equivalent entangled copies of each other. Therefore, the asymmetry 
of the final target qubit is entirely determined by the unfanout circuit,
which in this case selects the bottom qubit in the figure to be the target.

The initial fanned out state lives in a $2$-dimensional subspace. The
round of Hadamard gates increases its dimension to $2^n$, and the two
interleaved layers of CNOTs in a sense ``disentangle'' the qubits from
one another, up to a Pauli $Z$ correction. This correction, on the
final target qubit, is controlled by the parity of the classical measurements
on every ``even'' qubit ($j_2$ and $j_4$ in the figure), excluding the 
next-to-last qubit ($j_6$ in the figure). Each measurement projects the state of the
target qubit
into a subspace with half the dimension, so $n-1$ measurements project
the target qubit into a final $2$-dimensional subspace, which is the
qubit $\alpha\ket{0} + \beta\ket{1}$.

Although the circuit shown works for odd $n$, we can easily take into
account even $n$ with an initial CNOT to ``uncopy'' one qubit from its
neighbors. The unfanout circuit in Figure \ref{fig:cdu} is the
functional inverse of the fanout circuit in 
Figure \ref{fig:cdf}, which by itself only requires a 1D layout. However,
it relies on the fanned-out qubits
being teleported back into adjacent positions,
which is only possible in an overall 2D layout. This layout is not shown but can easily be constructed.
The target qubit of an unfanout is usually chosen to be in the same location
as the source qubit of the previous, corresponding fanout. 
The resources for unfanout are given in
Table \ref{tab:cd-resources}.

% From Notebook #16, p. 212
% From Notebook #16, p. 66
\begin{table}
\begin{displaymath}
\begin{tabular}{|c|c|c|c|}
\hline
\text{Circuit Name} & \text{Depth} & \text{Size} & \text{Width}\\
\hline
\text{Teleportation from Figure \ref{fig:cdt}} & $7$ & $3n + 4$ & $n+1$\\
\hline
\text{Fanout from Figure \ref{fig:cdf}} & $9$ & $10n - 9$ & $3n-1$ \\
\hline
\text{Unfanout from Figure \ref{fig:cdu}} & $ 6 $ & $ 3n+2 $ & $ n$ \\
\hline
\end{tabular}
\end{displaymath}
\centerline{}
\caption{Circuit resources for teleportation, fanout, and unfanout.}
\label{tab:cd-resources}
\end{table}

From an experimental perspective, it is physically efficient to create
a cat state in trapped ions using the M{\o}lmer-S{\o}rensen gate
\cite{Sorensen2000}\cite{Benhelm2008}. However, the fanout circuit for
the \textsf{2D CCNTCM} model would still be useful for other technologies, such
as superconducting qubits on a 2D lattice.


\section{Modules and Long-Range Interactions}
\label{sec:intro-modules}

%%%%%%%%%%%%%%%%%%%%%%%%%%%%%%%%%%%%%%%%%%%%%%%%%%%%%%%%%%%%%%%%%%%%%%%%%%%%%
%% Subsection 0: Motivation, why modules?

Unfortunately, the model \textsf{2D CCNTC} has several shortcomings
which are highlighted by the constant-depth communication in the
previous section. First and foremost, it is difficult to realize physically. 
A single, contiguous
2D lattice that contains an entire quantum architecture
may be prohibitively large to manufacture or too difficult to control
coherently. In practice,
scalable experiments already prefer to use many
smaller quantum computers that communicate by means of shared
entanglement \cite{Monroe2012}. Second, for highly parallelized
algorithms, the amount of quantum information undergoing active
computation is so large that communicating data across the
quantum processor consumes more resources than the computation
itself. Finally, as quantum architects we are interested in distilling
the essential locality of a computation, and distinguishing it from
communication (which itself may be an interesting problem). All of these
notions will be made more rigorous below, when we develop a new model to
overcome these limitations.

Running an
algorithm on a single \textsf{2D CCNTC} architecture makes it
sensitive to the geometry of the different computation sites and how they
are arranged on the same lattice. Calculating the circuit resources of different
geometric arrangements will have vastly different numerical constants but will asymptotically
be the same. We would like a model that abstracts away this sensitivity.
The key to characterizing this essential locality is the long-range interaction. Rather than simulating them using short-range interactions,
we account for them as separate circuit resources. We define our
new model as a
\emph{hybrid architectural model}, one which accounts for long-range
and short-range interactions separately and also constrains them
according to independent graphs.

In fact, an example of this hybrid approach has already been proposed
in the \emph{quantum multicomputer} of Van Meter et al.,
which we abbreviate as \textsf{QMC} \cite{VanMeter2008}.
This hybrid model is also called a distributed quantum computer since it contains several communicating nodes operating in parallel.
Our approach is
very similar in that we divide a computation, which normally occurs on a single connected qubit graph $G$, into smaller subdivisions connected in
a graph $\overline{G}$. If two qubits $\ket{u}$ and $\ket{v}$ are connected
in $G$, then their two containing subdivisions $\overline{u} \ni \ket{u}$
and $\overline{v} \ni \ket{v}$ are connected in $\overline{G}$.
Like \textsf{QMC}, we also include ``flying qubits,'' or shared
entanglement, to allow for long-range, inter-node interactions.
Finally, we also regard short-range interactions between
qubits in the same node as being faster and cheaper, although this is not
a formal part of the model.

However, our model is different from \textsf{QMC} in several crucial
respects. We call our circuit subdivisions \emph{modules}
to emphasize the following differences from
\textsf{QMC} nodes.\footnote{This should not be
confused with the word ``modular'' as in ``modular arithmetic'' or as
referring to the modulus $m$ that we are trying to factor.}
Each module is a self-contained \textsf{2D CCNTC} lattice, where the qubits
are constrained to have nearest-neighbor interactions. Qubits in a
\textsf{QMC} node,
in contrast, are \textsf{AC} circuits with arbitrary connectivity. The
Van Meter paper is more empirical, providing numerical calculations
based on estimates of current physical technology as well as detailed
schematic implementations of long-range teleportations. Our work is more
abstract, providing asymptotic formulae for circuit resources. Importantly,
our model allows concurrent teleportations, even among non-disjoint
pairs of modules, whereas \textsf{QMC} nodes can only perform long-range
teleportations one-at-a-time. Finally,
while both \textsf{QMC} nodes and modules have configurable size, we have come to the
conclusion that modules should contain $O(n)$ qubits each, whereas
the Van Meter paper concludes that nodes have an optimal $O(1)$ size for 
factoring.
We discuss module size in more detail in Section \ref{subsec:module-future}.

A related work by Beals et al. \cite{Beals2012} describes how to achieve
low-depth, generic simulations of any \textsf{AC} circuit, either by
rearranging qubits using a sorting network on \textsf{NTC} or
by mapping it to \textsf{QMC}
(even with fixed-size nodes connected in a \textsf{LNN} topology). The
depth overhead in either case for an $n$-qubit \textsf{AC} circuit is $O(\log^2 n)$. Their first approach of \textsf{NTC} sorting networks can be used
as an alternative to hybrid architectures. As we will see later in
Chapter \ref{chap:factor-polylog}, the asymptotic depth of the generic sorting network approach will be the same as our ``hand-crafted'' architecture for factoring,
but possibly with worse constants and asymptotically larger width and size. 

We have now described the motivation for a new hybrid architectural model.
We have described a general feature set which
allows us to capture the essential locality of a quantum algorithm while
still accounting for long-range interactions.

%%%%%%%%%%%%%%%%%%%%%%%%%%%%%%%%%%%%%%%%%%%%%%%%%%%%%%%%%%%%%%%%%%%%%%%%%%%%
\subsection{The Model \textsf{2D CCNTCM}}
\label{subsec:2dccntcm}

The main feature of our new submodel is that modules
and long-range teleportations between them are nodes and edges, respectively,
in a higher-level planar graph. The teleportations each transmit one qubit
from one module to another, from any location within the source module
to any location within the destination module, making use of the
omnipresent classical controller.

We do not discuss the physical technology underlying the long-range
teleportation other than acknowledging that it is based on shared
entanglement. Namely, the two modules which are the source and destination
of any long-range teleportation must each possess one half of an
Einstein-Podolsky-Rosen (EPR) pair, as shown in Equation \ref{eqn:epr}.
%
\begin{equation}
\normtwo \left( \ket{00} + \ket{11} \right)
\label{eqn:epr}
\end{equation}

%Therefore, each long-range teleportation includes a preprocessing step in
%which the EPR pairs are generated and transported, along the bounded-degree
%module graph, to the modules where future teleportations will take place.

 The modules can be arbitrarily far
apart physically and have arbitrary connectivity. This is a reasonable
assumption because entangled pairs can be generated and distributed between
any two modules. This generation and distribution is counted as part of
the (preprocessing) costs of the long-range teleportation itself as a
one-to-one correspondence in the worst case. One could imagine that 
generating and distributing many entangled pairs in a batch might be more
efficient physically than doing so for individual pairs. Moreover, the
distribution of entangled pairs may occur along a constrained graph of
module connectivity, but such constraints can be overcome with only
polylogarithmic depth overhead as previously noted \cite{Beals2012}.

A single module can be part of multiple teleportation operations in a single
timestep, as long as they involve disjoint qubits within the module.
We justify this assumption in that it is
possible for a quantum computer to share multiple EPR pairs with a
second quantum computer, and even multiple other quantum computers,
simultaneously.

We call this new model \textsf{2D CCNTCM}, since it generalizes the
model \textsf{2D CCNTC} to have more than one module. Equivalently,
each module corresponds exactly to a \textsf{2D CCNTC} architecture.
All modules can share a single omnipresent
classical controller or a collection of multiple, inter-communicating classical controllers.
A graphic depiction of three modules in \textsf{2D CCNTCM} is shown in
Figure \ref{fig:modules}.

\begin{figure}[btp!]
\begin{center}
\includegraphics[width=4in]{factor-polylog/figures/modules.pdf}
\end{center}
\caption{Three modules in the \textsf{2D CCNTCM} model.}
\label{fig:modules}
\end{figure}

\begin{definition}
A \textsf{2D CCNTCM} architecture consists of

\begin{itemize}
\item a quantum computer $\overline{QC}$ which is represented by a graph
$(\overline{V},\overline{E})$ of arbitrary degree.
A node $\overline{v} \in \overline{V}$ represents a module,
or a graph $(V,E)$
from a \textsf{2D CCNTC} architecture defined previously on page \pageref{subsec:2dccntc}.
An
undirected edge $(\overline{u},\overline{v}) \in \overline{E}$ represents an
allowed teleportation from any qubit in module $\overline{u}$ to
another qubit in module $\overline{v}$.
\item a circuit basis $\mathcal{G} = \{X, Z, H, T, T^{\dagger}, CNOT,
MeasureZ\}$
for the qubits \emph{within the same} modules which is the same as for \textsf{2D CCNTC}.
\item
an additional operation $Teleport$ which only operates on qubits
\emph{in
different} modules.
\item a deterministic machine (classical controller) $\overline{CC}$ that applies a sequence
of concurrent gates in each of $D+\overline{D}$ timesteps.
This can be a separate classical controller
for every pair of modules.
\item In timestep $i$, $\overline{CC}$ applies
gates $G_i = \{g_{i,j} : g_{i,j} \in \mathcal{G} \cup \{Teleport\} \}$.
That is, there are two kinds of timesteps with respect to the kinds of gates
which operate within them.
\begin{enumerate}
\item In the first kind, gates are exclusively from $\mathcal{G}$, and
they operate within modules as described
for \textsf{2D CCNTC} above. We say there are $D$ such timesteps.
\item In the second kind, gates are exclusively $Teleport$ gates between two qubits $v^{(1)}_{i,j} \in \overline{v}_1$ and
$v^{(2)}_{i,j} \in \overline{v}_2$ for
(possibly non-distinct) modules $\overline{v}_1, \overline{v}_2 \in \overline{V}$.
Again, all such qubits much be distinct within a timestep but the
modules need not be distinct.
We say there are $\overline{D}$ such timesteps.
\end{enumerate}

Again, we define the support of $G_i$
as $V_i$, the set of all qubits acted upon by any $g_{i.j}$, which
includes all the modules.
\begin{equation}
V_i = \bigcup_{j: g_{i,j} \in G_i} v_{i,j} \cup v^{(1)}_{i,j} \cup v^{(2)}_{i,j} 
\end{equation}

\end{itemize}
\end{definition}

We measure the efficiency of a circuit in this new module using not just
the three conventional circuit resources defined for \textsf{2D CCNTC}, but with three novel resources
based on modules.

\begin{description}

\item[circuit depth ($D$):] the depth of all operations over all modules.
\item[circuit size ($S$):] the total operations on all qubits within all
modules.
\item[circuit width ($W$):] the total number of qubits, equal to the sum of all the module sizes.
%, depicted in Figure \ref{fig:resources}:
\item[module depth ($\overline{D}$):] the depth of parallel teleportations
of disjoint qubits between modules.
\item[module size ($\overline{S}$):] the number of total qubits teleported between any two modules over all timesteps.
\item[module width ($\overline{W}$):] the number of modules whose qubits are
acted upon during any timestep.

\end{description}

We can define the size of a module $\overline{v} \in \overline{V}$
as the number of qubits within it,
where all the module sizes add up to the total circuit width.

\begin{equation}
\sum_{\overline{v} \in \overline{V}} |\overline{v}| = W
\end{equation}

%%%%%%%%%%%%%%%%%%%%%%%%%%%%%%%%%%%%%%%%%%%%%%%%%%%%%%%%%%%%%%%%%%%%%%%%%%%%
\subsection{Module Size and Model Comparisons}
\label{subsec:module-compare}

Modules in a \textsf{2D CCNTCM} architecture do not all need to have the same
size. However, it is useful to upper bound the size of any module and call
this the \emph{module size} for the entire architecture, denoted $\hat{W}$.
In this 
dissertation, we only consider \emph{uniformly-sized} \textsf{CCNTCM} architectures, ones where
the module size $\hat{W}$
is upper-bounded by the average of all the module sizes, up to a constant
factor. In the definition below, we iterate over all modules $\overline{v}$
in the set of all modules $\overline{V}$, where the number of modules is
defined as $\overline{W} = |\overline{V}|$.

\begin{equation}
\hat{W} = \max_{\overline{v} \in \overline{V}} |v| \equiv O(W / \overline{W})
\end{equation}

We argue that module size is a key, configurable parameter of any
hybrid architecture that greatly affects its resource usage. To see this,
we will now compare the overhead of simulating a generic architecture
(an \textsf{AC} circuit) on three different models with nearest-neighbor
interactions: \textsf{2D NTC}, \textsf{2D CCNTC}, and \textsf{2D CCNTCM}.
This will demonstrate how \textsf{2D CCNTCM} simplifies our view of computation
versus communication resources.

%Related to module
%size, we define two restrictions we can make to the \textsf{CCNTCM} model
%to simplify the analyses in this dissertation. Neither of these is strictly
%part of the \textsf{CCNTCM} model as we've defined it above.

%\begin{definition}
%A \emph{one-way} \textsf{CCNTCM} architecture is one where each module
%only interacts with other modules either before or after computation
%within the module itself. That is, all long-range, inter-module 
%teleportations that are counted as part of $\overline{S}$ take place in
%a cohort of operations either before or after the short-range interactions
%and gates that are counted as part of $S$. Furthermore, all
%inter-module teleportations occur in one-direction. Once 

%The second is that we only consider \emph{efficient} partitionings such
%that a module only participates in long long-range teleportations 
%\end{definition}

% IV. Compare & Contrast

In each of these three models, we must simulate the long-range
interactions of a completely general \textsf{AC} circuit (with
depth $D$, size $S$, and width $W$), using only the
constraints of each particular model. We will define a depth overhead
of $\tilde{D}$, a size overhead $\tilde{S}$, a width overhead $\tilde{W}$
associated purely with communication (not computation) associated with
this long-range simulation. The first two models do not use modules
($\overline{W} = 1$),
whereas the last model has a module size parameter $\hat{W}$ as defined above, with $1 \le \hat{W} \le W$ depending on the number of modules
$\overline{W}$. Modules are now seen as an elegant framework for
interpolating our hybrid model between a completely unconstrained and a completely nearest-neighbor architectural model. When $\overline{W} = 1$,
there is a contiguous lattice which corresponds exactly to \textsf{2D CCNTC}.
When $\overline{W} = W$, there is an unconstrained network of single qubits
which corresponds exactly to \textsf{AC}.
%% TODO I really love something about the sentences below,
%% find some way to put them back in
%This new model then, \textsf{2D CCNTCM}, is a hybrid
%-between \textsf{AC} and \textsf{2D CCNTC} with module size as a parameter.
%-\textsf{AC} and \textsf{2D CCNTC} represent the two extremes of long-range
%-interactions, where \textsf{AC} permits all long-range interactions and
%-\textsf{2D CCNTC} permits none.

Table \ref{tab:mod-comp} summarizes the resources used by these models
to to simulate an \textsf{AC} circuit \emph{with lowest depth}.
In this comparison, \textsf{AC} is
the most efficient architecture in all resources
but also the most unrealistic. All other
architectures pay an overhead for simulating interactions with
varying degrees of nearest-neighbor realism.

\begin{table}
\begin{tabular}{|c|c|c|c|c|}
\hline
Circuit Resources & \textsf{AC} & \textsf{2D NTC} & \textsf{2D CCNTC} & \textsf{2D CCNTCM} \\
\hline
$D + \tilde{D}$ & $D$ & $D\cdot (1 + O(\sqrt{W})$    & $D(1 + O(1))$   & $D(1 + O(1))$ \\
$S + \tilde{S}$ & $S$ & $S + D\cdot O(W)$ & $S + D\cdot O(W^2)$ & $S + D\cdot \overline{W} \cdot O(\hat{W}^2)$ \\
$W + \tilde{W}$ & $W$ & $W$               & $W(1 + W)$    & $W(1 + \hat{W})$\\
\hline
$\overline{D}$  & $D$ &                   &               & $\overline{D}$\\
$\overline{S}$  & $S$ &                   &               & $\overline{S}$\\
$\overline{W}$  & $W$ &                   &               & $\overline{W}$\\
\hline
\end{tabular}
\caption{A comparison of simulating \textsf{AC} interactions on three
other models. Unmarked entries signify $\overline{D}=\overline{S} = 0$ and
$\overline{W} = 1$.}
\label{tab:mod-comp}
\end{table}

A \textsf{2D NTC} circuit can only
use $SWAP$ gates to rearrange each of the $W$ qubits after each of $D$
timesteps, so that all gates can occur between nearest neighbors. We choose
to use a sorting network approach of Beals et al. \cite{Beals2012} which
has the following consequences. The sorting that occurs after every timestep
has depth $O(\sqrt{W})$ and total communication depth $\tilde{D} = D\cdot O(\sqrt{W})$.
This sorting after every timestep has size $O(W)$ (the maximum number
of $SWAP$'s required) and total communication size
$\tilde{S} = O(D\cdot W)$. Because sorting happens in-place,
no additional ancillary qubits are needed ($\tilde{W} = 0$). 

A \textsf{2D CCNTC} circuit achieves lowest-depth by using the constant-depth
communication of Section \ref{sec:intro-cdc} and Rosenbaum's
construction for reordering qubits \cite{Rosenbaum2012}. Other, more
efficient re-orderings may be possible for a specific algorithm, but
the Rosenbaum reordering is completely general. This doesn't
asymptotically increase the depth ($\tilde{D} = D\cdot O(1)$) but it does
increase size and width by the ancillary qubits needed for the reordering
grid ($\tilde{W} = W^2$, $\tilde{S} = O(W^2)$).

In the first two models, there was only one contiguous lattice
($\overline{W} = 1$) and therefore no long-range, inter-module interactions
($\overline{D} = \overline{S} = 0$). However, in the third model, we now
consider multiple modules which represent a partitioning of the qubits
in the original \textsf{AC} circuit. In the worst case, we still need to
perform a Rosenbaum reordering after each of $D$ timesteps $(\tilde{D} = D\cdot O(1)$), but this time
only within smaller modules of size at most $\hat{W}$. This gives the
number of communication operations $\tilde{S} = D\cdot \overline{W} \cdot O(\hat{W}^2)$.

The comparison between \textsf{2D NTC} and \textsf{2D CCNTC} underscores the
tradeoff between a large depth on one hand and a large ancillary qubit
overhead on the other hand.
The comparison between \textsf{2D CCNTC} and \textsf{2D CCNTCM} has two
noteworthy features.
There is a reduction in number of intra-module operations over \textsf{CCNTC} 
if the number of modules is non-constant:
$\overline{W} = \omega(1)$. However, this reduction in $S$ must be
balanced with the relative cost of $\overline{S}$, which is non-zero in
\textsf{CCNTCM} and zero in all other models.
Only physical experiments can determine the best tradeoff, but our
model will facilitate such a calculation. Achieving exact bounds on
$\overline{S}$ will be calculated later for factoring in Chapter
\ref{chap:factor-polylog}.

%We can make an observation analogous to Equation \ref{eqn:depth-width} but
%for modules in Equation \ref{eqn:module-depth-width}.

%\begin{equation}
%\overline{D} \le \overline{S} \le \overline{D}\cdot \overline{W}
%\label{eqn:module-depth-width}
%\end{equation}

%%%%%%%%%%%%%%%%%%%%%%%%%%%%%%%%%%%%%%%%%%%%%%%%%%%%%%%%%%%%%%%%%%%%%%%%%%%
\subsection{Objections to Modules and Future Directions}
\label{subsec:module-future}

Although we have demonstrate the usefulness of modules in capturing
the locality of a quantum algorithm, our new \textsf{CCNTCM} model also
contains some weaknesses. Chief among them are the possibilities of
wasteful modules
and the misrepresentation of the costs of long-range teleportation. We will
address objections to these weaknesses here before presenting useful
open questions needed to fully explore the potential of modules in
hybrid architectures.

It is allowable under the \textsf{CCNTCM} model for an architecture
to have only a small,
or constant, number of qubits in each module be useful for computation.
Alternatively, the qubits could be partitioned among modules such that most
computationally useful interactions must be long-range and involve
qubits in other modules.
Such modules are wasteful in the sense that they are not capturing any
locality.
We argue that such an architecture, and partitioning of qubits, can be
further optimized to use fewer resources, including both long-range
$\overline{S}$-type interactions and short-range $S$-type interactions.
Even an \textsf{AC} circuit can have extraneous
qubits and extraneous gates that are not useful for computation.
Bad implementations are possible with any model, and so the wasteful module
objection is not specific to \textsf{CCNTCM}.

Another objection to \textsf{CCNTCM} is that it may conceal the true
costs of long-range teleportation. Whereas the models \textsf{NTC} and
\textsf{CCNTC} present the communication costs of a circuit in a unified
way alongside computation, it can be argued that \textsf{CCNTCM} is unfairly
optimizing $S$ and $W$ compared to these two models. Indeed, for any
sufficiently localized algorithm, a \textsf{CCNTCM} architecture can be designed with asymptotically lower $S$ and $W$ than a corresponding \textsf{NTC} or \textsf{CCNTC}. However, the costs of communication are not being
misrepresented, they are just counted separately
in the module resources $\overline{D}$, $\overline{S}$, and
$\overline{W}$. Later, when physical experiments have determined the
true cost of each resource, architects can make informed decisions about
the best tradeoffs between module size and long-range interactions.
The separate accounting of \textsf{CCNTCM} reflects the belief that
practically, $\overline{S}$-type operations (or their equivalent simulation
on a contiguous lattice) will always be much slower
and expensive than $S$-type operations. A numerical comparison between
\textsf{2D CCNTC} and \textsf{2D CCNTCM} for one of our
factoring architectures is presented in Section \ref{sec:fpl-results}.

Finally, the distribution of entangled pairs necessary to allow
arbitrary connectivity of long-range teleportations has some
potential pitfalls. The polylogarithmic depth overhead
of EPR pair distribution is subsumed by the polylogarithmic
depth of our first factoring architecture in
Chapter \ref{chap:factor-polylog}, but not by our
second, sublogarithmic-depth architecture in
Chapter \ref{chap:factor-sublog}. These issues are
dealt with in more detail in the conclusions of those chapters.
More importantly, entangled pairs are likely to be generated
and distributed ``just-in-time'' to avoid unnecessary
error-correction in storing them.
Therefore, this part of shared entanglement is not included as part
of \textsf{CCNTCM} because
the technology and optimal strategy for entangled pair management
is still not well-developed.
Physical implementations could use a completely-connected
optical switch, such as the one described in \cite{Monroe2012}.
Furthermore, the qubits in a long-range teleportation always begin and end
on the peripheries of their source and destination modules,
on the edges of their respective \textsf{2D CCNTC} lattices.
Therefore, no additional teleportation is necessary to move
those qubits into position for computation.

Beyond these objections, there is still more work to be done to fully
understand the usefulness of modules and the \textsf{CCNTCM} model.
First and foremost, the connectivity
of the modules themselves. We allow teleportation between any two modules
because an entangled pair can be generated and distributed between any
two modules. However, in reality, this distribution might occur along
constrained paths between modules. Although 

Furthermore, \textsf{CCNTCM} can be defined for other dimensionalities,
which constrain the \textsf{CCNTC} lattices in each modules.
However, we concentrate on the \textsf{2D} case in this dissertation,
although the module graph $\overline{G}$ could be connected with
a \textsc{1D} topology independently of the modules, which could contain
\textsc{2D} lattices. It is a subject of future research to determine
the optimal setting of connectivities of the graphs $G$
and $\overline{G}$ independently.

Another useful investigation would be
the particular tradeoff between the number of
long-range interactions $\overline{S}$ and the module size $\hat{W}$.
In our comparison in the previous section, we make no assumptions about the 
partitioning of qubits into
modules or how the $\overline{S}$ long-range interactions are distributed
in the circuit. This prevents us from relating $\overline{S}$,
$\overline{W}$, and $W$ in the general case.
What particular features of an algorithm make this tradeoff better or worse?
Unfortunately, such investigations are beyond the scope of the current
work. In Chapter \ref{chap:factor-polylog}, we use the heuristic of linear 
size modules ($\hat{W} = O(n)$) to match the amount of computation and
communication within each module. This appears to give reasonably efficient
numerical results for our hybrid factoring architectures. We hope that
the importance of
both factoring and realistic hybrid architectures drive further work in this
area.



% Old leftover stuff
So now we have discovered that quantum computing is theoretically interesting because it can solve an important human problem. But can quantum computers be actually built?

Quantum architecture is the intermediate layer between algorithms and hardware.
It is the design of how quantum bits and their allowed interactions in order to solve these algorithms efficiently on realistic models of quantum hardware.
It aims to minimize circuit resources of interest, namely depth, size, and width. In analogy to classical circuits, the depth is the running time of an algorithm allowing parallelization, the size is the number of operations required over all parallel processors, and the width is the number of (quantum) bits required over all parallel processors.

In this dissertation, we study quantum architecture in the context of optimizing Shor's factoring algorithm on nearest-neighbor architectures with realistic constraints. It is hoped that lessons learned in this special-case can contribute to the general community general principles which can be used to generalize other quantum algorithms. We posit that the larger overall goal of quantum architecture should be the design of a general-purpose quantum processor, one which can execute any quantum algorithm with emphasis on being able to perform a core group of operations efficiently. This core group of operations is defined by the instruction set. Once quantum architecture has progressed to this point, we can leverage the remarkable strides in normal digital architecture over the past 80 years.

As well, we may be able to use the insights of quantum architecture to build quantum processors which can surpass digital architectures in the solution of large-scale problems over exponentially-sized solution spaces.

In Chapter \ref{chap:factor-polylog}, we present our first main result, a
nearest-neighbor architecture for factoring in polylogarithmic depth. This is an exponential improvement of the previous best-known result, which required quadratic depth.

In Chapter \ref{chap:factor-sublog}, we further improve our result with a
factoring architecture which executes in sublogarithmic depth, which represents another exponential improvement over our first result. Therefore, this achievement represents a remarkable doubly-exponential improvement over previous factoring architectures.

In Chapter \ref{chap:qcompile}, we study the relationship between quantum architecture and compiling. In digital computing, the boundary between architecture and compilers is quite porous and is determined by a processor's instruction set. Architecture studies processor resources to solve an algorithm given a particular instruction set which is fixed in hardware. This instruction set is produced by a compiler, a piece of (low-level) software which transforms over pieces of (high-level) software. This instruction set can change based on which algorithms it allows to solve efficiently as well as which processors it allows to manufacture efficiently as well as which operations it allows humans to understand easily. All of these factors combine to make architecture an art and an engineering discipline rather than merely a science.

Quantum computers make this problem even more difficult due to the nature of a quantum bit. Because transformations between quantum states vary continuously over the space of unitary matrices with complex coefficients, we can only approximate desired quantum logic gates using a fixed set, given to us by fault-tolerance.

This difficulty of approximation, which we call quantum compiling, represents the fundamental limit to the depth of factoring architectures. All known constant-depth factoring implementations, as well as sorting networks for those implementations which are not nearest-neighbor, assume arbitrary single-qubit rotations. That is, they assume the model where quantum compiling can be done in constant-depth, in polynomial size, for free. However, this is not the case when we combine these abstract circuit models with our model of fault-tolerant quantum computing. Under FTQC, quantum compiling depth represents a sufficient, but not known to be necessary, condition for constant-depth factoring. 

We leave this ultimately as an open problem and change directions.
Given that decreasing factoring depth with abandon increases size and width at relatively large polynomial growth rates (up to $n^8$). This is a rather undesirable time-space tradeoff.

Here, we insert the estimate of electricity and square miles and centuries of run-time based on our estimates given in the final exam.

In Chapter \ref{chap:coherence}, we introduce a novel quantum circuit resource called \emph{circuit coherence}.
