\section{Organization of Dissertation}
\label{sec:intro-conclude}

This introduction has motivated the field of quantum computing in
general, and quantum architecture in particular, by placing
Shor's factoring algorithm in context, both historically and
technologically. We have provided all the background necessary
to understand and reason about bases for quantum circuits,
quantum architectural models, constant-depth communication,
and hybrid architectural models. Although this is an
introductory chapter, we include one original result:
the constant-depth circuit for
quantum \emph{unfanout} in our new model.

The remainder of this dissertation is organized to
examine and expound upon the thesis statement above. In the first two
chapters, we focus on improving the depth of factoring architectures
in our hybrid 2D nearest-neighbor model using parallelization and
constant-depth communication. In contrast to nearest-neighbor factoring
architectures, we will refer to our results as hybrid factoring architectures,
or simply hybrid factoring.
Chapter \ref{chap:factor-polylog} presents our first main result:
hybrid factoring in polylogarithmic depth.
In a further exponential improvement, Chapter \ref{chap:factor-sublog} presents 
our second main result: hybrid factoring in sublogarithmic depth.

At this point, we have run into a seemingly fundamental limit on improving
hybrid factoring to be constant depth: the compilation of arbitrary
single-qubit gates to a fault-tolerant basis. We examine this
\emph{quantum compiling} limit in Chapter \ref{chap:qcompile}. First we
present a background of quantum compiling and a survey of the current
state-of-the-art in related literature to provide context. Then, in our
third main result, we improve
upon the Kitaev-Shen-Vyalyi procedure for generating quantum Fourier states,
a vital sub-problem for single-qubit quantum compiling.

Finally, we change directions and question the approach of minimizing depth
at all costs in the first two chapters. We argue that the most relevant
tradeoff to consider along with decreasing depth is the increasing amount
of error-correcting effort needed to maintain a useful quantum state for
computation. Therefore we introduce a new resource called
\emph{circuit coherence} in Chapter \ref{chap:coherence}. In this chapter,
we explore circuit coherence and its relationship to other circuit resources
in the hybrid model as well as to measurement-based quantum computing.
We provide definitions and an algorithm for upper-bounding coherence for
a given circuit. In our final
main result, we show the connection between the circuit coherence for a
layered quantum circuit and a reversible pebble game, for which existing
time-space tradeoffs are known. This sets the stage for proving an
asymptotic separation between circuit coherence and the circuit
depth-width product for the special case of factoring.

Finally we conclude our dissertation in Chapter \ref{chap:conclude} with a
summary of our main results along with related open conjectures and
directions for future research.