\section{Single-Qubit Rotations for Quantum Majority Gate}
\label{sec:qcompile-maj}

We now present a result delayed from Chapter \ref{chap:factor-sublog},
the quantum compiling procedure which completes the quantum majority gate.
Recall from that chapter that within each quantum majority gate, we
needed to implement single-qubit rotations of the form
$2\pi / m$, where $m = poly(n)$, where $n$ is the input size of the
number for factoring. We can augment any quantum
majority circuit with quantum compiler modules that produce
rotated ancillae of the form $\ket{0} + e^{i\phi}\ket{1}$.

To maintain our sub-logarithmic depth, we choose the KSV method for
generating quantum Fourier states and combine it with phase kickback
(PK-KSV).
The precision for quantum compiling is $m = 2^{-n'}$, and the
resulting resources for applying PK-KSV on \textsf{2D CCNTCM}
in this case are
given in Table \ref{tab:pk-ksv-resources}, based on calculations
from Section \ref{sec:qcompile-ksv}. Note that these resources
are deterministic, since they represent the worst case for any
single rotation $R_Z(e^{i\phi})$. To convert from $n'$ to $n$,
we use the relationship $n' = O(\log m) = O(\log n)$.
However, we note that it is possible to modify the Jones method of
Fourier state distillation to have similar performance.

\begin{table}[hbt!]
\begin{tabular}{|c|c|}
\hline
$D'$ & $O((\log \log n)^2)$ \\
$S'$ & $O()$ \\
$W'$ & $O()$ \\
\hline
\end{tabular}
\caption{Quantum compiling resources for PK-KSV for quantum majority gates
in factoring an $n$-bit number.}
\label{tab:pk-ksv-resources}
\end{table}

Given such a factory for producing such ancillae, we can compile the
rotations $R_Z(2\pi / m)$, for $m = poly(n)$ as in Theorem \ref{thm:maj-gate},
by directly using the KSV compiler with parameters $n' = \log n$ as in the
previous table.

We now present a more general result for
producing rotations which are rational multiples of $2\pi / m$ 
using a \emph{finite}
basis in constant depth and polynomial size.
This is not necessary for our sub-logarithmic factoring implementation,
since we can always produce our desired angles of $2\pi / m$ directly
with our quantum compiler.
In fact, no polynomial-size quantum circuit will require more than
polynomial precision for compiling single-qubit rotations. However, we
present our result in the hopes that it will be useful for other
quantum algorithms, perhaps one where we must produce the
PAR qubit $\ket{0} + e^{2\pi / m}\ket{1}$ ``offline'' and then produce
the rotation $R_Z(2\pi k / m)$ ``online.''

\begin{theorem}{\textbf{Compiling a single-qubit rotation over a non-fixed, finite basis.}}
The single-qubit rotation $R_Z(2\pi k /m)$, where $m = poly(n)$,
$k \in \mathbb{Z}_m$,
can be implemented in expected depth $O(1)$ and expected size and width $O(k)$ on
\textsf{2D CCNTCM} over the finite
(but not fixed) basis $\mathcal{G} \cup \{R_Z(2\pi / m)\}$.
\end{theorem}

\begin{proof}
We use the quantum parallelism method of Hoyer-Spalek \cite{Hoyer2002},
which relies on quantum fanout and unfanout on \textsf{2D CCNTCM}.
Our use of the basis $\mathcal{G} \cup \{R_Z(2\pi / m)\}$ implies that
we have access to quantum compiler modules for producing the
PAR qubits $\ket{0} + e^{2\pi / m}\ket{1}$. Teleport $O(k)$ such qubits
into our current circuit.
Our desired rotation of $R_Z(2\pi k / m)$ on a target qubit $\ket{\psi}$
can be produced as $k$
parallel applications of $R_Z(2\pi / m)$, which are already diagonal in
the same (computational) basis. Fan out the qubit $\ket{\psi}$ $k$ times,
apply the rotation $R_Z(2\pi /m)$ using the PAR procedure, then unfanout the
qubits.
This requires expected $O(k)$ PAR qubits.
\end{proof}

We note here two possible conjectures for improving the above result.
The first would allow us to achieve $O(\log m)$
expected size, expected width, and
expected number of teleported PAR qubits. The second would allow us to
compile arbitrary rotations to a basis that is both fixed and finite
in constant depth.

\begin{conjecture}{\textbf{Logarithmic Reduction of Compiling Circuit Size and Width.}}
The size and width of the above circuit depend on whatever additional,
finite, set of 
gates
$\{ R_Z(\phi_{k_i}) \}$ used to augment the usual \textsf{2D CCNTCM} basis
$\mathcal{G}$. Let $\phi_{k_i} = 2\pi k_i / m$, then the size and width of
a circuit applying \emph{only} $R_Z(\phi_{k_i})$
are proportional to the order of $k_i$ in
$\mathbb{Z}_m$, or equivalently, the number of times we must apply
the rotation $\phi_{k_i}$ in parallel to equal the desired rotation
$\phi_k$. Suppose we are able to find a Chinese Remainder number system
for $m$, that is, a set of pairwise coprime numbers $\{m_1, \ldots m_{t}\}$
such that $m = \prod_{i=1}^t m_i $, where $t$ and the number of bits
needed to encode each $m_i$ are $O(m)$ \cite{Yeh1996}.
The Chinese Remainder representation of $k$
is the set of $(\log_2 m)$-bit numbers
$x_i = k \bmod m_i$. 
Then we conjecture that
the finite basis $\mathcal{G} \cup \{R_Z(2\pi x_i / m\}$ satisfies the
properties above.
\end{conjecture}

\begin{conjecture}{\textbf{Constant Reduction of Compiling Depth.}}
The above bases are finite but still depend on the problem input
size $n$. It may be possible to find a basis that
is both fixed and finite that would allow for compiling
arbitrary single-qubit rotations in constant depth and polynomial
size and width, still to precision $1 / poly(n)$. This fixed
basis would be ``polynomially universal'' in that it would be
the same for all inputs of any size.
We conjecture that products of single-qubit gates in
$\mathcal{G}$ which, when diagonalized, represent $R_Z$ rotations
of irrational multiples of $\pi$, would form such a basis.
\end{conjecture}
