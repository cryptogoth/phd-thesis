\chapter{Quantum Compiling}
\label{chap:qcompile}

Quantum compiling is the approximation of an $n$-qubit
unitary operation, determined by a high-level algorithm,
using a fixed, finite, universal set
of fault-tolerant gates. Like quantum architecture, quantum
compiling plays an intermediate role between theory and experiment.
Quantum compiling helps mitigate one source of error,
the difference $\epsilon$ between a desired gate and its
finite approximation,
and consumes its own circuit resources as a function of
$(1 / \epsilon)$.

In a larger context, efficient quantum 
compiling ensures we don't lose any super-polynomial speedups
while implementing a quantum algorithm such as Shor's algorithm.
In this dissertation, we are particularly interested in
quantum compiling as a means of completing the
factoring architecture in Chapter \ref{chap:factor-sublog}.
Most interestingly, quantum compiling itself
is an algorithm and can be mapped to a hybrid nearest-neighbor
architecture with low depth. That is the subject of this chapter:
we combine a general pedagogical review of quantum compiling,
we describe a particular low-depth approach called phase kickback,
and then we present our main results: optimizing phase kickback
to \textsf{2D CCNTCM} and applying it to sub-logarithmic factoring.

In Section \ref{sec:qcompile-bg}, we notation and circuit resources,
building upon the primer on circuit bases in Section \ref{sec:intro-basis}.
We also discuss
variations and subtasks of quantum compiling that are cross-cutting themes
in the literature as well as their inter-relationships. This
provides a framework to understand the next two sections, which
survey existing related works.

In Section \ref{sec:qcompile-sk}, we review the foundational
result in this field, the Solovay-Kitaev algorithm, and how it
continues to shape quantum compiler research. We also discuss
lower bounds on any SK-style approach to quantum compiling.
Building upon this,
in Section \ref{sec:qcompile-review}, we review the large
amount of recent literature on single-qubit quantum compiling,
almost all of which has a logarithmic depth lower bound.
We then provide an
at-a-glance resource comparison of all known single-qubit quantum compiling methods
to date in Section \ref{sec:qcompile-compare}.

We then turn to the lowest-depth alternative to
the single-qubit quantum compilers discussed previously
in Section \ref{sec:qcompile-qfs}. This new approach
trades increased width for low depth by combining \emph{phase kickback}
with a quantum Fourier state.

In Section \ref{sec:qcompile-ksv}, we contribute an improved
algorithm for generating quantum Fourier states based on
Kitaev-Shen-Vyalyi \cite{Kitaev2002}. We also calculate the
parameters necessary for any practical implementation as well as
the particular circuit resources consumed on \textsf{2D CCNTCM}.
We then compare the KSV approach to a recent alternative
by Jones which distills quantum Fourier states recursively \cite{Jones2013}.
Finally, in Section \ref{sec:qcompile-maj}, we contribute the
missing piece from our sub-logarithmic factoring architecture
from Chapter \ref{chap:factor-sublog}: single-qubit rotations
compiled in sub-logarithmic depth.

These last two sections represent the main contributions of this chapter.

%Finally, in Section \ref{sec:qcompile-conclude}, we conclude by summarizing
%the overall themes of quantum compiling and presenting interesting
%directions for future research.

\section{Quantum Compiling Themes}
\label{sec:qcompile-bg}

Quantum compiling is a classical procedure for transforming quantum circuits.
The image to keep in mind throughout this entire section is shown in
Figure \ref{fig:qcompile}.

\begin{figure}
\begin{center}
\begin{displaymath}
\begin{array}{ccc}

%%%%%%%%%%%%%%%%
% Source circuit
%\underbrace{
\begin{array}{c}
S = 2 \\
\Qcircuit @C=0.5em @R=.5em { 
	& \multigate{4}{U_1} & \qw & \multigate{4}{U_2} & \qw \\ 
	& \ghost{U_1}        & \qw & \ghost{U_2}        & \qw \\
	& \ghost{U_1}        & \qw & \ghost{U_2}        & \qw \\
	& \ghost{U_1}        & \qw & \ghost{U_2}        & \qw \\
	& \ghost{U_1}        & \qw & \ghost{U_2}        & \qw 
	\gategroup{1}{2}{5}{4}{.7em}{--}
}\\
\xymatrix {
  & D=2 \ar[l] \ar[r] & \\
 }
\end{array}
%}_{C}

%& 
%\begin{array}{c}
%\textsc{Quantum Compiler} \\
\rightarrow
%\end{array}
%&

%%%%%%%%%%%%%%%%
% Target circuit
%\underbrace{
\begin{array}{c}
S' = 15 \\
\Qcircuit @C=0.5em @R=.5em { 
	& \gate{H} & \qw & \ctrl{1} & \gate{H} & \qw & \qw      & \ctrl{1} & \qw \\ 
	& \gate{H} & \qw & \targfix & \ctrl{2} & \qw & \gate{K} & \targfix & \qw \\
	& \gate{H} & \qw & \gate{K} & \qw      & \qw & \gate{H} & \qw      & \qw \\
	& \gate{H} & \qw & \ctrl{1} & \targfix & \qw & \gate{H} & \qw      & \qw \\
	& \gate{H} & \qw & \targfix & \gate{H} & \qw & \qw      & \qw      & \qw
	\gategroup{1}{2}{5}{9}{.7em}{--}
}\\
\xymatrix {
  & & D'=5 \ar[ll] \ar[rr] & & \\
 }
\end{array}
%}_{C}

\end{array}
\end{displaymath}

\caption{An arbitrary quantum circuit being compiled into single-qubit gates and $CNOT$.}
\label{fig:qcompile}
\end{center}
\end{figure}

General quantum compiling can be subdivided into more special-purpose tasks along several axes,
which are cross-cutting themes in any literature review of quantum compilers.
These themes also provide a context for understanding the resource consumption
for a wide variety of quantum compilers.

These axes are:

\begin{enumerate}
\item single-qubit compiling versus multi-qubit compiling
\item exact synthesis versus approximative quantum compiling
\item deterministic versus probabilistic quantum compiling
\item compilers with provable upper bounds versus conjectured upper bounds
\end{enumerate}

The first axis is 
single-qubit compiling
(mentioned previously in Section \ref{subsec:qcompile-single}) versus
multi-qubit compiling. Some algorithms which work on single-qubit compiling
can be generalized directly to the multi-qubit case. In fact, all known
examples of these generalized algorithms can accept an arbitrary circuit
basis $\mathcal{B}$ \cite{Amy2012,Dawson2005,Fowler2011,Booth2012}.
That is, they do not exploit any special structure of
a particular basis. The circuit basis is another input to the algorithm,
possibly to an additional classical preprocessing step. Whether the algorithm
is a single-qubit or a multi-qubit algorithm depends on whether the basis
is single-qubit or multi-qubit.

There is an intermediate point on this axis, between single-qubit and multi-qubit,
which is the reduction of a multi-qubit circuit into a basis of
single-qubit and two-qubit gates. This task is often called \emph{quantum circuit synthesis},
and we will discuss it in Section \ref{subsec:qcompile-multi}.

The second axis is compiling a circuit exactly or approximately.
Exact synthesis refers to the case of determining whether a
target circuit $C$ is implementable from a basis $\mathcal{B}$
with no error ($\epsilon = 0$). If this is possible, a quantum compiler
should return the sequence of gates which constitute the exact
synthesis. Furthermore, exact synthesizers often have a goal of
returning the \emph{optimal} sequence of compiled gates, that is,
one with minimal length $\ell$. In the compilers that we review
in Section \ref{sec:qcompile-review}, $\ell$ stands for the optimal
depth of non-Clifford resources in a basis which also contains Clifford
gates. Non-Clifford resources are always more expensive than
Clifford gates in most error-correcting codes. There is evidence
that the Clifford resources needed to synthesis the non-Clifford gates $T$ and $Toffoli$
are within a small constant factor of each other \cite{Eastin2012,Jones2012}

Exact synthesizers often enumerate over all circuits of
a certain length from a certain basis $\mathcal{B}$. Therefore, their
resources are upper bounded by a brute force search, which takes
time upper-bounded by $|\mathcal{B}|^{\ell}$.
Approximative quantum compiling conforms to our usual notion where
$\epsilon > 0$, and achieving smaller error costs more resources. Many
exact synthesis algorithms can be used to build basic approximations
for the Solovay-Kitaev algorithm more efficiently, and therefore help
achieve better approximative upper bounds as verified by numerical
simulation over random unitaries.

What is the relationship between $\ell$ and $\epsilon$? By a volume
argument, the minimum number of points in an $\epsilon_0$-net for
$SU(d=2^n)$ is $1/(\epsilon^{d^2 - 1})$. If we were to do an approximative
search within error $\epsilon_0$
for a circuit in $SU(d)$ which is known to have optimal length
$\ell$, we would have to enumerate all sequences from a basis $\mathcal{B}$
of up to length $\ell$ in the worst case, of which there are $|\mathcal{B}|^{\ell}$.
Therefore, we have the following relationship.

\begin{equation}
\ell \ge (d^2 - 1) \log_{|B|}(1/\epsilon) + O(1)
\end{equation}

The third axis is whether a quantum compiling algorithm uses randomness
or is completely deterministic. For known randomized algorithms, it is
an open problem whether the algorithm can be derandomized or not
\cite{Kliuchnikov2012a}, and numerical verification is necessary to
show the desired distribution of running times.

The fourth axis is whether a quantum compiler has upper bounds
(usually on running time or compiled sequence length) that are provable or
based on a conjecture. Both deterministic and randomized
algorithms can have provable upper bounds, although
in the latter case, one calculates the average-case and upper bounds the
variance. Likewise, both deterministic and randomized algorithms can
be based on a conjecture. One example is a deterministic algorithm
whose resources are too difficult to compute in any other way than
numerical simulation and fitting a curve to the data.

These four axes can be used to classify quantum compilers, although some
algorithms can be placed in multiple categories. For example, many
single-qubit quantum compilers which perform exact synthesis can be
incorporated into a hybrid algorithm which then performs
approximation. And of course, some single-qubit quantum compilers can be generalized
into multi-qubit algorithms.

A fifth axis could be formed, which is whether the compiled circuit requires
arbitrarily long interactions for $CNOT$ or is nearest-neighbor. Such a
quantum compiler could also divide up a compiled circuit into an optimal
number of modules on \textsf{2D CCNTCM} to also minimize module depth and
module size (inter-module teleportations). This is an interesting direction
for future research.

%%%%%%%%%%%%%%%%%%%%%%%%%%%%%%%%%%%%%%%%%%%%%%%%%%%%%%%%%%%%%%%%%%%%%%%%%%%%%%
\subsection{Quantum Compiler Resources}

Just as a quantum algorithm with arbitrary long-range interactions incurs
some overhead in being mapped to a nearest-neighbor architecture,
a quantum compiler itself is an algorithm. It always has a classical
component, which runs on a digital computer, and transforms a classical
description of an input quantum circuit into an output circuit from
a basis $\mathcal{B}$. The compiled output circuit then runs on a
quantum computer. In general, the compiled output circuit $\tilde{C}$ consumes
resources which are greater than those of the input circuit $C$.

Not all quantum compilers are ``total functions.'' Some of them, notably
single-qubit compilers, are ``promise functions'' in that they can
only compile gates of a certain form (usually $R_Z(\phi)$) and require
prior decomposition of a multi-qubit gate down to the set of
$Q \cup \{R_Z(\phi)\}$.

\begin{description}
\item[classical runtime $R$:] the classical time it takes to return a 
compiled quantum circuit.
\item[input depth $D$:] the depth of the input quantum circuit in arbitrary
$n$-qubit gates.
\item[input size $S$:] the size of the input quantum circuit in arbitrary
$n$-qubit gates.
\item[input width $W$:] the width of the input quantum circuit in qubits.
\item[compiled depth $D'$:] the compiled quantum circuit depth, equal to
the compiled sequence length for single-qubit circuits.
\item[compiled size $S$:] the compiled quantum circuit size, which is
identical to compiled depth if no ancillae are used (compiled width is zero).
\item[compiled width $W$:] the compiled quantum circuit width, which includes
the width of the input circuit as well as any ancillae introduced by
the compiler.
\end{description}

All but the first resource are quantum in nature, and follow the definitions
for circuit resources from Chapter \ref{chap:factor-polylog}. Because
compilation incurs some overhead, we have $D' \ge D$, $S' \ge S$, and
$W' \ge W$.

It's also known that
in order to approximate a circuit with $S$ gates to a total precision of
$\epsilon$
requires each gate to be approximated to a precision of
$n = O(\log(S/\epsilon)$ \cite{Lloyd1995}. We denote this per-gate precision
$n$, since it serves as an independent parameter for compiling. For
single-qubit gates, $S = 1$, and this corresponds exactly with our previous
definition for $n$ in Section \ref{sec:qcompile-basis}.

We do not measure classical space requirements, although these may be
exponential. This would be a useful metric for comparison for future work.

%%%%%%%%%%%%%%%%%%%%%%%%%%%%%%%%%%%%%%%%%%%%%%%%%%%%%%%%%%%%%%%%%%%%%%%%%%%%%%
\subsection{Decomposition to Bounded-Qubit Gates}
\label{subsec:qcompile-multi}

Restricting ourselves to the simplest case of
single-qubit circuits allows us to exploit a lot of structure
in the group $U(2)$ (or its related subgroups $SU(2)$ and $PSU(2)$).
From a volume argument, we can derive a general
lower bound for the efficiency of the multi-qubit case \cite{Harrow2002},
as well as determine how our compiling efficiency scales with dimensionality.
Any
SK-style algorithm produces worst-case sequence lengths $\ell_d$ that
are longer than worst-case single-qubit sequence lengths $\ell_1$ by a certain multiplicative
prefactor. This prefactor has a dependence that is at least
polynomial in $d = 2^n$. 

\begin{equation}
% TODO fact check this!
\ell_d / \ell_1 = \Omega \left( \frac{d^2 - 1}{ \log |\mathcal{B}| } \right )
\end{equation}

This is an example of task modularity which allows us
to divide the effort of quantum compiling between the
single-qubit case and then decomposition to single-qubit gates and
$CNOT$. It is a heuristic which often results
in simple decompositions to implement in (classical) software.
It may not be asymptotically optimal compared to generic 
multi-qubit protocols. However, for small input sizes, it is often
tractable to run on modern digital computers.

Now that we have handled the single-qubit case, how can we leverage this
to compile general $n$-qubit gates? We need a reduction to the basis
$\mathcal{Q} = U(2) \cup \{ CNOT \}$, as originally depicted in
Figure \ref{fig:qcompile}.
It turns out that almost any two-qubit gate plus arbitrary single-qubit
rotations are universal \cite{Bremner2002}. However, we will stick with CNOT
due to its other useful properties. A table of known
upper and lower bounds for this task are given in Table \ref{tab:multi}.
A standard two-level decomposition, such as provided on page 70 of \cite{Kitaev2002},
decomposes a general $U(d=2^n)$ matrix down to $O(d^2)$ ``two-level'' matrices which can
be implemented with multiply-controlled single qubit gates $\Lambda^{n-1}(U)$
for $U \in U(2)$. These gates are implementable with $O(n)$ $CNOT$ gates each.

The recent Giles-Selinger proves a conjecture by Kliuchnikov-Mosca-Maslov \cite{Kliuchnikov2012e}
that $n$-qubit circuits implementable by the basis $\mathcal{C}_2 \cup \{ T \}$ is
equivalent to all $U(2^n)$ matrices with elements from the ring $\mathbb{Z}\left[i,\frac{1}{\sqrt{2}}\right]$.
Their construction to find this exact synthesis of an $n$-qubit gate is meant to
prove this equivalent, and is not optimal.

The optimal bound needed for this in terms of $CNOT$ gates in
the compiled output (the dominant cost) is still exponential
$O(4^n)$ \cite{Shende2004}.

\begin{table}[hbt!]
\centerline{
\begin{tabular}{|c|c|}
\hline
Decomposition Method & $CNOT$ Cost 
\hline
Giles-Selinger \cite{Giles2012} & $O(9^n nk)$\\
Two-level \cite{Kitaev2002} & $O(4^n n)$ \\
Vartiainen-M\"{o}tt\"{o}nen-Salomaa \cite{Vartainen2003} & $o(11\cdot 4^n)$ \\
Aho-Svore \cite{Aho2003} & $o(1.17\cdot 4^n - 3.51\cdot 4^n + 3.34)$ \\
Shende-Bullock-Markov \cite{Shende2004a} & $o(0.48\cdot 4^n - 1.50\cdot 2^n + 1.34)$ \\
Shende-Bullock-Markov \cite{Shende2004} & $\omega(0.25\cdot 4^n - 3n - 1)$ \\
BBC+ \cite{Barenco1995a} & $\omega(0.10\cdot 4^n - 0.34n - 0.12)$ \\
\hline
\end{tabular}
}
\label{tab:multi}
\caption{Comparison of multi-qubit circuit synthesizers in $CNOT$ cost, both upper and lower bounds.
The $o(\cdot)$ and $\omega(\cdot)$ notation are used to indicate when multiplicative constants are known.}
\end{table}

\section{The Solovay-Kitaev Algorithm}
\label{sec:qcompile-sk}

As one of the central results of quantum computation, it is worth
reviewing here the venerable Solovay-Kitaev (SK) algorithm for the
approximation of $SU(d)$ gates by a fixed, finite basis $\mathcal{B}$.
Although many recent results have surpassed SK in terms of efficiency,
they often do so by improving the base-level approximation of the
original SK algorithm. Moreover, many techniques for analyzing
and understanding quantum compilers were developed for SK, which
continues to be the best way to learn them. Finally, the overall
structure of the original SK algorithm is so simple, it is surprising
that it works so well. Therefore, it is worth
our time to review it here.

The essential structure of SK is to recursively generate nets of unitary
operators with
successively finer precision. At any given level of recursion, the input 
gate is divided up into two halves which can be
approximated with less precision. Our pseudo-code and explanation
follows the exposition in \cite{Dawson2005} and \cite{Harrow2001}

\begin{algorithmic}[1]
\STATE \textsc{function} $\tilde{U}_i \leftarrow$ SK$(U,n)$
\IF{$i = 0$}
\STATE $\tilde{U}_i \leftarrow $ BASIC-APPROX$(U)$
\ELSE
\STATE $\tilde{U}_{i-1} \leftarrow$ SK$(U, i-1)$
\STATE $A,B \leftarrow $ FACTOR$(U\tilde{U}^\dagger_{i-1})$
\STATE $\tilde{A}_{i-1} \leftarrow $ SK$(A, i-1)$
\STATE $\tilde{B}_{i-1} \leftarrow $ SK$(B, i-1)$
\STATE $\tilde{U}_i \leftarrow \tilde{A}_{i-1}\tilde{B}_{i-1}\tilde{A}^\dagger_{i-1}\tilde{B}^\dagger_{i-1}\tilde{U}_{i-1}$
\ENDIF
\STATE return $\tilde{U}_i$
\end{algorithmic}

Since the recursion must eventually bottom out, we must precompute some sequences
of gates from $\mathcal{B}$ up to length $l_0$. This is the classical
preprocessing step which requires upfront storage space for this
coarsest-grained net, where
each sequence is no more than $\epsilon_0$ from its nearest neighbor. According
to \cite{Dawson2005} the values of $l_0=16$ and $\epsilon_0 = 0.14$ using
operator norm distance is sufficient for most applications.
This step can be
done offline and reused across multiple runs of the compiler, assuming
$\mathcal{B}$ for your quantum computer doesn't change.

The BASIC-APPROX function above does a lookup (e.g. using some kd-tree search
maneuvers through higher-dimensional vector spaces) using this $\epsilon_0$-net,
and all higher recursive calls to SK are effectively constructing
finer $\epsilon$-nets on the fly as needed.

The FACTOR function performs a balanced group commutator decomposition,
$U = ABA^\dagger B^\dagger$, and then recursively approximates the $A$ and $B$
operators, again using SK. We denote by $\tilde{U}_i$ as the approximation
of $U$ using $i$ levels of SK recursion. When they are multiplied
together again, along with their inverses, their errors (which go like
$\epsilon$) are symmetric and cancel out in such a way that the resulting
product $U$ has errors which go like $\epsilon^2$, using the properties of
the balanced group commutator. In this manner, we can
eventually sharpen our desired error down to any value. A
geometric decomposition is used in the Dawson-Nielsen implementation \cite{Dawson2005},
while one based on a Baker-Campbell-Hausdorff approximation is used in the
Harrow implementation \cite{Harrow2001}. It is not
known which method converges more quickly to a desired gate in general.

%%%%%%%%%%%%%%%%%%%%%%%%%%%%%%%%%%%%%%%%%%%%%%%%%%%%%%%%%%%%%%%%%%%%%%%%%%%%%%%
%%%%%%%%%%%%%%%%%%%%%%%%%%%%%%%%%%%%%%%%%%%%%%%%%%%%%%%%%%%%%%%%%%%%%%%%%%%%%%%
% we still need to reconcile this 5^n with log^{3.97}(1/\epsilon)
% this is an evernote item, maybe it is already reconciled

Each level $i$ of recursion solves the problem of compiling 
$U_i$ by combining five gates compiled at the lower $i-1$ level.
Therefore, the compiled circuit size at the top-level is upper-bounded by $5^n$,
and
this is in general the same as the circuit depth (since not all gates in
$\mathcal{B}$ commute).

%%%%%%%%%%%%%%%%%%%%%%%%%%%%%%%%%%%%%%%%%%%%%%%%%%%%%%%%%%%%%%%%%%%%%%%%%%%%%%%
%%%%%%%%%%%%%%%%%%%%%%%%%%%%%%%%%%%%%%%%%%%%%%%%%%%%%%%%%%%%%%%%%%%%%%%%%%%%%%%
% we still need a discussion somewhere of the relationship between commuting
% operators and parallelizing them into the same timestep

\section{Quantum Compiler Review}
\label{sec:qcompile-review}

\subsection{Alternative Bases and Resources}

Now we turn to three recent works which use magic state distillation to compile
arbitrary single-qubit gates. These works either compile to an alternate
basis from the Clifford group $\mathcal{C}_n$ and $T$

A recent approach by Bocharov-Gurevich-Svore \cite{Bocharov2013}
compiles to subsets of the Clifford group augmented with the non-Clifford $V$-basis, which was proven to permit the lower-bound of compiled sequence length $O(\log^1(1/\epsilon)$
\cite{Harrow2003}.

\begin{equation}
V_1 = TODO \qquad V_2 = TODO \qquad V_3 = TODO
\end{equation}

This work uses the properties of Lipschitz quaternions with norms $5^l$, ($l \in \mathbb{Z}, l \ge 0$). It
contains a randomized algorithm whose running time is based on a conjecture from geometric number theory.
There is currently no complete, fault-tolerant method of compiling all three gates from the $V$ basis into
our usual universal set of $\mathcal{C}_1 \cup \{T\}$. However, the appendix of \cite{Bocharov2013}
gives a method for implementing the exact $V_2$ gate using the (probabilistic) magic state distillation of
Duclos-Cianci and Svore \cite{DuclosCianci2012}. 
We cannot compare it directly to previous algorithms which consider the number of $T$ gates ($T_c$)
the primary resource, or the compiled sequence length $D'$ as an upper bound to $T_c$.

\section{Quantum Compiler Comparison}
\label{sec:qcompile-compare}

\begin{landscape}

\begin{table}[hbt!]
\begin{center}
\begin{tabular}{|c|c|c|c|c|c|c|c|c|}
\hline
Algorithm                  & A/E & D/R & P/C & Multi? & $R$                             & $D'$                    & $S'$                  & $W'$ \\
\hline
Fowler\cite{Fowler2011}     & A   & D   & P  & Y    & $O(\ell |\mathcal{B}|^{\ell})$    &                         &                       &     \\
Booth \cite{Booth2012}      & A   & D   & P  & Y    & $O(\ell |\mathcal{B}|^{\ell/2})$  &                         &                       & 1    \\
\hline
AMMR \cite{Amy2012}         & E   & D   & P  & Y    & $O(\ell |\mathcal{B}|^{\ell/2})$  &                         &                       & 1    \\
BS \cite{Bocharov2012}      & E   & D   & P  & N    & $O(\ell |\mathcal{B}|^{\ell/4})$  &                         &                       & 1    \\
%BSG \cite{Bocharov2013}     & A   & D   & N    & $O(\ell |\mathcal{B}|^{\ell/4})$ & $\ell$           & $D'$ & 1    \\
KMM-e\cite{Kliuchnikov2012e} & E   & D   & P  & ?    & $\ell^2$                          &                         &                       & 1    \\
\hline
SK-DN\cite{Dawson2005}      & A   & D   & P  & Y    & $O(\eta^{2.71} + \mathcal{B}^{l_0})$ & $O(D \eta^{3.97})$         &                       & 1    \\
SK-KSV\cite{Kitaev2002}     & A   & D   & P  & N    & $O(\eta^{3+o(1)})$                   & $O(D \eta^{3+o(1)})$        &                       & 1    \\
SK-BS \cite{Bocharov2012}   & A   & D   & C  & N    & $O(\eta^{2.71} + \mathcal{B}^{l_0})$ & $O(D\eta^{3.4})$           &                       & 1    \\
KMM-a\cite{Kliuchnikov2012a}& A   & P   & C  & N    & $O(\eta^2\log n)$                    & $O(D\eta)$                 &                       & 1    \\
Selinger\cite{Selinger2012} & A   & P   & C  & N    & $O(\eta^4)$                          & $D(48\eta + 44)$           &                       & 1    \\
\hline
PK-KSV \cite{Kitaev2002,Cleve2000} & E & D & P & N  & $O(1)$                            & $O(D\log \eta + \log^2 \eta)$ & $O(S \eta + \eta^2 \log \eta)$ & $O(\eta^2)$ \\
PK-Jones \cite{Jones2013}   & E  & D    & P  & N     & $O(1)$                           & $O(D\log \eta + \eta \log \eta)$ & $O(S \eta + \eta \log \eta)$   & $2\eta + O(1)$ \\
\hline
\end{tabular}
\caption[A comparison of single-qubit quantum compilers]{A comparison of single-qubit quantum compilers, where $\eta = \log_2(1/\epsilon)$, for a desired error $\epsilon$.
Blank depths $D'$ are $\ell$. The maximum basic approximation length is $\ell_0$ \cite{Dawson2005}. Blank sizes $S'$ are equal to $D'$. Blank widths $W'$ are equal to $1$.
A/E: Approximate versus exact-synthesis. D/R: Deterministic versus randomized. P/C: provable bounds versus conjectured bounds + numerical verification.
Multi?: Y if it can be generalized to a multi-qubit compiler.}
\label{tab:qcompile-compare}
\end{center}
\end{table}

\end{landscape}


\section{Phase Kickback and Quantum Fourier States}
\label{sec:qfs}

In this section, we discuss the \emph{phase kickback} procedure for
producing a qubit with arbitrary phase: $\ket{0} + e^{i\phi}\ket{1}$.
From the PAR procedure, a qubit could be prepared in such a state
``offline' and then teleported later into a \textsf{2D CCNTCM} module to
probabilistically enact rotations $R_Z(\phi)$ \cite{Jones2011}.

Phase kickback requires controlled addition on a target state in 
the quantum Fourier basis. This will provide us the foundation
to understand the KSV procedure for generating such quantum Fourier
states in the next section.

In Section \ref{subsec:qfs-basis}, we discuss the properties of the
basis of quantum Fourier states, including their relationship with
addition modulo $2^n$. In Section \ref{subsec:qfs-resources},
we calculate circuit resources for performing these adders
on \textsf{2D CCNTCM}. Although the adders are generic, they will be
used in the resource calculations of KSV phase estimation in
Section \ref{sec:ksv-pe}, which operates on quantum Fourier states.

%%%%%%%%%%%%%%%%%%%%%%%%%%%%%%%%%%%%%%%%%%%%%%%%%%%%%%%%%%%%%%%%%%%%%%%%%%%%%%
\subsection{Properties of the Quantum Fourier Basis}
\label{subsec:qfs-basis}

The following states are well-known as $n$-qubit quantum Fourier states,
the result of applying the quantum Fourier transform (QFT)
to the $n$-qubit computational basis state. 

\begin{equation}
\ket{\psi^{(k)}_{n}} = \frac{1}{\sqrt{2^n}} \sum_{j=0}^{2^n-1}
e^{-2\pi i j k / 2^n} \ket{j}
\end{equation}

These states form an alternate, orthonormal basis indexed by
$0 \le k < 2^n$. We will often just call these Fourier states,
and neglect the subscript $n$, which is implied.
Note that the state $\ket{\psi^{(0)}}$, sometimes
called the fundamental Fourier state, is simply the equal superposition
of all computational basis states, or the tensor product of
$n$ qubits in the state $\ket{+}$, or the result of applying
$n$ Hadamard's qubit-wise to a state beginning in $\ket{0}^n$.

Note that these states $\ket{\psi}^{(k)}_n$ are the
QFT states of the $n$-qubit computational basis. The usual method of
creating these states involves performing phase estimation of the
modular addition operator. These are implicitly hard in that
all known procedures take size $O(n\log n)$, even if the depths
can be decreased to $O(\log n)$ \cite{Jones2012} or in some cases $O(1)$
given unbounded quantum fanout
\cite{Browne2009}.

However, we can create a superposition in constant depth
over all odd $k=(2s-1)$
by starting in the state $\ket{0}^{\otimes n}$,
then applying a Hadamard and $\sigma^z$ to the most significant qubit.

\begin{equation}
\ket{\eta} = \normtwo \ket{0} - \normtwo \ket{2^{n-1}} =
\frac{1}{\sqrt{2^{n-1}}} \sum_{s=1}^{2^{n-1}} \ket{\psi_{n,2s-1}}
\end{equation}

More obviously relevant to our overall goal of approximating
$\Lambda(e^{i\phi})$, we can enact a phase
shift simply by performing the following modular addition operator, for
which $\ket{\psi_{n,k}}$ are eigenstates.

\begin{equation}
A\ket{j} \rightarrow \ket{j+1 \bmod 2^n}
\end{equation}

Applying this operator to its eigenstates results in a phase shift which
depends on the particular eigenstate.
 
\begin{equation}
A\ket{\psi_{n,k}} = e^{2\pi i \phi_k} \ket{\psi_{n,k}}
\end{equation}

Finding the eigenvalue $e^{2\pi i \phi_k}$ corresponds to finding
the phase $\phi_k = k / 2^n$.
Repeated application of $A$ (say $p$ times) would result in a phase
added to the eigenstate equal to a multiple of $e^{2\pi i p / 2^n}$

\begin{equation}
A^p\ket{\psi_{n,k}} = e^{2\pi i \phi_k / 2^n} \ket{\psi_{n,k}}
\end{equation}

This explains why we don't find even $k$ interesting,
since then we would not get a
cyclic distribution of $2^n$ different phases,
since only odd $k$
are coprime with $2^n$. The exception is $k=0$, since this is the
equal superposition of computational basis states, which we can also
efficiently create. This will be a useful starting point later on to
create addition eigenstates
for odd $k$.

\begin{displaymath}
\ket{\psi_{n,0}} = H^{\otimes n}\ket{0^n}
\end{displaymath}

Suppose we have a certain state $\ket{\psi_{n,k}}$ but we want to get enact
a phase shift $e^{2\pi i l / 2^n}$. We can do this by solving $p=p(s,l)$
in this equation:

\begin{equation}
\label{eqn:psl}
(2s-1)p \equiv l (\bmod 2^n)
\end{equation}

Stipulating $k$ to be odd guarantees that there is a unique solution $p$.

We then apply $A^p$ as follows, where $\Upsilon_n(A)$ means to
apply $A$ to the second register $p$ times, where $p$ is an $n$-qubit
number in the first register.

\begin{equation}
\label{eqn:upsilon}
\Upsilon_n(A) \ket{p}\ket{\psi_{n,k}} \rightarrow
e^{2\pi i l/2^n} \ket{p}\ket{\psi_{n,k}}
\end{equation}

If we control the operation of $\Upsilon_n(A)$ on a source qubit $\ket{+}$,
it will acquire the phase $e^{2\pi i l}$.

\begin{equation}
\Lambda(\Upsilon_n(A))\ket{+}\ket{p}\ket{\psi}^{(k)} \rightarrow
\left( \ket{0} + e^{2\pi i l}\ket{1} \right) \ket{p}\ket{\psi}^{(k)}
\end{equation}

This is not quite the phase kickback procedure, since
we must still solve for $p$ using the equation below by finding the
modular inverse $(2s - 1)^{-1} \bmod 2^n$,
making use of the following expansion from Section 13
of \cite{Kitaev2002}

\begin{equation}
p \equiv -l\sum_{j=0}^{m-1} (2s)^j \equiv -l \prod_{r=1}^{t-1}\left(1 + (2s)^{2^r}\right) \mod 2^n
\end{equation}

where $m = O(n)$ is $n$ rounded to the nearest power of $2$.  In general,
this requires a circuit of size $O(n^2 \log n)$ and depth $O((\log n)^2)$ and
represents the most expensive part of the KSV procedure as originally
presented in \cite{Kitaev2002}.

Ideally, we would obviate the need for the expensive circuit above
by ensuring that $k=1$, in which case
$p = l$. We will see how to do this in Section \ref{sec:ksv-pe}.

Finally, to copy the state $\ket{\psi_{n,k}}$ it suffices to apply the following
operator which only uses subtraction (addition with one addend and the
outcome negated in two's complement representation).

\begin{equation*}
\ket{\psi_{n,k}}^{\otimes m} = W^{-1}\left( \ket{\psi_{n,0}}^\otimes(m-1) \otimes \ket{\psi_{n,k}} \right)
\end{equation*}

where $W$ is the operator on $m$ registers, each of consisting of $n$ qubits,
which adds all the registers into its final register (modulo $2^n$).

\begin{multline}
W : \ket{x_1,\ldots,x_{m-1},x_m} \rightarrow \\
 \ket{x_1,\ldots,x_{m-1},x_1+\ldots+x_m \bmod 2^n}
\end{multline}

Having complained about the cost of modular division, how can we now
implement the operators $A$ and $W$ more efficiently on our
nearest-neighbor architecture?

%%%%%%%%%%%%%%%%%%%%%%%%%%%%%%%%%%%%%%%%%%%%%%%%%%%%%%%%%%%%%%%%%%%%%%%%%%%%%%
\subsection{Circuit Resources for Adders on Quantum Fourier States}

First we prove a general result for mapping any \textsf{AC} circuit
to a \textsf{2D CCNTCM} circuit.
Then we use it to 
map the operator $A$ (increment modulo $2^n$) and the operator
$W$ (multiple addition modulo $2^n$) to \textsf{2D CCNTCM}.

\begin{lemma}
Suppose an \textsf{AC} circuit $\mathcal{C}$ has
depth $D(n)$, size $S(n)$, and width $W(n)$. Then it can be mapped
onto a \textsf{2D CCNTCM} circuit $\mathcal{C'}$ with circuit depth
$D(n)$, circuit size $S(n)$, circuit width $D(n)\cdot W(n)$,
module depth $D(n)$, module size $D(n)\cdot W(n)$, and module width
$D(n)$.
\label{lem:ac-ccntcm}
\end{lemma}

\begin{proof}
Each of $D(n)$ layers in $\mathcal{C}$ could operate on non-nearest-neighbors.
We teleport all
$W(n)$ qubits from the first module to the last module in sequence,
reordering them at each module so that we can execute the gates of that
layer only on nearest-neighbor qubits.
This gives the desired circuit and module resources on
\textsf{2D CCNTCM}.
\end{lemma}

\begin{lemma}
The operator $A\ket{j} \rightarrow \ket{j+1 \bmod 2^n}$ on
$n$-qubit register can be
implemented in circuit/module depth $O(\log n)$, circuit size $O(n)$, 
circuit width $O(n\log n)$, module size of $O(n \log n)$, and
module width of $O(\log n)$ on
\textsf{2D CCNTCM}.
\label{lem:a}
\end{lemma}

\begin{proof}
We modify the QCLA adder from
Section \ref{subsec:qcla} mapped to \textsf{2D CCNTCM} using
Lemma \ref{lem:ac-ccntcm}.
To add $2 \times n$-bit numbers, the original \text{AC} adder has
depth $O(\log n)$ and size and width of $O(n)$.
Therefore, we have the desired resources for a \textsf{2D CCNTCM} circuit.
\end{proof}

\begin{lemma}
The operator $W\ket{x_1}\ket{x_2}\cdots\ket{x_m} \rightarrow \ket{x_1}\ket{x_2}\cdots\ket{x_1 + x_2 + \ldots + x_m \bmod 2^n}$,
which operates on $m \times n$-qubit registers, can be
implemented in circuit/module depth $O(\log n + \log nm)$, circuit size $O(mn)$, 
circuit width $O(mn + n\log n)$, module size of $O(mn + n \log n)$, and
module width of $O(m + \log n)$ on
\textsf{2D CCNTCM}.
\label{lem:w}
\end{lemma}

\begin{proof}
This reduces to the problem of modular multiple addition from
Section \ref{subsec:mma}, with the following variation. Instead of
addition modulo $m < 2^n$, we perform addition modulo $2^n$, which is
much simpler. It does not involve any truncation and adding back of
modular residues. Namely, we just perform one round of constant-depth
$3 \rightarrow 2$ addition, then we redo the computation of the
highest-order bit $v_n$ to uncompute it.

Therefore, we have a $O(\log m)$-depth binary tree of $O(m)$ modules,
which requires a total of $O(mn)$ teleportation of qubits between them.
Within each module, the addition takes depth $O(1)$ and size and width
of $O(n)$. Combine this in series with the circuit for $A$ in
Lemma \ref{lem:a} and we have the desired resource bounds.
\end{proof}

Armed with these properties, we're now ready to describe our version of the
KSV phase estimation procedure to product quantum Fourier states.

\section{Circuit Resources for the KSV Quantum Compiler}
\label{sec:qcompile-ksv}

This section gives a pedagogical review of the quantum compiling method
that combines phase kickback and QFS generation due to
Kitaev-Shen-Vyalyi \cite{Kitaev2002}. In this form, it compiles single-qubit
of the form $R_Z(\phi)$. Using a multi-qubit
decomposition such as those presented in Section \ref{subsec:qcompile-multi},
it can be extended to compile multi-qubit gates.
Furthermore, we present an
original optimization
called \emph{early measurement} which does not asymptotically increase
our compilation resources (see Appendix \ref{app:ksv-error}). We refer to
our optimized method as PK-KSVe, which still shares much in common with
PK-KSV up to the phase estimation step.
Finally, we contribute the architectural resources on $\textsf{2D CCNTCM}$ for running 
three different QFS generation.

First, in Section \ref{subsec:ksv-diff} we discuss our optimization of early measurement, which distinguished
PK-KSVe from the original PK-KSV algorithm.
Next,
in Section \ref{subsec:ksv-steps} we present the high-level overview
of the KSV algorithm. In Section \ref{subsec:ppe} the most important
and most resource-intensive step in PK-KSV and PK-KSVe.
Section \ref{subsec:ksv-classical} describes the classical post-processing
which completes phase estimation. While this is done on a classical computer,
we will discuss how this step affects our parameters and resources
for generating a quantum circuit. We also discuss a new, faster
phase estimation algorithm \cite{Svore2013}. Finally, we map the QFS
stage of PK-KSV,
PK-KSVe, and PK-Jones to \textsf{2D CCNTCM} and compare their
asymptotic resource usage in Section \ref{subsec:ksv-compare}.

%%%%%%%%%%%%%%%%%%%%%%%%%%%%%%%%%%%%%%%%%%%%%%%%%%%%%%%%%%%%%%%%%%%%%%%%%%%%%%
\subsection{The Optimization of Early Measurement}
\label{subsec:ksv-diff}

In this section, we describe the differences between PK-KSV, as originally
presented in Section 13 of \cite{Kitaev2002} and PK-KSVe, which includes our
novel contribution of early measurement.

Given a circuit $C$ to compile, we
precompile it into gates from $\mathcal{G} \cup \{R_Z(2\pi x / 2^n)\}$
using the results from Sections \ref{subsec:qcompile-single} and
\ref{subsec:qcompile-multi} in $O(1)$ time, depth, and size.
Now we are done with the single-qubit gates and CNOT, and we have computed
the values $\{x_1, \ldots , x_m\}$ that allow us to approximate our
desired $m$
$R_Z(\phi)$ gates as $\phi \approx \frac{\pi l}{2^n}$ to within precision
$2^{-n}$. We now need to generate the QFS $\ket{\psi^{(k)}}$ to use with
phase kickback to either approximate $R_Z(\phi)$ directly, or to generate
PAR qubits as intermediaries for later enacting $R_Z(\phi)$ probabilistically.

The original PK-KSV procedure first created the state $\ket{\psi^{(0)}}$
and applied phase estimation to it to enact the operator $\Upsilon(e^{-2\pi i / 2^n})$
and get $\ket{\psi^{(1)}}$ as a result. This state can be copied using
$W$: multiple addition modulo $2^n$. However, it involves using coherent measurement
in order to enact a phase $\ket{x} \rightarrow e^{2\pi i x / 2^n}\ket{x}$ on
all the components of $\ket{\psi^{(0)}} = \sum_{x = 0}^{2^n} \ket{x}$.

The PK-KSVe does not enact the operator $\Upsilon(e^{-2\pi i / 2^n})$ but instead
yields a QFS $\ket{\psi{(k)}}$ for random odd $k$, as well as the index $k$ itself.
This can be used to solve the modular inverse equation classically:

\begin{equation}
kp \equiv x \bmod 2^n
\end{equation}

to get the solution $p = p(k,x)$.
 This integer 
$p$ is added to $\ket{\psi^{(k)}}$, controlled on some qubit $\ket{phi}$
to get the desired phase shift: $R_Z(2\pi x / 2^n)\ket{\phi}$.
To copy $\ket{\psi^{(k)}}$, consider the controlled addition operator $\Upsilon(A)_n$
on two $n$-qubit registers.

\begin{equation}
\Upsilon(A)_n \ket{x}\ket{y} \rightarrow \ket{x}\ket{y + x \bmod 2^n}
\end{equation}

Applying this operator to two QFS's has the following interesting property
as noted in \cite{Jones2013}.

\begin{equation}
\Upsilon(A)_n \ket{\psi^{(k')}}\ket{\psi^{(k)}} \rightarrow \ket{\psi^{(k'-k)}}\ket{\psi^{(k)}}
\end{equation}

Therefore, we can copy a state $\ket{\psi^{(k)}}$ in the second register by
putting $\ket{\psi^{(0)}}$ in the first register to get $\ket{\psi^{(-k)}} = \ket{\psi^{(2^n - k)}}$.
This is not exactly the same state as $\ket{\psi^{(k)}}$, but it can similarly be used
with solving modular inverse to get an $x'$ and enact a desired rotation $R_Z(2\pi x' / 2^n)$.

%%%%%%%%%%%%%%%%%%%%%%%%%%%%%%%%%%%%%%%%%%%%%%%%%%%%%%%%%%%%%%%%%%%%%%%%%%%%%%
\subsection{PK-KSVe Steps}
\label{subsec:ksv-steps}

Here now are the main steps of PK-KSVe.

The first stage involves QFS generation.

\begin{enumerate}
\item Create the equal superposition of all $\ket{\psi^{(k)}}$ for odd $k$:
\begin{equation}
\ket{nu} = \normtwo \left( \ket{2^{n-1}} + \ket{0} \right)
\end{equation}
in register 1.
\item Use phase estimation with error $\epsilon = 2^{-3n}$ and early
measurement to get classical Bernoulli series outcomes $y$ in register 2
and a corresponding QFS $\ket{\psi^{(k)}}$. This takes a circuit of
depth $O(\log n)$ and size $O(n^2)$ (see Section \ref{subsec:ppe}).
\item Perform classical post-processing from Section \ref{subsec:ksv-classical} to
recover the phase $k$ by which can identify the state $\ket{\psi^{(k)}}$.
\end{enumerate}

Now we have $\ket{\psi^{(k)}}$, or a state that has high overlap with it
(calculated in Section \ref{subsec:ksv-compare}).
We can copy it $m$ times using multiple addition $W$ in depth
$O(m \log n)$ and size $O(mn)$. This is part of any complete
quantum compiler solution for PK-KSVe, but we omit it here and
focus on QFS generation.

%%%%%%%%%%%%%%%%%%%%%%%%%%%%%%%%%%%%%%%%%%%%%%%%%%%%%%%%%%%%%%%%%%%%%%%%%%%%%%
\subsection{Parallelized Phase Estimation}
\label{subsec:ppe}

Parallelized phase estimation is the central component of both
PK-KSV and PK-KSVe which distinguishes it from all other quantum
compiling approaches. It is used to randomly ``pick'' an
eigenstate $\ket{\psi_k}$ of a unitary operator $U$ and
measure its corresponding eigenvalue (phase) $\phi_k$
with some degree of precision $\delta = 2^{-n}$ and
error probability $\epsilon = 2^{-l}$. As $n$ increases, the phases
generally become closer together, which is why we need exponential precision
to distinguish between them.

\begin{displaymath}
U\ket{\psi_k} = e^{2\pi i \phi_k} \ket{\psi_k}
\end{displaymath}

Phase estimation holds some superposition of eigenstates
$\sum_{i} \alpha_i \ket{\psi_i}$
in an $n$-qubit target register, to which it applies repeated measuring
operators $\Lambda(U^{2^k})$
controlled on some $t$-qubit register, which holds an
approximation $\tilde{\phi}$ to the real phase $\phi$.
The unitary $U$ is applied in successive powers of two to get
power-of-two multiples of the phase for increased precision.
The error probability of approximating the phase to within a given
precision is given by the following:

\begin{displaymath}
\Pr\left[ | \phi - \tilde{\phi} | \ge \delta \right] \le \epsilon
\end{displaymath}

The parameter $t = t(\delta, \epsilon)$ encodes the dependence of the number
of $\Lambda(U)$ measuring operators as a function of our desired
$\delta$ and $\epsilon$.
It varies according to the exact phase estimation procedure
used.

The popular version of phase estimation presented in \cite{Nielsen2000},
requires $t$ repeated controlled applications of some unitary
$U$ (and its successive powers as $U^{2^k}$, $0<k<2^t$)
to a target state which holds some superposition of its eigenvectors,
controlled by $t$ bits which will hold the approximation to a corresponding
eigenvalue (phase).
This version requires applying an inverse quantum Fourier
transform (QFT).

To achieve our desired low-depth, we can ``parallelize'' the application of
$\Lambda(U)$ by interpreting the
$t = (n+2)s$ control bits as an $n$-bit number $q$ and
apply $\Upsilon(A)\ket{q}$ only once.
Recall that $A$ is the addition operator on an $n$ qubit
target register containing $\ket{\psi^{(k)}}$, so we can
only effectively add the lowest $n$ bits of $q$.
Furthermore, the eigenvalues of $A$ are rational with a fixed
denominator, $\phi_k = k / 2^n$.
To avoid the inverse QFT, which often has small rotations of the form 
$R_Z(\phi)$, 
we can do a classical post-processing step as described in Section \ref{subsec:ksv-classical}.

The main steps in parallelized phase estimation as applied to PK-KSV
are as follows. As our most important input parameter, we calculate $t$, the number
of measurements that we need (and the number of control qubits that we have).
For QFS generation, it suffices to determine the phase with resolution $\frac{1}{2^{n+2}}$,
where each bit of the phase is determined with a series of Bernoulli (biased coin flips)
of $s$ trials each. As in the original PK-KSV, we have two sets of measurements,
one to measure $\phi$ as a bias projected on the real axis of the unit circle (called
the cosine measurements), and one to perform the same measurements rotated by $\pi/2$
(called the sine measurements). These are illustrated below:

\begin{equation}
\Pr(1|k) = \frac{1 + \sin(2\pi\phi_k)}{2} \qquad \Pr(0|k) = \frac{1 + \cos(2\pi\phi_k)}{2}
\end{equation}

Therefore, we have:

\begin{equation}
t = 2(n+2)s
\end{equation}

\begin{enumerate}

\item Begin with a $t$-qubit ``phase'' register initialized to $\ket{0}^{\otimes t}$.
Recall that we have an $n$-qubit ``QFS'' register initialized to $\ket{\nu}$, the
equal superposition of $\ket{\psi^{(k)}}$ for all odd $0 < k  < 2^n$.
%\textsc{Resources} $= [0,0,0,0,0,t]$

\item Place the $t$-qubit register into an equal superposition by
applying $n$ Hadamard gates.
%\textsc{Resources} $= [0,0,0,n,1,0]$

\item Treat $t$ as $2s$ groups of bits, each encoding an $(n+2)$-bit number
which we call $\ket{q_i}$.
This does not involve any addition or other operation, it just determines
how we interpret its measurement outcome after we projectively measure it later.
%Sum them up out-of-place, retaining only the lowest $n$-bits,
%to get the superposition
%of all $n$-bit numbers, $1/(\sqrt{2^{n}}) \sum_{i=0}^{2^n-1} \ket{q_i}$.
%Call this register $\ket{q_i}$.
%\textsc{Resources} $= ADD-OUT(2s \times n)$

\item Apply the gate $\Upsilon(A)_{n+2}$ to a target ancilla register $\ket{h_i}$ controlled
on $\ket{q_i}$, which we interpret as an $(n+2)$-bit number. The register
$\ket{h_1}$ is initialized to all $\ket{0}$'s. This can be done in
parallel for all $0 \le i < 2s$ using constant-depth $3\rightarrow 2$ addition.
Now we have $2s$ quantum integers, plus
the original register $\ket{\nu}$, that we add down using modular multiple
addition in logarithmic depth, including the final addition of a CSE number
down to a conventional number using QCLA \cite{Draper2004}. We use the adders described
in Section \ref{subsec:qfs-adder} on \textsf{2D CCNTCM}, uncomputing all
other intermediate adder qubits back to $\ket{0}$
so that we are only left with $\ket{\nu}$.
%\textsc{Resources} $= ADD-IN(2 \times n)$

\item Reverse the second step by applying another $n$ Hadamards.
%\textsc{Resources} $= [0,0,0,n,1,0]$

\item Projectively measure all $t$ control
qubits in the phase register. These qubits
now contain classical $0$ or $1$ as outcomes of $(n+2)s$ Bernoulli trials.
The target $\ket{nu}$ now contains a QFS $\ket{\psi^{(k)}}$ for some
as yet unknown $k$ with high fidelity.

\item Read out these outcomes into our classical controller
and perform the post-processing in Section \ref{subsec:ksv-classical}.
We get an approximation of $\phi$ with precision $\delta$ and
success probability $1-\epsilon$.

\end{enumerate}

%%%%%%%%%%%%%%%%%%%%%%%%%%%%%%%%%%%%%%%%%%%%%%%%%%%%%%%%%%%%%%%%%%%%%%%%%%%%%%
\subsection{Classical Postprocessing}
\label{subsec:ksv-classical}

It is now the point to mention that Kitaev's phase estimation procedure
contains a post processing step which is completely classical in
character, in that they involve a measurement. If this measurement is
projective and the outcomes are completely classical, the remaining steps
can be done on our classical computer,
and the results fed back into our quantum algorithm to perform controlled
addition (phase kickback). Therefore, as long as we can perform these classical
algorithms in polynomial time (which we can), we don't really care
about the equivalent circuit size and depth.

The steps of classical postprocessing, which will determine some of the
parameters in the earlier, quantum part of phase estimation are as follows.

\begin{enumerate}

\item
Estimate the phase and its power-of-two multiples
$2^j \phi_k$ to
some constant, modest precision $\delta''$, where
$0 \le j < (n+2)$. For each $j$, we
apply a series of $s$ measuring operators targeting the state $\ket{\nu}$
controlled on $s$ qubits in the state $(\ket{0}+\ket{1})/\sqrt{2}$,
essentially encoding the $2^j \phi_k$ as a bias in a coin, and flipping the
coin $s$ times in a Bernoulli trial, counting the number of $1$ outcomes,
and using that fraction to approximate the real $2^j \phi_k$.
\item
Sharpen our estimate to exponential precision $\frac{1}{2^{n+2}}$ using the
$(n+2)$ estimates, each for different bits in the binary expansion of
$\phi_k$. Multiplication by successive powers-of-two shift these bits
up to a fixed position behind the zero in a binary fraction representation,
where we can use a finite-automata and a constant number of
bits to refine our $O(n)$-length running approximation.
\end{enumerate}

Three things are worth mentioning about the interrelation of the parameters
between these two steps. Since our phases all have a denominator of $2^n$,
there is no need to run the continued fractions algorithm on multiple
convergents, as is the case with period-finding in Shor's factoring algorithm.
Furthermore, the phases are $1/2^n$ apart, therefore it suffices to approximate
the phases to within $1/2^{n+2}$ in order to break ties, which is where
our range for $j$ comes from above.

The number of trials $s$ comes from the Chernoff bound:

\begin{displaymath}
\Pr \left[ | s^{-1}\sum_{r=1}^s v_r - p_* | \ge \delta'' \right]
\le 2e^{-2\delta'^2 s}
\end{displaymath}

Setting this equal to the desired error probability $\epsilon$ we get

\begin{displaymath}
s = \frac{1}{2\delta''^2}\ln \frac{1}{\epsilon}
\end{displaymath}

We are actually estimating the values $\cos(2\pi \cdot 2^j \phi_k)$ and
$\sin(2\pi \cdot 2^j \phi_k)$, so if we wish to know $2^j \phi_k$ with
precision $\delta''$, we actually need to determine the $\cos(\cdot)$ and
$\sin(\cdot)$ values with a different precision $\delta'$, lower-bounding
it with the steepest part of the cosine and sine curves.

\begin{displaymath}
\delta' = 1 + cos(\pi - \delta'')
\end{displaymath}

It is possible to improve this bound by adding a bias angle of $\pi/4$
before measurement, and then subtracting it from the recovered phase
angle after classical post-processing. This bias angle is
inspired by the SHF phase estimation \cite{Svore2013}. Then we have
$\delta'' \approx \delta'$.

The factor $\frac{1}{2\delta''^2}$ depends on the constant precision with
which we determine our $2^j \phi_k$ values. Since classical time is
cheap and quantum gates are expensive, it makes sense to minimize the number
of trials $s$. Table \ref{tab:ksv-parameters} shows the corresponding values of $1/(2\delta''^2)$
and $\delta'$ as a function of various choices for $\delta'$.

\begin{table}
\centerline{
\begin{tabular}{|c|c|c|c|}
\hline
$\delta''$ & $1/(2\delta''^2)$ & $\delta'$ & $1/(2\delta'^2)$ \\
\hline
$1/16$     & $128$             & $0.0019525$ & $131,160$\\
$1/8$      & $32$              & $0.0078023$ & $  8,213$\\
$1/4$      & $8$               & $0.0310880$ & $    517$\\
\hline
\end{tabular}
}
\caption{Parameters for the number of measurements in PK-KSV.}
\label{tab:ksv-parameters}
\end{table}

By making our $\delta'$
exponentially worse (doubling it) we are only increasing the range of
$j$ a linear amount (by one). In general, for $\delta'=\frac{1}{2^l}$, we get
a final estimate for $\phi = 2^{m-3}$

Projective measurements are irreversible, and it is not so important that
we are left with (classical, unentangled) garbage in our $t$-qubit ancillae register.
After all, we only run phase estimation once
to get our initial $\ket{\psi^{(k)}}$ state.
The reason why
the authors of \cite{Kitaev2002} go to some care to show that all the classical
postprocessing steps can be done in polynomial-size and logarithmic-depth
is that these must be done to a quantum state $\ket{\psi^{(0)}}$ in order
to turn it into $\ket{\psi^{(1)}}$ in original PK-KSV. However, our new procedure
sidesteps this requirement, so we are free to offload this processing to a
classical controller, which is available in \textsf{2D CCNTCM}.

Since we have seen KSV-style phase estimation in
Chapter \ref{chap:factor-polylog}, it is important to mention here several
differences between applying phase estimation to factoring versus
applying it to QFS generation. The first difference is that a constant
success probability of $\frac{3}{4}$ is no longer good enough. We would
like to drive down our error exponentially low as $\epsilon = 2^{-l}$.
Second, the operator whose phase we are estimating is now
addition modulo $2^n$, which is more efficient in circuit resources
than modular multiplication (see Section \ref{subsec:qfs-adder}).

In the procedure above, we measured power-of-two multiples of the phase
$2^{j}\phi_k$ to recover single bits of $\phi_k$ at a time.
However, a remarkable recent result by Svore-Hasting-Freedman (SHF) shows
how to use information theoretic and signal processing techniques to
improve the number of measurements needed from $O(n^2)$ as in conventional
KSV to $O(n\log^{*}n)$.
Their key technique is to select random measurements of this form:

\begin{equation}
(2^{j_1}+2^{j_2}+\ldots + 2^{j_S})\phi_k
\end{equation}

This would allow us to recover multiple bits at a time.
SHF empirically determines that the number of trials $s$ in
each Bernoulli series scales as $O(\log n)$ for $n = {1000,10000}$,
on the same order of magnitude for running factoring on
keys of 2048 to 4096 bits.
Furthermore, they extend this approach to multiple rounds. Given
that the number of measurements in the final round is
$s = O(\log n)$ and that each round requires logarithmically
fewer measurements than the previous round, they
achieve the iterated-logarithmic depth above.
These techniques could be used to improve PK-KSV, but we
remain conservative for now until numerical upper bounds can be
calculated.

%%%%%%%%%%%%%%%%%%%%%%%%%%%%%%%%%%%%%%%%%%%%%%%%%%%%%%%%%%%%%%%%%%%%%%%%%%%%%%%%%%%%%
\subsection{Resource Comparisons}
\label{subsec:ksv-compare}

We now map both the KSV and Jones algorithms for QFS generation
to \textsf{2D CCNTCM} and compare their resources. We do not consider
the phase kickback part of the quantum compiler, since this controlled addition
is the same for both QFS algorithms. Like PK-Jones \cite{Jones2012}, we
set our error as $\epsilon = \left(\frac{\pi}{2^n}\right)^2$. It is not clear why
this error was chosen, since for factoring or other polynomial-sized circuits,
an inverse-polynomial error precision is sufficient. However, for the
purpose of comparison, we set the same error.

The condition for error for PK-Jones is one minus the overlap between a
pure Fourier state $\ket{\psi}^{(1)}_n$ and a distilled
Fourier state $\ket{\tilde{\psi}}^{(1,m)}_n$ after $m$ rounds.

\begin{equation}
1 - | \braket{\tilde{\psi}^{(1,m)}_n}{\psi^{(1)}} |^2 = \epsilon \le \left( \frac{\pi}{2^n} \right)^2 
\end{equation}

The condition for error in PK-KSV is that an incorrect phase indexed by $k$ is returned,
one that does not correspond to the eigenstate $\ket{\psi}^{(1)}$. The target
register for phase estimation begins in a superposition of eigenstates
$\ket{1} = \sum_{s = 1}^{2^{n-1}} \ket{\psi^{(2s-1)}_n}$, and does not deteriorate
since none of the measuring operators mix between the orthogonal decomposition of the
QFS basis. We make use of Theorem \ref{thm:projective} in Appendix \ref{app:ksv-error},
where the measured index $y$ corresponds to a subspace $\mathcal{M}_y$ of all Bernoulli series outcomes $\Delta$ that would
correspond to the phase $k = (2s-1)$, and the state left in $\mathcal{N}$ is a (possibly impure)
QFS $\mathcal{\tilde{\psi}^{(k)}}$ which has large overlap with $\mathcal{\psi^{(k)}}$.
We take the set of $\Omega$ of subspaces to be the odd QFS indices $\{1, 3, 5, \ldots 2^n - 1\}$
of which there are $2^{n-1}$ elements. Taking our phase estimation error to be $\epsilon \left( \frac{1}{2^{3n-1}} \right)$,
we then have the following inequality.

\begin{equation}
1 - | \braket{\tilde{\psi}^{(y)}}{\psi^{(k)}} | = \sum_{s = 1}^{2^{n-1}} \sum_{y \in \Delta: f(j) \ne (2s-1)} \Pr(2s-1|y) \le |\Omega|\epsilon = \frac{1}{2^{2n}} \le \left( \frac{\pi}{2^n} \right)^2 
\end{equation}

Several differences exist in resource
calculation methods and those of PK-Jones \cite{Jones2012}. First, that
author calculates expected resources, since the distillation method is
probabilistic and involves post-selection. Our resources are worst-case,
with still a small probability of failure calculated to match the
case of PK-Jones with all successful post-selection. We also compare our
improvements with those of the original PK-KSV algorithm, for the QFS generation stage.
These are given in Table \ref{tab:ksv-resources}.

\begin{table}[hbt!]
\begin{tabular}{|c|c|c|c|}
\hline
Resource       & PK-KSVe     & PK-KSV         & PK-Jones\\
\hline
$D$            & $O(\log n)$ & $O(\log^2 n)$  & $O(\log^2 n)$ \\
$S$            & $O(n^2)$    & $O(n^2\log n)$ & $O(n \log n)$ \\
$W$            & $O(n^2)$    & $O(n^2)$       & $2n + O(1)$ \\
$\overline{D}$ & $O(\log n)$ & $O(\log^2 n)$  & $O(\log n)$ \\
$\overline{S}$ & $O(n^2)$    & $O(n^2)$       & $O(n^2)$ \\
$\overline{W}$ & $O(n)$      & $O(n)$         & $O(n)$ \\
\hline
\end{tabular}
\caption{A comparison of circuit resources for QFS generation for PK-KSVe, PK-KSV, and PK-Jones on \textsf{2D CCNTCM} for error $\epsilon \le \left( \frac{\pi}{2^n} \right)^2$.}
\label{tab:ksv-resources}
\end{table}

A useful extension of this work would be to calculate numerical upper bounds and compare the
two QFS methods for an application, such as Shor's factoring algorithm for realistic key sizes of
2048-4096 bits.

\section{Single-Qubit Rotations for Quantum Majority Gate}
\label{sec:qcompile-maj}

We now present a result delayed from Chapter \ref{chap:factor-sublog},
the quantum compiling procedure which completes the quantum majority gate.
Recall from that chapter that within each quantum majority gate, we
needed to implement single-qubit rotations of the form
$2\pi / k$, where $k = poly(n)$, where $n$ is the input size of the
number for factoring. We can augment any quantum
majority circuit with quantum compiler modules that produce
rotated ancillae of the form $\normtwo(\ket{0} + e^{i\phi}\ket{1})$.

To maintain our sub-logarithmic depth, we choose the KSV method for
generating quantum Fourier states and combine it with phase kickback
(PK-KSV).
The precision for quantum compiling is $k = 2^{-\eta}$, and the
resulting resources for applying PK-KSV on \textsf{2D CCNTCM}
in this case are
given in Table \ref{tab:pk-ksv-resources}, based on calculations
from Section \ref{sec:qcompile-ksv}. Note that these resources
are deterministic, since they represent the worst case for any
single rotation $R_Z(e^{i\phi})$. To convert from $\eta$ to $n$,
we use the relationship $\eta = O(\log k) = O(\log n)$.
These resources are for the generation of a single QFS, which is
reusable for creating PAR qubits one-at-a-time. To create
$m = poly(n)$ $R_Z(e^{i\phi})$ rotations for an entire
quantum majority circuit, we would need to copy the first
QFS $m$ times using the $W$ operator from Lemma \ref{lem:w}.
This would take depth $O(\log m) = O(\log n)$,
which is not sub-logarithmic. Therefore, it is more
depth-efficient to create a new QFS from scratch for each
PAR qubit needed, in parallel.

We note that it is possible to modify PK-Jones to have
similar sub-logarithmic depth for QFS generation if a
constant-depth adder were used.

\begin{table}[hbt!]
\begin{center}
\begin{tabular}{|c|c|}
\hline
$D'$ & $O((\log \log n)^2)$ \\
$S'$ & $O(\log^2 n)$ \\
$W'$ & $O(\log^2 n)$ \\
\hline
\end{tabular}
\caption{Quantum compiling resources for PK-KSV for quantum majority gates
in factoring an $n$-bit number.}
\label{tab:pk-ksv-resources}
\end{center}
\end{table}

Given such a factory for producing such ancillae, we can compile the
rotations $R_Z(2\pi / m)$, for $m = poly(n)$ as in Theorem \ref{thm:maj-gate},
by directly using the KSV compiler with parameters $\eta = \log n$ as in the
previous table.

We now present a more general result for
producing rotations which are integer multiples of $2\pi / m$ 
using a \emph{finite}
basis in constant depth and polynomial size.
This is not necessary for our sub-logarithmic factoring implementation,
since we can always produce our desired angles of $2\pi / m$ directly
with our quantum compiler.
In fact, no polynomial-size quantum circuit will require more than
polynomial precision for compiling single-qubit rotations. However, we
present our result in the hopes that it will be useful for other
quantum algorithms, perhaps one where we must produce the
PAR qubit $\normtwo ( \ket{0} + e^{2\pi / m}\ket{1} )$ ``offline'' and then produce
the rotation $R_Z(2\pi k / m)$ ``online.''

\begin{theorem}{\textbf{Compiling a single-qubit rotation over a non-fixed, finite basis.}}
The single-qubit rotation $R_Z(2\pi k /m)$, where $m = poly(n)$,
$k \in \mathbb{Z}_m$,
can be implemented in expected depth $O(1)$ and expected size and width $O(k)$ on
\textsf{2D CCNTCM} over the finite
(but not fixed) basis $\mathcal{G} \cup \{R_Z(2\pi / m)\}$.
\label{thm:qcompile}
\end{theorem}

\begin{proof}
We use the quantum parallelism method of Hoyer-Spalek \cite{Hoyer2002},
which relies on quantum fanout and unfanout on \textsf{2D CCNTCM}.
Our use of the basis $\mathcal{G} \cup \{R_Z(2\pi / m)\}$ implies that
we have access to quantum compiler modules for producing the
PAR qubits $\normtwo ( \ket{0} + e^{2\pi / m}\ket{1} )$. Teleport $O(k)$ such qubits
into our current circuit.
Our desired rotation of $R_Z(2\pi k / m)$ on a target qubit $\ket{\psi}$
can be produced as $k$
parallel applications of $R_Z(2\pi / m)$, which are already diagonal in
the same (computational) basis. Fan out the qubit $\ket{\psi}$ $k$ times,
apply the rotations $R_Z(2\pi /m)$ to each fanned out qubit in parallel
using the PAR procedure, then unfanout the
qubits.
This requires expected $O(k)$ PAR qubits.
\end{proof}

We note here two possible conjectures for improving the above result.
The first would allow us to achieve $O(\log m)$
expected size, expected width, and
expected number of teleported PAR qubits. The second would allow us to
compile arbitrary rotations to a basis that is both fixed and finite
in constant depth.

\begin{conjecture}{\textbf{Logarithmic Reduction of Compiling Circuit Size and Width.}}
Theorem \ref{thm:qcompile} can be accomplished in expected size
and width $O(\log n)$ rather than $O(n)$.
\end{conjecture}

The size and width of the above circuit depend on whatever additional,
finite, set of 
gates
$\{ R_Z(\phi_{k_i}) \}$ used to augment the usual \textsf{2D CCNTCM} basis
$\mathcal{G}$. Let $\phi_{k_i} = 2\pi k_i / m$, then the size and width of
a circuit applying \emph{only} $R_Z(\phi_{k_i})$
are proportional to the order of $k_i$ in
$\mathbb{Z}_m$, or equivalently, the number of times we must apply
the rotation $\phi_{k_i}$ in parallel to equal the desired rotation
$\phi_k$. Suppose we are able to find a Chinese Remainder number system
for $m$, that is, a set of pairwise coprime numbers $\{m_1, \ldots m_{t}\}$
such that $m = \prod_{i=1}^t m_i $, where $t$ and the number of bits
needed to encode each $m_i$ are $O(m)$ \cite{Yeh1996}.
The Chinese Remainder representation of $k$
is the set of $(\log_2 m)$-bit numbers
$x_i = k \bmod m_i$. 
Then we conjecture that
the finite basis $\mathcal{G} \cup \{R_Z(2\pi x_i / m\}$ satisfies the
properties above.

\begin{conjecture}{\textbf{Constant Reduction of Compiling Depth.}}
Theorem \ref{thm:qcompile} can be accomplished using a fixed,
finite basis in constant depth.
\end{conjecture}

The above bases are finite but still depend on the problem input
size $n$. It may be possible to find a basis that
is both fixed and finite that would allow for compiling
arbitrary single-qubit rotations in constant depth and polynomial
size and width, still to precision $1 / poly(n)$. This fixed
basis would be ``polynomially universal'' in that it would be
the same for all inputs of any size.
We conjecture that products of single-qubit gates in
$\mathcal{G}$ which, when diagonalized, represent $R_Z$ rotations
of irrational multiples of $\pi$, would form such a basis.

\section{Conclusion}
\label{sec:qcompile-conclude}

In this chapter, we examined quantum compiling as a necessary
puzzle piece to complete our previous factoring architectures
as well as
an active field of research in its own right. This chapter
combined a pedagogical review at the beginning of quantum
compiling in general and phase kickback quantum compiling
using quantum Fourier states (QFS) in particular. It then ended
with new, more depth-efficient results in QFS generation.

Expanding beyond single-qubit gates and circuit bases, we
studied the main themes of quantum compiler research,
including multi-qubit compiling, exact versus approximative,
deterministic versus randomized, and provably-efficient versus
conjectured-efficient. We provided a wide-ranging resource
comparison
of the current state-of-the-art in single-qubit
compiling. We then explicated in detail QFS generation
using the Kitaev-Shen-Vyalyi (KSV) method of parallelized
phase estimation with classical postprocessing. We called
this method PK-KSV.

In our main contribution, we presented
an optimized version of PK-KSV using early measurement
which we called PK-KSVe. We contributed depth-efficient
adders on \textsf{2D CCNTCM} for any phase kickback
compiler in Lemma \ref{lem:w}. Then we compared the resources required
for three different phase kick compilers,
PK-KSV, PK-KSVe, and a method by Jones \cite{Jones2013}
which we call PK-Jones. Our optimized approach
PK-KSVe has the lowest depth of these approaches.
Finally, we contribute a method for completing
our factoring architecture from the previous chapter
while maintaining sub-logarithmic depth in
Theorem \ref{thm:qcompile}.

This chapter has also raised interesting open question for
future research.
Among the quantum compiling themes of Section \ref{sec:qcompile-bg},
we could form a
fifth axis which is whether the compiled circuit obeys (hybrid) nearest-neighbor
constraints or not. Such compilers would be judged based on how well
they partitioned an input circuit to have an optimal
number of modules on \textsf{2D CCNTCM} to also minimize module depth and
module size (inter-module teleportations).
When comparing existing related works in Section \ref{sec:qcompile-review},
we did not measure classical space requirements, although these may be
exponential. This would be a useful metric for future comparison.
To extend
the comparison of PK-KSV(e) and PK-Jones in Section \ref{sec:qcompile-ksv},
one could calculate numerical upper bounds
for performing QFT rotations sufficient for
Shor's factoring algorithm on realistic key sizes of
2048-4096 bits. Finally, in Section \ref{sec:qcompile-maj},
we presented two conjectures which have deep implications for
depth-efficient fault-tolerance.