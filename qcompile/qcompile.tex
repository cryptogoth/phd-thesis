\chapter{Quantum Compiling}
\label{chap:qcompile}

Quantum compiling is the approximation of an $n$-qubit
unitary operation using a fixed, finite, universal set
of simpler gates. Like quantum architecture, quantum
compiling plays an intermediate role between quantum
algorithms (which determine the high-level gates in a circuit),
and quantum error correction (which determines the
fault-tolerant gate set).
Quantum compiling helps mitigate one source of error,
the difference $\epsilon$ between a desired gate and its
finite approximation,
and consumes its own circuit resources, usually measured in
$(1 / \epsilon)$.
Therefore, it is important to study efficient quantum 
compiling so that we don't lose any quantum algorithmic speedups.
Most interestingly to this dissertation, quantum compiling itself
is an algorithm and can be mapped to a low-depth, nearest-neighbor
architecture. That is the subject of this chapter.

In Section \ref{sec:qcompile-bg}, we build upon the background
of Section \ref{sec:intro-basis} to rigorously define the problem of
quantum compiling and define useful notation and circuit resources,
building upon the primer on circuit bases in Section \ref{sec:intro-basis}.
We also discuss
variations and subtasks of quantum compiling that are common themes
in the literature, as well as their inter-relationships.
These themes include exact synthesis versus approximation,
single-qubit gates versus multi-qubit gates, and deterministic
versus randomized compiling procedures.

In Section \ref{sec:qcompile-sk}, we review the foundational
result in this field, the Solovay-Kitaev algorithm, and how it
continues to shape quantum compiler research. We also discuss
lower bounds on any SK-style approach to quantum compiling.
Building upon this,
in Section \ref{sec:qcompile-review}, we review the large
amount of recent literature on single-qubit quantum compiling with
constant width. We present
the cross-cutting themes of this research and provide an
at-a-glance comparison of all known single-qubit quantum compiling methods
to date in Section \ref{sec:qcompile-compare}.

As an alternative method of single-qubit quantum compiling,
in Section \ref{sec:qcompile-qfs}, we
discuss a popular method of trading low depth for increased
width: combining phase kickback with a quantum Fourier state.

In Section \ref{sec:qcompile-ksv}, we contribute an improved
algorithm for generating quantum Fourier states based on
Kitaev-Shen-Vyalyi \cite{Kitaev2002}. We also provide the calculation of
parameters necessary for any practical implementation as well as
the particular circuit resources consumed on \textsf{2D CCNTCM}.
We then compare the KSV approach to a recent alternative
by Jones which distills quantum Fourier states recursively \cite{Jones2013}.
Finally, in Section \ref{sec:qcompile-maj}, we contribute a method for
quantum compiling the single-qubit rotations needed for a
sublogarithmic depth quantum majority gate from Chapter \ref{chap:factor-sublog}.

The last two sections represent the main contributions of this chapter.

%Finally, in Section \ref{sec:qcompile-conclude}, we conclude by summarizing
%the overall themes of quantum compiling and presenting interesting
%directions for future research.

%\section{Quantum Gates and Circuit Bases}
\label{sec:qcompile-basis}

To compile, or implement, arbitrary quantum algorithms, we must construct circuits
which draw gates from a universal set, which we call a \emph{circuit basis},
or just \emph{basis}.
This should not be confused with a basis for a vector space.
Therefore, we will now review quantum gates, how to combine them into
circuits, and what it means to be universal.

A quantum gate on $n$-qubits is a $2^n \times 2^n$ unitary matrix
(an element of $U(2^n)$). Often, we find it useful to neglect a
global phase, since these cannot be measured in quantum mechanics.
However, a global
phase on a particular system $S$ may result in a measurable relative phase
in a large system $S'$ of which $S$ is a subsystem. Therefore, for our
purposes we will only distinguish between $U(2^n)$ and
$SU(2^n) = U(2^n) \\ U(1)$ in the few cases where it matters for
quantum compiling. As in the discussion of classical circuits from
Section \ref{sec:fsl-circuits}, the distinction between a quantum circuit
and a quantum gate is relative; often we consider a quantum gate as a
fundamental primitive of our physical technology, and a circuit as a
composite of these gates corresponding to a quantum algorithm.

In Section \ref{subsec:pauli} we
will review the Pauli single-qubit gates and their corresponding group.
In Section \ref{subsec:clifford} we will introduce the Clifford group.
In Section \ref{subsec:controlled} we will introduce controlled operations
and the Toffoli gate.
In Section \ref{subsec:qcompile-single} we will discuss \emph{single-qubit compiling}
and how the structure of a
general single-qubit gate and how it can be compiled into simpler classes
of single-qubit gates, namely rotations about Bloch sphere axes.
In Section \ref{subsec:distance} we will present distance metrics to
measure the quality of our single-qubit (and later multi-qubit) approximations.
In Section \ref{subsec:circuit-basis} we will finally
define what it means to be a universal gate set, or a circuit basis.

%%%%%%%%%%%%%%%%%%%%%%%%%%%%%%%%%%%%%%%%%%%%%%%%%%%%%%%%%%%%%%%%%%%%%%%%%%%%%%
\subsection{Pauli Group}
\label{subsec:pauli}

We review here the Pauli group on one qubit, $\mathcal{P}_1 = \{I, X, Y, Z\}$.
These represent
rotations of $\pi$ on Bloch sphere about the $x$-axis, $y$-axis, and $z$-axis,
using the homomorphism between $SU(2)$ and $SO(3)$, as well as the
identity matrix $I$. The group $\mathcal{P}_1$ also serves as a vector
basis for generating elements of $U(2)$.

\begin{equation}
X = \sigma_x
 \left[
  \begin{array}{cc}
    0 & 1 \\
    1 & 0 \\
  \end{array} \right]
\qquad
Y = \sigma_z =
 \left[
  \begin{array}{cc}
    0 & i \\
   -i & 0 \\
  \end{array} \right]
\qquad
Z = \sigma_z =
 \left[
  \begin{array}{cc}
    1 & 0 \\
    0 & -1 \\
  \end{array} \right]
\qquad
I = \sigma_0 =
 \left[
  \begin{array}{cc}
    1 & 0 \\
    0 & 1 \\
  \end{array} \right]
\end{equation}

We define the Pauli group $\mathcal{P}_n$ on $n$ qubits as the set of
all $n$-qubit operators which are tensor products of elements from
$\mathcal{P}_1$.

%%%%%%%%%%%%%%%%%%%%%%%%%%%%%%%%%%%%%%%%%%%%%%%%%%%%%%%%%%%%%%%%%%%%%%%%%%%%%%
\subsection{The Clifford Group}
\label{subsec:clifford}

We define the normalizer of $\mathcal{P}_n$ as the
Clifford group $\mathcal{C}_n$ on $n$ qubits.

\begin{equation}
\mathcal{C}_n = \{ C \in U(n) | CPC^{\dagger} \in \mathcal{P}_n \forall P \in \mathcal{P}_n \}
\end{equation}

Of particular interest to us is the two-qubit Clifford group $\mathcal{C}_2$,
which is generated by the following matrices (and their adjoints):

\begin{equation}
\mathcal{C}_2 = = \langle H, S, CNOT \rangle
\end{equation}

The first two Clifford generator matrices are single-qubit gates ($2 \times 2$ unitary matrices) and
their inclusion means they can be applied on either the first or the second
qubit.
The matrix $H$ is known as the Hadamard gate, and it is a special case of the
general Walsh-Hadamard transform. It is its own adjoint: $H^{\dagger} = H$.
The matrix $S$ is known as the phase gate, and it can be considered the
``square root'' of the Pauli $Z$ gate (up to a phase): $S^2 = Z$.
Equivalently, it can be viewed as a $pi/2$ rotation about the Bloch sphere
$z$-axis, and its adjoint $S^{\dagger}$ is the reverse ($-\pi /2$) rotation.
These matrices are defined below.

\begin{displaymath}
H = \normtwo
 \left[
  \begin{array}{cc}
    1 & 1 \\
    1 & -1 \\
  \end{array} \right]
\qquad
S = 
 \left[
  \begin{array}{cc}
    1 & 0 \\
    0 & i \\
  \end{array} \right]
\qquad
S^{\dagger} = 
 \left[
  \begin{array}{cc}
    1 & 0 \\
    0 & -i \\
  \end{array} \right]
\end{displaymath}

The Hadamard matrix also has the special property that it changes between the
$X$ basis and the $Z$ basis, that is, the vector basis for single-qubit
states consisting of eigenstates of the Pauli $X$ and Pauli $Z$ gates,
respectively. In fact, using the identities $X = HZH$ and $S^2 = Z$, it
is easy to see why $X$ and $Z$ are often listed as generators of the
Clifford group as well.

The last Clifford generator matrix is a two-qubit gate (a $4 \times 4$ unitary matrix) which
also represents a \emph{controlled} operation. That is, based on the
$\ket{1}$ component of the \emph{control} qubit, it applies a single-qubit
gate (in this case, Pauli $X$) to the \emph{target} qubit.
In fact,
both $CNOT$ and $X$ are also fundamental gates in classical reversible
logic as well, where $X$ is also the Boolean $NOT$ gate on classical bits.
That is why the gate is called $CNOT$, for ``controlled-NOT.'' Its inclusion
in the generating set for $\mathcal{C}_2$ means that it can be applied
in either direction: with control on qubit 1 and target on qubit 2 or
vice versa. $CNOT$ is defined below.

\begin{displaymath}
CNOT = 
 \left[
  \begin{array}{cccc}
    1 & 0 & 0 & 0 \\
    0 & 1 & 0 & 0 \\
    0 & 0 & 0 & 1 \\
    0 & 0 & 1 & 0
  \end{array} \right]
\end{displaymath}

Likewise, general $\mathcal{C}_n$ can be generated from the same set
as $\mathcal{C}_2$, with gates understood to apply to any of the $n$ qubits.
The gate $CNOT$ has historical importance in quantum computing partly
due to its use in many
early quantum gate decompositions \cite{Barenco1995a} and its ability to
be performed fault-tolerantly in many physical technologies.
Therefore, we narrow our basis of interest from $\mathcal{Q}$ to a new
basis which contains only a single two-qubit gate, $CNOT$, along with
arbitrary single-qubit gates. As we will discuss later, this simplification
will not lose us any computing power.

\begin{equation}
\mathcal{Q}' = \{ U(2) \cup CNOT \}
\end{equation}

Therefore, we will continue to use $CNOT$ as our primary two-qubit gate.

%%%%%%%%%%%%%%%%%%%%%%%%%%%%%%%%%%%%%%%%%%%%%%%%%%%%%%%%%%%%%%%%%%%%%%%%%%%%%%
\subsection{Controlled Gates}
\label{subsec:controlled}

The principle of a controlled gate can be generalized to multiple
controls using the ``meta-operator'' notation from \cite{Kitaev2002}:
By $\Lambda^n(U)$, we mean an $(n+1)$-qubit gate ($2^{n+1} \times 2^{n+1}$
unitary matrix) with $n$ control qubits and a single-qubit target gate
$U \in U(2)$. An important multiply-controlled gate, which is universal
for classical reversible circuits, is the Toffoli gate, or controlled-controlled-$NOT$.

\begin{equation}
Toffoli = \Lambda^2(X) = 
 \left[
  \begin{array}{cccccccc}
    1 & 0 & 0 & 0 & 0 & 0 & 0 & 0 \\
    0 & 1 & 0 & 0 & 0 & 0 & 0 & 0 \\
    0 & 0 & 1 & 0 & 0 & 0 & 0 & 0 \\
    0 & 0 & 0 & 1 & 0 & 0 & 0 & 0 \\
    0 & 0 & 0 & 0 & 1 & 0 & 0 & 0 \\
    0 & 0 & 0 & 0 & 0 & 1 & 0 & 0 \\
    0 & 0 & 0 & 0 & 0 & 0 & 0 & 1 \\
    0 & 0 & 0 & 0 & 0 & 0 & 1 & 0
  \end{array} \right]
\end{equation}

As seen above, multiply-controlled single-qubit gates $\Lambda^n(U)$ have a
special, sparser structure than general $n$-qubit gates in $U(2^n)$. At the
same time, it is not known how to implement them generally on physical systems
which have nearest-neighbor constraints. Combining these two facts, we can
use a common heuristic for the general task of quantum compiling:
(a) first reducing them to $\Lambda^{n-1}(U)$ gates, and then (b) compiling
the $\Lambda^{n-1}(U)$ gates to $mathcal{Q}'$.
Task (a) is discussed in Section \ref{subsec:qcompile-multi}.
We will not discuss task (b) any further except for the special case of
singly-controlled $\Lambda(U)$ gates below. We refer the
interested reader to Rosenbaum \cite{Rosenbaum2012} who has shown
optimal circuits for $\Lambda^n(U)$ gates over the basis $\mathcal{Q}'$.

There is also a special case of a ``targetless'' controlled single-qubit
gate which simply rotates the $\ket{1}$ component of a single-qubit state.

\begin{equation}
\Lambda(e^{i\phi}) = 
 \left[
  \begin{array}{cc}
    1 & 0 \\
    0 & e^{i\phi} \\
  \end{array} \right]
\end{equation}

We can now approach the topic of single-qubit compiling.

%%%%%%%%%%%%%%%%%%%%%%%%%%%%%%%%%%%%%%%%%%%%%%%%%%%%%%%%%%%%%%%%%%%%%%%%%%%%%%
\subsection{Single-Qubit Compiling}
\label{subsec:qcompile-single}

A seemingly simpler task than general compiling is single-qubit compiling.
This will illustrate the basic principles of quantum compiling and the
structure that we will exploit later to choose an effective basis. Moreover,
it will reveal a general relationship between many of the single-qubit
gates that we have already introduced.

First, we review how to decompose a general $U \in U(2)$ into three single-qubit
rotations about the Bloch sphere $x$-axis and $z$-axis, the so-called
Euler angle decompositions \cite{Nielsen2000}. This is a common method used
in much of the literature, which is why you will often see a factor of $3$
in resource calculations. These works are first decomposing a general
$U(2)$ matrix into three rotations about fixed axes.

\begin{equation*}
U = e^{i\delta}R_Z(\gamma)R_X(\beta)R_Z(\alpha)
\end{equation*}

The gate $R_Z(\phi)$ represents a rotation about the Bloch sphere $z$-axis,
of which the Pauli $Z$ gate is a special case of a $\pi$ rotation. In fact,
it is the same as the controlled-phase gate we introduced in the previous section.

\begin{equation}
R_Z(\phi) = \Lambda(e^{i\phi}) =
\left[
  \begin{array}{cc}
    1 & 0 \\
    0 & e^{i\phi} \\
  \end{array} \right]
=
\left[
  \begin{array}{cc}
    e^{-i\phi/2) & 0 \\
    0 & e^{i\phi/2} \\
  \end{array} \right]
\end{equation}

We can now state the relationship between $S$ and $Z$, as well as introduce
an important new gate $T$ which is the square root of $S$ up to a phase. All three
are rotations about the Bloch $z$-axis by power-of-two fractions of $\pi$.

\begin{equation}
Z = R_Z(\pi) =
\left[
  \begin{array}{cc}
    1 & 0 \\
    0 & -1 \\
  \end{array} \right]
\qquad
S = R_Z(\pi/2) =
\left[
  \begin{array}{cc}
    1 & 0 \\
    0 & i \\
  \end{array} \right]
\qquad
T = R_Z(\pi/4) =
\left[
  \begin{array}{cc}
    1 & 0 \\
    0 & e^{i\pi / 4} \\
  \end{array} \right]
\end{equation}

Likewise, the gate $R_X(\phi)$ represents a rotation about the Bloch sphere $x$-axis,
of which the Pauli $X$ gate is a special case of a $\pi$ rotation.

\begin{equation}
R_X(\phi) =
\left[
  \begin{array}{cc}
    \cos \phi & -i \sin \phi \\
    -i \sin \phi & \cos \phi \\
  \end{array} \right]
\end{equation}

Similar decompositions can be given in terms of $R_X$ and $R_Y$, or in
terms of $R_Y$ and $R_Z$. Solving for the angles $\{ \alpha \beta, \gamma, \delta \}$
involves writing four equations in four variables. We will not say
more about their solution here, except that we can implement the
global phase shift $e^{i\delta}$ using the identities below, which are
adapted from \cite{Kitaev2002}.

\begin{eqnarray}
e^{i\delta} & = & R_Z(\phi)X R_Z(\phi) X \\
X & = & R_X(\pi) \\
Z & = & R_Z(\pi) \\
R_X(\phi) & = & H R_Z(\phi) H
\end{eqnarray}

It now seems that a reasonable basis for single-qubit compiling are
arbitrary $R_Z(\phi)$ and $R_X(\phi)$ rotations. However, we will see later
that there are physical problems with this basis, and therefore we must
consider $R_Z(\phi)$ and $R_X(\phi)$ are useful intermediate circuits,
to be themselves compiled down into even simpler gates. This is the
compilation process which dominates the literature review of 
Section \ref{sec:qcompile-review}.

%%%%%%%%%%%%%%%%%%%%%%%%%%%%%%%%%%%%%%%%%%%%%%%%%%%%%%%%%%%%%%%%%%%%%%%%%%%%%%
\subsection{Distance Metrics}
\label{subsec:distance}

Suppose that we are not able to implement the rotations $R_Z(\phi)$
and $R_X(\phi)$ from the previous section exactly, or moreover, that we
wish to choose some other basis. We can choose to approximate a
$U(2)$ matrix, or in general a $U(2^n)$ matrix, by combining multiple
gates from a particular basis and then measuring how different it is
from our desired target gate.

Each gate from our basis is a unitary matrix, and the action of an entire
$n$-qubit compiled circuit $\tilde{C}$
is also a $2^n \times 2^n$ unitary matrix. This matrix can be formed
by the product of $2^n \times 2^n$ simpler matrices $G_i$ with a certain
tensor product structure. In fact, the $G_i$'s are the timesteps from
our \textsf{2D CCNTCM} model defined in Section \ref{sec:fpl-arch}, and
are formed from the tensor product of all single-qubit and two-qubit gates
which execute concurrently. Our desired target matrix $C$
is itself
a matrix from $U(2^n)$, and therefore we will need a distance metric
that operates on matrices (specifically, differences of matrices).

\begin{equation}
\mathcal{C} = G_{D}G_{D-1}\cdots G_{2} G_{1}
\end{equation}

One distance metric used in theoretical literature
is the operator norm of a matrix $M$, $|| M || = || M ||_{\infty}$,
is defined as the maximum amount it scales the vector norm
of all unit-length vectors. This is sometimes also called the
infinity-norm, or supremum-norm (sup-norm).

\begin{equation}
|| M ||_{\infty} = \max_{|| \ket{v} || = 1} || M \ket{v} ||
\end{equation}

However, this is not an operational definition.
Moreover, we often wish to neglect a global phase in a unitary matrix,
which is not measurable in quantum physics. This is equivalent to
defining the set of valid $n$-qubit quantum gates as the
group $SU(2^n) = U(2^n) \ U(1)$. To measure phase-independent
distance between two unitary matrices, we can use the following
distance measure due to Fowler \cite{Fowler2011}.

\begin{equation}
dist(U, V) = ||U - V|| = \sqrt{\frac{2^n - |tr(U^{\dag}V)|}{2^n}}
\end{equation}

Now can quantify the quality of our approximations through an
error $\epsilon$.

\begin{equation}
|| C - \tilde{C} || = || C -  G_D\cdots G_1 || < \epsilon
\end{equation}

Often $\epsilon$ will be quite small, and we will be interested
in expressing it as a negative power-of-two. Therefore, we define a
new parameter which is the increasing number of bits needed to encode the exponent
of this increasingly small fraction.

\begin{equation}
\epsilon = \frac{1}{2^n} \qquad
n = \log(1/\epsilon)
\end{equation}

It is natural to suppose that compiling better approximations requires
more resources, and these resources are expressed as functions of these
parameters $\epsilon$ and $n$. In fact, often the efficiency and the
capabilities of a quantum
compiler depend on its basis, which comes next to conclude our discussion
of quantum gates.

%%%%%%%%%%%%%%%%%%%%%%%%%%%%%%%%%%%%%%%%%%%%%%%%%%%%%%%%%%%%%%%%%%%%%%%%%%%%%%
\subsection{Circuit Bases}
\label{subsec:qcompile-bases}

\begin{definition}{Circuit basis}
A basis for a quantum circuit is a universal set of
bounded-qubit (usually $3$-qubit) gates.
We call a basis \emph{finite} if it contains a finite
number of gates; that is, it contains discrete gates and not an infinite
continuum of gates. We call a basis \emph{fixed} if its members are independent
of any input size.
\end{definition}

For fault-tolerant quantum computing, we are interested in compiling
circuits to a fixed, finite basis. What does it mean for a fixed, finite
basis to be universal for an infinite group like $SU(2^n)$?

\begin{definition}{Universal approximation}
We call a fixed, finite set of gate $\mathcal{G}$ \emph{universal} for
a group $G$ iff for every desired target $C \in G$ and
desired error $\epsilon$, we can return a
sequence of gates $(g_1,g_2,\ldots,g_S)$ from $\mathcal{G}$ where

\begin{equation}
|| C - g_1 g_2 \cdots g_S || \le \epsilon
\end{equation}

\end{definition}

This defines whether a gateset is a basis, or whether universal approximation
is even possible (non-constructively). We will see in later sections that
quantum compilers are concerned with constructive approaches to
\emph{efficiently} return such a compiled sequence $\tilde{C} = \prod g_i$

What gatesets are known to be fixed, finite, and universal, and therefore
suitable bases for quantum compilation?

We recall one such basis from our definition of \textsf{2D CCNTCM} in
Section \ref{sec:fpl-arch}, with the exception of the non-unitary operation
$MeasureZ$. However, $MeasureZ$ is implicitly assumed as part of any basis,
and is the means by which we can offload postprocessing to a classical
controller.

\begin{equation}
\mathcal{G} = \{X, Z, H, Toffoli, CNOT\}
\end{equation}

It is important to note that the only non-Clifford gate in the above basis
is $Toffoli$.
The Clifford group $\mathcal{C}_n$ by itself is \emph{not}
universal.
This is unfortunate given that many quantum error-correcting codes have
efficient implementations for Clifford gates. In fact, it is provable
that \emph{any} universal gateset must possess at least one
non-Clifford gate \cite{Zheng2011}.

Two popular choices for the non-Clifford gate in a basis are the $T = R_Z(\pi/4)$
gate and the $Toffoli = \Lambda^2(X)$ gate. Since these are not ``natively''
supported (non-transveral) in many codes, they must often be implemented
probabilistically using only Clifford operations and $MeasureZ$, usually
by way of a so-called ``magic'' state.
Therefore, many quantum compilers
use the Clifford+$T$ basis or the Clifford+$Toffoli$ basis, and measure
the non-Clifford gate as the most expensive resource. It is an area of
active research
whether $T$ or $Toffoli$ is more efficient to implement
\cite{Jones2012a,Eastin2012}.

For single-qubit compilation, the $\{H,T\}$ gateset is universal and
plays an important role in the literature. Other compilers may add
the Clifford gates $S$ and $S^{\dagger}$ to the basis $\mathcal{G}$ above,
but this does not change its universality nor its asymptotic efficiency for
compiling.

We are not prepared to discuss resources for measuring the efficiency of
a quantum compiler.

\section{Quantum Compiling Background}
\label{sec:qcompile-bg}

Quantum compiling as a general task of approximating $SU(2^n)$ matrices
using 

The image to keep in mind throughout this entire section is shown in
Figure \ref{fig:qcompile}.

\begin{figure}
\begin{center}
\begin{displaymath}
\begin{array}{ccc}

%%%%%%%%%%%%%%%%
% Source circuit
%\underbrace{
\begin{array}{c}
S = 2 \\
\Qcircuit @C=0.5em @R=.5em { 
	& \multigate{4}{U_1} & \qw & \multigate{4}{U_2} & \qw \\ 
	& \ghost{U_1}        & \qw & \ghost{U_2}        & \qw \\
	& \ghost{U_1}        & \qw & \ghost{U_2}        & \qw \\
	& \ghost{U_1}        & \qw & \ghost{U_2}        & \qw \\
	& \ghost{U_1}        & \qw & \ghost{U_2}        & \qw 
	\gategroup{1}{2}{5}{4}{.7em}{--}
}\\
\xymatrix {
  & D=2 \ar[l] \ar[r] & \\
 }
\end{array}
%}_{C}

%& 
%\begin{array}{c}
%\textsc{Quantum Compiler} \\
\rightarrow
%\end{array}
%&

%%%%%%%%%%%%%%%%
% Target circuit
%\underbrace{
\begin{array}{c}
S' = 15 \\
\Qcircuit @C=0.5em @R=.5em { 
	& \gate{H} & \qw & \ctrl{1} & \gate{H} & \qw & \qw      & \ctrl{1} & \qw \\ 
	& \gate{H} & \qw & \targfix & \ctrl{2} & \qw & \gate{K} & \targfix & \qw \\
	& \gate{H} & \qw & \gate{K} & \qw      & \qw & \gate{H} & \qw      & \qw \\
	& \gate{H} & \qw & \ctrl{1} & \targfix & \qw & \gate{H} & \qw      & \qw \\
	& \gate{H} & \qw & \targfix & \gate{H} & \qw & \qw      & \qw      & \qw
	\gategroup{1}{2}{5}{9}{.7em}{--}
}\\
\xymatrix {
  & & D'=5 \ar[ll] \ar[rr] & & \\
 }
\end{array}
%}_{C}

\end{array}
\end{displaymath}

\caption{Quantum circuit synthesis into single-qubit gates and $CNOT$.}
\label{fig:qcompile}
\end{center}
\end{figure}

can be subdivided into more special-purpose tasks along several axes,
which are cross-cutting themes in any literature review of quantum compilers.
These themes also provide a context for understanding resource consumption
for quantum compiling, which we define in Section \ref{subsec:qcompile-resources}.

These axes are:

\begin{enumerate}
\item single-qubit compiling versus multi-qubit compiling
\item exact synthesis versus approximative quantum compiling
\item deterministic versus probabilistic quantum compiling
\item compilers with provable upper bounds versus conjectured upper bounds
\end{enumerate}

The first axis is 
single-qubit compiling
(mentioned previously in Section \ref{subsec:qcompile-single}) versus
multi-qubit compiling. Some algorithms which work on single-qubit compiling
can be generalized directly to the multi-qubit case. In fact, all known
examples of these generalized algorithms can accept an arbitrary circuit
basis $\mathcal{B}$ \cite{Amy2012,Solovay1995,Fowler2011,Booth2012}.
That is, they do not exploit any special structure of
a particular basis. The circuit basis is another input to the algorithm,
possibly to an additional classical preprocessing step. Whether the algorithm
is a single-qubit or a multi-qubit algorithm depends on whether the basis
is single-qubit or multi-qubit.

There is an intermediate point on this axis, between single-qubit and multi-qubit,
which is the reduction of a multi-qubit circuit into a basis of
single-qubit and two-qubit gates. This task is often called \emph{quantum circuit synthesis},
and we will discuss it in Section \ref{subsec:qcompile-multi}.

The second axis is compiling a circuit exactly or approximately.
Exact synthesis refers to the case of determining whether a
target circuit $C$ is implementable from a basis $\mathcal{B}$
with no error ($\epsilon = 0$). If this is possible, a quantum compiler
should return the sequence of gates which constitute the exact
synthesis. Furthermore, exact synthesizers often have a goal of
returning the \emph{optimal} sequence of compiled gates, that is,
one with minimal length $\ell$.
Exact synthesizers often enumerate over all circuits of
a certain length from a certain basis $\mathcal{B}$. Therefore, their
resources are upper bounded by a brute force search, which takes
time upper-bounded by $|\mathcal{B}|^{\ell}$.
Approximative quantum compiling conforms to our usual notion where
$\epsilon > 0$, and achieving smaller error costs more resources. Many
exact synthesis algorithms can be used to build basic approximations
for the Solovay-Kitaev algorithm more efficiently, and therefore help
achieve better approximative upper bounds as verified by numerical
simulation over random unitaries.

The third axis is whether a quantum compiling algorithm uses randomness
or is completely deterministic. For known randomized algorithms, it is
an open problem whether the algorithm can be derandomized or not
\cite{Kliuchnikov2012}, and numerical verification is necessary to
show the desired distribution of running times.

The fourth axis is whether a quantum compiler has upper bounds
(usually on running time or compiled sequence length) that are provable or
based on a conjecture. Both deterministic and randomized
algorithms can have provable upper bounds, although
in the latter case, one calculates the average-case and upper bounds the
variance. Likewise, both deterministic and randomized algorithms can
be based on a conjecture.

These four axes can be used to classify quantum compilers, although some
algorithms can be placed in multiple categories. For example, many
single-qubit quantum compilers which perform exact synthesis can be
incorporated into a hybrid algorithm which then performs
approximation. And of course, some single-qubit quantum compilers can be generalized
into multi-qubit algorithms.

A fifth axis could be formed, which is whether the compiled circuit requires
arbitrarily long interactions for $CNOT$ or is nearest-neighbor. Such a
quantum compiler could also divide up a compiled circuit into an optimal
number of modules on \textsf{2D CCNTCM} to also minimize module depth and
module size (inter-module teleportations). This is an interesting direction
for future research.

%%%%%%%%%%%%%%%%%%%%%%%%%%%%%%%%%%%%%%%%%%%%%%%%%%%%%%%%%%%%%%%%%%%%%%%%%%%%%%
\subsection{Quantum Compiler Resources}

Just as a quantum algorithm with arbitrary long-range interactions incurs
some overhead in being mapped to a nearest-neighbor architecture,
a quantum compiler itself is an algorithm. It always has a classical
component, which runs on a digital computer, and transforms a classical
description of an input quantum circuit into an output circuit from
a basis $\mathcal{B}$. The compiled output circuit then runs on a
quantum computer. In general, the compiled output circuit $\tilde{C}$ consumes
resources which are greater than those of the input circuit $C$.

\begin{description}
\item[classical runtime $R$:] the classical time it takes to return a 
compiled quantum circuit.
\item[input depth $D$:] the depth of the input quantum circuit in arbitrary
$n$-qubit gates.
\item[input size $S$:] the size of the input quantum circuit in arbitrary
$n$-qubit gates.
\item[input width $W$:] the width of the input quantum circuit in qubits.
\item[compiled depth $D'$:] the compiled quantum circuit depth, equal to
the compiled sequence length for single-qubit circuits.
\item[compiled size $S$:] the compiled quantum circuit size, which is
identical to compiled depth if no ancillae are used (compiled width is zero).
\item[compiled width $W$:] the compiled quantum circuit width, which includes
the width of the input circuit as well as any ancillae introduced by
the compiler.
\end{description}

All but the first resource are quantum in nature, and follow the definitions
for circuit resources from Chapter \ref{chap:factor-polylog}. Because
compilation incurs some overhead, we have $D' \ge D$, $S' \ge S$, and
$W' \ge W$.

It's also known that
in order to approximate a circuit with $S$ gates to a total precision of
$\epsilon$
requires each gate to be approximated to a precision of
$n = O(\log(S/\epsilon)$ \cite{Lloyd1995}. We denote this per-gate precision
$n$, since it serves as an independent parameter for compiling. For
single-qubit gates, $S = 1$, and this corresponds exactly with our previous
definition for $n$ in Section \ref{subsec:qcompile-basis}.

We do not measure classical space requirements, although these may be
exponential. This would be a useful metric for comparison for future work.

%%%%%%%%%%%%%%%%%%%%%%%%%%%%%%%%%%%%%%%%%%%%%%%%%%%%%%%%%%%%%%%%%%%%%%%%%%%%%%
\subsection{Decomposition to Single-Qubit and $CNOT$}
\label{subsec:qcompile-multi}

Restricting ourselves to the simplest case of
single-qubit circuits allows us to exploit a lot of structure
in the group $U(2)$ (or its related subgroups $SU(2)$ and $PSU(2)$).
From a volume argument, we can derive a general
lower bound for the efficiency of the multi-qubit case \cite{Harrow2003},
as well as determine how our compiling efficiency scales with dimensionality.
Any
SK-style algorithm produces worst-case sequence lengths $\ell_d$ that
are longer than worst-case single-qubit sequence lengths $\ell_1$ by a certain multiplicative
prefactor. This prefactor has a dependence that is at least
polynomial in $d = 2^n$. 

\begin{equation}
% TODO fact check this!
\ell_d / \ell_1 = \Omega \left( \frac{d^2 - 1}{ \log |\mathcal{B}| } \right )
\end{equation}

efficient. This is an example of task modularity which allows us
to divide the effort of quantum compiling between the
single-qubit case and then decomposition to single-qubit gates and
$CNOT$. It is a heuristic which often results
in simple decompositions to implement in (classical) software.
It may not be asymptotically optimal compared to generic 
multi-qubit protocols. However, for small input sizes, it beats the




Now that we have handled the single-qubit case, how can we leverage this
to compile general $n$-qubit gates? We need a reduction to the basis
$\mathcal{Q} = U(2) \cup \{ CNOT \}$, as originally depicted in
Figure \ref{fig:qcompile}.
It turns out that almost any two-qubit gate plus arbitrary single-qubit
rotations are universal \cite{Bremner2002}. However, we will stick with CNOT
due to its other useful properties.
an $SU(2^n)$ matrix to 
The usual reduction uses
a two-level decomposition, as shown in several places.
We need to fact-check this.
\cite{Kitaev2002}

The optimal bound needed for this in terms of $CNOT$ gates in
the compiled output (the dominant cost) is still exponential
$O(4^n)$ \cite{Shende2004}.

\begin{table}[hbt!]
\begin{tabular}{|c|c|}
\hline
BBC+ \cite{Barenco1995a} & \\
Svore-Aho \cite{Svore2003} & o(1.17\cdot 4^n - 3.51\cdot 4^n + 3.34) \\
Shende \cite{Shende2004a} & o(0.48\cdot 4^n - 1.50\cdot 2^n + 1.34) \\
Shende \cite{Shende2004} & \omega(0.25\cdot 4^n - 3n - 1 \\
\hline
\end{tabular}
\label{tab:multi}
\caption{}
\end{table}

\section{The Solovay-Kitaev Algorithm}
\label{sec:qcompile-sk}

As one of the central results of quantum computation, it is worth
reviewing here the venerable Solovay-Kitaev (SK) algorithm for the
approximation of $SU(d)$ gates by a fixed, finite basis $\mathcal{B}$.
Although many recent results have surpassed SK in terms of efficiency,
they often do so by improving the base-level approximation of the
original SK algorithm. Moreover, many techniques for analyzing
and understanding quantum compilers were developed for SK, which
continues to be the best way to learn them. Finally, the overall
struction of the original SK algorithm is so simple, it is surprising
that it works so well. Therefore, it is worth
our time to review it here.

The essential structure of SK is to recursively generate nets of unitary
operators with
successively finer precision. At any given level of recursion, the input 
gate is divided up into two halves which are further from identity [is this
really true] and can be
approximated with less precision. Our pseudo-code and explanation
follows the exposition in \cite{Dawson2005} and \cite{Harrow2001}

\begin{algorithmic}[1]
\STATE \textsc{function} $\tilde{U}_i \leftarrow$ SK$(U,n)$
\IF{$i == 0$}
\STATE $\tilde{U}_i \leftarrow $ BASIC-APPROX$(U)$
\ELSE
\STATE $\tilde{U}_{i-1} \leftarrow$ SK$(U, i-1)$
\STATE $A,B \leftarrow $ FACTOR$(U\tilde{U}^\dagger_{i-1})$
\STATE $\tilde{A}_{i-1} \leftarrow $ SK$(A, i-1)$
\STATE $\tilde{B}_{i-1} \leftarrow $ SK$(B, i-1)$
\STATE $\tilde{U}_i \leftarrow \tilde{A}_{i-1}\tilde{B}_{i-1}\tilde{A}^\dagger_{i-1}\tilde{B}^\dagger_{i-1}\tilde{U}_{i-1}$
\ENDIF
\STATE return $\tilde{U}_i$
\end{algorithmic}

Since the recursion must eventually bottom out, we must precompute some sequences
of gates from $\mathcal{G}$ up to length $l_0$. This is the classical
preprocessing step which requires upfront storage space for this
coarsest-grained net, where
each sequence is no more than $\epsilon_0$ from its nearest neighbor. According
to \cite{Dawson2005} the values of $l_0=16$ and $\epsilon_0 = 0.14$ using
operator norm distance is sufficient for most applications.
This step can be
done offline and reused across multiple runs of the compiler, assuming
$\mathcal{G}$ for your quantum computer doesn't change.

The BASIC-APPROX function above does a lookup (e.g. using some kd-tree search
maneuvers through higher-dimensional vector spaces) using this $\epsilon_0$-net,
and all higher recursive calls to SK are effectively constructing
finer $\epsilon$-nets on the fly as needed.

The FACTOR function performs a balanced group commutator decomposition,
$U = ABA^\dagger B^\dagger$, and then recursively approximates the $A$ and $B$
operators, again using SK. We denote by $\tilde{U}_i$ as the approximation
of $U$ using $i$ levels of SK recursion. Intuitively, when they are multiplied
together again, along with their inverses, their errors (which go like
$\epsilon$) are symmetric and cancel out in such a way that the resulting
product $U$ has errors which go like $\epsilon^2$. In this manner, we can
eventually sharpen our desired error down to any value. We use the
geometric decomposition in \cite{Dawson2005} rather than the decomposition
based on a BCH-like approximation as in \cite{Harrow2001}, although it is not
known which method converges more quickly to a desired gate in general.

%%%%%%%%%%%%%%%%%%%%%%%%%%%%%%%%%%%%%%%%%%%%%%%%%%%%%%%%%%%%%%%%%%%%%%%%%%%%%%%
%%%%%%%%%%%%%%%%%%%%%%%%%%%%%%%%%%%%%%%%%%%%%%%%%%%%%%%%%%%%%%%%%%%%%%%%%%%%%%%
% we still need to reconcile this 5^n with log^{3.97}(1/\epsilon)
% this is an evernote item, maybe it is already reconciled

Each level $i$ of recursion solves the problem of compiling 
$U_i$ by combining five gates compiled at the lower $i-1$ level.
Therefore, the compiled circuit size at the top-level is upper-bounded by $5^n$,
and
this is in general the same as the circuit depth (since not all gates in
$\mathcal{G}$ commute).

%%%%%%%%%%%%%%%%%%%%%%%%%%%%%%%%%%%%%%%%%%%%%%%%%%%%%%%%%%%%%%%%%%%%%%%%%%%%%%%
%%%%%%%%%%%%%%%%%%%%%%%%%%%%%%%%%%%%%%%%%%%%%%%%%%%%%%%%%%%%%%%%%%%%%%%%%%%%%%%
% we still need a discussion somewhere of the relationship between commuting
% operators and parallelizing them into the same timestep

\section{Quantum Compiler Review}
\label{sec:qcompile-review}

%%%%%%%%%%%%%%%%%%%%%%%%%%%%%%%%%%%%%%%%%%%%%%%%%%%%%%%%%%%%%%%%%%%%%%%%%%%%%%
\subsection{Quantum Compiling Before 2012}
\label{subsec:qcompiile-pre2012}

The first known quantum compiling result by Lloyd showed that almost any two
distinct single-qubit gates were universal for single-qubit compiling
\cite{Lloyd1995}. However, the runtime and length of a compiled sequence of gates
could take exponentially long in the desired error precision: $O(1/\epsilon)$
The first \emph{efficient} quantum compiler due to
Solovay-Kitaev (SK) as implemented by Dawson-Nielsen had an original depth overhead
of $O(\log^{2.71}(1/\epsilon))$ and classical
running time of $O(\log^{3.97}(1/\epsilon))$. This is a seminal result in
the field of quantum computation, and as such it was reviewed in the previous
section (\ref{sec:qcompile-sk}). We will abbreviate it afterwards as SK-DN.
It was later improved by Kitaev-Shen-Vyalyi
to have a longer running time but smaller depth overhead of $O(\log^{3+o(1)}(n/\epsilon))$
\cite{Kitaev2002}, an algorithm we will call SK-KSV.
This improvement was achieved by repeated rounds of sparsifying and
telescoping finer nets based on coarser nets. Both SK-DN and SK-KSV can be
generalized to $n$-qubit compiling, but in the latter case the multiplicative
prefactor has a doubly exponentially dependence of $exp(2^n)$. Suggestions
for improving this to be $poly(2^n)$ are provided in the text, but this remains
an interesting open problem.

In the same text \cite{Kitaev2002},
a more parallel compiling procedure was discovered by Kitaev-Shen-
Vyalyi with a depth overhead of $D' = O(D\log(1/\epsilon) + (\log\log(1/\epsilon))^2)$ but
a size overhead of $S' = O(S\log(1/epsilon) + (\log(1/\epsilon))^2 (\log\log(1/\epsilon))$
and a width overhead of $W' = O(W(\log(1/\epsilon))^2$
on \textsf{AC}.
This involves the generation of an $n$-qubit quantum Fourier state into a register
and the controlled-addition
modulo $2^n$ into this register. The first step is done using phase estimation
and an approximative inverse QFT using classical post-processing.
This controlled-addition step is known as
\emph{phase kickback} in recent literature \cite{Jones2012}.

Therefore, we
refer to this algorithm as PK-KSV.
Moreover, the KSV procedure assumes a basis of
relies on a prior decomposition into
single-qubit gates
a unitary matrix in $SU(N=2^n)$ into (controlled) single-qubit gates,
An equivalent, alternative proof is given by Cleve-Watrous \cite{Cleve2000}.
PK-KSV was the first quantum compiler to make a time-space (circuit depth
versus circuit width) tradeoff,
using ancillae to decrease depth. Therefore, it has non-trivial circuit space
and circuit width, and it is the first compiler where architectural concerns
matter.
We will calculate its resources when mapped to
\textsf{2D CCNTCM} in Section \ref{sec:qcompile-ksv}.

%Numerical comparisons between SK and KSV can be found in Section \ref{sec:ksv-results}.

As early as 2004, Austin Fowler pioneered the use of optimized enumeration
of gate sequences over a single-qubit basis, using efficient data structures
and heuristics to search and de-dupe (remove duplicates) a database of optimal, unique
sequences up to
a certain length \cite{Fowler2011}. He used the basis $\mathcal{C}_1 \cup \{ T, T^{\dagger} \}$
and gave efficient, fault-tolerant implementations of this basis in the
Steane code. Due to the fact that the Clifford group is closed under
matrix multiplication, every compiled Fowler sequence can be simplified as a
Clifford gate alternating with $T$. Asymptotically, this still runs in time
$|\mathcal{B}|^\ell$ to produce compiled sequences of optimal length $\ell$.
It is an exact-synthesis, deterministic, provably optimal compiler.
In its current form, it is a single-qubit compiler because it takes as input
$\mathcal{C}_1 \cup \{T\}$, but it can be generalized to $n$-qubits by enumerating
and feeding in $\mathcal{C}_n \cup \{T\}$. Unfortunately, the size of the
Clifford group increases at least exponentially in $n$, which places practical
limits on any such enumerative compiler for $n \ge 2$.

In 2007, Matsumoto-Amano used the idea of a \emph{normalized circuit} to
improve the enumeration and search approach to compiling \cite{Matsumoto2008}. Two circuits in
normalized form compute the same gate if and only if they are the same
circuit. If a normalized form can be computed efficiently, then a
database of normalized forms can be easily searched to determine the
uniqueness of a gate sequence. Their case exploits the special structure
of the single-qubit basis $\{H,T,T^{\dagger}\}$ and is an exact-synthesis,
deterministic, provably-optimal compiler over this basis. Both the Fowler
and the Matsumoto-Amano compiler can be plugged into Solovay-Kitaev to
optimize the basic (0-level) approximations. (Fact, check if anyone
has done this yet).

%%%%%%%%%%%%%%%%%%%%%%%%%%%%%%%%%%%%%%%%%%%%%%%%%%%%%%%%%%%%%%%%%%%%%%%%%%%%%%
\subsection{Quantum Compiling in 2012 and After}
\label{subsec:qcompile-post2012}

In 2012 and after, quantum compiling experienced a renewed interest in the
research community. Many different themes from the previous section were
taken up and developed. These developments are summarized below.

In 2012, two approaches improved upon the Fowler method by considering a
meet-in-the-middle approach: the work by Amy-Mosca-Maslov-Roetteler (AMMR)
\cite{Amy2012} and that by Booth (\cite{Booth2012}.
The Fowler approach searches for an
optimal path from identity to the desired target gate using an optimal
length of $\ell$. The meet-in-the-middle approach divides this into two tasks of
reaching a middle point, searching simultaneously from identity and the
target gate, to find optimal paths of length $\lceil \ell/2 \rceil$
and $\lfloor \ell/2 \rfloor$. This gives a quadratic speed improvement over
the Fowler approach.

Both the AMMR and Booth compilers can be generalized to
multi-qubit circuits by feeding in a multi-qubit basis. In fact, the AMMR compiler
was able to generate the two-qubit Clifford group with
size $|\mathcal{C}_2| = 11,520$ and the three-qubit Clifford
$|\mathcal{C}_3| = 92,897,280$, and therefore it is able to optimize
the depth of $T$ gates in a circuit, in addition to optimizing the total
count of $T$ gates. However, for multi-qubit gates of $n \ge 4$,
the running time of enumerative search, even with optimizations, is
doubly exponential (exponential enumeration over an exponential-sized
set of Clifford group elements), which quickly encounters practical limits
of modern digital computers.
On a more positive note, the AMMR compiler gives the first new decomposition
of the Toffoli gate in terms of $\mathcal{C}_1 \cup \{T, T^{\dagger}\}$
since 1995 \cite{Barenco1995a}.
%Interestingly, the AMMR paper states without
%proof the
%exact equivalence of single-qubit circuits over the basis
%$\{H, T, T^{\dagger}\}$ and $U(2)$ matrices with elements from the ring $\mathbb{Z}\left[i,\frac{1}{\sqrt{2}}\right]$.
%However, this is not proven until a later paper \cite{Kliuchnikov2013}.

The Bocharov-Svore (BS) compiler combines the idea of normalizing circuits
and enumerative search to compile \emph{canonical circuits} on single-qubits.
Canonical circuits
implement the same circuit if their matrices are the same up to
conjugation by two (possibly distinct) Clifford gates. This leads to
a fourth-root speedup of search time for this exact-synthesis, single-qubit
compiler over the $\{H, T, T^{\dagger} \}$ basis. While its circuit resources
can be expressed in terms of basis size $|\mathcal{B}|$ for comparison with
the AMMR, Booth, and Fowler compilers, in fact the BS compiler is only known
to work for one particular single-qubit basis. Furthermore, the BS exact
compiler can be plugged into SK to optimize the enumeration and storage of
basic (0-level) approximations. When this is done and simulated for
random unitaries, numerical experiments reveal an improved compiled
sequence length of $O(\log^{3.4}(1/\epsilon)$. We can call this hybrid
compiler SK-BS.

Kliuchnikov-Mosca-Maslov (KMM) provide significant advances for both the
exact-synthesis and approximative approaches to compiling, which we can
abbreviate KMMe and KMMa, respectively.

For the
exact-synthesis case \cite{Kliuchnikov2012e},
KMMe provides a rigorous proof of the equivalence
of circuits over the basis $\mathcal{C}_1 \cup \{T, T^{\dagger} \}$ and $U(2)$ matrices
with elements from the ring $\mathbb{Z}\left[i,\frac{1}{\sqrt{2}}\right]$.
Reducing the problem of circuit synthesis over this basis to state
generation, with elements from $\mathbb{Z}\left[i,\frac{1}{\sqrt{2}}\right]$,
KMMe remarkably finds sequences of optimal length $\ell$ in time
polynomial in $\ell$
($O(\ell ^2)$ bit operations), and not exponential in $\ell$ as
all previous approaches. This is a fair comparison because other 
exact-synthesis algorithms above implicitly
depend on a fixed-precision floating-point
arithmetic, and KMMe makes its bit-dependence explicit. The KMMe is
a single-qubit compiler, but its generalization to higher dimensions
depends on proving the conjecture that
$U(2^n)$ matrices of elements from $\mathbb{Z}\left[i,\frac{1}{\sqrt{2}}\right]$
are equivalent to $n$-qubit circuits over the same basis.

For the approximative case \cite{Kliuchnikov2012a}, KMMa saturates the lower
bound of $\Omega(\log(1/\epsilon)$ by using two ancillae prepared in the
$\ket{00}$ state and basis of $\mathcal{C}_1 \cup \{ T \}$.
It reduces the problem of approximating a
$R_Z(\phi)$ single-qubit rotation with expressing a large integer $M$ as the
sum of four squares. There is a known probabilistic algorithm to
solve this Diophantine equation
by Rabin-Shallit \cite{Rabin1985} that takes $O(\log^2(M)\log\log M)$. Therefore,
the KMMa compiler runs in classical time $O(\log^2(1/\epsilon)\log\log(1/\epsilon))$
and returns sequences of length $O(\log(1/\epsilon)$, which is optimal up to
a constant factor.

A natural improvement is to see if the ancillae of KMMa can be done away with
completely. This question is answered in the affirmative by a single-qubit,
approximative compiler by Selinger
\cite{Selinger2012}. In that work, general unitaries are approximated up to
error $\epsilon$ via the $R_Z(\phi)$ (or equivalently, $R_X(\phi)$ rotations)
of accuracy $\epsilon/3$ using elements of the
ring $\mathcal{R}$ so that a known, optimal exact-synthesizer such as KMMe
can be used. The approximation to $R_Z(\phi)$ involves solving
Diophantine equations to find approximations to real numbers in the
ring $\mathbb{Z}\left[\sqrt{2}\right]$. Solutions to this require
random sampling of candidate solutions from $\mathcal{R}$ whose average
success probability depends on an open conjecture about the distribution
of odd primes. The running time and optimal sequence-depth of the
Selinger compiler is verified by numerical simulation.

Improving SK-style approximative and Fowler-style enumerative compilers has
dominated most of the progress in the years 2012-2013. However, we now mention
a work which uniqely improves on phase-kickback and the PK-KSV-style of
compilation that allows $O(n)$ ancillae in order to achieve $o(\log(1/\epsilon)$
depth.

Based on previous multilevel distillation protocols for magic states \cite{Jones2012},
Jones devised a remarkable way to recursively distill Fourier states using an
initial approximation of $2$-qubit states using only Clifford gates. This provides
an alternative to generating quantum Fourier states to PK-KSV to be used with
phase kickback; therefore we abbreviate this method PK-Jones. Fourier state
distillation results
in an asymptotic improvement in the number of non-Clifford (Toffoli) gates
required versus PK-KSV ($O(n\log n)$ versus $O(n^2\log n)$ for an
$n$-qubit Fourier state), while maintaining the same asymptotic depth overhead
($O(\log n)$). The resources of PK-Jones are compared with an optimized version of PK-KSV in
Section \ref{sec:qcompile-ksv}.

All of these approaches, which admit direct comparison more-or-less, are
displayed in Table \ref{tab:qcompile-compare} in Section
\ref{sec:qcompile-compare}.

%%%%%%%%%%%%%%%%%%%%%%%%%%%%%%%%%%%%%%%%%%%%%%%%%%%%%%%%%%%%%%%%%%%%%%%%%%%%%%
\subsection{Magic States for Quantum Compiling}

Instead of using non-Clifford gates to complete our basis into a universal
gate set, we can include ``non-Clifford'' states as an additional resource.
These states are called \emph{magic states} if, combined with the Clifford
gates, enable universal quantum computation.
Quantum compilers which use these magic states count them as a non-Clifford
resource instead of $T$ or Toffoli gates, since such states cannot
be produced directly using Clifford gates. Rather, they must be
distilled in a probabilistic procedure which itself uses only 
Clifford gates and measurement.

There are two types of magic states, 
The most famous such example is the $+1$ eigenstate of the Hadamard operator,
often written as:

% TODO Fact check this, what about \ket{0} + e^{i\pi/4}\ket{1}
\begin{equation}
\ket{H} = \cos(\pi/8)\ket{0} + \sin(\pi/8)\ket{1}
\end{equation}

Magic state distillation was originally proposed by Bravyi and Kitaev
\cite{Bravyi2005} as a model of universal quantum computation (UQC) which
allowed for noisy, or imperfectly prepared, initial quantum states. These
noisy states could be ``purified'' or distilled down to certain states
that are ``magic'' in that, combined with the Clifford gates, they enable
UQC. These magic states come in two types, the eigenstates
of the $H$ or another operator, which we call $A$.

\begin{equation}
H = 
\normtwo
\left[ \begin{array}{cc}
1 & 1\\
1 & -1
\end{array} \right]
\qquad
A = e^{i\pi / 4} SH = \frac{e^{i\pi/4}}{\sqrt{2}}
\left[ \begin{array}{cc}
1 & 1\\
i & -i
\end{array} \right]
\label{eqn:T}
\end{equation}

The noisy initial states can be considered to be prepared
These states are distilled via successive rounds of recursive error
correction, where the efficiency of distillation with each round,
and the resources required to do so, depend on the code used.

% TODO Insert here a comparison of different codes used
% Original Knill result
% Bravyi-Kitaev
% Bravyi-Haah

\subsection{Alternative Bases and Resources}
\label{subsec:alt-resources}

Now we turn to three recent works which use magic state distillation to compile
arbitrary single-qubit gates. These works either compile to an alternate
basis from $\mathcal{C}_n \cup \{T,T^{\dagger}\}$.

A recent approach by Bocharov-Gurevich-Svore \cite{Bocharov2013}
compiles to subsets of the Clifford group augmented with the non-Clifford
$V$-basis, first discovered by Lubotsky-Phillips-Sarnak \cite{Lubotsky1987},
which was proven to permit the lower-bound of compiled sequence length
$O(\log^1(1/\epsilon)$
\cite{Harrow2002}.

\begin{equation}
V_1 = \frac{1}{\sqrt{5}}
\left[
\begin{array}{cc}
1  & 2i \\
2i & 1
\end{array}
\right]
\qquad
V_2 = \frac{1}{\sqrt{5}}
\left[
\begin{array}{cc}
1  & 2\\
-2 & 1
\end{array}
\right]
\qquad
V_3 = \frac{1}{\sqrt{5}}
\left[
\begin{array}{cc}
1+2i & 0 \\
   0 & 1-2i
\end{array}
\right]
\end{equation}

This work uses the properties of Lipschitz quaternions with norms $5^l$, ($l \in \mathbb{Z}, l \ge 0$). It
contains a randomized algorithm whose running time is based on a conjecture from geometric number theory.
There is currently no complete, fault-tolerant method of compiling all three gates from the $V$ basis into
our usual universal set of $\mathcal{C}_1 \cup \{T\}$. However, the appendix of \cite{Bocharov2013}
gives a method for implementing the exact $V_2$ gate using the (probabilistic) magic state distillation of
Duclos-Cianci and Svore \cite{DuclosCianci2012}. The average cost in $\ket{H}$
states for enacting a $V_2$ gate is $22.75$. It is an open problem how many
$\ket{H}$ states are needed to enact the gates $V_1$ and $V_3$.
We cannot compare it directly to previous algorithms which consider the number of $T$ gates ($T_c$)
the primary resource, or the compiled sequence length $D'$ as an upper bound to $T_c$.
However, the SK algorithm allows us to provide an alternate basis. By
comparing the compiled sequence lengths in the $V$ basis and noting any
improvements over the $\mathcal{C}_1 \cup \{T\}$ basis, 
measuring the length of compiled sequences 

The previous works all assume a minimal basis which only includes a fixed
number (usually one) non-Clifford gate (a $T$ gate or Toffoli). We can
call these ``reduced instruction set computing'' (RISC) bases, in analogy to
digital processor architectures.
In contrast, Landahl and
Cesare consider a ``complex instruction set computing'' (CISC) basis
\cite{Landahl2013b}
which
can include certain non-Clifford states of the form
$\ket{0} + e^{i\pi / 2^k} \ket{1} = R_Z(\pi/{2^k})\ket{+}$. These
non-Clifford states can be distilled to a certain error $\epsilon'$ using
a probabilistic procedure using only Clifford gates similar to distilling
the state $\ket{H}$ described above. This error $\epsilon'$ is proportional
to the desired error of gate compilation: $\epsilon' \tilde \epsilon$.
Therefore, the work of Landahl-Cesare also unifies into the sum of the 
exponent of the logarithm $\alpha + \beta + \gamma$ the
normally distinct procedures of
magic state distillation ($\alpha$),
quantum compiling ($\beta$),
and error-correction ($\gamma$) into
the sum of exponents of the logarithm.

\begin{equation}
R,D = O(\log ^{\alpha + \beta + \gamma} (1/\epsilon)
\end{equation}

By optimizing over magic state distillation and quantum compiling as a
single task, Landahl-Cesare achieves $\alpha + \beta \approx 1$, when the
best that can be achieved with these two procedures separately is
$\alpha + beta \ge 2$.

%%%%%%%%%%%%%%%%%%%%%%%%%%%%%%%%%%%%%%%%%%%%%%%%%%%%%%%%%%%%%%%%
%\subsection{Generalizations to Quantum Compiling}
%\label{subsec:alt-tasks}

%The literature of quantum compiling has traditionally drawn boundaries
%around itself, restricting its study to approximation of unitary gates
%with perfect operations. However, s
%around 

\section{Quantum Compiler Comparison}
\label{sec:qcompile-compare}

\begin{landscape}

\begin{table}[hbt!]
\begin{tabular}{|c|c|c|c|c|c|c|c|c|}
\hline
Algorithm                  & A/E & D/R & P/C & Multi? & $R$                             & $D'$                    & $S'$                  & $W'$ \\
\hline
Fowler\cite{Fowler2011}     & A   & D   & P  & Y    & $O(\ell |\mathcal{B}|^{\ell})$    &                         &                       &     \\
Booth \cite{Booth2012}      & A   & D   & P  & Y    & $O(\ell |\mathcal{B}|^{\ell/2})$  &                         &                       & 1    \\
AMMR \cite{Amy2012}         & E   & D   & P  & Y    & $O(\ell |\mathcal{B}|^{\ell/2})$  &                         &                       & 1    \\
BS \cite{Bocharov2012}      & E   & D   & P  & N    & $O(\ell |\mathcal{B}|^{\ell/4})$  &                         &                       & 1    \\
%BSG \cite{Bocharov2013}     & A   & D   & N    & $O(\ell |\mathcal{B}|^{\ell/4})$ & $\ell$           & $D'$ & 1    \\
KMM-e\cite{Kliuchnikov2012} & E   & D   & P  & ?    & $\ell^2$                          &                         &                       & 1    \\
SK-DN\cite{Dawson2005}      & A   & D   & P  & Y    & $O(n^{2.71} + \mathcal{B}^{l_0})$ & $O(D n^{3.97})$         &                       & 1    \\
SK-KSV\cite{Kitaev2002}     & A   & D   & P  & N    & $O(n^{3+o(1)})$                   & $O(D n^{3+\nu})$        &                       & 1    \\
SK-BS \cite{Bocharov2012}   & A   & D   & C  & N    & $O(n^{2.71} + \mathcal{B}^{l_0})$ & $O(Dn^{3.4})$           &                       & 1    \\
KMM-a\cite{Kliuchnikov2012b}& A   & P   & C  & N    & $O(n^2\log n)$                    & $O(Dn)$                 &                       & 1    \\
Selinger\cite{Selinger2012} & A   & P   & C  & N    & $O(n^4)$                          & $D(48n + 44)$           &                       & 1    \\
PK-KSV \cite{Kitaev2002,Cleve2000} & E & D & P & N  & $O(1)$                            & $O(D\log n + \log^2 n)$ & $O(S n + n^2 \log n)$ & $O(n^2)$ \\
PK-Jones \cite{Jones2013}   & E  & D    & P  & N     & $O(1)$                           & $O(D\log n + n \log n)$ & $O(S n + n \log n)$   & $2n + O(1)$ \\
\end{tabular}
\caption{A comparison of single-qubit quantum compilers, where $n = \log_2(1/\epsilon)$, for a desired error $\epsilon$.
Blank depths $D'$ are $\ell$. Blank sizes $S'$ are equal to $D'$. Blank widths $W'$ are equal to $1$.
A/E: Approximate versus exact-synthesis. D/R: Deterministic versus randomized. P/C: provable bounds versus conjecture + numerical verification.
Multi?: Y if it can be generalized to a multi-qubit compiler.}
\label{tab:qcompile-compare}
\end{table}

\end{landscape}


\section{Phase Kickback and Quantum Fourier States}
\label{sec:qcompile-qfs}

In this section, we discuss the \emph{phase kickback} procedure for
producing a qubit with arbitrary phase: $\normtwo (\ket{0} + e^{i\phi}\ket{1} )$.
From the PAR procedure, a qubit could be prepared in such a state
``offline,'' and then teleported later into a \textsf{2D CCNTCM} module to
probabilistically enact rotations $R_Z(\phi)$ \cite{Jones2012}.

Phase kickback requires controlled addition on an $\eta$-qubit target state in 
the quantum Fourier basis for desired precision $\epsilon = 2^{-\eta}$.
In this purely pedagogical review,
we will discuss known properties of the
basis of quantum Fourier states, including their relationship with
addition modulo $2^{\eta}$.
This will motivate our
understanding of the KSV procedure for generating such quantum Fourier
states in the next two sections, where we will also present our
main novel results in this chapter.

%In Section \ref{subsec:qfs-basis},  It is a purely pedagogical review which sets
%the stage for our new results in the next two sections.
%In Section \ref{subsec:qfs-adder},
%we calculate circuit resources for performing adders
%on \textsf{2D CCNTCM}. Although the adders are generic, they will be
%used in the resource calculations of KSV phase estimation in
%Section \ref{subsec:ppe}, which operates on quantum Fourier states.

%%%%%%%%%%%%%%%%%%%%%%%%%%%%%%%%%%%%%%%%%%%%%%%%%%%%%%%%%%%%%%%%%%%%%%%%%%%%%%%
%\subsection{Properties of the Quantum Fourier Basis}
%\label{subsec:qfs-basis}

The following states are well-known as $\eta$-qubit quantum Fourier states,
the result of applying the quantum Fourier transform (QFT)
to the $\eta$-qubit computational basis state. 
%
\begin{equation}
\ket{\psi^{(k)}_{\eta}} = \frac{1}{\sqrt{2^{\eta}}} \sum_{j=0}^{2^{\eta}-1}
e^{-2\pi i j k / 2^{\eta}} \ket{j}
\end{equation}
%
These states form an alternative, orthonormal basis indexed by
$0 \le k < 2^{\eta}$. We will often just call these Fourier states,
and neglect the subscript $\eta$, which is implied.
Note that the state $\ket{\psi^{(0)}}$, sometimes
called the fundamental Fourier state, is simply the equal superposition
of all computational basis states, or the tensor product of
$\eta$ qubits in the state $\ket{+}$, or the result of applying
$\eta$ Hadamards qubit-wise to a state beginning in $\ket{0}^\eta$.

Note that these states $\ket{\psi^{(k)}_n}$ are the
QFT states of the $\eta$-qubit computational basis. The usual method of
creating these states involves performing phase estimation of the
modular addition operator. These are implicitly hard in that
all known procedures take size $O(\eta\log \eta)$, even if rotations
are truncated to approximate the QFT in depth $O(\log \eta)$
\cite{Jones2012} or in some cases $O(1)$
given unbounded quantum fanout
\cite{Browne2009}.

However, we can create a superposition in constant depth
over all odd $k=(2s-1)$
by starting in the state $\ket{0}^{\otimes \eta}$,
then applying a Hadamard and $\sigma^z$ to the most significant qubit.

\begin{equation}
\ket{\eta} = \normtwo \ket{0} - \normtwo \ket{2^{\eta-1}} =
\frac{1}{\sqrt{2^{\eta-1}}} \sum_{s=1}^{2^{\eta-1}} \ket{\psi_{\eta,2s-1}}
\end{equation}

More obviously relevant to our overall goal of approximating
$\Lambda(e^{i\phi})$, we can enact a phase
shift simply by performing the following modular addition operator, for
which $\ket{\psi_{\eta}^{(k)}}$ are eigenstates.
%
\begin{equation}
A\ket{j} \rightarrow \ket{j+1 \bmod 2^{\eta}}
\end{equation}
%
Applying this operator to its eigenstates results in a phase shift which
depends on the particular eigenstate.
%
\begin{equation}
A\ket{\psi_{\eta}^{(k)}} = e^{2\pi i \phi_k} \ket{\psi_{\eta}^{(k)}}
\end{equation}
%
Finding the eigenvalue $e^{2\pi i \phi_k}$ corresponds to finding
the phase $\phi_k = k / 2^{\eta}$.
Repeated application of $A$ (say $p$ times) would result in a phase
added to the eigenstate equal to a multiple of $e^{2\pi i p / 2^{\eta}}$:
%
\begin{equation}
A^p\ket{\psi_{\eta}^{(k)}} = e^{2\pi i \phi_k / 2^{\eta}} \ket{\psi_{\eta}^{(k)}}
\end{equation}
%
This explains why we don't find even values of $k$ interesting,
since then we would not get a
cyclic distribution of $2^{\eta}$ different phases,
since only odd $k$
are coprime with $2^{\eta}$. The exception is $k=0$, since this is the
equal superposition of computational basis states, which we can also
efficiently create as in the equation below.
This will be a useful starting point later on to
create addition eigenstates
for odd $k$.
%
\begin{equation}
\ket{\psi_{\eta}^{(0)}} = H^{\otimes \eta}\ket{0}^{\otimes \eta}
\end{equation}
%
Suppose we have a certain state $\ket{\psi_{\eta}^{(k)}}$ but we want to get enact
a phase shift $e^{2\pi i l / 2^{\eta}}$. We can do this by solving $p=p(s,l)$
in this equation:
%
\begin{equation}
(2s-1)p \equiv l (\bmod 2^{\eta})
\label{eqn:psl}
\end{equation}
%
Stipulating $k$ to be odd guarantees that there is a unique solution $p$.

We then apply $A^p$ as follows, where $\Upsilon_{\eta}(A)$ means to
apply $A$ to the second register $p$ times, where $p$ is an $\eta$-qubit
number in the first register.
%
\begin{equation}
\Upsilon_{\eta}(A) \ket{p}\ket{\psi_{\eta}^{(k)}} \rightarrow
e^{2\pi i l/2^{\eta}} \ket{p}\ket{\psi_{\eta}^{(k)}}
\label{eqn:upsilon}
\end{equation}
%
If we control the operation of $\Upsilon_n(A)$ on a source qubit $\ket{+}$,
it will acquire the phase $e^{2\pi i l/2^{\eta}}$.
%
\begin{equation}
\Lambda(\Upsilon_n(A))\ket{+}\ket{p}\ket{\psi_{\eta}^{(k)}} \rightarrow
\left( \ket{0} + e^{2\pi i l}\ket{1} \right) \ket{p}\ket{\psi_{\eta}^{(k)}}
\end{equation}
%
This is not yet the phase kickback procedure, since
we must still solve for $p$ using the equation below by finding the
modular inverse $(2s - 1)^{-1} \bmod 2^{\eta}$,
making use of the following expansion from Section 13
of \cite{Kitaev2002}.
%
\begin{equation}
p \equiv -l\sum_{j=0}^{m-1} (2s)^j \equiv -l \prod_{r=1}^{t-1}\left(1 + (2s)^{2^r}\right) \mod 2^{\eta}
\label{eqn:mod-inverse}
\end{equation}
%
where $m = O(\eta)$ is $\eta$ rounded to the nearest power of $2$.  In general,
this requires a circuit of size $O(\eta^2 \log \eta)$ and depth $O((\log \eta)^2)$ and
represents the most expensive part of the KSV procedure as originally
presented in \cite{Kitaev2002}.

Ideally, we would obviate the need for the expensive circuit above
by ensuring that $k=1$, in which case
$p = l$. We will see how to do this in Section \ref{subsec:ppe}.

Finally, to copy the state $\ket{\psi_{\eta}^{(k)}}$ it suffices to apply the following
operator which only uses subtraction (addition with one addend and the
outcome negated in two's complement representation).
%
\begin{equation}
\ket{\psi_{\eta}^{(k)}}^{\otimes m} = W^{-1}\left( \ket{\psi_{\eta}^{(0)}}^{\otimes(m-1)} \otimes \ket{\psi_{\eta}^{(k)}} \right)
\end{equation}
%
where $W$ is the operator on $m$ registers, each consisting of $\eta$ qubits,
which adds all the registers into its final register (modulo $2^{\eta}$).
%
\begin{equation}
W : \ket{x_1}\ket{x_2}\ldots\ket{x_{m-1}}\ket{x_m} \rightarrow
 \ket{x_1}\ket{x_2}\ldots\ket{x_{m-1}}\ket{x_1+\ldots+x_m \bmod 2^{\eta}}
\end{equation}
%
Having complained about the cost of modular division, how can we now
implement the operators $\Upsilon_{\eta}(A)$ and $W$ more efficiently on our
hybrid nearest-neighbor architecture?

\section{Circuit Resources for the KSV Quantum Compiler}
\label{sec:ksv-resources}

This section gives a pedagogical review of the method by
Kitaev-Shen-Vyalyi \cite{Kitaev2002} to compile single-qubit gates,
specifically those of the form $R_Z(\phi)$. This method is henceforth
called the KSV quantum compiler. Furthermore, we present an
original optimization
called \emph{early measurement} which does not significantly increase
our compilation resources (see Appendix \ref{app:ksv-error}).
Finally, we contribute the architectural resources for running the KSV
compiler on \textsf{2D CCNTCM}.

In Section \ref{subsec:ksv-steps} we present the high-level overview
of the KSV algorithm. In Section \ref{subsec:precompile} we contribute
the constant-depth precompiling step. 
The rest of this section is organized as follows.
First, Section \ref{sec:prelims} defines terms and parameters
so that we can discuss quantum compilers with some rigor as well as
giving asymptotic bounds for specific algorithms.
Then Section
\ref{sec:related} gives a brief history of quantum compiling.
The next two sections describe the two compiling algorithms and how
to measure their relative performance.
Section \ref{sec:sk-algo} reviews the original SK result and
Section \ref{sec:main-algo} describes the building blocks of KSV in detail
along with its
most resource-intensive modules. Section \ref{sec:methods} describe
our methods for the performance comparisons, which are given in Section
\ref{sec:results}. Finally, we make some comments about these results
and suggest future directions for extending this work.

%%%%%%%%%%%%%%%%%%%%%%%%%%%%%%%%%%%%%%%%%%%%%%%%%%%%%%%%%%%%%%%%%%%%%%%%%%%%%%
\subsection{KSV Steps}
\label{subsec:ksv-steps}

Given a circuit $C$ to compile,

\begin{enumerate}
\item Precompile $C$ into gates from $\mathcal{G} \cup \{\Lambda(e^{2\pi i l / 2^n})\}$
using the results from Section \ref{subsec:precompile} in $O(1)$ time, depth,
and size.
Now we are done with the single-qubit gates and CNOT, and we have computed
the values $\{l_1, \ldots , l_m\}$ that allow us to approximate our
desired $m$
$\Lambda(e^{i\phi})$ gates as $\phi \approx l/2^n$ to within precision
$2^{-n}$.
\item Create the state $\ket{\psi^{(0)}_{n,0}}$ with $n$ Hadamards.
\item Turn it into $\ket{\psi_{n,1}} = \Upsilon(e^{-2\pi i / 2^n}) \ket{\psi_{n,0}}$
using phase kickback and parallelized phase estimation in Section \ref{subsec:phase-shift}
This is done with a circuit of size $O(n^2\log n)$ and $O(\log n)$ depth.
\item Make $m$ copies of the state $\ket{\psi_{n,1}}$ out of one copy by 
applying the addition operation $A$.
\item Simulate each $\Lambda(e^{2\pi i l / 2^n})$
using one copy each of $\ket{\psi_{n,1}}$, to which we can add our
values $l$ using $\Upsilon(A)$.
This takes size $O(mn)$ and depth $O(\log n)$, since we can enact
all these phase shifts in parallel.
\end{enumerate}

Now for the resource calculations of these individual steps and their
substeps.

%%%%%%%%%%%%%%%%%%%%%%%%%%%%%%%%%%%%%%%%%%%%%%%%%%%%%%%%%%%%%%%%%%%%%%%%%%%%%%
\subsection{Parallelized Phase Estimation}
\label{subsec:ppe}

One of the key components of the registered phase shifting procedure
described in the previous section is the ability to ``pick'' a random
eigenstate $\ket{\psi_k}$ of a unitary operator $U$ and
measure its corresponding eigenvalue (phase) $\phi_k$
with some degree of precision $\delta = 2^{-n}$ and
error probability $\epsilon = 2^{-l}$. As $n$ increases, the phases
generally become closer together, which is why we need exponential precision
to distinguish between them.
Of course, this exactly describes the phase
estimation procedure, a key technique in many quantum algorithms developed
by Kitaev in his derivation of Shor's factoring result \cite{kitaev}.

\begin{displaymath}
U\ket{\psi_k} = e^{2\pi i \phi_k} \ket{\psi_k}
\end{displaymath}

Phase estimation holds some superposition of eigenstates
$\sum_{i} \alpha_i \ket{\psi_i}$
in an $n$-qubit target register, to which it applies repeated measuring
operators $\Lambda(U^{2^k})$
controlled on some $t$-qubit register, which holds an
approximation $\tilde{phi}$ to the real phase $\phi$.
The unitary $U$ is applied in successive powers of two to get
power-of-two multiples of the phase for increased precision.
The error probability of approximating the phase to within a given
precision is given by the following:

\begin{displaymath}
\Pr\left[ | \phi - \tilde{phi} | \ge \delta \right] \le \epsilon
\end{displaymath}

The parameter $t = t(\delta, \epsilon)$ encodes the dependence of the number
of $\Lambda(U)$ measuring operators as a function of our desired
$\delta$ and $\epsilon$.
It varies according to the exact phase estimation procedure
used.

The popular version of phase estimation presented in \cite[nc00],
requires $t$ repeated controlled applications of some unitary
$U$ (and its successive powers as $U^{2^k}$, $0<k<2^t$)
to a target state which holds some superposition of its eigenvectors,
controlled by $t$ bits which will hold the approximation to a corresponding
eigenvalue (phase).
This version requires applying an inverse quantum Fourier
transform (QFT), which is already high-depth and way more inefficient than our
desired quantum compiler.

To achieve our desired low-depth, we can ``parallelize'' the application of
$\Lambda(U)$ by interpreting the
$t = (n+2)s$ control bits as an $n$-bit number $q$ and
apply $\Lambda(A^q)$ only once.
In the Super-Kitaev procedure, $A$ is the addition operator on an $n$ qubit
target register containing $\psi_{n,k}$, so we can
only effectively add the lowest $n$ bits of $q$.
Furthermore, the eigenvalues of $A$ are rational with a fixed
denominator, $\phi_k = k / 2^n$.
To avoid the inverse QFT,
we can do a classical postprocessing step, which we'll mostly skip over
for fairly good reasons, and then we'll do a detailed resource count of
parallelized phase estimation.

The main steps in parallelized phase estimation as applied to Super-Kitaev
are:

\begin{enumerate}

\item Begin with a $t$-qubit ancilla register initialized to $\ket{0}^{\otimes t}$.
%\textsc{Resources} $= [0,0,0,0,0,t]$

\item Place the $t$-qubit register into an equal superposition by
applying $n$ Hadamard gates.
%\textsc{Resources} $= [0,0,0,n,1,0]$

\item Treat $t$ as $2s$ groups of bits, each encoding an $n$-bit number.
Sum them up out-of-place, retaining only the lowest $n$-bits,
to get the superposition
of all $n$-bit numbers, $1/(\sqrt{2^{n}}) \sum_{i=0}^{2^n-1} \ket{q_i}$.
Call this register $\ket{q_i}$.
%\textsc{Resources} $= ADD-OUT(2s \times n)$

\item Reverse the first step by applying another $n$ Hadamards.
%\textsc{Resources} $= [0,0,0,n,1,0]$

\item Apply the gate $\Upsilon(A)$ to the target $\ket{\nu}$ controlled
on $\ket{q_i}$, which is equivalent to adding all $q_i$ in superposition
(in place).
%\textsc{Resources} $= ADD-IN(2 \times n)$

\item Measure the register $\ket{q_i}$ in the computational basis to get 
a particular value $q$ and collapse the register to $\ket{q}$. All $t=(n-2)s$
now contain classical $0$ or $1$ as outcomes of $(n+2)s$ Bernoulli trials.

\item Read out these outcomes into our classical controller
and perform the postprocessing
described above to get an approximation of $\phi$ with precision $\delta$.

\end{enumerate}

%%%%%%%%%%%%%%%%%%%%%%%%%%%%%%%%%%%%%%%%%%%%%%%%%%%%%%%%%%%%%%%%%%%%%%%%%%%%%%
\subsection{Classical Postprocessing}

It is now the point to mention that Kitaev's phase estimation procedure
contains a post processing step which is completely classical in
character, in that they involve a measurement. If this measurement is
projective and the outcomes are completely classical, the remaining steps
can be done on our classical computer (recall our quantum coprocessor model),
and the results fed back into our quantum subroutine, registered
phase shifting. Therefore, as long as we can perform these classical
algorithms in polynomial time (which we can), we don't really care
about the equivalent circuit size and depth.

The steps of classical postprocessing, which will determine some of the
parameters in the earlier, quantum part of phase estimation are as follows.

\begin{enumerate}

\item
Estimate the phase and its power-of-two multiples
$2^j \phi_k$ to
some constant, modest precision $\delta''$, where
$0 \le j < (n+2)$. For each $j$, we
apply a series of $s$ measuring operators targeting the state $\ket{\nu}$
controlled on $s$ qubits in the state $(\ket{0}+\ket{1})/\sqrt{2}$,
essentially encoding the $2^j \phi_k$ as a bias in a coin, and flipping the
coin $s$ times in a Bernoulli trial, counting the number of $1$ outcomes,
and using that fraction to approximate the real $2^j \phi_k$.
\item
Sharpen our estimate to exponential precision $1/2^(n+2)$ using the
$(n+2)$ estimates, each for different bits in the binary expansion of
$phi_k$. Multiplication by successive powers-of-two shift these bits
up to a fixed position behind the zero in a binary fraction representation,
where we can use a finite-automata and a constant number of
bits to refine our $O(n)$-length running approximation.
\end{enumerate}

Three things are worth mentioning about the interrelation of the parameters
between these two steps. Since our phases all have a denominator of $2^n$,
there is no need to run the continued fractions algorithm on multiple
convergents, as is the case with period-finding in Shor's factoring algorithm.
Furthermore, the phases are $1/2^n$ apart, therefore it suffices to approximate
the phases to within $1/2^{n+2}$ in order to break ties, which is where
our range for $j$ comes from above.

The number of trials $s$ comes from the Chernoff bound:

\begin{displaymath}
\Pr \left[ | s^{-1}\sum_{r=1}^s v_r - p_* | \ge \delta'' \right]
\le 2e^{-2\delta'^2 s}
\end{displaymath}

Setting this equal to the desired error probabiliy $\epsilon$ we get

\begin{displaymath}
s = \frac{1}{2\delta''^2}\ln \frac{1}{\epsilon}
\end{displaymath}

We are actually estimating the values $\cos(2\pi \cdot 2^j \phi_k)$ and
$\sin(2\pi \cdot 2^j \phi_k)$, so if we wish to know $2^j \phi_k$ with
precision $\delta''$, we actually need to determine the $\cos(\cdot)$ and
$\sin(\cdot)$ values with a different precision $\delta'$, lower-bounding
it with the steepest part of the cosine and sine curves.

\begin{displaymath}
\delta' = 1 + cos(\pi - \delta'')
\end{displaymath}

The factor $\frac{1}{2\delta''^2}$ depends on the constant precision with
which we determine our $2^j \phi_k$ values. Since classical time is
cheap and quantum gates are expensive, it makes sense to minimize the number
of trials $s$. The following table shows the corresponding values of $1/(2\delta''^2)$
and $\delta'$ as a function of various choices for $\delta'$.

\begin{center}
\begin{table}
\begin{tabular}{|c|c|c|}
\hline
$\delta''$ & $\delta'$   & $1/(2\delta''2)$\\
\hline
$1/16$     & $0.0019525$ & $131,160$\\
$1/8$      & $0.0078023$ & $  8,213$\\
$1/4$      & $0.0310880$ & $    517$\\
\hline
\end{tabular}
\end{table}
\end{center}

By making our $\delta'$
exponentially worse (doubling it) we are only increasing the range of
$j$ a linear amount (by one). In general, for $\delta'=\frac{1}{2^l}$, we get
a final estimate for $\phi = 2^{m-3}$

However, projective measurements are irreversible, and we may wish to
uncompute the phase estimation procedure to restore our ancilla to
their initialized state and reuse them later on. In that case, the
postprocessing can actually be done on a quantum computer using
reversible gates and without projective measurements. That's why
the authors of \cite{ksv02} go to some care to show that all the classical
postprocessing steps can be done in polynomial-size and logarithmic-depth
circuit.
However, to simplify our analysis, we assume the case
in the previous paragraph, and accept the
loss of $n$ ancilla qubits, After all, we only run phase estimation once
to get our initial $\ket{\psi_{n,1}}$ state.

\section{Single-Qubit Rotations for Quantum Majority Gate}
\label{sec:qcompile-maj}

We now present a result delayed from Chapter \ref{chap:factor-sublog},
the quantum compiling procedure which completes the quantum majority gate.
Recall from that chapter that within each quantum majority gate, we
needed to implement single-qubit rotations of the form
$2\pi / m$, where $m = poly(n)$, where $n$ is the input size of the
number for factoring. We can augment any quantum
majority circuit with quantum compiler modules that produce
rotated ancillae of the form $\ket{0} + e^{i\phi}\ket{1}$.

To maintain our sublogarithmic depth, we choose the KSV method for
generating quantum Fourier states and combine it with phase kickback
(PK-KSV).
The precision for quantum compiling is $m = 2^{-n'}$, and the
resulting resources for applying PK-KSV on \textsf{2D CCNTCM}
in this case are
given in Table \ref{tab:pk-ksv-resources}, based on calculations
from Section \ref{sec:qcompile-ksv}. Note that these resources
are deterministic, since they represent the worst case for any
single rotation $R_Z(e^{i\phi})$. To convert from $n'$ to $n$,
we use the relationship $n' = O(\log m) = O(\log n)$.
However, we note that it is possible to modify the Jones method of
Fourier state distillation to have similar performance.

\begin{table}[hbt!]
\begin{tabular}{|c|c|}
\hline
$D'$ & $O((\log \log n)^2)$ \\
$S'$ & $O()$ \\
$W'$ & $O()$ \\
\hline
\end{tabular}
\caption{Quantum compiling resources for PK-KSV for quantum majority gates
in factoring an $n$-bit number.}
\label{tab:pk-ksv-resources}
\end{table}

Given such a factory for producing such ancillae, we can compile the
rotations $R_Z(2\pi / m)$, for $m = poly(n)$ as in Theorem \ref{thm:maj-gate},
by directly using the KSV compiler with parameters $n' = \log n$ as in the
previous table.

We now present a more general result for
producing rotations which are rational multiples of $2\pi / m$ 
using a \emph{finite}
basis in constant depth and polynomial size.
This is not necessary for our sublogarithmic factoring implementation,
since we can always produce our desired angles of $2\pi / m$ directly
with our quantum compiler.
In fact, no polynomial-size quantum circuit will require more than
polynomial precision for compiling single-qubit rotations. However, we
present our result in the hopes that it will be useful for other
quantum algorithms, perhaps one where we must produce the
PAR qubit $\ket{0} + e^{2\pi / m}\ket{1}$ ``offline'' and then produce
the rotation $R_Z(2\pi k / m)$ ``online.''

\begin{theorem}{\textbf{Compiling a single-qubit rotation over a non-fixed, finite basis.}}
The single-qubit rotation $R_Z(2\pi k /m)$, where $m = poly(n)$,
$k \in \mathbb{Z}_m$,
can be implemented in expected depth $O(1)$ and expected size and width $O(k)$ on
\textsf{2D CCNTCM} over the finite
(but not fixed) basis $\mathcal{G} \cup \{R_Z(2\pi / m)\}$.
\end{theorem}

\begin{proof}
We use the quantum parallelism method of Hoyer-Spalek \cite{Hoyer2002},
which relies on quantum fanout and unfanout on \textsf{2D CCNTCM}.
Our use of the basis $\mathcal{G} \cup \{R_Z(2\pi / m)\}$ implies that
we have access to quantum compiler modules for producing the
PAR qubits $\ket{0} + e^{2\pi / m}\ket{1}$. Teleport $O(k)$ such qubits
into our current circuit.
Our desired rotation of $R_Z(2\pi k / m)$ on a target qubit $\ket{\psi}$
can be produced as $k$
parallel applications of $R_Z(2\pi / m)$, which are already diagonal in
the same (computational) basis. Fan out the qubit $\ket{\psi}$ $k$ times,
apply the rotation $R_Z(2\pi /m)$ using the PAR procedure, then unfanout the
qubits.
This requires expected $O(k)$ PAR qubits.
\end{proof}

We note here two possible conjectures for improving the above result.
The first would allow us to achieve $O(\log m)$
expected size, expected width, and
expected number of teleported PAR qubits. The second would allow us to
compile arbitrary rotations to a basis that is both fixed and finite
in constant depth.

\begin{conjecture}{\textbf{Logarithmic Reduction of Compiling Circuit Size and Width.}}
The size and width of the above circuit depend on whatever additional,
finite, set of 
gates
$\{ R_Z(\phi_{k_i}) \}$ used to augment the usual \textsf{2D CCNTCM} basis
$\mathcal{G}$. Let $\phi_{k_i} = 2\pi k_i / m$, then the size and width of
a circuit applying \emph{only} $R_Z(\phi_{k_i})$
are proportional to the order of $k_i$ in
$\mathbb{Z}_m$, or equivalently, the number of times we must apply
the rotation $\phi_{k_i}$ in parallel to equal the desired rotation
$\phi_k$. Suppose we are able to find a Chinese Remainder number system
for $m$, that is, a set of pairwise coprime numbers $\{m_1, \ldots m_{t}\}$
such that $m = \prod_{i=1}^t m_i $, where $t$ and the number of bits
needed to encode each $m_i$ are $O(m)$ \cite{Yeh1996}.
The Chinese Remainder representation of $k$
is the set of $(\log_2 m)$-bit numbers
$x_i = k \bmod m_i$. 
Then we conjecture that
the finite basis $\mathcal{G} \cup \{R_Z(2\pi x_i / m\}$ satisfies the
properties above.
\end{conjecture}

\begin{conjecture}{\textbf{Constant Reduction of Compiling Depth.}}
The above bases are finite but still depend on the problem input
size $n$. It may be possible to find a basis that
is both fixed and finite that would allow for compiling
arbitrary single-qubit rotations in constant depth and polynomial
size and width, still to precision $1 / poly(n)$. This fixed
basis would be ``polynomially universal'' in that it would be
the same for all inputs of any size.
We conjecture that products of single-qubit gates in
$\mathcal{G}$ which, when diagonalized, represent $R_Z$ rotations
of irrational multiples of $\pi$, would form such a basis.
\end{conjecture}


% Old stuff, from old intro, maybe we can fit it into here to add color
In digital computing, the boundary between architecture and compilers is quite porous and is determined by a processor's instruction set. Architecture studies processor resources to solve an algorithm given a particular instruction set which is fixed in hardware. This instruction set is produced by a compiler, a piece of (low-level) software which transforms over pieces of (high-level) software. This instruction set can change based on which algorithms it allows to solve efficiently as well as which processors it allows to manufacture efficiently as well as which operations it allows humans to understand easily. All of these factors combine to make architecture an art and an engineering discipline rather than merely a science.

Quantum computers make this problem even more difficult due to the nature of a quantum bit. Because transformations between quantum states vary continuously over the space of unitary matrices with complex coefficients, we can only approximate desired quantum logic gates using a fixed set, given to us by fault-tolerance.


%\section{Conclusion}
\label{sec:qcompile-conclude}

In conclusion, we note the following interesting open problems in
quantum compiling.

\begin{itemize}

\item Problem 1

\item Problem 2

\end{itemize}