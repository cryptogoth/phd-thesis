\chapter{Quantum Compiling}
\label{chap:qcompile}

Quantum compiling is the process of approximating arbitrary,
$n$-qubit unitary operations using a universal set which
is usually fixed, finite, and contains only single-qubit
and two-qubit gates. Like quantum architecture, quantum
compiling plays an intermediate role in between quantum
algorithms, which determine which gates to apply in a circuit,
and quantum error correction, which determines the
fault-tolerant gate set.
Quantum compiling helps mitigate one source of error,
the difference $\epsilon$ between a desired gate and its
finite approximation,
and consumes its own circuit resources, usually measured in
$(1 / \epsilon)$.
Therefore, it is important to study efficient quantum 
compiling so that we don't lose any quantum algorithmic speedups.
Most interestingly to this dissertation, quantum compiling itself
is an algorithm and can be mapped to a low-depth, nearest-neighbor
architecture. That is the subject of this chapter.

In Section \ref{sec:qcompile-bg}, we provide preliminary definitions
and key results in quantum compiling. We discuss both upper
and lower bound results, and the primary division of labor in
the literate: compiling single-qubit gates, and reducing
multi-qubit gates to single-qubit ones. We discuss the
two problems of approximating quantum gates and exact synthesis.
We also discuss the main techniques used in recent works
to improve the exponent in the SK works to meet the lower
bound.

In Section \ref{sec:ksv-resource}, we provide a pedagogical
review of the KSV algorithm, compared to the conventional SK
algorithm. We also calculate resources needed to implement
KSV on 2D CCNTCM.

In Section \ref{sec:ksv-error}, we discuss one particular
empirical optimization for quantum circuits, that of measuring
early. In the case of KSV quantum compiling, some
processing may be offloaded to a classical computer if one does
not perform coherent measurements. This trades off quantum
computer resources of circuit depth and size for width.

In Section \ref{sec:qcompile-misc}, we contribute several
results related to single-qubit quantum compiling. These
procedures are probabilistic and have some limitations.
One is not universal, the other does not compile to a
fixed basis. However, both may be useful as parts of
larger quantum compiling approaches in the future.

Finally, in Section \ref{sec:qcompile-conclude}, we present the results of
our comparison and conclude with interesting open problems and
future directions for results in this field.

We discuss the seminal work
in this field by Solovay-Kitaev \cite{Solovay1995} and a
parallelized version which trades circuit depth for circuit width
by Kitaev-Shen-Vyalyi \cite{Kitaev2002}. We discuss two primary
ways to divide up the problem of quantum compiling: 
compiling single-qubit gates, reducing multi-qubit gates to
single-qubit ones. We also provide a pedagogical review of
startling recent advances in quantum compiling

In the last year, many recent advances

\section{Quantum Compiling Themes}
\label{sec:qcompile-bg}

Quantum compiling is a classical procedure for transforming quantum circuits.
The image to keep in mind throughout this entire section is shown in
Figure \ref{fig:qcompile}.

\begin{figure}
\begin{center}
\begin{displaymath}
\begin{array}{ccc}

%%%%%%%%%%%%%%%%
% Source circuit
%\underbrace{
\begin{array}{c}
S = 2 \\
\Qcircuit @C=0.5em @R=.5em { 
	& \multigate{4}{U_1} & \qw & \multigate{4}{U_2} & \qw \\ 
	& \ghost{U_1}        & \qw & \ghost{U_2}        & \qw \\
	& \ghost{U_1}        & \qw & \ghost{U_2}        & \qw \\
	& \ghost{U_1}        & \qw & \ghost{U_2}        & \qw \\
	& \ghost{U_1}        & \qw & \ghost{U_2}        & \qw 
	\gategroup{1}{2}{5}{4}{.7em}{--}
}\\
\xymatrix {
  & D=2 \ar[l] \ar[r] & \\
 }
\end{array}
%}_{C}

%& 
%\begin{array}{c}
%\textsc{Quantum Compiler} \\
\rightarrow
%\end{array}
%&

%%%%%%%%%%%%%%%%
% Target circuit
%\underbrace{
\begin{array}{c}
S' = 15 \\
\Qcircuit @C=0.5em @R=.5em { 
	& \gate{H} & \qw & \ctrl{1} & \gate{H} & \qw & \qw      & \ctrl{1} & \qw \\ 
	& \gate{H} & \qw & \targfix & \ctrl{2} & \qw & \gate{K} & \targfix & \qw \\
	& \gate{H} & \qw & \gate{K} & \qw      & \qw & \gate{H} & \qw      & \qw \\
	& \gate{H} & \qw & \ctrl{1} & \targfix & \qw & \gate{H} & \qw      & \qw \\
	& \gate{H} & \qw & \targfix & \gate{H} & \qw & \qw      & \qw      & \qw
	\gategroup{1}{2}{5}{9}{.7em}{--}
}\\
\xymatrix {
  & & D'=5 \ar[ll] \ar[rr] & & \\
 }
\end{array}
%}_{C}

\end{array}
\end{displaymath}

\caption{An arbitrary quantum circuit being compiled into single-qubit gates and $CNOT$.}
\label{fig:qcompile}
\end{center}
\end{figure}

General quantum compiling can be subdivided into more special-purpose tasks along several axes,
which are cross-cutting themes in any literature review of quantum compilers.
These themes also provide a context for understanding the resource consumption
for a wide variety of quantum compilers.

These axes are:

\begin{enumerate}
\item single-qubit compiling versus multi-qubit compiling
\item exact synthesis versus approximative quantum compiling
\item deterministic versus probabilistic quantum compiling
\item compilers with provable upper bounds versus conjectured upper bounds
\end{enumerate}

The first axis is 
single-qubit compiling
(mentioned previously in Section \ref{subsec:qcompile-single}) versus
multi-qubit compiling. Some algorithms which work on single-qubit compiling
can be generalized directly to the multi-qubit case. In fact, all known
examples of these generalized algorithms can accept an arbitrary circuit
basis $\mathcal{B}$ \cite{Amy2012,Dawson2005,Fowler2011,Booth2012}.
That is, they do not exploit any special structure of
a particular basis. The circuit basis is another input to the algorithm,
possibly to an additional classical preprocessing step. Whether the algorithm
is a single-qubit or a multi-qubit algorithm depends on whether the basis
is single-qubit or multi-qubit.

There is an intermediate point on this axis, between single-qubit and multi-qubit,
which is the reduction of a multi-qubit circuit into a basis of
single-qubit and two-qubit gates. This task is often called \emph{quantum circuit synthesis},
and we will discuss it in Section \ref{subsec:qcompile-multi}.

The second axis is compiling a circuit exactly or approximately.
Exact synthesis refers to the case of determining whether a
target circuit $C$ is implementable from a basis $\mathcal{B}$
with no error ($\epsilon = 0$). If this is possible, a quantum compiler
should return the sequence of gates which constitute the exact
synthesis. Furthermore, exact synthesizers often have a goal of
returning the \emph{optimal} sequence of compiled gates, that is,
one with minimal length $\ell$. In the compilers that we review
in Section \ref{sec:qcompile-review}, $\ell$ stands for the optimal
depth of non-Clifford resources in a basis which also contains Clifford
gates. Non-Clifford resources are always more expensive than
Clifford gates in most error-correcting codes. There is evidence
that the Clifford resources needed to synthesis the non-Clifford gates $T$ and $Toffoli$
are within a small constant factor of each other \cite{Eastin2012,Jones2012}

Exact synthesizers often enumerate over all circuits of
a certain length from a certain basis $\mathcal{B}$. Therefore, their
resources are upper bounded by a brute force search, which takes
time upper-bounded by $|\mathcal{B}|^{\ell}$.
Approximative quantum compiling conforms to our usual notion where
$\epsilon > 0$, and achieving smaller error costs more resources. Many
exact synthesis algorithms can be used to build basic approximations
for the Solovay-Kitaev algorithm more efficiently, and therefore help
achieve better approximative upper bounds as verified by numerical
simulation over random unitaries.

What is the relationship between $\ell$ and $\epsilon$? By a volume
argument, the minimum number of points in an $\epsilon_0$-net for
$SU(d=2^n)$ is $1/(\epsilon^{d^2 - 1})$. If we were to do an approximative
search within error $\epsilon_0$
for a circuit in $SU(d)$ which is known to have optimal length
$\ell$, we would have to enumerate all sequences from a basis $\mathcal{B}$
of up to length $\ell$ in the worst case, of which there are $|\mathcal{B}|^{\ell}$.
Therefore, we have the following relationship.

\begin{equation}
\ell \ge (d^2 - 1) \log_{|B|}(1/\epsilon) + O(1)
\end{equation}

The third axis is whether a quantum compiling algorithm uses randomness
or is completely deterministic. For known randomized algorithms, it is
an open problem whether the algorithm can be derandomized or not
\cite{Kliuchnikov2012a}, and numerical verification is necessary to
show the desired distribution of running times.

The fourth axis is whether a quantum compiler has upper bounds
(usually on running time or compiled sequence length) that are provable or
based on a conjecture. Both deterministic and randomized
algorithms can have provable upper bounds, although
in the latter case, one calculates the average-case and upper bounds the
variance. Likewise, both deterministic and randomized algorithms can
be based on a conjecture. One example is a deterministic algorithm
whose resources are too difficult to compute in any other way than
numerical simulation and fitting a curve to the data.

These four axes can be used to classify quantum compilers, although some
algorithms can be placed in multiple categories. For example, many
single-qubit quantum compilers which perform exact synthesis can be
incorporated into a hybrid algorithm which then performs
approximation. And of course, some single-qubit quantum compilers can be generalized
into multi-qubit algorithms.

A fifth axis could be formed, which is whether the compiled circuit requires
arbitrarily long interactions for $CNOT$ or is nearest-neighbor. Such a
quantum compiler could also divide up a compiled circuit into an optimal
number of modules on \textsf{2D CCNTCM} to also minimize module depth and
module size (inter-module teleportations). This is an interesting direction
for future research.

%%%%%%%%%%%%%%%%%%%%%%%%%%%%%%%%%%%%%%%%%%%%%%%%%%%%%%%%%%%%%%%%%%%%%%%%%%%%%%
\subsection{Quantum Compiler Resources}

Just as a quantum algorithm with arbitrary long-range interactions incurs
some overhead in being mapped to a nearest-neighbor architecture,
a quantum compiler itself is an algorithm. It always has a classical
component, which runs on a digital computer, and transforms a classical
description of an input quantum circuit into an output circuit from
a basis $\mathcal{B}$. The compiled output circuit then runs on a
quantum computer. In general, the compiled output circuit $\tilde{C}$ consumes
resources which are greater than those of the input circuit $C$.

Not all quantum compilers are ``total functions.'' Some of them, notably
single-qubit compilers, are ``promise functions'' in that they can
only compile gates of a certain form (usually $R_Z(\phi)$) and require
prior decomposition of a multi-qubit gate down to the set of
$Q \cup \{R_Z(\phi)\}$.

\begin{description}
\item[classical runtime $R$:] the classical time it takes to return a 
compiled quantum circuit.
\item[input depth $D$:] the depth of the input quantum circuit in arbitrary
$n$-qubit gates.
\item[input size $S$:] the size of the input quantum circuit in arbitrary
$n$-qubit gates.
\item[input width $W$:] the width of the input quantum circuit in qubits.
\item[compiled depth $D'$:] the compiled quantum circuit depth, equal to
the compiled sequence length for single-qubit circuits.
\item[compiled size $S$:] the compiled quantum circuit size, which is
identical to compiled depth if no ancillae are used (compiled width is zero).
\item[compiled width $W$:] the compiled quantum circuit width, which includes
the width of the input circuit as well as any ancillae introduced by
the compiler.
\end{description}

All but the first resource are quantum in nature, and follow the definitions
for circuit resources from Chapter \ref{chap:factor-polylog}. Because
compilation incurs some overhead, we have $D' \ge D$, $S' \ge S$, and
$W' \ge W$.

It's also known that
in order to approximate a circuit with $S$ gates to a total precision of
$\epsilon$
requires each gate to be approximated to a precision of
$n = O(\log(S/\epsilon)$ \cite{Lloyd1995}. We denote this per-gate precision
$n$, since it serves as an independent parameter for compiling. For
single-qubit gates, $S = 1$, and this corresponds exactly with our previous
definition for $n$ in Section \ref{sec:qcompile-basis}.

We do not measure classical space requirements, although these may be
exponential. This would be a useful metric for comparison for future work.

%%%%%%%%%%%%%%%%%%%%%%%%%%%%%%%%%%%%%%%%%%%%%%%%%%%%%%%%%%%%%%%%%%%%%%%%%%%%%%
\subsection{Decomposition to Bounded-Qubit Gates}
\label{subsec:qcompile-multi}

Restricting ourselves to the simplest case of
single-qubit circuits allows us to exploit a lot of structure
in the group $U(2)$ (or its related subgroups $SU(2)$ and $PSU(2)$).
From a volume argument, we can derive a general
lower bound for the efficiency of the multi-qubit case \cite{Harrow2002},
as well as determine how our compiling efficiency scales with dimensionality.
Any
SK-style algorithm produces worst-case sequence lengths $\ell_d$ that
are longer than worst-case single-qubit sequence lengths $\ell_1$ by a certain multiplicative
prefactor. This prefactor has a dependence that is at least
polynomial in $d = 2^n$. 

\begin{equation}
% TODO fact check this!
\ell_d / \ell_1 = \Omega \left( \frac{d^2 - 1}{ \log |\mathcal{B}| } \right )
\end{equation}

This is an example of task modularity which allows us
to divide the effort of quantum compiling between the
single-qubit case and then decomposition to single-qubit gates and
$CNOT$. It is a heuristic which often results
in simple decompositions to implement in (classical) software.
It may not be asymptotically optimal compared to generic 
multi-qubit protocols. However, for small input sizes, it is often
tractable to run on modern digital computers.

Now that we have handled the single-qubit case, how can we leverage this
to compile general $n$-qubit gates? We need a reduction to the basis
$\mathcal{Q} = U(2) \cup \{ CNOT \}$, as originally depicted in
Figure \ref{fig:qcompile}.
It turns out that almost any two-qubit gate plus arbitrary single-qubit
rotations are universal \cite{Bremner2002}. However, we will stick with CNOT
due to its other useful properties. A table of known
upper and lower bounds for this task are given in Table \ref{tab:multi}.
A standard two-level decomposition, such as provided on page 70 of \cite{Kitaev2002},
decomposes a general $U(d=2^n)$ matrix down to $O(d^2)$ ``two-level'' matrices which can
be implemented with multiply-controlled single qubit gates $\Lambda^{n-1}(U)$
for $U \in U(2)$. These gates are implementable with $O(n)$ $CNOT$ gates each.

The recent Giles-Selinger proves a conjecture by Kliuchnikov-Mosca-Maslov \cite{Kliuchnikov2012e}
that $n$-qubit circuits implementable by the basis $\mathcal{C}_2 \cup \{ T \}$ is
equivalent to all $U(2^n)$ matrices with elements from the ring $\mathbb{Z}\left[i,\frac{1}{\sqrt{2}}\right]$.
Their construction to find this exact synthesis of an $n$-qubit gate is meant to
prove this equivalent, and is not optimal.

The optimal bound needed for this in terms of $CNOT$ gates in
the compiled output (the dominant cost) is still exponential
$O(4^n)$ \cite{Shende2004}.

\begin{table}[hbt!]
\centerline{
\begin{tabular}{|c|c|}
\hline
Decomposition Method & $CNOT$ Cost 
\hline
Giles-Selinger \cite{Giles2012} & $O(9^n nk)$\\
Two-level \cite{Kitaev2002} & $O(4^n n)$ \\
Vartiainen-M\"{o}tt\"{o}nen-Salomaa \cite{Vartainen2003} & $o(11\cdot 4^n)$ \\
Aho-Svore \cite{Aho2003} & $o(1.17\cdot 4^n - 3.51\cdot 4^n + 3.34)$ \\
Shende-Bullock-Markov \cite{Shende2004a} & $o(0.48\cdot 4^n - 1.50\cdot 2^n + 1.34)$ \\
Shende-Bullock-Markov \cite{Shende2004} & $\omega(0.25\cdot 4^n - 3n - 1)$ \\
BBC+ \cite{Barenco1995a} & $\omega(0.10\cdot 4^n - 0.34n - 0.12)$ \\
\hline
\end{tabular}
}
\label{tab:multi}
\caption{Comparison of multi-qubit circuit synthesizers in $CNOT$ cost, both upper and lower bounds.
The $o(\cdot)$ and $\omega(\cdot)$ notation are used to indicate when multiplicative constants are known.}
\end{table}

\input{qcompile/qcompile-ksv-resources.tex}

\input{qcompile/qcompile-ksv-error.tex}

\section{Conclusion}
\label{sec:qcompile-conclude}

In this chapter, we examined quantum compiling as a necessary
puzzle piece to complete our previous factoring architectures
as well as
an active field of research in its own right. This chapter
combined a pedagogical review at the beginning of quantum
compiling in general and phase kickback quantum compiling
using quantum Fourier states (QFS) in particular. It then ended
with new, more depth-efficient results in QFS generation.

Expanding beyond single-qubit gates and circuit bases, we
studied the main themes of quantum compiler research,
including multi-qubit compiling, exact versus approximative,
deterministic versus randomized, and provably-efficient versus
conjectured-efficient. We provided a wide-ranging resource
comparison
of the current state-of-the-art in single-qubit
compiling. We then explicated in detail QFS generation
using the Kitaev-Shen-Vyalyi (KSV) method of parallelized
phase estimation with classical postprocessing. We called
this method PK-KSV.

In our main contribution, we presented
an optimized version of PK-KSV using early measurement
which we called PK-KSVe. We contributed depth-efficient
adders on \textsf{2D CCNTCM} for any phase kickback
compiler in Lemma \ref{lem:w}. Then we compared the resources required
for three different phase kick compilers,
PK-KSV, PK-KSVe, and a method by Jones \cite{Jones2013}
which we call PK-Jones. Our optimized approach
PK-KSVe has the lowest depth of these approaches.
Finally, we contribute a method for completing
our factoring architecture from the previous chapter
while maintaining sub-logarithmic depth in
Theorem \ref{thm:qcompile}.

This chapter has also raised interesting open question for
future research.
Among the quantum compiling themes of Section \ref{sec:qcompile-bg},
we could form a
fifth axis which is whether the compiled circuit obeys (hybrid) nearest-neighbor
constraints or not. Such compilers would be judged based on how well
they partitioned an input circuit to have an optimal
number of modules on \textsf{2D CCNTCM} to also minimize module depth and
module size (inter-module teleportations).
When comparing existing related works in Section \ref{sec:qcompile-review},
we did not measure classical space requirements, although these may be
exponential. This would be a useful metric for future comparison.
To extend
the comparison of PK-KSV(e) and PK-Jones in Section \ref{sec:qcompile-ksv},
one could calculate numerical upper bounds
for performing QFT rotations sufficient for
Shor's factoring algorithm on realistic key sizes of
2048-4096 bits. Finally, in Section \ref{sec:qcompile-maj},
we presented two conjectures which have deep implications for
depth-efficient fault-tolerance.