\section{Quantum Compiler Review}
\label{sec:qcompile-review}

This section reviews all known single-qubit quantum compilers from
before 2012 (in Section \ref{subsec:qcompile-pre2012}) to the
present (in Section \ref{subsec:qcompile-post2012}) which use
the basis $\mathcal{C}_2 \cup \{T, \text{Toffoli} \}$.
Quantum compilers which use alternative, non-Clifford resources
are discussed in Section \ref{subsec:qcompile-alternate}.

%%%%%%%%%%%%%%%%%%%%%%%%%%%%%%%%%%%%%%%%%%%%%%%%%%%%%%%%%%%%%%%%%%%%%%%%%%%%%%
\subsection{Quantum Compiling Before 2012}
\label{subsec:qcompile-pre2012}

The first known quantum compiling result by Lloyd showed that almost any two
distinct single-qubit gates were universal for single-qubit compiling
\cite{Lloyd1995}. However, it could take exponential resources in the precision:
$R,D' = O(1/\epsilon) = O(2^{\eta})$
The first \emph{efficient} quantum compiler due to
Solovay-Kitaev (SK) as independently implemented by Harrow \cite{Harrow2001}
and Dawson-Nielsen \cite{Dawson2005} had depth overhead
$D' = O(\eta^{2.71})$ and classical running time
$R = O(\eta^{3.97})$. This is a seminal result in
the field of quantum computation, and as such it was reviewed in the previous
section (\ref{sec:qcompile-sk}). We will abbreviate it afterwards as SK-DN,
since most later works use the Dawson-Nielsen implementation.
SK was later improved by Kitaev-Shen-Vyalyi
to have a longer running time but smaller depth overhead of
$R,D' = O(\eta^{3+o(1)})$
\cite{Kitaev2002}, an algorithm we will call SK-KSV.
This improvement was achieved by repeated rounds of sparsifying and
telescoping finer nets based on coarser nets. Both SK-DN and SK-KSV can be
generalized to $n$-qubit compiling, but in the latter case the multiplicative
prefactor has a doubly exponentially dependence of $exp(2^{n})$. Suggestions
for improving this to be $poly(2^n)$ are provided in the text, but this remains
an interesting open problem.

Also in the same text \cite{Kitaev2002},
a more parallel compiling procedure was discovered by Kitaev-Shen-Vyalyi
with a depth overhead of $D' = O(D\eta + \log^2\eta)$ but
a size overhead of $S' = O(S\eta + \eta^2 \log\eta)$
and a width overhead of $W' = O(W\eta^2)$
on \textsf{AC}.
This involves the generation of a circuit-wide resource,
an $n$-qubit quantum Fourier state, which can be amortized over the compilation
cost for the entire circuit. Because this algorithm uses phase kickback,
we refer to it as PK-KSV.
An equivalent, alternative proof is given by Cleve-Watrous \cite{Cleve2000}.
PK-KSV was the first quantum compiler to make a time-space (circuit depth
versus circuit width) tradeoff,
using ancillae to decrease depth. Therefore, it has non-trivial circuit space
and circuit width, and it is the first compiler where architectural concerns
matter.
We calculate detailed parameters and circuit resources for it on
\textsf{2D CCNTCM} in Section \ref{sec:qcompile-ksv}.

%Numerical comparisons between SK and KSV can be found in Section \ref{sec:ksv-results}.

As early as 2004, Austin Fowler pioneered the use of optimized enumeration
of gate sequences over a single-qubit basis, using efficient data structures
and heuristics to search and de-dupe (remove duplicates from) a database of optimal, unique
sequences up to
a fixed length \cite{Fowler2011}. He used the basis $\mathcal{C}_1 \cup \{ T \}$
and gave efficient, fault-tolerant implementations of this basis in the
Steane code. Due to the fact that the Clifford group is closed under
matrix multiplication, every compiled Fowler sequence can be simplified as a
Clifford gate alternating with $T$. Asymptotically, this still runs in time
$|\mathcal{B}|^\ell$ to produce compiled sequences of optimal length $\ell$,
with an improved multiplicative constant.
It is an exact-synthesis, deterministic, provably optimal compiler.
In its current form, it is a single-qubit compiler because it takes as input
a single-qubit basis, but it can be generalized to $n$-qubits by enumerating
and feeding in $\mathcal{C}_n \cup \{T\}$. Unfortunately, the size of the
Clifford group increases at least exponentially in $n$, which places practical
limits on any such enumerative compiler for $n > 3$ on modern digital processors.

In 2007, Matsumoto-Amano used the idea of a \emph{normalized circuit} to
improve the enumeration and search approach to compiling \cite{Matsumoto2008}. Two circuits in
normalized form compute the same gate if and only if they are the same
circuit. If a normalized form can be computed efficiently, then a
database of normalized forms can be easily searched to determine the
uniqueness of a gate sequence. Their case exploits the special structure
of the single-qubit basis $\{H,T,T^{\dagger}\}$ and is an exact-synthesis,
deterministic, provably-optimal compiler over this basis. Both the Fowler
and the Matsumoto-Amano compiler can be plugged into Solovay-Kitaev to
optimize the base approximations.

%%%%%%%%%%%%%%%%%%%%%%%%%%%%%%%%%%%%%%%%%%%%%%%%%%%%%%%%%%%%%%%%%%%%%%%%%%%%%%
\subsection{Quantum Compiling in 2012 and After}
\label{subsec:qcompile-post2012}

In 2012 and after, quantum compiling experienced a renewed interest in the
research community. Many different themes from the previous section were
taken up and developed. These developments are summarized below.

In 2012, two approaches improved upon the Fowler method by considering a
meet-in-the-middle approach: the work by Amy-Mosca-Maslov-Roetteler (AMMR)
\cite{Amy2012} and that by Booth (\cite{Booth2012}.
The Fowler approach searches for an
optimal path from identity to the desired target gate whose length 
we define as $\ell$. The meet-in-the-middle approach divides this into two tasks of
reaching a middle point, searching simultaneously from identity and the
target gate, to find optimal paths of length $\lceil \ell/2 \rceil$
and $\lfloor \ell/2 \rfloor$. This gives a quadratic speed improvement over
the Fowler approach.

Both the AMMR and Booth compilers can be generalized to
multi-qubit circuits by feeding in a multi-qubit basis. In fact, the AMMR compiler
was able to generate the two-qubit Clifford group with
size $|\mathcal{C}_2| = 11,520$ and the three-qubit Clifford
$|\mathcal{C}_3| = 92,897,280$, and therefore it is able to optimize
the depth of $T$ gates in a circuit, in addition to optimizing the total
count of $T$ gates. However, for multi-qubit gates of $n \ge 4$,
the running time of enumerative search, even with optimizations, is
doubly exponential (exponential enumeration over an exponential-sized
set of Clifford group elements), which quickly encounters practical limits
of modern digital computers.
On a more positive note, the AMMR compiler gives the first improved decomposition
of the Toffoli gate in terms of $\mathcal{C}_1 \cup \{T, T^{\dagger}\}$
since 1995 \cite{Barenco1995a}.
%Interestingly, the AMMR paper states without
%proof the
%exact equivalence of single-qubit circuits over the basis
%$\{H, T, T^{\dagger}\}$ and $U(2)$ matrices with elements from the ring $\mathbb{Z}\left[i,\frac{1}{\sqrt{2}}\right]$.
%However, this is not proven until a later paper \cite{Kliuchnikov2013}.

The Bocharov-Svore (BS) compiler combines the idea of normalizing circuits
and enumerative search to compile \emph{canonical circuits} on single-qubits.
Canonical circuits
implement the same circuit if their matrices are the same up to
conjugation by two (possibly distinct) Clifford gates. This leads to
a fourth-root speedup of search time for this exact-synthesis, single-qubit
compiler using the $\{H, T, T^{\dagger} \}$ basis. While its circuit resources
can be expressed in terms of basis size $|\mathcal{B}|$ for comparison with
the AMMR, Booth, and Fowler compilers, in fact the BS compiler is only known
to work for the particular single-qubit basis above. Furthermore, the BS exact
compiler can be plugged into SK to optimize the enumeration and storage of
basic approximations. When this is done and simulated for
random unitaries, numerical experiments reveal an improved compiled
sequence length of $O(\eta^{3.4})$. We can call this hybrid
compiler SK-BS.

Kliuchnikov-Mosca-Maslov (KMM) provide significant advances for both the
exact-synthesis and approximative approaches to compiling, which we can
abbreviate KMMe and KMMa, respectively.

For the
exact-synthesis case \cite{Kliuchnikov2012e},
KMMe provides a rigorous proof of the equivalence
of circuits over the basis $\mathcal{C}_1 \cup \{T, T^{\dagger} \}$ and $U(2)$ matrices
with elements from the ring $\mathbb{Z}\left[i,\frac{1}{\sqrt{2}}\right]$.
Reducing the problem of circuit synthesis over this basis to state
generation, with elements from $\mathbb{Z}\left[i,\frac{1}{\sqrt{2}}\right]$,
KMMe remarkably finds sequences of optimal length $\ell$ in time
polynomial in $\ell$
($O(\ell ^2)$ single-bit operations), and not exponential in $\ell$ as
all previous approaches. This is a fair comparison because other 
exact-synthesis algorithms above implicitly
depend on a fixed-precision floating-point
arithmetic, and KMMe makes its bit-dependence explicit. The KMMe is
a single-qubit compiler, but its generalization to higher dimensions
depended on proving that
$U(2^n)$ matrices of elements from $\mathbb{Z}\left[i,\frac{1}{\sqrt{2}}\right]$
are equivalent to $n$-qubit circuits over the same basis, which was done by
Giles-Selinger \cite{Giles2013}.

For the approximative case \cite{Kliuchnikov2012a}, KMMa saturates the lower
bound of $\Omega(\eta)$ by using two ancillae prepared in the
$\ket{00}$ state and basis of $\mathcal{C}_1 \cup \{ T \}$.
It reduces the problem of approximating a
$R_Z(\phi)$ single-qubit rotation with expressing a large integer $M$ as the
sum of four squares. There is a known probabilistic algorithm to
solve this Diophantine equation
by Rabin-Shallit \cite{Rabin1985} that takes time $O(\log^2(M)\log\log M)$. Therefore,
the KMMa compiler runs in classical time $O(\eta^2\log\eta)$
and returns sequences of length $O(\eta)$, which is optimal up to
a constant factor.

A natural improvement is to see if the ancillae of KMMa can be done away with
completely. This question is answered in the affirmative by
Selinger using an approximative, randomized, single-qubit compiler
\cite{Selinger2012}. In that work, general unitaries are approximated up to
error $\epsilon$ via the $R_Z(\phi)$ (or equivalently, $R_X(\phi)$ rotations)
of accuracy $\epsilon/3$ using elements of the
ring $\mathcal{R}$ so that a known, optimal exact-synthesizer such as KMMe
can be used. The approximation to $R_Z(\phi)$ involves solving
Diophantine equations to find approximations to real numbers in the
ring $\mathbb{Z}\left[\sqrt{2}\right]$. Solutions to this require
random sampling of candidate solutions from $\mathcal{R}$ whose average
success probability depends on an open conjecture about the distribution
of odd primes. The running time and optimal sequence-depth of the
Selinger compiler is verified by numerical simulation.

Improving SK-style approximative and Fowler-style enumerative compilers has
dominated most of the progress in the years 2012-2013. However, we now mention
a work which uniquely improves on phase-kickback and the PK-KSV-style of
compilation that uses $O(\eta)$ ancillae in order to achieve $o(\eta)$
depth.

Based on previous multilevel distillation protocols for magic states \cite{Jones2012},
Jones devised a remarkable way to recursively distill Fourier states using an
initial approximation of $2$-qubit states using only Clifford gates. This provides
an alternative to generating quantum Fourier states to PK-KSV to be used with
phase kickback; therefore we abbreviate this method PK-Jones. Fourier state
distillation results
in an asymptotic improvement in the number of non-Clifford (Toffoli) gates
required versus PK-KSV ($O(\eta\log \eta)$ versus $O(\eta^2\log \eta)$ for an
$\eta$-qubit Fourier state), while maintaining the same asymptotic depth overhead
($O(\log \eta)$). The resources of PK-Jones are compared with an optimized version of PK-KSV in
Section \ref{sec:qcompile-ksv}.

All of these approaches, which admit direct comparison more-or-less, are
displayed in Table \ref{tab:qcompile-compare} in Section
\ref{sec:qcompile-compare}.

%%%%%%%%%%%%%%%%%%%%%%%%%%%%%%%%%%%%%%%%%%%%%%%%%%%%%%%%%%%%%%%%%%%%%%%%%%%%%%
\subsection{Magic States for Quantum Compiling}

Instead of using non-Clifford gates to complete our basis into a universal
gate set, we can include ``non-Clifford'' states as an additional resource.
These states are called \emph{magic states} if, combined with the Clifford
gates, they enable universal quantum computation.
Quantum compilers which use these magic states count them as a non-Clifford
resource instead of $T$ or Toffoli gates, since such states cannot
be produced directly using Clifford gates. Rather, they must be
distilled in a probabilistic procedure which itself uses only 
Clifford gates and measurement.

%There are two types of magic states. 
%The most famous such example is the $+1$ eigenstate of the Hadamard operator,
%often written as:
%
% TODO Fact check this, what about \ket{0} + e^{i\pi/4}\ket{1}
%\begin{equation}
%\ket{H} = \cos(\pi/8)\ket{0} + \sin(\pi/8)\ket{1}
%\end{equation}
%
Magic state distillation was originally proposed by Bravyi and Kitaev
\cite{Bravyi2005} as a model of universal quantum computation (UQC) which
allowed for noisy, or imperfectly prepared, initial quantum states. These
noisy states could be ``purified'' or distilled down to magic states.
These magic states come in two types, the eigenstates
of the $H$ or another operator, which we call $A$.
%
\begin{equation}
H = 
\normtwo
\left[ \begin{array}{cc}
1 & 1\\
1 & -1
\end{array} \right]
\qquad
A = e^{i\pi / 4} SH = \frac{e^{i\pi/4}}{\sqrt{2}}
\left[ \begin{array}{cc}
1 & 1\\
i & -i
\end{array} \right]
\label{eqn:T}
\end{equation}
%
The noisy initial states can be considered to be prepared with some
error in the basis of one of the eigenstates above.
These states are distilled via successive rounds of recursive error
correction, where the efficiency of distillation with each round,
and the resources required to do so, depend on the code used.

% TODO Insert here a comparison of different codes used
% Original Knill result
% Bravyi-Kitaev
% Bravyi-Haah

\subsection{Alternative Bases and Resources}
\label{subsec:qcompile-alternate}

Now we turn to three recent works which use magic state distillation to compile
arbitrary single-qubit gates. These works all compile to an alternative
basis different from $\mathcal{C}_n \cup \{T,T^{\dagger}\}$.

A recent approach by Bocharov-Gurevich-Svore \cite{Bocharov2013}
compiles to subsets of the Clifford group augmented with the non-Clifford
$V$-basis, first discovered by Lubotsky-Phillips-Sarnak \cite{Lubotsky1987},
which was proven to permit the lower-bound of compiled sequence length
$O(\eta)$
\cite{Harrow2002}.
%
\begin{equation}
V_1 = \frac{1}{\sqrt{5}}
\left[
\begin{array}{cc}
1  & 2i \\
2i & 1
\end{array}
\right]
\qquad
V_2 = \frac{1}{\sqrt{5}}
\left[
\begin{array}{cc}
1  & 2\\
-2 & 1
\end{array}
\right]
\qquad
V_3 = \frac{1}{\sqrt{5}}
\left[
\begin{array}{cc}
1+2i & 0 \\
   0 & 1-2i
\end{array}
\right]
\end{equation}
%
This work uses the properties of Lipschitz quaternions with norms $5^l$, ($l \in \mathbb{Z}, l \ge 0$). It
contains a randomized algorithm whose running time is based on a conjecture from geometric number theory.
There is currently no complete, fault-tolerant method of compiling all three gates from the $V$ basis into
our usual universal set of $\mathcal{C}_1 \cup \{T\}$. However, the appendix of \cite{Bocharov2013}
gives a method for implementing the exact $V_2$ gate using the (probabilistic) magic state distillation of
Duclos-Cianci and Svore \cite{DuclosCianci2012}. The average cost in $\ket{H}$
states for enacting a $V_2$ gate is $22.75$. It is an open problem how many
$\ket{H}$ states are needed to enact the gates $V_1$ and $V_3$.
We cannot compare it directly to previous algorithms which consider the number of $T$ gates ($T_c$)
the primary resource, or the compiled sequence length $D'$ as an upper bound to $T_c$.
However, the SK algorithm allows us to provide an alternative basis. By
comparing the compiled sequence lengths in the $V$ basis and noting any
improvements over the $\mathcal{C}_1 \cup \{T\}$ basis, 
measuring the length of compiled sequences 

The previous works all assume a minimal basis which only includes a fixed
number (usually one) non-Clifford gate (a $T$ gate or Toffoli). We can
call these ``reduced instruction set computing'' (RISC) bases, in analogy to
digital processor architectures.
In contrast, Landahl and
Cesare consider a ``complex instruction set computing'' (CISC) basis
\cite{Landahl2013b}
which
can include certain non-Clifford states of the form
$\normtwo (\ket{0} + e^{i\pi / 2^k} \ket{1} ) = R_Z(\pi/{2^k})\ket{+}$. These
non-Clifford states can be distilled to desired threshold error $\epsilon'$ using
a probabilistic procedure using only Clifford gates similar to distilling
the state $\ket{H}$ described above. This error $\epsilon'$ is proportional
to the desired error of gate compilation.%: $\epsilon' \tilde \epsilon$.
Therefore, the work of Landahl-Cesare also unifies into the sum of the 
exponent of the logarithm $\alpha + \beta + \gamma$ the
normally distinct procedures of
magic state distillation ($\alpha$),
quantum compiling ($\beta$),
and error-correction ($\gamma$) into
the sum of exponents of the following logarithm.
%
\begin{equation}
R,D = O(\eta ^{\alpha + \beta + \gamma})
\end{equation}
%
By optimizing over magic state distillation and quantum compiling as a
single task, Landahl-Cesare achieves $\alpha + \beta \approx 1$, when the
best that can be achieved with these two procedures separately is
$\alpha + \beta \ge 2$.

%%%%%%%%%%%%%%%%%%%%%%%%%%%%%%%%%%%%%%%%%%%%%%%%%%%%%%%%%%%%%%%%
%\subsection{Generalizations to Quantum Compiling}
%\label{subsec:alt-tasks}

%The literature of quantum compiling has traditionally drawn boundaries
%around itself, restricting its study to approximation of unitary gates
%with perfect operations. However, s
%around 