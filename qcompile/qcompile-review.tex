\section{Quantum Compiler Review}
\label{sec:qcompile-review}

\subsection{Alternative Bases and Resources}

Now we turn to three recent works which use magic state distillation to compile
arbitrary single-qubit gates. These works either compile to an alternate
basis from the Clifford group $\mathcal{C}_n$ and $T$

A recent approach by Bocharov-Gurevich-Svore \cite{Bocharov2013}
compiles to subsets of the Clifford group augmented with the non-Clifford $V$-basis, which was proven to permit the lower-bound of compiled sequence length $O(\log^1(1/\epsilon)$
\cite{Harrow2003}.

\begin{equation}
V_1 = TODO \qquad V_2 = TODO \qquad V_3 = TODO
\end{equation}

This work uses the properties of Lipschitz quaternions with norms $5^l$, ($l \in \mathbb{Z}, l \ge 0$). It
contains a randomized algorithm whose running time is based on a conjecture from geometric number theory.
There is currently no complete, fault-tolerant method of compiling all three gates from the $V$ basis into
our usual universal set of $\mathcal{C}_1 \cup \{T\}$. However, the appendix of \cite{Bocharov2013}
gives a method for implementing the exact $V_2$ gate using the (probabilistic) magic state distillation of
Duclos-Cianci and Svore \cite{DuclosCianci2012}. 
We cannot compare it directly to previous algorithms which consider the number of $T$ gates ($T_c$)
the primary resource, or the compiled sequence length $D'$ as an upper bound to $T_c$.