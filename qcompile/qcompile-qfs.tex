\section{Phase Kickback and Quantum Fourier States}
\label{sec:qfs}

In this section, we discuss the \emph{phase kickback} procedure for
producing a qubit with arbitrary phase: $\ket{0} + e^{i\phi}\ket{1}$.
From the PAR procedure, a qubit could be prepared in such a state
``offline' and then teleported later into a \textsf{2D CCNTCM} module to
probabilistically enact rotations $R_Z(\phi)$ \cite{Jones2011}.

Phase kickback requires controlled addition on a target state in 
the quantum Fourier basis. This will provide us the foundation
to understand the KSV procedure for generating such quantum Fourier
states in the next section.

In Section \ref{subsec:qfs-basis}, we discuss the properties of the
basis of quantum Fourier states, including their relationship with
addition modulo $2^n$. In Section \ref{subsec:qfs-resources},
we calculate circuit resources for performing these adders
on \textsf{2D CCNTCM}. Although the adders are generic, they will be
used in the resource calculations of KSV phase estimation in
Section \ref{sec:ksv-pe}, which operates on quantum Fourier states.

%%%%%%%%%%%%%%%%%%%%%%%%%%%%%%%%%%%%%%%%%%%%%%%%%%%%%%%%%%%%%%%%%%%%%%%%%%%%%%
\subsection{Properties of the Quantum Fourier Basis}
\label{subsec:qfs-basis}

The following states are well-known as $n$-qubit quantum Fourier states,
the result of applying the quantum Fourier transform (QFT)
to the $n$-qubit computational basis state. 

\begin{equation}
\ket{\psi^{(k)}_{n}} = \frac{1}{\sqrt{2^n}} \sum_{j=0}^{2^n-1}
e^{-2\pi i j k / 2^n} \ket{j}
\end{equation}

These states form an alternate, orthonormal basis indexed by
$0 \le k < 2^n$. We will often just call these Fourier states,
and neglect the subscript $n$, which is implied.
Note that the state $\ket{\psi^{(0)}}$, sometimes
called the fundamental Fourier state, is simply the equal superposition
of all computational basis states, or the tensor product of
$n$ qubits in the state $\ket{+}$, or the result of applying
$n$ Hadamard's qubit-wise to a state beginning in $\ket{0}^n$.

Note that these states $\ket{\psi}^{(k)}_n$ are the
QFT states of the $n$-qubit computational basis. The usual method of
creating these states involves performing phase estimation of the
modular addition operator. These are implicitly hard in that
all known procedures take size $O(n\log n)$, even if the depths
can be decreased to $O(\log n)$ \cite{Jones2012} or in some cases $O(1)$
given unbounded quantum fanout
\cite{Browne2009}.

However, we can create a superposition in constant depth
over all odd $k=(2s-1)$
by starting in the state $\ket{0}^{\otimes n}$,
then applying a Hadamard and $\sigma^z$ to the most significant qubit.

\begin{equation}
\ket{\eta} = \normtwo \ket{0} - \normtwo \ket{2^{n-1}} =
\frac{1}{\sqrt{2^{n-1}}} \sum_{s=1}^{2^{n-1}} \ket{\psi_{n,2s-1}}
\end{equation}

More obviously relevant to our overall goal of approximating
$\Lambda(e^{i\phi})$, we can enact a phase
shift simply by performing the following modular addition operator, for
which $\ket{\psi_{n,k}}$ are eigenstates.

\begin{equation}
A\ket{j} \rightarrow \ket{j+1 \bmod 2^n}
\end{equation}

Applying this operator to its eigenstates results in a phase shift which
depends on the particular eigenstate.
 
\begin{equation}
A\ket{\psi_{n,k}} = e^{2\pi i \phi_k} \ket{\psi_{n,k}}
\end{equation}

Finding the eigenvalue $e^{2\pi i \phi_k}$ corresponds to finding
the phase $\phi_k = k / 2^n$.
Repeated application of $A$ (say $p$ times) would result in a phase
added to the eigenstate equal to a multiple of $e^{2\pi i p / 2^n}$

\begin{equation}
A^p\ket{\psi_{n,k}} = e^{2\pi i \phi_k / 2^n} \ket{\psi_{n,k}}
\end{equation}

This explains why we don't find even $k$ interesting,
since then we would not get a
cyclic distribution of $2^n$ different phases,
since only odd $k$
are coprime with $2^n$. The exception is $k=0$, since this is the
equal superposition of computational basis states, which we can also
efficiently create. This will be a useful starting point later on to
create addition eigenstates
for odd $k$.

\begin{displaymath}
\ket{\psi_{n,0}} = H^{\otimes n}\ket{0^n}
\end{displaymath}

Suppose we have a certain state $\ket{\psi_{n,k}}$ but we want to get enact
a phase shift $e^{2\pi i l / 2^n}$. We can do this by solving $p=p(s,l)$
in this equation:

\begin{equation}
\label{eqn:psl}
(2s-1)p \equiv l (\bmod 2^n)
\end{equation}

Stipulating $k$ to be odd guarantees that there is a unique solution $p$.

We then apply $A^p$ as follows, where $\Upsilon_n(A)$ means to
apply $A$ to the second register $p$ times, where $p$ is an $n$-qubit
number in the first register.

\begin{equation}
\label{eqn:upsilon}
\Upsilon_n(A) \ket{p}\ket{\psi_{n,k}} \rightarrow
e^{2\pi i l/2^n} \ket{p}\ket{\psi_{n,k}}
\end{equation}

If we control the operation of $\Upsilon_n(A)$ on a source qubit $\ket{+}$,
it will acquire the phase $e^{2\pi i l}$.

\begin{equation}
\Lambda(\Upsilon_n(A))\ket{+}\ket{p}\ket{\psi}^{(k)} \rightarrow
\left( \ket{0} + e^{2\pi i l}\ket{1} \right) \ket{p}\ket{\psi}^{(k)}
\end{equation}

This is not quite the phase kickback procedure, since
we must still solve for $p$ using the equation below by finding the
modular inverse $(2s - 1)^{-1} \bmod 2^n$,
making use of the following expansion from Section 13
of \cite{Kitaev2002}

\begin{equation}
p \equiv -l\sum_{j=0}^{m-1} (2s)^j \equiv -l \prod_{r=1}^{t-1}\left(1 + (2s)^{2^r}\right) \mod 2^n
\end{equation}

where $m = O(n)$ is $n$ rounded to the nearest power of $2$.  In general,
this requires a circuit of size $O(n^2 \log n)$ and depth $O((\log n)^2)$ and
represents the most expensive part of the KSV procedure as originally
presented in \cite{Kitaev2002}.

Ideally, we would obviate the need for the expensive circuit above
by ensuring that $k=1$, in which case
$p = l$. We will see how to do this in Section \ref{sec:ksv-pe}.

Finally, to copy the state $\ket{\psi_{n,k}}$ it suffices to apply the following
operator which only uses subtraction (addition with one addend and the
outcome negated in two's complement representation).

\begin{equation*}
\ket{\psi_{n,k}}^{\otimes m} = W^{-1}\left( \ket{\psi_{n,0}}^\otimes(m-1) \otimes \ket{\psi_{n,k}} \right)
\end{equation*}

where $W$ is the operator on $m$ registers, each of consisting of $n$ qubits,
which adds all the registers into its final register (modulo $2^n$).

\begin{multline}
W : \ket{x_1,\ldots,x_{m-1},x_m} \rightarrow \\
 \ket{x_1,\ldots,x_{m-1},x_1+\ldots+x_m \bmod 2^n}
\end{multline}

Having complained about the cost of modular division, how can we now
implement the operators $A$ and $W$ more efficiently on our
nearest-neighbor architecture?

%%%%%%%%%%%%%%%%%%%%%%%%%%%%%%%%%%%%%%%%%%%%%%%%%%%%%%%%%%%%%%%%%%%%%%%%%%%%%%
\subsection{Circuit Resources for Adders on Quantum Fourier States}
\label{subsec:qfs-adder}

First we prove a general result for mapping any \textsf{AC} circuit
to a \textsf{2D CCNTCM} circuit.
Then we use it to 
map the operator $A$ (increment modulo $2^n$) and the operator
$W$ (multiple addition modulo $2^n$) to \textsf{2D CCNTCM}.

\begin{lemma}
Suppose an \textsf{AC} circuit $\mathcal{C}$ has
depth $D(n)$, size $S(n)$, and width $W(n)$. Then it can be mapped
onto a \textsf{2D CCNTCM} circuit $\mathcal{C'}$ with circuit depth
$D(n)$, circuit size $S(n)$, circuit width $D(n)\cdot W(n)$,
module depth $D(n)$, module size $D(n)\cdot W(n)$, and module width
$D(n)$.
\label{lem:ac-ccntcm}
\end{lemma}

\begin{proof}
Each of $D(n)$ layers in $\mathcal{C}$ could operate on non-nearest-neighbors.
We teleport all
$W(n)$ qubits from the first module to the last module in sequence,
reordering them at each module so that we can execute the gates of that
layer only on nearest-neighbor qubits.
This gives the desired circuit and module resources on
\textsf{2D CCNTCM}.
\end{lemma}

\begin{lemma}
The operator $A\ket{j} \rightarrow \ket{j+1 \bmod 2^n}$ on
$n$-qubit register can be
implemented in circuit/module depth $O(\log n)$, circuit size $O(n)$, 
circuit width $O(n\log n)$, module size of $O(n \log n)$, and
module width of $O(\log n)$ on
\textsf{2D CCNTCM}.
\label{lem:a}
\end{lemma}

\begin{proof}
We modify the QCLA adder from
Section \ref{subsec:qcla} mapped to \textsf{2D CCNTCM} using
Lemma \ref{lem:ac-ccntcm}.
To add $2 \times n$-bit numbers, the original \text{AC} adder has
depth $O(\log n)$ and size and width of $O(n)$.
Therefore, we have the desired resources for a \textsf{2D CCNTCM} circuit.
\end{proof}

\begin{lemma}
The operator $W\ket{x_1}\ket{x_2}\cdots\ket{x_m} \rightarrow \ket{x_1}\ket{x_2}\cdots\ket{x_1 + x_2 + \ldots + x_m \bmod 2^n}$,
which operates on $m \times n$-qubit registers, can be
implemented in circuit/module depth $O(\log n + \log nm)$, circuit size $O(mn)$, 
circuit width $O(mn + n\log n)$, module size of $O(mn + n \log n)$, and
module width of $O(m + \log n)$ on
\textsf{2D CCNTCM}.
\label{lem:w}
\end{lemma}

\begin{proof}
This reduces to the problem of modular multiple addition from
Section \ref{subsec:mma}, with the following variation. Instead of
addition modulo $m < 2^n$, we perform addition modulo $2^n$, which is
much simpler. It does not involve any truncation and adding back of
modular residues. Namely, we just perform one round of constant-depth
$3 \rightarrow 2$ addition, then we redo the computation of the
highest-order bit $v_n$ to uncompute it.

Therefore, we have a $O(\log m)$-depth binary tree of $O(m)$ modules,
which requires a total of $O(mn)$ teleportation of qubits between them.
Within each module, the addition takes depth $O(1)$ and size and width
of $O(n)$. Combine this in series with the circuit for $A$ in
Lemma \ref{lem:a} and we have the desired resource bounds.
\end{proof}

In fact, we also provide numerical upper bounds for circuit resources
of both $A$ and $W$ in Table \ref{tab:qfs-adder}.

\begin{table}[hbt!]
\begin{tabular}{|c|c|c|c|c|c|c|}
\hline
Operation & $D$ & $S$ & $W$ & $\overline{D}$ & $\overline{S}$ & $\overline{W}$ \\
\hline
$A$: CSE Addition of $3\times n$-bit Numbers Modulo $2^n$ & $66$ & $56n$ & $5n$ & 0 & 0 & 1 \\
$Q$: QCLA Adder from CSE & $4\log_2 n + 6$ & $10n - 48\log_2 n - 78$ & $8n \log_2 n + 12n$ & $D$ & $W'$ & $D$  \\
$W$: Multiple CSE Addition of $t\times n$-bit Numbers Modulo $2^n$ & $57\log_2 t$ & $100nt - 200n$ & $5nt - 10n$ & $1.71\log_2 t$ & $2nt + 3t - 4n - 6$ & $t-2$
\end{tabular}
\end{table}
Armed with these properties, we're now ready to describe our version of the
KSV phase estimation procedure to product quantum Fourier states.