\section{Phase Kickback and Quantum Fourier States}
\label{sec:qcompile-qfs}

In this section, we discuss the \emph{phase kickback} procedure for
producing a qubit with arbitrary phase: $\normtwo (\ket{0} + e^{i\phi}\ket{1} )$.
From the PAR procedure, a qubit could be prepared in such a state
``offline,'' and then teleported later into a \textsf{2D CCNTCM} module to
probabilistically enact rotations $R_Z(\phi)$ \cite{Jones2012}.

Phase kickback requires controlled addition on an $\eta$-qubit target state in 
the quantum Fourier basis for desired precision $\epsilon = 2^{-\eta}$.
In this purely pedagogical review,
we will discuss known properties of the
basis of quantum Fourier states, including their relationship with
addition modulo $2^{\eta}$.
This will motivate our
understanding of the KSV procedure for generating such quantum Fourier
states in the next two sections, where we will also present our
main novel results in this chapter.

%In Section \ref{subsec:qfs-basis},  It is a purely pedagogical review which sets
%the stage for our new results in the next two sections.
%In Section \ref{subsec:qfs-adder},
%we calculate circuit resources for performing adders
%on \textsf{2D CCNTCM}. Although the adders are generic, they will be
%used in the resource calculations of KSV phase estimation in
%Section \ref{subsec:ppe}, which operates on quantum Fourier states.

%%%%%%%%%%%%%%%%%%%%%%%%%%%%%%%%%%%%%%%%%%%%%%%%%%%%%%%%%%%%%%%%%%%%%%%%%%%%%%%
%\subsection{Properties of the Quantum Fourier Basis}
%\label{subsec:qfs-basis}

The following states are well-known as $\eta$-qubit quantum Fourier states,
the result of applying the quantum Fourier transform (QFT)
to the $\eta$-qubit computational basis state. 
%
\begin{equation}
\ket{\psi^{(k)}_{\eta}} = \frac{1}{\sqrt{2^{\eta}}} \sum_{j=0}^{2^{\eta}-1}
e^{-2\pi i j k / 2^{\eta}} \ket{j}
\end{equation}
%
These states form an alternative, orthonormal basis indexed by
$0 \le k < 2^{\eta}$. We will often just call these Fourier states,
and neglect the subscript $\eta$, which is implied.
Note that the state $\ket{\psi^{(0)}}$, sometimes
called the fundamental Fourier state, is simply the equal superposition
of all computational basis states, or the tensor product of
$\eta$ qubits in the state $\ket{+}$, or the result of applying
$\eta$ Hadamards qubit-wise to a state beginning in $\ket{0}^\eta$.

Note that these states $\ket{\psi^{(k)}_n}$ are the
QFT states of the $\eta$-qubit computational basis. The usual method of
creating these states involves performing phase estimation of the
modular addition operator. These are implicitly hard in that
all known procedures take size $O(\eta\log \eta)$, even if rotations
are truncated to approximate the QFT in depth $O(\log \eta)$
\cite{Jones2012} or in some cases $O(1)$
given unbounded quantum fanout
\cite{Browne2009}.

However, we can create a superposition in constant depth
over all odd $k=(2s-1)$
by starting in the state $\ket{0}^{\otimes \eta}$,
then applying a Hadamard and $\sigma^z$ to the most significant qubit.

\begin{equation}
\ket{\eta} = \normtwo \ket{0} - \normtwo \ket{2^{\eta-1}} =
\frac{1}{\sqrt{2^{\eta-1}}} \sum_{s=1}^{2^{\eta-1}} \ket{\psi_{\eta,2s-1}}
\end{equation}

More obviously relevant to our overall goal of approximating
$\Lambda(e^{i\phi})$, we can enact a phase
shift simply by performing the following modular addition operator, for
which $\ket{\psi_{\eta}^{(k)}}$ are eigenstates.
%
\begin{equation}
A\ket{j} \rightarrow \ket{j+1 \bmod 2^{\eta}}
\end{equation}
%
Applying this operator to its eigenstates results in a phase shift which
depends on the particular eigenstate.
%
\begin{equation}
A\ket{\psi_{\eta}^{(k)}} = e^{2\pi i \phi_k} \ket{\psi_{\eta}^{(k)}}
\end{equation}
%
Finding the eigenvalue $e^{2\pi i \phi_k}$ corresponds to finding
the phase $\phi_k = k / 2^{\eta}$.
Repeated application of $A$ (say $p$ times) would result in a phase
added to the eigenstate equal to a multiple of $e^{2\pi i p / 2^{\eta}}$:
%
\begin{equation}
A^p\ket{\psi_{\eta}^{(k)}} = e^{2\pi i \phi_k / 2^{\eta}} \ket{\psi_{\eta}^{(k)}}
\end{equation}
%
This explains why we don't find even values of $k$ interesting,
since then we would not get a
cyclic distribution of $2^{\eta}$ different phases,
since only odd $k$
are coprime with $2^{\eta}$. The exception is $k=0$, since this is the
equal superposition of computational basis states, which we can also
efficiently create as in the equation below.
This will be a useful starting point later on to
create addition eigenstates
for odd $k$.
%
\begin{equation}
\ket{\psi_{\eta}^{(0)}} = H^{\otimes \eta}\ket{0}^{\otimes \eta}
\end{equation}
%
Suppose we have a certain state $\ket{\psi_{\eta}^{(k)}}$ but we want to get enact
a phase shift $e^{2\pi i l / 2^{\eta}}$. We can do this by solving $p=p(s,l)$
in this equation:
%
\begin{equation}
(2s-1)p \equiv l (\bmod 2^{\eta})
\label{eqn:psl}
\end{equation}
%
Stipulating $k$ to be odd guarantees that there is a unique solution $p$.

We then apply $A^p$ as follows, where $\Upsilon_{\eta}(A)$ means to
apply $A$ to the second register $p$ times, where $p$ is an $\eta$-qubit
number in the first register.
%
\begin{equation}
\Upsilon_{\eta}(A) \ket{p}\ket{\psi_{\eta}^{(k)}} \rightarrow
e^{2\pi i l/2^{\eta}} \ket{p}\ket{\psi_{\eta}^{(k)}}
\label{eqn:upsilon}
\end{equation}
%
If we control the operation of $\Upsilon_n(A)$ on a source qubit $\ket{+}$,
it will acquire the phase $e^{2\pi i l/2^{\eta}}$.
%
\begin{equation}
\Lambda(\Upsilon_n(A))\ket{+}\ket{p}\ket{\psi_{\eta}^{(k)}} \rightarrow
\left( \ket{0} + e^{2\pi i l}\ket{1} \right) \ket{p}\ket{\psi_{\eta}^{(k)}}
\end{equation}
%
This is not yet the phase kickback procedure, since
we must still solve for $p$ using the equation below by finding the
modular inverse $(2s - 1)^{-1} \bmod 2^{\eta}$,
making use of the following expansion from Section 13
of \cite{Kitaev2002}.
%
\begin{equation}
p \equiv -l\sum_{j=0}^{m-1} (2s)^j \equiv -l \prod_{r=1}^{t-1}\left(1 + (2s)^{2^r}\right) \mod 2^{\eta}
\label{eqn:mod-inverse}
\end{equation}
%
where $m = O(\eta)$ is $\eta$ rounded to the nearest power of $2$.  In general,
this requires a circuit of size $O(\eta^2 \log \eta)$ and depth $O((\log \eta)^2)$ and
represents the most expensive part of the KSV procedure as originally
presented in \cite{Kitaev2002}.

Ideally, we would obviate the need for the expensive circuit above
by ensuring that $k=1$, in which case
$p = l$. We will see how to do this in Section \ref{subsec:ppe}.

Finally, to copy the state $\ket{\psi_{\eta}^{(k)}}$ it suffices to apply the following
operator which only uses subtraction (addition with one addend and the
outcome negated in two's complement representation).
%
\begin{equation}
\ket{\psi_{\eta}^{(k)}}^{\otimes m} = W^{-1}\left( \ket{\psi_{\eta}^{(0)}}^{\otimes(m-1)} \otimes \ket{\psi_{\eta}^{(k)}} \right)
\end{equation}
%
where $W$ is the operator on $m$ registers, each consisting of $\eta$ qubits,
which adds all the registers into its final register (modulo $2^{\eta}$).
%
\begin{equation}
W : \ket{x_1}\ket{x_2}\ldots\ket{x_{m-1}}\ket{x_m} \rightarrow
 \ket{x_1}\ket{x_2}\ldots\ket{x_{m-1}}\ket{x_1+\ldots+x_m \bmod 2^{\eta}}
\end{equation}
%
Having complained about the cost of modular division, how can we now
implement the operators $\Upsilon_{\eta}(A)$ and $W$ more efficiently on our
hybrid nearest-neighbor architecture?