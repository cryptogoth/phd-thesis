\section{Quantum Compiling Background}
\label{sec:qcompile-bg}

Quantum compiling, like
classical compiling, translates arbitrarily complex high-level ``software''
operations to physically simple,
universal, machine-dependent ``assembly'' instructions.
Equivalently, it transforms quantum circuits from those that are
convenient to specify in theoretical algorithms 

\begin{figure}
\begin{center}
\begin{displaymath}
\begin{array}{ccc}

%%%%%%%%%%%%%%%%
% Source circuit
%\underbrace{
\begin{array}{c}
S = 2 \\
\Qcircuit @C=0.5em @R=.5em { 
	& \multigate{4}{U_1} & \qw & \multigate{4}{U_2} & \qw \\ 
	& \ghost{U_1}        & \qw & \ghost{U_2}        & \qw \\
	& \ghost{U_1}        & \qw & \ghost{U_2}        & \qw \\
	& \ghost{U_1}        & \qw & \ghost{U_2}        & \qw \\
	& \ghost{U_1}        & \qw & \ghost{U_2}        & \qw 
	\gategroup{1}{2}{5}{4}{.7em}{--}
}\\
\xymatrix {
  & D=2 \ar[l] \ar[r] & \\
 }
\end{array}
%}_{C}

%& 
%\begin{array}{c}
%\textsc{Quantum Compiler} \\
\rightarrow
%\end{array}
%&

%%%%%%%%%%%%%%%%
% Target circuit
%\underbrace{
\begin{array}{c}
S' = 15 \\
\Qcircuit @C=0.5em @R=.5em { 
	& \gate{H} & \qw & \ctrl{1} & \gate{H} & \qw & \qw      & \ctrl{1} & \qw \\ 
	& \gate{H} & \qw & \targfix & \ctrl{2} & \qw & \gate{K} & \targfix & \qw \\
	& \gate{H} & \qw & \gate{K} & \qw      & \qw & \gate{H} & \qw      & \qw \\
	& \gate{H} & \qw & \ctrl{1} & \targfix & \qw & \gate{H} & \qw      & \qw \\
	& \gate{H} & \qw & \targfix & \gate{H} & \qw & \qw      & \qw      & \qw
	\gategroup{1}{2}{5}{9}{.7em}{--}
}\\
\xymatrix {
  & & D'=5 \ar[ll] \ar[rr] & & \\
 }
\end{array}
%}_{C}

\end{array}
\end{displaymath}

\caption{A quantum compiler in action}
\label{fig:qcompile}
\end{center}
\end{figure}

Quantum compilers take a
generic unitary matrix $U$ on $n$-qubits, that is, an element of $SU(2^n)$, and
attempts to approximate it with a sequence of gates
(that is, a product of some matrices $V_1\cdots V_k$ )
from some universal set using a distance metric.

One distance metric used in theoretical literature
is the operator norm of a matrix $M$, $|| M || = || M ||_{\infty}$,
is defined as the maximum amount it scales the vector norm
of all unit-length vectors. This is sometimes also called the
infinity-norm, or supremum-norm (sup-norm).

\begin{equation}
|| M || = \max_{|| \ket{v} || = 1} || M \ket{v} ||
\end{equation}

\begin{equation}
|| U - V_1\cdots V_k || < \epsilon
\end{equation}

However, this is not an operational definition.
Moreover, we often wish to neglect a global phase in a unitary matrix,
which is not measurable in quantum physics. This is equivalent to
defining the set of valid $n$-qubit quantum gates as the
group $SU(2^n) = U(2^n) \ U(1)$. To measure phase-independent
distance between two unitary matrices, we can use the following
distance measure due to Fowler \cite{Fowler2011}.

\begin{equation}
dist(U, V) = \sqrt{\frac{2^n - |tr(U^{\dag}V)|}{2^n}}
\end{equation}

Quantum compilers are generally classical algorithms that run on a
digital computer to produce a deterministic sequence of quantum gates.
They accept as
input a classical description of a quantum circuit, and their output
(called a \emph{compiled sequence} is
taken from 
the universal, fault-tolerant gate set which depends on a particular
physical quantum computer technology.
an input quantum circuit. The compiled sequence then runs on a quantum
computer to enact your original quantum circuit in the machine-dependent
``assembly language'' of your quantum computer. We can measure the
efficiency of a quantum compiler by the resources it consumes, which are
detailed below.

\begin{description}
\item[classical runtime:] the classical time it takes to return a 
compiled quantum circuit.
\item[compiled depth:] the compiled quantum circuit depth
\item[compiled size:] the compiled quantum circuit size, which is
identical to compiled depth if no ancillae are used (compiled width is zero).
\item[compiled width:] the compiled quantum circuit width, which includes
the width of the input circuit as well as any ancillae introduced by
the compiler.
\end{description}

The last three resources are quantum in nature. They include any

Quantum compilers in the literature are often divided up along
the following three axes.

\begin{description}
\item[single-qubit versus multi-qubit axis]
\item[exact sequence versus approximate sequence]
\item[deterministic protocols versus probabilistic protocols]
\item[provably optimal versus conjectured optimal with empirical verification]
\end{description}

These axes are described below.

%%%%%%%%%%%
\subsection{Single-Qubit Quantum Compiling}

Restricting ourselves to the simplest case of
single-qubit circuits allows us to exploit a lot of structure
in the group $SU(2)$ in order to make our compilers more
efficient. This is an example of modularity which allows us
to divide the effort of quantum compiling between the
single-qubit case and then decomposition to a single-qubit
and two-qubit gates. This is a heuristic which often results
in simple decompositions to implement in (classical) software.
However, it may not be optimal compared to generic 
multi-qubit protocols.

The single-qubit case often concentrate on enacting arbitrary
phase rotations about the $Z$-axis, discretized by a resolution of
$2^n$. The multiple $\ell \in \mathbb{Z}_{2^n}$ indicates the
closest approximation of an arbitrary angle $\phi \in [0,2\pi)$.
There is some error in representing an angle as a multiple of
$2\pi$ with a finite ($n)$ number of bits, which we upper-bound by
$\frac{1}{2^{n+1}}$, where ties are broken arbitrarily.

\begin{equation}
| \frac{2\pi \ell}{2^n} - \frac{\phi} | \le \frac{1}{2^{n+1}}
\end{equation}

These rotations are denoted as $R_Z(\phi)$, or in the case
of their discrete approximation, $R_Z(\frac{\pi \ell}{2^{n-1}})$.

\begin{equation}
 R_Z(\phi) = 
 \left[
  \begin{array}{cc}
    1 & 0 \\
    0 & e^{i\phi} \\
  \end{array} \right]
\end{equation}

Note that the Clifford gates Pauli $Z$ gate and the phase gate $S$,
as well as the non-Clifford gate $T$ are special cases of $R_Z(\phi)$
are shown below.

\begin{equation}
Z =  \left[
  \begin{array}{cc}
    1 & 0 \\
    0 & e^{i\pi} \\
  \end{array} \right]
\qquad
S =  \left[
  \begin{array}{cc}
    1 & 0 \\
    0 & e^{i\frac{\pi}{2}} \\
  \end{array} \right]
\qquad
T =  \left[
  \begin{array}{cc}
    1 & 0 \\
    0 & e^{i\frac{\pi}{4}} \\
  \end{array} \right]
\end{equation}


Once we have handled the single-qubit case.

%%%%%%%%%%%
\subsection{Decomposition to Single-Qubit and Two-Qubit Gates}

Now that we have handled the single-qubit case, how can we leverage this
to compile general $n$-qubit gates? The usual reduction uses
a two-level decomposition, as shown in several places.
We need to fact-check this.
\cite{Kitaev2002}

The optimal bound needed for this 

We can
also characterize the quantum compiler on the properties that we
have discussed above.which are
detailed below.



They accept an input circuit and produce a compiled
output circuit that approximates it, with some increase in circuit size,
width, and depth.

Compilation is therefore
one area where one can generically decrease circuit depth, without
regard to the particular algorithm being compiled.
%, independently of the
%algorithm.
The first efficient quantum compiler due to
Solovay and Kitaev (SK) had a depth overhead of $O(\log^3+\nu(n/\epsilon))$,
where $\nu$ is a constant which can be made arbitrarily small, and no
ancillae (the width was $O(n)$) \cite{Dawson2005}.
A more parallel compiling procedure was discovered by Kitaev, Shen, and
Vyalyi (KSV) with a depth overhead of $O(\log n + \log\log(1/\epsilon))$ but
with a width of $O(n^3)$ on an $\textsc{AC}$ architecture \cite{Kitaev2002}.
Numerical comparisons between SK and KSV can be found in Ref. \cite{Pham2012a}.
Moreover, the KSV procedure relies on a prior decomposition of
a unitary matrix in $SU(N=2^n)$ into (controlled) single-qubit gates,
for example the multiply-controlled single-qubit gates in
the palindrome transform of Aho and Svore \cite{Aho2003}.

Austin Fowler's compiler uses a brute force search over group elements of
$\mathcal{C}_1$ alternating with $R_Z(\pi/8)$ which still runs in time
exponential in sequence length.
This may be optimized using classical parallelization and hardware acceleration
\cite{Booth2012}, possibly also for scaling this approach to
two-qubit compilation with $\mathcal{C}_2$ (containing 11,520
elements).