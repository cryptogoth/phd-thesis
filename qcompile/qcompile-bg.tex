\section{Quantum Compiling Background}
\label{sec:qcompile-bg}

Quantum compiling, like
classical compiling, translates arbitrarily complex high-level ``software''
operations to physically simple,
universal, machine-dependent ``assembly'' instructions.
Equivalently, it transforms quantum circuits from those that are
convenient to specify in theoretical algorithms 

\begin{figure}
\begin{center}
\begin{displaymath}
\begin{array}{ccc}

%%%%%%%%%%%%%%%%
% Source circuit
%\underbrace{
\begin{array}{c}
S = 2 \\
\Qcircuit @C=0.5em @R=.5em { 
	& \multigate{4}{U_1} & \qw & \multigate{4}{U_2} & \qw \\ 
	& \ghost{U_1}        & \qw & \ghost{U_2}        & \qw \\
	& \ghost{U_1}        & \qw & \ghost{U_2}        & \qw \\
	& \ghost{U_1}        & \qw & \ghost{U_2}        & \qw \\
	& \ghost{U_1}        & \qw & \ghost{U_2}        & \qw 
	\gategroup{1}{2}{5}{4}{.7em}{--}
}\\
\xymatrix {
  & D=2 \ar[l] \ar[r] & \\
 }
\end{array}
%}_{C}

%& 
%\begin{array}{c}
%\textsc{Quantum Compiler} \\
\rightarrow
%\end{array}
%&

%%%%%%%%%%%%%%%%
% Target circuit
%\underbrace{
\begin{array}{c}
S' = 15 \\
\Qcircuit @C=0.5em @R=.5em { 
	& \gate{H} & \qw & \ctrl{1} & \gate{H} & \qw & \qw      & \ctrl{1} & \qw \\ 
	& \gate{H} & \qw & \targfix & \ctrl{2} & \qw & \gate{K} & \targfix & \qw \\
	& \gate{H} & \qw & \gate{K} & \qw      & \qw & \gate{H} & \qw      & \qw \\
	& \gate{H} & \qw & \ctrl{1} & \targfix & \qw & \gate{H} & \qw      & \qw \\
	& \gate{H} & \qw & \targfix & \gate{H} & \qw & \qw      & \qw      & \qw
	\gategroup{1}{2}{5}{9}{.7em}{--}
}\\
\xymatrix {
  & & D'=5 \ar[ll] \ar[rr] & & \\
 }
\end{array}
%}_{C}

\end{array}
\end{displaymath}

\caption{A quantum compiler in action}
\label{fig:qcompile}
\end{center}
\end{figure}

Quantum compilers take a
generic unitary matrix $U$ on $n$-qubits, that is, an element of $SU(2^n)$, and
attempts to approximate it with a sequence of gates
(that is, a product of some matrices $V_1\cdots V_k$ )
from some universal set using a distance metric.

One distance metric used in theoretical literature
is the operator norm of a matrix $M$, $|| M || = || M ||_{\infty}$,
is defined as the maximum amount it scales the vector norm
of all unit-length vectors. This is sometimes also called the
infinity-norm, or supremum-norm (sup-norm).

\begin{equation}
|| M || = \max_{|| \ket{v} || = 1} || M \ket{v} ||
\end{equation}

\begin{equation}
|| U - V_1\cdots V_k || < \epsilon
\end{equation}

However, this is not an operational definition.
Moreover, we often wish to neglect a global phase in a unitary matrix,
which is not measurable in quantum physics. This is equivalent to
defining the set of valid $n$-qubit quantum gates as the
group $SU(2^n) = U(2^n) \ U(1)$. To measure phase-independent
distance between two unitary matrices, we can use the following
distance measure due to Fowler \cite{Fowler2011}.

\begin{equation}
dist(U, V) = \sqrt{\frac{2^n - |tr(U^{\dag}V)|}{2^n}}
\end{equation}

Quantum compilers are generally classical algorithms that run on a
digital computer to produce a deterministic sequence of quantum gates.
They accept as
input a classical description of a quantum circuit, and their output
(called a \emph{compiled sequence} is
taken from 
the universal, fault-tolerant gate set which depends on a particular
physical quantum computer technology.
an input quantum circuit. The compiled sequence then runs on a quantum
computer to enact your original quantum circuit in the machine-dependent
``assembly language'' of your quantum computer. We can measure the
efficiency of a quantum compiler by the resources it consumes, which are
detailed below.

\begin{description}
\item[classical runtime:] the classical time it takes to return a 
compiled quantum circuit.
\item[compiled depth:] the compiled quantum circuit depth
\item[compiled size:] the compiled quantum circuit size, which is
identical to compiled depth if no ancillae are used (compiled width is zero).
\item[compiled width:] the compiled quantum circuit width, which includes
the width of the input circuit as well as any ancillae introduced by
the compiler.
\end{description}

The last three resources are quantum in nature. They include any

Quantum compilers in the literature are often divided up along
the following three axes.

\begin{description}
\item[single-qubit versus multi-qubit axis]
\item[exact sequence versus approximate sequence]
\item[deterministic protocols versus probabilistic protocols]
\item[provably optimal versus conjectured optimal with empirical verification]
\end{description}

These axes are described below.

%%%%%%%%%%%
\subsection{Single-Qubit Quantum Compiling}

Restricting ourselves to the simplest case of
single-qubit circuits allows us to exploit a lot of structure
in the group $SU(2)$ in order to make our compilers more
efficient. This is an example of modularity which allows us
to divide the effort of quantum compiling between the
single-qubit case and then decomposition to a single-qubit
and two-qubit gates. This is a heuristic which often results
in simple decompositions to implement in (classical) software.
However, it may not be optimal compared to generic 
multi-qubit protocols.

The single-qubit case often concentrate on enacting arbitrary
phase rotations about the $Z$-axis, discretized by a resolution of
$2^n$. The multiple $\ell \in \mathbb{Z}_{2^n}$ indicates the
closest approximation of an arbitrary angle $\phi \in [0,2\pi)$.
There is some error in representing an angle as a multiple of
$2\pi$ with a finite ($n)$ number of bits, which we upper-bound by
$\frac{1}{2^{n+1}}$, where ties are broken arbitrarily.

\begin{equation}
| \frac{2\pi \ell}{2^n} - \frac{\phi} | \le \frac{1}{2^{n+1}}
\end{equation}

These rotations are denoted as $R_Z(\phi)$, or in the case
of their discrete approximation, $R_Z(\frac{\pi \ell}{2^{n-1}})$.

\begin{equation}
 R_Z(\phi) = 
 \left[
  \begin{array}{cc}
    1 & 0 \\
    0 & e^{i\phi} \\
  \end{array} \right]
\end{equation}

Note that the Clifford gates Pauli $Z$ gate and the phase gate $S$,
as well as the non-Clifford gate $T$ are special cases of $R_Z(\phi)$
are shown below.

\begin{equation}
Z =  \left[
  \begin{array}{cc}
    1 & 0 \\
    0 & e^{i\pi} \\
  \end{array} \right]
\qquad
S =  \left[
  \begin{array}{cc}
    1 & 0 \\
    0 & e^{i\frac{\pi}{2}} \\
  \end{array} \right]
\qquad
T =  \left[
  \begin{array}{cc}
    1 & 0 \\
    0 & e^{i\frac{\pi}{4}} \\
  \end{array} \right]
\end{equation}


Once we have handled the single-qubit case.

%%%%%%%%%%%
\subsection{Decomposition to Single-Qubit and Two-Qubit Gates}

Now that we have handled the single-qubit case, how can we leverage this
to compile general $n$-qubit gates? The usual reduction uses
a two-level decomposition, as shown in several places.
We need to fact-check this.
\cite{Kitaev2002}

The optimal bound needed for this 

We can
also characterize the quantum compiler on the properties that we
have discussed above.which are
detailed below.



They accept an input circuit and produce a compiled
output circuit that approximates it, with some increase in circuit size,
width, and depth.

Compilation is therefore
one area where one can generically decrease circuit depth, without
regard to the particular algorithm being compiled.
%, independently of the
%algorithm.
The first efficient quantum compiler due to
Solovay and Kitaev (SK) had a depth overhead of $O(\log^3+\nu(n/\epsilon))$,
where $\nu$ is a constant which can be made arbitrarily small, and no
ancillae (the width was $O(n)$) \cite{Dawson2005}.
A more parallel compiling procedure was discovered by Kitaev, Shen, and
Vyalyi (KSV) with a depth overhead of $O(\log n + \log\log(1/\epsilon))$ but
with a width of $O(n^3)$ on an $\textsc{AC}$ architecture \cite{Kitaev2002}.
Numerical comparisons between SK and KSV can be found in Ref. \cite{Pham2012a}.
Moreover, the KSV procedure relies on a prior decomposition of
a unitary matrix in $SU(N=2^n)$ into (controlled) single-qubit gates,
for example the multiply-controlled single-qubit gates in
the palindrome transform of Aho and Svore \cite{Aho2003}.

Austin Fowler's compiler uses a brute force search over group elements of
$\mathcal{C}_1$ alternating with $R_Z(\pi/8)$ which still runs in time
exponential in sequence length.
This may be optimized using classical parallelization and hardware acceleration
\cite{Booth2012}, possibly also for scaling this approach to
two-qubit compilation with $\mathcal{C}_2$ (containing 11,520
elements).

%%%%%%%%%%%%%%%%%%%%%%%%%%%%%%%%%%%%%%%%%%%%%%%%%%%%%%%%%%%%%%%%%%%%%%%%%%%%%%
\subsection{Some Special Operator Notation}

Borrowing the notation in \cite{ksv02},
we define two controlled ``meta-operators'' which takes some unitary $U$ as
a parameter. The first describes a controlled-$U$ operation where the control
is a single qubit.

\begin{displaymath}
\Lambda(U) = \ket{0}\bra{0} \otimes I + \ket{1}\bra{1} \otimes U
\end{displaymath}

The second describes a registered-$U$ operation, which
can be thought of as applying $U^p$ to some target register
controlled on a separate $m$-qubit register
encoding the number $p$.

\begin{displaymath}
\Upsilon_m(U) : \ket{p} \otimes \ket{\psi} \rightarrow \ket{p}
\otimes U^p\ket{\psi}
\end{displaymath}

Note that $\Upsilon_1(U)$ and $\Lambda(U)$ are equivalent.

%%%%%%%%%%%%%%%%%%%%%%%%%%%%%%%%%%%%%%%%%%%%%%%%%%%%%%%%%%%%%%%%%%%%%%%%%%%%%%
\subsection{A Universal Set of Gates}

We use the following universal standard set of gates $\mathcal{G}$ throughout
this paper.

\begin{displaymath}
\mathcal{G} = \{ H, K, K^{\dagger}, X, Z,
\Lambda(\sigma_x), \Lambda^2(\sigma_x) \}
\end{displaymath}

These gates all operate on single qubits with the
exception of $\Lambda(\sigma_x)$ (the two-qubit CNOT gate)
and $\Lambda^2(\sigma_x)$ (the three-qubit Toffoli gate),
which can be interpreted as singly- and doubly-controlled
$X$ gates, respectively. $X$ and $Z$ are the standard Pauli matrices
$\sigma_x$ and $\sigma_z$, $H$ is the Hadamard matrix, and $K$ and its
Hermitian conjugate $K^{\dagger}$ are phase gates of $i$ and $-i$,
respectively.

\begin{displaymath}
Z = 
 \left[
  \begin{array}{cc}
    1 & 0 \\
    0 & -1 \\
  \end{array} \right]
\qquad
X = 
 \left[
  \begin{array}{cc}
    0 & 1 \\
    1 & 0 \\
  \end{array} \right]
\end{displaymath}

\begin{displaymath}
K = 
 \left[
  \begin{array}{cc}
    1 & 0 \\
    0 & i \\
  \end{array} \right]
\qquad
K^{\dagger} = 
 \left[
  \begin{array}{cc}
    1 & 0 \\
    0 & -i \\
  \end{array} \right]
\end{displaymath}

Without proving the universality of $\mathcal{G}$, we note that there are
methods for expressing any $2^n \times 2^n$ unitary matrix as a
tensor product of single- and two-qubit gates \cite{Bremner2002}. We call
the process of decomposing elements of $SU(N)$ into elements of
$SU(2)$ and $SU(4)$ \emph{quantum circuit synthesis}
[citation to Markov]
to distinguish them as
an intermediate stage in quantum compiling, for which procedures like SK and
KSV provide the final stage of producing circuits formed from $\mathcal{G}$.

%%%%%%%%%%%%%%%%%%%%%%%%%%%%%%%%%%%%%%%%%%%%%%%%%%%%%%%%%%%%%%%%%%%%%%%%%%%%%%
\subsection{Parameters}

The problem of quantum compiling is to translate
an entire circuit $C$ of $S$ gates with depth
$D$ to a new, compiled circuit $C'$ of size $S'$ and depth $D'$ which approximates
$C$ within error $\epsilon$ using some distance measure.

%%%%%%%%%%%%%%%%%%%%%%%%%%%%%%%%%%%%%%%%%%%%%%%%%%%%%%%%%%%%%%%%%%%%%%%%%%%%%%
%%%%%%%%%%%%%%%%%%%%%%%%%%%%%%%%%%%%%%%%%%%%%%%%%%%%%%%%%%%%%%%%%%%%%%%%%%%%%%
% Figure out what you are actually measuring / calculating in your code
$S$ is measured in the number of gates from $\mathcal{G}$. $D$ is measured in
number of concurrent timesteps of ... $W$ is measured in total number of 
ancillae qubits needed to maintain a coherent state for the duration of the
computation.

In our code, we use the trace measure introduced by Austin Fowler which disregards
the global phase factor. Here,
$d$ refers to the dimensionality
of our system ($d = 2^n$ for $n$ qubits).

\begin{equation}
d(U,\tilde{U}) = \sqrt{\frac{d - \norm{\mathrm{tr}(U^\dagger \tilde{U})}}{d}}
\end{equation}

We will use the terms ``error'', ``precision'', and
``accuracy'' interchangeably when approximating gates.
There is some overhead in the compiled circuit, so in
general $C'$ is larger (that is, $S' > S$ and $D' > D$). It's also known that
in order to approximate a circuit with $S$ gates to a total precision of
$\epsilon$
requires each gate to be approximated to a precision of
$n = O(\log(S/\epsilon)$ \cite{Lloyd1995}. We denote this per-gate precision
$n$, since it serves as an independent parameter for compiling. We will clarify
where there is confusion with using $n$ to denote the number of qubits in a
register or being acted upon by a circuit.
We denote at $T$ the classical preprocessing time to
produce $C'$.
 
Circuit depth is analogous to running time, or how long we have to wait from
feeding in inputs to getting a correct output. Relative to the circuit size,
it is a heuristic for how parallelizable our
circuit is. For example, in a single ion trap, if we had multiple lasers,
we could ``flatten'' our circuit into layers with bounded fan-in and
fan-out and operate on multiple ions in parallel.
We could also operate multiple ion traps in parallel which communicate by
teleportation.
All other things being equal, a circuit with low depth will complete
faster than one with high depth, although in practice we can only execute
fixed-width circuits.

%%%%%%%%%%%%%%%%%%%%%%%%%%%%%%%%%%%%%%%%%%%%%%%%%%%%%%%%%%%%%%%%%%%%%%%%%%%%%%
\subsection{Asymptotic Circuit Bounds}

The SK and KSV algorithms compile circuits with
a size and depth which depend on the desired per-gate precision $n$.
These asymptotic bounds are listed in
Table \ref{tab:asymptotics}, which actually refer to an improved version of
SK described in Theorem 8.5 of \cite{ksv02}, where 
$\nu$ is a small positive constant.
However, we describe a simpler version of SK in Section \ref{sec:sk-algo}
which has a fixed exponent of about $c=3.97$, which is easier to understand,
and does not significantly change the asymptotic relationship between SK and
KSV.

We will see these asymptotic bounds reflected
later in the actual numerical results in Section \ref{sec:results}.

\begin{center}
\begin{table}
\label{tab:asymptotics}
\begin{tabular}{|c|c|c|}
\hline
   & SK & KSV \\
\hline
$S'$ & $O(Sn^{3+\nu})$ & $O(Sn + n^2 \log n)$\\
$D'$ & $O(Dn^{3+\nu})$ & $O(D \log{n} + (\log{n})^2))$\\ 
$W'$ & $0$             & $$ \\
\hline
\end{tabular}
\caption{Asymptotic circuit resources for SK and KSV compilers}
\end{table}
\end{center}