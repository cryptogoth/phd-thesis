\section{Circuit Resources for the KSV Quantum Compiler}
\label{sec:qcompile-ksv}

This section gives a pedagogical review of the quantum compiling method
that combines phase kickback and QFS generation due to
Kitaev-Shen-Vyalyi \cite{Kitaev2002}. In this form, it compiles single-qubit
rotations of the form $R_Z(\phi)$. Using a multi-qubit
decomposition such as those presented in Section \ref{subsec:qcompile-multi},
it can be extended to compile multi-qubit gates.
Furthermore, we present an
original optimization
called \emph{early measurement} which does not asymptotically increase
our compilation resources (see Appendix \ref{app:ksv-error}). We refer to
our optimized method as PK-KSVe, which still shares much in common with
PK-KSV up to the phase estimation step.
Finally, we contribute the architectural resources on $\textsf{2D CCNTCM}$ for running 
three different QFS generation procedures

First, in Section \ref{subsec:qfs-adder} we contribute depth-efficient
adders for any version of phase kickback.
Then in Section \ref{subsec:ksv-diff} we discuss our optimization of early
measurement, which distinguishes
PK-KSVe from the original PK-KSV algorithm.
Next,
in Section \ref{subsec:ksv-steps} we present the high-level overview
of the KSV algorithm. In Section \ref{subsec:ppe} the most important
and most resource-intensive step in PK-KSV and PK-KSVe.
Section \ref{subsec:ksv-classical} describes the classical post-processing
which completes phase estimation. While this is done on a classical computer,
we will discuss how this step affects our parameters and resources
for generating a quantum circuit. We also discuss a new, faster
phase estimation algorithm \cite{Svore2013}. Finally, we map the QFS
stage of PK-KSV,
PK-KSVe, and PK-Jones to \textsf{2D CCNTCM} and compare their
asymptotic resource usage in Section \ref{subsec:ksv-compare}.

%%%%%%%%%%%%%%%%%%%%%%%%%%%%%%%%%%%%%%%%%%%%%%%%%%%%%%%%%%%%%%%%%%%%%%%%%%%%%%
\subsection{Circuit Resources for Adders on Quantum Fourier States}
\label{subsec:qfs-adder}

We need two kinds of adders, both mapped to \textsf{2D CCNTCM} in order to
create an $\eta$-qubit QFS.
The first adder is a version of the QCLA from Section \ref{sec:modexp}
on multiple modules instead of a single module. It sums two
conventional numbers into one conventional number modulo $2^{\eta}$;
call this adder $F$.
%
\begin{equation}
F\ket{a}\ket{b} \rightarrow \ket{a}\ket{a+b \bmod 2^{\eta}}
\end{equation}
%
The second adder is an optimized version of modular multipler addition
from
Chapter \ref{chap:factor-polylog},
from Sections \ref{sec:csa-mod-add} and \ref{subsec:mma}.
After modular multiple addition in carry-save encoding,
it uses the first adder $F$ to combine the last two numbers
into a conventional number. Then it uncomputes the intermediate stages.
You will recognize this adder as $W$ from the previous section.

First we prove a general result for mapping any \textsf{AC} circuit
to a \textsf{2D CCNTCM} circuit.
Then we use it to construct architectures for $F$ and $W$.

\begin{lemma}
Suppose an \textsf{AC} circuit $\mathcal{C}$ has
depth $D(\eta)$, size $S(\eta)$, and width $W(\eta)$. Then it can be mapped
onto a \textsf{2D CCNTCM} circuit $\mathcal{C'}$ with circuit depth
$D(\eta)$, circuit size $S(\eta)$, circuit width $D(\eta)\cdot W(\eta)$,
module depth $D(\eta)$, module size $D(\eta)\cdot W(\eta)$, and module width
$D(\eta)$.
\label{lem:ac-ccntcm}
\end{lemma}

\begin{proof}
Each of $D(\eta)$ layers in $\mathcal{C}$ could operate on non-nearest-neighbors.
We teleport all
$W(\eta)$ qubits from the first module to the last module in sequence,
reordering them at each module so that we can execute the gates of that
layer only on nearest-neighbor qubits.
This gives the desired circuit and module resources on
\textsf{2D CCNTCM}.
\end{proof}

\begin{lemma}
The operator $F$ on
two $\eta$-qubit registers can be
implemented in circuit/module depth $O(\log \eta)$, circuit size $O(\eta)$, 
circuit width $O(\eta\log \eta)$, module size of $O(\eta \log \eta)$, and
module width of $O(\log \eta)$ on
\textsf{2D CCNTCM}.
\label{lem:f}
\end{lemma}

\begin{proof}
We modify the QCLA adder from
Section \ref{subsec:qcla} by mapping it to multiple \textsf{2D CCNTCM} modules
using
Lemma \ref{lem:ac-ccntcm}.
This is an alternative to the version in Section \ref{subsec:qcla} which
contained the entire adder in a single module.
To add $2 \times \eta$-bit numbers, the original QCLA on \textsf{AC} had
depth $O(\log \eta)$ and size and width of $O(\eta)$ \cite{Draper2004}.
Therefore, we have the desired resources for a \textsf{2D CCNTCM} circuit.
\end{proof}

\begin{lemma}
The operator $W\ket{x_1}\ket{x_2}\cdots\ket{x_{m-1}}\ket{x_m} \rightarrow
\ket{x_1}\ket{x_2}\cdots\ket{x_{m-1}}\ket{x_1 + x_2 + \ldots + x_m \bmod 2^{\eta}}$,
which operates on $m \times \eta$-qubit registers, can be
implemented in circuit/module depth $O(\log nm)$, circuit size $O(m\eta)$, 
circuit width $O(m\eta + \eta\log \eta)$, module size of $O(m\eta + \eta \log \eta)$, and
module width of $O(m + \log \eta)$ on
\textsf{2D CCNTCM}.
\label{lem:w}
\end{lemma}

\begin{proof}
This reduces to the problem of modular multiple addition from
Section \ref{subsec:mma}, with the following variation. Instead of
addition modulo $m < 2^{\eta}$, we perform addition modulo $2^{\eta}$
exactly, which is
much simpler. It does not involve any truncation and adding back of
modular residues. Namely, we just perform one round of constant-depth
$3 \rightarrow 2$ addition, then we redo the computation of the
highest-order bit $v_n$ to uncompute it. Call this simpler
adder $J$:
%
\begin{equation}
J\ket{a}\ket{b} \rightarrow \ket{a}\ket{a+b \bmod 2^n}
\end{equation}
%
Therefore, we have a $O(\log m)$-depth binary tree of $O(m)$ modules,
each containing a $J$ adder,
which requires a total of $O(mn)$ teleportation of qubits between them.
Within each module, the addition takes depth $O(1)$ and size and width
of $O(\eta)$. Combine this in series with the circuit for $A$ in
Lemma \ref{lem:f} and we have the desired resource bounds.
\end{proof}

In fact, we also provide numerical upper bounds for circuit resources
of both $A$ and $W$ in Table \ref{tab:qfs-adder}.

\begin{table}[hbt!]
\begin{center}
\begin{tabular}{|c|c|c|c|}
\hline
Operation & $D$ & $S$ & $W$ \\
\hline
$J$ from Lemma \ref{lem:w} & $66$ & $56\eta$ & $5\eta$ \\
$F$ from Lemma \ref{lem:f} & $4\log_2 \eta + 6$ & $10\eta - 48\log_2 \eta - 78$ & $8\eta \log_2 \eta + 12\eta$ \\
$W$ from Lemma \ref{lem:w} & $57\log_2 m$ & $100\eta m - 200\eta$ & $5\eta m - 10\eta$ \\
\hline
                           & $\overline{D}$ & $\overline{S}$ & $\overline{W}$ \\
\hline
 $J$                       & 0 & 0 & 1 \\
 $F$                          & $4\log_2 \eta + 2$ & $12\eta \log_2 \eta + 4\eta$ & $4\log_2 \eta + 2$  \\
 $W$                          & $1.71\log_2 m$ & $2\eta m + 3m - 4\eta - 6$ & $m-2$\\
\hline
\end{tabular}
\caption[Circuit resources for adders used in generating an $\eta$ QFS generation on \textsf{2D CCNTCM}.]{Circuit resources for adders used in generating an $\eta$ QFS generation on \textsf{2D CCNTCM}.
Resources for $W$ are for adding $m\times \eta$-bit numbers.}
\label{tab:qfs-adder}
\end{center}
\end{table}

Armed with these properties, we're now ready to describe our version of the
KSV phase estimation procedure to produce quantum Fourier states.

%%%%%%%%%%%%%%%%%%%%%%%%%%%%%%%%%%%%%%%%%%%%%%%%%%%%%%%%%%%%%%%%%%%%%%%%%%%%%%
\subsection{The Optimization of Early Measurement}
\label{subsec:ksv-diff}

In this section, we describe the differences between PK-KSV, as originally
presented in Section 13 of \cite{Kitaev2002}, and PK-KSVe, which includes our
novel contribution of early measurement.

Given a circuit $C$ to compile, we
precompile it into gates from $\mathcal{G} \cup \{R_Z(2\pi x / 2^{\eta})\}$
using the results from Sections \ref{subsec:qcompile-single} and
in $R = O(1)$ classical time and $D'=S'=O(1)$ compiled depth and size.
Now we are done with the single-qubit gates and CNOT, and we have computed
the values $\{x_1, \ldots , x_m\}$ that allow us to approximate our
desired $m\times R_Z(\phi)$ gates as $\phi \approx \frac{\pi x_i}{2^{\eta}}$
to within precision
$2^{-(\eta+2)}$. We now need to generate the QFS $\ket{\psi_{\eta}^{(k)}}$ to use with
phase kickback to either approximate $R_Z(\phi)$ directly, or to generate
PAR qubits as intermediaries for later enacting $R_Z(\phi)$ probabilistically.

The original PK-KSV procedure first created the state $\ket{\psi_{\eta}^{(0)}}$
and applied phase estimation to it to enact the operator
$\Upsilon(e^{-2\pi i / 2^\eta})$
%
\begin{equation}
\Upsilon(e^{-2i \pi / 2^{\eta}})\ket{x} \rightarrow e^{-2i\pi x / 2^{\eta}}\ket{x}
\end{equation}
%
and get $\ket{\psi_{\eta}^{(1)}}$ as a result. This state can be copied using
$W$, multiple addition modulo $2^{\eta}$, from Lemma \ref{lem:w}.
However, it involves using coherent measurement
in order to enact a phase $\ket{x} \rightarrow e^{2\pi i x / 2^{\eta}}\ket{x}$ on
all the components of $\ket{\psi_{\eta}^{(0)}} = \sum_{x = 0}^{2^{\eta-1}} \ket{x}$.

The PK-KSVe does not enact the operator $\Upsilon(e^{-2\pi i / 2^{\eta}})$ but instead
yields a QFS $\ket{\psi_{\eta}^{(k)}}$ for random odd $k$, as well as the index $k$ itself.
This can be used to solve the modular inverse equation classically:
%
\begin{equation}
kp \equiv x \bmod 2^{\eta}
\end{equation}
%
to get the solution $p = p(k,x)$.
 This integer 
$p$ is added to $\ket{\psi_{\eta}^{(k)}}$, controlled on some qubit $\ket{\phi}$
to get the desired phase shift: $R_Z(2\pi x / 2^{\eta})\ket{\phi}$.
To copy $\ket{\psi_{\eta}^{(k)}}$, consider the controlled addition operator $\Upsilon(A)_n$
on two ${\eta}$-qubit registers.

\begin{equation}
\Upsilon(A)_n \ket{x}\ket{y} \rightarrow \ket{x}\ket{y + x \bmod 2^{\eta}}
\end{equation}

Applying this operator to two QFS's has the following interesting property
as noted in \cite{Jones2013}.

\begin{equation}
\Upsilon(A)_n \ket{\psi_{\eta}^{(k')}}\ket{\psi_{\eta}^{(k)}} \rightarrow \ket{\psi_{\eta}^{(k'-k)}}\ket{\psi_{\eta}^{(k)}}
\end{equation}

Therefore, we can copy a state $\ket{\psi_{\eta}^{(k)}}$ in the second register by
putting $\ket{\psi_{\eta}^{(0)}}$ in the first register to get
$\ket{\psi_{\eta}^{(-k)}} = \ket{\psi_{\eta}^{(2^{\eta} - k)}}$.
This is not exactly the same state as $\ket{\psi_{\eta}^{(k)}}$,
but it can similarly be used
with solving modular inverse to get an $x'$ and enact a desired rotation $R_Z(2\pi x' / 2^{\eta})$.

%%%%%%%%%%%%%%%%%%%%%%%%%%%%%%%%%%%%%%%%%%%%%%%%%%%%%%%%%%%%%%%%%%%%%%%%%%%%%%
\subsection{PK-KSVe Steps}
\label{subsec:ksv-steps}

Here now are the main steps of PK-KSVe.

The first stage involves QFS generation.

\begin{enumerate}
\item Create the equal superposition of all $\ket{\psi_{\eta}^{(k)}}$ for odd $k$:
\begin{equation}
\ket{\nu} = \normtwo \left( \ket{0} - \ket{2^{{\eta}-1}} \right)
\end{equation}
in register 1.
\item Use parallel phase estimation with error $\epsilon = 2^{-3\eta}$ and early
measurement to get classical Bernoulli series outcomes $y$ in register 2
and a corresponding QFS $\ket{\psi_{\eta}^{(k)}}$. This takes a circuit of
depth $O(\log {\eta})$ and size $O({\eta}^2)$ (see Section \ref{subsec:ppe}).
\item Perform classical post-processing from Section \ref{subsec:ksv-classical} to
recover the phase $k$ by which can identify the state $\ket{\psi_{\eta}^{(k)}}$.
\end{enumerate}

Now we have $\ket{\psi_{\eta}^{(k)}}$, or a state that has high overlap with it
(calculated in Section \ref{subsec:ksv-compare}).
We can copy it $m$ times using multiple addition $W$ in depth
$O(m \log {\eta})$ and size $O(m\eta)$. This is part of any complete
quantum compiler solution for PK-KSVe, but we omit it here and
focus on QFS generation.

%%%%%%%%%%%%%%%%%%%%%%%%%%%%%%%%%%%%%%%%%%%%%%%%%%%%%%%%%%%%%%%%%%%%%%%%%%%%%%
\subsection{Parallelized Phase Estimation}
\label{subsec:ppe}

Parallelized phase estimation is the central component of both
PK-KSV and PK-KSVe which distinguishes it from other,
non-phase-kickback quantum
compiling approaches. It is used to randomly ``pick'' an
eigenstate $\ket{\psi_{\eta}^{(k)}}$ of a unitary operator $U$ and
measure its corresponding eigenvalue (phase) $\phi_k$
with some degree of precision $\delta = 2^{-{\eta}}$ and
error probability $\epsilon = 2^{-l}$. Note that in this
case $\eta$ is the exponent of the precision we want from
our QFS, and $\epsilon$ denotes the error probability of
getting a phase estimate that does not correspond to the
QFS in the target register.
 As ${\eta}$ increases, the phases
generally become closer together, which is why we need exponential precision
to distinguish between them.
%
\begin{equation}
U\ket{\psi_k} = e^{2\pi i \phi_k} \ket{\psi_k}
\end{equation}
%
Phase estimation holds some superposition of eigenstates
$\sum_{i} \alpha_i \ket{\psi_i}$
in an ${\eta}$-qubit target register, to which it applies measuring
operators $\Lambda(U^{2^k})$
controlled on some $t$-qubit register, which holds an
approximation $\tilde{\phi}$ to the real phase $\phi$.
The unitary $U$ is applied in successive powers of two to get
power-of-two multiples of the phase for increased precision.
The error probability of approximating the phase to within a given
precision is given by the following:

\begin{displaymath}
\Pr\left[ | \phi - \tilde{\phi} | \ge \delta \right] \le \epsilon
\end{displaymath}

The parameter $t = t(\delta, \epsilon)$ encodes the dependence of the number
of $\Lambda(U)$ measuring operators as a function of our desired
$\delta$ and $\epsilon$.
It varies according to the exact phase estimation procedure
used.

The popular version of phase estimation presented in \cite{Nielsen2000}
requires $t$ repeated controlled applications of some unitary
$U$ (and its successive powers as $U^{2^k}$, $0<k<t$)
to a target state which holds some superposition of its eigenvectors,
controlled by $t$ bits which will hold the approximation to a corresponding
eigenvalue (phase).
This version requires applying an inverse quantum Fourier
transform (QFT).

The main steps in parallelized phase estimation as applied to PK-KSV
are as follows. As our most important input parameter, we calculate $t$, the number
of measurements that we need (and the number of control qubits that we have).
For QFS generation, it suffices to determine the phase with resolution $\frac{1}{2^{{\eta}+2}}$,
where each bit of the phase is determined with a series of Bernoulli (biased coin flips)
of $s=O(l + \log \eta)$ trials each. As in the original PK-KSV, we have two sets of measurements,
one to measure $\phi$ as a bias projected on the real axis of the unit circle (called
the cosine measurements), and one to perform the same measurements rotated by $\pi/2$
(called the sine measurements). These are illustrated below for these
measuring operators controlled on the first qubit and targeting the remaining
$\eta$ qubits with the operator whose phase we are trying to measure, $A$:
%
\begin{eqnarray}
\Xi(U) & = & (H\otimes I^{\otimes \eta})\Lambda(A)(H \otimes I^{\otimes \eta})\\
\Xi(iU) & = & (H\otimes I^{\otimes \eta})\Lambda(A)(SH \otimes I^{\otimes \eta})\\
\end{eqnarray}
%
\begin{equation}
\Pr(1|k,\Xi(iA)) = \frac{1 + \sin(2\pi\phi_k)}{2} \qquad \Pr(0|k,\Xi(A)) = \frac{1 + \cos(2\pi\phi_k)}{2}
\end{equation}
%
Therefore, we have:
%
\begin{equation}
t = 2({\eta}+2)s
\end{equation}
%
To achieve our desired low-depth, we can ``parallelize'' the application of
$\Lambda(U)$ by interpreting the
$t = ({\eta}+2)s$ control bits as $s\times O({\eta})$-bit numbers $q_i$,
adding them into a single $\eta$-bit number $\ket{q}$, and
applying $\Upsilon(A)\ket{q}$ only once.
Recall that $A$ is the addition operator on an ${\eta}$ qubit
target register containing $\ket{\psi_{\eta}^{(k)}}$, so we can
only effectively add the lowest ${\eta}$ bits of $q$.
Furthermore, the eigenvalues of $A$ are rational with a fixed
denominator, $\phi_k = k / 2^{\eta}$.
To avoid the inverse QFT, which often has small rotations of the form 
$R_Z(\phi)$, 
we can do a classical post-processing step as described in Section \ref{subsec:ksv-classical}.


\begin{enumerate}

\item Begin with a $t$-qubit ``phase'' register initialized to $\ket{0}^{\otimes t}$.
Recall that we have an ${\eta}$-qubit ``QFS'' register initialized to $\ket{\nu}$, the
equal superposition of $\ket{\psi_{\eta}^{(k)}}$ for all odd $0 < k  < 2^{\eta}$.
%\textsc{Resources} $= [0,0,0,0,0,t]$

\item Place the $t$-qubit register into an equal superposition by
applying ${\eta}$ Hadamard gates.
%\textsc{Resources} $= [0,0,0,{\eta},1,0]$

\item Treat $t$ as $2s$ groups of bits, each encoding an $({\eta}+2)$-bit number
which we call $\ket{q_i}$.
This does not involve any addition or other operation, it just determines
how we interpret its measurement outcome after we projectively measure it later.
%Sum them up out-of-place, retaining only the lowest ${\eta}$-bits,
%to get the superposition
%of all ${\eta}$-bit numbers, $1/(\sqrt{2^{{\eta}}}) \sum_{i=0}^{2^{\eta}-1} \ket{q_i}$.
%Call this register $\ket{q_i}$.
%\textsc{Resources} $= ADD-OUT(2s \times {\eta})$

\item Apply the gate $\Upsilon(A)_{{\eta}+2}$ to a target ancilla register $\ket{h_i}$ controlled
on $\ket{q_i}$, which we interpret as an $({\eta}+2)$-bit number. The register
$\ket{h_i}$ is initialized to all $\ket{0}$'s. This can be done in
parallel for all $0 \le i < 2s$ using constant-depth $3\rightarrow 2$ addition.
Now we have $2s$ quantum integers, plus
the original register $\ket{\nu}$, that we add down using modular multiple
addition in logarithmic depth, including the final addition of a CSE number
down to a conventional number using QCLA \cite{Draper2004}. We use the adders described
in Section \ref{subsec:qfs-adder} on \textsf{2D CCNTCM}, uncomputing all
other intermediate adder qubits back to $\ket{0}$
so that we are only left with $\ket{\nu}$, which we can also uncompute.
%\textsc{Resources} $= ADD-IN(2 \times {\eta})$

\item Reverse the second step by applying another ${\eta}$ Hadamards.
%\textsc{Resources} $= [0,0,0,{\eta},1,0]$

\item Projectively measure all $t$ control
qubits in the phase register. These qubits
now contain classical $0$ or $1$ as outcomes of $({\eta}+2)s$ Bernoulli trials.
The target $\ket{\nu}$ now contains a QFS $\ket{\psi_{\eta}^{(k)}}$ for some
as yet unknown $k$ with high fidelity.

\item Read out these outcomes into our classical controller
and perform the post-processing in Section \ref{subsec:ksv-classical}.
We get an approximation of $\phi$ with precision $\delta$ and
success probability $1-\epsilon$.

\end{enumerate}

%%%%%%%%%%%%%%%%%%%%%%%%%%%%%%%%%%%%%%%%%%%%%%%%%%%%%%%%%%%%%%%%%%%%%%%%%%%%%%
\subsection{Classical Postprocessing}
\label{subsec:ksv-classical}

It is now the point to mention that Kitaev's phase estimation procedure
contains a post processing step which is completely classical in
character, in that they involve a measurement. If this measurement is
projective and the outcomes are completely classical, the remaining steps
can be done on our classical computer,
and the results fed back into our quantum algorithm to perform controlled
addition (phase kickback). Therefore, as long as we can perform these classical
algorithms in polynomial time (which we can), we don't really care
about the equivalent circuit size and depth.

The steps of classical postprocessing, which will determine some of the
parameters in the earlier, quantum part of phase estimation are as follows.

\begin{enumerate}

\item
Estimate the phase and its power-of-two multiples
$2^j \phi_k$ to
some constant, modest precision $\delta''$, where
$0 \le j < ({\eta}+2)$. For each $j$, we
apply a series of $s$ measuring operators targeting the state $\ket{\nu}$
controlled on $s$ qubits in the state $\normtwo(\ket{0}+\ket{1})/\sqrt{2}$,
essentially encoding the $2^j \phi_k$ as a bias in a coin, and flipping the
coin $s$ times in a Bernoulli trial, counting the number of $1$ outcomes,
and using that fraction to approximate the real $2^j \phi_k$.
\item
Sharpen our estimate to exponential precision $\frac{1}{2^{{\eta}+2}}$ using the
$({\eta}+2)$ estimates, each for different bits in the binary expansion of
$\phi_k$. Multiplication by successive powers-of-two shift these bits
up to a fixed position behind the zero in a binary fraction representation,
where we can use a finite-automata and a constant number of
bits to refine our $O({\eta})$-length running approximation.
\end{enumerate}

Three things are worth mentioning about the interrelation of the parameters
between these two steps. Since our phases all have a denominator of $2^{\eta}$,
there is no need to run the continued fractions algorithm on multiple
convergents, as is the case with period-finding in Shor's factoring algorithm.
Furthermore, the phases are $1/2^{\eta}$ apart, therefore it suffices to approximate
the phases to within $1/2^{{\eta}+2}$ in order to break ties, which is where
our range for $j$ comes from above.

The number of trials $s$ comes from the Chernoff bound:
%
\begin{equation}
\Pr \left[ | s^{-1}\sum_{r=1}^s v_r - p_* | \ge \delta' \right]
\le 2e^{-2\delta'^2 s}
\end{equation}
%
Setting this equal to the desired error probability $\epsilon$ we get
%
\begin{equation}
s = \frac{1}{2\delta''^2}\ln \frac{1}{\epsilon}
\end{equation}
%
We are actually estimating the values $\cos(2\pi \cdot 2^j \phi_k)$ and
$\sin(2\pi \cdot 2^j \phi_k)$, so if we wish to know $2^j \phi_k$ with
precision $\delta''$, we actually need to determine the $\cos(\cdot)$ and
$\sin(\cdot)$ values with a different precision $\delta'$, upper-bounding
it with the steepest part of the cosine and sine curves.
%
\begin{equation}
\delta' = 1 + \cos(\pi - \delta'') = 1 - \cos(\delta'')
\end{equation}
%
The factor $\frac{1}{2\delta''^2}$ depends on the constant precision with
which we determine our $2^j \phi_k$ values. Since classical time is
cheap and quantum gates are expensive, it makes sense to minimize the number
of trials $s$. Table \ref{tab:ksv-parameters} shows the corresponding values of $1/(2\delta''^2)$
and $\delta'$ as a function of various choices for $\delta''$.

\begin{table}
\centerline{
\begin{tabular}{|c|c|c|c|}
\hline
$\delta''$ & $1/(2\delta''^2)$ & $\delta'(\delta'')$ & $1/(2\delta'^2)$ \\
\hline
$1/16$     & $128$             & $0.0019525$ & $131,160$\\
$1/8$      & $32$              & $0.0078023$ & $  8,213$\\
$1/4$      & $8$               & $0.0310880$ & $    517$\\
\hline
\end{tabular}
}
\caption[Parameters for the number of measurements in PK-KSV]{Parameters for the number of measurements in PK-KSV. Since
the number of trials $s \sim 1/(2\delta'^2)$, smaller is better.}
\label{tab:ksv-parameters}
\end{table}

By making our $\delta'$
exponentially worse (doubling it) we are only increasing the range of
$j$ a linear amount (by one).
% Who knows what the end of the below sentence is now
%In general, for $\delta'=\frac{1}{2^l}$, we get a final estimate for $\phi = 2^{m-3}$

Projective measurements are irreversible, and it is not so important that
we are left with (classical, unentangled) garbage in our $t$-qubit ancillae register.
After all, we only run phase estimation once
to get our initial $\ket{\psi_{\eta}^{(k)}}$ state.
The reason why
the authors of \cite{Kitaev2002} go to some care to show that all the classical
postprocessing steps can be done in polynomial-size and logarithmic-depth
is that these must be done coherently to a quantum state $\ket{\psi_{\eta}^{(0)}}$ in order
to turn it into $\ket{\psi_{\eta}^{(1)}}$ in original PK-KSV. However, our new procedure
sidesteps this requirement, so we are free to offload this processing to a
classical controller, which is available in \textsf{2D CCNTCM}.

Since we have seen KSV-style phase estimation in
Chapter \ref{chap:factor-polylog}, it is important to mention here several
differences between applying phase estimation to factoring versus
applying it to QFS generation. The first difference is that a constant
success probability of $\frac{3}{4}$ is no longer good enough. We would
like to drive down our error exponentially low as $\epsilon = 2^{-l}$.
Second, the operator whose phase we are estimating is now
addition modulo $2^{\eta}$, which is more efficient in circuit resources
than modular multiplication (see Section \ref{subsec:qfs-adder}).

In the procedure above, we measured power-of-two multiples of the phase
$2^{j}\phi_k$ to recover single bits of $\phi_k$ at a time.
However, a remarkable recent result by Svore-Hasting-Freedman (SHF) shows
how to use information theoretic and signal processing techniques to
improve the number of measurements needed from $O({\eta}^2)$ as in conventional
KSV to $O({\eta}\log^{*}{\eta})$.
Their key technique is to select random measurements of this form:
%
\begin{equation}
(2^{j_1}+2^{j_2}+\ldots + 2^{j_S})\phi_k
\end{equation}
%
This would allow us to recover multiple bits at a time.
SHF empirically determines that the number of trials $s$ in
each Bernoulli series scales as $O(\log {\eta})$ for ${\eta} = \{1000,10000\}$,
on the same order of magnitude for running factoring on
keys of 2048 to 4096 bits.
Furthermore, they extend this approach to multiple rounds. Given
that the number of measurements in the final round is
$s = O(\log {\eta})$ and that each round requires logarithmically
fewer measurements than the previous round, they
achieve the iterated-logarithmic depth above.
These techniques could be used to improve PK-KSV, but we
remain conservative for now until numerical upper bounds can be
calculated.

%%%%%%%%%%%%%%%%%%%%%%%%%%%%%%%%%%%%%%%%%%%%%%%%%%%%%%%%%%%%%%%%%%%%%%%%%%%%%%%%%%%%%
\subsection{Resource Comparisons}
\label{subsec:ksv-compare}

We now map both the KSV and Jones algorithms for QFS generation
to \textsf{2D CCNTCM} and compare their resources. We do not consider
the phase kickback part of the quantum compiler, since this controlled addition
is the same for both QFS algorithms. Like PK-Jones \cite{Jones2012}, we
set our error as $\epsilon = \left(\frac{\pi}{2^{\eta}}\right)^2$. It is not clear why
this error was chosen, since for factoring or other polynomial-sized circuits,
an inverse-polynomial error precision is sufficient. However, for the
purpose of comparison, we set the same error.

The error for PK-Jones is one minus the overlap between a
pure Fourier state $\ket{\psi_{\eta}^{(1)}}$ and a distilled
Fourier state $\ket{\tilde{\psi_{\eta}^{(1,m)}}}$ after $m$ rounds.
%
\begin{equation}
1 - | \braket{\tilde{\psi}^{(1,m)}_{\eta}}{\psi_{\eta}^{(1)}} |^2 = \epsilon \le \left( \frac{\pi}{2^{\eta}} \right)^2 
\end{equation}
%
The error in PK-KSV is that an incorrect phase indexed by $k'$ is returned,
one that does not correspond to the eigenstate $\ket{\psi_{\eta}^{(k)}}$.
The target
register for phase estimation begins in a superposition of eigenstates
$\ket{1} = \sum_{s = 1}^{2^{{\eta}-1}} \ket{\psi^{(2s-1)}_n}$, and does not deteriorate
since none of the measuring operators mix between the orthogonal decomposition of the
QFS basis. We make use of Theorem \ref{thm:projective} in Appendix \ref{app:ksv-error},
where the measured index $y$ corresponds to a subspace $\mathcal{M}_y$ of all Bernoulli series outcomes $\Delta$ that would
correspond to the phase $k = (2s-1)$, and the state left in $\mathcal{N}$ is a (possibly impure)
QFS $\ket{\tilde{\psi_{\eta}^{(k)}}}$ which has large overlap with
$\ket{\psi_{\eta}^{(k)}}$.
We take the set of $\Omega$ of subspaces to be the odd QFS indices $\{1, 3, 5, \ldots 2^{\eta} - 1\}$
of which there are $2^{{\eta}-1}$ elements. Taking our phase estimation error to be $\epsilon \left( \frac{1}{2^{3n-1}} \right)$,
we then have the following inequality.
%
\begin{equation}
1 - | \braket{\tilde{\psi}^{(y)}}{\psi^{(k)}} | = \sum_{s = 1}^{2^{{\eta}-1}} \sum_{y \in \Delta: f(j) \ne (2s-1)} \Pr(2s-1|y) \le |\Omega|\epsilon = \frac{1}{2^{2n}} \le \left( \frac{\pi}{2^{\eta}} \right)^2 
\end{equation}
%
Several differences exist in our resource
calculation methods and those of PK-Jones \cite{Jones2012}. First, that
author calculates expected resources, since the distillation method is
probabilistic and involves post-selection. Failures are heralded, and in
the worst case we fall back to using PK-KSV(e) for QFS generation.
Our resources are worst-case,
with still a small probability of failure calculated to match the
case of PK-Jones with all successful post-selection. We also compare our
improvements with those of the original PK-KSV algorithm, for the QFS generation stage.
Second, the resource measurements for PK-Jones are on
\textsf{AC}. Therefore, even in the most favorable (unconstrained, average-case) setting for
PK-Jones and a more constrained model for PK-KSVe, our result still have
lower depth at the cost of greater size and width.
These results are shown in Table \ref{tab:ksv-resources}.

\begin{table}[hbt!]
\begin{center}
\begin{tabular}{|c|c|c|c|c|}
\hline
\textsf{2D CCNTCM} &         &                          & \textsf{AC}     & \\
Resource  & PK-KSVe          & PK-KSV                   & Resource        & PK-Jones\\
\hline
$D$       & $O(\log {\eta})$ & $O(\log^2 {\eta})$       & $\avg{D}$       & $O(\log^2 {\eta})$ \\
$S$       & $O({\eta}^2)$    & $O({\eta}^2\log {\eta})$ & $\avg{S}$       & $O({\eta} \log {\eta})$ \\
$W$       & $O({\eta}^2)$    & $O({\eta}^2)$            & $\avg{W}$       & $2\eta + O(1)$ \\
$\bar{D}$ & $O(\log {\eta})$ & $O(\log^2 {\eta})$       & $\avg{\bar{D}}$ & $O(\log {\eta})$ \\
$\bar{S}$ & $O({\eta}^2)$    & $O({\eta}^2)$            & $\avg{\bar{S}}$ & $O({\eta}^2)$ \\
$\bar{W}$ & $O({\eta})$      & $O({\eta})$              & $\avg{\bar{W}}$ & $O({\eta})$ \\
\hline
\end{tabular}
\caption{A comparison of circuit resources for generating an $\eta$-qubit QFS using PK-KSVe, PK-KSV, and PK-Jones for error $\epsilon \le \left( \frac{\pi}{2^{\eta}} \right)^2$.}
\label{tab:ksv-resources}
\end{center}
\end{table}