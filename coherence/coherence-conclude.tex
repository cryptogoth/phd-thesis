\section{Conclusion}
\label{sec:coherence-conclude}

In this chapter, we turned away from minimizing depth to
examining its time-space tradeoff. Instead of using the
depth-width product, we introduced a new circuit resource
called circuit coherence
which we claim is asymptotically less than depth-width.
Operationally, we defined circuit coherence as an upper
bound for error-correcting effort needed to maintain
only those computationally useful qubits during
computationally useful timesteps. We defined a
plethora of concepts to support us in quantifying
coherence:
influencing and disentangling gates, qubit instances,
influencing paths, reachability, computational subsets
and sets, and the computational state.
Using these concepts, we defined an algorithm for
calculating the coherence for a given quantum circuit.
However, coherence still remains difficult to calculate
for general quantum circuits due to the need to know
whether gates are entangling or disentangling.

We then turn to an alternative model of quantum
computation, the measurement-based model (MBQC).
This model on a cluster state graph, in addition
to sharing many common properties with our
model \textsf{2D CCNTC}, also has a coherence that
is easy to calculate. Inspired by this, we defined
a layered form for quantum circuits which both
facilitates the calculation of circuit coherence
and corresponds to our hybrid factoring architectures
from Chapters \ref{chap:factor-polylog} and \ref{chap:factor-sublog}.
We present the important conjecture that
the optimal pebble-game strategies of Bennett and
Sherman-Levine can be used to uncompute intermediate
garbage in our factoring architectures, leading to
circuit coherence asymptotically less than
depth-width.

Returning to these architectures from the beginning
of our dissertation, we fulfilled our
promise of quantum architecture: providing configurable
parameters for future experimentalists.
We also present the conjecture that a mixed
hybrid-serial approach can lead to even
lower circuit coherence than either a purely
serial or a purely parallel approach.

The two major conjectures above indicate that
circuit coherence shows great promise in
characterizing future quantum architectures,
taking its place among circuit depth, size,
and width. Circuit coherence is well-defined
and has significance independently of
nearest-neighbor or hybrid architectures.
Determining better bounds for circuit coherence,
improving its definition, and applying it
to other quantum algorithms are all worthy
future directions to pursue.