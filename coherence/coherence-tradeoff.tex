\section{Circuit Coherence as a Time-Space Tradeoff}
\label{sec:cohere-tradeoff}

Although circuit coherence's motivation was to capture another resource
of interest that may not be 

%%%%%%%%%%%%%%%%%%%%%%%%%%%%%%%%%%%%%%%%%%%%%%%%%%%%%%%%%%%%%%%%%%%%%%%%%%%%%%
\subsection{Measurement-based Quantum Computing Time-Space Tradeoffs}
\label{subsec:cohere-mbqc}

Our model \textsf{2D CCNTCM} bears some resemblance to the
cluster state used to prove universality of the one-way quantum computer
model by Briegel=Raussendorf \cite{Briegel}. Therefore, a natural question
is to study the relationship between these two models and characterize
how the resources for an algorithm on these two models compare to each other.

First, we must make some simplifications to normalize these two models.
\textsf{2D CCNTCM} as defined in Chapter $\ref{chap:factor-polylog}$ allowed
an irregular planar graph, with constant but bounded degree (no more than
6 for factoring). We will constrain ourself to the regular
2D lattice, as in the cluster state graph. We state without proof that
every \textsf{2D CCNTCM} lattice with $n$ qubits can be embedded in
a regular 2D lattice with at most a constant factor increase in qubits
($O(n)$).

Both models contain a classical controller

\subsection{Other Time-space Tradeoffs}
\label{subsec:cohere-ts-other}

Another study of quantum time-space tradeoffs related to our notion of
circuit width is the bounded space regime by Klauck \cite{Klauck2003}.
In that model, the input is read-only and accessed through an oracle.
Time is counted as the number of 

Klauck discovered time-space tradeoff upper bounds for the specific
problem of sorting $n$ numbers.

In comparison to the classical time-space tradeoff for sorting
discovered by Borodin-Cook \cite{Borodin1982} of $\Omega(TS)$.	

%%%%%%%%%%%%%%%%%%%%%%%%%%%%%%%%%%%%%%%%%%%%%%%%%%%%%%%%%%%%%%%%%%%%%%%%%%%%%%
\subsection{The Pebble Game and Reversible Time-Space Tradeoffs}
\label{subsec:cohere-pebble}

An important time-space tradeoff for classical reversible Turing machines
originates from the pebble game as studied by Bennett \cite{Bennett1973}.
This is relevant to quantum time-space tradeoffs when simulating
completely classical circuits on quantum inputs, such as many arithmetic
functions and a large part of Shor's factoring algorithm. Moreover, this
pebble game models how a reversible machine can compute an irreversible
function. It has a direct connection to circuit coherence as we shall see
below, since quantum computations, especially low-depth ones, can leave
garbage behind which must be uncomputed.

The pebble game is a stylized setting for studying time-space tradeoffs.
Although it may take place on general graphs, we study a line graph
in analogy to the mechanism of an MBQC pattern and our factoring architectures
from Chapters \ref{chap:factor-polylog} and \ref{chap:factor-sublog}.
In short, imagine a row of $n$ tiles in sequence, each of which may
contain at most one pebble. One pebble is placed
on tile $1$ in the first timestep, and the goal is to place a pebble
on tile $n$. The only allowable move is that at every timestep,
you may add or remove a pebble from tile $i+1$ if there is a pebble on
tile $i$. Therefore, you cannot remove the pebble from tile $1$.

The number of timesteps it takes to place a pebble on tile $n$ is known
as the time $T$. The number of pebbles present on all tiles is known
as the space $S$. The obvious strategy for winning the pebble game is
to place a pebble on tile $i$ in timestep $i$, without removing any of them.
This completes in time $T=n$ and space $S = n$. In the case of unbounded
space (unlimited pebbles), this is the optimal depth. However, by
bounding space, we can introduce a time-space product $TS$ and attempt to
upper-bound and lower-bound it.

Knill gave a lower bound for the minimum pebble-game time-space tradeoff
\cite{Knill1995} which is bounded above by $n^3$.

\begin{equation}
TS(n) = 2^{2\sqrt{\log(n)}(1 + o(1))}n = o(n^3), \omega(n^2)
\end{equation}

As a consequence, he obtains a minimum time-space tradeoff for
Shor's factoring algorithm on \textsf{AC}.

\begin{equation}
TS(n) = 2^{2\sqrt{n}(1 + o(1))}n^3 = o(n^4), \omega(n^3)
\end{equation}

The minimum time-space tradeoff for factoring is indeed consistent with the depth-width product of all known
factoring implementations from Table \ref{tab:fpl-results}, including
the current work. The one exception is the approximate 1D NTC factoring
implementation by Kutin \cite{Kutin2006}, which beats the above lower bound.
This suggests that the earlier 1D NTC work by Fowler-Devitt-Hollenberg
\cite{Fowler2004} may achieve the optimal depth-width product.

TODO We need to define a standard form for coherent circuits

Note that in applying
pebble-game time-space tradeoffs to a quantum algorithm, each layer
in a parallel quantum circuit now takes the place of a tile in our
scenario above. Therefore, the space (circuit width) taken by placing a ``pebble'' on
each layer is no longer a constant. We will define the \emph{layer width}
$W_l$
of a layer $l$ in a quantum circuit as the number of new qubits which are operated on in
a given timestep. The maximum layer width $W_l^{max}$
is then well-defined over all the
layers, and the overall circuit width is upper-bounded by the sum of the
layer widths.

\begin{equation}
W \le \sum_{l} W_l
\end{equation}

In fact, $LW$ 

Finally, we conclude by describing the connection between the pebble game
and circuit coherence as defined in Section \ref{sec:cohere-def}.
Let us call the number of pebbles present at any timestep of the pebble game
the instantaneous space, and we will scale each pebble by the 
width of that layer in the quantum circuit (not the width of the entire circuit).
This scaled instantaneous space is equal to the computational subset, and
sum of these scaled instantaneous spaces (the scaled pebble-game space) is
an upper-bound, within a factor of the maximum layer width in a quantum
circuit