\section{Circuit Coherence as a Time-Space Tradeoff}
\label{sec:cohere-tradeoff}

Although circuit coherence's motivation was to capture another resource
of interest to optimize, any new resource may introduce tradeoffs with
existing resources. In this case, decreasing circuit coherence
may increase size and indirectly depth.

First, we will describe a special form for quantum circuits called
layered form in Section \ref{subsec:cohere-lqc}. This will make
it easier to use time-space tradeoff results from Bennett's
reversible pebble game, described in Section \ref{subsec:cohere-pebble}.
Then in Section \ref{subsec:cohere-factor}, we describe a conjectured asymptotic separation
between circuit coherence and the depth-width product for modular
multiplication in Shor's factoring algorithm. Finally, we conclude with
promising directions for future research regarding factoring architectures
and circuit coherence.

%%%%%%%%%%%%%%%%%%%%%%%%%%%%%%%%%%%%%%%%%%%%%%%%%%%%%%%%%%%%%%%%%%%%%%%%%%%%%%
\subsection{Circuits in Layered Form}
\label{subsec:cohere-lqc}

We now present a special form for quantum circuits which will be useful
later in proving facts about circuit coherence, called \emph{layered form}.
It applies to quantum algorithms in which gates execute
monotonically from input qubits to output qubits in parallel layers,
and therefore the physical layout of the qubits mimic the logical
execution of gates. This is useful for circuits which leave
garbage ancillae behind in each layer. In those cases,
circuit coherence can be improved at a negligible increase
in circuit depth and size. This is exactly the form of our
nearest-neighbor factoring architectures in earlier chapters.

Layered form is later used to leverage
results from reversible pebble games on a linear graph. It is also
similar to the circuits which can be parallelized to
constant-depth
MBQC patterns given in Section \ref{subsec:mbqc-par}, except that
each layer is allowed to introduce ancillae qubits.
%It is currently unknown whether
%all quantum circuits can be put into layered form.

\begin{definition}{\textbf{Classical layered form for quantum circuits.}}
We say an $n$-qubit quantum circuit is in \emph{layered form} if the following
properties apply. We assume that the circuit is part of an architecture
with a classical controller (\textsf{CCAC}).

\begin{enumerate}
\item
All gates are single-qubit or two-qubit gates.
\item
All qubits can be arranged in a directed, acyclic graph in $\tilde{D}+1$
layers $(l_0, l_1, \l_2, \ldots, l_{\tilde{D}})$
such that all two-qubit gates only operate between consecutive layers or
within a current layer. The size of a layer is the number of qubits within it,
which is polynomial in $n$: $|l_i| \in poly(n)$.
\item The $n$ input qubits are in $l_0$ and the $n$ output qubits are
in $l_{\tilde{D}}$.
\item All gates can be partitioned into $\tilde{D}$ cohorts $(C_1, \ldots, C_{\tilde{D}})$,
such that all gates in a cohort $C_j$
operate on the same consecutive layers $(l_{j}, l_{j+1})$
or within those two layers.
This is not a unique partitioning.

\item
All gates in cohort $C_j$ execute before all gates in cohort $C_{j+1}$.
%That is, there are no two groups $G_{j}$ (operating on layers $(l_{j}, l_{j+1})$
%and $G_{j+1}$ (operating on layers $(l_{j+1}, l_{j+2})$)
%such that a gate in $G_{j}$ operates on any qubit in $l_{j+1}$ while or
%after any gate in $G_{j'}$ operates on any qubit in the same layer $l_{j+1}$.
%We call $S''$ the \emph{layered circuit size}.
\end{enumerate}
\label{def:lqc}
\end{definition}

The layers $(\l_1, \ldots, l_{\tilde{D}})$ can be considered disjoint groupings in
physical space.
The gate groups $\{ G_j \}$ execute in disjoint cohorts which can be considered
groupings in time. At any one timestep, only gates from one cohort are executing,
and each cohort executes in order from $C_1$ to $C_{\tilde{D}}$.
%Once all the gates groups from exactly one cohort
%are executing. If the circuit has a classical controller, multiple gate groups
%can be operating concurrently within the cohort.

%Let us call the number of cohorts the cohort depth $D''$, where each gate group
%$G_{j,k}$ (now with two indices) belongs to exactly one cohort $C_k$.
%The number of timesteps in a cohort is the
%maximum that any of its gate groups $G_j$ requires to execute.
%That is, the cohorts represent a
%partitioning in time of gates which execute in groups.

We call the \emph{layer depth} of our layered circuit $\tilde{D}$, the number
of physical layers and also the number of cohorts.
It is at most the total circuit depth $D$, with saturation when each cohort
only contains gates which execute in a single timestep.
%cohort circuit depth $D''$, which itself
%is at least the total circuit depth $D$.

\begin{equation}
\tilde{D} \le D
\end{equation}

Layered quantum circuits as described above never uncompute
a layer. In the worst case, garbage is left behind in every
layer until all layers are potentially full of garbage,
and we are left with our answer in the output qubits and
our input qubits as before.

Most importantly for our proofs below, we can now define
\emph{layer coherence} $\tilde{Q}_i$  for timestep $i$
as the number of layers which intersect with the computational
subset $L_i$ from our algorithm in Section \ref{subsec:cohere-algo}.
The total layer coherence for a circuit is the sum of the
layer coherences in any timestep.

\begin{equation}
\tilde{Q} = \sum_{i=1}^{D} \tilde{Q}_i \le \tilde{D}^2
\end{equation}

Without uncomputation, each $\tilde{Q}_i$ is equal to $i$ and upper-bounded by $\tilde{D}$,
so the total layer coherence $\tilde{Q}$ is upper-bounded by the layer depth squared.

We can define the maximum number of qubits in any layer as $w_{max}$.

\begin{equation}
w_{max} = \max_{i \in [D]} |l_i|
\end{equation}

Then the total circuit coherence is upper-bounded by the layer coherence
multiplied by $w_{max}$, although this will asymptotically be the same as the depth-width product $D\cdot W$.

\begin{equation}
Q \le \sum_{i=1}^{D} \tilde{Q}_i \cdot w_{max} = O(D\cdot W)
\end{equation}

We can achieve tighter upper bounds by using the exact layer widths for every
layer $l_j$ which intersects with $L_i$ in every timestep $i$.
The inequality below is saturated for $D=1$.

\begin{equation}
Q \le \sum_{i=1}^{D} \quad \sum_{j: l_j \cap L_i} \quad |l_j| \le D\cdot W
\end{equation}

%The second inequality is saturated when each cohort only contains gates which
%execute within a single timestep. The first inequality is saturated when
%in each timestep, cohort $C_i$ operates only on layers $(l_{i},l_{i+1}$.

%If each cohort executes in $O(1)$ depth, then $D'' = O(D)$.

We can now verify that our factoring implementations presented in earlier
chapters are in fact layered circuits with the following layer and cohort
partitionings. For modular multiple addition,
each CSA layer is a cohort
which operates in constant depth, therefore, $\tilde{D} = O(D)$.
The results of
each CSA layer are propagated to the next CSA layer, where in the next cohort,
those numbers are also added in constant-depth, and so on.
For the partial product creation stage of modular multiplication, we can
define that entire stage as a layer for purposes of computing and uncomputing
garbage bits, even though it has depth $O(\log n)$, size $O(n^3)$, and
width $O(n^3)$.

So far, we have described a circuit which proceeds through CSA layers
(or just layers, to use the more general layered circuit term) to compute
a final answer, leaving behind garbage in each layer and never uncomputing
them until the end.
There is a single ``front edge'' of computation which
proceeds from input qubits to output qubits, both physically over the
circuit in layers and temporally over cohorts.

However,
it is natural to ask whether we
can perform any other kind of intermediate uncomputation to reduce
circuit coherence, at the possible increase of either circuit depth
or circuit size. The answer to this question is the subject of the
rest of this chapter.

%\subsection{Other Time-space Tradeoffs}
%\label{subsec:cohere-ts-other}

%Another study of quantum time-space tradeoffs related to our notion of
%circuit width is the bounded space regime by Klauck \cite{Klauck2003}.
%In that model, the input is read-only and accessed through an oracle.
%Time is counted as the number of 

%Klauck discovered time-space tradeoff upper bounds for the specific
%problem of sorting $n$ numbers.

%In comparison to the classical time-space tradeoff for sorting
%discovered by Borodin-Cook \cite{Borodin1982} of $\Omega(TS)$.	

%%%%%%%%%%%%%%%%%%%%%%%%%%%%%%%%%%%%%%%%%%%%%%%%%%%%%%%%%%%%%%%%%%%%%%%%%%%%%%
\subsection{The Pebble Game and Reversible Time-Space Tradeoffs}
\label{subsec:cohere-pebble}

An important time-space tradeoff for classical reversible Turing machines
originates from the pebble game as studied by Bennett \cite{Bennett1973}.
This is relevant to quantum time-space tradeoffs when executing
completely classical circuits on quantum inputs, as in many arithmetic
functions and a large part of Shor's factoring algorithm. Moreover, this
pebble game models how a reversible machine can compute an irreversible
function. It has a direct connection to circuit coherence as we shall see
below, since quantum computations, especially low-depth ones, can leave
garbage behind which must be uncomputed.

The pebble game is a stylized setting for studying time-space tradeoffs.
Although it may take place on general graphs, we study a line graph
in analogy to the mechanism of an MBQC pattern and our factoring architectures
from Chapters \ref{chap:factor-polylog} and \ref{chap:factor-sublog}.
In short, imagine a row of $n+1$ tiles $(t_0, t_1, \ldots, t_n)$
in sequence, each of which may
contain at most one pebble. One pebble is placed
on $t_0$ in the first timestep, and the goal is to place a pebble
on tile $t_n$. The only allowable move is that at every timestep,
you may add or remove a pebble from tile $i+1$ if there is a pebble on
$t_i$. Therefore, you can never remove the pebble from tile $t_0$.
(This is known as an input-saving pebble game.
There are other versions of the pebble game, similar to other kinds of
reversible computations, where the input can be removed).

In what follows, we modify the standard computer science integer set notation
to include $0$:
\begin{equation}
[T] = \{0, 1, 2, \ldots, T \}
\end{equation}

The number of timesteps it takes to place a pebble on tile $n$ is defined
as the time $T$.
The number of pebbles present on all tiles at any one move $i$
is known as the instantaneous space $S_i$. We call the space $S$ for the whole pebble
game as the maximum number of any pebbles on a game at any particular time:
$S = \max_{i \in [T]} S_i$.
The obvious strategy for winning the pebble game is
to place a pebble on $t_i$ in timestep $i$, without removing any of them.
This completes in time $T=n$ and space $S = n$. In the case of unbounded
space (unlimited pebbles), this is the optimal depth. However, by
bounding space, we can introduce a time-space product $TS$ and attempt to
upper-bound and lower-bound it. The product $TS$ of this naive strategy
is $n^2$.

Knill gave a lower bound for the minimum pebble-game time-space tradeoff
\cite{Knill1995} which is bounded above by $n^2$, thereby showing the
non-optimality of the naive strategy.

\begin{equation}
TS(n) = 2^{2\sqrt{\log(n)}(1 + o(1))}n = o(n^2), \omega(n)
\end{equation}

As a consequence, he obtains a minimum time-space tradeoff for
Shor's factoring algorithm on \textsf{AC}.

\begin{equation}
TS(n) = 2^{2\sqrt{n}(1 + o(1))}n^3 = o(n^4), \omega(n^3)
\end{equation}

These time-space tradeoffs are applicable not just to
pebble games but to classical properties of quantum
circuits such as layer coherence for factoring
architectures, as we will discuss in the  next section.

The minimum time-space tradeoff for factoring is indeed consistent with the depth-width product of all known
factoring implementations from Table \ref{tab:fpl-results}, including
the current work. The one exception is the approximate 1D NTC factoring
implementation by Kutin \cite{Kutin2006}, which suggests that a different
$TS$ tradeoff may apply for approximate modular exponentiation.
This also indicates that the earlier 1D NTC work by Fowler-Devitt-Hollenberg
\cite{Fowler2004} may come the closest to the optimal depth-width product
(and therefore optimal circuit coherence) for
exact modular exponentiation.

%%%%%%%%%%%%%%%%%%%%%%%%%%%%%%%%%%%%%%%%%%%%%%%%%%%%%%%%%%%%%%%%%%%%%%%%%%%%%%
\subsection{Pebble Games and Layered Quantum Circuits}
\label{subsec:cohere-pebble-lqc}

Taking a step back, how do we determine $T$? We must return to the original
motivation for the pebble game, in simulating an irreversible Turing machine
on a reversible one \cite{Bennett1989}. The irreversible Turing machine's
running time on a particular input is defined as $T$ and the maximum
space it uses over this time (including the input) is $S$. From this definition,
we get $S \ge T$. What is the corresponding pebble game for the most naive
reversible simulation strategy possible?

\begin{definition}{\textbf{Irreversible pebble game.}}
An irreversible pebble game is one which never removes any pebbles.
It corresponds to a computation on an irreversible Turing machine $M$.
By definition, it has time $T(n) = n$ and some space $S(n)$.
There is only one irreversible pebble game on $n$ tiles, and it is also called
the naive pebbling strategy.
In each move $i$ one can only
place a pebble in tile $t_i$.
We can define the number of pebbles after any move $i$ as the \emph{instantaneous space}
$S_i$, which in this case is just $i$, where $S = S_T$.
Therefore, $S = n$.
\end{definition}

\begin{definition}{\textbf{Pebble-instances.}}
We now define a new quantity \emph{pebble-instances} denoted by
$\hat{S}$ which is the total number of pebbles which appear in
any history of the pebble game. This is just the sum of all the
instantaneous spaces $S_i$ defined above.
\begin{equation}
\hat{S} = \sum{_i=1}^T S_i
\end{equation}

We also define the indicator random variable $t_{i,j}$ which is $1$
if a pebble is on tile $t_i$ at timestep $j$ and zero otherwise.
We can define the instantaneous space in terms of these indicators.
\begin{equation}
S_i = \sum_{j=1}^T t_{i,j}
\end{equation}
\end{definition}

We can now provide analogous constructions for both an irreversible pebble game
and a layered quantum circuit, where the parameters in the pebble game
upper bound those in the layered quantum circuit.

\begin{theorem}{\textbf{Irreversible pebble game for layered quantum circuit.}}
Consider a layered quantum circuit $C$
as defined in 
Definition \ref{def:lqc}. Then a pebble game $P$ can be constructed such that
time $T$ and space $S$ are equal to layer depth $\tilde{D}$,
and the number of pebbles at any timestep $i$, called $S_i$, equals the
layer coherence $\tilde{Q}_i$.
\label{thm:ipg-lqc}
\end{theorem}

\begin{proof}
For the given layered quantum circuit $C$, construct a pebble game $P$
with $\tilde{D}$ tiles. For every cohort execution of $C_{i-1}$, $i \in [\tilde{D}]$,
place a pebble on tile $t_i$.
In the first timestep $i=0$, note that setting the quantum inputs for $C$
is the same as executing cohort $C_0$ in layer $l_0$ and 
placing a pebble on tile $t_0$.

Then if there is a pebble on tile $t_j$,
the layer $l_j$ is part of the computation state. In every timestep $i$ then, the layer coherence
$\tilde{Q}_i = i$ and the number of pebbles on the tiles is $i$.
Both $C$ and $P$ complete in timestep $\tilde{D}$.
\end{proof}

During Bennett's reversible
simulation with time $T'$ and $S'$, there is some overhead. That is,
$T' \ge T$ and $S' \ge S$. Bennett's upper bounds for these figures
as well as Sherman-Levine's improvements \cite{Levine1990} are shown
in Table \ref{tab:pebble-ts}. We will make use of these constructions in
the next section.

\begin{table}[hbt!]
\centerline{
\begin{tabular}{|c|c|c|}
\hline
Tradeoff Construction           & $T'$                             & $S'$ \\
\hline
Naive                           & $T$                              & $ST$\\
Bennett \cite{Bennett1989}      & $O(T^{1+\epsilon})$              & $O(\epsilon 2^{1/\epsilon} S \log T)$ \\
Levine-Sherman \cite{Levine1990} & $O(T^{1+\epsilon}/S^{\epsilon})$ & $O(\epsilon 2^{1/\epsilon} S (1 + \log\frac{T}{S}))$ \\
\hline
\end{tabular}
}
\caption{Pebble game time-space tradeoffs for reversible simulation of
irreversible Turing machines.}
\label{tab:pebble-ts}
\end{table}

The pebble game as we have presented it can be called a serial, reversible
pebble game because only one pebble is added or removed at a time. This
simplifies our analysis in what follows, but one can also define
a parallel pebble game as one where multiple moves (exactly as
defined for the serial game above) can be made. The locations where these
moves can be made are the boundaries between
a pebble on $t_i$ and no pebble on $t_{i+1}$.
We will not discuss this variation any further, but it is a
interesting direction for future research to decrease both depth and
circuit coherence further.

%%%%%%%%%%%%%%%%%%%%%%%%%%%%%%%%%%%%%%%%%%%%%%%%%%%%%%%%%%%%%%%%%%%%%%%%%%%%%%
\subsection{Instantaneous Coherence and Pebble Game Space}
\label{subsec:cohere-equiv}

Finally, we conclude by describing the connection between the pebble game
and circuit coherence as defined in Section \ref{sec:cohere-def}. To do this,
we now construct a reversible pebble game for a modified layered quantum
circuit which now allows uncomputation.

We will then use the construction of Bennett \cite{Bennett1989},
which simulates an irreversible Turing machine computation on a
reversible Turing machine via a reversible pebble game. We will then
use to show how to execute a layered quantum circuit with reduced
circuit coherence by completing with only the input and output layers
in the computation state.

\begin{definition}{\textbf{Layered quantum circuit with uncomputation.}}
A layered quantum circuit can be augmented to allow layers to be uncomputed
by allowing in timestep $i$ that the cohort $C_i$ can operate as follows:

\begin{itemize}
\item $C_i$ can operate on any consecutive layers $(l_{j},l_{j+1})$,
for any $j \in [\tilde{D}]$,
not just layers $(l_{i},l_{i+1})$.
\item All $C_i$ that operate on a particular layer pair $(l_{j},l_{j+1})$
execute exactly the same gates in the same order, just occurring in
different timesteps. That is, all cohorts that operate on the
same layer are copies of each other shifted in time.
Therefore, given a computation state in layer $l_{j}$, any such
$C_i$ either extends the computation state into $l_{j+1}$ if it
was previously all $\ket{0}$, or it uncomputes $l_{j+1}$ completely
back to all $\ket{0}$ if that layer was already in the computation state.
\end{itemize}
\end{definition}

%Let us call the number of pebbles present at any timestep of the pebble game
%the instantaneous space, and we will scale each pebble by the 
%width of that layer in the quantum circuit (not the width of the entire circuit).
%This scaled instantaneous space is equal to the computational subset, and
%sum of these scaled instantaneous spaces (the scaled pebble-game space) is
%an upper-bound, within a factor of the maximum layer width in a quantum
%circuit

Now, we will show that a reversible pebble game can upper-bound the
layer coherence of a layered quantum circuit with uncomputation.

\begin{theorem}{\textbf{Circuit coherence and pebble game space.}}
Given a layered quantum circuit $C'$ with uncomputation with layer depth $D$ and width $W$, we
can construct a one-dimensional
reversible pebble game $P'$ that has time $T$
and space $S$, which executes in parallel timesteps.
Then the layer coherence $\tilde{Q}$ is upper-bounded by the
total pebble-timesteps $\mathcal{S}$, and the total circuit coherence
is upper-bounded by the following weighted sum:
\begin{equation}
\tilde{Q} \le \sum_{i=1}^D \sum_{j=1}^D t_{i,j} |l_j|
\end{equation}
\label{thm:pg-cc}
\end{theorem}

\begin{proof}
This construction mirrors the one in Theorem \ref{thm:ipg-lqc},
except now in timestep $i$, when the cohort $C_i$ in $C'$ uncomputes a layer $l_j$
from the layer $l_{j-1}$, in $P'$ we remove the pebble from tile $t_j$
and there is a pebble on tile $t_{j-1}$. Therefore, we maintain the
following invariants at every timestep, starting with $i=1$.

\begin{eqnarray}
        S_i & = & \tilde{Q}_i \\
\hat{S} & = & \tilde{Q}
\end{eqnarray}

Now, unlike in the pebble game, where all pebbles on any tile $t_i$
represent unit space, the corresponding circuit layer $l_i$ to that tile
represents $|l_i|$ qubits as part of the computation state. Therefore,
the instantaneous coherence $Q_i$ at timestep $i$ is equal to the
weighted sum of the pebbles which are present in the same timestep.

\begin{equation}
Q_i = \sum_{j=1}^{D} t_{i,j} |l_j|
\end{equation}

Since the total circuit coherence is the sum of the instantaneous coherences,
we have the desired result.

\begin{equation}
Q = \sum_{i=1}^D Q_i = \sum_{i,j=1}^D t_{i,j} |l_j| 
\end{equation}
\end{proof}