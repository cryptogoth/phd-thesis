\section{Circuit Coherence of Factoring Architectures}
\label{sec:cohere-factor}

In this section, we apply circuit coherence and pebble-game
uncomputing techniques from the previous section to our
factoring architectures. The techniques are generic for
any layered circuit, and are not specific to factoring or
even nearest-neighbor circuits, except that the layered circuits
are nearest-neighbor at the layer level.

In Section \ref{subsec:cohere-conject}, we provide a conjecture
for decreasing circuit coherence for modular multiplication while
only marginally increasing our depth.

In Section \ref{subsec:cohere-factor}, we provide a generalized,
configurable-depth factoring architecture based on
Chapter \ref{chap:factor-polylog}.

%%%%%%%%%%%%%%%%%%%%%%%%%%%%%%%%%%%%%%%%%%%%%%%%%%%%%%%%%%%%%%%%%%%%%%%%%%%%%%
\subsection{Reducing Circuit Coherence with Intermediate Uncomputing}
\label{subsec:cohere-conject}

The naive strategy for computing modular multiplication has
depth $D = O(\log n)$, size $O(n^2)$, and width $O(n^3)$. Therefore, the
circuit coherence is upper-bounded by $O(n^3 \log n)$. However,
Bennett \cite{Bennett1989} with corrections from
Sherman-Levine \cite{Levine1990} have shown that an irreversible
pebble game with time $T$ and space $S$ can be simulated reversibly
with overhead $T' = O(T^{1+\epsilon})$ and $S = O(\epsilon 2^{1/\epsilon} S \log T)$
Therefore, we propose the following conjecture.

\begin{conjecture}
Define a layered circuit $C$ for modular multiplication
with layer depth $\tilde{D} = O(\log n)$ and maximum layer width
$w_{max} = O(n^3)$.
Define $P$ as the pebble game corresponding to the optimal reversible simulation of
an irreversible pebble game on $\tilde{D}$ tiles on a reversible
pebble game with the same number of tiles, as proved by
Li \cite{Li1998} based on the authors above.
Using the results of Theorem \ref{thm:pg-cc}, we can create a new
layered circuit $C'$ with uncomputation that performs the
same modular multiplication with the following conjectured resources:
depth $D' = O(D^{1+\epsilon})$,
the same width $W' = W$
and reduced circuit coherence $Q = O(\epsilon 2^{1/\epsilon} n^3 \log\log n)$.
\end{conjecture}

% To improve: add this back in if time permits

%From the previously best-known architectural depth of
%$O(n^2)$, we improved the depth to $O(\log^2 n)$ in Chapter \ref{chap:factor-polylog},
%and then even beneath that to $O((\log\log n)^2)$ in Chapter \ref{chap:factor-sublog}.
%This last depth was limited only by the depth of quantum compiling a
%single-qubit rotation for the quantum majority gate. However, the improvement
%in depth came at the cost of increasing width and size.
%For the polylogarithmic depth, we calculated a size and width of $O(n^4)$.

%How does this compare with our technique of hand-optimized, nearest-neighbor
%mapping in Chapters \ref{chap:factor-polylog} and \ref{chap:factor-sublog}?
%We repeat the relevant rows from 
%Table \ref{tab:fpl-results} and Table \ref{tab:sublog-resources}, namely
%those that correspond to \textsf{AC} and \textsf{2D CCNTCM} implementations
%over the circuit basis of the PK-KSV quantum compiler, which is
%Clifford+$Toffoli$.

%\begin{table}[hbt!]
%\begin{tabular}{|c|c|c|c|c|}
%\hline
%Implementation                           & Architecture  & $D$           & $S$                                             & $W$ \\
%\hline
%Shor \cite{Shor1995}                     & \textsf{AC}   & [$O(n)$]      & $O(n^2\log n \log\log n)$                       & $O(n\log n \log\log n)$ \\
%Browne-Kashefi-Perdrix \cite{Browne2009} & \textsf{CCAC} & $O(1)$        & $O(\frac{1}{\epsilon}n^{6+2\epsilon}\log^{4}n)$ & $$ \\
%Cleve-Watrous \cite{Cleve2000}           & \textsf{AC}   & $O(\log^3 n)$ & $O(n^3)$                                        & $O(n^3)$ \\
%\hline
%Polylog Depth, Chapter \ref{chap:factor-polylog} & \textsf{2D CCNTCM} & $O((\log\log n)^2)$ & $$ & $$ \\
%\hline

%\end{tabular}
%\caption{A comparison for \textsf{AC} factoring architectures.}
%\label{tab:cohere-ac-factor}
%\end{table}



%\begin{table}[hbt!]
%\begin{tabular}{|c|c|c|c|c|}
%\hline
%Implementation                           & Architecture  & $D$ & $S$ & $W$ \\
%\hline
%Shor \cite{Shor1995}                     & \textsf{AC}   & [$O(n)$] & $O(n^2\log n \log\log n)$ & $O(n\log n \log\log n)$ \\
%Browne-Kashefi-Perdrix \cite{Browne2009} & \textsf{CCAC} & $O((\log\log n)^2)$ & $$ & $$ \\
%Cleve-Watrous \cite{Cleve2000}           & \textsf{AC}   & $O((\log\log n)^2)$ & $$ & $$ \\
%\hline
%Polylog Depth, Chapter \ref{chap:factor-polylog} & \textsf{2D CCNTCM} & $O((\log\log n)^2)$ & $$ & $$ \\
%\hline

%\end{tabular}
%\caption{A comparison of hand-optimized and automated nearest-neighbor mappings
%for \textsf{AC} factoring architectures.}
%\label{tab:mbqc-mapping}
%\end{table}

%%%%%%%%%%%%%%%%%%%%%%%%%%%%%%%%%%%%%%%%%%%%%%%%%%%%%%%%%%%%%%%%%%%%%%%%%%%%%%
\subsection{Configurable-Depth Factoring}
\label{subsec:cohere-factor}

When we decrease our nearest-neighbor factoring depth from $O(\log^2 n)$
in Chapter \ref{chap:factor-polylog} to $O((\log\log n)^2)$ in
Chapter \ref{chap:factor-sublog}, we calculated a disproportionate
increase in circuit size and width from
$O(n^4)$ to $\Omega(n^6\log^2 n)$. This seems to be quite an
unfavorable depth-width tradeoff, and it is natural to ask whether
we could have some configurable depth in between
poly-logarithmic and sub-logarithmic that would let a quantum
systems architect choose the right tradeoff for a particular
implementation.

In this section, we provide such a configurable depth
factoring architecture by generalizing our implementation in
Chapter \ref{chap:factor-polylog}. In that chapter, we
wanted to multiply $n\times n$-bit quantum integers in parallel.
To do so, we divided them up into $\lceil n/2 \rceil$ groups of two.
In each group of two quantum integers, call them $\ket{x}$ and
$\ket{y}$, in order to get all the product bits
$\ket{x_i\cdot y_j}$ we need to generate $n^2 \times n$-bit modular residues
$2^i 2^j \bmod m$ controlled on two qubits. We then add these
down with modular multiple addition back to an $n$-bit (CSE) number,
and we are then left with $\lceil n/2 \rceil$ numbers to multiply in
the second level. This takes place in depth $O(\log n)$ and takes
width and size $O(n^3)$.
Modular exponentiation has $\lceil \log_2 n \rceil + 1$ such
levels and perform $(n-1)$ multiplications total, giving us the
final depth of $O(\log n)$ and size and width $O(n^4)$.

The configurable parameter is how we group quantum integers for
expansion. If instead of groups of two, we used groups of
$4$, then each multiplication would require expansion
into $n^4 \times n$-bit numbers. These would still add down to
a single $O(n)$-bit CSE number in $O(\log n)$ depth, but
would now take size and width $O(n^5)$. Modular exponentiation
would still take total depth $O(\log^2 n)$ but would now take
size and width $O(n^6)$, with hidden constants dependent on
the parameter.

We now name the configurable parameter $d$, where in each level
of modular exponentiation we group quantum integers into groups of
$2^d$, and we also expand into $n^{2^d}\times n$ product bits.
We compare this new depth-width tradeoff in Table \ref{tab:config-tradeoff}.

\begin{table}[hbt!]
\centerline{
\begin{tabular}{|c|c|c|c|}
\hline
Implementation                           & $D$ & $W$ \\
\hline
Polylog Depth ($d=1$), Chapter \ref{chap:factor-polylog} & $O(\log^2 n)$ & $O(n^4)$ \\
Config Log Depth ($d$) & $O(\frac{2^d}{d}\log^2 n)$ & $O(\frac{1}{2^d}n^{{2^d}+2})$ \\
($d = \lceil \log_2 n \rceil$)  & $O(n\log n)$ & $O(n^{n+2})$ \\
($d = 1/n$)  & $O(n2^{1/n}\log^2 n)$ & $O(\frac{1}{2^{1/n}}n^{2^{1/n}+2})$ \\
Sublog Depth, Chapter \ref{chap:factor-polylog} & $O((\log\log n)^2)$ & $O(\frac{1}{\epsilon}n^{6 + 2\epsilon}\log^2 n)$ \\
\hline
\end{tabular}
}
\caption{A comparison of configurable-depth factoring architectures and their depth-width tradeoffs.}
\label{tab:config-tradeoff}
\end{table}

The depth-width product for configurable parameter $d$ is
$DW = O(n^{2^d + 3})$. By setting it to be less than one $d = 1/k$
for $k \ge 2$, we are
actually splitting each $n$-bit number into $2^{k-1}$ pieces and grouping
them together.

%Of all these implementations, the one that is the closest to the
%minimum $TS$ from Ref. \cite{Knill1995} is the $O(\log^2 n)$-depth
%factoring implementation from Chapter \ref{chap:factor-polylog).

%Other approaches to configuring 

Finally, we suggest a construction for low-coherence factoring circuits.
In the original quantum period-finding (QPF) procedure of
Nielsen-Chuang \cite{Nielsen2000}, $n$ quantum integers are multiplied in
series, giving a depth in multiplications of $O(n)$.
In the parallelized construction of Kitaev-Shen-Vyalyi \cite{Kitaev2002},
all $n$ quantum integers are multiplied in parallel with a multiplication
depth of $O(\log n)$. Instead, one can consider another kind of configurable-depth
QPF where $n/2^{d-1}$ integers are multiplied in parallel, % in depth $O(d)$,
and there are $2^{d-1}$ such groups multiplied in serial,
%, giving a total depth of $O(2^d \log n)$,
where $d = \lceil \log_2 n \rceil$ for the serial QPF above,
and $d=1$ for the parallel QPF above.

Determining other bounds for circuit coherence and extending this
to other quantum algorithms remains a promising area for future research.

%The best choice of $d$ for either these configurable-coh