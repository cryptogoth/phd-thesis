\section{Circuit Coherence of Factoring Architectures}
\label{sec:cohere-factor}


From the previously best-known architectural depth of
$O(n^2)$, we improved the depth to $O(\log^2 n)$ in Chapter \ref{chap:factor-polylog},
and then even beneath that to $O((\log\log n)^2)$ in Chapter \ref{chap:factor-sublog}.
This last depth was limited only by the depth of quantum compiling a
single-qubit rotation for the quantum majority gate. However, the improvement
in depth came at the cost of increasing width and size.
For the polylogarithmic depth, we calculated a size and width of $O(n^4)$.


How does this compare with our technique of hand-optimized, nearest-neighbor
mapping in Chapters \ref{chap:factor-polylog} and \ref{chap:factor-sublog}?
We repeat the relevant rows from 
Table \ref{tab:fpl-results} and Table \ref{tab:sublog-resources}, namely
those that correspond to \textsf{AC} and \textsf{2D CCNTCM} implementations
over the circuit basis of the PK-KSV quantum compiler, which is
Clifford+$Toffoli$.

\begin{table}[hbt!]
\begin{tabular}{|c|c|c|c|c|}
\hline
Implementation                           & Architecture  & $D$           & $S$                                             & $W$ \\
\hline
Shor \cite{Shor1995}                     & \textsf{AC}   & [$O(n)$]      & $O(n^2\log n \log\log n)$                       & $O(n\log n \log\log n)$ \\
Browne-Kashefi-Perdrix \cite{Browne2009} & \textsf{CCAC} & $O(1)$        & $O(\frac{1}{\epsilon}n^{6+2\epsilon}\log^{4}n)$ & $$ \\
Cleve-Watrous \cite{Cleve2000}           & \textsf{AC}   & $O(\log^3 n)$ & $O(n^3)$                                        & $O(n^3)$ \\
\hline
Polylog Depth, Chapter \ref{chap:factor-polylog} & \textsf{2D CCNTCM} & $O((\log\log n)^2)$ & $$ & $$ \\
\hline

\end{tabular}
\caption{A comparison for \textsf{AC} factoring architectures.}
\label{tab:cohere-ac-factor}
\end{table}



\begin{table}[hbt!]
\begin{tabular}{|c|c|c|c|c|}
\hline
Implementation                           & Architecture  & $D$ & $S$ & $W$ \\
\hline
Shor \cite{Shor1995}                     & \textsf{AC}   & [$O(n)$] & $O(n^2\log n \log\log n)$ & $O(n\log n \log\log n)$ \\
Browne-Kashefi-Perdrix \cite{Browne2009} & \textsf{CCAC} & $O((\log\log n)^2)$ & $$ & $$ \\
Cleve-Watrous \cite{Cleve2000}           & \textsf{AC}   & $O((\log\log n)^2)$ & $$ & $$ \\
\hline
Polylog Depth, Chapter \ref{chap:factor-polylog} & \textsf{2D CCNTCM} & $O((\log\log n)^2)$ & $$ & $$ \\
\hline

\end{tabular}
\caption{A comparison of hand-optimized and automated nearest-neighbor mappings
for \textsf{AC} factoring architectures.}
\label{tab:mbqc-mapping}
\end{table}

%%%%%%%%%%%%%%%%%%%%%%%%%%%%%%%%%%%%%%%%%%%%%%%%%%%%%%%%%%%%%%%%%%%%%%%%%%%%%%
\subsection{Configurable-Depth Factoring}
\label{subsec:cohere-config-factor}

When we decrease our nearest-neighbor factoring depth from $O(\log^2 n)$
in Chapter \ref{chap:factor-polylog} to $O((\log\log n)^2)$ in
Chapter \ref{chap:factor-sublog}, we calculated a disproportionate
increase in circuit size and width from
$O(n^4)$ to $\Omega(n^6\log^2 n)$. This seems to be quite an
unfavorable depth-width tradeoff, and it is natural to ask whether
we could have some configurable depth in between
polylogarithmic and sublogarithmic that would let a quantum
systems architect choose the right tradeoff for a particular
implementation.

In this section, we provide such a configurable depth
factoring architecture by generalizing our implementation in
Chapter \ref{chap:factor-polylog}. In that chapter, we
wanted to multiply $n\times n$-bit quantum integers in parallel.
To do so, we divided them up into $\lceil n/2 \rceil$ groups of two.
In each group of two quantum integers, call them $\ket{x}$ and
$\ket{y}$, in order to get all the product bits
$\ket{x_i\cdot y_j}$ we need to generate $n^2 \ n$-bit modular residues
$2^i 2^j \bmod m$ controlled on two qubits. We then add these
down with modular multiple addition back to an $n$-bit (CSE) number,
and we are then left with $\lceil n/2 \rceil$ numbers to multiply in
the second level. This takes place in depth $O(\log n)$ and takes
width and size $O(n^3)$.
Modular exponentiation has $\lceil \log_2 n \rceil + 1$ such
levels and perform $n-1$ multiplications total, giving us the
final depth of $O(\log n)$ and size and width $O(n^4)$.

The configurable parameter is how we group quantum integers for
expansion. If instead of groups of two, we used groups of
$4$, then each multiplication would require expansion
into $n^4 \times n$-bit numbers. These would still add down to
a single $O(n)$-bit CSE number in $O(\log n)$ depth, but
would now take size and width $O(n^5)$. Modular exponentiation
would still take total depth $O(\log^2 n)$ but would now take
size and width $O(n^6)$, with hidden constants dependent on
the parameter.

We now name the configurable parameter $d$, where in each level
of modular exponentation we group quantum integers into groups of
$2^d$.

In the polylogarithmic