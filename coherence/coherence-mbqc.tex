\section{Measurement-Based Quantum Computing Background}
\label{sec:cohere-mbqc}

We will now discuss a model of quantum computing that is very different
from the circuit model, but has already provided us with tools for
parallelism in our nearest-neighbor architectures 
in Chapters \ref{chap:factor-polylog} and \ref{chap:factor-sublog}.
Measurement-based quantum computing (MBQC) is a general model which creates
a large entangled-state on a graph of qubits and performs a pattern of
measurements \cite{Raussendorf2001}.
Later measurements may depend on previous outcomes, and so
a classical controller is required to make each measurement adaptive in this
way. Each measurement reduces the size of our entangled state, and the
measurement operation proceeds
physically across our lattice of qubits, from inputs to outputs.
We will discuss here a restricted form of MBQC called one-way computing
that was proved universal by Raussendorf-Brown-Briegel \cite{Raussendorf2003}
by translation from an arbitrary $n$-qubit circuit.
This model uses only single-qubit measurements on a
regular 2D lattice. We will exclusively consider this model and refer to it
at MBQC for the rest of this section, since it is sufficient for discussing
depth optimization of quantum circuits.

MBQC is very different from the circuit model, which describes unitary
evolution by the application of quantum gates to stationary qubits.
Quantum circuits start out with a product state and then slowly build up
more and more entanglement until it is finally projectively measured at the
end. Surprisingly, both the circuit model and MBQC are equivalent, but
have a tight depth separation for quantum algorithms.

In Section \ref{subsec:mbqc-bg}, we will review the MBQC model.
In Section \ref{subsec:mbqc-par}, we will discuss the work of
Broadbent-Kashefi in automated circuit parallelization, which is a
compilation-like process for reducing circuit depth at the expense of
size and width, another quantum time-space tradeoff. These represent
upper bounds on time-space tradeoffs for mapping certain circuit classes
to a nearest-neighbor circuit with classical controller. We will compare this
to the time-space tradeoff of
other re-ordering networks for mapping circuits to a nearest-neighbor
architecture.time-space tradeoffs for low-depth factoring.
We conclude by comparing and contrasting the
circuit coherence of an MBQC pattern with its other circuit resources.

%%%%%%%%%%%%%%%%%%%%%%%%%%%%%%%%%%%%%%%%%%%%%%%%%%%%%%%%%%%%%%%%%%%%%%%%%%%%%%
\subsection{MBQC Background}
\label{subsec:mbqc-bg}

We follow the exposition of Ref.'s \cite{Broadbent2007,DaSilva2013}
In the MBQC model, quantum computation is represented by three components:
a pattern, an entanglement graph of qubits and interactions, and a
classical controller.

A \emph{pattern} is a sequence of commands which come in five types:

\begin{description}
\item[$N_i$:]
preparation of a qubit $i$ into the state $\ket{+}$.

\item[$E_{ij}$:]
entanglement of qubits $i$ and $j$ with the two-qubit gate
$\Lambda(Z)$ defined below. Note that this gate is bidirectional, so it
does not matter which qubit is the control or the target.

\item[$M^{\alpha}_i$:] single qubit measurement on qubit $i$ which
projects onto the states
$\{ \ket{\pm_{\alpha}} \}$ where

\begin{equation}
\ket{\pm_{\alpha}} = \normtwo(\ket{0} \pm e^{i\alpha}\ket{1})\}
\end{equation}

Associated with every measurement is a signal $s_i \in \mathbb{Z}_2$
which is $0$ for outcome $\ket{+_{\alpha}}$ and $1$ for outcome
$\ket{-_{\alpha}}$.

\item[$X^{s}_i$:] a dependent Pauli $X$ correction, which applies $X$ to
qubit $i$ if the signal $s$ is $1$.

\item[$Z^{t}_j$:] a dependent Pauli $Z$ correction, which applies $Z$ to
qubit $i$ if the signal $t$ is $1$.

\end{description}

A pattern is a valid executable sequence which corresponds to well-defined
quantum and classical operations if no command depends on outcomes that
are not yet measured. Patterns are executed right to left, much like
composing the matrices that make up a sequence of gates to be applied to
a quantum state (column vector). Certain operations can be parallelized
if they occur on disjoint qubits. Furthermore, the operations in the
pattern describe the quantum operations above only. Implicitly inserted
in between them are classical layers which compute the dependent signals
$s$, which may be the parity
(sum modulo 2) of multiple signals: $s = \oplus_i s_i$.

We illustrate this via a simple pattern on two qubits below.

\begin{equation}
X^{s_1}_2 M^{-\alpha}_1 E_{12} N_2 N_1
\end{equation}

In this pattern, both qubits 1 and 2 are initially prepared in the
state $\ket{+}$ and then entangled with a $\Lambda(Z)$ gate.
Qubit 1 is measured in the basis $\ket{\pm_{\alpha}}$ and its
classical outcome is stored in the signal $s_1$. Finally,
qubit 2 is corrected with a Pauli $X$ based on the outcome of
the measurement.

The entanglement graph $G = (V,E)$ defines all the two-qubit entanglement
operations (edges in $E$) between qubits (vertices in $V$). Furthermore,
there are special vertices which represent the input
qubits $I \subset V$ and the output qubits $O \subset V$. In the one-way
MBQC model that we are exclusively considering,
the geometry of the entanglement graph always has the following form
corresponding to an $n$-qubit circuit: it is a
rectangular, regular 2D lattice of $n \times D(n)$ qubits where the
leftmost column of $n$ qubits are the inputs and the rightmost
column of $n$ qubits are the outputs. An
MBQC pattern can always be standardized so that measurements and corrections
always proceed from left-to-right across this so-called cluster state graph
\cite{Raussendorf2003}.

The preparation commands are often omitted since it is implied that they
are always done for all qubits except the input.
Measurements can also be done in a basis which depends on previous
measurement outcomes. These are written as measurements which are
preceded by some $X$ and $Z$ correction, which themselves are dependent.

\begin{equation}
_t\left[M^{\alpha}_i\right]^s \equiv M^{\alpha}_i X^s_i Z^t_i =
M_i^{(-1)^s \alpha + t\pi}
\end{equation}

An MBQC pattern, since it has circuit-like properties, also consumes
circuit depth, width, and size. The depth is divided up into
preparation depth (which involves applying the quantum operations
$N_i$ and $E_{ij}$) and computation depth (which involves applying
the operations $M^{\alpha}_i$, $X^{s}_i$, $Z^{t}_i$).

A pattern can be optimized in polynomial classical time so that
all preparation and entanglement occurs first in the preparation depth,
all measurements and $X$ corrections come next in interleaved layers of
quantum and classical processing (the computation depth),
and finally all the $Z$ corrections come
last. The preparation depth is equal to the maximum degree of the
underlying graph $G$, which is always $4$. The $Z$ corrections can
always be performed last.
Therefore our depth bottleneck
comes from our measurement commands and their dependencies.

Translations between MBQC patterns and quantum circuits are most easily
done using the following (universal) basis of $\{J(\alpha), \Lambda(Z)\}$:

\begin{equation}
J(\alpha) = \normtwo \left[
\begin{array}{cc}
1 & e^{i\alpha} \\
1 & -e^{i\alpha}
\end{array}
\right]
\qquad
\Lambda(Z) = \left[
\begin{array}{cccc}
1 & 0 & 0 & 0 \\
0 & 1 & 0 & 0 \\
0 & 0 & 1 & 0 \\
0 & 0 & 0 & -1
\end{array}
\right]
\end{equation}

Note that for the special angle of $0$,
$J(0) = H$, our usual Hadamard gate.
This is universal because it contains at least one entangling two-qubit
gate, and arbitrary single qubit rotations can be implemented using
the $\{J(\alpha)\}$ basis as shown below for angles $\phi$, $\beta$, $\gamma$,
and $\delta$.

\begin{equation}
U = e^{i\phi}J(0)J(\beta)J(\gamma)J(\delta)
\end{equation}

However, note that this basis is not
fixed and finite. Further restrictions must be placed on MBQC patterns
in order to meet fault-tolerance requirements. Namely, the angles
$\alpha$ should be drawn from a finite set that that can be compiled from
a fault-tolerant basis such as Clifford+$T$ or Clifford+$Toffoli$, as
done by the quantum compilers in Chapter \ref{chap:qcompile}.

In fact, we conjecture that because of this fundamental quantum
compiling limitation, no MBQC pattern can ever have depth smaller than
the corresponding quantum circuit over a fixed, finite basis. This does,
however, open up the question of efficient quantum compiling over
the $\{\Lambda(Z), J(\alpha)\}$ basis.

%%%%%%%%%%%%%%%%%%%%%%%%%%%%%%%%%%%%%%%%%%%%%%%%%%%%%%%%%%%%%%%%%%%%%%%%%%%%%%
\subsection{Automating Circuit Parallelism with MBQC}
\label{subsec:mbqc-par}

The work by Broadbent-Kashefi introduced the application of a
\emph{measurement calculus} to transform MBQC patterns and provide a
pattern for automated parallelization of quantum circuits. This is a
classical, compilation-like procedure which takes as input a quantum
circuit and returns as output a new quantum circuit with at least the
same depth and in some cases improved depth, with a corresponding
increase in circuit size and width.

The basic
technique involves translating a unitary circuit $C$ from a certain basis
into an MBQC pattern $P$. Two optimizations are used from the measurement
calculus: standardization and signal-shifting. Standardization applies
the rules below to make sure all patterns are well-formed: all
preparation commands precede all entanglement operations, which themselves
precede all measurements and corrections. This is useful for
standardizing two patterns $P^(1)$ and $P^(2)$, which themselves may be
standardized and which are concatenated together. Such a concatenation
may occur when we are translating two concatenated unitary circuits
which may individually be easy to translate to patterns but together may
be difficult. This example is illustrated below, where $C^{(x)}$,
$M^{(x)}$, and $E^{(x)}$ correspond to correction operations, measurement
operations, and entanglement operations for pattern $P^{(x)}$.
The symbol $\rightarrow^{*}$ indicates the transformation of standardization.

\begin{eqnarray}
P^{(1)} & = & C^{(1)}M^{(1)}E^{(1)} \\
P^{(2)} & = & C^{(2)}M^{(2)}E^{(2)} \\
P^{(1)}P^{(2)} & = & C^{(1)}M^{(1)}E^{(1)}C^{(2)}M^{(2)}E^{(2)} \\
P^{(1)}P^{(2)} & \rightarrow^{*} & C^{(1)}C^{(2)}M^{(1)}M^{(2)}E^{(1)}E^{(2)} \\
\end{eqnarray}

Signal-shifting further optimizes a pattern by moving all $Z$ corrections to
the end of the pattern, where they can all be performed in parallel. On a
cluster-state graph, the preparation depth and $Z$-correction depth are then
constant. The computation depth of the pattern is dominated by the
depth of dependent measurements and $X$ corrections.

Because the cluster state graph is a \textsf{2D CCNTC} lattice, it is natural to
wonder whether the circuit translation techniques of Broadbent-Kashefi
can be used to automatically map any quantum circuit to a \textsf{2D CCNTC}
architecture. Indeed, the discovery of a constant-depth teleportation
circuit on \textsf{2D CCNTC} proceeds directly from an MBQC pattern
for long-range teleportation. This is illustrated in the next equation,
which is Equation 15 from \cite{Broadbent2007},
which is a pattern for teleporting qubit $i$ to qubit $k$.

\begin{equation}
X^{s_j}_k Z^{s_i}_k M^0_j M^0_i E_{jk} E_{ij}
\end{equation}

Furthermore, the quintessential example of a tight logarithmic separation
between the MBQC and circuit model is the parity function.
On a non-adaptive quantum circuit (one without a classical controller),
the depth of computing parity is $\Omega(\log n)$ \cite{Fang2003}.
However, as an MBQC pattern, it takes constant depth \cite{Broadbent2007}.
Indeed the pattern for parity is very similar to our circuit for unbounded
quantum fanout, which is not surprising given that the two functions are
related by conjugation of Hadamards on every qubit \cite{Moore1999}.

Lemma 7.5 from \cite{Broadbent2007} answers affirmatively that any
non-nearest-neighbor circuit can be mapped to a nearest-neighbor
circuit with constant-depth overhead and the following time-space tradeoff.
We restate it here.

\begin{proposition}{\textbf{Time-space Tradeoff for Mapping Circuits to Nearest-Neighbor} \cite{Broadbent2007}}
Let $C$ be a quantum circuit with depth $D$, size $S$, and width $W$, with $J_c$ the number
of $J(\alpha)$ gates, and $m$ the number of places where two $\Lambda(Z)$ operate
consecutively on the same qubit. Then the corresponding MBQC pattern $P$
on a cluster-state graph, with the teleportation pattern above, has depth
$D' = O(D)$ and width $W' = W + J_c + m = O(W+S)$.
\label{prop:ts-mbqc}
\end{proposition}

This gives a time-space (depth-width) tradeoff for compiled,
nearest-neighbor circuits of $D'W' = O(D(W+S))$, where $D$, $S$, and $W$
are the depth, size, and width of the original, non-nearest-neighbor circuit.

Other than a nearest-neighbor mapping, however, automated techniques cannot
provide an asymptotically lower depth for generic circuits.
There are other special classes of circuits, namely those composed entirely of
Clifford gates and an initial layer of $J(\alpha)$ gates, which can be
parallelized to $O(1)$ depth.

We now compare MBQC to two similar automated mappings, or qubit re-ordering networks,
for any $n$-qubit circuit (on \textsf{AC}).
The main application of such a re-ordering is to convert quantum circuits
to a nearest-neighbor architecture. The re-ordering network of Rosenbaum
\cite{Rosenbaum2012} has the same constant-depth overhead of $D' = O(D)$
but a quadratic width overhead of $W' = O(W^2)$.
The re-ordering network of Beals et al. \cite{Beals2012} allows
a distributed quantum computer with nodes connected in a hypergraph
topology
(equivalent to \textsf{2D CCNTCM}) to execute the same circuit with
$D' = O(D\log^2n\log\log n)$.

An MBQC pattern, when executed on a cluster-state graph, has a well-defined
circuit coherence based on Section \ref{sec:cohere-def}. For a cluster-state
graph with $n \times D$ qubits, including the input and output qubits,
corresponding to a unitary circuit on $n$-qubits
and depth $D$, the circuit width is $W = nD$. The entire lattice starts out
in an entangled state, and measurements proceed column-by-column
left-to-right from the input qubits to the output qubits. Therefore,
the circuit coherence is $Q = D\cdot W = O(n^2D^2)$ whereas $S = O(nD)$, and in fact, there is no
asymptotic separation between circuit coherence and the depth-width product.