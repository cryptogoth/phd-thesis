\section{A New Problem: Hamiltonian Simulation}
\label{sec:cohere-hs-bg}

The original proposal for quantum computing of Feynman in 1982 \cite{Feynman1982}
took the form of simulation. He pointed out that one problem that
quantum physical systems would be inherently better at than classical systems
would be simulating other quantum systems. Today, this is still an interesting
problem for the field. While Shor's factoring algorithm, and other algorithms
which are inspired by processing digital information, exhibit great speedups,
they are also far out of reach of today's experimental capabilities. To
factor a 4096-bit key in the most conservative implementation, with only
one layer of error-correction, would still take tens of thousands of qubits.
In contrast, the problem of simulating the time evolution of a quantum system of 40 particles
is within the reach of experiments within the next decade and already
far surpasses the abilities of all but the most powerful classical supercomputers
today. A seminal work by Lloyd on universal quantum simulation still provides a
good introduction to the field \cite{Lloyd1996}.

Simulation is inter-related to computation. It
plays a complementary role to digital computing, much like the
now-defunct field of analog computing and other analog simulations. Instead
of digitizing variables and representing them indirectly in binary
encodings, abstracting away all physical effects, simulations encode variables
in other analog variables, where the physical state of the analog model
corresponds in some human-readable way to the simulated state of the
original variable. The problem we will discuss is how to approximate
the continuous evolution of a time-independent Hamiltonian using
a discretized-time, circuit-based approach of applying gates to a quantum
state.

Simulation Hamiltonians also has theoretical interest in that there are some
quantum physical systems which we can describe in a model
but can't even simulate quantum mechanical [need citation]. In that case,
there is an interesting twist in our expectations for a mathematical
to be \emph{predictive}. Furthermore, due to the "no fast-forwarding" theorem,
it turns out that even in a quantum simulation, we can't speed up time:
simulating a many-body quantum system for time $t$ still takes time
super-linear in $t$, although we can make it arbitrarily close to
linear \cite{Berry2005}.

Therefore, Hamiltonian simulation remains an interesting problem from the
time of Feynman until today, both for its practical applications, its
experimental feasibility, and its theoretical implications.

%%%%%%%%%%%%%%%%%%%%%%%%%%%%%%%%%%%%%%%%%%%%%%%%%%%%%%%%%%%%%%%%%%%%%%%%%%%%%%
\subsection{Hamiltonians for \\ Computer Scientists}

In a quantum mechanical system, observable quantities are eigenvalues
associated with Hermitian matrices. A Hamiltonian is the operator associated
with the energy of a system, which is a dual to time, similarly to how
position and momentum are duals in the famous Heisenberg uncertainty principle.
Therefore, we know
how a physical system evolves over time $t$ from its Hamiltonian $H$. For
an $n$-qubit system, $H$ has size $2^n \times 2^n$.
Starting from an initial state $\ket{\psi_0}$, we can simulate its time
evolution with the unitary operator which is the
matrix exponential of $H$ scaled by $t$, resulting in a final state $\ket{\psi_f}$.
The form of this matrix exponential is due to solutions of the Schrodinger
equation.

\begin{equation}
\ket{\psi_f} = e^{-iHt} \ket{\psi_0}
\end{equation}

In analogy to 
the Taylor expansion of the function $e^x$, which exponentiates a number,
we can define the Taylor expansion of the matrix $e^{-iHt}$ which is itself
a $2^n\times 2^n$ matrix.

\begin{equation}
e^{-iHt} = \sum_{j=0}^{\infty} \frac{(-iHt)^t}{j!}
\end{equation}

Since this is an infinite sum which converges to the true matrix
$e^{-iHt}$, we can take only the initial $k$ terms to get an approximation
which is accurate up to $O(t^k)$ terms.

%%%%%%%%%%%%%%%%%%%%%%%%%%%%%%%%%%%%%%%%%%%%%%%%%%%%%%%%%%%%%%%%%%%%%%%%%%%%%%
\subsection{The Simulation Procedure}

To state the problem more formally, suppose that you have an initial
quantum state $\ket{\psi_0}$ of $n$ qubits
and you would like to evolve it according to a Hamiltonian $H$ for time
$t$ to get
a final state $\ket{\psi_f}$. You can then operate on this simulated final state
to get some insight into the system defined by $H$, for example, by finding
its ground state energy.
To
enact this simulation, we perform a unitary which is equivalent to the
exponential $e^{-iHT}$.

\begin{equation}
\ket{\psi_f} = e^{-iHT} \ket{\psi_0}
\end{equation}

However, $H$ (and therefore $U_H$)
is an exponentially large matrix in the number of qubits you
are trying to simulate. If you were to generically compile $U_H$, 

We use the following variant of the Trotter formula \cite{Aharonov2003}.
Let $H = \sum_{m=1}^M H_m$ be the Hamiltonian we are trying to simulate, where
each $H_m$ is Hermitian and all have bounded norm, $||H||,||H_m|| \le \Lambda$.
Define an approximation to the time evolution of $H$ over time $t$ by using the
exponentials of its decomposed terms $H_m$ over a small
timestep $\delta$ as in Equation
\ref{eqn:hamsim-ut}.

\begin{eqnarray}
U_\delta & = & [e^{-iH_1 \delta}\cdot e^{-iH_2 \delta} \cdots e^{-iH_M \delta}]\cdot\\
         &   & [e^{-iH_M \delta}\cdot e^{-iH_{M-1} \delta} \cdots e^{-iH_1 \delta}]
\label{eqn:hamsim-ut}
\end{eqnarray}

Then Equation \ref{eqn:hamsim-trotter} bounds the error of approximation
as a function of the simulation timestep $\delta$.

\begin{equation}
||U_{\delta}^{\lfloor \frac{t}{2\delta} \rfloor} - e^{-iHt}||
\le O(M\cdot\Lambda \cdot \delta + M\Lambda^3 t \delta^2)
\label{eqn:hamsim-trotter}
\end{equation}

For fixed total evolution time $t$, Hamiltonian term count $M$,
and norm bound $\Lambda$, we can decrease the error arbitrarily as
$\delta$ goes to zero. In practice, we can make $\delta$ a polynomially
small timestep.

%%%%%%%%%%%%%%%%%%%%%%%%%%%%%%%%%%%%%%%%%%%%%%%%%%%%%%%%%%%%%%%%%%%%%%%%%%%%%%
\subsection{Possibilities for Parallelization}

Berry et al. showed that if we restrict our Hamiltonian $H$ to be
$d$-sparse, we can achieve a bound for the total
number of terms in our decomposition which is $M=6d^2$ \cite{Berry2005},
an improvement over the previous results by Aharonov and Ta-Shma
\cite{Aharonov2003}.
Furthermore, they showed that sublinear simulation time was impossible.
This gives indirect evidence that for a Hamiltonian of this form,
there is no way to decompose it into terms whose exponentials
pairwise commute, since then
it would be possible to parallelize the simulation of the
exponentials $e^{-iH_m t}$ using
the technique of H{\o}yer and {\v S}palek in
Section \ref{sec:parallel}.

Since we cannot parallelize the application of the exponentials themselves,
what about optimizing the depth of their compilations, following our
discussion in \ref{subsec:compile}? In the decomposition above, each $H_m$
was 1-sparse. An interesting open question is whether this structure of
$H_m$ would allow us to optimize the two-level decomposition of
Aho and Svore \cite{Aho2003}.

Now the sun is coming up again, so it's time to wrap things up.