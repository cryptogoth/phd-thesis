\section{Definition of Circuit Coherence}
\label{sec:cohere-def}

Usually quantum circuits neglect to draw identity gates. When a bare
quantum wire appears, what is meant is that the qubit maintains its
coherent state until the next non-identity gate comes along to transform it.
However, most quantum circuits are drawn at a logical qubit level,
assuming no errors occur and a coherent state is maintained.
So far we have maintained that abstraction in this dissertation by studying
quantum compiling and quantum architecture independently from
quantum error correction. In this chapter, we move closer to
this abstraction barrier by studying
the effort to maintain a coherent quantum state at the logical qubit level
only for those qubits and only during those timesteps which are useful
for computation.
We quantify this effort with a new circuit resource called \emph{circuit coherence}.

First, we will define what we mean by an \emph{influencing} gate and
a \emph{qubit instance}
in Section \ref{subsec:cohere-entangle}. Then we will build upon this
to define a computational subset of qubits in every
timestep, in Section \ref{subsec:cohere-subset}. Finally,
in Section \ref{subsec:cohere-algo}, we will use
the previous two definitions to provide an algorithm for
computing reachability and therefore the computational subset for a circuit.
This in turn lets us define the resource circuit coherence and
describe its relationship to the other circuit resources: depth, size, and
width. 

%%%%%%%%%%%%%%%%%%%%%%%%%%%%%%%%%%%%%%%%%%%%%%%%%%%%%%%%%%%%%%%%%%%%%%%%%%%%%%
\subsection{Influence and Qubit Instances}
\label{subsec:cohere-entangle}

In the rest of this chapter, we only consider pure quantum states, which are
sufficient for describing circuit coherence. While in reality
mixed quantum states may occur on implementations of our circuits, it
will not affect our definitions here. We will not precede every
instance of a quantum state with the word ``pure,'' but it should be
assumed.

The basic unit of circuit coherence is a \emph{qubit instance} which is
a qubit at a particular timestep. There are $D\cdot W$ qubit instances
in a circuit, but only some of them are computationally useful, depending
on how they interact with each other via gates. We will refer to
qubits and qubit instances with the same label ($u$ and $v$) if a particular
timestep is implied. Qubit instances which
need to be maintained by error-correction are said to \emph{influence}
the result of the output qubits in the last timestep, what we call
the \emph{output qubit instances}. We will make this more formal
as we build up to an algorithm for calculating coherence.

An \emph{entangled} pure quantum state is one which cannot be expressed as the
tensor product of two smaller states. An unentangled pure quantum state
is also known as a \emph{product} state. 
We call the gate $E_{uv}$ \emph{entangling} between two qubits $u$ and $v$
(and between $u$ and $L = V - \{u\}$) given the input states $\rho^{u}$ and $\rho^{L}$ if
the new global state $\rho^{V}$ after applying $E_{uv}$ cannot be expressed as a product state
$\sigma^{u} \otimes \sigma^{L}$.

However, it is useful to characterize a gate as ``potentially entangling''
as a simplification, so that we can reason about the circuit without
worrying about the input states to each gate.
%
\begin{definition}
A gate $E_{uv}$ is \emph{influencing}, or potentially entangling, if
there exists any pure, product quantum input state $\rho^{V} = \rho^{u} \otimes \rho^{L}$
(where $v \in L$) such that the output state $\sigma^{V}$ is entangled.
We can also say that the qubit instance $u$ influences
the qubit instance $v$, or $u$ influences
$L$, in the timestep in which $E_{uv}$ occurs.
We can also say $E_{uv}$ \emph{connects} influence from $u$ to $v$.
\end{definition}

Influencing gates are the means by which any qubit instance can influence
the output qubit instances. This will tell us when we need to \emph{begin}
maintaining a qubit's state, which is after a gate that is
\emph{influencing}.
But how do we know when to \emph{stop} maintaining a qubit's state?
For this purpose of conserving our error-correcting effort, we define
a related concept: an operator which is known to \emph{disconnect}
a qubit instance from influencing other qubit instances.

\begin{definition}
An operator $M$ on a pure $n$-qubit input state $\rho$ is 
\emph{disconnecting} with respect to a given input state $\rho^{V}$,
if the resulting output state is product:
$\rho^{u} \otimes \rho^{L}$ where $L = V - \{u\}$.
We say $u$ is the
qubit instance which has been disconnected.
\end{definition}

We only consider single-qubit and two-qubit disconnecting operators.
The only single-qubit disconnecting operator that we consider
is a projective measurement operator. This projector is always disconnecting,
regardless of the input state.

The disconnecting nature of a gate
may not be apparent just by examining a circuit locally.
To make things more difficult, a disconnecting gate is also often
influencing (for example, CNOT).
However, a quantum algorithm designer able to specify a circuit in terms of
single-qubit and
two-qubit gates often knows when gates are connecting influence and when
they are disconnecting it.
Moreover, the algorithm designer knows which qubits are garbage and that reversing
part of a quantum circuit will uncompute these garbage ancillae back to $\ket{0}$.
In those cases, it is known by design which gates are disconnecting.

This is the case for
quantum circuits in a certain layered form which we describe in
Section \ref{sec:cohere-tradeoff},
of which the 
well-known QFT and factoring circuits are special cases.
As an overestimate of circuit coherence, we can also consider all
two-qubit gates influencing in the worst case
with
only single-qubit projective measurement considered as disconnecting.
However, this will usually
not give an upper-bound separation between coherence and the depth-width product.
We only consider two-qubit gates that are either influencing or disconnecting
for some input states; all other two-qubit gates are a tensor product of
single-qubit gates and will be treated as such.

%%%%%%%%%%%%%%%%%%%%%%%%%%%%%%%%%%%%%%%%%%%%%%%%%%%%%%%%%%%%%%%%%%%%%%%%%%%%
\subsection{Reachability and Computational Subsets}
\label{subsec:cohere-subset}

We refer back to our definition of a quantum circuit on
\textsf{CCNTC}, which is represented by a graph $G = (V,E)$,
with input qubits $I \subseteq V$ and output qubits $O \subseteq V$, and a
classical controller. In particular, the set of all qubits is $V$,
and its size is $|V|=W$, the circuit width.
Our notion of circuit coherence will not depend
on the modules defined in \textsf{CCNTCM}.

%%%%%%%%%%%%%%%%%% DEFINITION
\begin{definition}{\textbf{Influencing Paths.}}
We denote by $E^{(i)}_{uv}$ a two-qubit gate which acts in
timestep $i$ which is influencing and \emph{not} disconnecting for its current states
on $\rho^{u}$ and $\rho^{L}$ for $L = V - \{u\}$, $v \in L$.
An \emph{influencing path} of gates from qubit $u$ in timestep $i_1$ to
qubit $v$ in timestep $i_n$ is
any sequence of influencing (and not disconnecting)
gates $(E^{(i_1)}, E^{(i_2)}, \ldots, E^{(i_n)})$
where the following conditions are met:

\begin{enumerate}
\item
$E^{(i_1)}$ operates on qubit $u$ and $E^{(i_n)}$ operates on qubit $v$.

%\item
%influencing paths always move forward in time: $i_j > i_{j-1}$.
\item
any two consecutive gates in the sequence $(E^{(i_j)},E^{(i_{j+1})})$
act on a common qubit $w$.
\item
any two consecutive gates in the sequence either occur in
consecutive timesteps ($i_j = i_{j-1} + 1$) or are only separated by
gates which are not single-qubit measurements on $w$ in intervening timesteps $i_j < i < i_{j+1}$.
%\item
%every gate $E^{(i_j)}$ encountered in the sequence satisfies the
%following two conditions:

%\begin{enumerate}
%\item it is influencing if
%the path exits it in the forward direction ($i_{j+1} = i_j + 1$)
%\item it is disentangling if the path exits it in the backward direction
%($i_{j+1} = i_j - 1$).
%\end{enumerate}
\end{enumerate}

\end{definition}

%%%%%%%%%%%%%%%%
We can now make our definition of influence more formal.

\begin{definition}{\textbf{Influence and Reachability.}}
A qubit instance $u$ at timestep $i$ is \emph{reachable} from another qubit
instance $v$ in
another (possibly the same) timestep $i'$ if there is some path of influencing
gates that connects them. Conversely, we also say $v$ influences $u$.
Influence is directed and asymmetric; qubit
instances can only be influenced from qubit instances in earlier timesteps.
\end{definition}

Influence also applies to classical measurement outcomes which are
connected by the classical controller.
If one qubit instance occurs right after a single-qubit measurement
(that is, its state contains a classical measurement outcome), any other
qubit instance (including another classical measurement outcome) which
depends on that outcome is influenced by it.

We now define a standard form for circuit in which circuit coherence will be
well-defined.

\begin{definition}{\textbf{Standard form circuits for calculating circuit coherence.}}
Standard form circuits must have the following properties:

\begin{description}
\item[output qubits $O \subseteq V$:] These qubits are semantically defined as
containing the useful outputs of a quantum circuit. They do not have to be
projectively measured. They may, for example, be the control for a
later coherent measurement when cascaded with another quantum circuit.
\item[input qubits $I \subseteq V$:] These qubits are prepared in a 
classical product state (the computational basis)
and are all reachable in timestep $1$ from the
output qubits in timestep $D$.
\item[ancillae qubits:] these are prepared in the product state of all $\ket{0}$'s.
\end{description}
\end{definition}.

We will assume all circuits from now on are in standard form.

\begin{definition}{\textbf{Computational subset, computational set, computational state.}}
The computational subset in timestep $i$ (abbreviated $L_i$) is the subset of
the qubit instances in that timestep
which are reachable from the output qubits $O$.
The \emph{computational set} is the set of computational subsets across
all timesteps.

\begin{equation}
M = \{L_1, L_2, \ldots, L_D \}
\end{equation}
\end{definition}

The computational set are those qubit instances on which an entangled, coherent
quantum state (which we will call the computational state) evolves over time from
the initial preparation of the input qubits $I \subset V$ in timestep $1$
until the output qubits $O \subset V$ are
measured in timestep $D$.
It is measured in qubit instances,
and potentially grows (if influencing, non-disconnecting gates are applied), shrinks
(if disconnecting gates including measurements are applied) or stays the same in size
in every timestep. We note the following
relationships for well-formed circuits.

\begin{equation}
L_1 = I \qquad L_D = O \qquad L_i \subseteq V
\end{equation}

The computational subset can be computed in two passes through
the quantum circuit, one forward to determine reachability and one backward
to determine the computational subset. We can perform this algorithm efficiently on a
layered quantum circuit as defined in Section \ref{subsec:cohere-lqc},
since it is known which gates are disconnecting in that case.

In each timestep $i$, we partition all $W$ qubits in $V$ into disjoint
subsets. We denote these other
(possibly non-computational) qubit subsets as $\{\tilde{L}^{(j)}_i\}$,
of which one is the same as the current
computational subset $L_i$.

This partitioning, like $L_i$, is updated in
every timestep.

Following the definition of \textsf{2D CCNTCM}, each qubit subset is a
contiguous subgraph of the main graph $G$. No two qubit subgraphs share any vertices,
but all vertices are covered and the subgraphs may share edges. All influencing/disconnecting gates $E^{(i)}_{uv}$
that occur during a timestep $i$ are contained in the set $G_i$.

Qubit subsets may potentially share common qubits and become connected by an influencing gate
in a past timestep $i' < i$ from the current timestep $i$.
We keep track of all qubit subsets at a given timestep $i$
in a collection $M_i = \{\tilde{L}^{(j)}_i \}$.

%%%%%%%%%%%%%%%%%%%%%%%%%%%%%%%%%%%%%%%%%%%%%%%%%%%%%%%%%%%%%%%%%%%%%%%%%%%%%%
\subsection{The Coherence Calculation Algorithm}
\label{subsec:cohere-algo}

The following algorithm takes as input a quantum circuit in standard form
with graph $G = (V,E)$, gates in groups $G_i$
over the basis of single-qubit and two-qubit gates
$U(2) \cup U(4)$ including single-qubit measurements in the $Z$-basis.
Gates in group $G_i$ execute in timestep $i$.
The algorithm returns as output the computational subset at every timestep $\{ L_1, L_2, \ldots, L_D \}$.

The data type of qubit subset $\tilde{L}^{(j)}_i$ is the tuple
$(\tilde{V}, P)$ where:

\begin{itemize}
\item $\tilde{V} \subseteq V$ are the qubits (qubit instances) in the subset
\item $P$ is a pointer to the parent qubit subset in timestep ${i-1}$, initially \textsc{NULL}.
\end{itemize}

When we assign one qubit subset to another, we assume we are only assigning
the qubits part of the tuple, $\tilde{V}$.

%%%%%%%%%%%%%%%%% ENUMERATE 1
\begin{enumerate}
\item
Initialize the following:
\begin{itemize}
\item
$L_1 \leftarrow \{ I \}$.
\item
$\tilde{L}^{(j)}_1 = v_j \in V \setminus I$ (all non-input qubits $v_j$ get their own subset)
\item
$M_1 = \left( \bigcup_j \{ \tilde{L}^{(j)}_1 \} \right) \cup \{ L_1 \} $.
\end{itemize}

\item
In timestep $i \in \{2, \ldots, D \}$:

%%%%%%%%%%%%%%%%%%%%%%%%%%%%% ENUMERATE 2
\begin{enumerate}

\item
Compute the classical description of the state on all the qubits $\rho^{V}_i$
from the state $\rho^{V}_{i-1}$
given $M_{i-1}$.
This is assumed to be efficient (e.g. for layered quantum circuits). 
\item
Initialize $M_i \leftarrow \{\}$.
\item
Create two sets:
\begin{itemize}
\item $T_e$ which contains every two-qubit
gate $E_{uv} \in G_i$ that is influencing and not disconnecting
\item $T_d$ which contains every two-qubit
gate $E_{uv} \in G_i$ that is disconnecting,
along with all single-qubit measurements $M_u \in G_i$.
\end{itemize}

\item
For every qubit subset $\tilde{L}^{(j)}_{i-1} \in M_{i-1}$:

%%%%%%%%%%%%%%%%%%%%%%%%%%%%%%%%%%%%%%%% ENUMERATE 3
\begin{enumerate}
\item \textbf{Connecting Case.} Check whether $\tilde{L}^{(j)}_{i-1}$ contains a qubit acted upon by a
$E_{uv} \in T_e$.

%%%%%%%%%%%%%%%%%%%%%%%%%%%%%%%%%%%%%%%%%%%%%%%%%%%% ENUMERATE 4
\begin{enumerate}
\item If it does, call that qubit
$u \in \tilde{L}^{(j)}_{i-1}$.
Check whether $v$ is in any other qubit subset
(call it $\tilde{L}^{(j')}_{i-1}$).

% Too deeply nested
%%%%%%%%%%%%%%%%%%%%%%%%%%%%%%%%%%%%%%%%%%%%%%%%%%%%%%%%%%%%%%%% ENUMERATE 5
%\begin{enumerate}
\item
If it is, create a new qubit subset
$\tilde{L}^{(j)}_{i}$ equal to the union of the two qubit subsets from
step $i-1$:

\begin{equation*}
\tilde{L}^{(j)}_{i} \leftarrow \tilde{L}^{(j)}_{i-1} \cup \tilde{L}^{(j')}_{i-1}
\end{equation*}

Update the parent pointers of $\tilde{L}^{(j)}_{i}$ accordingly.

%\item
%If $v$ is \emph{not} in any other qubit subset for timestep $i-1$,
%then simply add it to a new qubit subset for timestep $i$. Note that it will
%not have the tag \textsc{Computation Subset}.

%\begin{equation*}
%\tilde{L}^{(j)}_{i} = \{v\}
%\end{equation*}
%\end{enumerate}
%%%%%%%%%%%%%%%%%%%%%%%%%%%%%%%%%%%%%%%%%%%%%%%%%%%%%%%%%%%%%%% ENUMERATE 5

\item
Add the current qubit subset to the current timestep's set of qubit subsets $M_i$.

\begin{equation*}
M_i \leftarrow M_i \cup \{ \tilde{L}^{(j)}_{i} \}
\end{equation*}

\end{enumerate}
%%%%%%%%%%%%%%%%%%%%%%%%%%%%%%%%%%%%%%%%%%%%%%%%%%%% ENUMERATE 4

\item \textbf{Disconnecting Case.} Check whether $\tilde{L}^{(j)}_{i-1}$ matches the following two cases
and take the corresponding actions.

%%%%%%%%%%%%%%%%%%%%%%%%%%%%%%%%%%%%%%%%%%%%%%%%%%%% ENUMERATE 4
\begin{enumerate}
\item If $\tilde{L}^{(j)}_{i-1}$ contains two qubits $u$ and $v$
acted upon by some
$E_{uv} \in T_d$, then check whether $E_{uv}$ is
disconnecting between any partitioning of $\tilde{L}^{(j)}_{i-1}$ into two
subsets $V_1 \ni u$ and $V_2 \ni v$.

% Too deeply nested
%%%%%%%%%%%%%%%%%%%%%%%%%%%%%%%%%%%%%%%%%%%%%%%%%%%%%%%%%%%%%%%% ENUMERATE 5
%\begin{enumerate}
\item
If it does, add these two subsets to our collection $M_i$.
Set their parent pointers to $\tilde{L}^{(j)}_{i-1}$.

\begin{equation*}
M_i \leftarrow M_i \cup \{ V_1, V_2 \}
\end{equation*}

\item
Otherwise, just set the current subset $\tilde{L}^{(j)}_{i} \leftarrow \tilde{L}^{(j)}_{i-1}$
with the appropriate parent pointer, and add it.

\begin{equation*}
M_i \leftarrow M_i \cup \{ \tilde{L}^{(j)}_{i} \}
\end{equation*}

%\end{enumerate}
%%%%%%%%%%%%%%%%%%%%%%%%%%%%%%%%%%%%%%%%%%%%%%%%%%%%%%%%%%%%%%%% ENUMERATE 5

\item
If $\tilde{L}^{(j)}_{i+1}$ contains a qubit $u$ acted upon by some $M_u \in T_d$,
then create two new qubit subsets. One just removes the qubit $u$
from the current qubit subset.
%, inheriting the tag \textsc{Computational Subset}
%if present.
The other is a single-qubit subset consisting
only of $u$.

\begin{eqnarray*}
\tilde{L}^{(j)}_{i} & \leftarrow & \tilde{L}^{(j)}_{i-1} - \{u\} \\
\tilde{L}^{(j')}_{i} & \leftarrow & \{ u \}
\end{eqnarray*}

Set the parent pointer of $\tilde{L}^{(j)}_{i}$ accordingly.
Add these to our collection.

\begin{equation*}
M_i \leftarrow M_i \cup \{ \tilde{L}^{(j)}_{i}, \tilde{L}^{(j')}_{i} \}
\end{equation*}

\end{enumerate}
%%%%%%%%%%%%%%%%%%%%%%%%%%%%%%%%%%%%%%%%%%%%%%%%%%%%% ENUMERATE 4

\end{enumerate}
%%%%%%%%%%%%%%%%%%%%%%%%%%%%%%%%%%%%%%%% ENUMERATE 3

\item For every qubit subset $\tilde{L}^{(j)}_{i+1} \in M_{i-1}$ not
operated upon by any of the previous steps, copy it unmodified 
as $\tilde{L}^{(j)}_i$ into
$M_i$ with the appropriate parent pointer.

\end{enumerate}
%%%%%%%%%%%%%%%%%%%%%%%%%%%%% ENUMERATE 2

\item
Do a backward pass from the outputs to the inputs to discover the computational subset $L_i$ in each timestep.

\begin{enumerate}
\item
Verify that the output qubits exactly correspond to one of the qubit subsets
in $M_D$. Call this $L_D$.
\item
Initialize the computational set $M \leftarrow \{ L_D \}$.
\item
Working backwards for timestep $i$ in $(D, D-1, D-2, \ldots, 3, 2)$:
\begin{enumerate}
\item
Find the parent(s) of $L_i$. Create a new set that is the union of them called $L_{i-1}$
and add them to $M$.
\begin{equation}
M \leftarrow M \cup \{L_{i-1}\}
\end{equation}
\end{enumerate}

\item Verify that $L_1 = I$ are exactly the input qubits.

\item Output $M = \{L_1, \ldots, L_D\}$. This is the computational (sub)set of qubits.
\end{enumerate}

\end{enumerate}
%%%%%%%%%%%%%%% ENUMERATE 1

\begin{definition}{\textbf{Instantaneous coherence}.}
Instantaneous coherence $Q_i$ is the size of the computational subset
(in qubits) in timestep $i$. From the algorithm above,
\begin{equation}
Q_i = |L_i|\text{.}
\end{equation}
\end{definition}

\begin{definition}{\textbf{Circuit coherence}.}
Circuit coherence $Q$ is the sum of the computation subset size (in qubits)
over all $D$ timesteps of a quantum circuit's execution. It is measured
in qubit-timesteps, which is the amount of error-correcting effort to
maintain the coherent state of one logical qubit for one timestep of a circuit.

\begin{equation}
Q = \sum_{i=1}^D Q_i = \sum_{i=1}^D |L_i|
\end{equation}
\end{definition}

We will say that a circuit has greater coherence than another circuit if
its $Q$ has a higher value. This should not be confused with the
meaning of coherence as resisting decoherence.

The following relationships hold with other circuit resources.

\begin{equation}
D \le S \le Q \le D\cdot W
\end{equation}

The first inequality holds in the least parallel case, each of $S$ gates is executed in sequence
and $S=D$. The second inequality holds in the least coherent case, when of $S$ gates
either connects or disconnects another qubit from the computational subset in every timestep, and
there are no identity wires within the circuit. The third inequality holds in the
most coherent case, all of $W$ qubits are part of the computational subset for each of $D$ timesteps.

As an example, we can bound the circuit coherence of modular multiplication of $2\times n$-bit
CSE numbers, as described in Section \ref{subsec:mma}. The overall width is $W = O(n^3)$ and
depth is $D=O(\log n)$, so the coherence is upper-bounded by $O(n^3\log n)$. We will suggest possible
ways to reduce this resource in Section \ref{sec:cohere-tradeoff}.
