\section{Definition of Circuit Coherence}
\label{sec:cohere-def}

Usually quantum circuits neglect to draw identity gates. When a bare
quantum wire appears, what is meant is that the qubit maintains its
coherent state until the next non-identity gate comes along to transform it.
However, most quantum circuits are drawn at a logical qubit level,
assuming no errors occur and a coherent state is maintained. While
so far we have maintained that abstraction in this thesis by studying
quantum compiling and quantum architecture independently from
quantum error correction, we acknowledge it as an important area for
optimization and future study. In this chapter, we peek beneath
this abstraction barrier to study
the effort to maintain a coherent quantum state throughout a circuit
by defining a new circuit resource called \emph{circuit coherence}.

First, we will define what we mean by an entangling (or disentangling) gate
in Section \ref{subsec:cohere-entangle}. Then we will build upon this
to define a layer width, or a computational subset of qubits in every
timestep, in Section \ref{subsec:cohere-subset}. Finally,
in Section \ref{subsec:cohere-algo}, we will use
the previous two definitions to provide an algorithm for
computing reachability and therefore the computational subset for acircuit.
This in turn lets us define the resource circuit coherence and
describe its relationship to the other circuit resources: depth, size, and
width. 

%%%%%%%%%%%%%%%%%%%%%%%%%%%%%%%%%%%%%%%%%%%%%%%%%%%%%%%%%%%%%%%%%%%%%%%%%%%%%%
\subsection{Entangling Gates}
\label{subsec:cohere-entangle}

An entangled quantum state is one which cannot be expressed as the
tensor product of two smaller states. This does not depend on what basis
we consider for the smaller states. Using the density operator formalism,
we can say that a quantum state over two subsystems $A$ and $B$ is
entangled if tracing over one of the subsystem \emph{does not} yield the other subsystem
as a reduced density matrix.

\begin{equation}
\rho^{AB} \text{ entangled }
\iff \left( \tr_{A}(\rho^{AB}) \ne \rho^{B} \right) \land
%\left(\tr_{B}(\rho^{AB}) \ne \rho^{A}
\end{equation}

A general density matrix for a state across two subsystems $A$ and $B$ can be
written as
\begin{equation}
\rho^{AB} = \sum_{i,i',j,j'} p_{ii'jj'} \ket{a_i}\bra{a_{i'}} \otimes \ket{b_j}\bra{b_{j'}}
\end{equation}
where $\ket{a_i},\ket{a_{i'}}$ are any two states on $A$ and
$\ket{b_j},\ket{b_{j'}}$ are any two states on $B$.

We review here that the trace is a linear operator which distributes across
a general density matrix.

\begin{equation}
\tr(\rho^{AB}) = \sum_{i,i',j,j'} p_{ii'jj'} \tr (\ket{a_i}\bra{a_{i'}} \otimes \ket{b_j}\bra{b_{j'}})
\end{equation}

A reduced density matrix for a particular term is obtained by tracing out
one subsystem.

\begin{equation}
tr_A(\ket{a_i}\bra{a_{i'}} \otimes \ket{b_j}\bra{b_{j'}}) = \tr(\ket{a_i}\bra{a_{i'}}) \ket{b_j}\bra{b_{j'}}
\end{equation}

Consider a two-qubit gate $E_{uv}$ on single qubits $u$ and $v$ which exist
in a larger system.
Without loss of generality, we assume that $u$ is
not entangled with some other, possibly multi-qubit,
state on another set of vertices $L \ni v$.
The total state on $u \cup L$ is called $\rho^{u}\otimes \rho^{L}$.

We call the action of $E_{uv}$ on this combined state a new state $\sigma^{uL}$:

\begin{equation}
\sigma^{uL} = E_{uv} (\rho^{u}\otimes \rho^{L}) E^{\dagger}_{uv}
\end{equation}

We call the gate $E_{uv}$ \emph{entangling} between $u$ and $v$
(and between $u$ and $L$) given the states $\rho^{u}$ and $\rho^{L}$ if
the new state after applying $E_{uv}$ is entangled, which corresponds to
the condition below.

\begin{equation}
\tr_{u}( \sigma^{uL} ) \ne \rho^{L}
\end{equation}

More generally, if the above condition is true,
we call $E_{uv}$ entangling for any multi-qubit
states that $V_1 \ni u$ and $V_2 \ni v$ before the application of $E_{uv}$.

Now consider a slightly different setting, where $L$ contains both $u$ and $v$
which are entangled in a larger state $\rho^{L}$. We denote by $\sigma^{L}$ the resulting
state after applying $E_{uv}$ to $\rho^{L}$.

\begin{equation}
\sigma^{L} = E_{uv} (\rho^{L}) E^{\dagger}_{uv}
\end{equation}

We call the $E_{uv}$ \emph{disentangling} between $u$ and $L$ (without loss
of generality) if after its action on the state $\rho^{L}$ it is
separable into the product state $\sigma^{L} = \sigma^{u} \otimes \sigma^{L \\ \{u\}}$,
which corresponds to the following condition:

\begin{equation}
\tr_{u}(\sigma^{L}) = \sigma^{L - \{u\}}
\end{equation}

More generally, we can say the $E_{uv}$ is disentangling for any two
larger subsystems $V_1 \ni u$ and $V_2 \ni v$, if it is disentangling
for any pairs of qubits $(u',v')$ such that $u' \in V_1$ and $v' \in V_2$
for a particular state $\rho^{V_1 V_2}$.
Operationally, it is
usually more difficult to show that a gate is disentangling.

Given these definition, a gate $E_{uv}$
that is entangling in the forward time direction on input state $\rho$
is disentangling in the backward time direction ($E^{\dagger}_{uv}$) on output
state $\sigma = E^{\dagger}_{uv}\rho E_{uv}$. When we do not specify a
time direction, an entangling gate $E_{uv}$ is entangling in the forward
direction and disentangling in the backward direction, and vice versa
for a disentangling gate.

This definition for entangling or disentangling quantum gates
depends on knowing the actual states before these gates are applied.
Therefore, the entangling or disentangling nature of a gate
may not be apparent just by examining a circuit locally,
but may require simulation of the entire circuit. This gives an operational
definition for identifying entangling/disentangling quantum gates, but it does not give
a compact, theoretical description that can be applied generically. This is
currently a drawback of our definition, especially for characterizing the
behavior of new quantum algorithms that are not yet well-studied.

However, a quantum algorithm designer able to specify a circuit in terms of single-qubit and
two-qubit gates often knows when gates are entangling or disentangling.
Moreover, the algorithm designer knows which qubits are garbage and that reversing
part of a quantum circuit will uncompute these garbage ancillae back to $\ket{0}$.
This is
the case for well-known quantum algorithms such as the QFT or factoring, or
for quantum circuits in a certain layered form which we describe in Section \ref{sec:cohere-tradeoff}.
In fact, we will rely on this layered form when performing
calculations later in this section.
As an overestimate, we can also consider all two-qubit gates entangling in the worst case, and
only single-qubit projective measurement as disentangling.
We only consider two-qubit gates that are either potentially entangling or disentangling
for some input states; all other two-qubit gates are a tensor product of
single-qubit gates and will be treated as such.

%%%%%%%%%%%%%%%%%%%%%%%%%%%%%%%%%%%%%%%%%%%%%%%%%%%%%%%%%%%%%%%%%%%%%%%%%%%%%%
\subsection{Reachability and Computational Subsets}
\label{subsec:cohere-subset}

We refer back to our definition of a quantum circuit on
\textsf{CCNTC}, which is represented by a graph $G = (V,E)$ and a
classical controller. In particular, the set of all qubits is $V$,
and its size is $|V|=W$, the circuit width.
Our notion of circuit coherence will not depend
on the modules defined in \textsc{CCNTCM}.

%%%%%%%%%%%%%%%%%% DEFINITION
\begin{definition}{\textbf{Entangling Paths}}
We denote by $E^{(i)}_{uv}$ a two-qubit gate which acts in
timestep $i$ which is either entangling or disentangling for its current states on $\ket{u}$ and $\ket{v}$.
An \emph{entangling path} of gates from qubit $u$ in timestep $i_1$ to
qubit $v$ in timestep $i_n$ is
any sequence of entangling gates $(E^{(i_1)}, E^{(i_2)}, \ldots, E^{(i_n)})$
where the following conditions are met:

\begin{enumerate}
\item
$E^{(i_1)}$ operates on qubit $u$ and $E^{(i_n)}$ operates on qubit $v$.

\item
any two consecutive gates in the sequence $(E^{(i_j)},E^{(i_{j+1})})$
act on a common qubit $w$.
\item
any two consecutive gates in the sequence either occur in
consecutive timesteps ($i_j = i_{j+1} \pm 1$) or are only separated by
single-qubit gates on $w$ in intervening timesteps $i_j < i < i_{j+1}$.
No single-qubit measurements are allowed.
\item
every gate $E^{(i_j)}$ encountered in the sequence satisfies the
following two conditions:

\begin{enumerate}
\item it is entangling if
the path exits it in the forward direction ($i_{j+1} = i_j + 1$)
\item it is disentangling if the path exits it in the backward direction
($i_{j+1} = i_j - 1$).
\end{enumerate}
\end{enumerate}

\end{definition}
%%%%%%%%%%%%%%%%

\begin{definition}{\textbf{Reachability.}}
A qubit $u$ at timestep $i$ is \emph{reachable} from another qubit $v$ in
another (possibly the same) timestep $i'$ if there is some path of entangling gates that
connects them.
\end{definition}

We now define a standard form for circuit in which circuit coherence will be
well-defined. Standard form circuits must have the following properties:

\begin{description}
\item[output qubits $O \subseteq V$:] These qubits are semantically defined as
containing the useful outputs of a quantum circuit. They do not have to be
projectively measured. They may, for example, be the control for a
later coherent measurement when cascaded with another quantum circuit.
\item[input qubits $I \subseteq V$:] These qubits are prepared in a 
classical product state (the computational basis)
and are all reachable in timestep $1$ from the
output qubits in timestep $D$.
\item[ancillae qubits:] these are prepared in the product state of all $\ket{0}$'s.
\end{description}

\begin{definition}{\textbf{Computational subset}}
The computational subset in timestep $i$ (abbreviated $L_i$) is the subset of the qubits
which are reachable from the output qubits $O$.
When we do not specify a timestep, the \emph{computational subset} simply
refers to the union of all the computational subsets in any timestep:

\begin{equation}
L = \bigcup_{i=1}^D L_i = \subseteq V
\end{equation}
\end{definition}

The computational subset are those qubits on which an entangled, coherent
quantum state evolves over time from
the initial preparation of the input qubits $I \subset V$ in timestep $1$
until the output qubits $O \subset V$ are
measured in timestep $D$.
It is measured in qubits, and potentially grows (if entangling gates are applied), shrinks
(if disentangling gates including measurements are applied) or stays the same in size
in every timestep. We now give a procedure for determining the computational subset formally,
starting from timestep $i$ and working forwards to timestep $D$. We note the following
relationships for well-formed circuits.

\begin{equation}
L_1 = I \qquad L_D = O \qquad L_i \subseteq V
\end{equation}

The computational subset can be computed in two passes through
the quantum circuit, one forward to determine reachability and one backward
to determine the computational subset. This in general requires classical simulation of a
quantum circuit, but we can perform this algorithm efficiently on a
layered quantum circuit as defined in Section \ref{subsec:cohere-lqc}.

In each timestep $i$, we partition all $W$ qubits in $V$ into disjoint, but completely
covering, subsets which each contain an entangled state. We denote these other
qubit subsets as $\{\tilde{L}^{(j)}_i\}$, of which one is the same as the current
computational subset $L_i$.

This partitioning, like $L_i$, is updated in
every timestep.
We call these \emph{layer widths}, of which the computational subset is
a subset.

Following the definition of \textsf{2D CCNTCM}, each qubit subset is a
contiguous subgraphs of the main graph $G$. All qubit subsets are
disjoint subgraphs from each other in that they do not share any vertices,
but they may share edges. All entangling/disentangling gates $E^{(i)}_{uv}$
that occur during a timestep $i$ are contained in the set $G_i$.

Qubit subsets may potentially share common qubits and become entangled by an entangling gate
in a past timestep $i' < i$ from the current timestep $i$.
We keep track of all qubit subsets in $M$ (a set of qubit subsets) whose state at any
timestep $i$ is given by $M_i = \{\tilde{L}^{(j)}_i\} \cup \{ L_i \}$.

%%%%%%%%%%%%%%%%%%%%%%%%%%%%%%%%%%%%%%%%%%%%%%%%%%%%%%%%%%%%%%%%%%%%%%%%%%%%%%
\subsection{The Coherence Calculation Algorithm}
\label{subsec:cohere-algo}

The following algorithm takes as input a well-formed quantum circuit
with graph $G = (V,E)$ over the basis of single-qubit and two-qubit gates
$U(2) \cup U(4)$ including single-qubit measurements in the $Z$-basis.
It returns as output the computational subset at each timestep $\{ L_i \}$.

The data type of qubit subset $\tilde{L}^{(j)}_i$ is the tuple
$(\tilde{V}, P)$ where:

\begin{itemize}
\item $\tilde{V} \subseteq V$ are the qubits in the subset
\item $P$ is a pointer to the parent qubit subset in timestep ${i-1}$, initially \textsc{NULL}.
\end{itemize}

When we assign one qubit subset to another, we assume we are only assigning the qubits part of the tuple, $\tilde{V}$.

%%%%%%%%%%%%%%%%% ENUMERATE 1
\begin{enumerate}
\item
Initialize the following:
\begin{itemize}
\item
$L_1 \leftarrow^{\star} \{ I \}$.
\item
$\tilde{L}^{(j)}_1 = v_j \in V \setminus I$ for all non-input qubits $v_j$
\item
$M_1 = \{ L_1 \} \cup \{ L^{(j)}_1 \}$.
\end{itemize}

\item
In timestep $i \in \{2, \ldots, D \}$:

%%%%%%%%%%%%%%%%%%%%%%%%%%%%% ENUMERATE 2
\begin{enumerate}

\item
Compute the classical description of the state on all the qubits $\rho^{V}_i$ from the state $\rho^{V}_{i-1}$
given $M_{i-1}$.
This is assumed to efficient (e.g. for layered quantum circuits). 
\item
Initialize $M_i \leftarrow \{\}$.
\item
Create two sets:
\begin{itemize}
\item $T_e$ which contains every two-qubit
gate $E_{uv}$ that is entangling in the forward direction based on
their state $\rho^{uv}$ in timestep $i$
\item $T_d$ which contains every two-qubit
gate $E_{uv}$ that is disentangling in the forward direction
 based on
their state $\rho^{uv}$ in timestep $i$,
along with all single-qubit measurements $\{ M_u \}$.
\end{itemize}

\item
For every qubit subset $\tilde{L}^{(j)}_{i-1} \in M_{i-1}$:

%%%%%%%%%%%%%%%%%%%%%%%%%%%%%%%%%%%%%%%% ENUMERATE 3
\begin{enumerate}
\item \textbf{Entangling Case.} Check whether $\tilde{L}^{(j)}_{i-1}$ contains a qubit acted upon by a
$E_{uv} \in T_e$, where we assume without loss of generality that
$u \in \tilde{L}^{(j)}_{i-1}$. If it does:

%%%%%%%%%%%%%%%%%%%%%%%%%%%%%%%%%%%%%%%%%%%%%%%%%%%% ENUMERATE 4
\begin{enumerate}
\item Check whether $v$ is in any other qubit subset
(call it $\tilde{L}^{(j')}_{i-1}$).

% Too deeply nested
%%%%%%%%%%%%%%%%%%%%%%%%%%%%%%%%%%%%%%%%%%%%%%%%%%%%%%%%%%%%%%%% ENUMERATE 5
%\begin{enumerate}
\item
If it is, create a new qubit subset
$\tilde{L}^{(j)}_{i}$ equal to the union of the two qubit subsets from
step $i-1$:

\begin{equation*}
\tilde{L}^{(j)}_{i} \leftarrow \tilde{L}^{(j)}_{i-1} \cup \tilde{L}^{(j')}_{i-1}
\end{equation*}

Update the parent pointers of $\tilde{L}^{(j)}_{i}$ accordingly.

%\item
%If $v$ is \emph{not} in any other qubit subset for timestep $i-1$,
%then simply add it to a new qubit subset for timestep $i$. Note that it will
%not have the tag \textsc{Computation Subset}.

%\begin{equation*}
%\tilde{L}^{(j)}_{i} = \{v\}
%\end{equation*}
%\end{enumerate}
%%%%%%%%%%%%%%%%%%%%%%%%%%%%%%%%%%%%%%%%%%%%%%%%%%%%%%%%%%%%%%% ENUMERATE 5

\item
Add the current qubit subset to the current timestep's set of qubit subsets $M_i$.

\begin{equation*}
M_i \leftarrow M_i \cup \{ \tilde{L}^{(j)}_{i} \}
\end{equation*}

\end{enumerate}
%%%%%%%%%%%%%%%%%%%%%%%%%%%%%%%%%%%%%%%%%%%%%%%%%%%% ENUMERATE 4

\item \textbf{Disentangling Case.} Check whether $\tilde{L}^{(j)}_{i-1}$ matches the following two cases
and take the corresponding actions.

%%%%%%%%%%%%%%%%%%%%%%%%%%%%%%%%%%%%%%%%%%%%%%%%%%%% ENUMERATE 4
\begin{enumerate}
\item If $\tilde{L}^{(j)}_{i-1}$ contains two qubits $u$ and $v$
acted upon by some
$E_{uv} \in T_d$, then check whether $E_{uv}$ is
disentangling between any partitioning of $\tilde{L}^{(j)}_{i-1}$ into two
subsets $V_1 \ni u$ and $V_2 \ni v$. (This takes time
polynomial in the size of $\tilde{L}^{(j)}_{i-1}$).

% Too deeply nested
%%%%%%%%%%%%%%%%%%%%%%%%%%%%%%%%%%%%%%%%%%%%%%%%%%%%%%%%%%%%%%%% ENUMERATE 5
%\begin{enumerate}
\item
If it does, add these two subsets to our collection $M_i$.
Set their parent pointers to point to $\tilde{L}^{(j)}_{i-1}$.

\begin{equation*}
M_i \leftarrow M_i \cup \{ V_1, V_2 \}
\end{equation*}

\item
Otherwise, just set the current subset $\tilde{L}^{(j)}_{i} = \tilde{L}^{(j)}_{i-1}$
with the appropriate parent pointer.

\begin{equation*}
M_i \leftarrow M_i \cup \{ \tilde{L}^{(j)}_{i} \}
\end{equation*}

%\end{enumerate}
%%%%%%%%%%%%%%%%%%%%%%%%%%%%%%%%%%%%%%%%%%%%%%%%%%%%%%%%%%%%%%%% ENUMERATE 5

\item
If $\tilde{L}^{(j)}_{i+1}$ contains a qubit $u$ acted upon by some $M_u \in T_e$,
then create two new qubit subsets. One just removes the qubit $u$
from the current qubit subset.
%, inheriting the tag \textsc{Computational Subset}
%if present.
The other is a single-qubit subset consisting
only of $u$.

\begin{eqnarray*}
\tilde{L}^{(j)}_{i} & \leftarrow & \tilde{L}^{(j)}_{i+1} - \{u\} \\
\tilde{L}^{(j')}_{i} & \leftarrow & \{ u \}
\end{eqnarray*}

Set the parent pointer of $\tilde{L}^{(j)}_{i}$ accordingly.
Add these to our collection.

\begin{equation*}
M_i \leftarrow M_i \cup \{ \tilde{L}^{(j)}_{i}, \tilde{L}^{(j')}_{i} \}
\end{equation*}

\end{enumerate}
%%%%%%%%%%%%%%%%%%%%%%%%%%%%%%%%%%%%%%%%%%%%%%%%%%%%% ENUMERATE 4

\end{enumerate}
%%%%%%%%%%%%%%%%%%%%%%%%%%%%%%%%%%%%%%%% ENUMERATE 3

\item For every qubit subset $\tilde{L}^{(j)}_{i+1} \in M_{i-1}$ not
operated upon by any of the previous steps, copy it unmodified into
$M_i$ with the appropriate parent pointer.

\end{enumerate}
%%%%%%%%%%%%%%%%%%%%%%%%%%%%% ENUMERATE 2

\item
Do a backward pass to discover the computational subset $L_i$ in each timstep.

\begin{enumerate}
\item
Verify that the output qubits exactly correspond to one of the qubit subsets
in $M_D$. Call this $L_D$.
\item
Initialize the computational set $M \leftarrow \{ L_D \}$.
\item
Working backwards for timestep $i$ in $(D, D-1, \ldots, 1)$:
\begin{enumerate}
\item
Find the parent(s) of $L_i$. Create a new set that is the union of them called $L_{i-1}$
and add them to $M$.
\begin{equation}
M \leftarrow M \cup \{L_{i-1}\}
\end{equation}
\end{enumerate}

\item Verify that $L_1 = I$ are exactly the input qubits.

\item Output $M = \{L_1, \ldots, L_D\}$. This is the computational (sub)set of qubits.
\end{enumerate}

\end{enumerate}
%%%%%%%%%%%%%%% ENUMERATE 1


\begin{definition}{\textbf{Circuit coherence}.}
Circuit coherence $Q$ is the sum of the computation subset size (in qubits)
over all $D$ timesteps of a quantum circuit's execution. It is measured
in qubit-timesteps, which is the amount of error-correcting effort to
maintain the coherent state of one logical qubit for one timestep of a circuit.

\begin{equation}
Q = \sum_{i=1}^D |L_i|
\end{equation}
\end{definition}


\begin{definition}{\textbf{Instantaneous coherence}.}
Instantaneous coherence $Q_i$ is the size of the computational subset
(in qubits) in timestep $i$. From the algorithm above,
\begin{equation}
Q_i = |L_i|\text{.}
\end{equation}

We also have the relationship that the total circuit coherence is equal
to the sum of all instantaneous coherences over each timestep.
\begin{equation}
Q = \sum_{i=1}^D Q_i
\end{equation}
\end{definition}

The following relationships hold with other circuit resources.

\begin{equation}
D \le S \le Q \le D\cdot W
\end{equation}

The first inequality holds, because in the least parallel case, each of $S$ gates is executed in sequence
and $S=D$. The second inequality holds, because in the least coherent case, each of $S$ gates
either entangles or disentangles another qubit from the computational subset in every timestep, and
there are no identity wires within the circuit. The third inequality holds because in the
most coherent case, all of $W$ qubits are part of the computational (sub)set for each of $D$ timesteps.

As an example, we can bound the circuit coherence of modular multiplication of $2\times n$-bit
CSE numbers, as described in Section \ref{subsec:mma}. The overall width is $W = O(n^3)$ and
depth is $D=O(\log n)$, so the coherence is upper-bounded by $O(n^3\log n)$. We will possible
ways to reduce this using intermediate uncomputation in Section \ref{sec:cohere-tradeoff}.