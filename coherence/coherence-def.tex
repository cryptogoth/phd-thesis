\section{Definition of Circuit Coherence}
\label{sec:cohere-def}

Usually quantum circuits neglect to draw identity gates. When a bare
quantum wire appears, what is meant is that the qubit maintains its
coherent state until the next non-identity gate comes along to transform it.
However, most quantum circuits are drawn at a logical qubit level,
assuming no errors occur and a coherent state is maintained. While
we continue to maintain that abstraction in this thesis by studying
quantum compiling and quantum architecture abstracted away from
quantum error correction, we acknowledge it as an important area for
optimization and future study. Our one concession here will be to study
the effort to maintain a coherent state in between non-identity.

%%%%%%%%%%%%%%%%%%%%%%%%%%%%%%%%%%%%%%%%%%%%%%%%%%%%%%%%%%%%%%%%%%%%%%%%%%%%%%
\subsection{Entangling Gates}

An entangled quantum state is one which cannot be expressed as the
tensor product of two smaller states. This does not depend on what basis
we consider for the smaller states. Using the density operator formalism,
we can say that a quantum state over two subsystems $A$ and $B$ is
entangled if tracing over one of the subsystem \emph{does not} yield the other subsystem
as a reduced density matrix.

\begin{equation}
\rho^{AB} \text{ entangled } \iff \left(\tr_{A}(\rho^{AB}) \ne \rho^{B} \right) \land
\left(\tr_{B}(\rho^{AB}) \ne \rho^{A}
\end{equation}

A general density matrix for a state across two subsystems $A$ and $B$ can be
written as
\begin{equation}
\rho^{AB} = \sum_{i,i',j,j'} p_{ii'jj'} \ket{a_i}\bra{a_{i'}} \otimes \ket{b_j}\bra{b_{j'}}
\end{equation}
where $\ket{a_i},\ket{a_{i'}}$ are any two states on $A$ and
$\ket{b_j},\ket{b_{j'}}$ are any two states on $B$.

We review here that the trace is a linear operator which distributes across
a general density matrix.

\begin{equation}
\tr(\rho^{AB}) = \sum_{i,i',j,j'} p_{ii'jj'} \tr (\ket{a_i}\bra{a_{i'}} \otimes \ket{b_j}\bra{b_{j'}})
\end{equation}

A reduced density matrix for a particular term.
is obtained by tracing out
one subsystem.

\begin{equation}
tr_A(\ket{a_i}\bra{a_{i'} \otimes \ket{b_j}\bra{b_{j'}}) = \tr(\ket{a_i}\bra{a_{i'}}) \ket{b_j}\bra{b_{j'}}
\end{equation}

Consider a two-qubit gate $E_{uv}$ on single qubits $u$ and $v$ which exist
in a larger system.
Without loss of generality, we assume that $u$ is a single-qubit
that is not entangled with some other, possibly multi-qubit,
state on another set of vertices $L$ where $v \in L$.
The total state on $u \cup L$ is called $\rho^{u}\otimes \rho^{L}$.

We call the action of $E_{uv}$ on this combined state a new state $\sigma^{uL}$:

\begin{equation}
\sigma^{uL} = E^{\dagger}_{uv} (\rho^{u}\otimes \rho^{L}) E^{\dagger}_{uv}
\end{equation}

We call the gate $E_{uv}$ \emph{entangling} between $u$ and $v$
(and between $u$ and $L$) given the states $\rho^{u}$ and $\rho^{L}$ if
the new state after applying $E_uv}$ is entangled, which corresponds to
the condition below.

\begin{equation}
\tr_{u}( \sigma^{uL} ) \ne \rho^{L}
\end{equation}

More generally, we can define $E_{uv}$ as entangling for any multi-qubit
states that $u$ and $v$ are a part of before the application of $E_{uv}$.

Now consider a slightly different setting, where $L$ contains both $u$ and $v$
which are entangled in a larger state $\rho^{L}$. We denote by $\sigma^{L}$ the resulting
state after applying $E_{uv}$ to $\rho^{L}$.

\begin{equation}
\sigma^{L} = E^{\dagger}_{uv} (\rho^{L}) E^{\dagger}_{uv}
\end{equation}

We call the $E_{uv}$ \emph{disentangling} between $u$ and $L$ (without loss
of generality) if after its action on the state $\rho^{L}$ it is
separable into the product state $\sigma^{L} = \sigma^{u} \otimes \sigma^{L \\ \{u\}}$,
which corresponds to the following condition:

\begin{equation}
tr_{u}(\sigma^{L}) = \sigma_^{L \\ \{u\}
\end{equation}

More generally, we can stay the $E_{uv}$ is disentangling for any two
larger subsystems that $u$ and $v$ are a part of, if it is disentangling
for for one of the qubit.

Given these definition, a gate $E_{uv}$
that is entangling in the forward direction on input state $\rho$
is disentangling in the backward direction $E^{\dagger}_{uv}$ on output
state $\sigma = E_{uv}\rho E^{\dagger}_{uv}$.

Note that this definition for entangling or disentangling quantum gates
depends on knowing the actual states before these gates are applied.
Therefore, they may not be apparent just by examining a circuit locally,
but may require simulation of the entire circuit. This gives an operational
definition for entangling/disentangling quantum gates, but it does not give
a compact, theoretical description that can be applied generically. This is
currently a drawback of our definition, especially for characterizing the
behavior of new quantum algorithms that are not yet well-studied.

However, an quantum algorithm designer able to specify a circuit in terms of single-qubit and
two-qubit gates often knows when gates are entangling or disentangling. This is
the case for well-known quantum algorithms such as the QFT or factoring,
and in fact, we will rely on this ``insider knowledge'' when performing
calculations later in this section.
As an overestimate, we can also consider all two-qubit gates entangling, and
only single-qubit projective measurement as disentangling.

%%%%%%%%%%%%%%%%%%%%%%%%%%%%%%%%%%%%%%%%%%%%%%%%%%%%%%%%%%%%%%%%%%%%%%%%%%%%%%
\subsection{Computational Subsets}

In this section, we define subsets of qubits in each timestep
called \emph{computational subsets}.

We refer back to our definition of a quantum circuit on
\textsf{CCNTC}, which is represented by a graph $G = (V,E)$ and a
classical controller. We will later extend our definition to cover
the notion of modules for \textsc{CCNTCM}.

The computational subset in timestep $i$ (abbreviated $L_i$) is the subset of the qubits
in a coherent quantum state which persists from
the initial preparation of the input qubits $I \subset V$ in timestep $1$
until the output qubits $O \subset V$ are
projectively measured in timestep $D$.
It is measured in qubits, and potentially grows or shrinks in size
in every timestep, depending on whether entangling/disentangling gates
(as defined in the previous section) or measurements
are performed. We define the computational subset formally by
induction on timesteps, starting backwards from timestep $D$ where we
note the following relationships:

\begin{equation}
L_1 \subseteq I \qquad L_D = O \qquad L_i \subseteq V
\end{equation}

In each timestep $i$, there may be several entangled states on other
subsets of qubits
(call them $\{\tilde{L}^{(j)}_i\}$) in addition to the current
computational subset $L_i$. These are all
disconnected from each other (do not share a common qubit in timesteps $i \le i' \le D$),
since if they did, they would be the same computational subset by definition.
We call these \emph{interesting subsets}, of which the computational subset is
a special case. While the computational subset is allowed to be a single qubit,
all other interesting subsets must consists of two or more qubits.

Due to the definition of \textsf{2D CCNTCM}, each interesting subset is a
contiguous subgraphs of the main graph $G$, and all interesting subsets are
disjoint subgraphs from each other.

Interesting subsets may potentially share common qubits and become entangled by an entangling gate
in a past timestep $i < D$, so we keep track of them as states of interest.
We keep track of this set of sets of qubits $M$ whose state at any
timestep $i$ is given by $M_i = \{\tilde{L}^{(j)}_i\} \cup \{ L_i \}$,
with the understanding that there is only one interesting subset at either the
end or beginning of a circuit.

\begin{equation}
M_D = \{ L_D \} \qquad M_1 = \{ L_1 \}
\end{equation}

\begin{enumerate}
\item
Initialize $L_D = \{ O \}$ and $M_D = \{ L_D \}$.
\item
In timestep $i$, starting from $D-1$ and counting backwards to $1$:
\begin{enumerate}
\item
\item
Initialize $M_i \leftarrow \{\}$.
\item
Create two sets: =$T_e$ which contains every two-qubit
gate $E_{uv}$ that is entangling, and $T_d$ which contains every two-qubit
gate $E_{uv}$ that is disentangling, given the definition
in the previous section and the current of state of the qubits $\ket{u}$
and $\ket{v}$. The gates are
chosen
from $G_i$, the set of gates applied during timestep $i$
from the definition of \textsf{2D CCNTCM}.

\item
For every interesting subset $\tilde{L}^{(j)}_{i+1} \in M_{i+1}$:

\begin{enumerate}
\item Check whether $\tilde{L}^{(j)}_{i+1}$ contains a qubit acted upon by a
$E_{uv} \in T_d$, where we assume without loss of generality that
$u \in \tilde{L}^{(j)}_{i+1}$. If it does:

%%%%%%%%%%%%%%%%%%%%%%%%%%%%%%%%%%%%%%%%%%%%%%%%%%%%%%%%%%%%%%%%%%%%%%%%%%%%%%
\begin{enumerate}
\item Check whether $v$ is in any other interesting subset
(call it $\tilde{L}^{(j')}_{i+1}$).

\begin{enumerate}
\item
If it is, create a new interesting subset
$\tilde{L}^{(j)}_{i}$ equal to the union of the two interesting subsets from
step $i+1$:

\begin{equation}
\tilde{L}^{(j)}_{i} = \tilde{L}^{(j)}_{i+1} \cup \tilde{L}^{(j')}_{i+1}
\end{equation}

\item
If $v$ is \emph{not} in any other interesting subset for timestep $i+1$,
then simply add it to the current interesting subset for timestep $i$.

\begin{equation}
\tilde{L}^{(j)}_{i} = \{v\}
\end{equation}

\item
Add the current interesting subset to the current timestep's set of interesting subsets $M_i$.

\begin{equation}
M_i \leftarrow M_i \cup \{ \tilde{L}^{(j)}_{i} \}
\end{equation}

\end{enumerate}

\item Check whether $\tilde{L}^{(j)}_{i+1}$ contains a qubit acted upon by a
$E_{uv} \in T_e$, where we assume without loss of generality that
$u \in \tilde{L}^{(j)}_{i+1}$. If it does:

\begin{enumerate}
\item Check whether $v$ is in any other interesting subset
(call it $\tilde{L}^{(j')}_{i+1}$). If it is, create a new interesting subset
$\tilde{L}^{(j)}_{i}$ equal to the union of the two interesting subsets from
step $i+1$:

\begin{equation}
\tilde{L}^{(j)}_{i} = \tilde{L}^{(j)}_{i+1} \cup \tilde{L}^{(j')}_{i+1}
\end{equation}

\item If $v$ is not in any other interesting subset for timestep $i+1$,
then simply add it to the current interesting subset for timestep $i$.

\begin{equation}
\tilde{L}^{(j)}_{i} = \{v\}
\end{equation}
\end{enumerate}

\item
\end{enumerate}
\end{enumerate}

Finally 
\begin{description}
\item[$E_{i}$:] the set of qubits which are involved in two-qubit entangling
gates in timestep $i$.
\item[$M_{i}$:] the set of qubits which are projectively measured in timestep $i$,
and go into a product state with respect to the rest of $V$ afterwards.
\end{description}

We now give an operational definiton by working backwards from the final timestep.
In timestep $i < D$,

\begin{equation}
L_i = (L_{i+1} \cup (L_{i+1} \cap E_{i})) \\ M_{i}
\end{equation}

That is, the computation state $L_i$ in timestep $i$ is the
computation state in the preceding timestep

where $E_{i}$ are the qubits that a
the computation state consists of the computation
state in timestep $i+1$ with the follow changes:
\begin{itemize}
\item
If the two-qubit entangling gate $g_{u,v}$ occurs on qubits $u,v \in V$
occurs in timestep $i$, where either $u$ or $v$ are in $L_{$
\item
\end{itemize}
\end{itemize}

The coherent computation history is the union of the computation state size
(in qubits) over
all $D$ timesteps of a quantum circuit's execution.

Circuit coherence is the sum of the computation state size (in qubits)
over all $D$ timesteps of a quantum circuit's execution. It is measured
in qubit-timesteps.

\begin{definition}{\textbf{Circuit coherence}.}
Circuit coherence is the amount of error-correction required on the logical
qubits of a quantum circuit in standard coherent form
in order to achieve a correct result.
\end{definition}

The standard coherent form for a quantum circuit imposes some
additional restrictions to make sure the circuit coherence is
well-defined. First, it delays initialization
of qubits (both ancillae and input qubits) until the timestep before
they are first entangled with the computation state. Second, if
an input qubit $v \in I$ is not in the computation state in timestep 1,
as defined above, it is removed from $I$.

\begin{definition}{\textbf{Computation state}.}
The computation state is the coherent history 
\end{definition}

\begin{definition}{\textbf{Qubit-timestep}.}
A qubit-timestep is a unit of coherent quantum computation through time,
or the unit for measuring quantum time-space tradeoffs. It is equivalent
to one qubit in a coherent state for a unit of depth.
In \textsf{2D CCNTCM} 