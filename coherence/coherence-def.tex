\section{Definition of Circuit Coherence}
\label{sec:cohere-def}

Usually quantum circuits neglect to draw identity gates. When a bare
quantum wire appears, what is meant is that the qubit maintains its
coherent state until the next non-identity gate comes along to transform it.
However, most quantum circuits are drawn at a logical qubit level,
assuming no errors occur and a coherent state is maintained. While
we continue to maintain that abstraction in this thesis by studying
quantum compiling and quantum architecture independently from
quantum error correction, we acknowledge it as an important area for
optimization and future study. Our one concession here will be to study
the effort to maintain a coherent quantum state throughout a circuit
by defining a new circuit resource called \emph{circuit coherence}.

First, we will define what we mean by an entangling (or disentangling) gate
in Section \ref{subsec:cohere-entangle}. Then we will build upon this
to define a layer width, or an subset of interesting qubits in every
timestep, in Section \ref{subsec:cohere-subset}. Finally, we will use
the previous two definitions to define the resource circuit coherence and
describe its relationship to the other circuit resources: depth, size, and
width. 

%%%%%%%%%%%%%%%%%%%%%%%%%%%%%%%%%%%%%%%%%%%%%%%%%%%%%%%%%%%%%%%%%%%%%%%%%%%%%%
\subsection{Entangling Gates}
\label{subsec:cohere-entangle}

An entangled quantum state is one which cannot be expressed as the
tensor product of two smaller states. This does not depend on what basis
we consider for the smaller states. Using the density operator formalism,
we can say that a quantum state over two subsystems $A$ and $B$ is
entangled if tracing over one of the subsystem \emph{does not} yield the other subsystem
as a reduced density matrix.

\begin{equation}
\rho^{AB} \text{ entangled } \iff \left(\tr_{A}(\rho^{AB}) \ne \rho^{B} \right) \land
\left(\tr_{B}(\rho^{AB}) \ne \rho^{A}
\end{equation}

A general density matrix for a state across two subsystems $A$ and $B$ can be
written as
\begin{equation}
\rho^{AB} = \sum_{i,i',j,j'} p_{ii'jj'} \ket{a_i}\bra{a_{i'}} \otimes \ket{b_j}\bra{b_{j'}}
\end{equation}
where $\ket{a_i},\ket{a_{i'}}$ are any two states on $A$ and
$\ket{b_j},\ket{b_{j'}}$ are any two states on $B$.

We review here that the trace is a linear operator which distributes across
a general density matrix.

\begin{equation}
\tr(\rho^{AB}) = \sum_{i,i',j,j'} p_{ii'jj'} \tr (\ket{a_i}\bra{a_{i'}} \otimes \ket{b_j}\bra{b_{j'}})
\end{equation}

A reduced density matrix for a particular term.
is obtained by tracing out
one subsystem.

\begin{equation}
tr_A(\ket{a_i}\bra{a_{i'} \otimes \ket{b_j}\bra{b_{j'}}) = \tr(\ket{a_i}\bra{a_{i'}}) \ket{b_j}\bra{b_{j'}}
\end{equation}

Consider a two-qubit gate $E_{uv}$ on single qubits $u$ and $v$ which exist
in a larger system.
Without loss of generality, we assume that $u$ is a single-qubit
that is not entangled with some other, possibly multi-qubit,
state on another set of vertices $L$ where $v \in L$.
The total state on $u \cup L$ is called $\rho^{u}\otimes \rho^{L}$.

We call the action of $E_{uv}$ on this combined state a new state $\sigma^{uL}$:

\begin{equation}
\sigma^{uL} = E^{\dagger}_{uv} (\rho^{u}\otimes \rho^{L}) E^{\dagger}_{uv}
\end{equation}

We call the gate $E_{uv}$ \emph{entangling} between $u$ and $v$
(and between $u$ and $L$) given the states $\rho^{u}$ and $\rho^{L}$ if
the new state after applying $E_uv}$ is entangled, which corresponds to
the condition below.

\begin{equation}
\tr_{u}( \sigma^{uL} ) \ne \rho^{L}
\end{equation}

More generally, we can define $E_{uv}$ as entangling for any multi-qubit
states that $u$ and $v$ are a part of before the application of $E_{uv}$.

Now consider a slightly different setting, where $L$ contains both $u$ and $v$
which are entangled in a larger state $\rho^{L}$. We denote by $\sigma^{L}$ the resulting
state after applying $E_{uv}$ to $\rho^{L}$.

\begin{equation}
\sigma^{L} = E^{\dagger}_{uv} (\rho^{L}) E^{\dagger}_{uv}
\end{equation}

We call the $E_{uv}$ \emph{disentangling} between $u$ and $L$ (without loss
of generality) if after its action on the state $\rho^{L}$ it is
separable into the product state $\sigma^{L} = \sigma^{u} \otimes \sigma^{L \\ \{u\}}$,
which corresponds to the following condition:

\begin{equation}
tr_{u}(\sigma^{L}) = \sigma_^{L \\ \{u\}
\end{equation}

More generally, we can stay the $E_{uv}$ is disentangling for any two
larger subsystems $V_1 \ni u$ and $V_2 \ni v$, if it is disentangling
for any pairs of qubits $(u',v')$ such that $u' \in V_1, v' \in V_2$. For this
reason, it is
usually more difficult to show that a gate is disentangling in a particular
direction in time.

Given these definition, a gate $E_{uv}$
that is entangling in the forward direction on input state $\rho$
is disentangling in the backward direction $E^{\dagger}_{uv}$ on output
state $\sigma = E_{uv}\rho E^{\dagger}_{uv}$. When we do not specify a
time direction, an entangling gate $E_{uv}$ is entangling in the forward
direction and disentangling in the backward direction.

Note that this definition for entangling or disentangling quantum gates
depends on knowing the actual states before these gates are applied.
Therefore, they may not be apparent just by examining a circuit locally,
but may require simulation of the entire circuit. This gives an operational
definition for entangling/disentangling quantum gates, but it does not give
a compact, theoretical description that can be applied generically. This is
currently a drawback of our definition, especially for characterizing the
behavior of new quantum algorithms that are not yet well-studied.

However, an quantum algorithm designer able to specify a circuit in terms of single-qubit and
two-qubit gates often knows when gates are entangling or disentangling. This is
the case for well-known quantum algorithms such as the QFT or factoring,
and in fact, we will rely on this ``insider knowledge'' when performing
calculations later in this section.
As an overestimate, we can also consider all two-qubit gates entangling, and
only single-qubit projective measurement as disentangling.

%%%%%%%%%%%%%%%%%%%%%%%%%%%%%%%%%%%%%%%%%%%%%%%%%%%%%%%%%%%%%%%%%%%%%%%%%%%%%%
\subsection{Reachability and Computational Subsets}
\label{subsec:cohere-subset}

We refer back to our definition of a quantum circuit on
\textsf{CCNTC}, which is represented by a graph $G = (V,E)$ and a
classical controller. In particular, the set of all qubits is $V$,
and its size is $|V|=W$, the circuit width.
Our notion of circuit coherence will not depend
on the modules from \textsc{CCNTCM}.

%%%%%%%%%%%%%%%%%% DEFINITION
\begin{definition}{\textbf{Entangling Paths}}
We denote by $E^{(i)}$ an entangling two-qubit gate which acts in
timestep $i$.
An \emph{entangling path} of gates from qubit $u$ in timestep $i_1$ to
qubit $v$ in timestep $i_n$ is
any sequence of entangling gates $(E^{(i_1)}, E^{(i_2)}, \ldots, E^{(i_n)})$
where the following conditions are met:

\begin{enumerate}
\item
$E^{(i_1)}$ operates on qubit $u$ and $E^{(i_n)}$ operates on qubit $v$.

\item
any two consecutive gates in the sequence $(E^{(i_j)},E^{(i_{j+1})})$
act on a common qubit $w$.
\item
any two consecutive gates in the sequence either occur in
consecutive timesteps ($i_j = i_{j+1} \pm 1$) or are only separated by
single-qubit gates on $w$ in intervening timesteps $i_j < i < i_{j+1}$.
No single-qubit measurements are allowed.
\item
every gate $E^{(i_j)}$ encountered in the sequence satisfies the
following two conditions:

\begin{enumerate}
\item it is entangling if
the path exits it in the forward direction ($i_{j+1} = i_j + 1$)
\item it is disentangling if the path exits it in the backward direction
($i_{j+1} = i_j - 1$).
\end{enumerate}
\end{enumerate}

\end{definition}
%%%%%%%%%%%%%%%%

\begin{definition}{\textbf{Reachability}}
A qubit $u$ at timestep $i$ is \emph{reachable} from another qubit $v$ in
another timestep $i' > i$ if there is some path of entangling gates that
connects them.
\end{definition}

We now define a standard form for circuit in which circuit coherence will be
well-defined. Standard form circuits must have the following properties:

\begin{description}
\item[output qubits $O \subseteq V$:] These qubits are semantically defined as
containing the useful outputs of a quantum circuit. They do not have to be
projectively measured. They may, for example, be the control for a
later coherent measurement when cascaded with another quantum circuit.
\item[input qubits $I \subseteq V$:] These qubits are prepared in a 
classical product state (the computational basis)
and are all reachable from the
output qubits.
\item[ancillae qubits:] these are prepared in the product state of all $\ket{0}$'s.
\end{description}

\begin{definition}{\textbf{Computational subset}}
The computational subset in timestep $i$ (abbreviated $L_i$) is the subset of the qubits
which are reachable from the output qubits $O$.
When we do not specify a timestep, the \emph{computational subset} simply
refers to the union of all the computational subsets in any timestep:

\begin{equation}
L = \bigcup_{i=1^D L_i = \subseteq V
\end{equation}
\end{definition}

Intuitively, the notion of a computational subset is
in a coherent quantum state which evolves over time from
the initial preparation of the input qubits $I \subset V$ in timestep $1$
until the output qubits $O \subset V$ are
measured in timestep $D$.
It is measured in qubits, and potentially grows or shrinks in size
in every timestep, depending on whether entangling/disentangling gates
(as defined in the previous section) or measurements
are performed. We now give a procedure for determining the computational subset formally,
starting backwards from timestep $D$ where we
note the following relationships:

\begin{equation}
L_1 \subseteq I \qquad L_D = O \qquad L_i \subseteq V
\end{equation}

The computational subset can be computed from one backwards pass through
the quantum circuit, from the output qubits.

In each timestep $i$, we partition all $W$ qubits into disjoint, but completely
covering, subsets which each contain an entangled state. We denote these other
qubit subsets as $\{\tilde{L}^{(j)}_i\}$, of which one is the same as the current
computational subset $L_i$. The title of ``computational subset'' in any timestep
$i+1$ is inherited by any subset in timestep $i$ according to the rules below.
This partitioning, like $L_i$, is updated in
every timestep. No pair of subsets share a common qubit in timesteps $i < i' \le D$,
since if they did, they would be the same subset in timestep $i$ by definition.
We call these \emph{layer widths}, of which the computational subset is
a subset. While the computational subset in any particular timestep is allowed to be a single qubit,
all other interesting subsets must consists of two or more qubits.

Following the definition of \textsf{2D CCNTCM}, each qubit subset is a
contiguous subgraphs of the main graph $G$. All qubit subsets are
disjoint subgraphs from each other in that they do not share any vertices,
but they may share edges. All entangling/disentangling gates $E^{(i)}_{uv}$
that occur during a timestep $i$ are contained in the set $G_i$.

Qubit subsets may potentially share common qubits and become entangled by an entangling gate
in a past timestep $i' < i$ from the current timestep $i$.
We keep track of all qubit subsets in $M$ (a set of qubit subsets) whose state at any
timestep $i$ is given by $M_i = \{\tilde{L}^{(j)}_i\} \cup \{ L_i \}$.

%%%%%%%%%%%%%%%%% ENUMERATE 1
\begin{enumerate}
\item
Initialize the following:
\begin{itemize}
\item
$L_D = \{ O \}$
\item
$\tilde{L}^{(j)}_D = v_j \in V \\ O$ for all non-output qubits $v_j$
\item
$M_D = \{ L_D \} \cup \{ L^{(j)}_D \}$.
\end{itemize}

\item
In timestep $i$, starting from $D-1$ and counting backwards to $1$:

%%%%%%%%%%%%%%%%%%%%%%%%%%%%% ENUMERATE 2
\begin{enumerate}

\item
Initialize $M_i \leftarrow \{\}$.
\item
Create two sets:
\begin{itemize}
\item $T_e$ which contains every two-qubit
gate $E_{uv}$ that is entangling in the forward direction based on
its state $\ket{u}$ and $\ket{v}$ in timestep $i$
\item $T_d$ which contains every two-qubit
gate $E_{uv}$ that is disentangling in the forward direction
 based on
its state $\ket{u}$ and $\ket{v}$ in timestep $i$,
along with all single-qubit measurements $\{ M_u \}$.
\end{itemize}

\item
For every qubit subset $\tilde{L}^{(j)}_{i+1} \in M_{i+1}$:

%%%%%%%%%%%%%%%%%%%%%%%%%%%%%%%%%%%%%%%% ENUMERATE 3
\begin{enumerate}
\item Check whether $\tilde{L}^{(j)}_{i+1}$ contains a qubit acted upon by a
$E_{uv} \in T_d$, where we assume without loss of generality that
$u \in \tilde{L}^{(j)}_{i+1}$. If it does:

%%%%%%%%%%%%%%%%%%%%%%%%%%%%%%%%%%%%%%%%%%%%%%%%%%%% ENUMERATE 4
\begin{enumerate}
\item Check whether $v$ is in any other qubit subset
(call it $\tilde{L}^{(j')}_{i+1}$).

%%%%%%%%%%%%%%%%%%%%%%%%%%%%%%%%%%%%%%%%%%%%%%%%%%%%%%%%%%%%%%%% ENUMERATE 5
\begin{enumerate}
\item
If it is, create a new qubit subset
$\tilde{L}^{(j)}_{i}$ equal to the union of the two qubit subsets from
step $i+1$:

\begin{equation*}
\tilde{L}^{(j)}_{i} = \tilde{L}^{(j)}_{i+1} \cup \tilde{L}^{(j')}_{i+1}
\end{equation*}

\item
If $v$ is \emph{not} in any other interesting subset for timestep $i+1$,
then simply add it to a new qubit subset for timestep $i$.

\begin{equation*}
\tilde{L}^{(j)}_{i} = \{v\}
\end{equation*}
\end{enumerate}
%%%%%%%%%%%%%%%%%%%%%%%%%%%%%%%%%%%%%%%%%%%%%%%%%%%%%%%%%%%%%%% ENUMERATE 5

\item
Add the current interesting subset to the current timestep's set of interesting subsets $M_i$.

\begin{equation*}
M_i \leftarrow M_i \cup \{ \tilde{L}^{(j)}_{i} \}
\end{equation*}

\end{enumerate}
%%%%%%%%%%%%%%%%%%%%%%%%%%%%%%%%%%%%%%%%%%%%%%%%%%%% ENUMERATE 4

\item Check whether $\tilde{L}^{(j)}_{i+1}$ matches the following two cases
and take the corresponding actions.

%%%%%%%%%%%%%%%%%%%%%%%%%%%%%%%%%%%%%%%%%%%%%%%%%%%% ENUMERATE 4
\begin{enumerate}
\item If $\tilde{L}^{(j)}_{i+1}$ contains two qubits $u$ and $v$
acted upon by some
$E_{uv} \in T_e$, then check whether $E_{uv}$ is
disentangling between any partitioning of $\tilde{L}^{(j)}_{i+1}$ into two
subsets $V_1 \ni u$ and $V_2 \ni v$ is in the other. (This takes time
polynomial in the size of $\tilde{L}^{(j)}_{i+1}$).

%%%%%%%%%%%%%%%%%%%%%%%%%%%%%%%%%%%%%%%%%%%%%%%%%%%%%%%%%%%%%%%% ENUMERATE 5
\begin{enumerate}
\item
If it does, add these two subsets to our collection $M_i$.

\begin{equation*}
M_i \leftarrow M_i \cup \{ V_1, V_2 \}
\end{equation*}

\item
Otherwise, just set the current subset $\tilde{L}^{(j)}_{i+1} = \tilde{L}^{(j_}_{i}$

\begin{equation*}
M_i \leftarrow M_i \cup \{ \tilde{L}^{(j)}_{i} \}
\end{equation*}
\end{enumerate}
%%%%%%%%%%%%%%%%%%%%%%%%%%%%%%%%%%%%%%%%%%%%%%%%%%%%%%%%%%%%%%%% ENUMERATE 5

\item
If $\tilde{L}^{(j)}_{i+1}$ contains a qubit $u$ acted upon by some $M_u \in T_e$,
then create a new qubit subset $\tilde{L}^{(j)}_{i} = \tilde{L}^{(j)}_{i+1} - \{u\}$
which just removes the qubit $u$. Add this to our collection.

\begin{equation*}
M_i \leftarrow M_i \cup \{ \tilde{L}^{(j)}_{i} \}
\end{equation*}

\end{enumerate}
%%%%%%%%%%%%%%%%%%%%%%%%%%%%%%%%%%%%%%%%%%%%%%%%%%%%% ENUMERATE 4

\end{enumerate}
%%%%%%%%%%%%%%%%%%%%%%%%%%%%%%%%%%%%%%%% ENUMERATE 3

\item For every qubit subset $\tilde{L}^{(j)}_{i+1} \in M_{i+1}$ not
operated upon by any of the previous steps, copy it unmodified into
$M_i$.

\end{enumerate}
%%%%%%%%%%%%%%%%%%%%%%%%%%%%% ENUMERATE 2

\item


\end{enumerate}
%%%%%%%%%%%%%%% ENUMERATE 1


Circuit coherence is the sum of the computation state size (in qubits)
over all $D$ timesteps of a quantum circuit's execution. It is measured
in qubit-timesteps.

\begin{definition}{\textbf{Circuit coherence}.}
Circuit coherence is the amount of error-correction required on the logical
qubits of a quantum circuit in standard coherent form
in order to achieve a correct result.
\end{definition}

The standard coherent form for a quantum circuit imposes some
additional restrictions to make sure the circuit coherence is
well-defined. First, it delays initialization
of qubits (both ancillae and input qubits) until the timestep before
they are first entangled with the computation state. Second, if
an input qubit $v \in I$ is not in the computation state in timestep 1,
as defined above, it is removed from $I$.

\begin{definition}{\textbf{Computation state}.}
The computation state is the coherent history 
\end{definition}

\begin{definition}{\textbf{Qubit-timestep}.}
A qubit-timestep is a unit of coherent quantum computation through time,
or the unit for measuring quantum time-space tradeoffs. It is equivalent
to one qubit in a coherent state for a unit of depth.
In \textsf{2D CCNTCM} 