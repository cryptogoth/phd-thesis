\chapter{Quantum Circuit Coherence}
\label{chap:coherence}

In the first two chapters, we presented low-depth nearest-neighbor
architectures for factoring an $n$-bit number. We improved the depth first to
be sublinear and then sublogarithmic, but at a polynomial increase in size
and width. This represents a time-space tradeoff which can be upper-bounded
by the product of the circuit depth and circuit width. Although the title of
this dissertation indicates that it is beneficial to decrease depth, in
experimental implementations, we may be constrained by other real-world
resources.

Toward this end, we introduce a new circuit resource called \emph{coherence}
which quantifies the amount of error-correction that must be performed
to maintain a coherent quantum computing state. This can be analogous to
the amount of classical controller time or electrical power that a
quantum computing experiment consumes while running an algorithm. We
define our new resource in Section \ref{sec:cohere-def}, discuss its
relationship to other circuit resources on \textsf{2D CCNTCM}, and
calculate coherence for some simple examples to illustrate how it
captures the notion of parallelizability for a circuit.

Decreasing circuit coherence can increase circuit depth or size, which introduces
a new tradeoff for quantifying circuit parallelism.
Therefore, we compare it to other (time-space) tradeoffs that are common in
the literature, including measurement-based quantum computer (MBQC) in
Section \ref{sec:cohere-mbqc} and the reversible pebble game in
Section \ref{sec:cohere-tradeoff}. In both cases, we relate these
other tradeoffs to circuit coherence.

Finally, in Section \ref{sec:cohere-factor}, we calculate the circuit coherence
for an arithmetic building block useful in factoring: modular multiple addition.
We present fascinating conjectures about ways to further decrease
either depth or coherence for factoring architectures. Finally, we conclude
by presenting configurable-depth factoring circuits
that a future experimental architect can use to make the appropriate choice.

%Finally, having exhausted our fascination with factoring,
%we apply our low-depth techniques (compiling and circuit coherence)
%to a new quantum algorithm: hamiltonian
%simulation. In Section \ref{cohere-hs-bg}, we discuss the background of
%this problem, and in Section \ref{cohere-hs-calc} we parallelize one
%aspect of it, that of decreasing the depth of simulating
%a $1$-sparse Hamiltonian matrix.

\section{Definition of Circuit Coherence}
\label{sec:cohere-def}

Usually quantum circuits neglect to draw identity gates. When a bare
quantum wire appears, what is meant is that the qubit maintains its
coherent state until the next non-identity gate comes along to transform it.
However, most quantum circuits are drawn at a logical qubit level,
assuming no errors occur and a coherent state is maintained. While
we continue to maintain that abstraction in this thesis by studying
quantum compiling and quantum architecture independently from
quantum error correction, we acknowledge it as an important area for
optimization and future study. Our one concession here will be to study
the effort to maintain a coherent quantum state throughout a circuit
by defining a new circuit resource called \emph{circuit coherence}.

First, we will define what we mean by an entangling (or disentangling) gate
in Section \ref{subsec:cohere-entangle}. Then we will build upon this
to define a layer width, or an subset of interesting qubits in every
timestep, in Section \ref{subsec:cohere-subset}. Finally, we will use
the previous two definitions to define the resource circuit coherence and
describe its relationship to the other circuit resources: depth, size, and
width. 

%%%%%%%%%%%%%%%%%%%%%%%%%%%%%%%%%%%%%%%%%%%%%%%%%%%%%%%%%%%%%%%%%%%%%%%%%%%%%%
\subsection{Entangling Gates}
\label{subsec:cohere-entangle}

An entangled quantum state is one which cannot be expressed as the
tensor product of two smaller states. This does not depend on what basis
we consider for the smaller states. Using the density operator formalism,
we can say that a quantum state over two subsystems $A$ and $B$ is
entangled if tracing over one of the subsystem \emph{does not} yield the other subsystem
as a reduced density matrix.

\begin{equation}
\rho^{AB} \text{ entangled }
%\iff \left(\tr_{A}(\rho^{AB}) \ne \rho^{B} \right) \land
%\left(\tr_{B}(\rho^{AB}) \ne \rho^{A}
\end{equation}

A general density matrix for a state across two subsystems $A$ and $B$ can be
written as
\begin{equation}
\rho^{AB} = \sum_{i,i',j,j'} p_{ii'jj'} \ket{a_i}\bra{a_{i'}} \otimes \ket{b_j}\bra{b_{j'}}
\end{equation}
where $\ket{a_i},\ket{a_{i'}}$ are any two states on $A$ and
$\ket{b_j},\ket{b_{j'}}$ are any two states on $B$.

We review here that the trace is a linear operator which distributes across
a general density matrix.

\begin{equation}
\tr(\rho^{AB}) = \sum_{i,i',j,j'} p_{ii'jj'} \tr (\ket{a_i}\bra{a_{i'}} \otimes \ket{b_j}\bra{b_{j'}})
\end{equation}

A reduced density matrix for a particular term.
is obtained by tracing out
one subsystem.

\begin{equation}
tr_A(\ket{a_i}\bra{a_{i'}} \otimes \ket{b_j}\bra{b_{j'}}) = \tr(\ket{a_i}\bra{a_{i'}}) \ket{b_j}\bra{b_{j'}}
\end{equation}

Consider a two-qubit gate $E_{uv}$ on single qubits $u$ and $v$ which exist
in a larger system.
Without loss of generality, we assume that $u$ is a single-qubit
that is not entangled with some other, possibly multi-qubit,
state on another set of vertices $L$ where $v \in L$.
The total state on $u \cup L$ is called $\rho^{u}\otimes \rho^{L}$.

We call the action of $E_{uv}$ on this combined state a new state $\sigma^{uL}$:

\begin{equation}
\sigma^{uL} = E^{\dagger}_{uv} (\rho^{u}\otimes \rho^{L}) E^{\dagger}_{uv}
\end{equation}

We call the gate $E_{uv}$ \emph{entangling} between $u$ and $v$
(and between $u$ and $L$) given the states $\rho^{u}$ and $\rho^{L}$ if
the new state after applying $E_{uv}$ is entangled, which corresponds to
the condition below.

\begin{equation}
\tr_{u}( \sigma^{uL} ) \ne \rho^{L}
\end{equation}

More generally, we can define $E_{uv}$ as entangling for any multi-qubit
states that $u$ and $v$ are a part of before the application of $E_{uv}$.

Now consider a slightly different setting, where $L$ contains both $u$ and $v$
which are entangled in a larger state $\rho^{L}$. We denote by $\sigma^{L}$ the resulting
state after applying $E_{uv}$ to $\rho^{L}$.

\begin{equation}
\sigma^{L} = E^{\dagger}_{uv} (\rho^{L}) E^{\dagger}_{uv}
\end{equation}

We call the $E_{uv}$ \emph{disentangling} between $u$ and $L$ (without loss
of generality) if after its action on the state $\rho^{L}$ it is
separable into the product state $\sigma^{L} = \sigma^{u} \otimes \sigma^{L \\ \{u\}}$,
which corresponds to the following condition:

\begin{equation}
\tr_{u}(\sigma^{L}) = \sigma^{L} \\ \{u\}
\end{equation}

More generally, we can stay the $E_{uv}$ is disentangling for any two
larger subsystems $V_1 \ni u$ and $V_2 \ni v$, if it is disentangling
for any pairs of qubits $(u',v')$ such that $u' \in V_1, v' \in V_2$. For this
reason, it is
usually more difficult to show that a gate is disentangling in a particular
direction in time.

Given these definition, a gate $E_{uv}$
that is entangling in the forward direction on input state $\rho$
is disentangling in the backward direction $E^{\dagger}_{uv}$ on output
state $\sigma = E_{uv}\rho E^{\dagger}_{uv}$. When we do not specify a
time direction, an entangling gate $E_{uv}$ is entangling in the forward
direction and disentangling in the backward direction.

Note that this definition for entangling or disentangling quantum gates
depends on knowing the actual states before these gates are applied.
Therefore, they may not be apparent just by examining a circuit locally,
but may require simulation of the entire circuit. This gives an operational
definition for entangling/disentangling quantum gates, but it does not give
a compact, theoretical description that can be applied generically. This is
currently a drawback of our definition, especially for characterizing the
behavior of new quantum algorithms that are not yet well-studied.

However, an quantum algorithm designer able to specify a circuit in terms of single-qubit and
two-qubit gates often knows when gates are entangling or disentangling. This is
the case for well-known quantum algorithms such as the QFT or factoring,
and in fact, we will rely on this ``insider knowledge'' when performing
calculations later in this section.
As an overestimate, we can also consider all two-qubit gates entangling, and
only single-qubit projective measurement as disentangling.

%%%%%%%%%%%%%%%%%%%%%%%%%%%%%%%%%%%%%%%%%%%%%%%%%%%%%%%%%%%%%%%%%%%%%%%%%%%%%%
\subsection{Reachability and Computational Subsets}
\label{subsec:cohere-subset}

We refer back to our definition of a quantum circuit on
\textsf{CCNTC}, which is represented by a graph $G = (V,E)$ and a
classical controller. In particular, the set of all qubits is $V$,
and its size is $|V|=W$, the circuit width.
Our notion of circuit coherence will not depend
on the modules from \textsc{CCNTCM}.

%%%%%%%%%%%%%%%%%% DEFINITION
\begin{definition}{\textbf{Entangling Paths}}
We denote by $E^{(i)}$ an entangling two-qubit gate which acts in
timestep $i$.
An \emph{entangling path} of gates from qubit $u$ in timestep $i_1$ to
qubit $v$ in timestep $i_n$ is
any sequence of entangling gates $(E^{(i_1)}, E^{(i_2)}, \ldots, E^{(i_n)})$
where the following conditions are met:

\begin{enumerate}
\item
$E^{(i_1)}$ operates on qubit $u$ and $E^{(i_n)}$ operates on qubit $v$.

\item
any two consecutive gates in the sequence $(E^{(i_j)},E^{(i_{j+1})})$
act on a common qubit $w$.
\item
any two consecutive gates in the sequence either occur in
consecutive timesteps ($i_j = i_{j+1} \pm 1$) or are only separated by
single-qubit gates on $w$ in intervening timesteps $i_j < i < i_{j+1}$.
No single-qubit measurements are allowed.
\item
every gate $E^{(i_j)}$ encountered in the sequence satisfies the
following two conditions:

\begin{enumerate}
\item it is entangling if
the path exits it in the forward direction ($i_{j+1} = i_j + 1$)
\item it is disentangling if the path exits it in the backward direction
($i_{j+1} = i_j - 1$).
\end{enumerate}
\end{enumerate}

\end{definition}
%%%%%%%%%%%%%%%%

\begin{definition}{\textbf{Reachability}}
A qubit $u$ at timestep $i$ is \emph{reachable} from another qubit $v$ in
another timestep $i' > i$ if there is some path of entangling gates that
connects them.
\end{definition}

We now define a standard form for circuit in which circuit coherence will be
well-defined. Standard form circuits must have the following properties:

\begin{description}
\item[output qubits $O \subseteq V$:] These qubits are semantically defined as
containing the useful outputs of a quantum circuit. They do not have to be
projectively measured. They may, for example, be the control for a
later coherent measurement when cascaded with another quantum circuit.
\item[input qubits $I \subseteq V$:] These qubits are prepared in a 
classical product state (the computational basis)
and are all reachable from the
output qubits.
\item[ancillae qubits:] these are prepared in the product state of all $\ket{0}$'s.
\end{description}

\begin{definition}{\textbf{Computational subset}}
The computational subset in timestep $i$ (abbreviated $L_i$) is the subset of the qubits
which are reachable from the output qubits $O$.
When we do not specify a timestep, the \emph{computational subset} simply
refers to the union of all the computational subsets in any timestep:

\begin{equation}
L = \bigcup_{i=1}^D L_i = \subseteq V
\end{equation}
\end{definition}

Intuitively, the notion of a computational subset is
in a coherent quantum state which evolves over time from
the initial preparation of the input qubits $I \subset V$ in timestep $1$
until the output qubits $O \subset V$ are
measured in timestep $D$.
It is measured in qubits, and potentially grows or shrinks in size
in every timestep, depending on whether entangling/disentangling gates
(as defined in the previous section) or measurements
are performed. We now give a procedure for determining the computational subset formally,
starting backwards from timestep $D$ where we
note the following relationships:

\begin{equation}
L_1 \subseteq I \qquad L_D = O \qquad L_i \subseteq V
\end{equation}

The computational subset can be computed from one backwards pass through
the quantum circuit, from the output qubits.

In each timestep $i$, we partition all $W$ qubits into disjoint, but completely
covering, subsets which each contain an entangled state. We denote these other
qubit subsets as $\{\tilde{L}^{(j)}_i\}$, of which one is the same as the current
computational subset $L_i$. The title of ``computational subset'' in any timestep
$i+1$ is inherited by any subset in timestep $i$ according to the rules below.
This partitioning, like $L_i$, is updated in
every timestep. No pair of subsets share a common qubit in timesteps $i < i' \le D$,
since if they did, they would be the same subset in timestep $i$ by definition.
We call these \emph{layer widths}, of which the computational subset is
a subset. While the computational subset in any particular timestep is allowed to be a single qubit,
all other interesting subsets must consists of two or more qubits.

Following the definition of \textsf{2D CCNTCM}, each qubit subset is a
contiguous subgraphs of the main graph $G$. All qubit subsets are
disjoint subgraphs from each other in that they do not share any vertices,
but they may share edges. All entangling/disentangling gates $E^{(i)}_{uv}$
that occur during a timestep $i$ are contained in the set $G_i$.

Qubit subsets may potentially share common qubits and become entangled by an entangling gate
in a past timestep $i' < i$ from the current timestep $i$.
We keep track of all qubit subsets in $M$ (a set of qubit subsets) whose state at any
timestep $i$ is given by $M_i = \{\tilde{L}^{(j)}_i\} \cup \{ L_i \}$.

%%%%%%%%%%%%%%%%% ENUMERATE 1
\begin{enumerate}
\item
Initialize the following:
\begin{itemize}
\item
$L_1 = \{ O \}$
\item
$\tilde{L}^{(j)}_1 = v_j \in V \\ I$ for all non-input qubits $v_j$
\item
$M_1 = \{ L_1 \} \cup \{ L^{(j)}_1 \}$.
\end{itemize}

\item
In timestep $i \in \{2, \ldots, D \}$:

%%%%%%%%%%%%%%%%%%%%%%%%%%%%% ENUMERATE 2
\begin{enumerate}

\item
Initialize $M_i \leftarrow \{\}$.
\item
Create two sets:
\begin{itemize}
\item $T_e$ which contains every two-qubit
gate $E_{uv}$ that is entangling in the forward direction based on
its state $\ket{u}$ and $\ket{v}$ in timestep $i$
\item $T_d$ which contains every two-qubit
gate $E_{uv}$ that is disentangling in the forward direction
 based on
its state $\ket{u}$ and $\ket{v}$ in timestep $i$,
along with all single-qubit measurements $\{ M_u \}$.
\end{itemize}

\item
For every qubit subset $\tilde{L}^{(j)}_{i-1} \in M_{i-1}$:

%%%%%%%%%%%%%%%%%%%%%%%%%%%%%%%%%%%%%%%% ENUMERATE 3
\begin{enumerate}
\item Check whether $\tilde{L}^{(j)}_{i-1}$ contains a qubit acted upon by a
$E_{uv} \in T_e$, where we assume without loss of generality that
$u \in \tilde{L}^{(j)}_{i-1}$. If it does:

%%%%%%%%%%%%%%%%%%%%%%%%%%%%%%%%%%%%%%%%%%%%%%%%%%%% ENUMERATE 4
\begin{enumerate}
\item Check whether $v$ is in any other qubit subset
(call it $\tilde{L}^{(j')}_{i-1}$).

% Too deeply nested
%%%%%%%%%%%%%%%%%%%%%%%%%%%%%%%%%%%%%%%%%%%%%%%%%%%%%%%%%%%%%%%% ENUMERATE 5
%\begin{enumerate}
\item
If it is, create a new qubit subset
$\tilde{L}^{(j)}_{i}$ equal to the union of the two qubit subsets from
step $i+1$, which inherits the tag \textsc{Computational Subset} from
either of its source subsets:

\begin{equation*}
\tilde{L}^{(j)}_{i} \leftarrow \tilde{L}^{(j)}_{i-1} \cup \tilde{L}^{(j')}_{i-1}
\end{equation*}

\item
If $v$ is \emph{not} in any other interesting subset for timestep $i-1$,
then simply add it to a new qubit subset for timestep $i$. Note that it will
not have the tag \textsc{Computation Subset}.

\begin{equation*}
\tilde{L}^{(j)}_{i} = \{v\}
\end{equation*}
%\end{enumerate}
%%%%%%%%%%%%%%%%%%%%%%%%%%%%%%%%%%%%%%%%%%%%%%%%%%%%%%%%%%%%%%% ENUMERATE 5

\item
Add the current qubit subset to the current timestep's set of qubit subsets $M_i$.

\begin{equation*}
M_i \leftarrow M_i \cup \{ \tilde{L}^{(j)}_{i} \}
\end{equation*}

\end{enumerate}
%%%%%%%%%%%%%%%%%%%%%%%%%%%%%%%%%%%%%%%%%%%%%%%%%%%% ENUMERATE 4

\item Check whether $\tilde{L}^{(j)}_{i-1}$ matches the following two cases
and take the corresponding actions.

%%%%%%%%%%%%%%%%%%%%%%%%%%%%%%%%%%%%%%%%%%%%%%%%%%%% ENUMERATE 4
\begin{enumerate}
\item If $\tilde{L}^{(j)}_{i-1}$ contains two qubits $u$ and $v$
acted upon by some
$E_{uv} \in T_d$, then check whether $E_{uv}$ is
disentangling between any partitioning of $\tilde{L}^{(j)}_{i-1}$ into two
subsets $V_1 \ni u$ and $V_2 \ni v$ is in the other. (This takes time
polynomial in the size of $\tilde{L}^{(j)}_{i-1}$).

% Too deeply nested
%%%%%%%%%%%%%%%%%%%%%%%%%%%%%%%%%%%%%%%%%%%%%%%%%%%%%%%%%%%%%%%% ENUMERATE 5
%\begin{enumerate}
\item
If it does, add these two subsets to our collection $M_i$.

\begin{equation*}
M_i \leftarrow M_i \cup \{ V_1, V_2 \}
\end{equation*}

\item
Otherwise, just set the current subset $\tilde{L}^{(j)}_{i} = \tilde{L}^{(j)}_{i-1}$

\begin{equation*}
M_i \leftarrow M_i \cup \{ \tilde{L}^{(j)}_{i} \}
\end{equation*}
%\end{enumerate}
%%%%%%%%%%%%%%%%%%%%%%%%%%%%%%%%%%%%%%%%%%%%%%%%%%%%%%%%%%%%%%%% ENUMERATE 5

\item
If $\tilde{L}^{(j)}_{i+1}$ contains a qubit $u$ acted upon by some $M_u \in T_e$,
then create two new qubit subsets. One just removes the qubit $u$
from the current qubit subset, inheriting the tag \textsc{Computational Subset}
if present. The other is a single-qubit subset consisting
only of $u$.

\begin{eqnarray*}
\tilde{L}^{(j)}_{i} & \leftarrow & \tilde{L}^{(j)}_{i+1} - \{u\} \\
\tilde{L}^{(j')}_{i} & \leftarrow & \{ u \}
\end{eqnarray*}

which just removes the qubit $u$. Add this to our collection.

\begin{equation*}
M_i \leftarrow M_i \cup \{ \tilde{L}^{(j)}_{i}, \tilde{L}^{(j')}_{i} \}
\end{equation*}

\end{enumerate}
%%%%%%%%%%%%%%%%%%%%%%%%%%%%%%%%%%%%%%%%%%%%%%%%%%%%% ENUMERATE 4

\end{enumerate}
%%%%%%%%%%%%%%%%%%%%%%%%%%%%%%%%%%%%%%%% ENUMERATE 3

\item For every qubit subset $\tilde{L}^{(j)}_{i+1} \in M_{i-1}$ not
operated upon by any of the previous steps, copy it unmodified into
$M_i$, inheriting the tag \textsc{Computational Subset}.

\end{enumerate}
%%%%%%%%%%%%%%%%%%%%%%%%%%%%% ENUMERATE 2

\item
Verify that the output qubits are in the computational subset $O \subseteq L_D$,
where $L_D$ is the subset $\tilde{L}^{(j)}_D$ in the last timestep which has
inherited the tag \textsc{Computational Subset}.

\end{enumerate}
%%%%%%%%%%%%%%% ENUMERATE 1


\begin{definition}{\textbf{Circuit coherence}.}
Circuit coherence $Q$ is the sum of the computation subset size (in qubits)
over all $D$ timesteps of a quantum circuit's execution. It is measured
in qubit-timesteps, which is the amount of error-correcting effort to
maintain the coherent state of one logical qubit for one timestep of a circuit.

\begin{equation}
Q = \sum_{i=1}^D |L_i|
\end{equation}
\end{definition}


\begin{definition}{\textbf{Instantaneous coherence}.}
Instantaneous coherence $Q_i$ is the size of the computational subset
(in qubits) in timestep $i$. From the algorithm above,
\begin{equation}
Q_i = |L_i|\text{.}
\end{equation}

We also have the relationship that the total circuit coherence is equal
to the sum of all instantaneous coherences over each timestep.
\begin{equation}
Q = \sum_{i=1}^D Q_i
\end{equation}
\end{definition}

\section{Measurement-Based Quantum Computing Background}
\label{sec:cohere-mbqc}

We will now discuss a model of quantum computing that is very different
from the circuit model, but has already provided us with tools for
parallelism in our nearest-neighbor architectures 
in Chapters \ref{chap:factor-polylog} and \ref{chap:factor-sublog}.
Measurement-based quantum computing (MBQC) is a general model which creates
a large entangled-state on a graph of qubits and performs a pattern of
measurements \cite{Raussendorf2001}.
Later measurements may depend on previous outcomes, and so
a classical controller is required to make each measurement adaptive in this
way. Each measurement reduces the size of our entangled state, and the
measurement operation proceeds
physically across our lattice of qubits, from inputs to outputs.
We will discuss here a restricted form of MBQC called one-way computing
that was proved universal by Raussendorf-Brown-Briegel \cite{Raussendorf2003}
by translation from an arbitrary $n$-qubit circuit.
This model uses only single-qubit measurements on a
regular 2D lattice. We will exclusively consider this model and refer to it
at MBQC for the rest of this section, since it is sufficient for discussing
depth optimization of quantum circuits.

MBQC is very different from the circuit model, which describes unitary
evolution by the application of quantum gates to stationary qubits.
Quantum circuits start out with a product state and then slowly build up
more and more entanglement until it is finally projectively measured at the
end. Surprisingly, both the circuit model and MBQC are equivalent, but
have a tight depth separation for quantum algorithms.

In Section \ref{subsec:mbqc-bg}, we will review the MBQC model.
In Section \ref{subsec:mbqc-par}, we will discuss the work of
Broadbent-Kashefi in automated circuit parallelization, which is a
compilation-like process for reducing circuit depth at the expense of
size and width, another quantum time-space tradeoff. These represent
upper bounds on time-space tradeoffs for mapping certain circuit classes
to a nearest-neighbor circuit with classical controller. We will compare this
to the time-space tradeoff of
other re-ordering networks for mapping circuits to a nearest-neighbor
architecture.time-space tradeoffs for low-depth factoring.
We conclude by comparing and contrasting the
circuit coherence of an MBQC pattern with its other circuit resources.

%%%%%%%%%%%%%%%%%%%%%%%%%%%%%%%%%%%%%%%%%%%%%%%%%%%%%%%%%%%%%%%%%%%%%%%%%%%%%%
\subsection{MBQC Background}
\label{subsec:mbqc-bg}

We follow the exposition of Ref.'s \cite{Broadbent2007,DaSilva2013}
In the MBQC model, quantum computation is represented by three components:
a pattern, an entanglement graph of qubits and interactions, and a
classical controller.

A \emph{pattern} is a sequence of commands which come in five types:

\begin{description}
\item[$N_i$:]
preparation of a qubit $i$ into the state $\ket{+}$.

\item[$E_{ij}$:]
entanglement of qubits $i$ and $j$ with the two-qubit gate
$\Lambda(Z)$ defined below. Note that this gate is bidirectional, so it
does not matter which qubit is the control or the target.

\item[$M^{\alpha}_i$:] single qubit measurement on qubit $i$ which
projects onto the states
$\{ \ket{\pm_{\alpha}} \}$ where

\begin{equation}
\ket{\pm_{\alpha}} = \normtwo(\ket{0} \pm e^{i\alpha}\ket{1})\}
\end{equation}

Associated with every measurement is a signal $s_i \in \mathbb{Z}_2$
which is $0$ for outcome $\ket{+_{\alpha}}$ and $1$ for outcome
$\ket{-_{\alpha}}$.

\item[$X^{s}_i$:] a dependent Pauli $X$ correction, which applies $X$ to
qubit $i$ if the signal $s$ is $1$.

\item[$Z^{t}_j$:] a dependent Pauli $Z$ correction, which applies $Z$ to
qubit $i$ if the signal $t$ is $1$.

\end{description}

A pattern is a valid executable sequence which corresponds to well-defined
quantum and classical operations if no command depends on outcomes that
are not yet measured. Patterns are executed right to left, much like
composing the matrices that make up a sequence of gates to be applied to
a quantum state (column vector). Certain operations can be parallelized
if they occur on disjoint qubits. Furthermore, the operations in the
pattern describe the quantum operations above only. Implicitly inserted
in between them are classical layers which compute the dependent signals
$s$, which may be the parity
(sum modulo 2) of multiple signals: $s = \oplus_i s_i$.

We illustrate this via a simple pattern on two qubits below.

\begin{equation}
X^{s_1}_2 M^{-\alpha}_1 E_{12} N_2 N_1
\end{equation}

In this pattern, both qubits 1 and 2 are initially prepared in the
state $\ket{+}$ and then entangled with a $\Lambda(Z)$ gate.
Qubit 1 is measured in the basis $\ket{\pm_{\alpha}}$ and its
classical outcome is stored in the signal $s_1$. Finally,
qubit 2 is corrected with a Pauli $X$ based on the outcome of
the measurement.

The entanglement graph $G = (V,E)$ defines all the two-qubit entanglement
operations (edges in $E$) between qubits (vertices in $V$). Furthermore,
there are special vertices which represent the input
qubits $I \subset V$ and the output qubits $O \subset V$. In the one-way
MBQC model that we are exclusively considering,
the geometry of the entanglement graph always has the following form
corresponding to an $n$-qubit circuit: it is a
rectangular, regular 2D lattice of $n \times D(n)$ qubits where the
leftmost column of $n$ qubits are the inputs and the rightmost
column of $n$ qubits are the outputs. An
MBQC pattern can always be standardized so that measurements and corrections
always proceed from left-to-right across this so-called cluster state graph
\cite{Raussendorf2003}.

The preparation commands are often omitted since it is implied that they
are always done for all qubits except the input.
Measurements can also be done in a basis which depends on previous
measurement outcomes. These are written as measurements which are
preceded by some $X$ and $Z$ correction, which themselves are dependent.

\begin{equation}
_t\left[M^{\alpha}_i\right]^s \equiv M^{\alpha}_i X^s_i Z^t_i =
M_i^{(-1)^s \alpha + t\pi}
\end{equation}

An MBQC pattern, since it has circuit-like properties, also consumes
circuit depth, width, and size. The depth is divided up into
preparation depth (which involves applying the quantum operations
$N_i$ and $E_{ij}$) and computation depth (which involves applying
the operations $M^{\alpha}_i$, $X^{s}_i$, $Z^{t}_i$).

A pattern can be optimized in polynomial classical time so that
all preparation and entanglement occurs first in the preparation depth,
all measurements and $X$ corrections come next in interleaved layers of
quantum and classical processing (the computation depth),
and finally all the $Z$ corrections come
last. The preparation depth is equal to the maximum degree of the
underlying graph $G$, which is always $4$. The $Z$ corrections can
always be performed last.
Therefore our depth bottleneck
comes from our measurement commands and their dependencies.

Translations between MBQC patterns and quantum circuits are most easily
done using the following (universal) basis of $\{J(\alpha), \Lambda(Z)\}$:

\begin{equation}
J(\alpha) = \normtwo \left[
\begin{array}{cc}
1 & e^{i\alpha} \\
1 & -e^{i\alpha}
\end{array}
\right]
\qquad
\Lambda(Z) = \left[
\begin{array}{cccc}
1 & 0 & 0 & 0 \\
0 & 1 & 0 & 0 \\
0 & 0 & 1 & 0 \\
0 & 0 & 0 & -1
\end{array}
\right]
\end{equation}

Note that for the special angle of $0$,
$J(0) = H$, our usual Hadamard gate.
This is universal because it contains at least one entangling two-qubit
gate, and arbitrary single qubit rotations can be implemented using
the $\{J(\alpha)\}$ basis as shown below for angles $\phi$, $\beta$, $\gamma$,
and $\delta$.

\begin{equation}
U = e^{i\phi}J(0)J(\beta)J(\gamma)J(\delta)
\end{equation}

However, note that this basis is not
fixed and finite. Further restrictions must be placed on MBQC patterns
in order to meet fault-tolerance requirements. Namely, the angles
$\alpha$ should be drawn from a finite set that that can be compiled from
a fault-tolerant basis such as Clifford+$T$ or Clifford+$Toffoli$, as
done by the quantum compilers in Chapter \ref{chap:qcompile}.

In fact, we conjecture that because of this fundamental quantum
compiling limitation, no MBQC pattern can ever have depth smaller than
the corresponding quantum circuit over a fixed, finite basis. This does,
however, open up the question of efficient quantum compiling over
the $\{\Lambda(Z), J(\alpha)\}$ basis.

%%%%%%%%%%%%%%%%%%%%%%%%%%%%%%%%%%%%%%%%%%%%%%%%%%%%%%%%%%%%%%%%%%%%%%%%%%%%%%
\subsection{Automating Circuit Parallelism with MBQC}
\label{subsec:mbqc-par}

The work by Broadbent-Kashefi introduced the application of a
\emph{measurement calculus} to transform MBQC patterns and provide a
pattern for automated parallelization of quantum circuits. This is a
classical, compilation-like procedure which takes as input a quantum
circuit and returns as output a new quantum circuit with at least the
same depth and in some cases improved depth, with a corresponding
increase in circuit size and width.

The basic
technique involves translating a unitary circuit $C$ from a certain basis
into an MBQC pattern $P$. Two optimizations are used from the measurement
calculus: standardization and signal-shifting. Standardization applies
the rules below to make sure all patterns are well-formed: all
preparation commands precede all entanglement operations, which themselves
precede all measurements and corrections. This is useful for
standardizing two patterns $P^(1)$ and $P^(2)$, which themselves may be
standardized and which are concatenated together. Such a concatenation
may occur when we are translating two concatenated unitary circuits
which may individually be easy to translate to patterns but together may
be difficult. This example is illustrated below, where $C^{(x)}$,
$M^{(x)}$, and $E^{(x)}$ correspond to correction operations, measurement
operations, and entanglement operations for pattern $P^{(x)}$.
The symbol $\rightarrow^{*}$ indicates the transformation of standardization.

\begin{eqnarray}
P^{(1)} & = & C^{(1)}M^{(1)}E^{(1)} \\
P^{(2)} & = & C^{(2)}M^{(2)}E^{(2)} \\
P^{(1)}P^{(2)} & = & C^{(1)}M^{(1)}E^{(1)}C^{(2)}M^{(2)}E^{(2)} \\
P^{(1)}P^{(2)} & \rightarrow^{*} & C^{(1)}C^{(2)}M^{(1)}M^{(2)}E^{(1)}E^{(2)} \\
\end{eqnarray}

Signal-shifting further optimizes a pattern by moving all $Z$ corrections to
the end of the pattern, where they can all be performed in parallel. On a
cluster-state graph, the preparation depth and $Z$-correction depth are then
constant. The computation depth of the pattern is dominated by the
depth of dependent measurements and $X$ corrections.

Because the cluster state graph is a \textsf{2D CCNTC} lattice, it is natural to
wonder whether the circuit translation techniques of Broadbent-Kashefi
can be used to automatically map any quantum circuit to a \textsf{2D CCNTC}
architecture. Indeed, the discovery of a constant-depth teleportation
circuit on \textsf{2D CCNTC} proceeds directly from an MBQC pattern
for long-range teleportation. This is illustrated in the next equation,
which is Equation 15 from \cite{Broadbent2007},
which is a pattern for teleporting qubit $i$ to qubit $k$.

\begin{equation}
X^{s_j}_k Z^{s_i}_k M^0_j M^0_i E_{jk} E_{ij}
\end{equation}

Furthermore, the quintessential example of a tight logarithmic separation
between the MBQC and circuit model is the parity function.
On a non-adaptive quantum circuit (one without a classical controller),
the depth of computing parity is $\Omega(\log n)$ \cite{Fang2003}.
However, as an MBQC pattern, it takes constant depth \cite{Broadbent2007}.
Indeed the pattern for parity is very similar to our circuit for unbounded
quantum fanout, which is not surprising given that the two functions are
related by conjugation of Hadamards on every qubit \cite{Moore1999}.

Lemma 7.5 from \cite{Broadbent2007} answers affirmatively that any
non-nearest-neighbor circuit can be mapped to a nearest-neighbor
circuit with constant-depth overhead and the following time-space tradeoff.
We restate it here.

\begin{proposition}{\textbf{Time-space Tradeoff for Mapping Circuits to Nearest-Neighbor} \cite{Broadbent2007}}
Let $C$ be a quantum circuit with depth $D$, size $S$, and width $W$, with $J_c$ the number
of $J(\alpha)$ gates, and $m$ the number of places where two $\Lambda(Z)$ operate
consecutively on the same qubit. Then the corresponding MBQC pattern $P$
on a cluster-state graph, with the teleportation pattern above, has depth
$D' = O(D)$ and width $W' = W + J_c + m = O(W+S)$.
\label{prop:ts-mbqc}
\end{proposition}

This gives a time-space (depth-width) tradeoff for compiled,
nearest-neighbor circuits of $D'W' = O(D(W+S))$, where $D$, $S$, and $W$
are the depth, size, and width of the original, non-nearest-neighbor circuit.

Other than a nearest-neighbor mapping, however, automated techniques cannot
provide an asymptotically lower depth for generic circuits.
There are other special classes of circuits, namely those composed entirely of
Clifford gates and an initial layer of $J(\alpha)$ gates, which can be
parallelized to $O(1)$ depth.

We now compare MBQC to two similar automated mappings, or qubit re-ordering networks,
for any $n$-qubit circuit (on \textsf{AC}).
The main application of such a re-ordering is to convert quantum circuits
to a nearest-neighbor architecture. The re-ordering network of Rosenbaum
\cite{Rosenbaum2012} has the same constant-depth overhead of $D' = O(D)$
but a quadratic width overhead of $W' = O(W^2)$.
The re-ordering network of Beals et al. \cite{Beals2012} allows
a distributed quantum computer with nodes connected in a hypergraph
topology
(equivalent to \textsf{2D CCNTCM}) to execute the same circuit with
$D' = O(D\log^2n\log\log n)$.

An MBQC pattern, when executed on a cluster-state graph, has a well-defined
circuit coherence based on Section \ref{sec:cohere-def}. For a cluster-state
graph with $n \times D$ qubits, including the input and output qubits,
corresponding to a unitary circuit on $n$-qubits
and depth $D$, the circuit width is $W = nD$. The entire lattice starts out
in an entangled state, and measurements proceed column-by-column
left-to-right from the input qubits to the output qubits. Therefore,
the circuit coherence is $Q = D\cdot W = O(n^2D^2)$ whereas $S = O(nD)$, and in fact, there is no
asymptotic separation between circuit coherence and the depth-width product.

\section{Circuit Coherence as a Time-Space Tradeoff}
\label{sec:cohere-tradeoff}

Although circuit coherence's motivation was to capture another resource
of interest to optimize, any new resource may introduce tradeoffs with
existing resources. In this case, decreasing circuit coherence
may increase size and indirectly depth.

First, we will describe a special form for quantum circuits called
layered form in Section \ref{subsec:cohere-lqc}. This will make
it easier to use time-space tradeoff results from Bennett's
reversible pebble game, described in Section \ref{subsec:cohere-pebble}.
Then in Section \ref{subsec:cohere-factor}, we describe a conjectured asymptotic separation
between circuit coherence and the depth-width product for modular
multiplication in Shor's factoring algorithm. Finally, we conclude with
promising directions for future research regarding factoring architectures
and circuit coherence.

%%%%%%%%%%%%%%%%%%%%%%%%%%%%%%%%%%%%%%%%%%%%%%%%%%%%%%%%%%%%%%%%%%%%%%%%%%%%%%
\subsection{Circuits in Layered Form}
\label{subsec:cohere-lqc}

We now present a special form for quantum circuits which will be useful
later in proving facts about circuit coherence, called \emph{layered form}.
It applies to quantum algorithms in which gates execute
monotonically from input qubits to output qubits in parallel layers,
and therefore the physical layout of the qubits mimic the logical
execution of gates. This is useful for circuits which leave
garbage ancillae behind in each layer. In those cases,
circuit coherence can be improved at a negligible increase
in circuit depth and size. This is exactly the form of our
nearest-neighbor factoring architectures in earlier chapters.

Layered form is later used to leverage
results from reversible pebble games on a linear graph. It is also
similar to the circuits which can be parallelized to
constant-depth
MBQC patterns given in Section \ref{subsec:mbqc-par}, except that
each layer is allowed to introduce ancillae qubits, and multiple layers
can be operated on in parallel.
It is currently open whether
all quantum circuits can be put into layered form.

\begin{definition}{\textbf{Classical layered form for quantum circuits.}}
We say an $n$-qubit quantum circuit is in \emph{layered form} if the following
properties apply. We assume that the circuit is part of an architecture
with a classical controller (\textsf{AC}).

\begin{enumerate}
\item
All gates are single-qubit or two-qubit gates.
\item
All qubits can be arranged in a directed, acyclic graph in $\tilde{D}+1$
layers $(l_0, l_1, \l_2, \ldots, l_{\tilde{D}})$
such that all two-qubit gates only operate between consecutive layers or
within a current layer. The size of a layer is the number of qubits within it,
which is polynomial in $n$: $|l_i| \in poly(n)$.
\item The $n$ input qubits are in $l_0$ and the $n$ output qubits are
in $l_{\tilde{D}}$. All other layers $l_j$ have a number of qubits
bounded by $poly(n)$.
\item All gates can be partitioned into $\tilde{D}$ cohorts $(C_1, \ldots, C_{\tilde{D}})$,
such that all gates in a cohort $C_j$
operate on the same consecutive layers $(l_{j}, l_{j+1})$
or within those two layers.
This is a unique partitioning, modulo the gates which operate entirely within each layer.

\item
All gates in cohort $C_j$ execute before all gates in cohort $G_{j+1}$.
%That is, there are no two groups $G_{j}$ (operating on layers $(l_{j}, l_{j+1})$
%and $G_{j+1}$ (operating on layers $(l_{j+1}, l_{j+2})$)
%such that a gate in $G_{j}$ operates on any qubit in $l_{j+1}$ while or
%after any gate in $G_{j'}$ operates on any qubit in the same layer $l_{j+1}$.
%We call $S''$ the \emph{layered circuit size}.
\end{enumerate}
\label{def:lqc}
\end{definition}

The layers $(\l_1, \ldots, l_{\tilde{D}})$ can be considered layers in
physical space.
The gate groups $\{ G_j \}$ execute in disjoint cohorts which can be considered
layers in time. At any one timestep, only gates from one cohort are executing,
and each cohort executes in order from $C_1$ to $C_{\tilde{D}}$.
%Once all the gates groups from exactly one cohort
%are executing. If the circuit has a classical controller, multiple gate groups
%can be operating concurrently within the cohort.

%Let us call the number of cohorts the cohort depth $D''$, where each gate group
%$G_{j,k}$ (now with two indices) belongs to exactly one cohort $C_k$.
%The number of timesteps in a cohort is the
%maximum that any of its gate groups $G_j$ requires to execute.
%That is, the cohorts represent a
%partitioning in time of gates which execute in groups.

We call the \emph{layer depth} of our layered circuit $\tilde{D}$, the number
of physical layers and also the number of cohorts.
It is at least the total circuit depth $D$, with saturation when each cohort
only contains gates which execute in a single timestep.
%cohort circuit depth $D''$, which itself
%is at least the total circuit depth $D$.

\begin{equation}
\tilde{D} \le D
\end{equation}

Layered quantum circuits as described above never uncompute
a layer. In the worst case, garbage is left behind in every
layer until all layers are potentially full of garbage,
and we are left with our answer in the output qubits and
our input qubits as before. We will allow uncomputation
later.

Most importantly for our proofs below, we can now define
\emph{layer coherence} $\tilde{Q}_i$  for timestep $i$
as the number of layers which intersect with the computational
subset $L_i$ from our algorithm in Section \ref{subsec:cohere-algo}.
The total layer coherence for a circuit is the sum of the
layer coherences in any timestep.

\begin{equation}
\tilde{Q} = \sum{i=1}^{D} \tilde{Q}_i \le \tilde{D}^2
\end{equation}

Without uncomputation, each $\tilde{Q}_i$ is equal to $i$ and upper-bounded by $\tilde{D}$,
so the total layer conherence $\tilde{Q}$ is upper-bounded by the layer depth squared.

We can define the maximum number of qubits in any layer as $w_{max}$.

\begin{equation}
w_{max} = \max{i \in [D]} |l_i|
\end{equation}

Then the total circuit coherence is upper-bounded by the layer coherence
times $w_{max}$, although this will asymptotically be the same as the depth-width product $D\cdot W$.

\begin{equation}
Q \le \sum{i=1}^{D} \tilde{Q}_i \cdot w_{max} = O(D\cdot W)
\end{equation}

We can achieve tighter upper bounds by using the exact layer widths for every
layer $l_j$ which intersects with $L_i$ in every timestep $i$.

\begin{equation}
Q \le \sum{i=1}^{D} \sum{j: l_j \cap L_i} |l_j| \le D\cdot W
\end{equation}

%The second inequality is saturated when each cohort only contains gates which
%execute within a single timestep. The first inequality is saturated when
%in each timestep, cohort $C_i$ operates only on layers $(l_{i},l_{i+1}$.

Then the depth of our circuit is $D$ as defined previously for \textsf{NTC} architectures,
and we have circuit width $W = Dn$.

%If each cohort executes in $O(1)$ depth, then $D'' = O(D)$.

We can now verify that our factoring implementations presented in earlier
chapters is in fact a layered circuit. For modular multiple addition,
each CSA layer is a cohort
which operates in constant depth, therefore, $\tilde{D} = O(D)$.
The results of
each CSA layer are propagated to the next CSA layer, where in another cohort,
those numbers are then added in constant-depth, and so on.
For the partial product creation stage of modular multiplication, we can
define that entire stage as a layer for purposes of computing and uncomputing
garbage bits, even though it has depth $O(\log n)$, size $O(n\log^2 n)$, and
width $O(n^3)$.

So far, we have described a circuit which proceeds through CSA layers
(or just layers, to use the more general layered circuit term) to compute
a final answer, leaving behind garbage in each layer and never uncomputing
them until it has reached the last layer (if ever).
There is a single ``front'' of computation which
proceeds from input qubits to output qubits, both physically over the
circuit in layers and temporally over cohorts, until finally at the end,
the front reverses and uncomputes all the garbage left behind from before.

However,
it is natural to ask whether we
can perform any other kind of intermediate uncomputation to reduce
circuit coherence, at the possible increase of either circuit depth
or circuit size. The answer to this question is the subject of the
rest of this chapter.

%\subsection{Other Time-space Tradeoffs}
%\label{subsec:cohere-ts-other}

%Another study of quantum time-space tradeoffs related to our notion of
%circuit width is the bounded space regime by Klauck \cite{Klauck2003}.
%In that model, the input is read-only and accessed through an oracle.
%Time is counted as the number of 

%Klauck discovered time-space tradeoff upper bounds for the specific
%problem of sorting $n$ numbers.

%In comparison to the classical time-space tradeoff for sorting
%discovered by Borodin-Cook \cite{Borodin1982} of $\Omega(TS)$.	

%%%%%%%%%%%%%%%%%%%%%%%%%%%%%%%%%%%%%%%%%%%%%%%%%%%%%%%%%%%%%%%%%%%%%%%%%%%%%%
\subsection{The Pebble Game and Reversible Time-Space Tradeoffs}
\label{subsec:cohere-pebble}

An important time-space tradeoff for classical reversible Turing machines
originates from the pebble game as studied by Bennett \cite{Bennett1973}.
This is relevant to quantum time-space tradeoffs when simulating
completely classical circuits on quantum inputs, such as many arithmetic
functions and a large part of Shor's factoring algorithm. Moreover, this
pebble game models how a reversible machine can compute an irreversible
function. It has a direct connection to circuit coherence as we shall see
below, since quantum computations, especially low-depth ones, can leave
garbage behind which must be uncomputed.

The pebble game is a stylized setting for studying time-space tradeoffs.
Although it may take place on general graphs, we study a line graph
in analogy to the mechanism of an MBQC pattern and our factoring architectures
from Chapters \ref{chap:factor-polylog} and \ref{chap:factor-sublog}.
In short, imagine a row of $n$ tiles $(t_1, \ldots, t_n)$
in sequence, each of which may
contain at most one pebble. One pebble is placed
on $t_1$ in the first timestep, and the goal is to place a pebble
on tile $t_n$. The only allowable move is that at every timestep,
you may add or remove a pebble from tile $i+1$ if there is a pebble on
$t_i$. Therefore, you can never remove the pebble from tile $t_1$.
(This is known as an input-saving pebble game.
There are other versions of the pebble game, similar to other kinds of
reversible computations, where the input can be removed).

The number of timesteps it takes to place a pebble on tile $n$ is known
as the time $T$. In fact, we can consider them synonymous.
The number of pebbles present on all tiles at any one move $i$
is known as the space $S_i$. We call the space $S$ for the whole pebble
game as the maximum number of any pebbles on a game at any particular time:
$S = \max_{i \in [T]} S_i$.
The obvious strategy for winning the pebble game is
to place a pebble on tile $i$ in timestep $i$, without removing any of them.
This completes in time $T=n$ and space $S = n$. In the case of unbounded
space (unlimited pebbles), this is the optimal depth. However, by
bounding space, we can introduce a time-space product $TS$ and attempt to
upper-bound and lower-bound it.

Knill gave a lower bound for the minimum pebble-game time-space tradeoff
\cite{Knill1995} which is bounded above by $n^3$.

\begin{equation}
TS(n) = 2^{2\sqrt{\log(n)}(1 + o(1))}n = o(n^3), \omega(n^2)
\end{equation}

As a consequence, he obtains a minimum time-space tradeoff for
Shor's factoring algorithm on \textsf{AC}.

\begin{equation}
TS(n) = 2^{2\sqrt{n}(1 + o(1))}n^3 = o(n^4), \omega(n^3)
\end{equation}

The minimum time-space tradeoff for factoring is indeed consistent with the depth-width product of all known
factoring implementations from Table \ref{tab:fpl-results}, including
the current work. The one exception is the approximate 1D NTC factoring
implementation by Kutin \cite{Kutin2006}, which beats the above lower bound.
This suggests that the earlier 1D NTC work by Fowler-Devitt-Hollenberg
\cite{Fowler2004} may achieve the optimal depth-width product.

%%%%%%%%%%%%%%%%%%%%%%%%%%%%%%%%%%%%%%%%%%%%%%%%%%%%%%%%%%%%%%%%%%%%%%%%%%%%%%
\subsection{Pebble Games and Layered Quantum Circuits}
\label{subsec:cohere-pebble-lqc}

Taking a step back, how do we determine $T$? We must return to the original
motivation for the pebble game, in simulating an irreversible Turing machine
on a reversible one \cite{Bennett1989}. The irreversible Turing machine's
running time on a particular input is defined as $T$ and the maximum
space it uses over this time (including the input) is $S$. From this definition,
we get $S \ge T$. What is the corresponding pebble game for the most naive
reversible simulation strategy possible?

\begin{definition}{\textbf{Irreversible pebble game.}}
An irreversible pebble game is one which never removes any pebbles.
It corresponds to a computation on an irreversible Turing machine $M$.
By definition, it has time $T \equiv n$ and some space $S$.
There is only one irreversible pebble game on $n$ tiles, and it is also called
the naive pebbling strategy.
In each move $i$ one can only
read or write the space in tile $t_i$.
We can define the number of pebbles after any move $i$ as the \emph{instantaneous space}
$S_i$, which in this case is just $i$, where $S = S_T$.
Therefore, $S = n$.
\end{definition}

\begin{definition}{\textbf{Pebble-instances.}}
We now define a new quantity \emph{pebble-instances} denoted by
$\mathcal{S}$ which is the total number of pebbles which appear in
any history of the pebble game. This is just the sum of all the
instantaneous spaces $S_i$ defined above.
\begin{equation}
\mathcal{S} = \sum{i=1}^D S_i
\end{equation}

We also define the indicator random variable $t_{i,j}$ which is $1$
if a pebble is on tile $t_i$ at timestep $j$ and zero otherwise.
We can define the instantanous space in terms of these indicators.
\begin{equation}
S_i = \sum_{j=1}^D t_{i,j}
\end{equation}
\end{definition}

We can now provide parallel constructions for an irreversible pebble game
and a layered quantum circuit, where the parameters in the pebble game
upper bound those in the layered quantum circuit.

\begin{theorem}{\textbf{Irreversible pebble game for layered quantum circuit.}}
Consider an irreversible pebble game $P$ for a layered quantum circuit $C$
as defined in 
Definition \ref{def:lqc}. Then pebble game time $T$ and space $S$ are equal to layer depth $\tilde{D}$,
and the number of pebbles at any timestep $i$, called $S_i$, equals the
layer coherence $\tilde{Q}_i$.
\label{thm:ipg-lqc}
\end{theorem}

\begin{proof}
For the given layered quantum circuit $C$, construct a pebble game $P$
with $\tilde{D}$ tiles. For every cohort execution of $C_{i-1}$, $i \in [\tilde{D}]$,
place a pebble on tile $t_i$, for $i = 1, \ldots, D$.
In the first timestep $i=1$, note that setting the quantum inputs for $C$
is the same as executing cohort $C_0$ in layer $l_0$ and 
placing a pebble on tile $t_0$.

Then if there is a pebble on tile $t_j$,
the layer $l_j$ is part of the computation state. In every timestep $i$ then, the layer coherence
$\tilde{Q}_i = i$ and the number of pebbles on the tiles is $i$.
Both $C$ and $P$ complete in timestep $\tilde{D}$.
\end{proof}

During Bennett's reversible
simulation with time $T'$ and $S'$, there is some overhead. That is,
$T' \ge T$ and $S' \ge S$. Bennett's upper bounds for these figures
as well as Sherman-Levine's improvements \cite{Levine1990} are shown
in Table \ref{tab:pebble-ts}. We will make use of these constructions in
the next section.

\begin{table}[hbt!]
\begin{tabular}{|c|c|c|}
\hline
Tradeoff Construction           & $T'$                             & $S'$ \\
\hline
Naive                           & $T$                              & $ST$\\
Bennett \cite{Bennett1989}      & $O(T^{1+\epsilon})$              & $O(\epsilon 2^{1/\epsilon} S \log T)$ \\
Levine-Sherman \cite{Levine1990} & $O(T^{1+\epsilon}/S^{\epsilon})$ & $O(\epsilon 2^{1/\epsilon} S (1 + \log\frac{T}{S}))$ \\
\hline
\hline

\end{tabular}
\caption{Pebble game time-space tradeoffs for reversible simulation of
irreversible Turing machines.}
\label{tab:pebble-ts}
\end{table}

The pebble game as we have presented it can be called a serial, reversible
pebble game because only one pebble is added or removed at a time. This
simplifies our analysis in what follows, but one can also define
a parallel pebble game as one where multiple moves (exactly as
defined for the serial game above) can be made. The locations where these
moves can be made are the boundaries between
a pebble on $t_i$ and no pebble on $t_{i+1}$. For simplicity,
we will not discuss this variation any further.

%%%%%%%%%%%%%%%%%%%%%%%%%%%%%%%%%%%%%%%%%%%%%%%%%%%%%%%%%%%%%%%%%%%%%%%%%%%%%%
\subsection{Instantaneous Coherence and Pebble Game Space}
\label{subsec:cohere-equiv}

Finally, we conclude by describing the connection between the pebble game
and circuit coherence as defined in Section \ref{sec:cohere-def}. To do this,
we now construct a reversible pebble game for a modified layered quantum
circuit which now allows uncomputation.

We will then use the construction of Bennett \cite{Bennett1989},
which simulates an irreversible Turing machine computation on a
reversible Turing machine via a reversible pebble game. We will then
use to show how to execute a layered quantum circuit with reduced
circuit coherence by completing with only the input and output layers
in the computation state.

\begin{definition}{\textbf{Layered quantum circuit with uncomputation.}}
A layered quantum circuit can be augmented to allow layers to be uncomputed
by allowing in timestep $i$ that the cohort $C_i$ can operate as follows:

\begin{itemize}
\item $C_i$ can operate on any consecutive layers $(l_{j},l_{j+1})$,
for any $j \in [\tilde{D}]$,
not just layers $(l_{i},l_{i+1})$.
\item All $C_i$ that operate on a particular layer pair $(l_{j},l_{j+1})$
execute exactly the same gates in the same order, just shifted in time.
Therefore, given a computation state in layer $l_{j}$, any such
$C_i$ either extends the computation state into $l_{j+1}$ if it
was previously all $\ket{0}$, or it uncompute $l_{j+1}$ completely
back to all $\ket{0}$ if that layer was already in the computation state.
\end{itemize}
\end{definition}

%Let us call the number of pebbles present at any timestep of the pebble game
%the instantaneous space, and we will scale each pebble by the 
%width of that layer in the quantum circuit (not the width of the entire circuit).
%This scaled instantaneous space is equal to the computational subset, and
%sum of these scaled instantaneous spaces (the scaled pebble-game space) is
%an upper-bound, within a factor of the maximum layer width in a quantum
%circuit

Now, we will show that reversible pebble games can upper-bound the
layer coherence of a layered quantum circuit with uncomputation.

\begin{theorem}{\textbf{Circuit coherence and pebble game space.}}
Given a layered quantum circuit $C'$ with layer depth $D$ and width $W$, we
can construct a one-dimensional
reversible pebble game $P'$ that has time $T$
and space $S$, which executes in parallel timesteps.
Then the layer coherence $\tilde{Q}$ is upper-bounded by the
total pebble-timesteps $\mathcal{S}$, and the total circuit coherence
is upper-bounded by the following weighted sum:
\begin{equation}
Q \le \sum{i=1}^D \sum{j=1}^D t_{i,j} |l_j|
\end{equation}
\label{thm:pg-cc}
\end{theorem}

\begin{proof}
This construction mirrors the one in Theorem \ref{thm:ipg-lqc},
except now in timestep $i$, when the cohort $C_i$ in $C'$ uncomputes a layer $l_j$
from the layer $l_{j-1}$, in $P'$ we remove the pebble from tile $t_j$
and there is a pebble on tile $t_{j-1}$. Therefore, we maintain the
following invariants at every timestep, starting with $i=1$.

\begin{eqnarray}
        S_i & = & \tilde{Q}_i \\
\mathcal{S} & = & \tilde{Q}
\end{eqnarray}

Now, unlike in the pebble game, where all pebbles on any tile $t_i$
represent unit space, the corresponding circuit layer $l_i$ to that tile
represents $|l_i|$ qubits as part of the computation state. Therefore,
the instantaneous coherence $Q_i$ at timestep $i$ is equal to the
weighted sum of the pebbles which are present in the same timestep.

\begin{equation}
Q_i = \sum{j=1}^{D} t_{i,j} |l_j|
\end{equation}

Since the total circuit coherence is the sum of the instantaneous coherences,
we have the desired result.

\begin{equation}
Q = \sum{i=1}^D Q_i = \sum{i,j=1}^D t_{i,j} |l_j| 
\end{equation}
\end{proof}

\section{Circuit Coherence of Factoring Architectures}
\label{sec:cohere-factor}

In this section, we apply circuit coherence and pebble-game
uncomputing techniques from the previous section to our
factoring architectures. The techniques are generic for
any layered circuit, and are not specific to factoring or
even nearest-neighbor circuits, except that the layered circuits
are nearest-neighbor at the layer level.

In Section \ref{subsec:cohere-conject}, we provide a conjecture
for decreasing circuit coherence for modular multiplication while
only marginally increasing our depth.

In Section \ref{subsec:cohere-factor}, we provide a generalized,
configurable-depth factoring architecture based on
Chapter \ref{chap:factor-polylog}.

%%%%%%%%%%%%%%%%%%%%%%%%%%%%%%%%%%%%%%%%%%%%%%%%%%%%%%%%%%%%%%%%%%%%%%%%%%%%%%
\subsection{Reducing Circuit Coherence with Intermediate Uncomputing}
\label{subsec:cohere-conject}

The naive strategy for computing modular multiplication has
depth $D = O(\log n)$, size $O(n^2)$, and width $O(n^3)$. Therefore, the
circuit coherence is upper-bounded by $O(n^3 \log n)$. However,
Bennett \cite{Bennett1989} with corrections from
Sherman-Levine \cite{Levine1990} have shown that an irreversible
pebble game with time $T$ and space $S$ can be simulated reversibly
with overhead $T' = O(T^{1+\epsilon})$ and $S = O(\epsilon 2^{1/\epsilon} S \log T)$
Therefore, we propose the following conjecture.

\begin{conjecture}
Define a layered circuit $C$ for modular multiplication
with layer depth $\tilde{D} = O(\log n)$ and maximum layer width
$w_{max} = O(n^3)$.
Define $P$ as the pebble game corresponding to the optimal reversible simulation of
an irreversible pebble game on $\tilde{D}$ tiles on a reversible
pebble game with the same number of tiles, as proved by
Li \cite{Li1998} based on the authors above.
Using the results of Theorem \ref{thm:pg-cc}, we can create a new
layered circuit $C'$ with uncomputation that performs the
same modular multiplication with the following conjectured resources:
depth $D' = O(D^{1+\epsilon})$,
the same width $W' = W$
and reduced circuit coherence $Q = O(\epsilon 2^{1/\epsilon} n^3 \log\log n)$.
\end{conjecture}

% To improve: add this back in if time permits

%From the previously best-known architectural depth of
%$O(n^2)$, we improved the depth to $O(\log^2 n)$ in Chapter \ref{chap:factor-polylog},
%and then even beneath that to $O((\log\log n)^2)$ in Chapter \ref{chap:factor-sublog}.
%This last depth was limited only by the depth of quantum compiling a
%single-qubit rotation for the quantum majority gate. However, the improvement
%in depth came at the cost of increasing width and size.
%For the polylogarithmic depth, we calculated a size and width of $O(n^4)$.

%How does this compare with our technique of hand-optimized, nearest-neighbor
%mapping in Chapters \ref{chap:factor-polylog} and \ref{chap:factor-sublog}?
%We repeat the relevant rows from 
%Table \ref{tab:fpl-results} and Table \ref{tab:sublog-resources}, namely
%those that correspond to \textsf{AC} and \textsf{2D CCNTCM} implementations
%over the circuit basis of the PK-KSV quantum compiler, which is
%Clifford+$Toffoli$.

%\begin{table}[hbt!]
%\begin{tabular}{|c|c|c|c|c|}
%\hline
%Implementation                           & Architecture  & $D$           & $S$                                             & $W$ \\
%\hline
%Shor \cite{Shor1995}                     & \textsf{AC}   & [$O(n)$]      & $O(n^2\log n \log\log n)$                       & $O(n\log n \log\log n)$ \\
%Browne-Kashefi-Perdrix \cite{Browne2009} & \textsf{CCAC} & $O(1)$        & $O(\frac{1}{\epsilon}n^{6+2\epsilon}\log^{4}n)$ & $$ \\
%Cleve-Watrous \cite{Cleve2000}           & \textsf{AC}   & $O(\log^3 n)$ & $O(n^3)$                                        & $O(n^3)$ \\
%\hline
%Polylog Depth, Chapter \ref{chap:factor-polylog} & \textsf{2D CCNTCM} & $O((\log\log n)^2)$ & $$ & $$ \\
%\hline

%\end{tabular}
%\caption{A comparison for \textsf{AC} factoring architectures.}
%\label{tab:cohere-ac-factor}
%\end{table}



%\begin{table}[hbt!]
%\begin{tabular}{|c|c|c|c|c|}
%\hline
%Implementation                           & Architecture  & $D$ & $S$ & $W$ \\
%\hline
%Shor \cite{Shor1995}                     & \textsf{AC}   & [$O(n)$] & $O(n^2\log n \log\log n)$ & $O(n\log n \log\log n)$ \\
%Browne-Kashefi-Perdrix \cite{Browne2009} & \textsf{CCAC} & $O((\log\log n)^2)$ & $$ & $$ \\
%Cleve-Watrous \cite{Cleve2000}           & \textsf{AC}   & $O((\log\log n)^2)$ & $$ & $$ \\
%\hline
%Polylog Depth, Chapter \ref{chap:factor-polylog} & \textsf{2D CCNTCM} & $O((\log\log n)^2)$ & $$ & $$ \\
%\hline

%\end{tabular}
%\caption{A comparison of hand-optimized and automated nearest-neighbor mappings
%for \textsf{AC} factoring architectures.}
%\label{tab:mbqc-mapping}
%\end{table}

%%%%%%%%%%%%%%%%%%%%%%%%%%%%%%%%%%%%%%%%%%%%%%%%%%%%%%%%%%%%%%%%%%%%%%%%%%%%%%
\subsection{Configurable-Depth Factoring}
\label{subsec:cohere-factor}

When we decrease our nearest-neighbor factoring depth from $O(\log^2 n)$
in Chapter \ref{chap:factor-polylog} to $O((\log\log n)^2)$ in
Chapter \ref{chap:factor-sublog}, we calculated a disproportionate
increase in circuit size and width from
$O(n^4)$ to $\Omega(n^6\log^2 n)$. This seems to be quite an
unfavorable depth-width tradeoff, and it is natural to ask whether
we could have some configurable depth in between
poly-logarithmic and sub-logarithmic that would let a quantum
systems architect choose the right tradeoff for a particular
implementation.

In this section, we provide such a configurable depth
factoring architecture by generalizing our implementation in
Chapter \ref{chap:factor-polylog}. In that chapter, we
wanted to multiply $n\times n$-bit quantum integers in parallel.
To do so, we divided them up into $\lceil n/2 \rceil$ groups of two.
In each group of two quantum integers, call them $\ket{x}$ and
$\ket{y}$, in order to get all the product bits
$\ket{x_i\cdot y_j}$ we need to generate $n^2 \times n$-bit modular residues
$2^i 2^j \bmod m$ controlled on two qubits. We then add these
down with modular multiple addition back to an $n$-bit (CSE) number,
and we are then left with $\lceil n/2 \rceil$ numbers to multiply in
the second level. This takes place in depth $O(\log n)$ and takes
width and size $O(n^3)$.
Modular exponentiation has $\lceil \log_2 n \rceil + 1$ such
levels and perform $(n-1)$ multiplications total, giving us the
final depth of $O(\log n)$ and size and width $O(n^4)$.

The configurable parameter is how we group quantum integers for
expansion. If instead of groups of two, we used groups of
$4$, then each multiplication would require expansion
into $n^4 \times n$-bit numbers. These would still add down to
a single $O(n)$-bit CSE number in $O(\log n)$ depth, but
would now take size and width $O(n^5)$. Modular exponentiation
would still take total depth $O(\log^2 n)$ but would now take
size and width $O(n^6)$, with hidden constants dependent on
the parameter.

We now name the configurable parameter $d$, where in each level
of modular exponentiation we group quantum integers into groups of
$2^d$, and we also expand into $n^{2^d}\times n$ product bits.
We compare this new depth-width tradeoff in Table \ref{tab:config-tradeoff}.

\begin{table}[hbt!]
\centerline{
\begin{tabular}{|c|c|c|c|}
\hline
Implementation                           & $D$ & $W$ \\
\hline
Polylog Depth ($d=1$), Chapter \ref{chap:factor-polylog} & $O(\log^2 n)$ & $O(n^4)$ \\
Config Log Depth ($d$) & $O(\frac{2^d}{d}\log^2 n)$ & $O(\frac{1}{2^d}n^{{2^d}+2})$ \\
($d = \lceil \log_2 n \rceil$)  & $O(n\log n)$ & $O(n^{n+2})$ \\
($d = 1/n$)  & $O(n2^{1/n}\log^2 n)$ & $O(\frac{1}{2^{1/n}}n^{2^{1/n}+2})$ \\
Sublog Depth, Chapter \ref{chap:factor-polylog} & $O((\log\log n)^2)$ & $O(\frac{1}{\epsilon}n^{6 + 2\epsilon}\log^2 n)$ \\
\hline
\end{tabular}
}
\caption{A comparison of configurable-depth factoring architectures and their depth-width tradeoffs.}
\label{tab:config-tradeoff}
\end{table}

The depth-width product for configurable parameter $d$ is
$DW = O(n^{2^d + 3})$. By setting it to be less than one $d = 1/k$
for $k \ge 2$, we are
actually splitting each $n$-bit number into $2^{k-1}$ pieces and grouping
them together. Unfortunately, this configurability does not give us
improved depth-width product; the optimal factoring architecture
configuration appears to be our poly-logarithmic-depth implementation
with $d=1$. However, circuit coherence may provide us with a
better solution.

%Of all these implementations, the one that is the closest to the
%minimum $TS$ from Ref. \cite{Knill1995} is the $O(\log^2 n)$-depth
%factoring implementation from Chapter \ref{chap:factor-polylog).

%Other approaches to configuring 

We end with a conjecture for constructing for low-coherence factoring circuits.
In the original quantum period-finding (QPF) procedure of
Nielsen-Chuang \cite{Nielsen2000}, $n$ quantum integers are multiplied in
series, giving a depth in multiplications of $O(n)$.
In the parallelized construction of Kitaev-Shen-Vyalyi \cite{Kitaev2002},
all $n$ quantum integers are multiplied in parallel with a multiplication
depth of $O(\log n)$. Instead, one can consider another kind of configurable-depth
QPF where $n/2^{d-1}$ integers are multiplied in parallel, % in depth $O(d)$,
and there are $2^{d-1}$ such groups multiplied in serial,
%, giving a total depth of $O(2^d \log n)$,
where $d = \lceil \log_2 n \rceil$ for the serial QPF above,
and $d=1$ for the parallel QPF above.

\begin{conjecture}
Mixed serial-parallel factoring approaches described above,
using uncomputing strategies similar to Bennett or
Sherman-Levine in the previous section,
which interpolate between the Nielsen-Chuang serial QPF and
KSV parallel QPF, will achieve asymptotically lower circuit
coherence than either approach alone.
\end{conjecture}

%The best choice of $d$ for either these configurable-coh

%\section{A New Problem: Hamiltonian Simulation}
\label{sec:cohere-hs-bg}

The original proposal for quantum computing of Feynman in 1982 \cite{Feynman1982}
took the form of simulation. He pointed out that one problem that
quantum physical systems would be inherently better at than classical systems
would be simulating other quantum systems. Today, this is still an interesting
problem for the field. While Shor's factoring algorithm, and other algorithms
which are inspired by processing digital information, exhibit great speedups,
they are also far out of reach of today's experimental capabilities. To
factor a 4096-bit key in the most conservative implementation, with only
one layer of error-correction, would still take tens of thousands of qubits.
In contrast, the problem of simulating the time evolution of a quantum system of 40 particles
is within the reach of experiments within the next decade and already
far surpasses the abilities of all but the most powerful classical supercomputers
today. A seminal work by Lloyd on universal quantum simulation still provides a
good introduction to the field \cite{Lloyd1996}.

Simulation is inter-related to computation. It
plays a complementary role to digital computing, much like the
now-defunct field of analog computing and other analog simulations. Instead
of digitizing variables and representing them indirectly in binary
encodings, abstracting away all physical effects, simulations encode variables
in other analog variables, where the physical state of the analog model
corresponds in some human-readable way to the simulated state of the
original variable. The problem we will discuss is how to approximate
the continuous evolution of a time-independent Hamiltonian using
a discretized-time, circuit-based approach of applying gates to a quantum
state.

Simulation Hamiltonians also has theoretical interest in that there are some
quantum physical systems which we can describe in a model
but can't even simulate quantum mechanical [need citation]. In that case,
there is an interesting twist in our expectations for a mathematical
to be \emph{predictive}. Furthermore, due to the "no fast-forwarding" theorem,
it turns out that even in a quantum simulation, we can't speed up time:
simulating a many-body quantum system for time $t$ still takes time
super-linear in $t$, although we can make it arbitrarily close to
linear \cite{Berry2005}.

Therefore, Hamiltonian simulation remains an interesting problem from the
time of Feynman until today, both for its practical applications, its
experimental feasibility, and its theoretical implications.

%%%%%%%%%%%%%%%%%%%%%%%%%%%%%%%%%%%%%%%%%%%%%%%%%%%%%%%%%%%%%%%%%%%%%%%%%%%%%%
\subsection{Hamiltonians for \\ Computer Scientists}

In a quantum mechanical system, observable quantities are eigenvalues
associated with Hermitian matrices. A Hamiltonian is the operator associated
with the energy of a system, which is a dual to time, similarly to how
position and momentum are duals in the famous Heisenberg uncertainty principle.
Therefore, we know
how a physical system evolves over time $t$ from its Hamiltonian $H$. For
an $n$-qubit system, $H$ has size $2^n \times 2^n$.
Starting from an initial state $\ket{\psi_0}$, we can simulate its time
evolution with the unitary operator which is the
matrix exponential of $H$ scaled by $t$, resulting in a final state $\ket{\psi_f}$.
The form of this matrix exponential is due to solutions of the Schrodinger
equation.

\begin{equation}
\ket{\psi_f} = e^{-iHt} \ket{\psi_0}
\end{equation}

In analogy to 
the Taylor expansion of the function $e^x$, which exponentiates a number,
we can define the Taylor expansion of the matrix $e^{-iHt}$ which is itself
a $2^n\times 2^n$ matrix.

\begin{equation}
e^{-iHt} = \sum_{j=0}^{\infty} \frac{(-iHt)^t}{j!}
\end{equation}

Since this is an infinite sum which converges to the true matrix
$e^{-iHt}$, we can take only the initial $k$ terms to get an approximation
which is accurate up to $O(t^k)$ terms.

%%%%%%%%%%%%%%%%%%%%%%%%%%%%%%%%%%%%%%%%%%%%%%%%%%%%%%%%%%%%%%%%%%%%%%%%%%%%%%
\subsection{The Simulation Procedure}

To state the problem more formally, suppose that you have an initial
quantum state $\ket{\psi_0}$ of $n$ qubits
and you would like to evolve it according to a Hamiltonian $H$ for time
$t$ to get
a final state $\ket{\psi_f}$. You can then operate on this simulated final state
to get some insight into the system defined by $H$, for example, by finding
its ground state energy.
To
enact this simulation, we perform a unitary which is equivalent to the
exponential $e^{-iHT}$.

\begin{equation}
\ket{\psi_f} = e^{-iHT} \ket{\psi_0}
\end{equation}

However, $H$ (and therefore $U_H$)
is an exponentially large matrix in the number of qubits you
are trying to simulate. If you were to generically compile $U_H$, 

We use the following variant of the Trotter formula \cite{Aharonov2003}.
Let $H = \sum_{m=1}^M H_m$ be the Hamiltonian we are trying to simulate, where
each $H_m$ is Hermitian and all have bounded norm, $||H||,||H_m|| \le \Lambda$.
Define an approximation to the time evolution of $H$ over time $t$ by using the
exponentials of its decomposed terms $H_m$ over a small
timestep $\delta$ as in Equation
\ref{eqn:hamsim-ut}.

\begin{eqnarray}
U_\delta & = & [e^{-iH_1 \delta}\cdot e^{-iH_2 \delta} \cdots e^{-iH_M \delta}]\cdot\\
         &   & [e^{-iH_M \delta}\cdot e^{-iH_{M-1} \delta} \cdots e^{-iH_1 \delta}]
\label{eqn:hamsim-ut}
\end{eqnarray}

Then Equation \ref{eqn:hamsim-trotter} bounds the error of approximation
as a function of the simulation timestep $\delta$.

\begin{equation}
||U_{\delta}^{\lfloor \frac{t}{2\delta} \rfloor} - e^{-iHt}||
\le O(M\cdot\Lambda \cdot \delta + M\Lambda^3 t \delta^2)
\label{eqn:hamsim-trotter}
\end{equation}

For fixed total evolution time $t$, Hamiltonian term count $M$,
and norm bound $\Lambda$, we can decrease the error arbitrarily as
$\delta$ goes to zero. In practice, we can make $\delta$ a polynomially
small timestep.

%%%%%%%%%%%%%%%%%%%%%%%%%%%%%%%%%%%%%%%%%%%%%%%%%%%%%%%%%%%%%%%%%%%%%%%%%%%%%%
\subsection{Possibilities for Parallelization}

Berry et al. showed that if we restrict our Hamiltonian $H$ to be
$d$-sparse, we can achieve a bound for the total
number of terms in our decomposition which is $M=6d^2$ \cite{Berry2005},
an improvement over the previous results by Aharonov and Ta-Shma
\cite{Aharonov2003}.
Furthermore, they showed that sublinear simulation time was impossible.
This gives indirect evidence that for a Hamiltonian of this form,
there is no way to decompose it into terms whose exponentials
pairwise commute, since then
it would be possible to parallelize the simulation of the
exponentials $e^{-iH_m t}$ using
the technique of H{\o}yer and {\v S}palek in
Section \ref{sec:parallel}.

Since we cannot parallelize the application of the exponentials themselves,
what about optimizing the depth of their compilations, following our
discussion in \ref{subsec:compile}? In the decomposition above, each $H_m$
was 1-sparse. An interesting open question is whether this structure of
$H_m$ would allow us to optimize the two-level decomposition of
Aho and Svore \cite{Aho2003}.

Now the sun is coming up again, so it's time to wrap things up.

%\section{Depth-Reduction for Hamiltonian Simulation}
\label{sec:cohere-hs-calc}