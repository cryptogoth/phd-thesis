\chapter{Quantum Circuit Coherence}
\label{chap:coherence}

In the first two chapters, we presented low-depth nearest-neighbor
architectures for factoring an $n$-bit number. We improved the depth first to
be sublinear and then sublogarithmic, but at a polynomial increase in size
and width. This represents a time-space tradeoff which can be upper-bounded
by the product of the circuit depth and circuit width. Although the title of
this dissertation indicates that it is beneficial to decrease depth, in
experimental implementations, we may be constrained by other real-world
resources.

Toward this end, we introduce a new circuit resource called \emph{coherence}
which quantifies the amount of error-correction that must be performed
to maintain a coherent quantum computing state. This can be analogous to
the amount of classical controller time or electrical power that a
quantum computing experiment consumes while running an algorithm. We
define our new resource in Section \ref{sec:cohere-def}, discuss its
relationship to other circuit resources on \textsf{2D CCNTCM}, and
calculate coherence for some simple examples to illustrate how it
captures the notion of parallelizability for a circuit.

Decreasing circuit coherence can increase circuit depth or size, which introduces
a new tradeoff for quantifying circuit parallelism.
Therefore, we compare it to other (time-space) tradeoffs that are common in
the literature, including measurement-based quantum computer (MBQC) in
Section \ref{sec:cohere-mbqc} and the reversible pebble game in
Section \ref{sec:cohere-tradeoff}. In both cases, we relate these
other tradeoffs to circuit coherence.

Finally, in Section \ref{sec:cohere-factor}, we calculate the circuit coherence
for an arithmetic building block useful in factoring: modular multiple addition.
We present fascinating conjectures about ways to further decrease
either depth or coherence for factoring architectures. Finally, we conclude
by presenting configurable-depth factoring circuits
that a future experimental architect can use to make the appropriate choice.

%Finally, having exhausted our fascination with factoring,
%we apply our low-depth techniques (compiling and circuit coherence)
%to a new quantum algorithm: hamiltonian
%simulation. In Section \ref{cohere-hs-bg}, we discuss the background of
%this problem, and in Section \ref{cohere-hs-calc} we parallelize one
%aspect of it, that of decreasing the depth of simulating
%a $1$-sparse Hamiltonian matrix.

\section{Definition of Circuit Coherence}
\label{sec:cohere-def}

Usually quantum circuits neglect to draw identity gates. When a bare
quantum wire appears, what is meant is that the qubit maintains its
coherent state until the next non-identity gate comes along to transform it.
However, most quantum circuits are drawn at a logical qubit level,
assuming no errors occur and a coherent state is maintained.
So far we have maintained that abstraction in this dissertation by studying
quantum compiling and quantum architecture independently from
quantum error correction. In this chapter, we move closer to
this abstraction barrier by studying
the effort to maintain a coherent quantum state at the logical qubit level
only for those qubits and only during those timesteps which are useful
for computation.
We quantify this effort with a new circuit resource called \emph{circuit coherence}.

First, we will define what we mean by an \emph{influencing} gate and
a \emph{qubit instance}
in Section \ref{subsec:cohere-entangle}. Then we will build upon this
to define a computational subset of qubits in every
timestep, in Section \ref{subsec:cohere-subset}. Finally,
in Section \ref{subsec:cohere-algo}, we will use
the previous two definitions to provide an algorithm for
computing reachability and therefore the computational subset for a circuit.
This in turn lets us define the resource circuit coherence and
describe its relationship to the other circuit resources: depth, size, and
width. 

%%%%%%%%%%%%%%%%%%%%%%%%%%%%%%%%%%%%%%%%%%%%%%%%%%%%%%%%%%%%%%%%%%%%%%%%%%%%%%
\subsection{Influence and Qubit Instances}
\label{subsec:cohere-entangle}

In the rest of this chapter, we only consider pure quantum states, which are
sufficient for describing circuit coherence. While in reality
mixed quantum states may occur on implementations of our circuits, it
will not affect our definitions here. We will not precede every
instance of a quantum state with the word ``pure,'' but it should be
assumed.

The basic unit of circuit coherence is a \emph{qubit instance} which is
a qubit at a particular timestep. There are $D\cdot W$ qubit instances
in a circuit, but only some of them are computationally useful, depending
on how they interact with each other via gates. We will refer to
qubits and qubit instances with the same label ($u$ and $v$) if a particular
timestep is implied. Qubit instances which
need to be maintained by error-correction are said to \emph{influence}
the result of the output qubits in the last timestep, what we call
the \emph{output qubit instances}. We will make this more formal
as we build up to an algorithm for calculating coherence.

An \emph{entangled} pure quantum state is one which cannot be expressed as the
tensor product of two smaller states. An unentangled pure quantum state
is also known as a \emph{product} state. 
We call the gate $E_{uv}$ \emph{entangling} between two qubits $u$ and $v$
(and between $u$ and $L = V - \{u\}$) given the input states $\rho^{u}$ and $\rho^{L}$ if
the new global state $\rho^{V}$ after applying $E_{uv}$ cannot be expressed as a product state
$\sigma^{u} \otimes \sigma^{L}$.

However, it is useful to characterize a gate as ``potentially entangling''
as a simplification, so that we can reason about the circuit without
worrying about the input states to each gate.
%
\begin{definition}
A gate $E_{uv}$ is \emph{influencing}, or potentially entangling, if
there exists any pure, product quantum input state $\rho^{V} = \rho^{u} \otimes \rho^{L}$
(where $v \in L$) such that the output state $\sigma^{V}$ is entangled.
We can also say that the qubit instance $u$ influences
the qubit instance $v$, or $u$ influences
$L$, in the timestep in which $E_{uv}$ occurs.
We can also say $E_{uv}$ \emph{connects} influence from $u$ to $v$.
\end{definition}

Influencing gates are the means by which any qubit instance can influence
the output qubit instances. This will tell us when we need to \emph{begin}
maintaining a qubit's state, which is after a gate that is
\emph{influencing}.
But how do we know when to \emph{stop} maintaining a qubit's state?
For this purpose of conserving our error-correcting effort, we define
a related concept: an operator which is known to \emph{disconnect}
a qubit instance from influencing other qubit instances.

\begin{definition}
An operator $M$ on a pure $n$-qubit input state $\rho$ is 
\emph{disconnecting} with respect to a given input state $\rho^{V}$,
if the resulting output state is product:
$\rho^{u} \otimes \rho^{L}$ where $L = V - \{u\}$.
We say $u$ is the
qubit instance which has been disconnected.
\end{definition}

We only consider single-qubit and two-qubit disconnecting operators.
The only single-qubit disconnecting operator that we consider
is a projective measurement operator. This projector is always disconnecting,
regardless of the input state.

The disconnecting nature of a gate
may not be apparent just by examining a circuit locally.
To make things more difficult, a disconnecting gate is also often
influencing (for example, CNOT).
However, a quantum algorithm designer able to specify a circuit in terms of
single-qubit and
two-qubit gates often knows when gates are connecting influence and when
they are disconnecting it.
Moreover, the algorithm designer knows which qubits are garbage and that reversing
part of a quantum circuit will uncompute these garbage ancillae back to $\ket{0}$.
In those cases, it is known by design which gates are disconnecting.

This is the case for
quantum circuits in a certain layered form which we describe in
Section \ref{sec:cohere-tradeoff},
of which the 
well-known QFT and factoring circuits are special cases.
As an overestimate of circuit coherence, we can also consider all
two-qubit gates influencing in the worst case
with
only single-qubit projective measurement considered as disconnecting.
However, this will usually
not give an upper-bound separation between coherence and the depth-width product.
We only consider two-qubit gates that are either influencing or disconnecting
for some input states; all other two-qubit gates are a tensor product of
single-qubit gates and will be treated as such.

%%%%%%%%%%%%%%%%%%%%%%%%%%%%%%%%%%%%%%%%%%%%%%%%%%%%%%%%%%%%%%%%%%%%%%%%%%%%
\subsection{Reachability and Computational Subsets}
\label{subsec:cohere-subset}

We refer back to our definition of a quantum circuit on
\textsf{CCNTC}, which is represented by a graph $G = (V,E)$,
with input qubits $I \subseteq V$ and output qubits $O \subseteq V$, and a
classical controller. In particular, the set of all qubits is $V$,
and its size is $|V|=W$, the circuit width.
Our notion of circuit coherence will not depend
on the modules defined in \textsf{CCNTCM}.

%%%%%%%%%%%%%%%%%% DEFINITION
\begin{definition}{\textbf{Influencing Paths.}}
We denote by $E^{(i)}_{uv}$ a two-qubit gate which acts in
timestep $i$ which is influencing and \emph{not} disconnecting for its current states
on $\rho^{u}$ and $\rho^{L}$ for $L = V - \{u\}$, $v \in L$.
An \emph{influencing path} of gates from qubit $u$ in timestep $i_1$ to
qubit $v$ in timestep $i_n$ is
any sequence of influencing (and not disconnecting)
gates $(E^{(i_1)}, E^{(i_2)}, \ldots, E^{(i_n)})$
where the following conditions are met:

\begin{enumerate}
\item
$E^{(i_1)}$ operates on qubit $u$ and $E^{(i_n)}$ operates on qubit $v$.

%\item
%influencing paths always move forward in time: $i_j > i_{j-1}$.
\item
any two consecutive gates in the sequence $(E^{(i_j)},E^{(i_{j+1})})$
act on a common qubit $w$.
\item
any two consecutive gates in the sequence either occur in
consecutive timesteps ($i_j = i_{j-1} + 1$) or are only separated by
gates which are not single-qubit measurements on $w$ in intervening timesteps $i_j < i < i_{j+1}$.
%\item
%every gate $E^{(i_j)}$ encountered in the sequence satisfies the
%following two conditions:

%\begin{enumerate}
%\item it is influencing if
%the path exits it in the forward direction ($i_{j+1} = i_j + 1$)
%\item it is disentangling if the path exits it in the backward direction
%($i_{j+1} = i_j - 1$).
%\end{enumerate}
\end{enumerate}

\end{definition}

%%%%%%%%%%%%%%%%
We can now make our definition of influence more formal.

\begin{definition}{\textbf{Influence and Reachability.}}
A qubit instance $u$ at timestep $i$ is \emph{reachable} from another qubit
instance $v$ in
another (possibly the same) timestep $i'$ if there is some path of influencing
gates that connects them. Conversely, we also say $v$ influences $u$.
Influence is directed and asymmetric; qubit
instances can only be influenced from qubit instances in earlier timesteps.
\end{definition}

Influence also applies to classical measurement outcomes which are
connected by the classical controller.
If one qubit instance occurs right after a single-qubit measurement
(that is, its state contains a classical measurement outcome), any other
qubit instance (including another classical measurement outcome) which
depends on that outcome is influenced by it.

We now define a standard form for circuit in which circuit coherence will be
well-defined.

\begin{definition}{\textbf{Standard form circuits for calculating circuit coherence.}}
Standard form circuits must have the following properties:

\begin{description}
\item[output qubits $O \subseteq V$:] These qubits are semantically defined as
containing the useful outputs of a quantum circuit. They do not have to be
projectively measured. They may, for example, be the control for a
later coherent measurement when cascaded with another quantum circuit.
\item[input qubits $I \subseteq V$:] These qubits are prepared in a 
classical product state (the computational basis)
and are all reachable in timestep $1$ from the
output qubits in timestep $D$.
\item[ancillae qubits:] these are prepared in the product state of all $\ket{0}$'s.
\end{description}
\end{definition}.

We will assume all circuits from now on are in standard form.

\begin{definition}{\textbf{Computational subset, computational set, computational state.}}
The computational subset in timestep $i$ (abbreviated $L_i$) is the subset of
the qubit instances in that timestep
which are reachable from the output qubits $O$.
The \emph{computational set} is the set of computational subsets across
all timesteps.

\begin{equation}
M = \{L_1, L_2, \ldots, L_D \}
\end{equation}
\end{definition}

The computational set are those qubit instances on which an entangled, coherent
quantum state (which we will call the computational state) evolves over time from
the initial preparation of the input qubits $I \subset V$ in timestep $1$
until the output qubits $O \subset V$ are
measured in timestep $D$.
It is measured in qubit instances,
and potentially grows (if influencing, non-disconnecting gates are applied), shrinks
(if disconnecting gates including measurements are applied) or stays the same in size
in every timestep. We note the following
relationships for well-formed circuits.

\begin{equation}
L_1 = I \qquad L_D = O \qquad L_i \subseteq V
\end{equation}

The computational subset can be computed in two passes through
the quantum circuit, one forward to determine reachability and one backward
to determine the computational subset. We can perform this algorithm efficiently on a
layered quantum circuit as defined in Section \ref{subsec:cohere-lqc},
since it is known which gates are disconnecting in that case.

In each timestep $i$, we partition all $W$ qubits in $V$ into disjoint
subsets. We denote these other
(possibly non-computational) qubit subsets as $\{\tilde{L}^{(j)}_i\}$,
of which one is the same as the current
computational subset $L_i$.

This partitioning, like $L_i$, is updated in
every timestep.

Following the definition of \textsf{2D CCNTCM}, each qubit subset is a
contiguous subgraph of the main graph $G$. No two qubit subgraphs share any vertices,
but all vertices are covered and the subgraphs may share edges. All influencing/disconnecting gates $E^{(i)}_{uv}$
that occur during a timestep $i$ are contained in the set $G_i$.

Qubit subsets may potentially share common qubits and become connected by an influencing gate
in a past timestep $i' < i$ from the current timestep $i$.
We keep track of all qubit subsets at a given timestep $i$
in a collection $M_i = \{\tilde{L}^{(j)}_i \}$.

%%%%%%%%%%%%%%%%%%%%%%%%%%%%%%%%%%%%%%%%%%%%%%%%%%%%%%%%%%%%%%%%%%%%%%%%%%%%%%
\subsection{The Coherence Calculation Algorithm}
\label{subsec:cohere-algo}

The following algorithm takes as input a quantum circuit in standard form
with graph $G = (V,E)$, gates in groups $G_i$
over the basis of single-qubit and two-qubit gates
$U(2) \cup U(4)$ including single-qubit measurements in the $Z$-basis.
Gates in group $G_i$ execute in timestep $i$.
The algorithm returns as output the computational subset at every timestep $\{ L_1, L_2, \ldots, L_D \}$.

The data type of qubit subset $\tilde{L}^{(j)}_i$ is the tuple
$(\tilde{V}, P)$ where:

\begin{itemize}
\item $\tilde{V} \subseteq V$ are the qubits (qubit instances) in the subset
\item $P$ is a pointer to the parent qubit subset in timestep ${i-1}$, initially \textsc{NULL}.
\end{itemize}

When we assign one qubit subset to another, we assume we are only assigning
the qubits part of the tuple, $\tilde{V}$.

%%%%%%%%%%%%%%%%% ENUMERATE 1
\begin{enumerate}
\item
Initialize the following:
\begin{itemize}
\item
$L_1 \leftarrow \{ I \}$.
\item
$\tilde{L}^{(j)}_1 = v_j \in V \setminus I$ (all non-input qubits $v_j$ get their own subset)
\item
$M_1 = \left( \bigcup_j \{ \tilde{L}^{(j)}_1 \} \right) \cup \{ L_1 \} $.
\end{itemize}

\item
In timestep $i \in \{2, \ldots, D \}$:

%%%%%%%%%%%%%%%%%%%%%%%%%%%%% ENUMERATE 2
\begin{enumerate}

\item
Compute the classical description of the state on all the qubits $\rho^{V}_i$
from the state $\rho^{V}_{i-1}$
given $M_{i-1}$.
This is assumed to be efficient (e.g. for layered quantum circuits). 
\item
Initialize $M_i \leftarrow \{\}$.
\item
Create two sets:
\begin{itemize}
\item $T_e$ which contains every two-qubit
gate $E_{uv} \in G_i$ that is influencing and not disconnecting
\item $T_d$ which contains every two-qubit
gate $E_{uv} \in G_i$ that is disconnecting,
along with all single-qubit measurements $M_u \in G_i$.
\end{itemize}

\item
For every qubit subset $\tilde{L}^{(j)}_{i-1} \in M_{i-1}$:

%%%%%%%%%%%%%%%%%%%%%%%%%%%%%%%%%%%%%%%% ENUMERATE 3
\begin{enumerate}
\item \textbf{Connecting Case.} Check whether $\tilde{L}^{(j)}_{i-1}$ contains a qubit acted upon by a
$E_{uv} \in T_e$.

%%%%%%%%%%%%%%%%%%%%%%%%%%%%%%%%%%%%%%%%%%%%%%%%%%%% ENUMERATE 4
\begin{enumerate}
\item If it does, call that qubit
$u \in \tilde{L}^{(j)}_{i-1}$.
Check whether $v$ is in any other qubit subset
(call it $\tilde{L}^{(j')}_{i-1}$).

% Too deeply nested
%%%%%%%%%%%%%%%%%%%%%%%%%%%%%%%%%%%%%%%%%%%%%%%%%%%%%%%%%%%%%%%% ENUMERATE 5
%\begin{enumerate}
\item
If it is, create a new qubit subset
$\tilde{L}^{(j)}_{i}$ equal to the union of the two qubit subsets from
step $i-1$:

\begin{equation*}
\tilde{L}^{(j)}_{i} \leftarrow \tilde{L}^{(j)}_{i-1} \cup \tilde{L}^{(j')}_{i-1}
\end{equation*}

Update the parent pointers of $\tilde{L}^{(j)}_{i}$ accordingly.

%\item
%If $v$ is \emph{not} in any other qubit subset for timestep $i-1$,
%then simply add it to a new qubit subset for timestep $i$. Note that it will
%not have the tag \textsc{Computation Subset}.

%\begin{equation*}
%\tilde{L}^{(j)}_{i} = \{v\}
%\end{equation*}
%\end{enumerate}
%%%%%%%%%%%%%%%%%%%%%%%%%%%%%%%%%%%%%%%%%%%%%%%%%%%%%%%%%%%%%%% ENUMERATE 5

\item
Add the current qubit subset to the current timestep's set of qubit subsets $M_i$.

\begin{equation*}
M_i \leftarrow M_i \cup \{ \tilde{L}^{(j)}_{i} \}
\end{equation*}

\end{enumerate}
%%%%%%%%%%%%%%%%%%%%%%%%%%%%%%%%%%%%%%%%%%%%%%%%%%%% ENUMERATE 4

\item \textbf{Disconnecting Case.} Check whether $\tilde{L}^{(j)}_{i-1}$ matches the following two cases
and take the corresponding actions.

%%%%%%%%%%%%%%%%%%%%%%%%%%%%%%%%%%%%%%%%%%%%%%%%%%%% ENUMERATE 4
\begin{enumerate}
\item If $\tilde{L}^{(j)}_{i-1}$ contains two qubits $u$ and $v$
acted upon by some
$E_{uv} \in T_d$, then check whether $E_{uv}$ is
disconnecting between any partitioning of $\tilde{L}^{(j)}_{i-1}$ into two
subsets $V_1 \ni u$ and $V_2 \ni v$.

% Too deeply nested
%%%%%%%%%%%%%%%%%%%%%%%%%%%%%%%%%%%%%%%%%%%%%%%%%%%%%%%%%%%%%%%% ENUMERATE 5
%\begin{enumerate}
\item
If it does, add these two subsets to our collection $M_i$.
Set their parent pointers to $\tilde{L}^{(j)}_{i-1}$.

\begin{equation*}
M_i \leftarrow M_i \cup \{ V_1, V_2 \}
\end{equation*}

\item
Otherwise, just set the current subset $\tilde{L}^{(j)}_{i} \leftarrow \tilde{L}^{(j)}_{i-1}$
with the appropriate parent pointer, and add it.

\begin{equation*}
M_i \leftarrow M_i \cup \{ \tilde{L}^{(j)}_{i} \}
\end{equation*}

%\end{enumerate}
%%%%%%%%%%%%%%%%%%%%%%%%%%%%%%%%%%%%%%%%%%%%%%%%%%%%%%%%%%%%%%%% ENUMERATE 5

\item
If $\tilde{L}^{(j)}_{i+1}$ contains a qubit $u$ acted upon by some $M_u \in T_d$,
then create two new qubit subsets. One just removes the qubit $u$
from the current qubit subset.
%, inheriting the tag \textsc{Computational Subset}
%if present.
The other is a single-qubit subset consisting
only of $u$.

\begin{eqnarray*}
\tilde{L}^{(j)}_{i} & \leftarrow & \tilde{L}^{(j)}_{i-1} - \{u\} \\
\tilde{L}^{(j')}_{i} & \leftarrow & \{ u \}
\end{eqnarray*}

Set the parent pointer of $\tilde{L}^{(j)}_{i}$ accordingly.
Add these to our collection.

\begin{equation*}
M_i \leftarrow M_i \cup \{ \tilde{L}^{(j)}_{i}, \tilde{L}^{(j')}_{i} \}
\end{equation*}

\end{enumerate}
%%%%%%%%%%%%%%%%%%%%%%%%%%%%%%%%%%%%%%%%%%%%%%%%%%%%% ENUMERATE 4

\end{enumerate}
%%%%%%%%%%%%%%%%%%%%%%%%%%%%%%%%%%%%%%%% ENUMERATE 3

\item For every qubit subset $\tilde{L}^{(j)}_{i+1} \in M_{i-1}$ not
operated upon by any of the previous steps, copy it unmodified 
as $\tilde{L}^{(j)}_i$ into
$M_i$ with the appropriate parent pointer.

\end{enumerate}
%%%%%%%%%%%%%%%%%%%%%%%%%%%%% ENUMERATE 2

\item
Do a backward pass from the outputs to the inputs to discover the computational subset $L_i$ in each timestep.

\begin{enumerate}
\item
Verify that the output qubits exactly correspond to one of the qubit subsets
in $M_D$. Call this $L_D$.
\item
Initialize the computational set $M \leftarrow \{ L_D \}$.
\item
Working backwards for timestep $i$ in $(D, D-1, D-2, \ldots, 3, 2)$:
\begin{enumerate}
\item
Find the parent(s) of $L_i$. Create a new set that is the union of them called $L_{i-1}$
and add them to $M$.
\begin{equation}
M \leftarrow M \cup \{L_{i-1}\}
\end{equation}
\end{enumerate}

\item Verify that $L_1 = I$ are exactly the input qubits.

\item Output $M = \{L_1, \ldots, L_D\}$. This is the computational (sub)set of qubits.
\end{enumerate}

\end{enumerate}
%%%%%%%%%%%%%%% ENUMERATE 1

\begin{definition}{\textbf{Instantaneous coherence}.}
Instantaneous coherence $Q_i$ is the size of the computational subset
(in qubits) in timestep $i$. From the algorithm above,
\begin{equation}
Q_i = |L_i|\text{.}
\end{equation}
\end{definition}

\begin{definition}{\textbf{Circuit coherence}.}
Circuit coherence $Q$ is the sum of the computation subset size (in qubits)
over all $D$ timesteps of a quantum circuit's execution. It is measured
in qubit-timesteps, which is the amount of error-correcting effort to
maintain the coherent state of one logical qubit for one timestep of a circuit.

\begin{equation}
Q = \sum_{i=1}^D Q_i = \sum_{i=1}^D |L_i|
\end{equation}
\end{definition}

We will say that a circuit has greater coherence than another circuit if
its $Q$ has a higher value. This should not be confused with the
meaning of coherence as resisting decoherence.

The following relationships hold with other circuit resources.

\begin{equation}
D \le S \le Q \le D\cdot W
\end{equation}

The first inequality holds in the least parallel case, each of $S$ gates is executed in sequence
and $S=D$. The second inequality holds in the least coherent case, when of $S$ gates
either connects or disconnects another qubit from the computational subset in every timestep, and
there are no identity wires within the circuit. The third inequality holds in the
most coherent case, all of $W$ qubits are part of the computational subset for each of $D$ timesteps.

As an example, we can bound the circuit coherence of modular multiplication of $2\times n$-bit
CSE numbers, as described in Section \ref{subsec:mma}. The overall width is $W = O(n^3)$ and
depth is $D=O(\log n)$, so the coherence is upper-bounded by $O(n^3\log n)$. We will suggest possible
ways to reduce this resource in Section \ref{sec:cohere-tradeoff}.


\section{Measurement-Based Quantum Computing Background}
\label{sec:cohere-mbqc}

We will now discuss a model of quantum computing that is very different
from the circuit model, but has already provided us with tools for
parallelism in our nearest-neighbor architectures 
in Chapters \ref{chap:factor-polylog} and \ref{chap:factor-sublog}.
Measurement-based quantum computing (MBQC) is a general model which creates
a large entangled-state on a graph of qubits and performs a pattern of
measurements \cite{Raussendorf2001}.
Later measurements may depend on previous outcomes, and so
a classical controller is required to make each measurement adaptive in this
way. Each measurement reduces the size of our entangled state, and the
measurement operation proceeds
physically across our lattice of qubits, from inputs to outputs.
We will discuss here a restricted form of MBQC called one-way computing
that was proved universal by Raussendorf-Brown-Briegel \cite{Raussendorf2003}
by translation from an arbitrary $n$-qubit circuit.
This model uses only single-qubit measurements on a
regular 2D lattice. We will exclusively consider this model and refer to it
at MBQC for the rest of this section, since it is sufficient for discussing
depth optimization of quantum circuits.

MBQC is very different from the circuit model, which describes unitary
evolution by the application of quantum gates to stationary qubits.
Quantum circuits start out with a product state and then slowly build up
more and more entanglement until it is finally projectively measured at the
end. Surprisingly, both the circuit model and MBQC are equivalent, but
have a tight depth separation for quantum algorithms.

In Section \ref{subsec:mbqc-bg}, we will review the MBQC model.
In Section \ref{subsec:mbqc-par}, we will discuss the work of
Broadbent-Kashefi in automated circuit parallelization, which is a
compilation-like process for reducing circuit depth at the expense of
size and width, another quantum time-space tradeoff. These represent
upper bounds on time-space tradeoffs for mapping certain circuit classes
to a nearest-neighbor circuit with classical controller. We will compare this
to the time-space tradeoff of
other re-ordering networks for mapping circuits to a nearest-neighbor
architecture.time-space tradeoffs for low-depth factoring.
We conclude by comparing and contrasting the
circuit coherence of an MBQC pattern with its other circuit resources.

%%%%%%%%%%%%%%%%%%%%%%%%%%%%%%%%%%%%%%%%%%%%%%%%%%%%%%%%%%%%%%%%%%%%%%%%%%%%%%
\subsection{MBQC Background}
\label{subsec:mbqc-bg}

We follow the exposition of Ref.'s \cite{Broadbent2007,DaSilva2013}
In the MBQC model, quantum computation is represented by three components:
a pattern, an entanglement graph of qubits and interactions, and a
classical controller.

A \emph{pattern} is a sequence of commands which come in five types:

\begin{description}
\item[$N_i$:]
preparation of a qubit $i$ into the state $\ket{+}$.

\item[$E_{ij}$:]
entanglement of qubits $i$ and $j$ with the two-qubit gate
$\Lambda(Z)$ defined below. Note that this gate is bidirectional, so it
does not matter which qubit is the control or the target.

\item[$M^{\alpha}_i$:] single qubit measurement on qubit $i$ which
projects onto the states
$\{ \ket{\pm_{\alpha}} \}$ where

\begin{equation}
\ket{\pm_{\alpha}} = \normtwo(\ket{0} \pm e^{i\alpha}\ket{1})\}
\end{equation}

Associated with every measurement is a signal $s_i \in \mathbb{Z}_2$
which is $0$ for outcome $\ket{+_{\alpha}}$ and $1$ for outcome
$\ket{-_{\alpha}}$.

\item[$X^{s}_i$:] a dependent Pauli $X$ correction, which applies $X$ to
qubit $i$ if the signal $s$ is $1$.

\item[$Z^{t}_j$:] a dependent Pauli $Z$ correction, which applies $Z$ to
qubit $i$ if the signal $t$ is $1$.

\end{description}

A pattern is a valid executable sequence which corresponds to well-defined
quantum and classical operations if no command depends on outcomes that
are not yet measured. Pattern and executed right to left, much like
composing the matrices that make up a sequence of gates to be applied to
a quantum state (column vector). Certain operations can be parallelized
if they occur on disjoint qubits. Furthermore, the operations in the
pattern describe the quantum operations above only. Implicitly inserted
in between them are classical layers which compute the dependent signals
$s$, which may be the parity
(sum modulo 2) of multiple signals: $s = \oplus_i s_i$.

We illustrate this via a simple pattern on two qubits below.

\begin{equation}
X^{s_1}_2 M^{-\alpha}_1 E_{12} N^{0}_2 N^{0}_1
\end{equation}

In this pattern, both qubits 1 and 2 are initially prepared in the
state $\ket{+}$ and then entangled with a $\Lambda(Z)$ gate.
Qubit 1 is measured in the basis $\ket{\pm_{\alpha}}$ and its
classical outcome is stored in the signal $s_1$. Finally,
qubit 2 is corrected with a Pauli $X$ based on the outcome of
the measurement.

The entanglement graph $G = (V,E)$ defines all the two-qubit entanglement
operations (edges in $E$) between qubits (vertices in $V$). Furthermore,
there are special vertices which represent the input
qubits $I \subset V$ and the output qubits $O \subset V$. In the one-way
MBQC model that we are exclusively considering,
the geometry of the entanglement graph always has the following form
corresponding to an $n$-qubit circuit: it is a
rectangular, regular 2D lattice of $n \times D(n)$ qubits where the
leftmost column of $n$ qubits are the inputs and the rightmost
column of $n$ qubits are the outputs. We state without proof that an
MBQC pattern can be standardized so that measurements and corrections
always proceed from left-to-right across this graph.

The preparation commands are often omitted since it is implied that they
are always done for all qubits except the input.
Measurements can also be done in a basis which depends on previous
measurement outcomes. These are written as measurements which are
preceded by some $X$ and $Z$ correction, which themselves are dependent.

\begin{equation}
_t\left[M^{\alpha}_i\right]^s \equiv M^{\alpha}_i X^s_i Z^t_i =
M_i^{(-1)^s \alpha + t\pi}
\end{equation}

An MBQC pattern, since it has circuit-like properties, also consumes
circuit depth, width, and size. The depth is divided up into
preparation depth (which involves applying the quantum operations
$N_i$ and $E_{ij}$) and computation depth (which involves applying
the operations $M^{\alpha}_i$, $X^{s}_i$, $Z^{t}_i$).

A pattern can be optimized in polynomial classical time so that
all preparation and entanglement occurs first in the preparation depth,
all measurements and $X$ corrections come next in interleaved layers of
quantum and classical processing (the computation depth),
and finally all the $Z$ corrections come
last. The preparation depth is equal to the maximum degree of the
underlying graph $G$, which is always $4$. The $Z$ corrections can
always be performed last.
Therefore our depth bottleneck
comes from our measurement commands and their dependencies.

Translations between MBQC patterns and quantum circuits are most easily
done using the following (universal) basis of $\{J(\alpha), \Lambda(Z)\}$:

\begin{equation}
J(\alpha) = \normtwo \left[
\begin{array}{cc}
1 & e^{i\alpha} \\
1 & -e^{i\alpha}
\end{array}
\right]
\qquad
\Lambda(Z) = \left[
\begin{array}{cccc}
1 & 0 & 0 & 0 \\
0 & 1 & 0 & 0 \\
0 & 0 & 1 & 0 \\
0 & 0 & 0 & -1
\end{array}
\right]
\end{equation}

Note that for the special angle of $0$,
$J(0) = H$, our usual Hadamard gate.
This is universal because it contains at least one entangling two-qubit
gate, and arbitrary single qubit rotations can be implemented using
the $\{J(\alpha)\}$ basis as shown below for angles $\phi$, $\beta$, $\gamma$,
and $\delta$.

\begin{equation}
U = e^{i\phi}J(0)J(\beta)J(\gamma)J(\delta)
\end{equation}

However, note that this basis is not
fixed and finite. Further restrictions must be placed on MBQC patterns
in order to meet fault-tolerance requirements. Namely, the angles
$\alpha$ should be drawn from a finite set that that can be compiled from
a fault-tolerant basis such as Clifford+$T$ or Clifford+$Toffoli$, as
done by the quantum compilers in Chapter \ref{chap:qcompile}.

In fact, we conjecture that because of this fundamental quantum
compiling limitation, no MBQC pattern can ever have depth smaller than
the corresponding quantum circuit over a fixed, finite basis. This does,
however, open up the question of efficient quantum compiling over
the $\{\Lambda(Z), J(\alpha)\}$ basis.

%%%%%%%%%%%%%%%%%%%%%%%%%%%%%%%%%%%%%%%%%%%%%%%%%%%%%%%%%%%%%%%%%%%%%%%%%%%%%%
\subsection{Automating Circuit Parallelism with MBQC}
\label{subsec:mbqc-par}

The work by Broadbent-Kashefi introduced the application of a
\emph{measurement calculus} to transform MBQC patterns and provide a
pattern for automated parallelization of quantum circuits. This is a
classical, compilation-like procedure which takes as input a quantum
circuit and returns as output a new quantum circuit with at least the
same depth and in some cases improved depth, with a corresponding
increase in circuit size and width.

The basic
technique involves translating a unitary circuit $C$ from a certain basis
into an MBQC pattern $P$. Two optimizations are used from the measurement
calculus: standardization and signal-shifting. Standardization applies
the rules below to make sure all patterns are well-formed: all
preparation commands precede all entanglement operations, which themselves
precede all measurements and corrections. This is useful for
standardizing two patterns $P^(1)$ and $P^(2)$, which themselves may be
standardized and which are concatenated together. Such a concatenation
may occur when we are translating two concatenated unitary circuits
which may individually be easy to translate to patterns but together may
be difficult. This example is illustrated below, where $C^{(x)}$,
$M^{(x)}$, and $E^{(x)}$ correspond to correction operations, measurement
operations, and entanglement operations for pattern $P^{(x)}$.
The symbol $\rightarrow^{*}$ indicates the transformation of standardization.

\begin{eqnarray}
P^{(1)} & = & C^{(1)}M^{(1)}E^{(1)} \\
P^{(2)} & = & C^{(2)}M^{(2)}E^{(2)} \\
P^{(1)}P^{(2)} & = & C^{(1)}M^{(1)}E^{(1)}C^{(2)}M^{(2)}E^{(2)} \\
P^{(1)}P^{(2)} & \rightarrow^{*} & C^{(1)}C^{(2)}M^{(1)}M^{(2)}E^{(1)}E^{(2)} \\
\end{eqnarray}

Signal-shifting further optimizes a pattern by moving all $Z$ corrections to
the end of the pattern, where they can all be performed in parallel. On a
cluster-state graph, the preparation depth and $Z$-correction depth are then
constant. The computation depth of the pattern is dominated by the
depth of dependent measurements and $X$ corrections.

Because the cluster state graph is a 2D CCNTC lattice, it is natural to
wonder whether the circuit translation techniques of Broadbent-Kashefi
can be used to automatically map any quantum circuit to a 2D CCNTC
architecture. Indeed, the discovery of a constant-depth teleportation
circuit on \textsf{2D CCNTC} proceeds directly from an MBQC pattern
for long-range teleportation. This is illustrated in the next equation,
which is a pattern for teleporting qubit $i$ to qubit $k$.

\begin{equation}
X^{s_j}_k X^{s_i}_k M^0_j M^0_i E_{jk} E_{ij}
\end{equation}

Furthermore, the quintessential example of a tight logarithmic separation
between the MBQC and circuit model is the parity function.
On a non-adaptive quantum circuit (one without a classical controller),
the depth of computing parity is $\Omega(\log n)$ \cite{Fang2003}.
However, as an MBQC pattern, it takes constant depth \cite{Broadbent2007}.
Indeed the pattern for parity is very similar to our circuit for unbounded
quantum fanout, which is not surprising given that the two functions are
related by conjugation of Hadamards on every qubit \cite{Moore1999}.

Lemma 7.5 from \cite{Broadbent2007} answers affirmatively that any
non-nearest-neighbor circuit can be mapped to a nearest-neighbor
circuit with constant-depth overhead and the following time-space tradeoff.
We restate it here.

\begin{proposition}{\textbf{Time-space Tradeoff for Mapping Circuits to Nearest-Neighbor} \cite{Broadbent2007}}
Let $C$ be a quantum circuit with depth $D$, size $S$, and width $W$, with $J_c$ the number
of $J(\alpha)$ gates, and $m$ the number of places where two $\Lambda(Z)$ operate
consecutively on the same qubit. Then the corresponding MBQC pattern $P$
on a cluster-state graph, with the teleportation pattern above, has depth
$D' = O(D)$ and width $W' = W + J_c + m = O(W+S)$.
\label{prop:ts-mbqc}
\end{proposition}

This gives a time-space (depth-width) tradeoff for compiled,
nearest-neighbor circuits of $D'W' = O(D(W+S))$, where $D$, $S$, and $W$
are the depth, size, and width of the original, non-nearest-neighbor circuit.

Other than a nearest-neighbor mapping, however, automated techniques cannot
provide an asymptotically lower depth for generic circuits.
There are other special classes of circuits, namely those composed entirely of
Clifford gates and an initial layer of $J(\alpha)$ gates, which can be
parallelized to $O(1)$ depth.

We now compare MBQC to two similar automated mappings, or qubit re-ordering networks,
for any $n$-qubit circuit (on \textsf{AC}).
The main application of such a re-ordering is to convert quantum circuits
to a nearest-neighbor architecture. The re-ordering network of Rosenbaum
\cite{Rosenbaum2012} has the same constant-depth overhead of $D' = O(D)$
but a quadratic width overhead of $W' = O(W^2)$.
The re-ordering network of Beals et al. \cite{Beals2012} allows
a distributed quantum computer with nodes connected in a hypergraph
topology
(equivalent to \textsf{2D CCNTCM}) to execute the same circuit with
$D' = O(D\log^2n\log\log n)$.

An MBQC pattern, when executed on a cluster-state graph, has a well-defined
circuit coherence based on Section \ref{sec:cohere-def}. For a cluster-state
graph with $n \times D$ qubits, including the input and output qubits,
corresponding to a unitary circuit on $n$-qubits
and depth $D$, the circuit width is $W = nD$. The entire lattice starts out
in an entangled state, and measurements proceed column-by-column
left-to-right from the input qubits to the output qubits. Therefore,
the circuit coherence is $Q = D\cdot W = O(n^2D^2)$ whereas $S = O(nD)$, and in fact, there is no
asymptotic separation between circuit coherence and the depth-width product.

\section{Circuit Coherence as a Time-Space Tradeoff}
\label{sec:cohere-tradeoff}

Although circuit coherence's motivation was to capture another resource
of interest that may not be 

%%%%%%%%%%%%%%%%%%%%%%%%%%%%%%%%%%%%%%%%%%%%%%%%%%%%%%%%%%%%%%%%%%%%%%%%%%%%%%
\subsection{Measurement-based Quantum Computing Time-Space Tradeoffs}
\label{subsec:cohere-mbqc}

Our model \textsf{2D CCNTCM} bears some resemblance to the
cluster state used to prove universality of the one-way quantum computer
model by Briegel=Raussendorf \cite{Briegel}. Therefore, a natural question
is to study the relationship between these two models and characterize
how the resources for an algorithm on these two models compare to each other.

First, we must make some simplifications to normalize these two models.
\textsf{2D CCNTCM} as defined in Chapter $\ref{chap:factor-polylog}$ allowed
an irregular planar graph, with constant but bounded degree (no more than
6 for factoring). We will constrain ourself to the regular
2D lattice, as in the cluster state graph. We state without proof that
every \textsf{2D CCNTCM} lattice with $n$ qubits can be embedded in
a regular 2D lattice with at most a constant factor increase in qubits
($O(n)$).

Both models contain a classical controller

\subsection{Other Time-space Tradeoffs}
\label{subsec:cohere-ts-other}

Another study of quantum time-space tradeoffs related to our notion of
circuit width is the bounded space regime by Klauck \cite{Klauck2003}.
In that model, the input is read-only and accessed through an oracle.
Time is counted as the number of 

Klauck discovered time-space tradeoff upper bounds for the specific
problem of sorting $n$ numbers.

In comparison to the classical time-space tradeoff for sorting
discovered by Borodin-Cook \cite{Borodin1982} of $\Omega(TS)$.	

%%%%%%%%%%%%%%%%%%%%%%%%%%%%%%%%%%%%%%%%%%%%%%%%%%%%%%%%%%%%%%%%%%%%%%%%%%%%%%
\subsection{The Pebble Game and Reversible Time-Space Tradeoffs}
\label{subsec:cohere-pebble}

An important time-space tradeoff for classical reversible Turing machines
originates from the pebble game as studied by Bennett \cite{Bennett1973}.
This is relevant to quantum time-space tradeoffs when simulating
completely classical circuits on quantum inputs, such as many arithmetic
functions and a large part of Shor's factoring algorithm. Moreover, this
pebble game models how a reversible machine can compute an irreversible
function. It has a direct connection to circuit coherence as we shall see
below, since quantum computations, especially low-depth ones, can leave
garbage behind which must be uncomputed.

The pebble game is a stylized setting for studying time-space tradeoffs.
Although it may take place on general graphs, we study a line graph
in analogy to the mechanism of an MBQC pattern and our factoring architectures
from Chapters \ref{chap:factor-polylog} and \ref{chap:factor-sublog}.
In short, imagine a row of $n$ tiles in sequence, each of which may
contain at most one pebble. One pebble is placed
on tile $1$ in the first timestep, and the goal is to place a pebble
on tile $n$. The only allowable move is that at every timestep,
you may add or remove a pebble from tile $i+1$ if there is a pebble on
tile $i$. Therefore, you cannot remove the pebble from tile $1$.

The number of timesteps it takes to place a pebble on tile $n$ is known
as the time $T$. The number of pebbles present on all tiles is known
as the space $S$. The obvious strategy for winning the pebble game is
to place a pebble on tile $i$ in timestep $i$, without removing any of them.
This completes in time $T=n$ and space $S = n$. In the case of unbounded
space (unlimited pebbles), this is the optimal depth. However, by
bounding space, we can introduce a time-space product $TS$ and attempt to
upper-bound and lower-bound it.

Knill gave a lower bound for the minimum pebble-game time-space tradeoff
\cite{Knill1995} which is bounded above by $n^3$.

\begin{equation}
TS(n) = 2^{2\sqrt{\log(n)}(1 + o(1))}n = o(n^3), \omega(n^2)
\end{equation}

As a consequence, he obtains a minimum time-space tradeoff for
Shor's factoring algorithm on \textsf{AC}.

\begin{equation}
TS(n) = 2^{2\sqrt{n}(1 + o(1))}n^3 = o(n^4), \omega(n^3)
\end{equation}

The minimum time-space tradeoff for factoring is indeed consistent with the depth-width product of all known
factoring implementations from Table \ref{tab:fpl-results}, including
the current work. The one exception is the approximate 1D NTC factoring
implementation by Kutin \cite{Kutin2006}, which beats the above lower bound.
This suggests that the earlier 1D NTC work by Fowler-Devitt-Hollenberg
\cite{Fowler2004} may achieve the optimal depth-width product.

TODO We need to define a standard form for coherent circuits

Note that in applying
pebble-game time-space tradeoffs to a quantum algorithm, each layer
in a parallel quantum circuit now takes the place of a tile in our
scenario above. Therefore, the space (circuit width) taken by placing a ``pebble'' on
each layer is no longer a constant. We will define the \emph{layer width}
$W_l$
of a layer $l$ in a quantum circuit as the number of new qubits which are operated on in
a given timestep. The maximum layer width $W_l^{max}$
is then well-defined over all the
layers, and the overall circuit width is upper-bounded by the sum of the
layer widths.

\begin{equation}
W \le \sum_{l} W_l
\end{equation}

In fact, $LW$ 

Finally, we conclude by describing the connection between the pebble game
and circuit coherence as defined in Section \ref{sec:cohere-def}.
Let us call the number of pebbles present at any timestep of the pebble game
the instantaneous space, and we will scale each pebble by the 
width of that layer in the quantum circuit (not the width of the entire circuit).
This scaled instantaneous space is equal to the computational subset, and
sum of these scaled instantaneous spaces (the scaled pebble-game space) is
an upper-bound, within a factor of the maximum layer width in a quantum
circuit

\section{Circuit Coherence of Factoring Architectures}
\label{sec:cohere-factor}

In this section, we apply circuit coherence and pebble-game
uncomputing techniques from the previous section to our
factoring architectures. The techniques are generic for
any layered circuit, and are not specific to factoring or
even nearest-neighbor circuits, except that the layered circuits
are nearest-neighbor at the layer level.

In Section \ref{subsec:cohere-conject}, we provide a conjecture
for decreasing circuit coherence for modular multiplication while
only marginally increasing our depth.

In Section \ref{subsec:cohere-factor}, we provide a generalized,
configurable-depth factoring architecture based on
Chapter \ref{chap:factor-polylog}.

%%%%%%%%%%%%%%%%%%%%%%%%%%%%%%%%%%%%%%%%%%%%%%%%%%%%%%%%%%%%%%%%%%%%%%%%%%%%%%
\subsection{Reducing Circuit Coherence with Intermediate Uncomputing}
\label{subsec:cohere-conject}

The naive strategy for computing modular multiplication has
depth $D = O(\log n)$, size $O(n^2)$, and width $O(n^3)$. Therefore, the
circuit coherence is upper-bounded by $O(n^3 \log n)$. However,
Bennett \cite{Bennett1989} with corrections from
Sherman-Levine \cite{Levine1990} have shown that an irreversible
pebble game with time $T$ and space $S$ can be simulated reversibly
with overhead $T' = O(T^{1+\epsilon})$ and $S = O(\epsilon 2^{1/\epsilon} S \log T)$
Therefore, we propose the following conjecture.

\begin{conjecture}
Define a layered circuit $C$ for modular multiplication
with layer depth $\tilde{D} = O(\log n)$ and maximum layer width
$w_{max} = O(n^3)$.
Define $P$ as the pebble game corresponding to the optimal reversible simulation of
an irreversible pebble game on $\tilde{D}$ tiles on a reversible
pebble game with the same number of tiles, as proved by
Li \cite{Li1998} based on the authors above.
Using the results of Theorem \ref{thm:pg-cc}, we can create a new
layered circuit $C'$ with uncomputation that performs the
same modular multiplication with the following conjectured resources:
depth $D' = O(D^{1+\epsilon})$,
the same width $W' = W$
and reduced circuit coherence $Q = O(\epsilon 2^{1/\epsilon} n^3 \log\log n)$.
\end{conjecture}

% To improve: add this back in if time permits

%From the previously best-known architectural depth of
%$O(n^2)$, we improved the depth to $O(\log^2 n)$ in Chapter \ref{chap:factor-polylog},
%and then even beneath that to $O((\log\log n)^2)$ in Chapter \ref{chap:factor-sublog}.
%This last depth was limited only by the depth of quantum compiling a
%single-qubit rotation for the quantum majority gate. However, the improvement
%in depth came at the cost of increasing width and size.
%For the polylogarithmic depth, we calculated a size and width of $O(n^4)$.

%How does this compare with our technique of hand-optimized, nearest-neighbor
%mapping in Chapters \ref{chap:factor-polylog} and \ref{chap:factor-sublog}?
%We repeat the relevant rows from 
%Table \ref{tab:fpl-results} and Table \ref{tab:sublog-resources}, namely
%those that correspond to \textsf{AC} and \textsf{2D CCNTCM} implementations
%over the circuit basis of the PK-KSV quantum compiler, which is
%Clifford+$Toffoli$.

%\begin{table}[hbt!]
%\begin{tabular}{|c|c|c|c|c|}
%\hline
%Implementation                           & Architecture  & $D$           & $S$                                             & $W$ \\
%\hline
%Shor \cite{Shor1995}                     & \textsf{AC}   & [$O(n)$]      & $O(n^2\log n \log\log n)$                       & $O(n\log n \log\log n)$ \\
%Browne-Kashefi-Perdrix \cite{Browne2009} & \textsf{CCAC} & $O(1)$        & $O(\frac{1}{\epsilon}n^{6+2\epsilon}\log^{4}n)$ & $$ \\
%Cleve-Watrous \cite{Cleve2000}           & \textsf{AC}   & $O(\log^3 n)$ & $O(n^3)$                                        & $O(n^3)$ \\
%\hline
%Polylog Depth, Chapter \ref{chap:factor-polylog} & \textsf{2D CCNTCM} & $O((\log\log n)^2)$ & $$ & $$ \\
%\hline

%\end{tabular}
%\caption{A comparison for \textsf{AC} factoring architectures.}
%\label{tab:cohere-ac-factor}
%\end{table}



%\begin{table}[hbt!]
%\begin{tabular}{|c|c|c|c|c|}
%\hline
%Implementation                           & Architecture  & $D$ & $S$ & $W$ \\
%\hline
%Shor \cite{Shor1995}                     & \textsf{AC}   & [$O(n)$] & $O(n^2\log n \log\log n)$ & $O(n\log n \log\log n)$ \\
%Browne-Kashefi-Perdrix \cite{Browne2009} & \textsf{CCAC} & $O((\log\log n)^2)$ & $$ & $$ \\
%Cleve-Watrous \cite{Cleve2000}           & \textsf{AC}   & $O((\log\log n)^2)$ & $$ & $$ \\
%\hline
%Polylog Depth, Chapter \ref{chap:factor-polylog} & \textsf{2D CCNTCM} & $O((\log\log n)^2)$ & $$ & $$ \\
%\hline

%\end{tabular}
%\caption{A comparison of hand-optimized and automated nearest-neighbor mappings
%for \textsf{AC} factoring architectures.}
%\label{tab:mbqc-mapping}
%\end{table}

%%%%%%%%%%%%%%%%%%%%%%%%%%%%%%%%%%%%%%%%%%%%%%%%%%%%%%%%%%%%%%%%%%%%%%%%%%%%%%
\subsection{Configurable-Depth Factoring}
\label{subsec:cohere-factor}

When we decrease our nearest-neighbor factoring depth from $O(\log^2 n)$
in Chapter \ref{chap:factor-polylog} to $O((\log\log n)^2)$ in
Chapter \ref{chap:factor-sublog}, we calculated a disproportionate
increase in circuit size and width from
$O(n^4)$ to $\Omega(n^6\log^2 n)$. This seems to be quite an
unfavorable depth-width tradeoff, and it is natural to ask whether
we could have some configurable depth in between
poly-logarithmic and sub-logarithmic that would let a quantum
systems architect choose the right tradeoff for a particular
implementation.

In this section, we provide such a configurable depth
factoring architecture by generalizing our implementation in
Chapter \ref{chap:factor-polylog}. In that chapter, we
wanted to multiply $n\times n$-bit quantum integers in parallel.
To do so, we divided them up into $\lceil n/2 \rceil$ groups of two.
In each group of two quantum integers, call them $\ket{x}$ and
$\ket{y}$, in order to get all the product bits
$\ket{x_i\cdot y_j}$ we need to generate $n^2 \times n$-bit modular residues
$2^i 2^j \bmod m$ controlled on two qubits. We then add these
down with modular multiple addition back to an $n$-bit (CSE) number,
and we are then left with $\lceil n/2 \rceil$ numbers to multiply in
the second level. This takes place in depth $O(\log n)$ and takes
width and size $O(n^3)$.
Modular exponentiation has $\lceil \log_2 n \rceil + 1$ such
levels and perform $(n-1)$ multiplications total, giving us the
final depth of $O(\log n)$ and size and width $O(n^4)$.

The configurable parameter is how we group quantum integers for
expansion. If instead of groups of two, we used groups of
$4$, then each multiplication would require expansion
into $n^4 \times n$-bit numbers. These would still add down to
a single $O(n)$-bit CSE number in $O(\log n)$ depth, but
would now take size and width $O(n^5)$. Modular exponentiation
would still take total depth $O(\log^2 n)$ but would now take
size and width $O(n^6)$, with hidden constants dependent on
the parameter.

We now name the configurable parameter $d$, where in each level
of modular exponentiation we group quantum integers into groups of
$2^d$, and we also expand into $n^{2^d}\times n$ product bits.
We compare this new depth-width tradeoff in Table \ref{tab:config-tradeoff}.

\begin{table}[hbt!]
\centerline{
\begin{tabular}{|c|c|c|c|}
\hline
Implementation                           & $D$ & $W$ \\
\hline
Polylog Depth ($d=1$), Chapter \ref{chap:factor-polylog} & $O(\log^2 n)$ & $O(n^4)$ \\
Config Log Depth ($d$) & $O(\frac{2^d}{d}\log^2 n)$ & $O(\frac{1}{2^d}n^{{2^d}+2})$ \\
($d = \lceil \log_2 n \rceil$)  & $O(n\log n)$ & $O(n^{n+2})$ \\
($d = 1/n$)  & $O(n2^{1/n}\log^2 n)$ & $O(\frac{1}{2^{1/n}}n^{2^{1/n}+2})$ \\
Sublog Depth, Chapter \ref{chap:factor-polylog} & $O((\log\log n)^2)$ & $O(\frac{1}{\epsilon}n^{6 + 2\epsilon}\log^2 n)$ \\
\hline
\end{tabular}
}
\caption{A comparison of configurable-depth factoring architectures and their depth-width tradeoffs.}
\label{tab:config-tradeoff}
\end{table}

The depth-width product for configurable parameter $d$ is
$DW = O(n^{2^d + 3})$. By setting it to be less than one $d = 1/k$
for $k \ge 2$, we are
actually splitting each $n$-bit number into $2^{k-1}$ pieces and grouping
them together.

%Of all these implementations, the one that is the closest to the
%minimum $TS$ from Ref. \cite{Knill1995} is the $O(\log^2 n)$-depth
%factoring implementation from Chapter \ref{chap:factor-polylog).

%Other approaches to configuring 

Finally, we suggest a construction for low-coherence factoring circuits.
In the original quantum period-finding (QPF) procedure of
Nielsen-Chuang \cite{Nielsen2000}, $n$ quantum integers are multiplied in
series, giving a depth in multiplications of $O(n)$.
In the parallelized construction of Kitaev-Shen-Vyalyi \cite{Kitaev2002},
all $n$ quantum integers are multiplied in parallel with a multiplication
depth of $O(\log n)$. Instead, one can consider another kind of configurable-depth
QPF where $n/2^{d-1}$ integers are multiplied in parallel, % in depth $O(d)$,
and there are $2^{d-1}$ such groups multiplied in serial,
%, giving a total depth of $O(2^d \log n)$,
where $d = \lceil \log_2 n \rceil$ for the serial QPF above,
and $d=1$ for the parallel QPF above.

Determining other bounds for circuit coherence and extending this
to other quantum algorithms remains a promising area for future research.

%The best choice of $d$ for either these configurable-coh

%\input{coherence/coherence-hs-bg.tex}

%\section{Depth-Reduction for Hamiltonian Simulation}
\label{sec:cohere-hs-calc}