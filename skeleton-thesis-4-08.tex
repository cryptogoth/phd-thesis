\documentclass{article}

\usepackage{amssymb}

\input{Qcircuit}

\begin{document}

\section{The Overall View}

For today, we want to find out some things:

\begin{enumerate}
\item What is the meaning of the $\log_2 n$-bit block size in the proofs for
the block-save adder? It seems that we could have larger block sizes, say
$\log^2_2 n$ and still have a polynomial number of input variables for each
output block?

\item
How can we calculate the $L_1$ norm for the bits of multiplication 
function, since this seems to depend still, in the worst case, of
$n$ partial product bits, for up to $2$ bits.

\item
Why is it that a polynomial with a $L_1$ norm whose difference from that
of the desired target function is inverse polynomially
bounded is a good approximation for the given function? It seems that
two functions could have similar spectral coefficients, and therefore
similar $L_1$ norms, but we very different, for example, just
permuting the spectral coefficients of one polynomial representation.

\item
What is the relationship between MAJORITY and linear threshold elements?

\end{enumerate}

However, the scheme that has become clear from yesterday is that we have
a lower-bound on the number of threshold gates (circuit size) for a 
particular boolean function, and that comes from calculating its $L_1$
norm. We have examples of how to do this for the COMPARE and ADD functions,
which might give us clues for how to do this for MULTIPLY and DIVIDE.
However, the proofs for these seem to involve finding polynomial
threshold functions which approximate the boolean function, by bounding
the difference in spectral norms. I don't seem to understand this part.
There something about needing to approximate taking the logical AND
of two approximating polynomials.

However, to make a true and useful comparison to our polylogarithmic depth
case, we need also upper bounds. This can be done without much understanding
by using the explicit function for ADDITION given in the Bruck-Alon paper.
There are ways to unify layers rather than just naive combination.
How does that affect circuit size? Unknown. Eliminating layers might
change what multinomial terms there are, or polynomial threshold functions,
where each one that has $m$ terms and $n$ inputs takes $mn$ threshold gates
to implement.

So the two prongs of our approach are constructive (upper bounding) 
and non-constructive (lower bounding).

\section{Lower Bounding Terms in the Polynomial Representation}

\section{Polynomial Weights in Threshold Functions}

The way that the quantum threshold gate works in Hoyer-Spalek, and
Takahashi-Tani, is that the rotations by Hamming weight work on a phase
out of $2\pi$. Therefore, an exponentially-large weight function, such
as COMPARE $\in LT_1$, would need to be simulated in three layers of
poly-weighted functions ($\in \hat{LT}_3$). However, it is difficult
to get exponential precision in the phase, so we would prefer it be
polynomially-weighted. In fact, it is difficult to get more than linear
precision in the phase. Rounding off is basically the same approach
that allows us to truncate terms on the QFT.

For the explicit constructions of ADD, we can tell that the weights
are simply $+1$ and $-1$ because they are in class $MAJ_k$.
So this doesn't seem to be a huge problem.

\section{The Activities of a Bodhisattva}

By which I mean, a thesis writer. Keeping notecards is a good idea,
writing what I learned each day, as a tangible reminder. I should date
them too. They make a huge difference in writing, in giving me something
to write about at the end of the day. God it's not even 30 minutes yet.

Other things to write about. Trying to figure out how something works by
bullshitting into LaTeX. Even trying to write your thesis is writing your 
thesis.

\section{DIVISION, EXPONENTIATION, Chinese Remainder Theorem}

It seems really useful to understand this mixed radix representation.
Why would it be easier to work in the coefficients $r_i$ of the
mixed radix $m_i$. Let's try to recreate something from memory.

We have an exponentially large number $Z$ that is larger than
a prime $P_n$. We are going to need to fact-check the hell out of this.

\begin{equation}
P_n = \prod_{i=1}^n p_i^{\alpha_i}
\end{equation}

We define the resides of $Z$ modulo $p_i$.

\begin{equation}
z_i = Z \bmod p_i
\end{equation}

So maybe instead of dividing $x$ and $y$ directly, we divide the mixed 
radix coefficients of one from the other.

Okay compile now. Ten minutes left.

\section{Literature Search}

Tomorrow I should also figure out more recent citations by the Bruck papers,
to see if someone has improved these in recent years.

\end{document}