\section{The Iterated Carry-Save Technique}
\label{sec:blocksave}

Now we mention an alternative technique to sub-logarithmic
factoring by generalizing
the carry-save technique in Section \ref{sec:csa}. We will call this
\emph{iterated carry-save} to distinguish it from the block-save
technique, which is another generalization using classical threshold
circuits \cite{Siu1993}. However, this technique is currently incomplete
in that it depends on a currently unresolved conjecture. It repeatedly
re-encodes the sum of $t\times O(n)$-bit numbers into the
sum of $O(\log t) \times O(n)$-bit numbers We present it
here in the hopes that the reader can solve it.

In Section \ref{subsec:fsl-itlog}, we define the iterated logarithm
function and present the unresolved depth of this technique.

\subsection{Iterated Logarithm and Diagonalization Depth}
\label{subsec:fsl-itlog}

Using iterated carry-save, we can replace the modular multiple adder in
Chapter \ref{chap:factor-polylog}. This
would reduce our modular multiplier from $O(\log n)$ depth to 
a $O((\log^{(*)}n)\cdot f(x))$-depth algorithm. The quantity $f(x)$ indicates
the compiled depth, in $\textsf{CCNTC}$, of a unitary matrix $T$ such that $TMT^{\dag} = D$
where $M$ is the increment operator modulo a power of two, and $D$ is its diagonal form.
The need for this 

\begin{equation}
M\ket{x} \rightarrow \ket{x \bmod 2^k}
\end{equation}

The function $\log^{(*)}n$ indicates
the iterated logarithm function, or the number of times the
logarithm can be applied to a number before it is less than a
constant.
\begin{eqnarray}
\log^{(1)}_2 n & \equiv & \log_2 n \\
\log^{(2)}_2 n & \equiv & \log_2\log_2 n \\
\log^{(*)}_2 n & \equiv & \log_2 \cdots \log_2 n < k \text{ for some positive constant } k\\
\end{eqnarray}
For our purposes, we are interested in base $2$ and
the constant is defined as $\log^{(*)}_2 n \le 2$. This
means we would halt modular multiple addition as before, when we
had converted the sum of $t'\times n$-bit numbers to $2 \times n$-bit numbers.

While this functions grows
slowly (it only has meaningful values less than
$6$ given the number of particles in the universe), it is not
constant.

\subsection{Converting Back to a Conventional Number}
Furthermore, the output is a non-unique carry-save encoding (CSE)
and must be combined into a single unique conventional number.
This can be done by a classical threshold circuit, using a quantum $\text{MAJ}_n$
gate from Section \ref{subsec:fsl-majority},  in
$O(\log n)$ depth and $O(n^2 \log n)$ size as done in Section \ref{subsec:qcla}.
Therefore, by itself, it cannot lead to a sub-logarithmic factoring
implementation.
We mention the proof here for completeness. Future work
may discover that this approach yields lower-depth empirical
results than the majority circuit approach in Section \ref{subsec:fsl-majority},
especially since it does not require quantum compiling single-qubit rotations.

\subsection{The Re-encoding Procedure}
Now we sketch the proof of iterated carry-save.
Instead of re-encoding the sum of 3 input bits ($x = 2^2 x_2 + 2 x_1  + x_0$)
as the sum of 2 output bits ($y = 2 y_1 + y_0$),
we could also re-encode the sum of 7 bits to be the
sum of 3 bits. In analogy to the $3 \rightarrow 2$ adder, we call this a 7-3 adder.
Moreover, the sum of $2^n - 1$ input bits can be re-encoded into the sum
of $n$ output bits. If there are in general $k$ input bits, for $2^{n-1} < k \ge 2^n$,
they can be padded with zero bits up to $2^n$.

\begin{equation}
\sum_{i=0}^{2^n - 2} 2^i x_i = \sum_{j=0}^{n-1} 2^i y_i
\end{equation}

This is the result of Theorem \ref{thm:log-sum}.
However, this only re-encodes bits of the same significance. For adding
$n$-bit numbers, we must regroup the output bits of difference
significances. This is the result of Theorem \ref{thm:par-icsa}.

%%%%%%%%%%%%%%%
\begin{theorem}{Logarithmic Reduction of Multiple-Qubit Sums in Constant-Depth}
The sum of $2^n -1$ qubits $\ket{x_i}$
can be re-encoded into the sum of $n$ qubits $\ket{y_j}$ in
constant depth on CCNTC. Analogously, we can reduce the sum of
$n$ qubits into the sum of $\lceil \log_2 n \rceil$ qubits in
constant depth on CCNTC.
\label{thm:log-sum}
\end{theorem}

\begin{proof}
We provide a circuit which is entirely classical, but can be applied to
quantum inputs on a CCNTC architecture in order to achieve
constant-depth communication.
In the $3 \rightarrow 2$ adder, $y_1$ is the parity of the $x_i$ bits and
$y_0$ is their majority. In fact, the 
generalization of this re-encoding is as follows:
$y_0=1$ when $|x|$ is evenly divisible by $2$, $y_1=1$ when
$|x|$ is evenly divisible by $4$, and so forth. The general
form of the output bits is given below.

\begin{equation}
y_i = \left\{
\begin{array}{rl}
1 & \text{ if } |x| \bmod 2^i = 0 \\
0 & \text{ otherwise } 
\end{array} \right.
\end{equation}

Every round of the re-encoding reduces the number of bits in the
sum by a logarithmic factor. Therefore, the number of re-encodings
needed to reduce the sum of $n$ bits to a sum of $2$ bits is
$\lceil \log^*n \rceil$. It remains to show that round of re-encoding
can be done in constant-depth using quantum fanout.

Each of the $y_i$ bits in a re-encoding round is the output of a
$\text{MOD}[q]$ gate. Using Theorem 2 in \cite{Hoyer2002} and the
quantum fanout and unfanout circuits in Chapter \cite{chap:factor-polylog},
we can
implement this in constant depth. We fanout a target register of
$\lceil \log_2 n \rceil$ qubits $n$ times, one per input qubit $\ket{x_i}$.
In parallel, and using constant-depth teleportation, we can increment
each copy of the target register controlled on each $\ket{x_i}$ modulo
$q$. Since $q$ is fixed, this can be done in constant-depth. Then
we unfanout the target registers and compare the result with all $\ket{0}$'s
using, for example, a quantum $\text{OR}_n$ or $\text{EX}_n$. Such exact
(non-approximate) gates can be accomplished on CCNTC in constant-depth
using the circuits of \cite{Takahashi2011}. The results of this are later shown
in Theorem \ref{thm:ex-ccntc}.
\end{proof}

Now we can perform the reduction in Theorem \ref{thm:log-sum} multiple times
on each bit in parallel of $t' \times n$-bit numbers. To do this, we need
to bound how fast the output numbers grow after every round of parallel,
bitwise applications of the previous logarithmic reduction. The size of
the output numbers produced after $k$ rounds of reduction is given below.

\begin{equation}
n_k = n + \sum_{l=1}^k \lceil \log_2^{(l)}n
\end{equation}

Since any $n_k$ is smaller than the final size of the output numbers,
we can upper-bound any $n_k$ as follows.

\begin{equation}
n_{\log^{(*)}n} \le 2n = O(n)
\end{equation}

%%%%%%%%%%%%%%%
\begin{theorem}{Parallel Multiple Sum in Iterated-Logarithmic Depth}
The sum of $t' \times n$-qubit numbers can be reduced to the sum
of $2 \times 2n$-qubit numbers in $O(\log^{*}n)$ depth on CCNTC.
\end{theorem}

\begin{proof}
Again, we use completely classical circuits, but with quantum inputs and
using quantum fanout and unfanout techniques to map them to CCNTC.

\begin{enumerate}

\item For round $k$ of $\lceil \log_2^{*} n \rceil$ rounds, perform the following
steps on $t'\log_2^{(k)}$ input numbers of size $n_k = O(n)$. 

\begin{enumerate}

\item
For each of (at most) $\log_2^{(k)}t'$ bits of significance $2^i$ in the summands, for 
$0 \le i < n$, apply Theorem \ref{thm:log-sum} to re-encode them as the sum of
$\lceil \log_2^{(k+1)}t' \rceil$ bits (of different significance). 

\end{enumerate}

\end{enumerate}

At the end of the last round, by definition, we are left with $2$ numbers of size
at most $2n$ bits.
\end{proof}