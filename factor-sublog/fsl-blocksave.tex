\section{The Block-Save Technique}

This doesn't really go anywhere, but I will mention here out of
completeness.
As an aside, one may think that one can iterate the carry-save
adder. Instead of re-encoding the sum of 3 bits as the sum of
2 bits, we could also re-encode the sum of 7 bits to be the
sum of 3 bits. In analogy to the 3-2 adder, we call this a 7-3 adder.
How would this re-encoding work?

In the 3-2 adder, there were two output bits of significance $1$
and $2$. The $1$-bit was the parity of the three bits and
the $2$-bit was the majority of the three bits. The truth tables
for these functions are shown in Table \ref{tab:3-2}

\begin{table}
\begin{tabular}{cc|c}
\hline
$x_0$ & $x_1$ & $c$ \\
\hline
0 & 0 & 0 \\
0 & 1 & 1 \\
1 & 0 & 1 \\
1 & 1 & 0 \\
\hline
\end{tabular}
\caption{Truth tables for}
\label{tab:3-2}
\end{table}

 In a 7-3 adder,
likewise, the $1$-bit is the parity of \emph{single} bits.
The $2$-bit is the parity of \emph{pairs} of bits, 
