\section{Quantum Majority Circuits for Modular Exponentiation}
\label{sec:fsl-majority}

A quantum majority circuit is made from quantum $\text{MAJ}_n$ gates.
As mentioned before, depth-$k$
majority circuits are equivalent in power to depth-$k$ LTE circuits
with polynomially-bounded weights: $\textsf{MAJ}_k = \hat{\textsf{LT}}_k$
\cite{Alon1994,Goldmann1994}.

We are also interest in majority gates of polynomial fanin.
In a majority circuit, the fanin of any one majority gate is
bounded above by the circuit size, since in the worst case, one
gate receives an output from every other gate as input. As long
as we restrict ourselves to polynomial-size circuits, we are
assured that our circuit fanin is also polynomial. This is the
primary consideration in achieving sublogarithmic depth for
quantum compiling single-qubit rotations, and therefore for
factoring.

We now contribute a quantum majority circuit for 
modular exponentiation
of an $n$-bit modulus on \textsf{2D CCNTCM}. In this section, we show that each majority circuit
can be implemented in a single module with $O(n)$ qubits. Any
reordering needed between the output qubits of one timestep in a majority circuit
and the input qubits of the next timestep is handled by teleportation
in between modules, as allowed on \textsf{2D CCNTCM}. Therefore, it suffices for
us to do the following:

\begin{enumerate}
\item Show that a (classical) majority circuit exists for (classical) modular exponentiation.
We can translate this to a \textsf{2D CCNTCM} architecture where each $\text{MAJ}_n$ gate
is translated to $O(1)$ modules of $\Omega(n)$ width.
\item Show a \textsf{2D CCNTCM} quantum architecture for a single $\text{MAJ}_n$ gate in
$O(1)$ modules of $\Omega(n)$ width,
assuming the incoming teleportation of qubits of this form $\normtwo(\ket{0} + e^{i\phi}\ket{1})$.
These qubits are used to perform the corresponding single-qubit rotations using the
procedure of Jones et al. \cite{Jones2012}.
\item Include the circuit resources for (KSV) quantum compiler modules that can produce
the qubits in the previous step.
\end{enumerate}

We do the first item by relying on the following two theorems from Yeh-Varvarigos,
which we re-state below without proof. Both of these theorems
allow an extra parameter $\epsilon \in (0,1]$ which determines the
tradeoff between circuit depth and circuit size. They apply to \emph{classical majority circuits}.
Therefore, we must scale all their circuit resources by those for a quantum majority
gate, calculated in Section \ref{subsec:maj-gate}.

\begin{theorem}{\textbf{(Yeh-Varvarigos) Multiple product in constant depth and polynomial size: \cite{Yeh1996}.}}
The $n^2$-bit product of $n\times n$-bit numbers can be computed by a
majority circuit of depth $O(\frac{1}{\epsilon})$,
size and width $O(\frac{1}{\epsilon}n^{3+2\epsilon})$, and
fanin $O(n)$.
\label{thm:mult-prod}
\end{theorem}

\begin{theorem}{\textbf{(Yeh-Varvarigos) Modular reduction in constant depth and polynomial size \cite{Yeh1996}.}}
The $n$-bit binary representation of the modular residue $x \bmod m$, where
$x$ is an $n^2$-bit number and $m$ is an $n$-bit modulus, can be computed
by a majority circuit of depth $O(\frac{1}{\epsilon})$,
size and width $O(\frac{1}{\epsilon}n^{1 + 2\epsilon})$, and
fanin $O(n^2)$.
\label{thm:mod-reduce}
\end{theorem}

Both of these theorems rely on a Chinese Remainder representation for
an $n$-bit number. A conventional binary representation of a number
treats bits as coefficients for weights of $O(2^n)$, which requires exponential
weight to represent in an LTE. In a Chinese Remainder representation,
a number is given a ``mixed-radix'' representation, where each coefficient
is associated with a modular residue for a prime with $O(n)$ bits. In this
way, the weights needed to represent each coefficient are bounded
polynomially and not exponentially.
A more detailed reference of this technique can be found in \cite{Reif1992}.

We delay discussion of quantum compiler considerations until Section
\ref{sec:fsl-qcompile}.

We now accomplish the second item (a concrete architecture for a quantum majority gate)
in the remainder of this section by a sequence of building blocks, each on
\textsf{2D CCNTC} with constant depths and polynomially-bounded sizes and widths.

\begin{itemize}

\item a $\text{BIAS}^{t,\phi}_n$ gate which distinguishes between $|x| = t$ and $|x| = (\lceil n/2 \rceil - t) \bmod n$ with a measurement bias of $e^{i\phi}$.
This is described in Section \ref{subsec:mu-gate}.
\item an $\text{EX}^t_{n\rightarrow \log_n}$ gate which reduces from $\text{EX}^t_n$ (on $n$ qubits)
to $\text{EX}^t_{\log n}$ (on $\lceil \log_2(n+1) \rceil$ qubits).
This is described in Section \ref{subsec:ex-reduce}.
\item an $\text{EX}^t_{\log_n}$ gate which acts on $O(\log n)$ qubits.
This is described in Section \ref{subsec:or-log}.
\end{itemize}

%%%%%%%%%%%
\subsection{BIAS Gate}
\label{subsec:mu-gate}

We define the BIAS gate using the results in \cite{Hoyer2002}, where it is called a $\mu^{|x|-t}_{\phi}$ gate.
We can also think of it as rotating the output qubit $\ket{+}$ by Hamming weight with a threshold $t$ subtracted.
It operates as follows:

\begin{equation}
\text{BIAS}^{t,\phi}_n\ket{x}\ket{+} \rightarrow \ket{x}\ket{\mu^{|x|-t}_{\phi}}
\end{equation}

The output qubit begins in the state $\ket{+}$, which has equal probability of
being measured in the $\ket{0}$ state or the $\ket{1}$ state. It ends in
the following state, which introduces a bias between measuring $\ket{0}$
or $\ket{1}$ proportional to the difference $(|x|-t)$.

\begin{equation}
\ket{\mu^{|x|-t}_{\phi}} = \frac{1 + e^{i\phi(|x|-t)}}{2}\ket{0} + \frac{1 - e^{i\phi(|x|-t)}}{2}\ket{1}
\end{equation}

When $\phi = 2\pi / n$, then the BIAS gate allows us to distinguish
between the case of $|x| = t$ or $|x| = (\lceil n/2 \rceil - t) \bmod n$. As we will
see in the next section, rotations by multiples of $2\pi / n$ will allow us to reduce the
size of an $\text{EX}^t_n$ gate.

\begin{theorem}{\textbf{Constant-depth BIAS gate.}}
The $BIAS^{t,\phi}_n$ gate can be implemented on \textsf{2D CCNTCM} with
a depth of $O(1)$, a size and width of $O(n)$, and
expected $O(n)$ teleported PAR qubits of the form $(\ket{0} + e^{i\phi}\ket{1})$
and $(\ket{0} + e^{-i\phi\cdot t}\ket{1})$.
\label{thm:bias}
\end{theorem}

\begin{proof}
We can lay out the circuit from Figure 1 in \cite{Takahashi2011} on a \textsf{2D CCNTC} lattice as
shown in Figure \ref{fig:mu-circuit}. The size includes a Hadamard to transform
the output qubit into $\ket{+}$, a fanout of this qubit over $n+1$ qubits,
$O(n)$ gates to apply the rotations $R_Z(\phi)$ and $O(1)$ gates to apply
the rotation $R_Z(-\phi\cdot t)$, and a corresponding unfanout, which is
$O(n)$ total. This occurs on $O(n)$ qubits, and can be arranged to take
$O(1)$ depth. This circuit is contained within a single module.
\end{proof}

% TODO: Fill this in
\begin{figure}
\caption{The layout for a BIAS gate on 2D CCNTC.}
\label{fig:mu-circuit}
\end{figure}

%%%%%%%%%%%
\subsection{EX Logarithmic Reduction}
\label{subsec:ex-reduce}

Now we wish to show how to reduce $\text{EX}^t_n$ gate (on $n$ qubits) to an $\text{EX}^t_{\log n}$
gate (on $O(\log n)$ qubits). That is, we wish to implement the following gate
$\text{EX}^t_{n\rightarrow \log_2 n}$ on an $n$-qubit input register $\ket{x}$
to produce an $m$-qubit output register $\ket{y}$, where
$m = \lceil \log_2 n + 1 \rceil$. Running $\text{EX}^t_n$
on $\ket{x}$ should produce the same output qubit $\ket{z}$ as running
$\text{EX}^t_{m}$ on $\ket{y}$. This is formally defined below.

\begin{eqnarray}
\text{EX}^t_{n\rightarrow \log_2 n} \ket{x}\ket{0^m} & \rightarrow &\ket{x}\ket{y} \\
\text{EX}^t_n \ket{x}\ket{0} & \rightarrow & \ket{x}\ket{z} \\
\text{EX}^t_m \ket{y}\ket{0} & \rightarrow & \ket{y}\ket{z}
\end{eqnarray}

\begin{theorem}{\textbf{Constant-depth EX reduction gate.}}
The $EX^t_{n\rightarrow \log_2 n}$ gate can be implemented on \textsf{2D CCNTCM} with
a depth of $O(1)$, a size and width of $O(n\log n)$, and expected
$O(n\log n)$ teleported PAR qubits of the form $(\ket{0} + e^{i\phi_k}\ket{1})$
and $(\ket{0} + e^{-i\phi_k\cdot t}\ket{1})$
\label{thm:ex-reduce}
\end{theorem}

\begin{proof}
We map the construction from Theorem 19 in \cite{Hoyer2002} onto
\textsf{2D CCNTCM}.
This step involves $m = \lceil \log_2 (n+1) \rceil$ parallel $\text{BIAS}^{t,\phi_k}_n$ gates from the last section for
$1 \le k \le m$, where $\phi_k = \frac{2\pi}{m}k$, each in their own modules.
This maintains constant circuit depth while only increasing circuit size by
a $O(\log n)$ factor.
\end{proof}

%%%%%%%%%%%
\subsection{OR on Logarithmic Qubits}
\label{subsec:or-log}

We now map an exact OR gate from Takahashi-Tani \cite{Takahashi2011} to \textsf{2D CCNTCM}.
Unlike the approximate OR gate (and its variation, an approximate EXACT gate) from Hoyer-Spalek \cite{Hoyer2002}, it is both
complete and sound with probability $1$ and it completes in
constant depth. However, it has size exponential in its input size $n'$: $O(n'2^{n'})$.
Therefore, the previous logarithmic reduction in Section \ref{subsec:ex-reduce}
is necessary to reduce $O(n)$ qubits, the fanin of our majority circuit,
to $n' = O(\log n)$, to give us a polynomial circuit size of $O(n\log n)$.
We will denote this gate as $OR^t_{\log n} = OR^t_{n'}$ to emphasize that it
can only be efficiently used when the number of qubits have been reduced to
be logarithmic in the input size of the overall problem (in this case, factoring).
The circuit depth will be constant no matter what, but circuit size is
only polynomially-bounded if the previous condition is met.

\begin{theorem}{\textbf{Constant-depth exact OR gate on logarithmic qubits.}}
The $OR^t_{\log n}$ gate can be implemented on \textsf{2D CCNTCM} with a
circuit depth of $O(1)$, a circuit size and width of $O(n\log n)$, and expected
$O(n \log n)$ teleported PAR qubits of the form $(\ket{0} + e^{i \frac{2\pi}{O(n)}}\ket{1})$.
This takes module depth of $O(1)$, module size of $O(n\log n)$, and
module width $O(n\log n)$.
\label{thm:or-log}
\end{theorem}

\begin{proof}
Following the notation of Lemma 2 from \cite{Takahashi2011}, we have two
modules:

\begin{enumerate}
\item One module contains the input qubits $x_i$ and the $R_j$ and $S$
registers. We will call this the $RS$ module.
\item One module contains the $T$ register, which we call the $T$ module.
\end{enumerate}

First, we consider the $RS$ module.
This has a total of $n'2^{n'} = O(n\log n)$ qubits each.
We use a sorting network from Rosenbaum \cite{Rosenbaum2012} to
reorder qubits and allow for nearest-neighbor interactions.
This requires size and width $O(n^2\log^2 n)$ and depth $O(1)$
using constant-depth teleportation. We also need to perform $O(n\log n)$
Hadamards.
In addition, we need to
perform the following fanouts (and corresponding unfanout).
This is illustrated in Figure 3 from
\cite{Takahashi2011}, which we repeat in Figure \ref{fig:exact-or}.

\begin{itemize}
\item
$n' = O(\log n)$ constant-depth fanouts of $2^{n'} = O(n)$
target qubits from each $x_i$ to each $R_j$ register.
\item
$2^{n'} = O(n)$ constant-depth fanouts of $2^{n'} = O(n)$
target qubits from each $S$ qubit to the $R_j$ registers.
\end{itemize}

The two kinds of fanouts above take, in total, $O(1)$ depth, $O(n^2)$ size,
and $O(n^2)$ width. These are subsumed by the resources needed
for reordering qubits above. The operations in the $RS$ module
do not require any PAR ancillae qubits to be teleported in.

Second, we consider the $T$ module, into which the $O(\log n)$ input qubits
$x_i$ and the $O(n)$ qubits from $S$ are teleported from the $RS$ module.
There is a single
Hadamard gate on the output qubit and a fanout
(and corresponding unfanout) of $O(2^{n'}) = O(n)$ qubits,
which takes a depth of $O(1)$ and a size and width of $O(n)$.
Finally, we are left with $O(n)$ controlled-$R_z$ rotations of
angle $\pi / 2^{n-1}$ which can be done in expected depth of $O(1)$,
 size and width of $O(n)$, and $O(n)$ teleported PAR qubits
 of the form $(\ket{0} + e^{i\phi_k}\ket{1})$.

 This gives us the required resources stated in the theorem.
\end{proof}

\begin{figure}[hbt!]
\caption{A circuit for exact $\text{OR}_3$ from Takahashi-Tani \cite{Takahashi2011}.}
\label{fig:exact-or}
\end{figure}

%%%%%%%%%%%
\subsection{A Majority Gate in \textsf{2D CCNTCM} in Sublogarithmic Depth}
\label{subsec:maj-gate}

Now we have all the building blocks needed to construct a
$\text{MAJ}_n$ gate in sublogarithmic depth. Using the constant-depth
majority circuits of $O(n)$ fanin from \cite{Yeh1996} mentioned at
the beginning of this section, we will then have a sublogarithmic
quantum circuit for modular exponetiation, and therefore for
Shor's factoring algorithm.

\begin{theorem}{\textbf{Constant-depth quantum $\text{MAJ}_n$ gate on \textsf{2D CCNTCM}.}}
A quantum $\text{MAJ}_n$ gate can be implemented on \textsf{2D CCNTCM} with
circuit depth $O(1)$, circuit size and width $O(n^2\log^2 n)$,
module depth $O(1)$, and module size and width $O(n \log n)$.
\label{thm:maj-gate}
\end{theorem}

\begin{proof}
We combine the gates 
Below we give the construction for $\text{MAJ}_{n}(x)$ on the
$n$-qubit input $x$.



\begin{enumerate}

\item
We compute in parallel the gates $\text{EX}^i_{n}(x)$ for
$0 \le i \le \lceil n/2 \rceil$ to determine if the quantum
threshold for majority is reached. There are at most $(n/2) + 1$
such gates. To do this, we need to use Theorem \ref{thm:or-log}.
Each gate $\text{EX}^i_{n}$ requires the following steps:

\begin{enumerate}
\item 
Compute the constant-depth reduction from $\text{EX}^t_{n}$ to
$\text{EX}^t_{m}$ where $m = \lceil \log_2(n+1) \rceil$, using
the reduction from $\text{OR}_n$ to $\text{OR}_{\log_n}$ \cite{Hoyer2002}.
For $1 \le k \le m$, do the following:

\begin{enumerate}
\item
Compute the qubit $\ket{\mu^{|x|-t}_{\phi_k}}$, which is the rotation by Hamming 
weight of $x$, with a threshold $t$ subtracted, by the angle $\phi_k = 2\pi / 2^k$. Note that the
required precision
of this angle is $O(\log \log n)$. This uses the result of Theorem \ref{thm:bias}.
This can be done by a \textsf{2D CCNTCM} circuit
with $O(1)$-depth, $O(n^2)$-size, and $O(n^2)$-width.
\end{enumerate}

At the end of this step, we have $m = O(\log_2 n)$ bits $\ket{y_k}$. If
$|x| \ge t$ then the qubits $\ket{y_k}$ will be orthogonal
to the state $\ket{0^m}$.

\item
Apply the \textsf{2D CCNTCM} circuit for exact $\text{OR}_{\log n}$ from
Theorem \ref{thm:or-log} of
\cite{Takahashi2011} to the output of the previous step. This can
be done with a \textsf{2D CCNTCM} circuit with $O(1)$-depth, $O(n \log n)$-size,
and $O()$ width.

\end{enumerate}

At the end of this step, we have used expected circuit depth of
$O(1)$, expected circuit size and width of $O(n^3 \log^2 n)$,
expected module depth of $O(1)$, and expected module size and
width of $O(n \log n)$.

\item
Apply the gate $\text{PA}_{\lceil n/2 \rceil}$ to the result of
the previous step. This can be done by a \textsf{2D CCNTCM} circuit of
$O(1)$-depth, $O(n)$-size, and $O(n)$-width using constant-depth
fanout, as described in Section \ref{sec:cdc}, and conjugated by
Hadamards on every qubit as described in \cite{Moore1998}. See
the equivalence of these two steps in Figure \ref{fig:pa-fanout}.

\begin{figure}[htb!]
% TODO insert Figure 1 from Hoyer-Spalek
\caption{The equivalence of $\text{PA}_n$ and $\text{FANOUT}_n$ conjugated by Hadamards.}
\label{fig:pa-fanout}
\end{figure}

\item
Apply a NOT to the output of the previous step. This final
output is the output of the quantum majority gate $MAJ_{n}$.

\end{enumerate}

The final resources for a quantum $\text{MAJ}_n$ gate is shown in
Table \ref{tab:maj-resources}, including the KSV quantum compiler
resources from Table \ref{tab:ksv-qcompile}.

\begin{table}[htb!]
\begin{tabular}{c|c|}
\hline
$\langle D \rangle$ & $O(1)$ \\
\hline
$\langle S \rangle$ & $O(n^3\log^2 n)$ \\
\hline
$\langle W \rangle$ & $O(n^3 \log^2 n)$ \\
\hline
$\langle \overline{D} \rangle$ & $O(1)$ \\
\hline
$\langle \overline{S} \rangle$ & $O(n\log n)$ \\
\hline
$\langle \overline{W} \rangle$ & $O(n\log n)$ \\
\hline
\end{tabular}
\caption{Expected circuit resources for a quantum $\text{MAJ}_n$ gate.}
\label{tab:maj-resources}
\end{table}

\end{proof}