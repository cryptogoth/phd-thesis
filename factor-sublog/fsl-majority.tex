\section{Quantum Majority Circuits for Modular Exponentiation}

We now consider circuits made from majority gates, which we denote
as $\text{MAJ}_n \equiv \text{TH}^{\lceil n/2 \rceil}_n$. These have the
advantage of being simpler to implement than general threshold gates
$\text{TH}^t_n$ while being equivalent in power. Namely, depth-$k$
majority circuits are equivalent in power to depth-$k$ threshold circuits
with polynomially-bounded weights: $\textsf{MAJ}_k = \hat{\textsf{LT}}_k$
\cite{Alon1994,Goldmann1994}.

In many works on classical majority circuits, the fanin of a majority
gate is not treated as an important consideration and can in 

\begin{theorem}{Multiple product in constant depth and polynomial size: \cite{Yeh1996}}
The $n^2$-bit product of $n\times n$-bit numbers can be computed by a
circuit of depth $O(\frac{1}{\epsilon})$,
size and width $O(\frac{1}{\epsilon}n^{3+2\epsilon})$, and
fanin $O(n)$.
\end{theorem}

\begin{theorem}{Modular reduction in constant depth and polynomial size \cite{Yeh1996}}
The $n$-bit binary representation of the modular residue $x \bmod m$, where
$x$ is an $n^2$-bit number and $m$ is an $n$-bit modulus, can be computed
by a circuit of depth $O(\frac{1}{\epsilon})$,
size and width $O(\frac{1}{\epsilon}n^{1 + 2\epsilon})$, and
fanin $O(n^2)$.
\end{theorem}



\section{A Majority Gate in 2D CCNTCM in Sub-logarithmic Depth}

To map a majority gate to 2D CCNTCM, we use the $\textsf{AC}_f^0$ 
constructions from Takahashi-Tani \cite{Takahashi2011},
based on the results of
Hoyer-Spalek \cite{Hoyer2002}.
We combine this with our architectural techniques
from the previous chapter, including the 2D CCNTCM circuits
for constant-depth teleportation, fanout, and unfanout.

The majority gate is a threshold gate, with unit weights, a
fan-in (input size) $n' = O(n)$ in the size of the input number to Shor's 
factoring algorithm, and a threshold of $\lceil n/2 \rceil$.

We describe the construction for $\text{MAJ}_{n'}$ below, given the
$n'$-qubit input $x$.

\begin{enumerate}

\item
We compute in parallel the gates $\text{EX}^i_{n}$ for
$0 \le i \le \lceil n/2 \rceil$ to determine if the quantum
threshold for majority is reached. There are at most $(n/2) + 1$
such gates.

\begin{enumerate}
\item 
Compute the constant-depth reduction from $\text{EX}^t_{n'}$ to
$\text{EX}^t_{m}$ where $m = \lceil \log_2(n'=1) \rceil$, using
the reduction from $OR_n$ to $OR_{\log_n}$ \cite{Hoyer2002}.
For $1 \le k \le m$, do the following:

\begin{enumerate}
\item
Compute the qubit $\mu^{|x|-t}_{\phi_k}$, which is the rotation by Hamming 
weight of $x$, with a threshold $t$ subtracted, by the angle $\phi_k = 2\pi / 2^k$. Note that the precision
of this angle is $O(\log \log n)$. This can be done by a 2D CCNTCM circuit
with $O(1)$-depth, $O(n^2)$-size, and $O(n^2)$-width, as described in
Section \ref{subsec:or-reduce}.
\end{enumerate}

At the end of this step, we have $O(\log_2 n)$ bits $\ket{y_k}$ which 
represent the degree to which the Hamming weight of $x$ is close to the
threshold $t = \lceil n/2 \rceil$.

\item
Apply the circuit for exact $\text{OR}_{\log n}$ from Lemma 2 of
\cite{Takahashi2011} to the output of the previous step. This can
be done with a 2D CCNTCM circuit with $O(1)$-depth, $O(n \log n)$-size,
and $O()$.

\end{enumerate}

\item
Apply the gate $\text{PA}_{\lceil n/2 \rceil}$ to the result of
the previous step. This can be done by a 2D CCNTCM circuit of
$O(1)$-depth, $O(n)$-size, and $O(n)$-width using constant-depth
fanout, as described in Section \ref{sec:cdc}, and conjugated by
Hadamards on every qubit as described in \cite{Moore1998}.

\item
Apply a NOT to the output of the previous step. This final
output is the output of the quantum majority gate $MAJ_{n}$.

\end{enumerate}


\subsection{OR Reduction of Hoyer-Spalek}
\label{subsec:or-reduce}

This step involves $m \le (n/2)+1$ gates $\mu^{|x|-t}_{\phi_k}$ for
$1 \le k \le m$. Each gate is the same as Hoyer and Spalek's reduction from
$O(n)$ bits to $O(\log n)$ bits as described in \cite{Hoyer2002}.

The $\mu$ circuits 

\subsection{Exact OR Circuit of Takahashi-Tani}
\label{subsec:or-exact}



\subsection{Parity Circuit}
\label{subsec:parity}

