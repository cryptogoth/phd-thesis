\section{Quantum Majority Circuits for Modular Exponentiation}

A quantum majority circuit is made from majority gates, a special case
of a threshold gate with unit weights and threshold $n/2$. In this section, we refer
to quantum majority gates by this label:
$\text{MAJ}_n \equiv \text{TH}^{\lceil n/2 \rceil}_n$. There has been
a lot of interest in majority gates in the classical circuit complexity
research community due to their simplicity relative to general LTEs
while being equivalent in power. Namely, depth-$k$
majority circuits are equivalent in power to depth-$k$ LTE circuits
with polynomially-bounded weights: $\textsf{MAJ}_k = \hat{\textsf{LT}}_k$
\cite{Alon1994,Goldmann1994}.

We are also interest in majority gates of polynomial fanin.
In a majority circuit, the fanin of any one majority gate is
bounded above by the circuit size, since in the worst case, one
gate receives an output from every other gate as input. As long
as we restrict ourselves to polynomial-size circuits, we are
assured that our circuit fanin is also polynomial. This is the
primary consideration in achieving sublogarithmic depth for
quantum compiling single-qubit rotations, and therefore for
factoring.

We now contribute a quantum majority circuit for 
modular exponentiation
of an $n$-bit modulus on 2D CCNTCM. In this section, we show that each majority circuit
can be implemented in a single module with $O(n)$ qubits. Any
reordering needed between the output qubits of one timestep in a majority circuit
and the input qubits of the next timestep is handled by teleportation
in between modules, as allowed on 2D CCNTCM. Therefore, it suffices for
us to do the following:

\begin{enumerate}
\item Show that a classical majority circuit exists for modular exponentiation
\item Show a 2D CCNTC quantum architecture for a single majority gate, inside its own module.
\end{enumerate}

We do the first item by relying on the following two theorems from Yeh and Varvarigos,
which we state again without proof. Both of these theorems
allow an extra parameter $\epsilon \in (0,1]$ which determines the
tradeoff between circuit depth and circuit size.

\begin{theorem}{Multiple product in constant depth and polynomial size: \cite{Yeh1996}}
The $n^2$-bit product of $n\times n$-bit numbers can be computed by a
circuit of depth $O(\frac{1}{\epsilon})$,
size and width $O(\frac{1}{\epsilon}n^{3+2\epsilon})$, and
fanin $O(n)$.
\end{theorem}

\begin{theorem}{Modular reduction in constant depth and polynomial size \cite{Yeh1996}}
The $n$-bit binary representation of the modular residue $x \bmod m$, where
$x$ is an $n^2$-bit number and $m$ is an $n$-bit modulus, can be computed
by a circuit of depth $O(\frac{1}{\epsilon})$,
size and width $O(\frac{1}{\epsilon}n^{1 + 2\epsilon})$, and
fanin $O(n^2)$.
\end{theorem}

Both of these theorems rely on a Chinese Remainder representation for
an $n$-bit number. A conventional binary representation of a number
treats bits as coefficients for weights of $O(2^n)$, which requires exponential
weight to represent in a threshold circuit. In a Chinese Remainder representation,
a number is given a ``mixed-radix'' representation, where each coefficient
is associated with a modular residue for a prime with $O(n)$ bits.
A more detailed reference of this technique can be found in \cite{Reif1992}.

We now accomplish the second item (a concrete architecture for a quantum majority gate)
in the remainder of this section by a sequence of building blocks, each on 2D
CCNTC with constant depths and polynomially-bounded sizes and widths.

\begin{itemize}

\item a $\text{BIAS}^{t,\phi}_n$ gate which distinguishes between $|x| = t$ and $|x| = (\lceil n/2 \rceil - t) \bmod n$ with a measurement bias of $e^{i\phi}$.
This is described in Section \ref{subsec:mu-gate}.
\item a $\text{EX}^t_{n\rightarrow \log_n}$ gate which reduces from $\text{EX}^t_n$ (on $n$ qubits)
to $\text{EX}^t_{\log n}$ (on $\lceil \log_2(n+1) \rceil$ qubits).
This is described in Section \ref{subsec:ex-reduce}.
\item a $\text{EX}^t_{\log_n}$ gate which acts on $O(\log n)$ qubits.
This is described in Section \ref{subsec:ex-log}.
\end{itemize}

%%%%%%%%%%%
\subsection{BIAS Gate}
\label{subsec:mu-gate}

We define the BIAS gate using the results in \cite{Hoyer2002}, where it is called $\mu^{|x|-t}_{\phi}$ gate.
It operates as follows:

\begin{equation}
\text{BIAS}^{t,\phi}_n\ket{x}\ket{+} \rightarrow \ket{x}\ket{\mu^{|x|-t}_{\phi}}
\end{equation}

The output qubit begins in the state $\ket{+}$, which has equal probability of
being measured in the $\ket{0}$ state or the $\ket{1}$ state. It ends in
the following state, which introduces a bias between measuring $\ket{0}$
or $\ket{1}$ proportional to the difference Hamming weight of the input
($|x|$) of a threshold $t$.

\begin{equation}
\ket{\mu^{|x|-t}_{\phi}} = \frac{1 + e^{i\phi(|x|-t)}}{2}\ket{0} + \frac{1 - e^{i\phi(|x|-t)}}{2}\ket{1}
\end{equation}

When $\phi = 2\pi / n$, then the BIAS gate allows us to distinguish perfectly
between the case of $|x| = t$ or $|x| = (\lceil n/2 \rceil - t) \bmod n$. As we will
see in the next section, multiples of $2\pi / n$ will allow us to reduce the
size of an $\text{EX}^t_n$ gate.

\begin{theorem}{Constant-depth BIAS gate}
The $BIAS^{t,\phi}_n$ gate can be implemented on 2D CCNTCM with
a depth of $O(1)$, a size and width of $O(n)$, and
expected $O(n)$ teleported PAR qubits of the form $(\ket{0} + e^{i\phi}\ket{1})$
and $(\ket{0} + e^{-i\phi\cdot t}\ket{1})$.
\label{thm:bias}
\end{theorem}

\begin{proof}
We can lay out the circuit from Figure 1 in \cite{Takahashi2011} on a 2D CCNTC lattice as
shown in Figure \ref{fig:mu-circuit}. The size includes a Hadamard to transform
the output qubit into $\ket{+}$, a fanout of this qubit over $n+1$ qubits,
$O(n)$ gates to apply the rotations $R_Z(\phi)$ and $O(1)$ gates to apply
the rotation $R_Z(-\phi\cdot t)$, and a corresponding unfanout, which is
$O(n)$ total. This occurs on $O(n)$ qubits, and can be arranged to take
$O(1)$ depth.
\end{proof}

% TODO: Fill this in
\begin{figure}
\caption{The circuit for a BIAS gate on 2D CCNTC.}
\label{fig:mu-circuit}
\end{figure}

%%%%%%%%%%%
\subsection{EX Reduction}
\label{subsec:ex-reduce}

Now we wish to show how to reduce the EXACT gate on $n$ bits to an EXACT
gate on $\log n$ qubits. That is, we wish to implement the following gate
$\text{EX}^t_{n\rightarrow \log_2 n}$ on an $n$-qubit input register $\ket{x}$
to produce an $m$-qubit output register $\ket{y}$, where
$m = \lceil \log_2 n + 1 \rceil$. Running $\text{EX}^t{n}$
on $\ket{x}$ should produce the same output qubit $\ket{z}$ as running
$\text{EX}^t_{m}$ on $\ket{y}$. This is formally defined below.

\begin{eqnarray}
\text{EX}^t_{n\rightarrow \log_2 n} \ket{x}\ket{0^m} & \rightarrow &\ket{x}\ket{y} \\
\text{EX}^t_n \ket{x}\ket{0} & \rightarrow & \ket{x}\ket{z} \\
\text{EX}^t_m \ket{y}\ket{0} & \rightarrow & \ket{y}\ket{z}
\end{eqnarray}

\begin{theorem}{Constant-depth EX reduction gate}
The $EX^t_{n\rightarrow \log_2 n}$ gate can be implemented on 2D CCNTCM with
a depth of $O(1)$, a size and width of $O(n\log n)$, and expected
$O(n\log n)$ teleported PAR qubits of the form $(\ket{0} + e^{i\phi_k}\ket{1})$
$(\ket{0} + e^{-i\phi_k\cdot t}\ket{1})$
\label{thm:ex-reduce}
\end{theorem}

\begin{proof}
We map the construction from Theorem 19 in \cite{Hoyer2002} onto
2D CCNTCM.
This step involves $m \le (n/2)+1$ parallel $\text{BIAS}^{t,\phi_k}_n$ gates from the last section for
$1 \le k \le m$, where $\phi_k = \frac{2\pi}{m}k$, each in their own modules.
This maintains constant circuit depth while 
\end{proof}

%%%%%%%%%%%
\subsection{EX on Logarithmic Qubits}
\label{subsec:or-exact}

We now present an exact EXACT gate.


%%%%%%%%%%%
\subsection{A Majority Gate in 2D CCNTCM in Sub-logarithmic Depth}

To map a majority gate to 2D CCNTCM, we use the $\textsf{AC}_f^0$ 
constructions from Takahashi-Tani \cite{Takahashi2011},
based on the results of
Hoyer-Spalek \cite{Hoyer2002}.
We combine this with our architectural techniques
from the previous chapter, including the 2D CCNTCM circuits
for constant-depth teleportation, fanout, and unfanout. 

We describe the construction for $\text{MAJ}_{n}$ below, given the
$n$-qubit input $x$.

\begin{enumerate}

\item
We compute in parallel the gates $\text{EX}^i_{n}$ for
$0 \le i \le \lceil n/2 \rceil$ to determine if the quantum
threshold for majority is reached. There are at most $(n/2) + 1$
such gates.

\begin{enumerate}
\item 
Compute the constant-depth reduction from $\text{EX}^t_{n}$ to
$\text{EX}^t_{m}$ where $m = \lceil \log_2(n+1) \rceil$, using
the reduction from $OR_n$ to $OR_{\log_n}$ \cite{Hoyer2002}.
For $1 \le k \le m$, do the following:

\begin{enumerate}
\item
Compute the qubit $\mu^{|x|-t}_{\phi_k}$, which is the rotation by Hamming 
weight of $x$, with a threshold $t$ subtracted, by the angle $\phi_k = 2\pi / 2^k$. Note that the precision
of this angle is $O(\log \log n)$. This can be done by a 2D CCNTCM circuit
with $O(1)$-depth, $O(n^2)$-size, and $O(n^2)$-width, as described in
Section \ref{subsec:or-reduce}.
\end{enumerate}

At the end of this step, we have $O(\log_2 n)$ bits $\ket{y_k}$ which 
represent the degree to which the Hamming weight of $x$ is close to the
threshold $t = \lceil n/2 \rceil$.

\item
Apply the circuit for exact $\text{OR}_{\log n}$ from Lemma 2 of
\cite{Takahashi2011} to the output of the previous step. This can
be done with a 2D CCNTCM circuit with $O(1)$-depth, $O(n \log n)$-size,
and $O()$ width.

\end{enumerate}

\item
Apply the gate $\text{PA}_{\lceil n/2 \rceil}$ to the result of
the previous step. This can be done by a 2D CCNTCM circuit of
$O(1)$-depth, $O(n)$-size, and $O(n)$-width using constant-depth
fanout, as described in Section \ref{sec:cdc}, and conjugated by
Hadamards on every qubit as described in \cite{Moore1998}.

\item
Apply a NOT to the output of the previous step. This final
output is the output of the quantum majority gate $MAJ_{n}$.

\end{enumerate}


