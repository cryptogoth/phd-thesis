\section{Conclusion}
\label{sec:fsl-conclude}

In this section, we have contributed a nearest-neighbor factoring architecture with
sub-logarithmic depth based on majority circuits. To do so, we've combined results from classical threshold
circuit complexity and our low-depth quantum architectural techniques from
Chapter $\ref{chap:factor-polylog}$.
We dsicuss the effect of quantum compiling single-qubit rotations to a fixed, finite basis.
To that end, we've given a concrete circuit for
a quantum majority gate on \textsf{2D CCNTCM} to fit into a classical, constant-depth majority circuit
for factoring. Table \ref{tab:sublog-resources} summarizes the final $\textsf{2D CCNTCM}$ resources for our sublogarithmic
factoring architecture, combining Theorems \ref{thm:mult-prod}, \ref{thm:mod-reduce}, and \ref{thm:maj-gate}.

\begin{table}[htb!]
\begin{tabular}{c|c|}
\hline
$\langle D \rangle$ & $O(\frac{1}{\epsilon}(\log\log n)^2)$ \\
\hline
$\langle S \rangle$ & $O(\frac{1}{\epsilon}n^{6 + 2\epsilon}\log^4 n\log\log n)$ \\
\hline
$\langle W \rangle$ & $O(\frac{1}{\epsilon}n^{6 + 2\epsilon}\log^2 n)$ \\
\hline
$\langle \overline{D} \rangle$ & $O(\frac{1}{\epsilon})$ \\
\hline
$\langle \overline{S} \rangle$ & $O(\frac{1}{\epsilon}n^{4+2\epsilon}\log n)$ \\
\hline
$\langle \overline{W} \rangle$ & $O(\frac{1}{\epsilon}n^{4+2\epsilon}\log n)$ \\
\hline
\end{tabular}
\caption{Expected circuit resources for a sublogarithmic factoring architecture with time-space tradeoff $\epsilon = (0,1]$.}
\label{tab:sublog-resources}
\end{table}

Now we conclude with some interesting open questions.
Although we are able to compile arbitrary rotations down to \textsf{CCNTC} in $O((\log \log n)^2)$-depth
for the resolution needed for factoring, can we reduce this to $O(1)$ depth if we relax our
requirements? For example, if we have a finite basis which is fault-tolerant, but which is not fixed;
it may vary based on the input size. quantum compiling makes truly constant-depth factoring architecture a challenging
open problem. Perhaps we will solve this in Chapter \ref{chap:qcompile}, when we discuss quantum compiling on
nearest-neighbor architectures.
Another open question is: is our factoring architectures now optimal?
What is the lower-bound for factoring on \textsf{CCNTC}, versus the $\Theta(1)$ bound for factoring on \textsf{CCAC}?
Can we determine the optimality of our factoring architectures using some other time-space product?
Perhaps we will this in Chapter \ref{chap:coherence}, when we discuss quantum circuit coherence. 

%We've also discussed an alternative
%approach to sub-logarithmic factor called iterated carry-save, which could
%potentially beat our majority circuit construction.


% In addition,
%Quantum compiling itself is a procedure which can be mapped to a quantum
%architecture, which is the topic of the next chapter.