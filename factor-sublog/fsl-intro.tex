\section{Improving the Depth Beyond Sublogarithmic}
\label{fsl-intro}

It is now natural to ask, given such dramatic improvement in
nearest-neighbor circuit
depth from quadratic \cite{Kutin2006} to polylogarithmic in the last chapter, can we
decrease depth further? Surprisingly, the answer is yes. In this
chapter we now decrease the depth below polylogarithmic, in fact,
to be $O((\log \log n)^2)$. To do this, we take inspiration from
classical circuit complexity to augment our universal quantum gate
set to get circuits of ``near-constant'' depth. At the end, we show how
to compile these augmented gate sets to the sublogarithmic depth above
on $2D CCNTCM$.

By augmenting a classical universal
gate set with a new primitive logic element, the \emph{threshold gate}, we can
get strictly more powerful circuit classes with constant depth. A brief
history of threshold circuit complexity is given in Section \ref{sec:fsl-threshold}.
However, these
definitions must be modified to allow reversible quantum gates and
superpositions of logic functions. Hoyer-Spalek showed that using
constant-depth fanout, we can implement the quantum threshold gate and
other such powerful multi-qubit functions.
The quantum threshold
gate can be decomposed into simpler operations, namely CNOT and
arbitrary single-qubit gates, while maintaining constant-depth.
These quantum modifications and results
are summarized in Section \ref{sec:fsl-circuits}.

A special case of the quantum threshold gate is the quantum majority
gate on $n$ qubits, which has unit weights, zero bias, and a
threshold of $\lceil n/2 \rceil$. This is a popular example from
classical threshold circuits, where the classical majority gate
is simpler to implement and is equivalent in computing power to
the general threshold gate, up to a constant factor in depth and
a polynomial increase in size. Using results for
arithmetic in constant-depth majority circuits due to
Reif-Tate \cite{Reif1992} and Yeh-Varvarigos \cite{Yeh1996},
we calculate the resources needed
for constant-depth quantum majority circuits for Shor's factoring
algorithm in Section \ref{sec:fsl-majority}.

In terms of physical realism,
we are still interested in implementing a quantum majority gate using
our $\textsf{2D CCNTCM}$ model, which due to the requirements of
quantum fault-tolerance must use a fixed, finite basis. This is where
we encounter the $O((\log \log n)^2)$ depth overhead. The resources
needed for quantum compiling the arbitrary single-qubit rotations above
are given in Section \ref{sec:fsl-qcompile}.