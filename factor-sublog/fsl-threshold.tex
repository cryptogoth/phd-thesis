\section{Previous Threshold Circuits and Related Work}

In order to understand techniques of decreasing depth for quantum 
circuits, we can leverage a well-studied area of classical circuit
complexity, namely that of threshold circuits.

There is a more general class of threshold gates than $\text{TH}_n^t$
defined earlier. These allow weights other than $+/-1$. In general, we
allow them to be any real number, although this doesn't allow them to
be 

Classical circuit complexity studies the resources needed
to implement a boolean function
$f:\{0,1\}^n \rightarrow {0,1}$. The resources are usually the
depth and size of the circuit in gates taken from some universal
set, which may be of fixed size or vary with the size of the input.
By adding gates, restricting their properties, or setting
constraints on circuit depth and size, we can define what boolean
functions are computable using the resulting circuit complexity
class.

A \emph{linear threshold element} (LTE) is a logic gate whose output is the
sign of the weighted sum of $n$ input bits $x = \{x_1, x_2, \ldots, x_n\}$,
where initially we take
the weights $w_i$ to be any real number. There is a weight that is not
associated with any input bit which is called the \emph{bias} of the
LTE. It is the most elementary logic element in threshold circuits.
Analogous to a simplified model of neurons in the human brain, an LTE can
accept many inputs through its dendrites, which are the outputs of other
neurons or sensory organs, and compute a function based on ``learned'' 
weights. If the weighted sum passes some threshold (the bias), the neuron
``fires,'' or sends signal down its axon, which then becomes the input to
another neuron or ends in an actuator, like a muscle.

\begin{equation}
f(x) = \sgn\left( w_0 \sum_{i=1}^n w_i x_i \right)
\end{equation}

The class of boolean functions implementable with a single layer
of LTE's is called
$\textsf{LT}_1$. In fact, a circuit of LTEs of depth $1$ consists only
of $1$ LTE gate. We are primarily interested in constant depth threshold
circuits, which we denote $\textsf{LT}_k$ for depths $k>0$.

An alternative way of expressing boolean functions is using a polynomial
threshold element (PTE), which is a weighted sum over \emph{multilinear terms}
in the input variables. In this representation, it is useful to
map the bit values $\{0,1\}$ to $\{+1,-1\}$.

\begin{equation}
f(x) = \sgn\left( w_0 \sum_{a \in \{0,1\}^n} w_a X_a \right) \qquad X_a \equiv \prod_{i=1}^n x_i^{a_i}
\end{equation}

We call the class of functions implementable by $k$ layers of PTE's
$\textsf{PT}_k$, where $\textsf{PT}_1$ indicates the class of functions
implementable by a single PTE.
It is easy to see that any boolean function can be implemented using
a sum-of-products, with one multilinear term for each of the $2^n$
rows in the function's truth table. The surprising relationship between
these two primitive gates 

As a practical consideration, we would like to restrict the weights
to be rational numbers, and moreover, those that are bounded
in magnitude by a polynomial in the number of inputs. 
We can restrict the weights to be rational numbers without
greatly diminishing their power, as well as limit  \cite{}
\begin{enumerate}

\item
consider from exponential number of terms to polynomial number
of terms. This definitely decreases power, but has a practical
motivation in that we will quickly run out of space.

\item
change weights from real numbers to rational numbers. in fact,
integers with $n \log n$ bits can be used. (doesn't change power) need citation. motivation is because we often have limited precision is
implementing gates physically.

\item
change exponential weights to polynomially bounded weights (can simulate in constant-depth, polynomial increase in size), need
citation.

\item
change to unit weights (majority gates). can simulate in
constant depth polynomial increase in size) need citation.
We can allow weights of $+1$ and $-1$ simply by putting a
NOT gate in front of an input we want.

\end{enumerate}

There are (non-constructive) existence proofs of many interesting 
arithmetic functions that can be implemented by threshold circuits
of constant depth (called \emph{bounded depth} in the literature),
often of very small sizes (5 or less). These exist in the class
$\textsf{TC}^0$, and it is especially surprising because they are known
not to be in the class $\textsf{AC}^0$, which proves the proper
containment $\textsf{TC}^0 \subsetneq \textsf{AC}^0$. That is, in
the classical setting, threshold gates appear to contain some
special power that cannot be captured by AND and OR, gates, even those
with unbounded fanin, in contant depth.

In \cite{Reif1992}, it is shown that threshold circuits
are equivalent (up to constant depth) to another class of
circuits known as finite field ($Z_{P(n)}$) circuits, where each node
computes either multiple sums or products of integers modulo a
prime $P(n)$. However, in both of these cases, the fanin

An important technique in any operation involving multiplication is
the conversion between a binary representation, where numbers are
treated as a linear combination from the basis of elements of
size $O(2^n)$, to a so-called ``Chinese Remainder'' representation,
where numbers are treated as a linear combination of residues
modulo primes of size $O(n)$. This latter representation can be thought
of as a ``mixed-radix'' representation, and is responsible for
the depth-reduction of multiplication related operations in
threshold circuits.