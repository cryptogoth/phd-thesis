\section{Controlled Rotations for Factoring}
\label{sec:fsl-qcompile}

In Section \ref{sec:fpl-related} of Chapter \ref{chap:factor-polylog},
we stated that we wished to avoid
factoring implementations that used a QFT due to the fine
single-qubit rotations involved. Due to the requirements of
fault-tolerance on a particular physical implementation,
we can usually implement only a set of gates that is fixed
(it does not change with the problem input size), discrete (of finite size),
and universal. This last property is necessary for us to approximate any
other gate \emph{not} in our set, especially single-qubit phase rotations
of the form $\Lambda(e^{i \phi})$. Such an approximation would involve
a quantum compiling procedure, such as Solovay-Kitaev, which is the
subject of Chapter \ref{chap:qcompile}. However, we mention it here
because the choice of our universal set determines the true depth
of any circuit.

%%%%%%%%%%%%%%%%%%%%%%%%%%%%%%%%%%%%%%%%%%%%%%%%%%%%%%%%%%%%%%%%%%%%%%%%%%%
\subsection{Controlled-$R_Z$ Rotations}

In our poly-logarithmic factoring implementation, we were able to reduce
all our arithmetic circuits to such a fixed, discrete universal set.
These arithmetic circuits are discrete and classical and nature, so it is
not surprising that we can implement them in a discrete way.
However, to reduce the depth further, we need to introduce the idea of
a quantum threshold gate, which requires controlled-rotations similar to
the QFT. This controlled-rotation gate is shown in Figure \ref{fig:crz}
along with its decomposition into CNOTs and three single-qubit $R_Z(\phi)$
rotations. An alternative decomposition using only one single-qubit
$R_Z(\phi)$ and controlled-SWAPs is given in Figure \ref{fig:crz2} which
is often preferred, but asymptotically has the same resources as
Figure \ref{fig:crz}.
These rotations depend on the input size to a particular gate called
BIAS which we introduce in Theorem \ref{thm:bias}.

If we were given pre-generated ancillae of the form
$\normtwo (\ket{0} + e^{i\phi}\ket{1})$,
we could use the method of programmable ancillae rotation (PAR) from
\cite{Jones2012}
to apply the gate $R_Z(\phi)$ probabilistically
using the circuit from Figure \ref{fig:par}.
We will call ancilla qubits in this form \emph{PAR qubits} from now on.
Note that we have a $50\%$
chance
of applying the opposite rotation $-\phi$, and so we must chain
several rounds of the given circuit. In expectation, we achieve the
desired rotation in two rounds, which has constant depth.
Consequently, we must now consider
all circuit resources in terms of expected (average) values, which we will
write in between angle brackets such as $\rangle D \langle$ for expected
depth, rather than just $D$.

The 
probability of $k$ consecutive PAR rounds all failing (and the need to run a
more expensive KSV quantum compiling procedure) is $2^{-k}$, giving
the following expected depth. For $k = O(1)$, one can make the
expected depth of both PAR and quantum compiling grow with an arbitrarily small multiplicative constant.
However the asymptotic cost will still track that of whatever
quantum compiling procedure is used. For $k = \omega(1)$, we make
the expected depth of quantum compiling constant; unfortunately, now
the expected depth of PAR rounds is no longer constant. This is an
inherent tradeoff in using PAR.
An analogous calculation can be done
for the circuit size.

\begin{equation}
\left( \sum_{m=1}^k \frac{1}{2^m} O(1) \right) + 2^{-k}O((\log\log n)^2)
\end{equation}

\begin{figure}[tbp!]
\begin{displaymath}
\begin{array}{ccc}
\Qcircuit @C=1.5em @R=1.5em {
\lstick{\ket{\psi}}                  & \qw & \targfix  & \qw & \gate{X^j} & \qw & \rstick{R_Z((-1)^j \phi)\ket{\psi}} \\
 %\normtwo \left( \ket{0} + e^{i\phi}\ket{1} \right) & \qw & \ctrl{-1} & \qw & \measureD  & \cw & \rstick{j}
\lstick{ \normtwo \left( \ket{0} + e^{i\phi}\ket{1} \right) } & \qw & \ctrl{-1} & \qw & \measureD{Z}  & \cw & \rstick{j}
 }
 &
\begin{array}{c}
\\
\\
\\
= \\
\end{array}
&
\Qcircuit @C=1.5em @R=1.5em {
   & \qw & \gate{\frac{\pi}{2^{d}}} & \qw \\
   & \qw & \qw                      & \qw \\
 }
\end{array}
 \end{displaymath}
\caption{One round of programmable ancillae rotation (PAR) to probabilistically achieve arbitrary single-qubit rotations \cite{Jones2012}.}
\label{fig:par}
\end{figure}

\begin{figure}[tb!]
\begin{center}
\begin{displaymath}
\begin{array}{ccc}
\Qcircuit @C=1.5em @R=1.5em {
   & \qw      & \ctrl{1}                   & \qw \\
   & \qw      & \gate{\frac{\pi}{2^{d}}} & \qw \\
 }
&
\begin{array}{c}
\\
\\
\\
= \\
\end{array}
&
\Qcircuit @C=1.5em @R=1.5em {
& \qw & \qw & \qw & \ctrl{1} & \qw & \gate{\frac{\pi}{2^{d+1}}} & \qw & \ctrl{1} & \qw\\
 & \qw & \gate{\frac{\pi}{2^{d+1}}} & \qw & \targfix & \qw & \gate{\frac{\pi}{2^{d+1}}} & \qw & \targfix & \qw
}
\end{array}
\end{displaymath}
\caption{Decomposition of a controlled-$Z$ ($\Lambda(R_Z(\phi))$) rotation using three single-qubit rotations and CNOTs.}
\label{fig:crz}
\end{center}\end{figure}

\begin{figure}[tb!]
\begin{center}
\begin{displaymath}
\begin{array}{ccc}
\Qcircuit @C=1.5em @R=1.5em {
   & \ctrl{1}                 & \qw \\
   & \gate{\frac{\pi}{2^{d}}} & \qw \\
   & \qw                      & \qw \\
 }
&
\begin{array}{c}
\\
\\
\\
= \\
\end{array}
&
\Qcircuit @C=0.75em @R=1.5em {
& \qw & \ctrl{1} & \qw & \qw                        & \qw & \ctrl{1} & \qw\\
& \qw & \qswap   & \qw & \qw                        & \qw & \qswap   & \qw\\
& \qw & \qswap \qwx     & \qw & \gate{\frac{\pi}{2^{d}}} & \qw & \qswap \qwx     & \qw
}
\end{array}
\end{displaymath}
\caption{Decomposition of a controlled-$Z$ ($\Lambda(R_Z(\phi))$) rotation using one single-qubit rotation controlled-SWAPs, and an ancilla.}
\label{fig:crz2}
\end{center}\end{figure}

%%%%%%%%%%%%%%%%%%%%%%%%%%%%%%%%%%%%%%%%%%%%%%%%%%%%%%%%%%%%%%%%%%%%%%%%%%%
\subsection{Quantum Compiler Modules}

We can augment any \textsf{2D CCNTCM} circuit $\mathcal{C}$ with a set of modules that run a
separate quantum compiler procedure, which is a source of PAR qubits.
The angles $\phi$ are determined by
the original $\mathcal{C}$ and are known in advance (classically-precomputed,
or non-adaptive). In the case of majority gates for factoring,
as described in Section \ref{sec:fsl-majority}, the angles are of the
form $\phi_k = \frac{2\pi}{2^k}$ where $k \le (\log_2 n' + 2)$ for a majority
gate with fanin $n'$. Since this represents the most fine-grained resolution of
rotation, our allowable error is $2^{-(k+1)} = 2^{-(\log_2 n' + 3)}$.

This quantum compiler resource has a parameter $\epsilon$ which is the
allowable error in the angles of the returned PAR qubits.
We assume they have some precision $\epsilon = 2^{-k}$ from the
true desired angle $\tilde{\phi}$.
%
\begin{equation}
| \phi - \tilde{\phi} | \le \epsilon = 2^{-(k+1)} = 2^{-(\log_2 n' + 3)}
\end{equation}.
%
Often, quantum compiler resources are given in terms of the quantity
$(1/\epsilon)$, which we can re-express in terms of the
majority gate fanin as $O(1/n')$. 
While we will delay a discussion and comparison of
quantum compilers until Chapter \ref{chap:qcompile}, for now
we assume an upper bound of the Kitaev-Shen-Vyalyi (KSV) algorithm
using parallel phase estimation from Section 13 of \cite{Kitaev2002}.
This has the resources given in Table \ref{tab:fsl-ksv}.

\begin{table}[tbp!]
\begin{center}
\begin{tabular}{|c|c|c|}
\hline
Circuit Size & $O((\log n')^2\log\log n')$ \\
\hline
Circuit Depth & $O((\log\log n')^2)$\\
\hline
\end{tabular}
\caption{KSV quantum compiling resources for single-qubit rotations in majority gates of fanin $n'$.}
\label{tab:fsl-ksv}
\end{center}
\end{table}

As we will discover later, the fanin of each majority gate is
linear in the size of the modulus to factor: $n' = O(n)$.
Therefore, if we are not allowed PAR qubits for free in our model,
our construction will actually have minimum depth $O((\log\log n)^2)$,
which is not constant, but is still sub-logarithmic. This relationship still
holds if the fanin is merely $n' = poly(n)$. It is still an
open question \cite{Hoyer2002} whether a constant-depth
quantum majority gate exists even on \textsf{AC} with a fixed finite basis,
let alone \textsf{NTC} models such as \textsf{CCNTC} or \textsf{CCNTCM}.

From the argument in the previous section: even though the PAR procedure is
probabilistic, the asymptotic circuit resources required can be
deterministically upper-bounded.