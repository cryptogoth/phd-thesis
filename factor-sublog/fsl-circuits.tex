\section{Circuit Complexity Classes}
\label{sec:fsl-circuits}

A circuit can be thought of as a directed acyclic graph in which the nodes are
logical gates drawn from a certain (universal) set and the edges 
represent
the connection of the output of one gate to the input of another
gate. This graph is not equivalent to, but is related to, the graph of an architecture
as described in Chapter \ref{chap:factor-polylog}. The edges of a circuit graph can
be mapped to the nodes (qubits) of an architectural graph; the nodes of the
circuit graph, for 2D CCNTC, can be mapped to single nodes or connected pairs of nodes
in the architectural graph.

\begin{figure}
\caption{An example of a classical circuit implementing a Boolean function.}
\end{figure}

We can also define special nodes which are not gates, but rather
are placeholder ``sources'' which provide the inputs to the circuit and 
``sinks'' which provide the outputs to the circuit. The in-degree of a 
node is also known as its \emph{fanin} and the out-degree of a node is
also known as its \emph{fanout}.
Classical circuits implement Boolean functions, which take in $n$ input
bits to one output bit.

\begin{equation}
f:{0,1}^n \rightarrow {0,1}
\end{equation}

We denote a gate by its fanin as a subscript and an optional
second parameter as a superscript.

\begin{equation}
\text{GATE}_n^k
\end{equation}

The fanin $n$ will be neglected where it is obvious,
such as for the following well-known gate set which is universal
for classical circuits:

\begin{itemize}
\item $\text{NOT} = \text{NOT}_1$
\item $\text{AND} = \text{AND}_2$
\item $\text{OR} = \text{OR}_2$
\end{itemize}

\subsection{Gates Based on the Hamming Weight}

While the gates above have a simple truth table, it is
useful to describe a wider class of gates with general
fanin $n$ in a compact way as a function on the input
Hamming weight. These include the gates named below, which
are $1$ when the condition to their right is met, and $0$
otherwise.

\begin{itemize}
\item The logical OR gate $\text{OR}_n: |x| > 0$
\item The logical AND gate $\text{AND}_n: |x| = n$
\item The modular gate $\text{MOD}[q]_n: |x| \bmod q = 0$
\item The parity gate $\text{PA}_n: |x| \bmod 2 = 0$
\item The exact gate $\text{EX}^t_n: |x| = t$
\item The threshold gate $\text{TH}^t_n: |x| \ge t$
\item The majority $\text{MAJ}_n: |x| \ge n/2$
\end{itemize}

Many of these gates are also related in interesting ways,
as noted in \cite{Takahashi2011}.
$\text{OR}_n$ and $\text{EX}^0_n$ are
negations of each other. $\text{PA}_n$ is equivalent to
$\text{MOD}[2]_n$. $\text{TH}^t_n$ can be implemented with
$n-t$ parallel copies of $\text{EX}^k_n$ for $t \ge k \ge n$
followed by $\text{PA}_{n-t}$ on the outputs.

\subsection{Classical Circuit Complexity Classes}

We have introduced a menagerie of interesting gates,
all of which are universal with the $NOT$ gate. However depending
on what gates are in our universal set, our circuits may have
different depth and size. Therefore, we can define complexity classes
of circuits based on the allowed gate set and study more general
relationships among these classes beyond the special cases mentioned
as the end of the last section. This will let us formalize the notion of
which gates are more powerful than other gates and which are equivalent.
 
In classical circuits, we take unbounded fanout
for granted (any node can have arbitrary out-degree). These are common
in the literature of classical circuits. We will list them in order
of the size of their universal set, where each subsequent class adds
more gates.

\begin{definition}
\item[\textsf{NC}]
circuits consisting of $\text{NOT}_1$ and $\text{AND}_2$ and
$\text{OR}_2$ gates.
\item[\textsf{AC}]
NC circuits augmented with $\text{AND}_n$ and $\text{OR}_n$ gates,
for $n \ge 2$.
\item[\textsf{TC}]
AC circuits augmented with $\text{TH}_n^t$ gates, for $n \ge 2$ and
$0 \le t \le n$.
\end{definition}

A $\text{TH}^t_n$ gate is a threshold gate that is $1$ if the number
of input bits in greater than or equal to the threshold $t$ and $0$
otherwise. This is a special case of the linear threshold element (LTE)
in Section \ref{sec:threshold}, but we can simulate any
LTE with a circuit of $\text{TH}^t_n$ gates in constant depth
and polynomial size (citation needed).

\begin{equation}
% TODO insert block piecewise bracket
\end{equation}

We are often interested in the computing power of the above
circuit classes restricted in some way, usually shallow depth.
We denote by a superscript $k$ a complexity class of
functions implementable by circuits of depth $k$.

For these classical circuit classes, it is known that containment
is proper.

\begin{equation}
\textsf{NC}^0 \subsetneq \textsf{AC}^0 \subsetneq \textsf{TC}^0
\end{equation}

\subsection{Quantum Gates}

Unbounded fanout is taken for granted in classical circuit classes
because it is physically realistic to implement using electrical
devices. However, for quantum circuits, we must make use of the
unbounded quantum fanout, $\text{FANOUT}_n$, to copy one output
qubit of each gate.
We can define quantum analogues of the above circuit complexity 
classes by defining quantum analogues for each of the gates which
is reversible.
$NOT_1$ is already reversible, and we can use it as is (the Pauli $X$ gate).
To replace $AND_2$ we can use the reversible $3$-qubit Toffoli gate,
the so-called controlled-controlled-NOT.
To replace $OR_2$ we can use the circuit given in
Figure \ref{fig:or2}

\begin{figure}
% TODO 
\caption{A quantum $\text{OR}_2$ gate (citation needed).}
\label{fig:or2}
\end{figure}

In fact this is special case of a much more powerful construction
that will let us define a quantum $OR_n$ gate on unbounded inputs.
We will use this construction in Section \ref{sec:fsl-majority}
to build a quantum majority gate.

\begin{definition}
\item[$\textsf{QNC}^0_f$]
constant-depth quantum circuits consisting of CNOT and single-qubit gates.
\item[$\textsf{QAC}^0_f$]
constant-depth $\textsf{QNC}^0_f$ circuits augmented with quantum $\text{AND}_n$ and $\text{OR}_n$ gates,
for $n \ge 2$.
\item[$\textsf{QTC}^0_f$]
constant-depth $\textsf{QAC}^0_f$ circuits augmented with $\text{TH}_n^t$ gates, for $n \ge 2$ and
$0 \le t \le n$.
\end{definition}

\subsection{}