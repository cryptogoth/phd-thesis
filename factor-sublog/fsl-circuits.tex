\section{Circuit Complexity Background}
\label{sec:fsl-circuits}

A circuit can be thought of as a directed acyclic graph in which the nodes are
logical gates drawn from a certain (universal) set and the edges 
represent
the connection of the output of one gate to the input of another
gate. This graph is not equivalent to, but is related to, the graph of an architecture
as described in Chapter \ref{chap:factor-polylog}. The edges of a circuit graph can
be mapped to the nodes (qubits) of an architectural graph; the nodes of the
circuit graph, for 2D CCNTC, can be mapped to single nodes or connected pairs of nodes
in the architectural graph. The notion of a gate and a circuit are not mutually
exclusive: both implement a boolean function, but a gate is usually treated as
a primitive, subcircuit element for the purpose of counting resources in a larger circuit.

%\begin{figure}
%\caption{An example of a classical circuit implementing a Boolean function.}
%\end{figure}

We can also define special nodes which are not gates, but rather
are placeholder ``sources'' which provide the inputs to the circuit (they only
have out-degree) and 
``sinks'' which consume the outputs to the circuit (they only have in-degree).
The in-degree of a 
node is also known as its \emph{fanin} and the out-degree of a node is
also known as its \emph{fanout}.
Classical circuits implement boolean functions, which take in $n$ input
bits to one output bit.
%
\begin{equation}
f:\{0,1\}^n \rightarrow \{0,1\}
\end{equation}
%
We denote a gate by its fanin as a subscript and other optional
parameters as superscripts ($n$ and $k$, respectively, in the equation below).
%
\begin{equation}
\text{GATE}_n^k
\end{equation}
%
The fanin $n$ will be neglected where it is obvious,
such as for the following well-known gate set which is universal
for classical circuits: $\text{NOT} = \text{NOT}_1$, 
$\text{AND} = \text{AND}_2$, $\text{OR} = \text{OR}_2$.

%%%%%%%%%%%%%%%%%%%%%%%%%%%%%%%%%%%%%%%%%%%%%%%%%%%%%%%%%%%%%%%%%%%%%%%%%%%%
\subsection{Gates Based on the Hamming Weight}

While the gates above have a simple truth table, it is
useful to describe a wider class of gates with general
fanin $n$ in a compact way as a function on the input
Hamming weight. These include the gates named below, which
are $1$ when the condition to their right is met, and $0$
otherwise.
%
\begin{itemize}
\item The logical OR gate $\text{OR}_n: |x| > 0$
\item The logical AND gate $\text{AND}_n: |x| = n$
\item The modulo gate $\text{MOD}[q]_n: |x| \bmod q = 0$
\item The parity gate $\text{PA}_n: |x| \bmod 2 = 0$
\item The exact gate $\text{EX}^t_n: |x| = t$
\item The threshold gate $\text{TH}^t_n: |x| \ge t$
\item The majority gate $\text{MAJ}_n: |x| \ge n/2$
\end{itemize}
%
Many of these gates are also related in interesting ways,
as noted in \cite{Takahashi2011}.
$\text{OR}_n$ and $\text{EX}^0_n$ are
negations of each other. $\text{PA}_n$ is equivalent to
$\text{MOD}[2]_n$. $\text{TH}^t_n$ can be implemented with
$n-t$ parallel copies of $\text{EX}^k_n$ for $t \ge k \ge n$
followed by $\text{PA}_{n-t}$ (or $\text{OR}_{n-t}$) on the outputs.

%%%%%%%%%%%%%%%%%%%%%%%%%%%%%%%%%%%%%%%%%%%%%%%%%%%%%%%%%%%%%%%%%%%%%%%%%%%%
\subsection{Classical Circuit Complexity Classes}

We have introduced a menagerie of interesting gates,
all of which are universal with the $NOT$ gate. However depending
on what gates are in our universal set, our circuits may have
different depth and size. Therefore, we define complexity classes
of circuits based on the allowed gate set and study more general
relationships among these classes. As is usual in the literature,
we will interchangeably use ``circuits'' to mean uniform
circuit families parameterized by their input size $n$.
 This will let us formalize the notion of
which gates are more powerful than other gates and which are equivalent.
 
In classical circuits, we take unbounded fanout
for granted (any node can have arbitrary out-degree). These are common
in the literature of classical circuits. We will list them in order
of the size of their universal set, where each subsequent class adds
more gates. Note that each class must include polynomial-sized circuits
in order to be complete.
%
\begin{description}
\item[\textsf{NC}:]
circuits consisting of $\text{NOT}_1$ and $\text{AND}_2$ and
$\text{OR}_2$ gates.
\item[\textsf{AC}:]
\textsf{NC} circuits augmented with $\text{AND}_n$ and $\text{OR}_n$ gates,
for $n \ge 2$.
\item[\textsf{AC[q]}:] \textsf{AC} circuits augmented with $\text{MOD}[q]_n$ gates.
\item[\textsf{ACC}:] the union of $\textsf{AC}[q]$ for all positive integers $q > 2$.
\item[\textsf{TC}:]
\textsf{AC} circuits augmented with $\text{TH}_n^t$ gates, for $n \ge 2$ and
$0 \le t \le n$.
\end{description}
%
We are often interested in the computing power of the above
circuit classes restricted in some way, usually shallow depth.
We denote by a superscript $k$ a complexity class of
functions implementable by circuits of depth bounded by $O(\log^k n)$.

For these classical circuit classes, it is known that containment
is proper between $\textsf{NC}^0$, $\textsf{AC}^0$, and $\textsf{TC}^0$.
%
\begin{equation}
\textsf{NC}^0 \subsetneq \textsf{AC}^0 \subsetneq \textsf{TC}^0
\end{equation}
%
For example, the function $\text{AND}_n$ and $\text{OR}_n$ is known
\emph{not} to be in $\textsf{NC}^0$ but is in $\textsf{AC}^0$ by
definition. Likewise, the gate $\text{PA}_n$ is
known \emph{not} to be in $\text{AC}^0$ but is in 
$\text{TC}^0$ \cite{Bruck1990}.

It is also known that $\textsf{AC}^0[q] \ne \textsf{AC}^0[p]$
for $p$ and $q$ being powers of distinct primes \cite{Smolensky1987}.

%%%%%%%%%%%%%%%%%%%%%%%%%%%%%%%%%%%%%%%%%%%%%%%%%%%%%%%%%%%%%%%%%%%%%%%%%%%%
\subsection{Linear Threshold Elements}

For historical reasons, we will mention here a more general
version of $\text{TH}^t_n$ which was studied extensively in
the 1990s. This threshold gate, 
called a \emph{linear threshold element} (LTE), is a boolean
function defined as
the sign of a weighted sum of its input bits. There is a weight
not associated with any input, called a bias,
denoted as $w_0$ in the equation below.
%
\begin{equation}
f:\{0,1\}^n \rightarrow \{0,1\} = \sgn\left( w_0 + \sum_{i=1}^n w_i x_i \right)
\end{equation}
%
The LTE was biologically inspired by neurons in the human brain and
subsequently the neural network model of computation. Computational
neurons are simple elements which can be combined in highly parallel,
shallow depth (less than $6$) networks to compute seemingly complicated
arithmetic functions such as (multiple) addition, (multiple) multiplication,
division, and comparison. These results are summarized in Siu et al. \cite{Siu1993}
based on techniques by Beame et al. \cite{Beame1986}. LTE's are arguably the
predecessors of logic elements (LE's) in modern day field-programmable
gate-arrays (FPGA's).

However, the most general kind of LTE can have real-valued weights,
which can be difficult to implement fault-tolerantly on digital logic.
A single LTE's power can be changed dramatically by restricting
its weights, but this power goes away when we consider \emph{threshold circuits}
of LTEs restricted to constant depth.
Through a succession of results, it was shown that restricting the weights
to be rational numbers bounded by a polynomial in $n$ \cite{Siu1991a}
did not decrease the power of such constant-depth threshold circuits.
That is, any threshold circuit with unrestricted-weight LTE's could be simulated
with a constant-depth overhead with restricted-weight LTE's.

Remarkably, this constant-depth simulability applies even when restricting
LTE's to unit weight (weights of only $+1$) and integer bias $0 \le t \le n$,
which is equivalent to the definition of $\text{TH}_n$ given above. It
even applies when we restrict the threshold to be $\lceil n/2 \rceil$, which
is the definition of $\text{MAJ}_n$ given above \cite{Goldmann1994}.
In fact, where many arithmetic functions above have been shown to
be implementable in majority circuits \cite{Reif1992,Yeh1996}.
Therefore, without loss of generality, we can concentrate on
majority circuits for the remainder of this chapter.

%%%%%%%%%%%%%%%%%%%%%%%%%%%%%%%%%%%%%%%%%%%%%%%%%%%%%%%%%%%%%%%%%%%%%%%%%%%%
\subsection{Quantum Gates}

We can define analogous quantum circuit complexity classes by
considering circuits over (reversible) quantum gates. From
now on, when we refer to a gate $\text{GATE}^k_n$, we mean
its quantum version which behaves as follows on an $n$-qubit
``input'' register $\ket{x}$ and a single-qubit ``output''
register $\ket{y}$. We denote the single-bit output of the
classical function $\text{GATE}^k_n(x)$ as $g(x)$.
%
\begin{equation}
\text{GATE}^k_n\ket{x}\ket{y} \rightarrow \ket{x}\ket{y \oplus g(x)}
\end{equation}
%
Unbounded fanout is taken for granted in classical circuit classes
because it is physically realistic to implement using electrical
devices. However, for quantum circuits, we must make use of
unbounded quantum fanout (described in detail in Chapter \ref{chap:factor-polylog})
to create entangled copies of the output qubit for every subsequent
gate which consumes it as an input. We can even define the
gate $\text{FANOUT}_n$ on the source qubit $x$ onto target qubits
$y_1, \ldots, y_n$.
%
\begin{equation}
\text{FANOUT}_n\ket{x}\ket{y_1}\ldots\ket{y_n} \rightarrow \ket{x}\ket{y_1 \oplus x}\ldots\ket{y_n \oplus x}
\end{equation}
%
There is not a clear, unique quantum version of the classical
circuit classes above. However, the following quantum circuit
complexity classes, taken from \cite{Hoyer2002}
begin with the simplest universal set and augment it with
quantum versions of the corresponding classical gates. Each
class has a subscript $f$ to indicate the inclusion of
$\text{FANOUT}_n$.
%
\begin{description}
\item[$\textsf{QNC}^0_f$:]
constant-depth quantum circuits consisting of CNOT and single-qubit gates.
\item[$\textsf{QAC}^0_f$:]
constant-depth $\textsf{QNC}^0_f$ circuits augmented with quantum $\text{AND}_n$ and $\text{OR}_n$ gates,
for $n \ge 2$.
\item[$\textsf{QAC[q]}^0_f$:]
constant-depth $\textsf{QAC}^0_f$ circuits augmented with quantum $\text{MOD}[q]_n$ gates,
for $n \ge 2$.
\item[$\textsf{QACC}^0_f$:] the union of all $\textsf{QAC[q]}^0_f$ for positive integers $q$.
\item[$\textsf{QTC}^0_f$:]
constant-depth $\textsf{QAC}^0_f$ circuits augmented with $\text{TH}_n^t$ gates, for $n \ge 2$ and
$0 \le t \le n$.
\end{description}
%
It is an important result of Takahashi-Tani \cite{Takahashi2011} that the presence of
$\text{FANOUT}_n$ equalizes the three classes below.
%
\begin{eqnarray}
\textsf{QNC}^0 \subsetneq \textsf{QAC}^0 \subsetneq \textsf{QTC}^0\\
\textsf{QNC}^0_f = \textsf{QAC}^0_f = \textsf{QTC}^0_f
\end{eqnarray}
%
Moore shows that parity is equivalent to fanout in a quantum setting \cite{Moore1999}.
%
\begin{equation}
\textsf{QAC}^0_f = \textsf{QAC}[2]^0_f
\end{equation}
%
Furthermore, it is has been discovered by Green et al. \cite{Green2001} that including any $\text{MOD}[q]^0_f$
leads to the same complexity class in constant depth.

\begin{equation}
\textsf{QACC}^0_f = \textsf{QAC}[q]^0_f \qquad \forall q \ge 2
\end{equation}