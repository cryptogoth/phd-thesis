\documentclass{article}

%    Q-circuit version 1.06
%    Copyright (C) 2004  Steve Flammia & Bryan Eastin

%    This program is free software; you can redistribute it and/or modify
%    it under the terms of the GNU General Public License as published by
%    the Free Software Foundation; either version 2 of the License, or
%    (at your option) any later version.
%
%    This program is distributed in the hope that it will be useful,
%    but WITHOUT ANY WARRANTY; without even the implied warranty of
%    MERCHANTABILITY or FITNESS FOR A PARTICULAR PURPOSE.  See the
%    GNU General Public License for more details.
%
%    You should have received a copy of the GNU General Public License
%    along with this program; if not, write to the Free Software
%    Foundation, Inc., 59 Temple Place, Suite 330, Boston, MA  02111-1307  USA

\usepackage[matrix,frame,arrow]{xy}
\usepackage{amsmath}
\newcommand{\bra}[1]{\left\langle{#1}\right\vert}
\newcommand{\ket}[1]{\left\vert{#1}\right\rangle}
    % Defines Dirac notation.
\newcommand{\qw}[1][-1]{\ar @{-} [0,#1]}
    % Defines a wire that connects horizontally.  By default it connects to the object on the left of the current object.
    % WARNING: Wire commands must appear after the gate in any given entry.
\newcommand{\qwx}[1][-1]{\ar @{-} [#1,0]}
    % Defines a wire that connects vertically.  By default it connects to the object above the current object.
    % WARNING: Wire commands must appear after the gate in any given entry.
\newcommand{\cw}[1][-1]{\ar @{=} [0,#1]}
    % Defines a classical wire that connects horizontally.  By default it connects to the object on the left of the current object.
    % WARNING: Wire commands must appear after the gate in any given entry.
\newcommand{\cwx}[1][-1]{\ar @{=} [#1,0]}
    % Defines a classical wire that connects vertically.  By default it connects to the object above the current object.
    % WARNING: Wire commands must appear after the gate in any given entry.
\newcommand{\gate}[1]{*{\xy *+<.6em>{#1};p\save+LU;+RU **\dir{-}\restore\save+RU;+RD **\dir{-}\restore\save+RD;+LD **\dir{-}\restore\POS+LD;+LU **\dir{-}\endxy} \qw}
    % Boxes the argument, making a gate.
\newcommand{\meter}{\gate{\xy *!<0em,1.1em>h\cir<1.1em>{ur_dr},!U-<0em,.4em>;p+<.5em,.9em> **h\dir{-} \POS <-.6em,.4em> *{},<.6em,-.4em> *{} \endxy}}
    % Inserts a measurement meter.
\newcommand{\measure}[1]{*+[F-:<.9em>]{#1} \qw}
    % Inserts a measurement bubble with user defined text.
\newcommand{\measuretab}[1]{*{\xy *+<.6em>{#1};p\save+LU;+RU **\dir{-}\restore\save+RU;+RD **\dir{-}\restore\save+RD;+LD **\dir{-}\restore\save+LD;+LC-<.5em,0em> **\dir{-} \restore\POS+LU;+LC-<.5em,0em> **\dir{-} \endxy} \qw}
    % Inserts a measurement tab with user defined text.
\newcommand{\measureD}[1]{*{\xy*+=+<.5em>{\vphantom{#1}}*\cir{r_l};p\save*!R{#1} \restore\save+UC;+UC-<.5em,0em>*!R{\hphantom{#1}}+L **\dir{-} \restore\save+DC;+DC-<.5em,0em>*!R{\hphantom{#1}}+L **\dir{-} \restore\POS+UC-<.5em,0em>*!R{\hphantom{#1}}+L;+DC-<.5em,0em>*!R{\hphantom{#1}}+L **\dir{-} \endxy} \qw}
    % Inserts a D-shaped measurement gate with user defined text.
\newcommand{\multimeasure}[2]{*+<1em,.9em>{\hphantom{#2}} \qw \POS[0,0].[#1,0];p !C *{#2},p \drop\frm<.9em>{-}}
    % Draws a multiple qubit measurement bubble starting at the current position and spanning #1 additional gates below.
    % #2 gives the label for the gate.
    % You must use an argument of the same width as #2 in \ghost for the wires to connect properly on the lower lines.
\newcommand{\multimeasureD}[2]{*+<1em,.9em>{\hphantom{#2}}\save[0,0].[#1,0];p\save !C *{#2},p+LU+<0em,0em>;+RU+<-.8em,0em> **\dir{-}\restore\save +LD;+LU **\dir{-}\restore\save +LD;+RD-<.8em,0em> **\dir{-} \restore\save +RD+<0em,.8em>;+RU-<0em,.8em> **\dir{-} \restore \POS !UR*!UR{\cir<.9em>{r_d}};!DR*!DR{\cir<.9em>{d_l}}\restore \qw}
    % Draws a multiple qubit D-shaped measurement gate starting at the current position and spanning #1 additional gates below.
    % #2 gives the label for the gate.
    % You must use an argument of the same width as #2 in \ghost for the wires to connect properly on the lower lines.
\newcommand{\control}{*-=-{\bullet}}
    % Inserts an unconnected control.
\newcommand{\controlo}{*!<0em,.04em>-<.07em,.11em>{\xy *=<.45em>[o][F]{}\endxy}}
    % Inserts a unconnected control-on-0.
\newcommand{\ctrl}[1]{\control \qwx[#1] \qw}
    % Inserts a control and connects it to the object #1 wires below.
\newcommand{\ctrlo}[1]{\controlo \qwx[#1] \qw}
    % Inserts a control-on-0 and connects it to the object #1 wires below.
\newcommand{\targ}{*{\xy{<0em,0em>*{} \ar @{ - } +<.4em,0em> \ar @{ - } -<.4em,0em> \ar @{ - } +<0em,.4em> \ar @{ - } -<0em,.4em>},*+<.8em>\frm{o}\endxy} \qw}
    % Inserts a CNOT target.
\newcommand{\qswap}{*=<0em>{\times} \qw}
    % Inserts half a swap gate. 
    % Must be connected to the other swap with \qwx.
\newcommand{\multigate}[2]{*+<1em,.9em>{\hphantom{#2}} \qw \POS[0,0].[#1,0];p !C *{#2},p \save+LU;+RU **\dir{-}\restore\save+RU;+RD **\dir{-}\restore\save+RD;+LD **\dir{-}\restore\save+LD;+LU **\dir{-}\restore}
    % Draws a multiple qubit gate starting at the current position and spanning #1 additional gates below.
    % #2 gives the label for the gate.
    % You must use an argument of the same width as #2 in \ghost for the wires to connect properly on the lower lines.
\newcommand{\ghost}[1]{*+<1em,.9em>{\hphantom{#1}} \qw}
    % Leaves space for \multigate on wires other than the one on which \multigate appears.  Without this command wires will cross your gate.
    % #1 should match the second argument in the corresponding \multigate. 
\newcommand{\push}[1]{*{#1}}
    % Inserts #1, overriding the default that causes entries to have zero size.  This command takes the place of a gate.
    % Like a gate, it must precede any wire commands.
    % \push is useful for forcing columns apart.
    % NOTE: It might be useful to know that a gate is about 1.3 times the height of its contents.  I.e. \gate{M} is 1.3em tall.
    % WARNING: \push must appear before any wire commands and may not appear in an entry with a gate or label.
\newcommand{\gategroup}[6]{\POS"#1,#2"."#3,#2"."#1,#4"."#3,#4"!C*+<#5>\frm{#6}}
    % Constructs a box or bracket enclosing the square block spanning rows #1-#3 and columns=#2-#4.
    % The block is given a margin #5/2, so #5 should be a valid length.
    % #6 can take the following arguments -- or . or _\} or ^\} or \{ or \} or _) or ^) or ( or ) where the first two options yield dashed and
    % dotted boxes respectively, and the last eight options yield bottom, top, left, and right braces of the curly or normal variety.
    % \gategroup can appear at the end of any gate entry, but it's good form to pick one of the corner gates.
    % BUG: \gategroup uses the four corner gates to determine the size of the bounding box.  Other gates may stick out of that box.  See \prop. 
\newcommand{\rstick}[1]{*!L!<-.5em,0em>=<0em>{#1}}
    % Centers the left side of #1 in the cell.  Intended for lining up wire labels.  Note that non-gates have default size zero.
\newcommand{\lstick}[1]{*!R!<.5em,0em>=<0em>{#1}}
    % Centers the right side of #1 in the cell.  Intended for lining up wire labels.  Note that non-gates have default size zero.
\newcommand{\ustick}[1]{*!D!<0em,-.5em>=<0em>{#1}}
    % Centers the bottom of #1 in the cell.  Intended for lining up wire labels.  Note that non-gates have default size zero.
\newcommand{\dstick}[1]{*!U!<0em,.5em>=<0em>{#1}}
    % Centers the top of #1 in the cell.  Intended for lining up wire labels.  Note that non-gates have default size zero.
\newcommand{\Qcircuit}{\xymatrix @*=<0em>}
    % Defines \Qcircuit as an \xymatrix with entries of default size 0em.


\begin{document}

\section{Polynomial bounds for circuit complexity}

It is a well-known result (citation needed here) from classical circuit
complexity that any symmetric Boolean function that depends on 
polynomially-weighted inputs can be enacted in constant depth and polynomial
sizein a circuit
that allows threshold gates. This is surprising, since there are
exponentially many rows in a Boolean function. The method from Siu and Bruck
declares the existence of intervals within $[-N,N]$, where $N$ is the sum
of the magnitudes of all the weights, and represents the entire range
of the weighted sum, where $0$ is the threshold over the whole circuit.

\begin{equation}
N = \sum_{i=0}^n | w_i |
\end{equation}

That is, the final threshold gate in the circuit is the sign function of
the weighted sum of the inputs.

\begin{equation}
f(X) = \text{sgn} \left( w_0 + \sum_{i=0}^n w_i x_i \right)
\end{equation}

In their proof, Siu and Bruck find that at least two layers of threshold
gates are needed to enact any symmetric boolean function. In the first layer,
they calculate $y_k$'s which are the sum of the differences of each weighted
input $w_i x_i$ with one of $\ell$ intervals $[k_j, \tilde{k}_j]$. If the
sum of the weighted inputs are within a particular interval, then that particular
$y_k$ will be 1, otherwise, all the $y_k$'s will be zero.
The second layer sums up the $y_k$'s. If at least one of them is $1$ then their
sum is 1, and the circuit outputs 1. Otherwise, none of the $y_k$'s are one,
and the circuit outputs 0. Thus, the trick then is to find the intervals
$[k_j, \tilde{k}_j]$ that correspond to a given symmetric Boolean function and
to guarantee that the number of intervals $\ell$ is polynomially bounded by
$n$, i.e. $\ell \le n^c$ for some $c>0$.

The existence of such an $\ell$ derives from the theory of $GF(q)$. However,
this is non-constructive. Explicit constructions for the addition gate
$ADD$ and the comparison gate $COMP$ are given in [citation needed],
using the majority gate $MAJ$, which is just a special case for of the
threshold gate where the weights are either $+1$ or $-1$. Equivalent results
can be found for the canonical majority gate where all the weights are
$+!$.

The $ADD$ gates depend on expressing each of the $2\log_2 n$ bits of
the block-sum $s$ as a threshold function on a polynomial number of inputs.
In the given construction, $n\times n$-bit numbers are summed together,
and each bit $s_i$ of the sum, for $0 \le i < n$,
depends not only on the $n$ bits of significance
$2^i$ but also on any carry bits generated and propagated for $0 \le j < i$.
Each bit of the sum depends not only on the $n$

The original construction says it suffices to show the final high-order
carry bit $c_n$, which potentially depends on the most number of inputs,
that is, all $n \log_2 n$ bits of the entire block. However, in this thesis,
we give a general formula for each individual sum and carry bit. The odd
and even blocks are summed in parallel, with alternating blocks set to zero.

Therefore, the existence of these polynomials provides an upper bound on
the number of terms in the weighted sum, and therefore on the number of
intervals in the threshold gate. This is in contrast to the polynomial method
used to lower bound quantum query complexity, as in Beals et al. and 
Ambainis.

The construction depends on \emph{delta polynomials}, which in analogy to the
Dirac delta function, has much of its weight on a particular input, namely,
all $+1$'s or all $-1$'s, and a much smaller weight on all other inputs.
Furthermore, we are only interested in polynomials with a polynomially
bounded number of monomials, 
We say that a polynomial $P(x_n, x_{n-1}, \ldots, x_1, x_0)$ is a
$(c,\epsilon)$ sparse delta-polynomial if for a particular
$\epsilon = \{\epsilon_n, \ldots, \epsilon_0\}$

\begin{equation}
P(\epsilon) \ge a
\end{equation}

and for all other input vectors $\epsilon'$

\begin{equation}
P(\epsilon') < a/c
\end{equation}

In particular, the input $\epsilon$ signifies the condition where we want
the threshold gate to trigger, that is, we want its output to be 1. We can
alter the weights $w_i$ so that this triggering happens for the right
input vector $\epsilon$, in a sense, ``programming'' the polynomial.

However, this suffices just for the addition function. In general, we would
like to solve the harder problem of multiplication, and finally, of greatest
interest to factoring, modular exponentiation, or powering, or one $n$-bit number
by another. Therefore, finding the explicit construction may rely on
similar techniques of calculating the character of a finite field.

\end{document}