\chapter{Conclusion}
\label{chap:conclude}

I. Thesis statement
	A. Review of the thesis statement
	B. Organization of the conclusion

II. Review of Chapter 0
	A. Introduction, motivation, background
	B. quantum architecture-specific background
		1. circuit basis
		2. arch & arch model
		3. constant-depth communication
		4. modules

II. Review of Chapter 1
	A. main results, each module of factoring, leading up to polylog hybrid factoring

III. Review of Chapter 2
	A. main results, each module of quantum majority gates, leading up to sublog hybrid factoring
	B. conjecture of sublog factoring without using quantum majority gates

IV. Review of Chapter 3
	A. main result: quantum fourier state generation with early measurement and reduced resources
	B. based on result in the appendix

V. Review of Chapter 4
	A. main results: connection of circuit coherence for layered quantum circuits with pebble game space
	B. main result: configurable depth factoring from Chapter 1, in two ways
	B. conjecture: circuit coherence for factoring is O(n^3\log n \log\log n) and is asymptotically separated from depth*width

VI. Future directions
	A. Exploit locality in hamiltonian simulation, apply general principles to another algorithm
	B. make a more general quantum processor for running multiple algorithms

Quantum architecture is the intermediate layer between algorithms and hardware.
It is the design of how quantum bits and their allowed interactions in order
to solve these algorithms efficiently on realistic models of quantum hardware.
It aims to minimize circuit resources of interest, namely depth, size, and
width. In analogy to classical circuits, the depth is the running time of an
algorithm allowing parallelization, the size is the number of operations
required over all parallel processors, and the width is the number of
(quantum) bits required over all parallel processors.

In this dissertation, we study quantum architecture in the context of
optimizing Shor's factoring algorithm on nearest-neighbor architectures with
realistic constraints. It is hoped that lessons learned in this special-case
can contribute to the general community general principles which can be used
to generalize other quantum algorithms. We posit that the larger overall goal
of quantum architecture should be the design of a general-purpose quantum
processor, one which can execute any quantum algorithm with emphasis on being
able to perform a core group of operations efficiently. This core group of
operations is defined by the instruction set. Once quantum architecture has
progressed to this point, we can leverage the remarkable strides in normal
digital architecture over the past 80 years.

As well, we may be able to use the insights of quantum architecture to build
quantum processors which can surpass digital architectures in the solution of
large-scale problems over exponentially-sized solution spaces.
