\chapter{KSV Error from Early Measurement}
\label{chap:ksv-error}

In this note, we provide a detailed calculation for the error in
phase estimation due to Kitaev, Shen, and Vyalyi (KSV) where projective
measurement is used instead of a coherent measurement. We show an
increase from $2\sqrt{\epsilon}$ to $\sqrt{2}\sqrt[4]{\epsilon}$. The use
of projective measurement is to offload many trigonometric operations
onto a classical controller instead of doing them reversibly on a quantum
computer. The cost of projective measurement is leaving garbage qubits in
the ancillae of the target register, but this is a constant amount relative to
the size of circuits we may wish to compile using the KSV procedure.
Our goal is to show that this increase in error by a square root
factor is negligible and may be preferrable in realistic implementations.

To begin with, we review (coherent) measurement operators as a generalization of
controlled quantum operators, and then we further extend them to the case
of approximately measuring functions on orthogonal basis decompositions
where there are garbage bits left over in an ancillary register which must
be uncomputed.
Next, we calculate the error of this approximate measurement with ancillae.
Then, we show how this general measurement procedure corresponds to
estimating the phase of the modular multiplication operator used in
Shor's factoring algorithm. Finally, we show that the error only
increases by a square root factor when we projectively measure the garbage
in the ancillae instead of coherently simulating the measurement.

%%%%%%%%%%%%%%%%%%%%%%%%%%%%%%%%%%%%%%%%%%%%%%%%%%%%%%%%%%%%%%%%%%%%%%%%%%%%%%
%%%%%%%%%%%%%%%%%%%%%%%%%%%%%%%%%%%%%%%%%%%%%%%%%%%%%%%%%%%%%%%%%%%%%%%%%%%%%%
\section{Measurement Operators}
\label{sec:meas-ops}

First we introduce some preliminary definitions related to measurement.

A very common quantum operation entangles the results of one register
(called the \emph{target}) based on the value of another register (called the
\emph{control}). The most basic case of this is CNOT, or $\Lambda(X)$,
operator, which operates
on a control register of one qubit and a target register of one qubit.

\begin{equation}
\Lambda(X)\ket{y,z} \rightarrow \ket{y, z \oplus y}
\end{equation}

We have two ways of describing how CNOT is measuring in this case.
We can say CNOT is \emph{measuring with respect to the decomposition} of
the control registers, namely the computational basis, in that the
operation on the target register (flipping the bit) depend on decomposing the
control register in the basis $\{0,1\}$. We can also say that CNOT is
\emph{measuring a function} $f:\{0,1\} \rightarrow \{0,1\}$ which in this
case is simply the copy operator, $f(x) = x$. We can rewrite the equation
above as:

\begin{equation}
\Lambda(X)\ket{y,z} \rightarrow \ket{y, z \oplus f(y)}
\end{equation}

However, we know that the distinction between control and target registers
often depends on a particular basis. For example, we can flip the direction
of control and target in the CNOT case by conjugating both qubits with
Hadamard operators. The general characteristic of measurement operators
is that they are entangling, and that they can encode information about
one register in another in a very general way.
From the point of view of measurement, we call the control register the
\emph{object}, as in the state we are trying to measure, and the target register
is the \emph{instrument}, as in the state that we transform according to
projecting the measurement object in some orthogonal decomposition.

Let's begin with a simple but more general case of a measuring operator
$W$ which operates on a space decomposed into the subsystems $\mathcal{N}$
(the measurement object) and $\mathcal{K}$ (the measurement instrument)
according to the orthogonal decomposition
$\mathcal{N} = \bigoplus_j \mathcal{L}_j$, so-called because
each of the $\mathcal{L}_j$ are pairwise orthogonal subspaces.
$W$ will perform a different unitary
operator $U_j$ on the subsystem $\mathcal{K}$
depending on the projection of $\mathcal{N}$ into each $\mathcal{L}_j$.

\begin{equation}
W = \sum_{j \in \Omega} \Pi_{\mathcal{L}_j} \otimes U_j
\end{equation}

An interesting fact about this definition of measurement operators is
that approximativeness is preserved. If each unitary $U_j$ is replaced with
another unitary $\tilde{U}_j$ that approximates it with precision $\delta$,
then the new measuring operator $\tilde{W}$ also approximates the original
$W$ with the precision $\delta$.

This precision is defined in terms of the inner product between any state
$\ket{\zeta}$
operated
on by $W$ and by $\tilde{W}$. We can decompose $\ket{\zeta}$ into
two subsystems $\ket{\psi}$ and $\ket{\phi}$
corresponding to the spaces $\mathcal{N}$ and $\mathcal{K}$
above. We will use this fact later.

\begin{equation}
\bra{\zeta} W^{\dag} \tilde{W} \ket{\zeta} =
\bra{\psi} \otimes \bra{\phi} W^{\dag} \tilde{W} \ket{\psi} \otimes \ket{\phi} =
\sum_{j \in \Omega} \left( \bra{\psi} \Pi_{\mathcal{L}_j} \ket{\psi} \otimes
\bra{\phi} U^{\dag}_j \tilde{U}_j \ket{\phi} \right) \le \delta
\end{equation}

%%%%%%%%%%%%%%%%%%%%%%%%%%%%%%%%%%%%%%%%%%%%%%%%%%%%%%%%%%%%%%%%%%%%%%%%%%%%%%
%%%%%%%%%%%%%%%%%%%%%%%%%%%%%%%%%%%%%%%%%%%%%%%%%%%%%%%%%%%%%%%%%%%%%%%%%%%%%%
\section{Notation}

A random section on notation. I will find a better place to put this later.

In \cite{Kitaev2002}, ancillae are denoted as $\mathcal{B}^{\otimes N}$ to mean
appending $N$ qubits to a quantum state. However, we will use the more general
symbol $\mathcal{J}$ to mean any ancillary register of $N$ qudits of dimension $d$.
In context,
when we mean an ancillae state initialized to all zeros, we will write
$\ket{0^N}$ where $N = \log_d(\dim(\mathcal{J}))$.

In these notes we will mainly deal with unitary operators $U$, which by
definition operate on
states from some vector space $\mathcal{V}$
without enlarging or shrinking the space.
We can think of them as being square
$\dim(\mathcal{V}) \times \dim(\mathcal{V})$
unitary matrices.
However, mathematically adding ancillae qubits enlarges our Hilbert space
(say from $\mathcal{V}$ to $\mathcal{V}\otimes \mathcal{J}$),
and discarding ancillae qubits shrinks our Hilbert space
(say from $\mathcal{V}\otimes \mathcal{J}$ back to $\mathcal{V}$),
and these are more properly represented by isometries $\hat{U}$,
or non-square matrices
of dimension
$\dim(\mathcal{V})\times \dim(\mathcal{V})\cdot\dim(J)$
(or the transpose of that), which are unital in that
$\hat{U}\hat{U}^\dagger = I$. To convert between the two matrix sizes,
so that we can combine unitary and unital operators, we can use the
standard embedding as in \cite{Kitaev2002}:

\begin{equation}
\forall \ket{\xi} \in mathcal{V} \qquad V\ket{\xi} = \ket{\xi} \otimes \ket{0^N}
\end{equation}

However, in these notes, we take the equivalent approach of just
tensoring a unitary operator without ancillae with identity on the
ancillae Hilbert space ($U \otimes I_{\mathcal{B}^{\otimes N}}$).

For unitary operators, the input and output vector spaces are the same,
so we will often just write them in the following notation instead of
the usual $U: \mathcal{V} \rightarrow \mathcal{V}$.

\begin{equation}
U : \mathcal{V}
\end{equation}

%%%%%%%%%%%%%%%%%%%%%%%%%%%%%%%%%%%%%%%%%%%%%%%%%%%%%%%%%%%%%%%%%%%%%%%%%%%%%%
%%%%%%%%%%%%%%%%%%%%%%%%%%%%%%%%%%%%%%%%%%%%%%%%%%%%%%%%%%%%%%%%%%%%%%%%%%%%%%
\section{Operators That Measure a Function}
\label{sec:meas-func}

We now introduce the idea that an operator can measure a function
from the indices of the measurement object space $\mathcal{N}$
to the measurement instrument space $\mathcal{K}$ with respect to
some fixed orthogonal decompositions of these spaces.

\begin{equation}
\mathcal{N} = \bigotimes_{j \in \Omega} \qquad
\mathcal{K} = \bigotimes_{y \in \Delta} \qquad
f:\Omega \rightarrow \Delta
\end{equation}

Saying that an operator $Y$ is measuring with respect to a fixed
orthogonal decomposition $\Omega$ is equivalent to saying that
$Y$ measures the function $f$, which need not even be reversible.
The simplest, but not the most general, form of such an operator projects
the first subsystem $\mathcal{N}$ into a subspace $\mathcal{L}_j$
and performs the
corresponding operation $Q_{f(j)}$ on the second subsystem $\mathcal{K}$.
In Equation \ref{eqn:y-ideal}, we define $Y$ as the sum of these projectors,
and our notation means it operates on the space of $\mathcal{N}$
tensored with $\mathcal{K}$.

\begin{equation}
Y = \sum_{j \in \Omega} \Pi_{\mathcal{L}_j} \otimes Q_{f(j)} : \mathcal{N} \otimes \mathcal{K}
\label{eqn:y-ideal}
\end{equation}



%%%%%%%%%%%%%%%%%%%%%%%%%%%%%%%%%%%%%%%%%%%%%%%%%%%%%%%%%%%%%%%%%%%%%%%%%%%%%%
%%%%%%%%%%%%%%%%%%%%%%%%%%%%%%%%%%%%%%%%%%%%%%%%%%%%%%%%%%%%%%%%%%%%%%%%%%%%%%
\section{Measurement Operators with Ancillae}
\label{sec:meas-ancillae}

However, this is not the most general case of a measurement operator.
We make \emph{three extensions} here which require cascading
two rounds of measurement through
an ancillary register. We now define a new
composite measurement operator $\tilde{Y}$
on this space with three subsystems corresponding to the input register
$\mathcal{N}$,
the intermediate ancillary register $\mathcal{B}^N$, and a final output register
$\mathcal{K}$ which approximates $Y$ from Equation \ref{eqn:y-ideal}.

\begin{equation}
\tilde{Y} : \mathcal{N} \otimes \mathcal{B}^N \otimes \mathcal{K}
\end{equation}

We will build up to an operational definition of $\tilde{Y}$ later, but first
we must motivate what it must achieve.

In the first round, we measure with our object in $\mathcal{N}$ and our
instrument in $\mathcal{B}^N$. Here we allow for garbage, which is our first
extension, described in \ref{subsec:garbage}.
In the second round, we measure with our object being the
instrument of the first round, in $\mathcal{B}^N$, and our instrument is
in $\mathcal{K}$. Here we allow for operations other than just copying from
$\mathcal{B}^N$ to $\mathcal{K}$, which is our second extension, described
in \ref{subsec:non-copy}. Finally, instead of implementing our ideal
$Y$ directly, we allow ourselves, and quantify what it means, to
approximate it as $\tilde{Y}$.

%%%%%%%%%%%%%%%%%%%%%%%%%%%%%%%%%%%%%%%%%%%%%%%%%%%%%%%%%%%%%%%%%%%%%%%%%%%%%%
\subsection{First Extension: Measurement with Garbage}
\label{subsec:garbage}

To motivate why we need this new composite measurement $Y$ instead of just
using $W$ directly from the previous section, let's consider the problem of
garbage.
In general, only some of the information computed by $W$ into its
target register $\mathcal{K}$ above may be useful, but we generate some
garbage qubits (say, to make the operation reversible).
To be concrete, let's say that we have the following:

\begin{equation}
W: \mathcal{N} \otimes \mathcal{B}^N =
\sum_{j \in \Omega} \Pi_{\mathcal{L}_j} \otimes U_j \qquad
U_j : \mathcal{B}^N \rightarrow \mathcal{B}^N \qquad
U_j\ket{0} = \sum_{y,z} c_{y,z}(j)\ket{y,z}
\end{equation}

where $y \in \mathbb{B}^m$ represents the useful part of the result,
$z \in \mathbb{B}^{N-m}$ is garbage, and the complex weights
$c_{y,z} \in \mathbb{C}$
are functions of the index of the orthogonal decomposition of $\mathcal{N}$,
the number $j$. In many cases, we would like to uncompute this garbage in
order to use space efficiently.
We can do this using an uncomputing trick due to Charlie Bennett
\cite{Bennett1973} by first running an operation $U$ which may produce
garbage, copying out the useful part of the result to a
second register, and finally running $U^{\dagger}$ to uncompute the
input register.

Now we will give a more concrete definition for $Y$ in terms of
$W$ and the operation $V$ which simply copies a state $\ket{\psi} \in \mathcal{B}^N$
from the ancillary register to a state $\ket{\phi} in \mathcal{K}$ in the
output register using bitwise CNOTs, or bitwise addition modulo $2^n$ for an
$n$-qubit register. Then $Y = W^{-1}VW$, where $W$ is our operator from above
which computes a function with garbage, $V$ is our copy operation, and
$W^{-1}$ uncomputes the original function and it garbage. We make this
more rigorous in Equation \ref{eqn:wvw}.

\begin{eqnarray}
Y & = & WVW^{-1} : \mathcal{N} \otimes \mathcal{B}^N \otimes \mathcal{K} \\
W & = &
\sum_{j \in \Omega} \Pi_{\mathcal{L}_j} \otimes U_j \otimes I_{\mathcal{K}}\\
V & : & \ket{x} \otimes \ket{y} \otimes \ket{z} \rightarrow \ket{x} \otimes \ket{y} \otimes \ket{z \oplus y}
\label{eqn:wvw}
\end{eqnarray}

%%%%%%%%%%%%%%%%%%%%%%%%%%%%%%%%%%%%%%%%%%%%%%%%%%%%%%%%%%%%%%%%%%%%%%%%%%%%%%
\subsection{Second Extension: Measurement with a Non-Copy Operation}
\label{subsec:non-copy}

Our notion of measurement with garbage is still not the most general it could be.
For example, there is no reason why the decomposition of $\mathcal{K}$ has to
be the same as that for $\mathcal{B}^N$ or $\mathcal{N}$, or why we are limited
to strictly copying the useful output from $\mathcal{B}^N$ into $\mathcal{K}$.
We don't need to limit the function $f$ measured by $W$ to be invertible.
Furthermore, for practical reasons, there may be a more efficient way of encoding
the useful part of $f$'s output, or we may need to do further processing on it
as part of the algorithm we want to run.

To allow this, in this section, we allow an arbitrary orthogonal decomposition
$\mathcal{K} = \bigotimes_{y \in \Delta} \mathcal{M}_y$. This is the same as
allowing $V$ to be an arbitrary sum of projectors onto $\{\mathcal{M}_y\}$
and corresponding unitary operators $Q_y$ on $\mathcal{K}$.

\begin{equation}
V = \sum_{y \in \Delta} I_{\mathcal{N}} \otimes Q_y \otimes \Pi_{\mathcal{M}_y}
\end{equation}

%%%%%%%%%%%%%%%%%%%%%%%%%%%%%%%%%%%%%%%%%%%%%%%%%%%%%%%%%%%%%%%%%%%%%%%%%%%%%%
\subsection{Third Extension: Approximate Measurement}
\label{subsec:approx}

We can now allow a very general form of
measurement which allows garbage, non-copying uncomputation, and finally,
approximating a function. We will explain this last feature in this section.

We review that in the last two subsections we have been building up a
composite measurement operator $\tilde{Y} = W^{-1}VW$ which consists of 
these two operations on a space with three subsystems.

\begin{equation}
W = \sum_{j \in \Omega} \Pi_{\mathcal{L}_j} \otimes R_j \otimes I_{\mathcal{K}} \qquad
V = \sum_{y \in \Delta} I_{\mathcal{N}} \otimes \Pi_{\mathcal{M}_y} \otimes Q_y
\end{equation}

We now reveal that $\tilde{Y}$ approximates $Y$ from Equation \ref{eqn:y-ideal}.
What does this mean? For one operator, say $\tilde{X}$, to approximate another,
say $X$, within an error $\delta$ means that the maximum overlap between
any state $\ket{\eta}$ that is operated on by the difference
$\bar{X} = (\tilde{X} - X$) is at most
$\delta$. This is shown in Equation \ref{eqn:overlap}, and is analogous
to the method in Equation \ref{eqn:op-diff}.

\begin{equation}
\bra{\eta} \bar{X}^{\dagger} \bar{X} \ket{\eta} \le \delta
\label{eqn:overlap}
\end{equation}

Now how do we relate this to approximating a function
$f$ between the orthogonal decomposition indices of $\mathcal{N}$ and 
$\mathcal{K}$? And what does this latter concept mean?
It means that the conditional probabilities of
getting a given output index $y \in \Delta$ given an input index $j \in \Omega$,
namely $P(y \mid j) = \bra{0^N} R^{\dagger}_j \Pi_{\mathcal{M}_y} R_j \ket{0^N}$
satisfies Equation \ref{eqn:f-approx}. Approximating a function $f$ with
error probability $\epsilon$ means that the probability of getting the
correct answer $f(j)$ is bounded below by $1 - \epsilon$, and the probability
of getting any $y \ne f(j)$ is bounded above by $\epsilon$.

\begin{eqnarray}
P(f(j) \mid j) & \ge & 1 - \epsilon \\
\sum_{y \ne f(j)} P(y \mid j) & < & \epsilon
\label{eqn:f-approx}
\end{eqnarray}

In the next section, we will delve into detailed calculations of relating
$\delta$ and $\epsilon$ given the framework we have built up until now.


%%%%%%%%%%%%%%%%%%%%%%%%%%%%%%%%%%%%%%%%%%%%%%%%%%%%%%%%%%%%%%%%%%%%%%%%%%%%%%
%%%%%%%%%%%%%%%%%%%%%%%%%%%%%%%%%%%%%%%%%%%%%%%%%%%%%%%%%%%%%%%%%%%%%%%%%%%%%%
\section{Error Calculations}

In this section, we will calculate two error probabilities $\delta$ for
an approximate measuring operator $\tilde{Y}$, given that we are able to
approximate a function $f$ with error probability $\epsilon$. The first
error probability, which follows exactly the development in Problem 12.2
of Ref \cite{Kitaev2002}, assumes that we measure \emph{coherently}, meaning
that at no point do we projectively measure, and we are able to perfectly
uncompute all garbage. This is the ideal case, which we calculate in
\ref{subsec:error-noproj}. The second error probability, which we use in
\cite{Pham2012a}, involves projectively measuring as part of the operator $W$,
which involves some purely classical reversible operations that can be
offloaded to a classical controller. Afterwards, we execute $V$ as before,
but it is now impossible to run $W^{-1}$ because of the projective measurement.
This leaves some amount of garbage in the ancillary
subsystem $\mathcal{B}^N$, and we calculate it in \ref{subsec:error-proj}.

%%%%%%%%%%%%%%%%%%%%%%%%%%%%%%%%%%%%%%%%%%%%%%%%%%%%%%%%%%%%%%%%%%%%%%%%%%%%%%
%%%%%%%%%%%%%%%%%%%%%%%%%%%%%%%%%%%%%%%%%%%%%%%%%%%%%%%%%%%%%%%%%%%%%%%%%%%%%%
\subsection{Error Probability With Coherent Measurement}
\label{subsec:error-noproj}

First, we use this preliminary lemma. using properties of
the operator norm.

%%%%%%%%%%%%%%%%%%%%%%%%%%%%%%%%%%%%%%%%%%%%%%%%%%%%%%%%%%%%%%%%%%%%%%%%%%%%%%
\begin{lemma}[Solution 12.1, p. 230 \cite{ksv02}]
\label{lemma:sum-norm}
Let $X_j : \mathcal{N}_j \rightarrow \mathcal{M}_j $
be a collection of operators which operate on pairwise orthogonal
subspaces, where each $X_j$ takes $\mathcal{N}_j$ to $\mathcal{M}_j$.
Then the norm of the operator $X$ formed as a direct product of these
$X_j$'s has an operator norm equal to the maximum of any of the
$X_j$'s.

\begin{equation}
X = \bigoplus_j X_j : \bigoplus_j \mathcal{N}_j \rightarrow \mathcal{M}_j
\end{equation}

\begin{equation}
|| X || = \max_{j} ||X_j||
\end{equation}
\end{lemma}

\begin{proof}
This follows from the fact that the operator norm measures how much
an operator scales any non-zero vector. If the vector comes from the
space which is the direct sum of the $\mathcal{N}_j$'s, it is in a
particular fixed subspace $\mathcal{N}_j$. Therefore, it cannot be scaled
more than the maximum operator norm of any of the $X_j$'s.
\end{proof}

We now apply this to the case of unitary operators
to show how to approximate a measuring operator with
ancillae which is the direct sum of projectors onto pairwise othogonal subspaces.

%%%%%%%%%%%%%%%%%%%%%%%%%%%%%%%%%%%%%%%%%%%%%%%%%%%%%%%%%%%%%%%%%%%%%%%%%%%%%%
\begin{lemma}[Problem 12.1 \cite{ksv02}]
\label{lemma:error-sum}
Let $W$ be a unitary operator which acts on a space with two subsystems
$\mathcal{N}$ and $\mathcal{K}$ and is the direct sum of projectors onto
pairwise orthogonal subspaces of $\mathcal{N} = \bigoplus_j \mathcal{L}_j$
tensored with unitary operators on $\mathcal{K}$.
Let $\tilde{W}$ be an analogous operator except that the unitaries $\tilde{U}_j$
now operate on $\mathcal{K}$ tensored with an ancillary subsystem
$\mathcal{B}^N$.

\begin{eqnarray}
U_j : \mathcal{K}\\
\tilde{U}_j : \mathcal{K} \otimes \mathcal{B}^{\otimes N}\\
W : \mathcal{N} \otimes \mathcal{K} = \bigoplus_j \Pi_{\mathcal{L}_j} \otimes U_j \\
\tilde{W} : \mathcal{N} \otimes \mathcal{K} \otimes \mathcal{B}^N =
\bigoplus_j \Pi_{\mathcal{L}_j} \otimes \tilde{U}_j
\end{eqnarray}

Suppose that for each $j$,
$\tilde{U}_j$ approximates (with ancillae) $U_j$ with error $\nu$
according to the
definition below.

\begin{equation}
\forall_j || \tilde{U}_j - (U_j \otimes I_{\mathcal{B}^{\otimes N}}) || \le \nu
\label{eqn:uj_approx}
\end{equation}

Then the measuring operator
$\tilde{W} = \oplus_j \Pi_{\mathcal{L}_j} \otimes \tilde{U}_j$ approximates
with ancillae the measuring operator
$W = \sum_j \Pi_{\mathcal{L}_j} \otimes U_j$ with the same error $\nu$.

\begin{equation}
|| \tilde{W} - (W \otimes I_{\mathcal{B}^{\otimes N}}) || \le \nu ||
\end{equation}

\end{lemma}

\begin{proof}
We decompose $W$ using its definition.

\begin{eqnarray}
|| \tilde{W} - (W \otimes I_{\mathcal{B}^{\otimes N}}) || & = &
|| (\bigoplus_j \Pi_{\mathcal{L}_j} \otimes \tilde{U}_j) -
   (\bigoplus_j \Pi_{\mathcal{L}_j} \otimes U_j \otimes I_{\mathcal{B}^{\otimes N}}\\
& = & || \bigoplus_j \Pi_{\mathcal{L}_j} \otimes (\tilde{U}_j - (U_j \otimes I_{\mathcal{B}^{\otimes N}})) || \\
& \le & \max_j ||    \Pi_{\mathcal{L}_j} \otimes (\tilde{U}_j - (U_j \otimes I_{\mathcal{B}^{\otimes N}})) ||
\end{eqnarray}

In the last step above, we use Lemma \ref{lemma:error-sum} to reduce the
error of approximating $W$ with ancillae to the largest error of approximating
any $U_j$ with ancillae. Now we use Equation \ref{eqn:uj_approx} and continue.

\begin{eqnarray}
|| \tilde{W} - (W \otimes I_{\mathcal{B}^{\otimes N}}) || & \le &
\max_j || (\tilde{U}_j - (U_j \otimes I_{\mathcal{B}^{\otimes N}}) || \\
& = & \nu
\end{eqnarray}

In the step above, we used the fact that when taking the norm of an operator
which is a projector onto a subspace
tensored with a unitary, this reduces to taking the norm
of just the unitary. This is because the vectors outside of the subspace,
which are projected away to the zero vector, cannot affect the operator norm.

Therefore, we have that $\tilde{W}$ \emph{with} ancillae approximates $\tilde{W}$
\emph{without} ancillae with the same error $\nu$ of approximating the unitaries
with ancillae within any subspace.
\end{proof}

We now return to the setting of Section \ref{subsec:approx} and repeat
its definitions here. Given a measuring operator $W$ which approximately
measures a function, what is the error of approximation
of a two stage measurement using $W$ and intermediate ancillae?
We will see later that such an approximate two-stage measurement
corresponds to parallelized phase estimation.

%%%%%%%%%%%%%%%%%%%%%%%%%%%%%%%%%%%%%%%%%%%%%%%%%%%%%%%%%%%%%%%%%%%%%%%%%%%%%%%
\begin{theorem}[Problem 12.2, \cite{ksv02}]
\label{thm:coherent}
We consider the space
$\mathcal{N} \otimes \mathcal{B}^{\otimes N} \otimes \mathcal{K}$ with the
following orthogonal subsystem decompositions:
$\mathcal{N} = \bigoplus_{j \in \Omega} \mathcal{L}_j$ and
$\mathcal{B}^{\otimes N} = \bigoplus_{y \in \Delta} \mathcal{M}_y$. We
define two operators, $W$ which measures $\mathcal{N}$ as object into
$\mathcal{B}^{\otimes N}$ as instrument and $V$ which measures
$\mathcal{B}^{\otimes N}$ as object into $\mathcal{K}$ as instrument.

\begin{eqnarray}
W : \mathcal{N} \otimes \mathcal{K}\\
\tilde{W} : \mathcal{N} \otimes \mathcal{K} \otimes \mathcal{B}^N
\end{eqnarray}

If $W$ approximately measures a function $f : \Omega \rightarrow \Delta$
with error $\epsilon$, then the operator $\tilde{Y} = W^{-1}VW$ approximates
the following operator $Y$ with error $2\sqrt{\epsilon}$.
\end{theorem}

%%%%%%%%%%%%%%%%%%%%%%%%%%%%%%%%%%%%%%%%%%%%%%%%%%%%%%%%%%%%%%%%%%%%%%%%%%%%%%%
\begin{proof}
Even though we have already defined $Y$ and $\tilde{Y}$ in terms of
operators on the whole space
$\mathcal{N} \otimes \mathcal{K} \otimes \mathcal{J}$, it is now useful to
define $\tilde{Y}$ in an alternate way: operators $P_j$ which operate on the
space $\mathcal{K} \otimes \mathcal{J}$ which can further be expressed
in terms of operators $Q_y$ operating on $\mathcal{K}$ and operators
$R_j$ operating on $\mathcal{J}$.

\begin{equation}
\tilde{Y} = \sum_{j \in \Omega} \Pi_{\mathcal{L}_j} \otimes P_j \qquad
P_j = \sum_{y \in \Delta} Q_y \otimes (R_j^{\dagger}\Pi_{\mathcal{M}_y}R_j)
\end{equation}

Using the results of Lemma \ref{lemma:error-sum}, we only need to show that
$P_j$ approximates (using ancillae) $Q_{f(j)}$ (without using ancillae) for
all $j$.

To begin with, let's examine the action of $P_j$ and $Q_{f(j)}$ on
an arbitrary state $\ket{\xi} \in \mathcal{K}$, possibly augmented with
ancillae $\ket{0^N}$.
We further define the following states and the difference between them.

\begin{eqnarray}
\ket{\eta}         & = & Q_{f(j)} \ket{\xi} \\
\ket{\tilde{\eta}} & = & P_j(\ket{\xi} \otimes \ket{0^N}) \\
\ket{\psi}         & = & \ket{\tilde{\eta}} - \ket{\eta}
\end{eqnarray}

We want to minimize the norm of $\ket{\psi}$, which represents the error
in a state operated on by the desired unitary $Q_{f(j)}$ and the state
operated on by the approximation with ancillae $P_j$.

That's how we begin these next calculations.

\begin{eqnarray}
\braket{\psi}{\psi} & = & \left[ \bra{\tilde{\eta}} - (\bra{\eta} \otimes \bra{0^N}) \right]
                    \left[ \ket{\tilde{\eta}} - (\ket{eta} \otimes \ket{0^N}) \right] \\
              & = & \left[ \braket{\tilde{\eta}}{\tilde{\eta}} \right ] - \\
              &   & \left[ (\bra{\eta} \otimes \bra{0^N})(\ket{\tilde{\eta}}) \right] - \\
              &   & \left[ (\bra{\tilde{\eta}} \otimes (\ket{\eta} \otimes \ket{0^N}) \right] + \\
              &   & \left[ \braket{\eta}{\eta} \right] \\
              & = & 2 - \left(\bra{\eta} \otimes \bra{0^N} \right) \ket{\tilde{\eta}} - \\
              &   & \bra{\tilde{eta}} \left( \ket{\eta} \otimes \ket{0^N} \right)
\end{eqnarray}

In the last line, we use the fact that $\ket{\eta} \otimes \ket{0^N}$ and
$\ket{\tilde{\eta}}$ are both normalized states of unit norm. The two braket
terms remaining are complex conjugates of each other (call them $a+bi$ and
$a-bi$) so their sum is just $2a$, or the real part of either complex number.

This is where we continue our calculations, using the definition of $\ket{\eta}$
and $\ket{\tilde{\eta}}$ to express their overlaps in terms of $Q_y$,
$R_j$, and projectors onto the orthogonal decomposition of $\mathcal{J}$.

\begin{eqnarray}
\braket{\psi}{\psi} & = & 2 - 2\Re\left[ \left(\bra{\eta} \otimes \bra{0^N}\right) \ket{\tilde{\eta}} \right] \\
                    & = & 2 - 2\Re\left[ \sum_{y \in \Delta} \bra{\xi}Q_{f(j)}^{\dagger}Q_y \ket{\xi}
                                                             \bra{0^N}R_j^{\dagger}\Pi_{\mathcal{M}_y}R_j\ket{0^N} \right] \label{eqn:sym-real}
\end{eqnarray}

In the last line above, we use the fact that we can distribute a braket
through a tensor product. That is, if $\ket{\alpha}$, $\ket{\beta}$, and
$\ket{\gamma}$ are normalized vectors, then
$\left( \bra{\alpha} \otimes \bra{\beta}\right) \ket{\gamma} = 
 \braket{\alpha}{\gamma}\braket{\beta}{\gamma}$.
 
We can now factor out the braket
$\bra{0^N}R_j^{\dagger}\Pi_{\mathcal{M}_y}R_j\ket{0^N}$ from the product
inside the real operation above since the braket is completely real. This is
because $\Pi_{\mathcal{M}_y}R_j \ket{0^N}$ is either the zero vector or
a normalized state. We will henceforce call this projected state
$\ket{\phi_{(y,j)}}$, and we will write its overlap with itself
as the braket $\braket{\phi}{\phi}$.

Now we also need to use the fact that the real part of a sum over complex
numbers is at most the sum of real parts of each complex number.

\begin{equation}
\Re\left[ \sum_{i} c_i \right] \quad \le \quad \sum_{i} \Re( c_i )
\end{equation}

Furthermore, the following inequality holds.

\begin{equation}
1 - \sum_{i} \Re(c_i) \quad \le \quad \sum_{i} \left(1 - \Re(c_i) \right)
\end{equation}

Substituting this
back in our original calculation, we get:

\begin{eqnarray}
\braket{\psi}{\psi} & \le & 2 - 2\sum_{y \in \Delta} \Re\left(
  \bra{\xi}Q_{f(j)}^{\dagger}Q_y \ket{\xi}
  \braket{\phi_{(y,j)}}{\phi_{(y,j)}} \right) \\
   & \le & 2 - 2\sum_{y \in \Delta} \Re\left(
           \bra{\xi}Q_{f(j)}^{\dagger}Q_y \ket{\xi}\Pr(y | j) \right) \\
   & \le & 2 - 2\sum_{y \in \Delta} \Re\left(
           \bra{\xi}Q_{f(j)}^{\dagger}Q_y \ket{\xi}\right) \Pr(y | j) \\
   & \le & 2 \left[ 1 - \sum_{y \in \Delta} \Re\left(
           \bra{\xi}Q_{f(j)}^{\dagger}Q_y \ket{\xi} \right) \right] \Pr(y | j) \\
   & \le & 2 \sum_{y \in \Delta}\left[
           1 - \Re\left( \bra{\xi}Q_{f(j)}^{\dagger}Q_y \ket{\xi} \right)
           \Pr(y | j)
           \right]
\end{eqnarray}

In the second line, we used the definition of
$\bra{0^N}R_j^{\dagger}\Pi_{\mathcal{M}_y}R_j\ket{0^N} = \braket{\phi_{(y,j)}}{\phi_{(y,h)}}$
as the probability of getting outcome $y$ in the register
$\mathcal{J}$ given that we have projected onto outcome $j$ in register
$\mathcal{N}$ according to Section \ref{subsec:approx}.

Next, we use the fact that the real part of the overlap
$\bra{\xi}Q_{f(j)}^{\dagger}Q_y \ket{\xi}$ is between $-1$ and $1$, inclusive,
because any $Q_y\ket{\xi}$ is a normalized state, and therefore one minus
this quantity is at most $2$. We also exclude the case where
$y = f(j)$, since in that case $Q_{f_(j)}^{\dagger}Q_y = I$ and the overlap
is 1, contributing zero to the sum.

\begin{eqnarray}
\braket{\psi}{\psi} & \le & 2 \sum_{y \ne f(j)} 2\Pr(y|j) \label{eqn:prob-sum}\\
                    & \le & 4\epsilon
\end{eqnarray}

In the final line above, we use the definition of $\Pr(y|j)$ from
Section \ref{subsec:approx}. The actual answer we want is the norm
of the difference vector $\ket{\psi}$, which is the square root of the
magnitude of the overlap.

\begin{equation}
\lvert\lvert \ket{\psi} \rvert\rvert \le 2\sqrt{\epsilon}
\end{equation}

This completes the proof.

As an additional note, if $V$ is the copy operator, we have
$Q_{f(j)}^{\dagger}Q_y = \delta_{y,f(j)}I_{\mathcal{K}}$.
For each $y = f(j)$, the overlap $\bra{\xi}Q_{f(j)}^{\dagger}Q_y \ket{\xi}$
is $1$ and contributes zero to the sum of probabilities.
For each $y \ne f(j)$, the overlap is exactly $0$, and contributes
$\Pr(y|j)$ to the sum.
Therefore, instead of line \ref{eqn:prob-sum} above, we get:

\begin{eqnarray}
\braket{\psi}{\psi} & \le & 2\sum_{y \ne f(j)} \Pr(y|j)\\
                    & \le & 2\epsilon
\end{eqnarray}

And the final error is

\begin{equation}
\lvert\lvert \ket{\psi} \rvert\rvert \le \sqrt{2\epsilon}
\end{equation}

This makes intuitive sense, since the first bound is overly general.
It works for any $V$. If we know $V$, we can actually make additional
assumptions and get a better (smaller) upper bound for the error, in this
case by a factor $\sqrt{2}$.
\end{proof}

%%%%%%%%%%%%%%%%%%%%%%%%%%%%%%%%%%%%%%%%%%%%%%%%%%%%%%%%%%%%%%%%%%%%%%%%%%%%%%
%%%%%%%%%%%%%%%%%%%%%%%%%%%%%%%%%%%%%%%%%%%%%%%%%%%%%%%%%%%%%%%%%%%%%%%%%%%%%%
\subsection{Error Probability With Projective Measurement}
\label{subsec:error-proj}

Now what happens if we perform the operator $\hat{Y} = VW$? That is, we
perform $W$ to measure some function $f$ and then we extract the useful
information using $V$, but we don't wish to uncompute the results by
performing $W^{-1}$? There are two main reasons we may wish to do this:
(1) because $W$ may be practically difficult to perform, and we don't wish
to essentially do it twice or (2) because we have projectively measured
in the register $\mathcal{K}$ so that we can offload some postprocessing to
a classical computer, and there is no way to uncompute $W$ at that point.

These are both practical reasons in that they don't affect the
asymptotic resources required by our algorithm. However, they do affect
the asymptotic error of our algorithm. In this section, we will calculate
this error based on Theorem \ref{thm:coherent}. This is the main original
contribution of this work beyond the exposition in \cite{ksv02}.

%%%%%%%%%%%%%%%%%%%%%%%%%%%%%%%%%%%%%%%%%%%%%%%%%%%%%%%%%%%%%%%%%%%%%%%%%%%%%%%
\begin{theorem}
\label{thm:projective}
Given the setting in Theorem \ref{thm:coherent}
where $W$ approximately measures a function $f : \Omega \rightarrow \Delta$
with error $\epsilon$, then the operator $\hat{Y} = VW$ approximates
the operator $Y = W^{-1}vW$ with error $\sqrt{2}\sqrt[4]{\epsilon}$.
\end{theorem}

\begin{proof}
We will take an almost identical approach to the proof for Theorem \ref{thm:coherent}
by measuring the difference between vectors which are the results of applying
$Y$ and $\hat{Y}$. However, now our definition of $P_j$ lacks the
uncomputation step $R_j^{\dagger}$.

\begin{equation}
\hat{Y} = \sum_{j \in \Omega} \Pi_{\mathcal{L}_j} \otimes P_j \qquad
P_j = \sum_{y \in \Delta} Q_y \otimes \Pi_{\mathcal{M}_y}R_j)
\end{equation}

Now we can jump ahead to equation \ref{eqn:sym-real}, which is the first
place in the proof of Theorem \ref{thm:coherent} where this asymmetry matters.

\begin{eqnarray}
\braket{\psi}{\psi} & = & 2 - 2\Re\left[ \sum_{y \in \Delta} \bra{\xi}Q_{f(j)}^{\dagger}Q_y \ket{\xi}
                                                             \bra{0^N}\Pi_{\mathcal{M}_y}R_j\ket{0^N} \right]
\end{eqnarray}

As in the proof of Theorem \ref{thm:coherent}, we can label the projected
ancillae state like so:

\begin{equation}
\ket{\phi_{(y,j)}} = \Pi_{\mathcal{M}_y}R_j \ket{0^N}
\end{equation}

and substitute it into our calculation:

\begin{eqnarray}
\braket{\psi}{\psi} & = & 2 - 2\Re\left[ \sum_{y \in \Delta} \bra{\xi}Q_{f(j)}^{\dagger}Q_y \ket{\xi}
                                                             \braket{0^N}{\phi_{(y,j)}} \right]
\end{eqnarray}

Here, the braket $\braket{0^N}{\phi_{(y,j)}}$ cannot be
directly factored out of the real operator since it is no longer the overlap of
a vector with itself, and therefore it is not guaranteed to be a
real number. Therefore, we must upper bound the real part of this complex
number with its magnitude, as in the inequality below with any two vectors
$\ket{\phi}$ and $\ket{\sigma}$.

\begin{equation}
\Re(\braket{\phi}{\sigma}) \quad \le \lvert \braket{\phi}{\sigma} \rvert
\end{equation}

\begin{eqnarray}
\braket{\psi}{\psi} & = & 2 - 2\Re\left[ \sum_{y \in \Delta} \bra{\xi}Q_{f(j)}^{\dagger}Q_y \ket{\xi}
                                                             \braket{0^N}{\phi_{(y,j)}} \right]
\end{eqnarray}

\end{proof}

