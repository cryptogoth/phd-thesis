%%%%%%%%%%%%%%%%%%%%%%%%%%%%%%%%%%%%%%%%%%%%%%%%%%%%%%%%%%%%%%%%%%%%%%%%%%%%%%
%%%%%%%%%%%%%%%%%%%%%%%%%%%%%%%%%%%%%%%%%%%%%%%%%%%%%%%%%%%%%%%%%%%%%%%%%%%%%%
\section{Error Calculations}

In this section, we will calculate two error probabilities $\delta$ for
an approximate measuring operator $\tilde{Y}$, given that we are able to
approximate a function $f$ with error probability $\epsilon$. The first
error probability, which follows exactly the development in Problem 12.2
of Ref \cite{Kitaev2002}, assumes that we measure \emph{coherently}, meaning
that at no point do we projectively measure, and we are able to perfectly
uncompute all garbage. This is the ideal case, which we calculate in
\ref{subsec:error-noproj}. The second error probability
involves projectively measuring as part of the operator $W$,
which involves some purely classical reversible operations that can be
offloaded to a classical controller. Afterwards, we execute $V$ as before,
but it is now impossible to run $W^{-1}$ because of the projective measurement.
This leaves some amount of garbage in the ancillary
subsystem $\mathcal{B}^N$, and we calculate it in \ref{subsec:error-proj}.

%%%%%%%%%%%%%%%%%%%%%%%%%%%%%%%%%%%%%%%%%%%%%%%%%%%%%%%%%%%%%%%%%%%%%%%%%%%%%%
%%%%%%%%%%%%%%%%%%%%%%%%%%%%%%%%%%%%%%%%%%%%%%%%%%%%%%%%%%%%%%%%%%%%%%%%%%%%%%
\subsection{Error Probability With Coherent Measurement}
\label{subsec:error-noproj}

First, we use this preliminary lemma. using properties of
the operator norm.

%%%%%%%%%%%%%%%%%%%%%%%%%%%%%%%%%%%%%%%%%%%%%%%%%%%%%%%%%%%%%%%%%%%%%%%%%%%%%%
\begin{lemma}[Solution 12.1, p. 230 \cite{Kitaev2002}]
\label{lemma:sum-norm}
Let $X_j : \mathcal{N}_j \rightarrow \mathcal{M}_j $
be a collection of operators which operate on pairwise orthogonal
subspaces, where each $X_j$ takes $\mathcal{N}_j$ to $\mathcal{M}_j$.
Then the norm of the operator $X$ formed as a direct product of these
$X_j$'s has an operator norm equal to the maximum of any of the
$X_j$'s.

\begin{equation}
X = \bigoplus_j X_j : \bigoplus_j \mathcal{N}_j \rightarrow \mathcal{M}_j
\end{equation}

\begin{equation}
|| X || = \max_{j} ||X_j||
\end{equation}
\end{lemma}

\begin{proof}
This follows from the fact that the operator norm measures how much
an operator scales any non-zero vector. If the vector comes from the
space which is the direct sum of the $\mathcal{N}_j$'s, it is in a
particular fixed subspace $\mathcal{N}_j$. Therefore, it cannot be scaled
more than the maximum operator norm of any of the $X_j$'s.
\end{proof}

We now apply this to the case of unitary operators
to show how to approximate a measuring operator with
ancillae which is the direct sum of projectors onto pairwise orthogonal subspaces.

%%%%%%%%%%%%%%%%%%%%%%%%%%%%%%%%%%%%%%%%%%%%%%%%%%%%%%%%%%%%%%%%%%%%%%%%%%%%%%
\begin{lemma}[Problem 12.1 \cite{Kitaev2002}]
\label{lemma:error-sum}
Let $W$ be a unitary operator which acts on a space with two subsystems
$\mathcal{N}$ and $\mathcal{K}$ and is the direct sum of projectors onto
pairwise orthogonal subspaces of $\mathcal{N} = \bigoplus_j \mathcal{L}_j$
tensored with unitary operators on $\mathcal{K}$.
Let $\tilde{W}$ be an analogous operator except that the unitaries $\tilde{U}_j$
now operate on $\mathcal{K}$ tensored with an ancillary subsystem
$\mathcal{B}^N$.
%
\begin{eqnarray}
U_j : \mathcal{K}\\
\tilde{U}_j : \mathcal{K} \otimes \mathcal{B}^{\otimes N}\\
W : \mathcal{N} \otimes \mathcal{K} = \bigoplus_j \Pi_{\mathcal{L}_j} \otimes U_j \\
\tilde{W} : \mathcal{N} \otimes \mathcal{K} \otimes \mathcal{B}^N =
\bigoplus_j \Pi_{\mathcal{L}_j} \otimes \tilde{U}_j
\end{eqnarray}
%
Suppose that for each $j$,
$\tilde{U}_j$ approximates (with ancillae) $U_j$ with error $\nu$
according to the
definition below.
%
\begin{equation}
\forall_j || \tilde{U}_j - (U_j \otimes I_{\mathcal{B}^{\otimes N}}) || \le \nu
\label{eqn:uj_approx}
\end{equation}
%
Then the measuring operator
$\tilde{W} = \oplus_j \Pi_{\mathcal{L}_j} \otimes \tilde{U}_j$ approximates
with ancillae the measuring operator
$W = \sum_j \Pi_{\mathcal{L}_j} \otimes U_j$ with the same error $\nu$.
%
\begin{equation}
|| \tilde{W} - (W \otimes I_{\mathcal{B}^{\otimes N}}) || \le \nu
\end{equation}
%
\end{lemma}

\begin{proof}
We decompose $W$ using its definition.
%
\begin{eqnarray}
|| \tilde{W} - (W \otimes I_{\mathcal{B}^{\otimes N}}) || & = &
|| (\bigoplus_j \Pi_{\mathcal{L}_j} \otimes \tilde{U}_j) -
   (\bigoplus_j \Pi_{\mathcal{L}_j} \otimes U_j \otimes I_{\mathcal{B}^{\otimes N}}\\
& = & || \bigoplus_j \Pi_{\mathcal{L}_j} \otimes (\tilde{U}_j - (U_j \otimes I_{\mathcal{B}^{\otimes N}})) || \\
& \le & \max_j ||    \Pi_{\mathcal{L}_j} \otimes (\tilde{U}_j - (U_j \otimes I_{\mathcal{B}^{\otimes N}})) ||
\end{eqnarray}
%
In the last step above, we use Lemma \ref{lemma:error-sum} to reduce the
error of approximating $W$ with ancillae to the largest error of approximating
any $U_j$ with ancillae. Now we use Equation \ref{eqn:uj_approx} and continue.
%
\begin{eqnarray}
|| \tilde{W} - (W \otimes I_{\mathcal{B}^{\otimes N}}) || & \le &
\max_j || (\tilde{U}_j - (U_j \otimes I_{\mathcal{B}^{\otimes N}}) || \\
& = & \nu
\end{eqnarray}
%
In the step above, we used the fact that when taking the norm of an operator
which is a projector onto a subspace
tensored with a unitary, this reduces to taking the norm
of just the unitary. This is because the vectors outside of the subspace,
which are projected away to the zero vector, cannot affect the operator norm.

Therefore, we have that $\tilde{W}$ \emph{with} ancillae approximates $\tilde{W}$
\emph{without} ancillae with the same error $\nu$ of approximating the unitaries
with ancillae within any subspace.
\end{proof}

We now return to the setting of Section \ref{subsec:approx} and repeat
its definitions here. Given a measuring operator $W$ which approximately
measures a function, what is the error of approximation
of a two stage measurement using $W$ and intermediate ancillae?
We will see later that such an approximate two-stage measurement
corresponds to parallelized phase estimation.

%%%%%%%%%%%%%%%%%%%%%%%%%%%%%%%%%%%%%%%%%%%%%%%%%%%%%%%%%%%%%%%%%%%%%%%%%%%%%%%
\begin{theorem}[Problem 12.2, \cite{Kitaev2002}]
We consider the space
$\mathcal{N} \otimes \mathcal{B}^{\otimes N} \otimes \mathcal{K}$ with the
following orthogonal subsystem decompositions:
$\mathcal{N} = \bigoplus_{j \in \Omega} \mathcal{L}_j$ and
$\mathcal{B}^{\otimes N} = \bigoplus_{y \in \Delta} \mathcal{M}_y$. We
define two operators, $W$ which measures $\mathcal{N}$ as object into
$\mathcal{B}^{\otimes N}$ as instrument and $V$ which measures
$\mathcal{B}^{\otimes N}$ as object into $\mathcal{K}$ as instrument.
%
\begin{eqnarray}
W : \mathcal{N} \otimes \mathcal{K}\\
\tilde{W} : \mathcal{N} \otimes \mathcal{K} \otimes \mathcal{B}^N
\end{eqnarray}
%
If $W$ approximately measures a function $f : \Omega \rightarrow \Delta$
with error $\epsilon$, then the operator $\tilde{Y} = W^{-1}VW$ approximates
the following operator $Y$ with error $2\sqrt{\epsilon}$.
\label{thm:coherent}
\end{theorem}

%%%%%%%%%%%%%%%%%%%%%%%%%%%%%%%%%%%%%%%%%%%%%%%%%%%%%%%%%%%%%%%%%%%%%%%%%%%%%%%
\begin{proof}
Even though we have already defined $Y$ and $\tilde{Y}$ in terms of
operators on the whole space
$\mathcal{N} \otimes \mathcal{K} \otimes \mathcal{J}$, it is now useful to
define $\tilde{Y}$ in an alternative way: operators $P_j$ which operate on the
space $\mathcal{K} \otimes \mathcal{J}$ which can further be expressed
in terms of operators $Q_y$ operating on $\mathcal{K}$ and operators
$R_j$ operating on $\mathcal{J}$.
%
\begin{equation}
\tilde{Y} = \sum_{j \in \Omega} \Pi_{\mathcal{L}_j} \otimes P_j \qquad
P_j = \sum_{y \in \Delta} Q_y \otimes (R_j^{\dagger}\Pi_{\mathcal{M}_y}R_j)
\end{equation}
%
Using the results of Lemma \ref{lemma:error-sum}, we only need to show that
$P_j$ approximates (using ancillae) $Q_{f(j)}$ (without using ancillae) for
all $j$.

To begin with, let's examine the action of $P_j$ and $Q_{f(j)}$ on
an arbitrary state $\ket{\xi} \in \mathcal{K}$, possibly augmented with
ancillae $\ket{0^N}$.
We further define the following states and the difference between them.
%
\begin{eqnarray}
\ket{\eta}         & = & Q_{f(j)} \ket{\xi} \\
\ket{\tilde{\eta}} & = & P_j(\ket{\xi} \otimes \ket{0^N}) \\
\ket{\psi}         & = & \ket{\tilde{\eta}} - \ket{\eta}
\end{eqnarray}
%
We want to minimize the norm of $\ket{\psi}$, which represents the error
in a state operated on by the desired unitary $Q_{f(j)}$ and the state
operated on by the approximation with ancillae $P_j$.

That's how we begin these next calculations.
%
\begin{eqnarray}
\braket{\psi}{\psi} & = & \left[ \bra{\tilde{\eta}} - (\bra{\eta} \otimes \bra{0^N}) \right]
                    \left[ \ket{\tilde{\eta}} - (\ket{\eta} \otimes \ket{0^N}) \right] \\
              & = & \left[ \braket{\tilde{\eta}}{\tilde{\eta}} \right ] - \\
              &   & \left[ (\bra{\eta} \otimes \bra{0^N})(\ket{\tilde{\eta}}) \right] - \\
              &   & \left[ (\bra{\tilde{\eta}} \otimes (\ket{\eta} \otimes \ket{0^N}) \right] + \\
              &   & \left[ \braket{\eta}{\eta} \right] \\
              & = & 2 - \left(\bra{\eta} \otimes \bra{0^N} \right) \ket{\tilde{\eta}} - \\
              &   & \bra{\tilde{\eta}} \left( \ket{\eta} \otimes \ket{0^N} \right)
\end{eqnarray}
%
In the last line, we use the fact that $\ket{\eta} \otimes \ket{0^N}$ and
$\ket{\tilde{\eta}}$ are both normalized states of unit norm. The two braket
terms remaining are complex conjugates of each other (call them $a+bi$ and
$a-bi$) so their sum is just $2a$, or the real part of either complex number.

This is where we continue our calculations, using the definition of $\ket{\eta}$
and $\ket{\tilde{\eta}}$ to express their overlaps in terms of $Q_y$,
$R_j$, and projectors onto the orthogonal decomposition of $\mathcal{J}$.
%
\begin{eqnarray}
\braket{\psi}{\psi} & = & 2 - 2\Re\left[ \left(\bra{\eta} \otimes \bra{0^N}\right) \ket{\tilde{\eta}} \right] \\
                    & = & 2 - 2\Re\left[ \sum_{y \in \Delta} \bra{\xi}Q_{f(j)}^{\dagger}Q_y \ket{\xi}
                                                             \bra{0^N}R_j^{\dagger}\Pi_{\mathcal{M}_y}R_j\ket{0^N} \right] \label{eqn:sym-real}
\end{eqnarray}
%
In the last line above, we use the fact that we can distribute a braket
through a tensor product.

We can now factor out the braket
$\bra{0^N}R_j^{\dagger}\Pi_{\mathcal{M}_y}R_j\ket{0^N}$ from the product
inside the real operation above since the braket is completely real. This is
because $\Pi_{\mathcal{M}_y}R_j \ket{0^N}$ is either the zero vector or
a normalized state. We will henceforce call this projected state
$\ket{\phi_{(y,j)}}$, and we will write its overlap with itself
as the braket $\braket{\phi}{\phi}$.

Now we also need to use the fact that the real part of a sum over complex
numbers is at most the sum of real parts of each complex number.
%
\begin{equation}
\Re\left[ \sum_{i} c_i \right] \quad \le \quad \sum_{i} \Re( c_i )
\end{equation}
%
Furthermore, the following equality holds.
%
\begin{equation}
1 - \sum_{i} \Re(c_i) \quad = \quad \sum_{i} \left(1 - \Re(c_i) \right)
\end{equation}
%
Substituting this
back in our original calculation, we get:
%
\begin{eqnarray}
\braket{\psi}{\psi} & \le & 2 - 2\sum_{y \in \Delta} \Re\left(
  \bra{\xi}Q_{f(j)}^{\dagger}Q_y \ket{\xi}
  \braket{\phi_{(y,j)}}{\phi_{(y,j)}} \right) \\
   & \le & 2 - 2\sum_{y \in \Delta} \Re\left(
           \bra{\xi}Q_{f(j)}^{\dagger}Q_y \ket{\xi}\Pr(y | j) \right) \\
   & \le & 2 - 2\sum_{y \in \Delta} \Re\left(
           \bra{\xi}Q_{f(j)}^{\dagger}Q_y \ket{\xi}\right) \Pr(y | j) \\
   & \le & 2 \sum_{y \in \Delta}\left[
           1 - \Re\left( \bra{\xi}Q_{f(j)}^{\dagger}Q_y \ket{\xi} \right)
           \Pr(y | j)
           \right]
\end{eqnarray}
%
In the second line, we used the definition of
$\bra{0^N}R_j^{\dagger}\Pi_{\mathcal{M}_y}R_j\ket{0^N} = \braket{\phi_{(y,j)}}{\phi_{(y,h)}}$
as the probability of getting outcome $y$ in the register
$\mathcal{J}$ given that we have projected onto outcome $j$ in register
$\mathcal{N}$ according to Section \ref{subsec:approx}.

Next, we use the fact that the real part of the overlap
$\bra{\xi}Q_{f(j)}^{\dagger}Q_y \ket{\xi}$ is between $-1$ and $1$, inclusive,
because any $Q_y\ket{\xi}$ is a normalized state, and therefore one minus
this quantity is at most $2$. We also exclude the case where
$y = f(j)$, since in that case $Q_{f_(j)}^{\dagger}Q_y = I$ and the overlap
is 1, contributing zero to the sum.
%
\begin{eqnarray}
\braket{\psi}{\psi} & \le & 2 \sum_{y \ne f(j)} 2\Pr(y|j) \label{eqn:prob-sum}\\
                    & \le & 4\epsilon
\end{eqnarray}
%
In the final line above, we use the definition of $\Pr(y|j)$ from
Section \ref{subsec:approx}. The actual answer we want is the norm
of the difference vector $\ket{\psi}$, which is the square root of the
magnitude of the overlap.
%
\begin{equation}
\lvert\lvert \ket{\psi} \rvert\rvert \le 2\sqrt{\epsilon}
\end{equation}
%
This completes the proof.

As an additional note, if $V$ is the copy operator, we have
$Q_{f(j)}^{\dagger}Q_y = \delta_{y,f(j)}I_{\mathcal{K}}$.
For each $y = f(j)$, the overlap $\bra{\xi}Q_{f(j)}^{\dagger}Q_y \ket{\xi}$
is $1$ and contributes zero to the sum of probabilities.
For each $y \ne f(j)$, the overlap is exactly $0$, and contributes
$\Pr(y|j)$ to the sum.
Therefore, instead of line \ref{eqn:prob-sum} above, we get:
%
\begin{eqnarray}
\braket{\psi}{\psi} & \le & 2\sum_{y \ne f(j)} \Pr(y|j)\\
                    & \le & 2\epsilon
\end{eqnarray}
%
And the final error is
%
\begin{equation}
\lvert\lvert \ket{\psi} \rvert\rvert \le \sqrt{2\epsilon}
\end{equation}
%
This makes intuitive sense, since the first bound is overly general.
It works for any $V$. If we know $V$, we can actually make additional
assumptions and get a better (smaller) upper bound for the error, in this
case by a factor $\sqrt{2}$.
\end{proof}

%%%%%%%%%%%%%%%%%%%%%%%%%%%%%%%%%%%%%%%%%%%%%%%%%%%%%%%%%%%%%%%%%%%%%%%%%%%%%%
%%%%%%%%%%%%%%%%%%%%%%%%%%%%%%%%%%%%%%%%%%%%%%%%%%%%%%%%%%%%%%%%%%%%%%%%%%%%%%
\subsection{Error Probability With Projective Measurement}
\label{subsec:error-proj}

Now what happens if we perform the operator $\hat{Y} = VW$? That is, we
perform $W$ to measure some function $f$ and then we extract the useful
information using $V$, but we don't wish to uncompute the results by
performing $W^{-1}$? There are four related reasons we may wish to do this:

\begin{enumerate}
\item
because $W$ may be practically difficult to perform, and we don't wish
to essentially do it twice
\item
because we have projectively measured
in the register $\mathcal{K}$ so that we can offload some post-processing to
a classical computer, and there is no way to uncompute $W$ at that point.
\item
the states in $\mathcal{K}$, after having been projected by $\hat{Y}$,
are close to classical, and we wish to avoid maintaining them as coherent
quantum states
\item
the states in $\mathcal{N}$, after having been entangled with $\mathcal{K}$ and
projected by $\hat{Y}$, are still ``mostly'' within their subspaces
$\mathcal{L}_j$, and we wish to use states in these subspaces as a resource.
\end{enumerate}

All of these reasons are true for the problem of quantum Fourier state
generation, as described in Section \ref{sec:qcompile-qfs}. However,
the last reason is the most important, since in the case of phase
estimation for phase kickback quantum compiling, a quantum Fourier
state (or something close to one) is left in the subsystem $\mathcal{N}$.

These are both practical reasons in that they don't affect the
asymptotic resources required by our algorithm. However, they do affect
the asymptotic error of our algorithm. In this section, we will calculate
this error based on Theorem \ref{thm:coherent}. This is the main original
contribution of this work beyond the exposition in \cite{Kitaev2002}.

%%%%%%%%%%%%%%%%%%%%%%%%%%%%%%%%%%%%%%%%%%%%%%%%%%%%%%%%%%%%%%%%%%%%%%%%%%%%%%%
\begin{theorem}
Given the setting in Theorem \ref{thm:coherent}
where $W$ approximately measures a function $f : \Omega \rightarrow \Delta$
with error $\epsilon$: If the operator $\hat{Y} = VW$ followed by projective
measurement on $\mathcal{K}$ yields classical outcome $y \in \Delta$,
then the
overlap of the state $\ket{\tilde{\psi}}$ in $\mathcal{N}$ with $\mathcal{L}_j$ such
that $f(j) = y$ is upper-bounded by the number of subspaces $|\Omega|$ times the
$\epsilon$.
%
\begin{equation}
\sum_{j \in \Omega: f(j) \ne y} \bra{\tilde{\psi}}\Pi_{\mathcal{L}_j}\ket{\tilde{\psi}} \le |\Omega|\epsilon
\end{equation}
\label{thm:projective}
\end{theorem}
%
\begin{proof}
Consider the probability distribution of states over $\mathcal{N} \otimes \mathcal{J} \otimes \mathcal{K}$.
Decomposing the entire system over the subspaces of $\mathcal{N} \bigoplus_{j \in \Omega} \mathcal{L}_j$,
within each $\mathcal{L}_j$ there is a probability of $\epsilon$ associated with error states $y \in \Delta$
such that $y \ne f(j)$.
%
\begin{equation}
\sum_{y \ne f(j)} \Pr(y | j) \le \epsilon
\end{equation}
%
We can define a total error probability by summing over all values of $j \in Omega$, or
$\mathcal{L_j} \in \mathcal{N}$.

\begin{equation}
\sum_{j \in \Omega} \sum_{y \ne f(j)} \Pr(y | j) \le |\Omega|\epsilon
\end{equation}

However, if we projectively measure $y$ in register $\mathcal{K}$, what is the probability
mass associated with ``impurities'' in register $\mathcal{N}$? That is, we want to
find $\sum_{j: f(j) \ne y} \Pr(j | y)$. The maximum contribution to this error for
any $y$ is the total error probability above. Therefore, we have:

\begin{equation}
\sum_{j: f(j) \ne y} \Pr(j | y) \le |\Omega|\epsilon
\end{equation}

This argument is independent of any ancillae.
\end{proof}

Theorem \ref{thm:projective} upper-bounds the error for
early measurement in KSV parallel phase estimation used to
generate a quantum Fourier state. It is necessary to compute the resources
for a quantum compiler in Section \ref{sec:qcompile-ksv}.