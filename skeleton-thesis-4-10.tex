\documentclass{article}

\usepackage{amssymb}

%    Q-circuit version 1.06
%    Copyright (C) 2004  Steve Flammia & Bryan Eastin

%    This program is free software; you can redistribute it and/or modify
%    it under the terms of the GNU General Public License as published by
%    the Free Software Foundation; either version 2 of the License, or
%    (at your option) any later version.
%
%    This program is distributed in the hope that it will be useful,
%    but WITHOUT ANY WARRANTY; without even the implied warranty of
%    MERCHANTABILITY or FITNESS FOR A PARTICULAR PURPOSE.  See the
%    GNU General Public License for more details.
%
%    You should have received a copy of the GNU General Public License
%    along with this program; if not, write to the Free Software
%    Foundation, Inc., 59 Temple Place, Suite 330, Boston, MA  02111-1307  USA

\usepackage[matrix,frame,arrow]{xy}
\usepackage{amsmath}
\newcommand{\bra}[1]{\left\langle{#1}\right\vert}
\newcommand{\ket}[1]{\left\vert{#1}\right\rangle}
    % Defines Dirac notation.
\newcommand{\qw}[1][-1]{\ar @{-} [0,#1]}
    % Defines a wire that connects horizontally.  By default it connects to the object on the left of the current object.
    % WARNING: Wire commands must appear after the gate in any given entry.
\newcommand{\qwx}[1][-1]{\ar @{-} [#1,0]}
    % Defines a wire that connects vertically.  By default it connects to the object above the current object.
    % WARNING: Wire commands must appear after the gate in any given entry.
\newcommand{\cw}[1][-1]{\ar @{=} [0,#1]}
    % Defines a classical wire that connects horizontally.  By default it connects to the object on the left of the current object.
    % WARNING: Wire commands must appear after the gate in any given entry.
\newcommand{\cwx}[1][-1]{\ar @{=} [#1,0]}
    % Defines a classical wire that connects vertically.  By default it connects to the object above the current object.
    % WARNING: Wire commands must appear after the gate in any given entry.
\newcommand{\gate}[1]{*{\xy *+<.6em>{#1};p\save+LU;+RU **\dir{-}\restore\save+RU;+RD **\dir{-}\restore\save+RD;+LD **\dir{-}\restore\POS+LD;+LU **\dir{-}\endxy} \qw}
    % Boxes the argument, making a gate.
\newcommand{\meter}{\gate{\xy *!<0em,1.1em>h\cir<1.1em>{ur_dr},!U-<0em,.4em>;p+<.5em,.9em> **h\dir{-} \POS <-.6em,.4em> *{},<.6em,-.4em> *{} \endxy}}
    % Inserts a measurement meter.
\newcommand{\measure}[1]{*+[F-:<.9em>]{#1} \qw}
    % Inserts a measurement bubble with user defined text.
\newcommand{\measuretab}[1]{*{\xy *+<.6em>{#1};p\save+LU;+RU **\dir{-}\restore\save+RU;+RD **\dir{-}\restore\save+RD;+LD **\dir{-}\restore\save+LD;+LC-<.5em,0em> **\dir{-} \restore\POS+LU;+LC-<.5em,0em> **\dir{-} \endxy} \qw}
    % Inserts a measurement tab with user defined text.
\newcommand{\measureD}[1]{*{\xy*+=+<.5em>{\vphantom{#1}}*\cir{r_l};p\save*!R{#1} \restore\save+UC;+UC-<.5em,0em>*!R{\hphantom{#1}}+L **\dir{-} \restore\save+DC;+DC-<.5em,0em>*!R{\hphantom{#1}}+L **\dir{-} \restore\POS+UC-<.5em,0em>*!R{\hphantom{#1}}+L;+DC-<.5em,0em>*!R{\hphantom{#1}}+L **\dir{-} \endxy} \qw}
    % Inserts a D-shaped measurement gate with user defined text.
\newcommand{\multimeasure}[2]{*+<1em,.9em>{\hphantom{#2}} \qw \POS[0,0].[#1,0];p !C *{#2},p \drop\frm<.9em>{-}}
    % Draws a multiple qubit measurement bubble starting at the current position and spanning #1 additional gates below.
    % #2 gives the label for the gate.
    % You must use an argument of the same width as #2 in \ghost for the wires to connect properly on the lower lines.
\newcommand{\multimeasureD}[2]{*+<1em,.9em>{\hphantom{#2}}\save[0,0].[#1,0];p\save !C *{#2},p+LU+<0em,0em>;+RU+<-.8em,0em> **\dir{-}\restore\save +LD;+LU **\dir{-}\restore\save +LD;+RD-<.8em,0em> **\dir{-} \restore\save +RD+<0em,.8em>;+RU-<0em,.8em> **\dir{-} \restore \POS !UR*!UR{\cir<.9em>{r_d}};!DR*!DR{\cir<.9em>{d_l}}\restore \qw}
    % Draws a multiple qubit D-shaped measurement gate starting at the current position and spanning #1 additional gates below.
    % #2 gives the label for the gate.
    % You must use an argument of the same width as #2 in \ghost for the wires to connect properly on the lower lines.
\newcommand{\control}{*-=-{\bullet}}
    % Inserts an unconnected control.
\newcommand{\controlo}{*!<0em,.04em>-<.07em,.11em>{\xy *=<.45em>[o][F]{}\endxy}}
    % Inserts a unconnected control-on-0.
\newcommand{\ctrl}[1]{\control \qwx[#1] \qw}
    % Inserts a control and connects it to the object #1 wires below.
\newcommand{\ctrlo}[1]{\controlo \qwx[#1] \qw}
    % Inserts a control-on-0 and connects it to the object #1 wires below.
\newcommand{\targ}{*{\xy{<0em,0em>*{} \ar @{ - } +<.4em,0em> \ar @{ - } -<.4em,0em> \ar @{ - } +<0em,.4em> \ar @{ - } -<0em,.4em>},*+<.8em>\frm{o}\endxy} \qw}
    % Inserts a CNOT target.
\newcommand{\qswap}{*=<0em>{\times} \qw}
    % Inserts half a swap gate. 
    % Must be connected to the other swap with \qwx.
\newcommand{\multigate}[2]{*+<1em,.9em>{\hphantom{#2}} \qw \POS[0,0].[#1,0];p !C *{#2},p \save+LU;+RU **\dir{-}\restore\save+RU;+RD **\dir{-}\restore\save+RD;+LD **\dir{-}\restore\save+LD;+LU **\dir{-}\restore}
    % Draws a multiple qubit gate starting at the current position and spanning #1 additional gates below.
    % #2 gives the label for the gate.
    % You must use an argument of the same width as #2 in \ghost for the wires to connect properly on the lower lines.
\newcommand{\ghost}[1]{*+<1em,.9em>{\hphantom{#1}} \qw}
    % Leaves space for \multigate on wires other than the one on which \multigate appears.  Without this command wires will cross your gate.
    % #1 should match the second argument in the corresponding \multigate. 
\newcommand{\push}[1]{*{#1}}
    % Inserts #1, overriding the default that causes entries to have zero size.  This command takes the place of a gate.
    % Like a gate, it must precede any wire commands.
    % \push is useful for forcing columns apart.
    % NOTE: It might be useful to know that a gate is about 1.3 times the height of its contents.  I.e. \gate{M} is 1.3em tall.
    % WARNING: \push must appear before any wire commands and may not appear in an entry with a gate or label.
\newcommand{\gategroup}[6]{\POS"#1,#2"."#3,#2"."#1,#4"."#3,#4"!C*+<#5>\frm{#6}}
    % Constructs a box or bracket enclosing the square block spanning rows #1-#3 and columns=#2-#4.
    % The block is given a margin #5/2, so #5 should be a valid length.
    % #6 can take the following arguments -- or . or _\} or ^\} or \{ or \} or _) or ^) or ( or ) where the first two options yield dashed and
    % dotted boxes respectively, and the last eight options yield bottom, top, left, and right braces of the curly or normal variety.
    % \gategroup can appear at the end of any gate entry, but it's good form to pick one of the corner gates.
    % BUG: \gategroup uses the four corner gates to determine the size of the bounding box.  Other gates may stick out of that box.  See \prop. 
\newcommand{\rstick}[1]{*!L!<-.5em,0em>=<0em>{#1}}
    % Centers the left side of #1 in the cell.  Intended for lining up wire labels.  Note that non-gates have default size zero.
\newcommand{\lstick}[1]{*!R!<.5em,0em>=<0em>{#1}}
    % Centers the right side of #1 in the cell.  Intended for lining up wire labels.  Note that non-gates have default size zero.
\newcommand{\ustick}[1]{*!D!<0em,-.5em>=<0em>{#1}}
    % Centers the bottom of #1 in the cell.  Intended for lining up wire labels.  Note that non-gates have default size zero.
\newcommand{\dstick}[1]{*!U!<0em,.5em>=<0em>{#1}}
    % Centers the top of #1 in the cell.  Intended for lining up wire labels.  Note that non-gates have default size zero.
\newcommand{\Qcircuit}{\xymatrix @*=<0em>}
    % Defines \Qcircuit as an \xymatrix with entries of default size 0em.


\begin{document}

\section{Settling Previous Questions}

\begin{enumerate}
\item
The relationship between majority gates and threshold gates are as follows.
Threshold gates of polynomially-bounded weight can be simulated by
(I think up to three layers of) threshold gates with unit weights, which
can then be trivially simulated by majority gates. I think the only 
difference between the last two are that majority gates never have a bias,
that is, the weight $w_0$. The reverse direction is trivial.

\item
As Aram pointed out in our meeting, the $O(\log n)$ block size in the
block-save adder is probably to polynomially bound the weights in the
threshold circuit, something that I had not previously considered, but
seems obvious in retrospect.

\end{enumerate}

Okay that's it folks, move along.

\section{The Overall 2D Layout}

There are several intermediate results, such as the $n^2$-bit result of the
multiple product, the quotient of this with the modulus $m$, which involves
finding the reciprocal of $m$, the product of
the quotient and the modulus, and the subtraction.

Using our model with modules, we can teleport / fanout copies to different
modules and not worry about the exact geometry of the relationship between
them.

\section{To Do After These Pages Are Written}

Deeper understanding of the Reif and Tate results are needed, but perhaps
that can be delayed or deferred until after we have a final version of
Chapter 1.

Soft failure can look like just using the results of Reif and Tate.
But some deeper understanding is needed. For example, how to compute the
reciprocal. This can be done using the expansion given in Kitaev, but
then this reduces to creating the products in constant depth.

We don't necessarily need to use threshold gates to do this, since the
Toffoli is much easier, and can still be made constant.

As pointed out in group meeting, David's reordering circuit could be
used to rearrange the qubits, after being fanned out, into the right
order. But we could do this in blocks. What are the size of the blocks?
What \emph{are} the blocks?

\section{Creating Partial Product Bits}

In this section, I will describe how the partial product creation I gave in
my 2D factoring paper.

\section{Multiple Product via Reif and Tate}

In the 1992 paper by Reif and Tate, which appears to have appeared
contemporaneously with a bunch of papers by Bruck, and so was not cited
until a 1996 paper by Yeh and Varvarigos when the dust had settled, they
address the question of multiple product. Actually it is multiple product
modulo a number $p$, which as far as I can tell does not strictly have to
be prime, or if it is prime, is a smaller prime and one chosen so it is
bigger than the largest possible output number, so that no bits are
truncated. It does not correspond to the number $m$, the modulus to be
factored in Shor's factoring algorithm.

It remains to be seen whether I can use this or not. Yeh and Varvarigos
seem to think so, as long as I choose $p$ large enough. But then I need
to do modular reduction.

\section{The equivalence of threshold circuits to $Z_p$ circuits}

They can simulate each other in constant depth and polynomial increase in 
size. Certainly, this seems useful for the so-called
``Chinese Remaindering'' procedure, since we are dividing up a number into
factors of this ``mixed radix'' where the weights never become exponential,
but rather, remain of size $O(n)$, or at least polynomial.

\section{Size and depth tradeoffs}

Also from reading the Yeh-Varvarigos paper, I became aware of new parameters
in threshold circuits, namely the tradeoff between size and depth. The
size is never very critical for me, as long as it's polynomial, and 
surprisingly even with the parameter $\epsilon$ set to its maximum value,
the size is never more than $O(n^2)$ and usually $O(n)$.

\section{Sum of $n$ bits}

There is a construction given in Yeh-Varvarigos for the sum of $n$ bits,
which can then be used, I guess, for iterated sum, or multiple addition.
At some point, hours need to be set aside for finding the exact
construction for the building blocks that I am going to use.

For iterated addition, we can pretty much use the Siu-Bruck original
construction. And time's up.

\section{Explicit construction of simulation}

The Goldmann and Karpinski paper showed an explicit construction for
simulating threshold gates with exponential weights with those of
polynomial size weights. This is in contrast to the probabilistic,
existential, non-constructive proof given in the Siu-Bruck
paper ``On the Dynamic Range of Linear Threshold Elements.'' and
one other paper where the division and multiplication functions are
given.

I think this might be useful for lower bounds, or back when it was thought
that multiplication and division required exponentially many terms.
(Citation needed!)

\end{document}